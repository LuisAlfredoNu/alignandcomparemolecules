%%%%%%%%%%%%%%%%%%%%%%%%%%%%%%%%%%%%%%%%%
% Report of comparations of the program
% alignandcomparemolecules
% LaTeX Template
% Version 0.1 (10/12/17)
%
% Part of this template was downloaded from:
% http://www.LaTeXTemplates.com
%
% Modifications by:
% Isaias Morales Salazar (Shiver) isaiasms117@gmail.com
% Luis Alfredo Nuñez (Fdx) luis.alfredo.nu@gmail.com
%
%%%%%%%%%%%%%%%%%%%%%%%%%%%%%%%%%%%%%%%%%
%-----------------------------------------------------------------------------% 
% Packages
%-----------------------------------------------------------------------------%
\documentclass[12pt]{article} 	
\usepackage[utf8]{inputenc}
\usepackage[spanish]{babel}
\usepackage[usenames]{color}
\usepackage{float}
\usepackage{graphicx}
\usepackage{amsmath,mathtools}
\usepackage[font={small},format=plain,labelfont=bf]{caption} % Size and format of foot of figures and tables
\usepackage[dvipsnames]{xcolor}
\usepackage[most]{tcolorbox}
\usepackage{listings} % Add parts of code
\usepackage{multicol}
\usepackage{array,multirow}
%-----------------------------------------------------------------------------%
% Margin settings 
%-----------------------------------------------------------------------------%
\usepackage{geometry}
\geometry{
	paper=letterpaper, % Paper type
	inner=2.5cm, % Inner margin
	outer=2.5cm, % Outer margin
	top=3.5cm, % Top margin
	bottom=3.5cm, % Bottom margin
}
%-----------------------------------------------------------------------------% 
% New Commands
%-----------------------------------------------------------------------------%
\renewcommand{\tablename}{Tabla} % Name of table foot 
\renewcommand{\baselinestretch}{1.5} % space between lines 1.5
\newcommand\tab[1][0.5cm]{\hspace*{#1}}
\newcommand\vtab[1][0.5cm]{\vspace*{#1}}
\newcommand{\classC}[1]{\textcolor{MidnightBlue}{\bf {#1}}}
\newcommand{\alert}[1]{\begin{center}
\fcolorbox{RoyalBlue}{Mahogany!90}{\tab\begin{minipage}[c]{0.8\textwidth}
\vtab[-7mm]\textcolor{Goldenrod}{
\begin{flushleft}{\bf{#1}}\end{flushleft}
}\vtab[-3mm]
\end{minipage}\tab}
\end{center}}
%-----------------------------------------------------------------------------% 
% Path
%-----------------------------------------------------------------------------% 
\newcommand*{\srcpath}{src/}
%-----------------------------------------------------------------------------%
%-----------------------------------------------------------------------------%
% Document
%-----------------------------------------------------------------------------%
%-----------------------------------------------------------------------------% 

\begin{document}
%-----------------------------------------------------------------------------%
% Cover Page
%-----------------------------------------------------------------------------% 
\pagestyle{empty} 

\begin{center}
\includegraphics[width=4cm]{escudo-buap.pdf}

BENEMÉRITA UNIVERSIDAD AUTÓNOMA DE PUEBLA\\
\rule{150mm}{0.1mm}
FACULTAD DE CIENCIAS QUÍMICAS\\
LABORATORIO DE FISICOQUÍMICA TEÓRICA
\rule{150mm}{0.1mm}

\vtab
\Large{Reporte de resultados para el programa \\ alignandcomparemolecules}\\
\vtab[.3cm]

\large{Título del proyecto: \\ \textbf{Desarrollo de software para comparar y resolver estructuras isoméricas mediante el tensor de inercia}}\\
\vtab[0.5cm]

PRESENTA\\
Luis Alfredo Nuñez Meneses \\
\vtab[0.5cm]

\tab[-0.5cm]
\begin{minipage}[t]{0.5\textwidth}
\begin{center}
DIRECTOR DE TESIS  \\
Dr. Juan Manuel Solano Altamirano \\
FQ-BUAP \\
\end{center}
\end{minipage}

\vtab[0.5cm]
\today\\
\end{center}

\begin{flushright}
  \large{Jefe del Departamento de Fisicoquímica\\
  Dr. Juan Carlos Ramírez García}\\
\end{flushright}

\newpage

%-----------------------------------------------------------------------------% 
% Resultados
%-----------------------------------------------------------------------------% 
\section{Resultados}
\subsection{Pruebas realizadas}
Las Pruebas que ya se realizaron son las siguientes y los resultados son óptimos y satisfactorios en todas las pruebas.\\
Realicé todas las pruebas que muestro a continuación, para hacer todas las comparaciones programé unos scripts que me ayudaran a 
automatizar todo el trabajo.

		%\tab[-2cm]
		%\begin{tabular}{c|m{8cm}|c|c}
		%\# & Moléculas & Restultado esperado & Resultado programa \\ \hline\hline
		%\multirow{4}{*}{\tab 1 \tab} & r-2-chlorobutane\_out\_G09\_inversion & \multirow{4}{*}{Estereisómeros} & \multirow{4}{*}{Estereisómeros} \\
		%						 & E = -618.05708 \tab Freq = 115.2921   &											& 											 \\ \cline{2-2}
		%						 & r-2-chlorobutane\_out\_G09 			  &											& 											 \\
		%						 & E = -618.05708 \tab Freq = 115.2921   &											& 											 \\ \hline
		%\multirow{4}{*}{\tab 1 \tab} & r-2-chlorobutane\_out\_G09\_inversion & \multirow{4}{*}{Estereisómeros} & \multirow{4}{*}{Estereisómeros} \\
		%						 & E = -618.05708 \tab Freq = 115.2921   &											& 											 \\ \cline{2-2}
		%						 & r-2-chlorobutane\_out\_G09 			  &											& 											 \\
		%						 & E = -618.05708 \tab Freq = 115.2921   &											& 											 \\ \hline
		%\end{tabular}
		%
		%\vtab[2cm]
		%\begin{tabular}{c|m{5cm}|m{5cm}|m{3cm}|m{3cm}}
		%\# & Molécula A & Molécula B & Resultado esperado & Resultado programa \\ \hline
		%1  & r-2-chlorobutane\_out\_G09\_inversion & r-2-chlorobutane\_out\_G09 & Estereisómeros & Estereisómeros \\
		%	& E = -618.057081944 					 & E = -618.057081944 & & \\
		%	& Freq = 115.2921 						 & Freq = 115.2921 & & \\ \hline
		%
		%\end{tabular}



	%\begin{multicols}{2}
	%\begin{center}
	%Molecule A: \\ SymmetricEnantiomers\/Opt\_Geo\/s-tmp\_rotated\_out\_G09\_output.xyz \\
	% Inertia Tensor - Molecule A \\
	%\vtab
	%\begin{tabular}{|c c c|}
	%            103.7 &  -4.21263e-05 & 8.53829e-06  \\ 
	%     -4.21263e-05 &       630.074 & 2.17164e-05  \\ 
	%      8.53829e-06 &   2.17164e-05 &     722.449   
	%\end{tabular}
	%
	%\vtab
	% EingenVectors - Molecule A     \\ 
	%\vtab
	%\begin{tabular}{|c c c|}
	%               -1  & -8.00311e-08  &1.37993e-08  \\ 
	%      8.00312e-08  &           -1  &2.35089e-07  \\ 
	%      1.37993e-08  &  2.35089e-07  &          1   
	%\end{tabular}
	%
	%\vtab
	% EingenValues - Molecule A       \\
	%\vtab
	%\begin{tabular}{|c c c|}
	%            103.7  &      630.074  &    722.449   
	%\end{tabular}
	%
	%\columnbreak
	%Molecule B: \\ SymmetricEnantiomers\/Opt\_Geo\/s-tmp\_rotated\_out\_G09.xyz\\
	%Inertia Tensor - Molecule B\\
	%
	%\vtab
	%\begin{tabular}{|c c c|}
	%         103.707   &     1.51142    &  1.05403  \\
	%         1.51142   &     630.103    &  1.67204  \\
	%         1.05403   &     1.67204    &   722.42  
	%\end{tabular}
	%                                                
	%\vtab
	%EingenVectors - Molecule B\\
	%\vtab
	%\begin{tabular}{|c c c|}
	%       -0.999994   &  0.00286582  & 0.00169581  \\
	%      -0.0028346   &   -0.999832  &   0.018136  \\
	%       0.0017475   &   0.0181311  &   0.999834  
	%\end{tabular}
	%                                                
	%\vtab
	%EingenValues - Molecule B\\
	%\vtab
	%\begin{tabular}{|c c c|}
	%         103.701   &     630.077   &   722.452   
	%\end{tabular}
	%
	%\end{center}
	%\end{multicols}
	%\textcolor{NavyBlue}{\large Resultado: Are Isomers}

% All results

\vtab[-3cm]
\begin{center}
{\large AsymmetricEnantiomers \tab Número 1}
\end{center}
\begin{multicols}{2}
\begin{center}

Molecule A \
r-2-chlorobutane\_out\_G09

\includegraphics[width=6cm]{../Comparisons/ImagesFromVMD/r-2-chlorobutane_out_G09.png}

Inertia Tensor - Molecule A \\
\begin{tabular}{|c c c|}
113.621	 & 	-8.52391	 & 	-0.0867925	 \\
-8.52391	 & 	164.234	 & 	-0.885702	 \\
-0.0867925	 & 	-0.885702	 & 	257.008
\end{tabular}

\vtab
 EingenVectors - Molecule A     \\
\begin{tabular}{|c c c|}
-0.986829	 & 	-0.161759	 & 	-0.00158111	 \\
0.161767	 & 	-0.986784	 & 	-0.00941026	 \\
-3.80259e-05	 & 	-0.00954208	 & 	0.999954
\end{tabular}

\vtab
 EingenValues - Molecule A     \\
\begin{tabular}{|c c c|}
112.224	 & 	165.623	 & 	257.016	 \\
\end{tabular}
\columnbreak

Molecule B \
r-2-chlorobutane\_out\_G09\_inversion

\includegraphics[width=6cm]{../Comparisons/ImagesFromVMD/r-2-chlorobutane_out_G09_inversion.png}

Inertia Tensor - Molecule B \\
\begin{tabular}{|c c c|}
113.621	 & 	-8.52381	 & 	-0.0867846	 \\
-8.52381	 & 	164.234	 & 	-0.885691	 \\
-0.0867846	 & 	-0.885691	 & 	257.008
\end{tabular}

\vtab
 EingenVectors - Molecule B     \\
\begin{tabular}{|c c c|}
-0.986829	 & 	-0.161756	 & 	-0.00158103	 \\
0.161764	 & 	-0.986785	 & 	-0.00941017	 \\
-3.79835e-05	 & 	-0.00954198	 & 	0.999954
\end{tabular}

\vtab
 EingenValues - Molecule B     \\
\begin{tabular}{|c c c|}
112.224	 & 	165.623	 & 	257.016	 \\
\end{tabular}

\end{center}
\end{multicols}

\vtab[-5mm]
\begin{tabular}{*{2}{m{0.38\textwidth}}}
\begin{center}
\textcolor{NavyBlue}{\Large Enantiomers}
\end{center}
&
\begin{center}
\includegraphics[height=6.5cm]{../Comparisons/Vectors/inertia_tensor_of_r-2-chlorobutane_out_G09_and_r-2-chlorobutane_out_G09_inversion.png}
\end{center}
\end{tabular}

 \newpage

\vtab[-3cm]
\begin{center}
{\large AsymmetricEnantiomers \tab Número 2}
\end{center}
\begin{multicols}{2}
\begin{center}

Molecule A \
r-2-chlorobutane\_out\_G09

\includegraphics[width=6cm]{../Comparisons/ImagesFromVMD/r-2-chlorobutane_out_G09.png}

Inertia Tensor - Molecule A \\
\begin{tabular}{|c c c|}
113.621	 & 	-8.52391	 & 	-0.0867925	 \\
-8.52391	 & 	164.234	 & 	-0.885702	 \\
-0.0867925	 & 	-0.885702	 & 	257.008
\end{tabular}

\vtab
 EingenVectors - Molecule A     \\
\begin{tabular}{|c c c|}
-0.986829	 & 	-0.161759	 & 	-0.00158111	 \\
0.161767	 & 	-0.986784	 & 	-0.00941026	 \\
-3.80259e-05	 & 	-0.00954208	 & 	0.999954
\end{tabular}

\vtab
 EingenValues - Molecule A     \\
\begin{tabular}{|c c c|}
112.224	 & 	165.623	 & 	257.016	 \\
\end{tabular}
\columnbreak

Molecule B \
r-2-chlorobutane\_rotated\_out\_G09

\includegraphics[width=6cm]{../Comparisons/ImagesFromVMD/r-2-chlorobutane_rotated_out_G09.png}

Inertia Tensor - Molecule B \\
\begin{tabular}{|c c c|}
113.621	 & 	-8.52391	 & 	-0.0867928	 \\
-8.52391	 & 	164.234	 & 	-0.885702	 \\
-0.0867928	 & 	-0.885702	 & 	257.008
\end{tabular}

\vtab
 EingenVectors - Molecule B     \\
\begin{tabular}{|c c c|}
-0.986829	 & 	-0.161759	 & 	-0.00158112	 \\
0.161767	 & 	-0.986784	 & 	-0.00941025	 \\
-3.80279e-05	 & 	-0.00954208	 & 	0.999954
\end{tabular}

\vtab
 EingenValues - Molecule B     \\
\begin{tabular}{|c c c|}
112.224	 & 	165.623	 & 	257.016	 \\
\end{tabular}

\end{center}
\end{multicols}

\vtab[-5mm]
\begin{tabular}{*{2}{m{0.38\textwidth}}}
\begin{center}
\textcolor{NavyBlue}{\Large Equal}
\end{center}
&
\begin{center}
\includegraphics[height=6.5cm]{../Comparisons/Vectors/inertia_tensor_of_r-2-chlorobutane_out_G09_and_r-2-chlorobutane_rotated_out_G09.png}
\end{center}
\end{tabular}

 \newpage

\vtab[-3cm]
\begin{center}
{\large AsymmetricEnantiomers \tab Número 3}
\end{center}
\begin{multicols}{2}
\begin{center}

Molecule A \
r-2-chlorobutane\_out\_G09

\includegraphics[width=6cm]{../Comparisons/ImagesFromVMD/r-2-chlorobutane_out_G09.png}

Inertia Tensor - Molecule A \\
\begin{tabular}{|c c c|}
113.621	 & 	-8.52391	 & 	-0.0867925	 \\
-8.52391	 & 	164.234	 & 	-0.885702	 \\
-0.0867925	 & 	-0.885702	 & 	257.008
\end{tabular}

\vtab
 EingenVectors - Molecule A     \\
\begin{tabular}{|c c c|}
-0.986829	 & 	-0.161759	 & 	-0.00158111	 \\
0.161767	 & 	-0.986784	 & 	-0.00941026	 \\
-3.80259e-05	 & 	-0.00954208	 & 	0.999954
\end{tabular}

\vtab
 EingenValues - Molecule A     \\
\begin{tabular}{|c c c|}
112.224	 & 	165.623	 & 	257.016	 \\
\end{tabular}
\columnbreak

Molecule B \
s-2-chlorobutane\_out\_G09

\includegraphics[width=6cm]{../Comparisons/ImagesFromVMD/s-2-chlorobutane_out_G09.png}

Inertia Tensor - Molecule B \\
\begin{tabular}{|c c c|}
113.621	 & 	8.52396	 & 	0.0868155	 \\
8.52396	 & 	164.235	 & 	-0.885715	 \\
0.0868155	 & 	-0.885715	 & 	257.008
\end{tabular}

\vtab
 EingenVectors - Molecule B     \\
\begin{tabular}{|c c c|}
-0.98683	 & 	0.161755	 & 	0.00158125	 \\
-0.161763	 & 	-0.986785	 & 	-0.00941041	 \\
3.81715e-05	 & 	-0.00954226	 & 	0.999954
\end{tabular}

\vtab
 EingenValues - Molecule B     \\
\begin{tabular}{|c c c|}
112.224	 & 	165.624	 & 	257.017	 \\
\end{tabular}

\end{center}
\end{multicols}

\vtab[-5mm]
\begin{tabular}{*{2}{m{0.38\textwidth}}}
\begin{center}
\textcolor{NavyBlue}{\Large Enantiomers}
\end{center}
&
\begin{center}
\includegraphics[height=6.5cm]{../Comparisons/Vectors/inertia_tensor_of_r-2-chlorobutane_out_G09_and_s-2-chlorobutane_out_G09.png}
\end{center}
\end{tabular}

 \newpage

\vtab[-3cm]
\begin{center}
{\large AsymmetricEnantiomers \tab Número 4}
\end{center}
\begin{multicols}{2}
\begin{center}

Molecule A \
r-2-chlorobutane\_out\_G09

\includegraphics[width=6cm]{../Comparisons/ImagesFromVMD/r-2-chlorobutane_out_G09.png}

Inertia Tensor - Molecule A \\
\begin{tabular}{|c c c|}
113.621	 & 	-8.52391	 & 	-0.0867925	 \\
-8.52391	 & 	164.234	 & 	-0.885702	 \\
-0.0867925	 & 	-0.885702	 & 	257.008
\end{tabular}

\vtab
 EingenVectors - Molecule A     \\
\begin{tabular}{|c c c|}
-0.986829	 & 	-0.161759	 & 	-0.00158111	 \\
0.161767	 & 	-0.986784	 & 	-0.00941026	 \\
-3.80259e-05	 & 	-0.00954208	 & 	0.999954
\end{tabular}

\vtab
 EingenValues - Molecule A     \\
\begin{tabular}{|c c c|}
112.224	 & 	165.623	 & 	257.016	 \\
\end{tabular}
\columnbreak

Molecule B \
s-2-chlorobutane\_out\_G09\_inversion

\includegraphics[width=6cm]{../Comparisons/ImagesFromVMD/s-2-chlorobutane_out_G09_inversion.png}

Inertia Tensor - Molecule B \\
\begin{tabular}{|c c c|}
113.622	 & 	8.52398	 & 	0.0868204	 \\
8.52398	 & 	164.235	 & 	-0.88572	 \\
0.0868204	 & 	-0.88572	 & 	257.009
\end{tabular}

\vtab
 EingenVectors - Molecule B     \\
\begin{tabular}{|c c c|}
-0.986829	 & 	0.161757	 & 	0.0015813	 \\
-0.161765	 & 	-0.986785	 & 	-0.00941039	 \\
3.82053e-05	 & 	-0.00954225	 & 	0.999954
\end{tabular}

\vtab
 EingenValues - Molecule B     \\
\begin{tabular}{|c c c|}
112.224	 & 	165.624	 & 	257.018	 \\
\end{tabular}

\end{center}
\end{multicols}

\vtab[-5mm]
\begin{tabular}{*{2}{m{0.38\textwidth}}}
\begin{center}
\textcolor{NavyBlue}{\Large Equal}
\end{center}
&
\begin{center}
\includegraphics[height=6.5cm]{../Comparisons/Vectors/inertia_tensor_of_r-2-chlorobutane_out_G09_and_s-2-chlorobutane_out_G09_inversion.png}
\end{center}
\end{tabular}

 \newpage

\vtab[-3cm]
\begin{center}
{\large AsymmetricEnantiomers \tab Número 5}
\end{center}
\begin{multicols}{2}
\begin{center}

Molecule A \
r-2-chlorobutane\_out\_G09

\includegraphics[width=6cm]{../Comparisons/ImagesFromVMD/r-2-chlorobutane_out_G09.png}

Inertia Tensor - Molecule A \\
\begin{tabular}{|c c c|}
113.621	 & 	-8.52391	 & 	-0.0867925	 \\
-8.52391	 & 	164.234	 & 	-0.885702	 \\
-0.0867925	 & 	-0.885702	 & 	257.008
\end{tabular}

\vtab
 EingenVectors - Molecule A     \\
\begin{tabular}{|c c c|}
-0.986829	 & 	-0.161759	 & 	-0.00158111	 \\
0.161767	 & 	-0.986784	 & 	-0.00941026	 \\
-3.80259e-05	 & 	-0.00954208	 & 	0.999954
\end{tabular}

\vtab
 EingenValues - Molecule A     \\
\begin{tabular}{|c c c|}
112.224	 & 	165.623	 & 	257.016	 \\
\end{tabular}
\columnbreak

Molecule B \
s-2-chlorobutane\_rotated\_out\_G09

\includegraphics[width=6cm]{../Comparisons/ImagesFromVMD/s-2-chlorobutane_rotated_out_G09.png}

Inertia Tensor - Molecule B \\
\begin{tabular}{|c c c|}
113.621	 & 	8.52397	 & 	0.0867992	 \\
8.52397	 & 	164.235	 & 	-0.885706	 \\
0.0867992	 & 	-0.885706	 & 	257.008
\end{tabular}

\vtab
 EingenVectors - Molecule B     \\
\begin{tabular}{|c c c|}
-0.98683	 & 	0.161756	 & 	0.00158113	 \\
-0.161763	 & 	-0.986785	 & 	-0.00941034	 \\
3.80628e-05	 & 	-0.00954217	 & 	0.999954
\end{tabular}

\vtab
 EingenValues - Molecule B     \\
\begin{tabular}{|c c c|}
112.224	 & 	165.624	 & 	257.017	 \\
\end{tabular}

\end{center}
\end{multicols}

\vtab[-5mm]
\begin{tabular}{*{2}{m{0.38\textwidth}}}
\begin{center}
\textcolor{NavyBlue}{\Large Enantiomers}
\end{center}
&
\begin{center}
\includegraphics[height=6.5cm]{../Comparisons/Vectors/inertia_tensor_of_r-2-chlorobutane_out_G09_and_s-2-chlorobutane_rotated_out_G09.png}
\end{center}
\end{tabular}

 \newpage

\vtab[-3cm]
\begin{center}
{\large AsymmetricEnantiomers \tab Número 6}
\end{center}
\begin{multicols}{2}
\begin{center}

Molecule A \
r-2-chlorobutane\_out\_G09\_inversion

\includegraphics[width=6cm]{../Comparisons/ImagesFromVMD/r-2-chlorobutane_out_G09_inversion.png}

Inertia Tensor - Molecule A \\
\begin{tabular}{|c c c|}
113.621	 & 	-8.52381	 & 	-0.0867846	 \\
-8.52381	 & 	164.234	 & 	-0.885691	 \\
-0.0867846	 & 	-0.885691	 & 	257.008
\end{tabular}

\vtab
 EingenVectors - Molecule A     \\
\begin{tabular}{|c c c|}
-0.986829	 & 	-0.161756	 & 	-0.00158103	 \\
0.161764	 & 	-0.986785	 & 	-0.00941017	 \\
-3.79835e-05	 & 	-0.00954198	 & 	0.999954
\end{tabular}

\vtab
 EingenValues - Molecule A     \\
\begin{tabular}{|c c c|}
112.224	 & 	165.623	 & 	257.016	 \\
\end{tabular}
\columnbreak

Molecule B \
r-2-chlorobutane\_rotated\_out\_G09

\includegraphics[width=6cm]{../Comparisons/ImagesFromVMD/r-2-chlorobutane_rotated_out_G09.png}

Inertia Tensor - Molecule B \\
\begin{tabular}{|c c c|}
113.621	 & 	-8.52391	 & 	-0.0867928	 \\
-8.52391	 & 	164.234	 & 	-0.885702	 \\
-0.0867928	 & 	-0.885702	 & 	257.008
\end{tabular}

\vtab
 EingenVectors - Molecule B     \\
\begin{tabular}{|c c c|}
-0.986829	 & 	-0.161759	 & 	-0.00158112	 \\
0.161767	 & 	-0.986784	 & 	-0.00941025	 \\
-3.80279e-05	 & 	-0.00954208	 & 	0.999954
\end{tabular}

\vtab
 EingenValues - Molecule B     \\
\begin{tabular}{|c c c|}
112.224	 & 	165.623	 & 	257.016	 \\
\end{tabular}

\end{center}
\end{multicols}

\vtab[-5mm]
\begin{tabular}{*{2}{m{0.38\textwidth}}}
\begin{center}
\textcolor{NavyBlue}{\Large Enantiomers}
\end{center}
&
\begin{center}
\includegraphics[height=6.5cm]{../Comparisons/Vectors/inertia_tensor_of_r-2-chlorobutane_out_G09_inversion_and_r-2-chlorobutane_rotated_out_G09.png}
\end{center}
\end{tabular}

 \newpage

\vtab[-3cm]
\begin{center}
{\large AsymmetricEnantiomers \tab Número 7}
\end{center}
\begin{multicols}{2}
\begin{center}

Molecule A \
r-2-chlorobutane\_out\_G09\_inversion

\includegraphics[width=6cm]{../Comparisons/ImagesFromVMD/r-2-chlorobutane_out_G09_inversion.png}

Inertia Tensor - Molecule A \\
\begin{tabular}{|c c c|}
113.621	 & 	-8.52381	 & 	-0.0867846	 \\
-8.52381	 & 	164.234	 & 	-0.885691	 \\
-0.0867846	 & 	-0.885691	 & 	257.008
\end{tabular}

\vtab
 EingenVectors - Molecule A     \\
\begin{tabular}{|c c c|}
-0.986829	 & 	-0.161756	 & 	-0.00158103	 \\
0.161764	 & 	-0.986785	 & 	-0.00941017	 \\
-3.79835e-05	 & 	-0.00954198	 & 	0.999954
\end{tabular}

\vtab
 EingenValues - Molecule A     \\
\begin{tabular}{|c c c|}
112.224	 & 	165.623	 & 	257.016	 \\
\end{tabular}
\columnbreak

Molecule B \
s-2-chlorobutane\_out\_G09

\includegraphics[width=6cm]{../Comparisons/ImagesFromVMD/s-2-chlorobutane_out_G09.png}

Inertia Tensor - Molecule B \\
\begin{tabular}{|c c c|}
113.621	 & 	8.52396	 & 	0.0868155	 \\
8.52396	 & 	164.235	 & 	-0.885715	 \\
0.0868155	 & 	-0.885715	 & 	257.008
\end{tabular}

\vtab
 EingenVectors - Molecule B     \\
\begin{tabular}{|c c c|}
-0.98683	 & 	0.161755	 & 	0.00158125	 \\
-0.161763	 & 	-0.986785	 & 	-0.00941041	 \\
3.81715e-05	 & 	-0.00954226	 & 	0.999954
\end{tabular}

\vtab
 EingenValues - Molecule B     \\
\begin{tabular}{|c c c|}
112.224	 & 	165.624	 & 	257.017	 \\
\end{tabular}

\end{center}
\end{multicols}

\vtab[-5mm]
\begin{tabular}{*{2}{m{0.38\textwidth}}}
\begin{center}
\textcolor{NavyBlue}{\Large Equal}
\end{center}
&
\begin{center}
\includegraphics[height=6.5cm]{../Comparisons/Vectors/inertia_tensor_of_r-2-chlorobutane_out_G09_inversion_and_s-2-chlorobutane_out_G09.png}
\end{center}
\end{tabular}

 \newpage

\vtab[-3cm]
\begin{center}
{\large AsymmetricEnantiomers \tab Número 8}
\end{center}
\begin{multicols}{2}
\begin{center}

Molecule A \
r-2-chlorobutane\_out\_G09\_inversion

\includegraphics[width=6cm]{../Comparisons/ImagesFromVMD/r-2-chlorobutane_out_G09_inversion.png}

Inertia Tensor - Molecule A \\
\begin{tabular}{|c c c|}
113.621	 & 	-8.52381	 & 	-0.0867846	 \\
-8.52381	 & 	164.234	 & 	-0.885691	 \\
-0.0867846	 & 	-0.885691	 & 	257.008
\end{tabular}

\vtab
 EingenVectors - Molecule A     \\
\begin{tabular}{|c c c|}
-0.986829	 & 	-0.161756	 & 	-0.00158103	 \\
0.161764	 & 	-0.986785	 & 	-0.00941017	 \\
-3.79835e-05	 & 	-0.00954198	 & 	0.999954
\end{tabular}

\vtab
 EingenValues - Molecule A     \\
\begin{tabular}{|c c c|}
112.224	 & 	165.623	 & 	257.016	 \\
\end{tabular}
\columnbreak

Molecule B \
s-2-chlorobutane\_out\_G09\_inversion

\includegraphics[width=6cm]{../Comparisons/ImagesFromVMD/s-2-chlorobutane_out_G09_inversion.png}

Inertia Tensor - Molecule B \\
\begin{tabular}{|c c c|}
113.622	 & 	8.52398	 & 	0.0868204	 \\
8.52398	 & 	164.235	 & 	-0.88572	 \\
0.0868204	 & 	-0.88572	 & 	257.009
\end{tabular}

\vtab
 EingenVectors - Molecule B     \\
\begin{tabular}{|c c c|}
-0.986829	 & 	0.161757	 & 	0.0015813	 \\
-0.161765	 & 	-0.986785	 & 	-0.00941039	 \\
3.82053e-05	 & 	-0.00954225	 & 	0.999954
\end{tabular}

\vtab
 EingenValues - Molecule B     \\
\begin{tabular}{|c c c|}
112.224	 & 	165.624	 & 	257.018	 \\
\end{tabular}

\end{center}
\end{multicols}

\vtab[-5mm]
\begin{tabular}{*{2}{m{0.38\textwidth}}}
\begin{center}
\textcolor{NavyBlue}{\Large Enantiomers}
\end{center}
&
\begin{center}
\includegraphics[height=6.5cm]{../Comparisons/Vectors/inertia_tensor_of_r-2-chlorobutane_out_G09_inversion_and_s-2-chlorobutane_out_G09_inversion.png}
\end{center}
\end{tabular}

 \newpage

\vtab[-3cm]
\begin{center}
{\large AsymmetricEnantiomers \tab Número 9}
\end{center}
\begin{multicols}{2}
\begin{center}

Molecule A \
r-2-chlorobutane\_out\_G09\_inversion

\includegraphics[width=6cm]{../Comparisons/ImagesFromVMD/r-2-chlorobutane_out_G09_inversion.png}

Inertia Tensor - Molecule A \\
\begin{tabular}{|c c c|}
113.621	 & 	-8.52381	 & 	-0.0867846	 \\
-8.52381	 & 	164.234	 & 	-0.885691	 \\
-0.0867846	 & 	-0.885691	 & 	257.008
\end{tabular}

\vtab
 EingenVectors - Molecule A     \\
\begin{tabular}{|c c c|}
-0.986829	 & 	-0.161756	 & 	-0.00158103	 \\
0.161764	 & 	-0.986785	 & 	-0.00941017	 \\
-3.79835e-05	 & 	-0.00954198	 & 	0.999954
\end{tabular}

\vtab
 EingenValues - Molecule A     \\
\begin{tabular}{|c c c|}
112.224	 & 	165.623	 & 	257.016	 \\
\end{tabular}
\columnbreak

Molecule B \
s-2-chlorobutane\_rotated\_out\_G09

\includegraphics[width=6cm]{../Comparisons/ImagesFromVMD/s-2-chlorobutane_rotated_out_G09.png}

Inertia Tensor - Molecule B \\
\begin{tabular}{|c c c|}
113.621	 & 	8.52397	 & 	0.0867992	 \\
8.52397	 & 	164.235	 & 	-0.885706	 \\
0.0867992	 & 	-0.885706	 & 	257.008
\end{tabular}

\vtab
 EingenVectors - Molecule B     \\
\begin{tabular}{|c c c|}
-0.98683	 & 	0.161756	 & 	0.00158113	 \\
-0.161763	 & 	-0.986785	 & 	-0.00941034	 \\
3.80628e-05	 & 	-0.00954217	 & 	0.999954
\end{tabular}

\vtab
 EingenValues - Molecule B     \\
\begin{tabular}{|c c c|}
112.224	 & 	165.624	 & 	257.017	 \\
\end{tabular}

\end{center}
\end{multicols}

\vtab[-5mm]
\begin{tabular}{*{2}{m{0.38\textwidth}}}
\begin{center}
\textcolor{NavyBlue}{\Large Equal}
\end{center}
&
\begin{center}
\includegraphics[height=6.5cm]{../Comparisons/Vectors/inertia_tensor_of_r-2-chlorobutane_out_G09_inversion_and_s-2-chlorobutane_rotated_out_G09.png}
\end{center}
\end{tabular}

 \newpage

\vtab[-3cm]
\begin{center}
{\large AsymmetricEnantiomers \tab Número 10}
\end{center}
\begin{multicols}{2}
\begin{center}

Molecule A \
r-2-chlorobutane\_rotated\_out\_G09

\includegraphics[width=6cm]{../Comparisons/ImagesFromVMD/r-2-chlorobutane_rotated_out_G09.png}

Inertia Tensor - Molecule A \\
\begin{tabular}{|c c c|}
113.621	 & 	-8.52391	 & 	-0.0867928	 \\
-8.52391	 & 	164.234	 & 	-0.885702	 \\
-0.0867928	 & 	-0.885702	 & 	257.008
\end{tabular}

\vtab
 EingenVectors - Molecule A     \\
\begin{tabular}{|c c c|}
-0.986829	 & 	-0.161759	 & 	-0.00158112	 \\
0.161767	 & 	-0.986784	 & 	-0.00941025	 \\
-3.80279e-05	 & 	-0.00954208	 & 	0.999954
\end{tabular}

\vtab
 EingenValues - Molecule A     \\
\begin{tabular}{|c c c|}
112.224	 & 	165.623	 & 	257.016	 \\
\end{tabular}
\columnbreak

Molecule B \
s-2-chlorobutane\_out\_G09

\includegraphics[width=6cm]{../Comparisons/ImagesFromVMD/s-2-chlorobutane_out_G09.png}

Inertia Tensor - Molecule B \\
\begin{tabular}{|c c c|}
113.621	 & 	8.52396	 & 	0.0868155	 \\
8.52396	 & 	164.235	 & 	-0.885715	 \\
0.0868155	 & 	-0.885715	 & 	257.008
\end{tabular}

\vtab
 EingenVectors - Molecule B     \\
\begin{tabular}{|c c c|}
-0.98683	 & 	0.161755	 & 	0.00158125	 \\
-0.161763	 & 	-0.986785	 & 	-0.00941041	 \\
3.81715e-05	 & 	-0.00954226	 & 	0.999954
\end{tabular}

\vtab
 EingenValues - Molecule B     \\
\begin{tabular}{|c c c|}
112.224	 & 	165.624	 & 	257.017	 \\
\end{tabular}

\end{center}
\end{multicols}

\vtab[-5mm]
\begin{tabular}{*{2}{m{0.38\textwidth}}}
\begin{center}
\textcolor{NavyBlue}{\Large Enantiomers}
\end{center}
&
\begin{center}
\includegraphics[height=6.5cm]{../Comparisons/Vectors/inertia_tensor_of_r-2-chlorobutane_rotated_out_G09_and_s-2-chlorobutane_out_G09.png}
\end{center}
\end{tabular}

 \newpage

\vtab[-3cm]
\begin{center}
{\large AsymmetricEnantiomers \tab Número 11}
\end{center}
\begin{multicols}{2}
\begin{center}

Molecule A \
r-2-chlorobutane\_rotated\_out\_G09

\includegraphics[width=6cm]{../Comparisons/ImagesFromVMD/r-2-chlorobutane_rotated_out_G09.png}

Inertia Tensor - Molecule A \\
\begin{tabular}{|c c c|}
113.621	 & 	-8.52391	 & 	-0.0867928	 \\
-8.52391	 & 	164.234	 & 	-0.885702	 \\
-0.0867928	 & 	-0.885702	 & 	257.008
\end{tabular}

\vtab
 EingenVectors - Molecule A     \\
\begin{tabular}{|c c c|}
-0.986829	 & 	-0.161759	 & 	-0.00158112	 \\
0.161767	 & 	-0.986784	 & 	-0.00941025	 \\
-3.80279e-05	 & 	-0.00954208	 & 	0.999954
\end{tabular}

\vtab
 EingenValues - Molecule A     \\
\begin{tabular}{|c c c|}
112.224	 & 	165.623	 & 	257.016	 \\
\end{tabular}
\columnbreak

Molecule B \
s-2-chlorobutane\_out\_G09\_inversion

\includegraphics[width=6cm]{../Comparisons/ImagesFromVMD/s-2-chlorobutane_out_G09_inversion.png}

Inertia Tensor - Molecule B \\
\begin{tabular}{|c c c|}
113.622	 & 	8.52398	 & 	0.0868204	 \\
8.52398	 & 	164.235	 & 	-0.88572	 \\
0.0868204	 & 	-0.88572	 & 	257.009
\end{tabular}

\vtab
 EingenVectors - Molecule B     \\
\begin{tabular}{|c c c|}
-0.986829	 & 	0.161757	 & 	0.0015813	 \\
-0.161765	 & 	-0.986785	 & 	-0.00941039	 \\
3.82053e-05	 & 	-0.00954225	 & 	0.999954
\end{tabular}

\vtab
 EingenValues - Molecule B     \\
\begin{tabular}{|c c c|}
112.224	 & 	165.624	 & 	257.018	 \\
\end{tabular}

\end{center}
\end{multicols}

\vtab[-5mm]
\begin{tabular}{*{2}{m{0.38\textwidth}}}
\begin{center}
\textcolor{NavyBlue}{\Large Equal}
\end{center}
&
\begin{center}
\includegraphics[height=6.5cm]{../Comparisons/Vectors/inertia_tensor_of_r-2-chlorobutane_rotated_out_G09_and_s-2-chlorobutane_out_G09_inversion.png}
\end{center}
\end{tabular}

 \newpage

\vtab[-3cm]
\begin{center}
{\large AsymmetricEnantiomers \tab Número 12}
\end{center}
\begin{multicols}{2}
\begin{center}

Molecule A \
r-2-chlorobutane\_rotated\_out\_G09

\includegraphics[width=6cm]{../Comparisons/ImagesFromVMD/r-2-chlorobutane_rotated_out_G09.png}

Inertia Tensor - Molecule A \\
\begin{tabular}{|c c c|}
113.621	 & 	-8.52391	 & 	-0.0867928	 \\
-8.52391	 & 	164.234	 & 	-0.885702	 \\
-0.0867928	 & 	-0.885702	 & 	257.008
\end{tabular}

\vtab
 EingenVectors - Molecule A     \\
\begin{tabular}{|c c c|}
-0.986829	 & 	-0.161759	 & 	-0.00158112	 \\
0.161767	 & 	-0.986784	 & 	-0.00941025	 \\
-3.80279e-05	 & 	-0.00954208	 & 	0.999954
\end{tabular}

\vtab
 EingenValues - Molecule A     \\
\begin{tabular}{|c c c|}
112.224	 & 	165.623	 & 	257.016	 \\
\end{tabular}
\columnbreak

Molecule B \
s-2-chlorobutane\_rotated\_out\_G09

\includegraphics[width=6cm]{../Comparisons/ImagesFromVMD/s-2-chlorobutane_rotated_out_G09.png}

Inertia Tensor - Molecule B \\
\begin{tabular}{|c c c|}
113.621	 & 	8.52397	 & 	0.0867992	 \\
8.52397	 & 	164.235	 & 	-0.885706	 \\
0.0867992	 & 	-0.885706	 & 	257.008
\end{tabular}

\vtab
 EingenVectors - Molecule B     \\
\begin{tabular}{|c c c|}
-0.98683	 & 	0.161756	 & 	0.00158113	 \\
-0.161763	 & 	-0.986785	 & 	-0.00941034	 \\
3.80628e-05	 & 	-0.00954217	 & 	0.999954
\end{tabular}

\vtab
 EingenValues - Molecule B     \\
\begin{tabular}{|c c c|}
112.224	 & 	165.624	 & 	257.017	 \\
\end{tabular}

\end{center}
\end{multicols}

\vtab[-5mm]
\begin{tabular}{*{2}{m{0.38\textwidth}}}
\begin{center}
\textcolor{NavyBlue}{\Large Enantiomers}
\end{center}
&
\begin{center}
\includegraphics[height=6.5cm]{../Comparisons/Vectors/inertia_tensor_of_r-2-chlorobutane_rotated_out_G09_and_s-2-chlorobutane_rotated_out_G09.png}
\end{center}
\end{tabular}

 \newpage

\vtab[-3cm]
\begin{center}
{\large AsymmetricEnantiomers \tab Número 13}
\end{center}
\begin{multicols}{2}
\begin{center}

Molecule A \
s-2-chlorobutane\_out\_G09

\includegraphics[width=6cm]{../Comparisons/ImagesFromVMD/s-2-chlorobutane_out_G09.png}

Inertia Tensor - Molecule A \\
\begin{tabular}{|c c c|}
113.621	 & 	8.52396	 & 	0.0868155	 \\
8.52396	 & 	164.235	 & 	-0.885715	 \\
0.0868155	 & 	-0.885715	 & 	257.008
\end{tabular}

\vtab
 EingenVectors - Molecule A     \\
\begin{tabular}{|c c c|}
-0.98683	 & 	0.161755	 & 	0.00158125	 \\
-0.161763	 & 	-0.986785	 & 	-0.00941041	 \\
3.81715e-05	 & 	-0.00954226	 & 	0.999954
\end{tabular}

\vtab
 EingenValues - Molecule A     \\
\begin{tabular}{|c c c|}
112.224	 & 	165.624	 & 	257.017	 \\
\end{tabular}
\columnbreak

Molecule B \
s-2-chlorobutane\_out\_G09\_inversion

\includegraphics[width=6cm]{../Comparisons/ImagesFromVMD/s-2-chlorobutane_out_G09_inversion.png}

Inertia Tensor - Molecule B \\
\begin{tabular}{|c c c|}
113.622	 & 	8.52398	 & 	0.0868204	 \\
8.52398	 & 	164.235	 & 	-0.88572	 \\
0.0868204	 & 	-0.88572	 & 	257.009
\end{tabular}

\vtab
 EingenVectors - Molecule B     \\
\begin{tabular}{|c c c|}
-0.986829	 & 	0.161757	 & 	0.0015813	 \\
-0.161765	 & 	-0.986785	 & 	-0.00941039	 \\
3.82053e-05	 & 	-0.00954225	 & 	0.999954
\end{tabular}

\vtab
 EingenValues - Molecule B     \\
\begin{tabular}{|c c c|}
112.224	 & 	165.624	 & 	257.018	 \\
\end{tabular}

\end{center}
\end{multicols}

\vtab[-5mm]
\begin{tabular}{*{2}{m{0.38\textwidth}}}
\begin{center}
\textcolor{NavyBlue}{\Large Enantiomers}
\end{center}
&
\begin{center}
\includegraphics[height=6.5cm]{../Comparisons/Vectors/inertia_tensor_of_s-2-chlorobutane_out_G09_and_s-2-chlorobutane_out_G09_inversion.png}
\end{center}
\end{tabular}

 \newpage

\vtab[-3cm]
\begin{center}
{\large AsymmetricEnantiomers \tab Número 14}
\end{center}
\begin{multicols}{2}
\begin{center}

Molecule A \
s-2-chlorobutane\_out\_G09

\includegraphics[width=6cm]{../Comparisons/ImagesFromVMD/s-2-chlorobutane_out_G09.png}

Inertia Tensor - Molecule A \\
\begin{tabular}{|c c c|}
113.621	 & 	8.52396	 & 	0.0868155	 \\
8.52396	 & 	164.235	 & 	-0.885715	 \\
0.0868155	 & 	-0.885715	 & 	257.008
\end{tabular}

\vtab
 EingenVectors - Molecule A     \\
\begin{tabular}{|c c c|}
-0.98683	 & 	0.161755	 & 	0.00158125	 \\
-0.161763	 & 	-0.986785	 & 	-0.00941041	 \\
3.81715e-05	 & 	-0.00954226	 & 	0.999954
\end{tabular}

\vtab
 EingenValues - Molecule A     \\
\begin{tabular}{|c c c|}
112.224	 & 	165.624	 & 	257.017	 \\
\end{tabular}
\columnbreak

Molecule B \
s-2-chlorobutane\_rotated\_out\_G09

\includegraphics[width=6cm]{../Comparisons/ImagesFromVMD/s-2-chlorobutane_rotated_out_G09.png}

Inertia Tensor - Molecule B \\
\begin{tabular}{|c c c|}
113.621	 & 	8.52397	 & 	0.0867992	 \\
8.52397	 & 	164.235	 & 	-0.885706	 \\
0.0867992	 & 	-0.885706	 & 	257.008
\end{tabular}

\vtab
 EingenVectors - Molecule B     \\
\begin{tabular}{|c c c|}
-0.98683	 & 	0.161756	 & 	0.00158113	 \\
-0.161763	 & 	-0.986785	 & 	-0.00941034	 \\
3.80628e-05	 & 	-0.00954217	 & 	0.999954
\end{tabular}

\vtab
 EingenValues - Molecule B     \\
\begin{tabular}{|c c c|}
112.224	 & 	165.624	 & 	257.017	 \\
\end{tabular}

\end{center}
\end{multicols}

\vtab[-5mm]
\begin{tabular}{*{2}{m{0.38\textwidth}}}
\begin{center}
\textcolor{NavyBlue}{\Large Equal}
\end{center}
&
\begin{center}
\includegraphics[height=6.5cm]{../Comparisons/Vectors/inertia_tensor_of_s-2-chlorobutane_out_G09_and_s-2-chlorobutane_rotated_out_G09.png}
\end{center}
\end{tabular}

 \newpage

\vtab[-3cm]
\begin{center}
{\large AsymmetricEnantiomers \tab Número 15}
\end{center}
\begin{multicols}{2}
\begin{center}

Molecule A \
s-2-chlorobutane\_out\_G09\_inversion

\includegraphics[width=6cm]{../Comparisons/ImagesFromVMD/s-2-chlorobutane_out_G09_inversion.png}

Inertia Tensor - Molecule A \\
\begin{tabular}{|c c c|}
113.622	 & 	8.52398	 & 	0.0868204	 \\
8.52398	 & 	164.235	 & 	-0.88572	 \\
0.0868204	 & 	-0.88572	 & 	257.009
\end{tabular}

\vtab
 EingenVectors - Molecule A     \\
\begin{tabular}{|c c c|}
-0.986829	 & 	0.161757	 & 	0.0015813	 \\
-0.161765	 & 	-0.986785	 & 	-0.00941039	 \\
3.82053e-05	 & 	-0.00954225	 & 	0.999954
\end{tabular}

\vtab
 EingenValues - Molecule A     \\
\begin{tabular}{|c c c|}
112.224	 & 	165.624	 & 	257.018	 \\
\end{tabular}
\columnbreak

Molecule B \
s-2-chlorobutane\_rotated\_out\_G09

\includegraphics[width=6cm]{../Comparisons/ImagesFromVMD/s-2-chlorobutane_rotated_out_G09.png}

Inertia Tensor - Molecule B \\
\begin{tabular}{|c c c|}
113.621	 & 	8.52397	 & 	0.0867992	 \\
8.52397	 & 	164.235	 & 	-0.885706	 \\
0.0867992	 & 	-0.885706	 & 	257.008
\end{tabular}

\vtab
 EingenVectors - Molecule B     \\
\begin{tabular}{|c c c|}
-0.98683	 & 	0.161756	 & 	0.00158113	 \\
-0.161763	 & 	-0.986785	 & 	-0.00941034	 \\
3.80628e-05	 & 	-0.00954217	 & 	0.999954
\end{tabular}

\vtab
 EingenValues - Molecule B     \\
\begin{tabular}{|c c c|}
112.224	 & 	165.624	 & 	257.017	 \\
\end{tabular}

\end{center}
\end{multicols}

\vtab[-5mm]
\begin{tabular}{*{2}{m{0.38\textwidth}}}
\begin{center}
\textcolor{NavyBlue}{\Large Enantiomers}
\end{center}
&
\begin{center}
\includegraphics[height=6.5cm]{../Comparisons/Vectors/inertia_tensor_of_s-2-chlorobutane_out_G09_inversion_and_s-2-chlorobutane_rotated_out_G09.png}
\end{center}
\end{tabular}

 \newpage

\vtab[-3cm]
\begin{center}
{\large ClustersCu \tab Número 16}
\end{center}
\begin{multicols}{2}
\begin{center}

Molecule A \
G\_08\_Cu

\includegraphics[width=6cm]{../Comparisons/ImagesFromVMD/G_08_Cu.png}

Inertia Tensor - Molecule A \\
\begin{tabular}{|c c c|}
1654.82	 & 	51.548	 & 	-216.332	 \\
51.548	 & 	1555.14	 & 	-242.791	 \\
-216.332	 & 	-242.791	 & 	1259.16
\end{tabular}

\vtab
 EingenVectors - Molecule A     \\
\begin{tabular}{|c c c|}
0.286837	 & 	0.406743	 & 	0.867343	 \\
0.655465	 & 	-0.743609	 & 	0.13195	 \\
0.698634	 & 	0.530665	 & 	-0.479901
\end{tabular}

\vtab
 EingenValues - Molecule A     \\
\begin{tabular}{|c c c|}
1073.76	 & 	1552.79	 & 	1842.57	 \\
\end{tabular}
\columnbreak

Molecule B \
G\_08\_Cu\_AFTER\_DFT

\includegraphics[width=6cm]{../Comparisons/ImagesFromVMD/G_08_Cu_AFTER_DFT.png}

Inertia Tensor - Molecule B \\
\begin{tabular}{|c c c|}
1597.81	 & 	73.7499	 & 	-186.12	 \\
73.7499	 & 	1509.17	 & 	-199.941	 \\
-186.12	 & 	-199.941	 & 	1250.28
\end{tabular}

\vtab
 EingenVectors - Molecule B     \\
\begin{tabular}{|c c c|}
0.274867	 & 	0.38536	 & 	0.880878	 \\
0.652168	 & 	-0.747915	 & 	0.123691	 \\
0.706488	 & 	0.540482	 & 	-0.456897
\end{tabular}

\vtab
 EingenValues - Molecule B     \\
\begin{tabular}{|c c c|}
1104.74	 & 	1477.93	 & 	1774.59	 \\
\end{tabular}

\end{center}
\end{multicols}

\vtab[-5mm]
\begin{tabular}{*{2}{m{0.38\textwidth}}}
\begin{center}
\textcolor{NavyBlue}{\Large Different}
\end{center}
&
\begin{center}
\includegraphics[height=6.5cm]{../Comparisons/Vectors/inertia_tensor_of_G_08_Cu_and_G_08_Cu_AFTER_DFT.png}
\end{center}
\end{tabular}

 \newpage

\vtab[-3cm]
\begin{center}
{\large ClustersCu \tab Número 17}
\end{center}
\begin{multicols}{2}
\begin{center}
Molecule A \\ 
G\_08\_Cu
\includegraphics[width=8cm]{../Comparisons/ImagesFromVMD/G_08_Cu.png}
\\
\vtab

\columnbreak
Molecule B \\ 
G\_09\_Cu
\includegraphics[width=8cm]{../Comparisons/ImagesFromVMD/G_09_Cu.png}
\\
\vtab


\end{center}
\end{multicols}
\begin{center}
\textcolor{NavyBlue}{\Large Different}
\end{center}

 \newpage

\vtab[-3cm]
\begin{center}
{\large ClustersCu \tab Número 18}
\end{center}
\begin{multicols}{2}
\begin{center}
Molecule A \\ 
G\_08\_Cu
\includegraphics[width=8cm]{../Comparisons/ImagesFromVMD/G_08_Cu.png}
\\
\vtab

\columnbreak
Molecule B \\ 
G\_09\_Cu\_AFTER\_DFT
\includegraphics[width=8cm]{../Comparisons/ImagesFromVMD/G_09_Cu_AFTER_DFT.png}
\\
\vtab


\end{center}
\end{multicols}
\begin{center}
\textcolor{NavyBlue}{\Large Different}
\end{center}

 \newpage

\vtab[-3cm]
\begin{center}
{\large ClustersCu \tab Número 19}
\end{center}
\begin{multicols}{2}
\begin{center}
Molecule A \\ 
G\_08\_Cu
\includegraphics[width=8cm]{../Comparisons/ImagesFromVMD/G_08_Cu.png}
\\
\vtab

\columnbreak
Molecule B \\ 
G\_10\_Cu
\includegraphics[width=8cm]{../Comparisons/ImagesFromVMD/G_10_Cu.png}
\\
\vtab


\end{center}
\end{multicols}
\begin{center}
\textcolor{NavyBlue}{\Large Different}
\end{center}

 \newpage

\vtab[-3cm]
\begin{center}
{\large ClustersCu \tab Número 20}
\end{center}
\begin{multicols}{2}
\begin{center}
Molecule A \\ 
G\_08\_Cu
\includegraphics[width=8cm]{../Comparisons/ImagesFromVMD/G_08_Cu.png}
\\
\vtab

\columnbreak
Molecule B \\ 
G\_11\_Cu
\includegraphics[width=8cm]{../Comparisons/ImagesFromVMD/G_11_Cu.png}
\\
\vtab


\end{center}
\end{multicols}
\begin{center}
\textcolor{NavyBlue}{\Large Different}
\end{center}

 \newpage

\vtab[-3cm]
\begin{center}
{\large ClustersCu \tab Número 21}
\end{center}
\begin{multicols}{2}
\begin{center}

Molecule A \
G\_08\_Cu

\includegraphics[width=6cm]{../Comparisons/ImagesFromVMD/G_08_Cu.png}

Inertia Tensor - Molecule A \\
\begin{tabular}{|c c c|}
1654.82	 & 	51.548	 & 	-216.332	 \\
51.548	 & 	1555.14	 & 	-242.791	 \\
-216.332	 & 	-242.791	 & 	1259.16
\end{tabular}

\vtab
 EingenVectors - Molecule A     \\
\begin{tabular}{|c c c|}
0.286837	 & 	0.406743	 & 	0.867343	 \\
0.655465	 & 	-0.743609	 & 	0.13195	 \\
0.698634	 & 	0.530665	 & 	-0.479901
\end{tabular}

\vtab
 EingenValues - Molecule A     \\
\begin{tabular}{|c c c|}
1073.76	 & 	1552.79	 & 	1842.57	 \\
\end{tabular}
\columnbreak

Molecule B \
SC\_08\_Cu

\includegraphics[width=6cm]{../Comparisons/ImagesFromVMD/SC_08_Cu.png}

Inertia Tensor - Molecule B \\
\begin{tabular}{|c c c|}
1331.27	 & 	0.00958178	 & 	0.0062677	 \\
0.00958178	 & 	1331.26	 & 	0.00713637	 \\
0.0062677	 & 	0.00713637	 & 	1331.28
\end{tabular}

\vtab
 EingenVectors - Molecule B     \\
\begin{tabular}{|c c c|}
-0.515886	 & 	0.847718	 & 	-0.123437	 \\
-0.684575	 & 	-0.321331	 & 	0.654296	 \\
0.514994	 & 	0.422044	 & 	0.746096
\end{tabular}

\vtab
 EingenValues - Molecule B     \\
\begin{tabular}{|c c c|}
1331.25	 & 	1331.27	 & 	1331.28	 \\
\end{tabular}

\end{center}
\end{multicols}

\vtab[-5mm]
\begin{tabular}{*{2}{m{0.38\textwidth}}}
\begin{center}
\textcolor{NavyBlue}{\Large Different}
\end{center}
&
\begin{center}
\includegraphics[height=6.5cm]{../Comparisons/Vectors/inertia_tensor_of_G_08_Cu_and_SC_08_Cu.png}
\end{center}
\end{tabular}

 \newpage

\vtab[-3cm]
\begin{center}
{\large ClustersCu \tab Número 22}
\end{center}
\begin{multicols}{2}
\begin{center}

Molecule A \
G\_08\_Cu

\includegraphics[width=6cm]{../Comparisons/ImagesFromVMD/G_08_Cu.png}

Inertia Tensor - Molecule A \\
\begin{tabular}{|c c c|}
1654.82	 & 	51.548	 & 	-216.332	 \\
51.548	 & 	1555.14	 & 	-242.791	 \\
-216.332	 & 	-242.791	 & 	1259.16
\end{tabular}

\vtab
 EingenVectors - Molecule A     \\
\begin{tabular}{|c c c|}
0.286837	 & 	0.406743	 & 	0.867343	 \\
0.655465	 & 	-0.743609	 & 	0.13195	 \\
0.698634	 & 	0.530665	 & 	-0.479901
\end{tabular}

\vtab
 EingenValues - Molecule A     \\
\begin{tabular}{|c c c|}
1073.76	 & 	1552.79	 & 	1842.57	 \\
\end{tabular}
\columnbreak

Molecule B \
SC\_08\_Cu\_AFTER\_DFT

\includegraphics[width=6cm]{../Comparisons/ImagesFromVMD/SC_08_Cu_AFTER_DFT.png}

Inertia Tensor - Molecule B \\
\begin{tabular}{|c c c|}
1394.63	 & 	1.12138	 & 	-5.89124	 \\
1.12138	 & 	1385.89	 & 	0.251188	 \\
-5.89124	 & 	0.251188	 & 	1398.86
\end{tabular}

\vtab
 EingenVectors - Molecule B     \\
\begin{tabular}{|c c c|}
0.189307	 & 	-0.97651	 & 	0.102909	 \\
0.794609	 & 	0.213922	 & 	0.568184	 \\
-0.576853	 & 	-0.0257885	 & 	0.816441
\end{tabular}

\vtab
 EingenValues - Molecule B     \\
\begin{tabular}{|c c c|}
1385.64	 & 	1390.72	 & 	1403.02	 \\
\end{tabular}

\end{center}
\end{multicols}

\vtab[-5mm]
\begin{tabular}{*{2}{m{0.38\textwidth}}}
\begin{center}
\textcolor{NavyBlue}{\Large Different}
\end{center}
&
\begin{center}
\includegraphics[height=6.5cm]{../Comparisons/Vectors/inertia_tensor_of_G_08_Cu_and_SC_08_Cu_AFTER_DFT.png}
\end{center}
\end{tabular}

 \newpage

\vtab[-3cm]
\begin{center}
{\large ClustersCu \tab Número 23}
\end{center}
\begin{multicols}{2}
\begin{center}
Molecule A \\ 
G\_08\_Cu
\includegraphics[width=8cm]{../Comparisons/ImagesFromVMD/G_08_Cu.png}
\\
\vtab

\columnbreak
Molecule B \\ 
SC\_09\_Cu
\includegraphics[width=8cm]{../Comparisons/ImagesFromVMD/SC_09_Cu.png}
\\
\vtab


\end{center}
\end{multicols}
\begin{center}
\textcolor{NavyBlue}{\Large Different}
\end{center}

 \newpage

\vtab[-3cm]
\begin{center}
{\large ClustersCu \tab Número 24}
\end{center}
\begin{multicols}{2}
\begin{center}
Molecule A \\ 
G\_08\_Cu
\includegraphics[width=8cm]{../Comparisons/ImagesFromVMD/G_08_Cu.png}
\\
\vtab

\columnbreak
Molecule B \\ 
SC\_09\_Cu\_AFTER\_DFT
\includegraphics[width=8cm]{../Comparisons/ImagesFromVMD/SC_09_Cu_AFTER_DFT.png}
\\
\vtab


\end{center}
\end{multicols}
\begin{center}
\textcolor{NavyBlue}{\Large Different}
\end{center}

 \newpage

\vtab[-3cm]
\begin{center}
{\large ClustersCu \tab Número 25}
\end{center}
\begin{multicols}{2}
\begin{center}

Molecule A \
G\_08\_Cu

\includegraphics[width=6cm]{../Comparisons/ImagesFromVMD/G_08_Cu.png}

Inertia Tensor - Molecule A \\
\begin{tabular}{|c c c|}
1654.82	 & 	51.548	 & 	-216.332	 \\
51.548	 & 	1555.14	 & 	-242.791	 \\
-216.332	 & 	-242.791	 & 	1259.16
\end{tabular}

\vtab
 EingenVectors - Molecule A     \\
\begin{tabular}{|c c c|}
0.286837	 & 	0.406743	 & 	0.867343	 \\
0.655465	 & 	-0.743609	 & 	0.13195	 \\
0.698634	 & 	0.530665	 & 	-0.479901
\end{tabular}

\vtab
 EingenValues - Molecule A     \\
\begin{tabular}{|c c c|}
1073.76	 & 	1552.79	 & 	1842.57	 \\
\end{tabular}
\columnbreak

Molecule B \
lj\_08\_Cu

\includegraphics[width=6cm]{../Comparisons/ImagesFromVMD/lj_08_Cu.png}

Inertia Tensor - Molecule B \\
\begin{tabular}{|c c c|}
1617.21	 & 	-91.1093	 & 	-437.808	 \\
-91.1093	 & 	1798.78	 & 	-179.618	 \\
-437.808	 & 	-179.618	 & 	957.828
\end{tabular}

\vtab
 EingenVectors - Molecule B     \\
\begin{tabular}{|c c c|}
0.43919	 & 	0.180681	 & 	0.880038	 \\
-0.601213	 & 	-0.668782	 & 	0.437348	 \\
-0.667574	 & 	0.721169	 & 	0.185095
\end{tabular}

\vtab
 EingenValues - Molecule B     \\
\begin{tabular}{|c c c|}
702.459	 & 	1834.34	 & 	1837.02	 \\
\end{tabular}

\end{center}
\end{multicols}

\vtab[-5mm]
\begin{tabular}{*{2}{m{0.38\textwidth}}}
\begin{center}
\textcolor{NavyBlue}{\Large Different}
\end{center}
&
\begin{center}
\includegraphics[height=6.5cm]{../Comparisons/Vectors/inertia_tensor_of_G_08_Cu_and_lj_08_Cu.png}
\end{center}
\end{tabular}

 \newpage

\vtab[-3cm]
\begin{center}
{\large ClustersCu \tab Número 26}
\end{center}
\begin{multicols}{2}
\begin{center}
Molecule A \\ 
G\_08\_Cu
\includegraphics[width=8cm]{../Comparisons/ImagesFromVMD/G_08_Cu.png}
\\
\vtab

\columnbreak
Molecule B \\ 
lj\_09\_Cu\_AFTER\_DFT
\includegraphics[width=8cm]{../Comparisons/ImagesFromVMD/lj_09_Cu_AFTER_DFT.png}
\\
\vtab


\end{center}
\end{multicols}
\begin{center}
\textcolor{NavyBlue}{\Large Different}
\end{center}

 \newpage

\vtab[-3cm]
\begin{center}
{\large ClustersCu \tab Número 27}
\end{center}
\begin{multicols}{2}
\begin{center}
Molecule A \\ 
G\_08\_Cu\_AFTER\_DFT
\includegraphics[width=8cm]{../Comparisons/ImagesFromVMD/G_08_Cu_AFTER_DFT.png}
\\
\vtab

\columnbreak
Molecule B \\ 
G\_09\_Cu
\includegraphics[width=8cm]{../Comparisons/ImagesFromVMD/G_09_Cu.png}
\\
\vtab


\end{center}
\end{multicols}
\begin{center}
\textcolor{NavyBlue}{\Large Different}
\end{center}

 \newpage

\vtab[-3cm]
\begin{center}
{\large ClustersCu \tab Número 28}
\end{center}
\begin{multicols}{2}
\begin{center}
Molecule A \\ 
G\_08\_Cu\_AFTER\_DFT
\includegraphics[width=8cm]{../Comparisons/ImagesFromVMD/G_08_Cu_AFTER_DFT.png}
\\
\vtab

\columnbreak
Molecule B \\ 
G\_09\_Cu\_AFTER\_DFT
\includegraphics[width=8cm]{../Comparisons/ImagesFromVMD/G_09_Cu_AFTER_DFT.png}
\\
\vtab


\end{center}
\end{multicols}
\begin{center}
\textcolor{NavyBlue}{\Large Different}
\end{center}

 \newpage

\vtab[-3cm]
\begin{center}
{\large ClustersCu \tab Número 29}
\end{center}
\begin{multicols}{2}
\begin{center}
Molecule A \\ 
G\_08\_Cu\_AFTER\_DFT
\includegraphics[width=8cm]{../Comparisons/ImagesFromVMD/G_08_Cu_AFTER_DFT.png}
\\
\vtab

\columnbreak
Molecule B \\ 
G\_10\_Cu
\includegraphics[width=8cm]{../Comparisons/ImagesFromVMD/G_10_Cu.png}
\\
\vtab


\end{center}
\end{multicols}
\begin{center}
\textcolor{NavyBlue}{\Large Different}
\end{center}

 \newpage

\vtab[-3cm]
\begin{center}
{\large ClustersCu \tab Número 30}
\end{center}
\begin{multicols}{2}
\begin{center}
Molecule A \\ 
G\_08\_Cu\_AFTER\_DFT
\includegraphics[width=8cm]{../Comparisons/ImagesFromVMD/G_08_Cu_AFTER_DFT.png}
\\
\vtab

\columnbreak
Molecule B \\ 
G\_11\_Cu
\includegraphics[width=8cm]{../Comparisons/ImagesFromVMD/G_11_Cu.png}
\\
\vtab


\end{center}
\end{multicols}
\begin{center}
\textcolor{NavyBlue}{\Large Different}
\end{center}

 \newpage

\vtab[-3cm]
\begin{center}
{\large ClustersCu \tab Número 31}
\end{center}
\begin{multicols}{2}
\begin{center}

Molecule A \
G\_08\_Cu\_AFTER\_DFT

\includegraphics[width=6cm]{../Comparisons/ImagesFromVMD/G_08_Cu_AFTER_DFT.png}

Inertia Tensor - Molecule A \\
\begin{tabular}{|c c c|}
1597.81	 & 	73.7499	 & 	-186.12	 \\
73.7499	 & 	1509.17	 & 	-199.941	 \\
-186.12	 & 	-199.941	 & 	1250.28
\end{tabular}

\vtab
 EingenVectors - Molecule A     \\
\begin{tabular}{|c c c|}
0.274867	 & 	0.38536	 & 	0.880878	 \\
0.652168	 & 	-0.747915	 & 	0.123691	 \\
0.706488	 & 	0.540482	 & 	-0.456897
\end{tabular}

\vtab
 EingenValues - Molecule A     \\
\begin{tabular}{|c c c|}
1104.74	 & 	1477.93	 & 	1774.59	 \\
\end{tabular}
\columnbreak

Molecule B \
SC\_08\_Cu

\includegraphics[width=6cm]{../Comparisons/ImagesFromVMD/SC_08_Cu.png}

Inertia Tensor - Molecule B \\
\begin{tabular}{|c c c|}
1331.27	 & 	0.00958178	 & 	0.0062677	 \\
0.00958178	 & 	1331.26	 & 	0.00713637	 \\
0.0062677	 & 	0.00713637	 & 	1331.28
\end{tabular}

\vtab
 EingenVectors - Molecule B     \\
\begin{tabular}{|c c c|}
-0.515886	 & 	0.847718	 & 	-0.123437	 \\
-0.684575	 & 	-0.321331	 & 	0.654296	 \\
0.514994	 & 	0.422044	 & 	0.746096
\end{tabular}

\vtab
 EingenValues - Molecule B     \\
\begin{tabular}{|c c c|}
1331.25	 & 	1331.27	 & 	1331.28	 \\
\end{tabular}

\end{center}
\end{multicols}

\vtab[-5mm]
\begin{tabular}{*{2}{m{0.38\textwidth}}}
\begin{center}
\textcolor{NavyBlue}{\Large Different}
\end{center}
&
\begin{center}
\includegraphics[height=6.5cm]{../Comparisons/Vectors/inertia_tensor_of_G_08_Cu_AFTER_DFT_and_SC_08_Cu.png}
\end{center}
\end{tabular}

 \newpage

\vtab[-3cm]
\begin{center}
{\large ClustersCu \tab Número 32}
\end{center}
\begin{multicols}{2}
\begin{center}

Molecule A \
G\_08\_Cu\_AFTER\_DFT

\includegraphics[width=6cm]{../Comparisons/ImagesFromVMD/G_08_Cu_AFTER_DFT.png}

Inertia Tensor - Molecule A \\
\begin{tabular}{|c c c|}
1597.81	 & 	73.7499	 & 	-186.12	 \\
73.7499	 & 	1509.17	 & 	-199.941	 \\
-186.12	 & 	-199.941	 & 	1250.28
\end{tabular}

\vtab
 EingenVectors - Molecule A     \\
\begin{tabular}{|c c c|}
0.274867	 & 	0.38536	 & 	0.880878	 \\
0.652168	 & 	-0.747915	 & 	0.123691	 \\
0.706488	 & 	0.540482	 & 	-0.456897
\end{tabular}

\vtab
 EingenValues - Molecule A     \\
\begin{tabular}{|c c c|}
1104.74	 & 	1477.93	 & 	1774.59	 \\
\end{tabular}
\columnbreak

Molecule B \
SC\_08\_Cu\_AFTER\_DFT

\includegraphics[width=6cm]{../Comparisons/ImagesFromVMD/SC_08_Cu_AFTER_DFT.png}

Inertia Tensor - Molecule B \\
\begin{tabular}{|c c c|}
1394.63	 & 	1.12138	 & 	-5.89124	 \\
1.12138	 & 	1385.89	 & 	0.251188	 \\
-5.89124	 & 	0.251188	 & 	1398.86
\end{tabular}

\vtab
 EingenVectors - Molecule B     \\
\begin{tabular}{|c c c|}
0.189307	 & 	-0.97651	 & 	0.102909	 \\
0.794609	 & 	0.213922	 & 	0.568184	 \\
-0.576853	 & 	-0.0257885	 & 	0.816441
\end{tabular}

\vtab
 EingenValues - Molecule B     \\
\begin{tabular}{|c c c|}
1385.64	 & 	1390.72	 & 	1403.02	 \\
\end{tabular}

\end{center}
\end{multicols}

\vtab[-5mm]
\begin{tabular}{*{2}{m{0.38\textwidth}}}
\begin{center}
\textcolor{NavyBlue}{\Large Different}
\end{center}
&
\begin{center}
\includegraphics[height=6.5cm]{../Comparisons/Vectors/inertia_tensor_of_G_08_Cu_AFTER_DFT_and_SC_08_Cu_AFTER_DFT.png}
\end{center}
\end{tabular}

 \newpage

\vtab[-3cm]
\begin{center}
{\large ClustersCu \tab Número 33}
\end{center}
\begin{multicols}{2}
\begin{center}
Molecule A \\ 
G\_08\_Cu\_AFTER\_DFT
\includegraphics[width=8cm]{../Comparisons/ImagesFromVMD/G_08_Cu_AFTER_DFT.png}
\\
\vtab

\columnbreak
Molecule B \\ 
SC\_09\_Cu
\includegraphics[width=8cm]{../Comparisons/ImagesFromVMD/SC_09_Cu.png}
\\
\vtab


\end{center}
\end{multicols}
\begin{center}
\textcolor{NavyBlue}{\Large Different}
\end{center}

 \newpage

\vtab[-3cm]
\begin{center}
{\large ClustersCu \tab Número 34}
\end{center}
\begin{multicols}{2}
\begin{center}
Molecule A \\ 
G\_08\_Cu\_AFTER\_DFT
\includegraphics[width=8cm]{../Comparisons/ImagesFromVMD/G_08_Cu_AFTER_DFT.png}
\\
\vtab

\columnbreak
Molecule B \\ 
SC\_09\_Cu\_AFTER\_DFT
\includegraphics[width=8cm]{../Comparisons/ImagesFromVMD/SC_09_Cu_AFTER_DFT.png}
\\
\vtab


\end{center}
\end{multicols}
\begin{center}
\textcolor{NavyBlue}{\Large Different}
\end{center}

 \newpage

\vtab[-3cm]
\begin{center}
{\large ClustersCu \tab Número 35}
\end{center}
\begin{multicols}{2}
\begin{center}

Molecule A \
G\_08\_Cu\_AFTER\_DFT

\includegraphics[width=6cm]{../Comparisons/ImagesFromVMD/G_08_Cu_AFTER_DFT.png}

Inertia Tensor - Molecule A \\
\begin{tabular}{|c c c|}
1597.81	 & 	73.7499	 & 	-186.12	 \\
73.7499	 & 	1509.17	 & 	-199.941	 \\
-186.12	 & 	-199.941	 & 	1250.28
\end{tabular}

\vtab
 EingenVectors - Molecule A     \\
\begin{tabular}{|c c c|}
0.274867	 & 	0.38536	 & 	0.880878	 \\
0.652168	 & 	-0.747915	 & 	0.123691	 \\
0.706488	 & 	0.540482	 & 	-0.456897
\end{tabular}

\vtab
 EingenValues - Molecule A     \\
\begin{tabular}{|c c c|}
1104.74	 & 	1477.93	 & 	1774.59	 \\
\end{tabular}
\columnbreak

Molecule B \
lj\_08\_Cu

\includegraphics[width=6cm]{../Comparisons/ImagesFromVMD/lj_08_Cu.png}

Inertia Tensor - Molecule B \\
\begin{tabular}{|c c c|}
1617.21	 & 	-91.1093	 & 	-437.808	 \\
-91.1093	 & 	1798.78	 & 	-179.618	 \\
-437.808	 & 	-179.618	 & 	957.828
\end{tabular}

\vtab
 EingenVectors - Molecule B     \\
\begin{tabular}{|c c c|}
0.43919	 & 	0.180681	 & 	0.880038	 \\
-0.601213	 & 	-0.668782	 & 	0.437348	 \\
-0.667574	 & 	0.721169	 & 	0.185095
\end{tabular}

\vtab
 EingenValues - Molecule B     \\
\begin{tabular}{|c c c|}
702.459	 & 	1834.34	 & 	1837.02	 \\
\end{tabular}

\end{center}
\end{multicols}

\vtab[-5mm]
\begin{tabular}{*{2}{m{0.38\textwidth}}}
\begin{center}
\textcolor{NavyBlue}{\Large Different}
\end{center}
&
\begin{center}
\includegraphics[height=6.5cm]{../Comparisons/Vectors/inertia_tensor_of_G_08_Cu_AFTER_DFT_and_lj_08_Cu.png}
\end{center}
\end{tabular}

 \newpage

\vtab[-3cm]
\begin{center}
{\large ClustersCu \tab Número 36}
\end{center}
\begin{multicols}{2}
\begin{center}
Molecule A \\ 
G\_08\_Cu\_AFTER\_DFT
\includegraphics[width=8cm]{../Comparisons/ImagesFromVMD/G_08_Cu_AFTER_DFT.png}
\\
\vtab

\columnbreak
Molecule B \\ 
lj\_09\_Cu\_AFTER\_DFT
\includegraphics[width=8cm]{../Comparisons/ImagesFromVMD/lj_09_Cu_AFTER_DFT.png}
\\
\vtab


\end{center}
\end{multicols}
\begin{center}
\textcolor{NavyBlue}{\Large Different}
\end{center}

 \newpage

\vtab[-3cm]
\begin{center}
{\large ClustersCu \tab Número 37}
\end{center}
\begin{multicols}{2}
\begin{center}

Molecule A \
G\_09\_Cu

\includegraphics[width=6cm]{../Comparisons/ImagesFromVMD/G_09_Cu.png}

Inertia Tensor - Molecule A \\
\begin{tabular}{|c c c|}
1643.08	 & 	-103.447	 & 	113.093	 \\
-103.447	 & 	2240.72	 & 	205.863	 \\
113.093	 & 	205.863	 & 	1748.04
\end{tabular}

\vtab
 EingenVectors - Molecule A     \\
\begin{tabular}{|c c c|}
-0.747587	 & 	-0.277171	 & 	0.603564	 \\
0.657992	 & 	-0.185481	 & 	0.729825	 \\
-0.0903363	 & 	0.942748	 & 	0.321039
\end{tabular}

\vtab
 EingenValues - Molecule A     \\
\begin{tabular}{|c c c|}
1513.42	 & 	1797.68	 & 	2320.74	 \\
\end{tabular}
\columnbreak

Molecule B \
G\_09\_Cu\_AFTER\_DFT

\includegraphics[width=6cm]{../Comparisons/ImagesFromVMD/G_09_Cu_AFTER_DFT.png}

Inertia Tensor - Molecule B \\
\begin{tabular}{|c c c|}
1625.38	 & 	-92.1303	 & 	110.251	 \\
-92.1303	 & 	2185.43	 & 	185.921	 \\
110.251	 & 	185.921	 & 	1731.97
\end{tabular}

\vtab
 EingenVectors - Molecule B     \\
\begin{tabular}{|c c c|}
-0.760255	 & 	-0.26577	 & 	0.592773	 \\
0.644373	 & 	-0.192718	 & 	0.740029	 \\
-0.0824395	 & 	0.944577	 & 	0.317769
\end{tabular}

\vtab
 EingenValues - Molecule B     \\
\begin{tabular}{|c c c|}
1507.21	 & 	1779.55	 & 	2256.02	 \\
\end{tabular}

\end{center}
\end{multicols}

\vtab[-5mm]
\begin{tabular}{*{2}{m{0.38\textwidth}}}
\begin{center}
\textcolor{NavyBlue}{\Large Different}
\end{center}
&
\begin{center}
\includegraphics[height=6.5cm]{../Comparisons/Vectors/inertia_tensor_of_G_09_Cu_and_G_09_Cu_AFTER_DFT.png}
\end{center}
\end{tabular}

 \newpage

\vtab[-3cm]
\begin{center}
{\large ClustersCu \tab Número 38}
\end{center}
\begin{multicols}{2}
\begin{center}
Molecule A \\ 
G\_09\_Cu
\includegraphics[width=8cm]{../Comparisons/ImagesFromVMD/G_09_Cu.png}
\\
\vtab

\columnbreak
Molecule B \\ 
G\_10\_Cu
\includegraphics[width=8cm]{../Comparisons/ImagesFromVMD/G_10_Cu.png}
\\
\vtab


\end{center}
\end{multicols}
\begin{center}
\textcolor{NavyBlue}{\Large Different}
\end{center}

 \newpage

\vtab[-3cm]
\begin{center}
{\large ClustersCu \tab Número 39}
\end{center}
\begin{multicols}{2}
\begin{center}
Molecule A \\ 
G\_09\_Cu
\includegraphics[width=8cm]{../Comparisons/ImagesFromVMD/G_09_Cu.png}
\\
\vtab

\columnbreak
Molecule B \\ 
G\_11\_Cu
\includegraphics[width=8cm]{../Comparisons/ImagesFromVMD/G_11_Cu.png}
\\
\vtab


\end{center}
\end{multicols}
\begin{center}
\textcolor{NavyBlue}{\Large Different}
\end{center}

 \newpage

\vtab[-3cm]
\begin{center}
{\large ClustersCu \tab Número 40}
\end{center}
\begin{multicols}{2}
\begin{center}
Molecule A \\ 
G\_09\_Cu
\includegraphics[width=8cm]{../Comparisons/ImagesFromVMD/G_09_Cu.png}
\\
\vtab

\columnbreak
Molecule B \\ 
SC\_08\_Cu
\includegraphics[width=8cm]{../Comparisons/ImagesFromVMD/SC_08_Cu.png}
\\
\vtab


\end{center}
\end{multicols}
\begin{center}
\textcolor{NavyBlue}{\Large Different}
\end{center}

 \newpage

\vtab[-3cm]
\begin{center}
{\large ClustersCu \tab Número 41}
\end{center}
\begin{multicols}{2}
\begin{center}
Molecule A \\ 
G\_09\_Cu
\includegraphics[width=8cm]{../Comparisons/ImagesFromVMD/G_09_Cu.png}
\\
\vtab

\columnbreak
Molecule B \\ 
SC\_08\_Cu\_AFTER\_DFT
\includegraphics[width=8cm]{../Comparisons/ImagesFromVMD/SC_08_Cu_AFTER_DFT.png}
\\
\vtab


\end{center}
\end{multicols}
\begin{center}
\textcolor{NavyBlue}{\Large Different}
\end{center}

 \newpage

\vtab[-3cm]
\begin{center}
{\large ClustersCu \tab Número 42}
\end{center}
\begin{multicols}{2}
\begin{center}

Molecule A \
G\_09\_Cu

\includegraphics[width=6cm]{../Comparisons/ImagesFromVMD/G_09_Cu.png}

Inertia Tensor - Molecule A \\
\begin{tabular}{|c c c|}
1643.08	 & 	-103.447	 & 	113.093	 \\
-103.447	 & 	2240.72	 & 	205.863	 \\
113.093	 & 	205.863	 & 	1748.04
\end{tabular}

\vtab
 EingenVectors - Molecule A     \\
\begin{tabular}{|c c c|}
-0.747587	 & 	-0.277171	 & 	0.603564	 \\
0.657992	 & 	-0.185481	 & 	0.729825	 \\
-0.0903363	 & 	0.942748	 & 	0.321039
\end{tabular}

\vtab
 EingenValues - Molecule A     \\
\begin{tabular}{|c c c|}
1513.42	 & 	1797.68	 & 	2320.74	 \\
\end{tabular}
\columnbreak

Molecule B \
SC\_09\_Cu

\includegraphics[width=6cm]{../Comparisons/ImagesFromVMD/SC_09_Cu.png}

Inertia Tensor - Molecule B \\
\begin{tabular}{|c c c|}
1697.74	 & 	-70.4883	 & 	5.56477	 \\
-70.4883	 & 	1438.99	 & 	-78.8974	 \\
5.56477	 & 	-78.8974	 & 	2188.92
\end{tabular}

\vtab
 EingenVectors - Molecule B     \\
\begin{tabular}{|c c c|}
0.23798	 & 	0.966449	 & 	0.0966559	 \\
0.970922	 & 	-0.234051	 & 	-0.0503052	 \\
0.025995	 & 	-0.105817	 & 	0.994046
\end{tabular}

\vtab
 EingenValues - Molecule B     \\
\begin{tabular}{|c c c|}
1413.74	 & 	1714.44	 & 	2197.47	 \\
\end{tabular}

\end{center}
\end{multicols}

\vtab[-5mm]
\begin{tabular}{*{2}{m{0.38\textwidth}}}
\begin{center}
\textcolor{NavyBlue}{\Large Different}
\end{center}
&
\begin{center}
\includegraphics[height=6.5cm]{../Comparisons/Vectors/inertia_tensor_of_G_09_Cu_and_SC_09_Cu.png}
\end{center}
\end{tabular}

 \newpage

\vtab[-3cm]
\begin{center}
{\large ClustersCu \tab Número 43}
\end{center}
\begin{multicols}{2}
\begin{center}

Molecule A \
G\_09\_Cu

\includegraphics[width=6cm]{../Comparisons/ImagesFromVMD/G_09_Cu.png}

Inertia Tensor - Molecule A \\
\begin{tabular}{|c c c|}
1643.08	 & 	-103.447	 & 	113.093	 \\
-103.447	 & 	2240.72	 & 	205.863	 \\
113.093	 & 	205.863	 & 	1748.04
\end{tabular}

\vtab
 EingenVectors - Molecule A     \\
\begin{tabular}{|c c c|}
-0.747587	 & 	-0.277171	 & 	0.603564	 \\
0.657992	 & 	-0.185481	 & 	0.729825	 \\
-0.0903363	 & 	0.942748	 & 	0.321039
\end{tabular}

\vtab
 EingenValues - Molecule A     \\
\begin{tabular}{|c c c|}
1513.42	 & 	1797.68	 & 	2320.74	 \\
\end{tabular}
\columnbreak

Molecule B \
SC\_09\_Cu\_AFTER\_DFT

\includegraphics[width=6cm]{../Comparisons/ImagesFromVMD/SC_09_Cu_AFTER_DFT.png}

Inertia Tensor - Molecule B \\
\begin{tabular}{|c c c|}
1766.59	 & 	-64.0178	 & 	3.81884	 \\
-64.0178	 & 	1532.55	 & 	-76.7432	 \\
3.81884	 & 	-76.7432	 & 	2248.22
\end{tabular}

\vtab
 EingenVectors - Molecule B     \\
\begin{tabular}{|c c c|}
0.238514	 & 	0.966075	 & 	0.0990463	 \\
0.970895	 & 	-0.234926	 & 	-0.046613	 \\
0.0217631	 & 	-0.107281	 & 	0.99399
\end{tabular}

\vtab
 EingenValues - Molecule B     \\
\begin{tabular}{|c c c|}
1508.88	 & 	1781.9	 & 	2256.58	 \\
\end{tabular}

\end{center}
\end{multicols}

\vtab[-5mm]
\begin{tabular}{*{2}{m{0.38\textwidth}}}
\begin{center}
\textcolor{NavyBlue}{\Large Different}
\end{center}
&
\begin{center}
\includegraphics[height=6.5cm]{../Comparisons/Vectors/inertia_tensor_of_G_09_Cu_and_SC_09_Cu_AFTER_DFT.png}
\end{center}
\end{tabular}

 \newpage

\vtab[-3cm]
\begin{center}
{\large ClustersCu \tab Número 44}
\end{center}
\begin{multicols}{2}
\begin{center}
Molecule A \\ 
G\_09\_Cu
\includegraphics[width=8cm]{../Comparisons/ImagesFromVMD/G_09_Cu.png}
\\
\vtab

\columnbreak
Molecule B \\ 
lj\_08\_Cu
\includegraphics[width=8cm]{../Comparisons/ImagesFromVMD/lj_08_Cu.png}
\\
\vtab


\end{center}
\end{multicols}
\begin{center}
\textcolor{NavyBlue}{\Large Different}
\end{center}

 \newpage

\vtab[-3cm]
\begin{center}
{\large ClustersCu \tab Número 45}
\end{center}
\begin{multicols}{2}
\begin{center}

Molecule A \
G\_09\_Cu

\includegraphics[width=6cm]{../Comparisons/ImagesFromVMD/G_09_Cu.png}

Inertia Tensor - Molecule A \\
\begin{tabular}{|c c c|}
1643.08	 & 	-103.447	 & 	113.093	 \\
-103.447	 & 	2240.72	 & 	205.863	 \\
113.093	 & 	205.863	 & 	1748.04
\end{tabular}

\vtab
 EingenVectors - Molecule A     \\
\begin{tabular}{|c c c|}
-0.747587	 & 	-0.277171	 & 	0.603564	 \\
0.657992	 & 	-0.185481	 & 	0.729825	 \\
-0.0903363	 & 	0.942748	 & 	0.321039
\end{tabular}

\vtab
 EingenValues - Molecule A     \\
\begin{tabular}{|c c c|}
1513.42	 & 	1797.68	 & 	2320.74	 \\
\end{tabular}
\columnbreak

Molecule B \
lj\_09\_Cu\_AFTER\_DFT

\includegraphics[width=6cm]{../Comparisons/ImagesFromVMD/lj_09_Cu_AFTER_DFT.png}

Inertia Tensor - Molecule B \\
\begin{tabular}{|c c c|}
2617.21	 & 	-111.803	 & 	97.8221	 \\
-111.803	 & 	1523.61	 & 	773.326	 \\
97.8221	 & 	773.326	 & 	1971.07
\end{tabular}

\vtab
 EingenVectors - Molecule B     \\
\begin{tabular}{|c c c|}
-0.0874373	 & 	-0.795928	 & 	0.599044	 \\
0.136388	 & 	-0.60525	 & 	-0.784265	 \\
-0.986789	 & 	-0.0131283	 & 	-0.161476
\end{tabular}

\vtab
 EingenValues - Molecule B     \\
\begin{tabular}{|c c c|}
929.299	 & 	2550.86	 & 	2631.73	 \\
\end{tabular}

\end{center}
\end{multicols}

\vtab[-5mm]
\begin{tabular}{*{2}{m{0.38\textwidth}}}
\begin{center}
\textcolor{NavyBlue}{\Large Different}
\end{center}
&
\begin{center}
\includegraphics[height=6.5cm]{../Comparisons/Vectors/inertia_tensor_of_G_09_Cu_and_lj_09_Cu_AFTER_DFT.png}
\end{center}
\end{tabular}

 \newpage

\vtab[-3cm]
\begin{center}
{\large ClustersCu \tab Número 46}
\end{center}
\begin{multicols}{2}
\begin{center}
Molecule A \\ 
G\_09\_Cu\_AFTER\_DFT
\includegraphics[width=8cm]{../Comparisons/ImagesFromVMD/G_09_Cu_AFTER_DFT.png}
\\
\vtab

\columnbreak
Molecule B \\ 
G\_10\_Cu
\includegraphics[width=8cm]{../Comparisons/ImagesFromVMD/G_10_Cu.png}
\\
\vtab


\end{center}
\end{multicols}
\begin{center}
\textcolor{NavyBlue}{\Large Different}
\end{center}

 \newpage

\vtab[-3cm]
\begin{center}
{\large ClustersCu \tab Número 47}
\end{center}
\begin{multicols}{2}
\begin{center}
Molecule A \\ 
G\_09\_Cu\_AFTER\_DFT
\includegraphics[width=8cm]{../Comparisons/ImagesFromVMD/G_09_Cu_AFTER_DFT.png}
\\
\vtab

\columnbreak
Molecule B \\ 
G\_11\_Cu
\includegraphics[width=8cm]{../Comparisons/ImagesFromVMD/G_11_Cu.png}
\\
\vtab


\end{center}
\end{multicols}
\begin{center}
\textcolor{NavyBlue}{\Large Different}
\end{center}

 \newpage

\vtab[-3cm]
\begin{center}
{\large ClustersCu \tab Número 48}
\end{center}
\begin{multicols}{2}
\begin{center}
Molecule A \\ 
G\_09\_Cu\_AFTER\_DFT
\includegraphics[width=8cm]{../Comparisons/ImagesFromVMD/G_09_Cu_AFTER_DFT.png}
\\
\vtab

\columnbreak
Molecule B \\ 
SC\_08\_Cu
\includegraphics[width=8cm]{../Comparisons/ImagesFromVMD/SC_08_Cu.png}
\\
\vtab


\end{center}
\end{multicols}
\begin{center}
\textcolor{NavyBlue}{\Large Different}
\end{center}

 \newpage

\vtab[-3cm]
\begin{center}
{\large ClustersCu \tab Número 49}
\end{center}
\begin{multicols}{2}
\begin{center}
Molecule A \\ 
G\_09\_Cu\_AFTER\_DFT
\includegraphics[width=8cm]{../Comparisons/ImagesFromVMD/G_09_Cu_AFTER_DFT.png}
\\
\vtab

\columnbreak
Molecule B \\ 
SC\_08\_Cu\_AFTER\_DFT
\includegraphics[width=8cm]{../Comparisons/ImagesFromVMD/SC_08_Cu_AFTER_DFT.png}
\\
\vtab


\end{center}
\end{multicols}
\begin{center}
\textcolor{NavyBlue}{\Large Different}
\end{center}

 \newpage

\vtab[-3cm]
\begin{center}
{\large ClustersCu \tab Número 50}
\end{center}
\begin{multicols}{2}
\begin{center}

Molecule A \
G\_09\_Cu\_AFTER\_DFT

\includegraphics[width=6cm]{../Comparisons/ImagesFromVMD/G_09_Cu_AFTER_DFT.png}

Inertia Tensor - Molecule A \\
\begin{tabular}{|c c c|}
1625.38	 & 	-92.1303	 & 	110.251	 \\
-92.1303	 & 	2185.43	 & 	185.921	 \\
110.251	 & 	185.921	 & 	1731.97
\end{tabular}

\vtab
 EingenVectors - Molecule A     \\
\begin{tabular}{|c c c|}
-0.760255	 & 	-0.26577	 & 	0.592773	 \\
0.644373	 & 	-0.192718	 & 	0.740029	 \\
-0.0824395	 & 	0.944577	 & 	0.317769
\end{tabular}

\vtab
 EingenValues - Molecule A     \\
\begin{tabular}{|c c c|}
1507.21	 & 	1779.55	 & 	2256.02	 \\
\end{tabular}
\columnbreak

Molecule B \
SC\_09\_Cu

\includegraphics[width=6cm]{../Comparisons/ImagesFromVMD/SC_09_Cu.png}

Inertia Tensor - Molecule B \\
\begin{tabular}{|c c c|}
1697.74	 & 	-70.4883	 & 	5.56477	 \\
-70.4883	 & 	1438.99	 & 	-78.8974	 \\
5.56477	 & 	-78.8974	 & 	2188.92
\end{tabular}

\vtab
 EingenVectors - Molecule B     \\
\begin{tabular}{|c c c|}
0.23798	 & 	0.966449	 & 	0.0966559	 \\
0.970922	 & 	-0.234051	 & 	-0.0503052	 \\
0.025995	 & 	-0.105817	 & 	0.994046
\end{tabular}

\vtab
 EingenValues - Molecule B     \\
\begin{tabular}{|c c c|}
1413.74	 & 	1714.44	 & 	2197.47	 \\
\end{tabular}

\end{center}
\end{multicols}

\vtab[-5mm]
\begin{tabular}{*{2}{m{0.38\textwidth}}}
\begin{center}
\textcolor{NavyBlue}{\Large Different}
\end{center}
&
\begin{center}
\includegraphics[height=6.5cm]{../Comparisons/Vectors/inertia_tensor_of_G_09_Cu_AFTER_DFT_and_SC_09_Cu.png}
\end{center}
\end{tabular}

 \newpage

\vtab[-3cm]
\begin{center}
{\large ClustersCu \tab Número 51}
\end{center}
\begin{multicols}{2}
\begin{center}

Molecule A \
G\_09\_Cu\_AFTER\_DFT

\includegraphics[width=6cm]{../Comparisons/ImagesFromVMD/G_09_Cu_AFTER_DFT.png}

Inertia Tensor - Molecule A \\
\begin{tabular}{|c c c|}
1625.38	 & 	-92.1303	 & 	110.251	 \\
-92.1303	 & 	2185.43	 & 	185.921	 \\
110.251	 & 	185.921	 & 	1731.97
\end{tabular}

\vtab
 EingenVectors - Molecule A     \\
\begin{tabular}{|c c c|}
-0.760255	 & 	-0.26577	 & 	0.592773	 \\
0.644373	 & 	-0.192718	 & 	0.740029	 \\
-0.0824395	 & 	0.944577	 & 	0.317769
\end{tabular}

\vtab
 EingenValues - Molecule A     \\
\begin{tabular}{|c c c|}
1507.21	 & 	1779.55	 & 	2256.02	 \\
\end{tabular}
\columnbreak

Molecule B \
SC\_09\_Cu\_AFTER\_DFT

\includegraphics[width=6cm]{../Comparisons/ImagesFromVMD/SC_09_Cu_AFTER_DFT.png}

Inertia Tensor - Molecule B \\
\begin{tabular}{|c c c|}
1766.59	 & 	-64.0178	 & 	3.81884	 \\
-64.0178	 & 	1532.55	 & 	-76.7432	 \\
3.81884	 & 	-76.7432	 & 	2248.22
\end{tabular}

\vtab
 EingenVectors - Molecule B     \\
\begin{tabular}{|c c c|}
0.238514	 & 	0.966075	 & 	0.0990463	 \\
0.970895	 & 	-0.234926	 & 	-0.046613	 \\
0.0217631	 & 	-0.107281	 & 	0.99399
\end{tabular}

\vtab
 EingenValues - Molecule B     \\
\begin{tabular}{|c c c|}
1508.88	 & 	1781.9	 & 	2256.58	 \\
\end{tabular}

\end{center}
\end{multicols}

\vtab[-5mm]
\begin{tabular}{*{2}{m{0.38\textwidth}}}
\begin{center}
\textcolor{NavyBlue}{\Large Different}
\end{center}
&
\begin{center}
\includegraphics[height=6.5cm]{../Comparisons/Vectors/inertia_tensor_of_G_09_Cu_AFTER_DFT_and_SC_09_Cu_AFTER_DFT.png}
\end{center}
\end{tabular}

 \newpage

\vtab[-3cm]
\begin{center}
{\large ClustersCu \tab Número 52}
\end{center}
\begin{multicols}{2}
\begin{center}
Molecule A \\ 
G\_09\_Cu\_AFTER\_DFT
\includegraphics[width=8cm]{../Comparisons/ImagesFromVMD/G_09_Cu_AFTER_DFT.png}
\\
\vtab

\columnbreak
Molecule B \\ 
lj\_08\_Cu
\includegraphics[width=8cm]{../Comparisons/ImagesFromVMD/lj_08_Cu.png}
\\
\vtab


\end{center}
\end{multicols}
\begin{center}
\textcolor{NavyBlue}{\Large Different}
\end{center}

 \newpage

\vtab[-3cm]
\begin{center}
{\large ClustersCu \tab Número 53}
\end{center}
\begin{multicols}{2}
\begin{center}

Molecule A \
G\_09\_Cu\_AFTER\_DFT

\includegraphics[width=6cm]{../Comparisons/ImagesFromVMD/G_09_Cu_AFTER_DFT.png}

Inertia Tensor - Molecule A \\
\begin{tabular}{|c c c|}
1625.38	 & 	-92.1303	 & 	110.251	 \\
-92.1303	 & 	2185.43	 & 	185.921	 \\
110.251	 & 	185.921	 & 	1731.97
\end{tabular}

\vtab
 EingenVectors - Molecule A     \\
\begin{tabular}{|c c c|}
-0.760255	 & 	-0.26577	 & 	0.592773	 \\
0.644373	 & 	-0.192718	 & 	0.740029	 \\
-0.0824395	 & 	0.944577	 & 	0.317769
\end{tabular}

\vtab
 EingenValues - Molecule A     \\
\begin{tabular}{|c c c|}
1507.21	 & 	1779.55	 & 	2256.02	 \\
\end{tabular}
\columnbreak

Molecule B \
lj\_09\_Cu\_AFTER\_DFT

\includegraphics[width=6cm]{../Comparisons/ImagesFromVMD/lj_09_Cu_AFTER_DFT.png}

Inertia Tensor - Molecule B \\
\begin{tabular}{|c c c|}
2617.21	 & 	-111.803	 & 	97.8221	 \\
-111.803	 & 	1523.61	 & 	773.326	 \\
97.8221	 & 	773.326	 & 	1971.07
\end{tabular}

\vtab
 EingenVectors - Molecule B     \\
\begin{tabular}{|c c c|}
-0.0874373	 & 	-0.795928	 & 	0.599044	 \\
0.136388	 & 	-0.60525	 & 	-0.784265	 \\
-0.986789	 & 	-0.0131283	 & 	-0.161476
\end{tabular}

\vtab
 EingenValues - Molecule B     \\
\begin{tabular}{|c c c|}
929.299	 & 	2550.86	 & 	2631.73	 \\
\end{tabular}

\end{center}
\end{multicols}

\vtab[-5mm]
\begin{tabular}{*{2}{m{0.38\textwidth}}}
\begin{center}
\textcolor{NavyBlue}{\Large Different}
\end{center}
&
\begin{center}
\includegraphics[height=6.5cm]{../Comparisons/Vectors/inertia_tensor_of_G_09_Cu_AFTER_DFT_and_lj_09_Cu_AFTER_DFT.png}
\end{center}
\end{tabular}

 \newpage

\vtab[-3cm]
\begin{center}
{\large ClustersCu \tab Número 54}
\end{center}
\begin{multicols}{2}
\begin{center}
Molecule A \\ 
G\_10\_Cu
\includegraphics[width=8cm]{../Comparisons/ImagesFromVMD/G_10_Cu.png}
\\
\vtab

\columnbreak
Molecule B \\ 
G\_11\_Cu
\includegraphics[width=8cm]{../Comparisons/ImagesFromVMD/G_11_Cu.png}
\\
\vtab


\end{center}
\end{multicols}
\begin{center}
\textcolor{NavyBlue}{\Large Different}
\end{center}

 \newpage

\vtab[-3cm]
\begin{center}
{\large ClustersCu \tab Número 55}
\end{center}
\begin{multicols}{2}
\begin{center}
Molecule A \\ 
G\_10\_Cu
\includegraphics[width=8cm]{../Comparisons/ImagesFromVMD/G_10_Cu.png}
\\
\vtab

\columnbreak
Molecule B \\ 
SC\_08\_Cu
\includegraphics[width=8cm]{../Comparisons/ImagesFromVMD/SC_08_Cu.png}
\\
\vtab


\end{center}
\end{multicols}
\begin{center}
\textcolor{NavyBlue}{\Large Different}
\end{center}

 \newpage

\vtab[-3cm]
\begin{center}
{\large ClustersCu \tab Número 56}
\end{center}
\begin{multicols}{2}
\begin{center}
Molecule A \\ 
G\_10\_Cu
\includegraphics[width=8cm]{../Comparisons/ImagesFromVMD/G_10_Cu.png}
\\
\vtab

\columnbreak
Molecule B \\ 
SC\_08\_Cu\_AFTER\_DFT
\includegraphics[width=8cm]{../Comparisons/ImagesFromVMD/SC_08_Cu_AFTER_DFT.png}
\\
\vtab


\end{center}
\end{multicols}
\begin{center}
\textcolor{NavyBlue}{\Large Different}
\end{center}

 \newpage

\vtab[-3cm]
\begin{center}
{\large ClustersCu \tab Número 57}
\end{center}
\begin{multicols}{2}
\begin{center}
Molecule A \\ 
G\_10\_Cu
\includegraphics[width=8cm]{../Comparisons/ImagesFromVMD/G_10_Cu.png}
\\
\vtab

\columnbreak
Molecule B \\ 
SC\_09\_Cu
\includegraphics[width=8cm]{../Comparisons/ImagesFromVMD/SC_09_Cu.png}
\\
\vtab


\end{center}
\end{multicols}
\begin{center}
\textcolor{NavyBlue}{\Large Different}
\end{center}

 \newpage

\vtab[-3cm]
\begin{center}
{\large ClustersCu \tab Número 58}
\end{center}
\begin{multicols}{2}
\begin{center}
Molecule A \\ 
G\_10\_Cu
\includegraphics[width=8cm]{../Comparisons/ImagesFromVMD/G_10_Cu.png}
\\
\vtab

\columnbreak
Molecule B \\ 
SC\_09\_Cu\_AFTER\_DFT
\includegraphics[width=8cm]{../Comparisons/ImagesFromVMD/SC_09_Cu_AFTER_DFT.png}
\\
\vtab


\end{center}
\end{multicols}
\begin{center}
\textcolor{NavyBlue}{\Large Different}
\end{center}

 \newpage

\vtab[-3cm]
\begin{center}
{\large ClustersCu \tab Número 59}
\end{center}
\begin{multicols}{2}
\begin{center}
Molecule A \\ 
G\_10\_Cu
\includegraphics[width=8cm]{../Comparisons/ImagesFromVMD/G_10_Cu.png}
\\
\vtab

\columnbreak
Molecule B \\ 
lj\_08\_Cu
\includegraphics[width=8cm]{../Comparisons/ImagesFromVMD/lj_08_Cu.png}
\\
\vtab


\end{center}
\end{multicols}
\begin{center}
\textcolor{NavyBlue}{\Large Different}
\end{center}

 \newpage

\vtab[-3cm]
\begin{center}
{\large ClustersCu \tab Número 60}
\end{center}
\begin{multicols}{2}
\begin{center}
Molecule A \\ 
G\_10\_Cu
\includegraphics[width=8cm]{../Comparisons/ImagesFromVMD/G_10_Cu.png}
\\
\vtab

\columnbreak
Molecule B \\ 
lj\_09\_Cu\_AFTER\_DFT
\includegraphics[width=8cm]{../Comparisons/ImagesFromVMD/lj_09_Cu_AFTER_DFT.png}
\\
\vtab


\end{center}
\end{multicols}
\begin{center}
\textcolor{NavyBlue}{\Large Different}
\end{center}

 \newpage

\vtab[-3cm]
\begin{center}
{\large ClustersCu \tab Número 61}
\end{center}
\begin{multicols}{2}
\begin{center}
Molecule A \\ 
G\_11\_Cu
\includegraphics[width=8cm]{../Comparisons/ImagesFromVMD/G_11_Cu.png}
\\
\vtab

\columnbreak
Molecule B \\ 
SC\_08\_Cu
\includegraphics[width=8cm]{../Comparisons/ImagesFromVMD/SC_08_Cu.png}
\\
\vtab


\end{center}
\end{multicols}
\begin{center}
\textcolor{NavyBlue}{\Large Different}
\end{center}

 \newpage

\vtab[-3cm]
\begin{center}
{\large ClustersCu \tab Número 62}
\end{center}
\begin{multicols}{2}
\begin{center}
Molecule A \\ 
G\_11\_Cu
\includegraphics[width=8cm]{../Comparisons/ImagesFromVMD/G_11_Cu.png}
\\
\vtab

\columnbreak
Molecule B \\ 
SC\_08\_Cu\_AFTER\_DFT
\includegraphics[width=8cm]{../Comparisons/ImagesFromVMD/SC_08_Cu_AFTER_DFT.png}
\\
\vtab


\end{center}
\end{multicols}
\begin{center}
\textcolor{NavyBlue}{\Large Different}
\end{center}

 \newpage

\vtab[-3cm]
\begin{center}
{\large ClustersCu \tab Número 63}
\end{center}
\begin{multicols}{2}
\begin{center}
Molecule A \\ 
G\_11\_Cu
\includegraphics[width=8cm]{../Comparisons/ImagesFromVMD/G_11_Cu.png}
\\
\vtab

\columnbreak
Molecule B \\ 
SC\_09\_Cu
\includegraphics[width=8cm]{../Comparisons/ImagesFromVMD/SC_09_Cu.png}
\\
\vtab


\end{center}
\end{multicols}
\begin{center}
\textcolor{NavyBlue}{\Large Different}
\end{center}

 \newpage

\vtab[-3cm]
\begin{center}
{\large ClustersCu \tab Número 64}
\end{center}
\begin{multicols}{2}
\begin{center}
Molecule A \\ 
G\_11\_Cu
\includegraphics[width=8cm]{../Comparisons/ImagesFromVMD/G_11_Cu.png}
\\
\vtab

\columnbreak
Molecule B \\ 
SC\_09\_Cu\_AFTER\_DFT
\includegraphics[width=8cm]{../Comparisons/ImagesFromVMD/SC_09_Cu_AFTER_DFT.png}
\\
\vtab


\end{center}
\end{multicols}
\begin{center}
\textcolor{NavyBlue}{\Large Different}
\end{center}

 \newpage

\vtab[-3cm]
\begin{center}
{\large ClustersCu \tab Número 65}
\end{center}
\begin{multicols}{2}
\begin{center}
Molecule A \\ 
G\_11\_Cu
\includegraphics[width=8cm]{../Comparisons/ImagesFromVMD/G_11_Cu.png}
\\
\vtab

\columnbreak
Molecule B \\ 
lj\_08\_Cu
\includegraphics[width=8cm]{../Comparisons/ImagesFromVMD/lj_08_Cu.png}
\\
\vtab


\end{center}
\end{multicols}
\begin{center}
\textcolor{NavyBlue}{\Large Different}
\end{center}

 \newpage

\vtab[-3cm]
\begin{center}
{\large ClustersCu \tab Número 66}
\end{center}
\begin{multicols}{2}
\begin{center}
Molecule A \\ 
G\_11\_Cu
\includegraphics[width=8cm]{../Comparisons/ImagesFromVMD/G_11_Cu.png}
\\
\vtab

\columnbreak
Molecule B \\ 
lj\_09\_Cu\_AFTER\_DFT
\includegraphics[width=8cm]{../Comparisons/ImagesFromVMD/lj_09_Cu_AFTER_DFT.png}
\\
\vtab


\end{center}
\end{multicols}
\begin{center}
\textcolor{NavyBlue}{\Large Different}
\end{center}

 \newpage

\vtab[-3cm]
\begin{center}
{\large ClustersCu \tab Número 67}
\end{center}
\begin{multicols}{2}
\begin{center}

Molecule A \
SC\_08\_Cu

\includegraphics[width=6cm]{../Comparisons/ImagesFromVMD/SC_08_Cu.png}

Inertia Tensor - Molecule A \\
\begin{tabular}{|c c c|}
1331.27	 & 	0.00958178	 & 	0.0062677	 \\
0.00958178	 & 	1331.26	 & 	0.00713637	 \\
0.0062677	 & 	0.00713637	 & 	1331.28
\end{tabular}

\vtab
 EingenVectors - Molecule A     \\
\begin{tabular}{|c c c|}
-0.515886	 & 	0.847718	 & 	-0.123437	 \\
-0.684575	 & 	-0.321331	 & 	0.654296	 \\
0.514994	 & 	0.422044	 & 	0.746096
\end{tabular}

\vtab
 EingenValues - Molecule A     \\
\begin{tabular}{|c c c|}
1331.25	 & 	1331.27	 & 	1331.28	 \\
\end{tabular}
\columnbreak

Molecule B \
SC\_08\_Cu\_AFTER\_DFT

\includegraphics[width=6cm]{../Comparisons/ImagesFromVMD/SC_08_Cu_AFTER_DFT.png}

Inertia Tensor - Molecule B \\
\begin{tabular}{|c c c|}
1394.63	 & 	1.12138	 & 	-5.89124	 \\
1.12138	 & 	1385.89	 & 	0.251188	 \\
-5.89124	 & 	0.251188	 & 	1398.86
\end{tabular}

\vtab
 EingenVectors - Molecule B     \\
\begin{tabular}{|c c c|}
0.189307	 & 	-0.97651	 & 	0.102909	 \\
0.794609	 & 	0.213922	 & 	0.568184	 \\
-0.576853	 & 	-0.0257885	 & 	0.816441
\end{tabular}

\vtab
 EingenValues - Molecule B     \\
\begin{tabular}{|c c c|}
1385.64	 & 	1390.72	 & 	1403.02	 \\
\end{tabular}

\end{center}
\end{multicols}

\vtab[-5mm]
\begin{tabular}{*{2}{m{0.38\textwidth}}}
\begin{center}
\textcolor{NavyBlue}{\Large Different}
\end{center}
&
\begin{center}
\includegraphics[height=6.5cm]{../Comparisons/Vectors/inertia_tensor_of_SC_08_Cu_and_SC_08_Cu_AFTER_DFT.png}
\end{center}
\end{tabular}

 \newpage

\vtab[-3cm]
\begin{center}
{\large ClustersCu \tab Número 68}
\end{center}
\begin{multicols}{2}
\begin{center}
Molecule A \\ 
SC\_08\_Cu
\includegraphics[width=8cm]{../Comparisons/ImagesFromVMD/SC_08_Cu.png}
\\
\vtab

\columnbreak
Molecule B \\ 
SC\_09\_Cu
\includegraphics[width=8cm]{../Comparisons/ImagesFromVMD/SC_09_Cu.png}
\\
\vtab


\end{center}
\end{multicols}
\begin{center}
\textcolor{NavyBlue}{\Large Different}
\end{center}

 \newpage

\vtab[-3cm]
\begin{center}
{\large ClustersCu \tab Número 69}
\end{center}
\begin{multicols}{2}
\begin{center}
Molecule A \\ 
SC\_08\_Cu
\includegraphics[width=8cm]{../Comparisons/ImagesFromVMD/SC_08_Cu.png}
\\
\vtab

\columnbreak
Molecule B \\ 
SC\_09\_Cu\_AFTER\_DFT
\includegraphics[width=8cm]{../Comparisons/ImagesFromVMD/SC_09_Cu_AFTER_DFT.png}
\\
\vtab


\end{center}
\end{multicols}
\begin{center}
\textcolor{NavyBlue}{\Large Different}
\end{center}

 \newpage

\vtab[-3cm]
\begin{center}
{\large ClustersCu \tab Número 70}
\end{center}
\begin{multicols}{2}
\begin{center}

Molecule A \
SC\_08\_Cu

\includegraphics[width=6cm]{../Comparisons/ImagesFromVMD/SC_08_Cu.png}

Inertia Tensor - Molecule A \\
\begin{tabular}{|c c c|}
1331.27	 & 	0.00958178	 & 	0.0062677	 \\
0.00958178	 & 	1331.26	 & 	0.00713637	 \\
0.0062677	 & 	0.00713637	 & 	1331.28
\end{tabular}

\vtab
 EingenVectors - Molecule A     \\
\begin{tabular}{|c c c|}
-0.515886	 & 	0.847718	 & 	-0.123437	 \\
-0.684575	 & 	-0.321331	 & 	0.654296	 \\
0.514994	 & 	0.422044	 & 	0.746096
\end{tabular}

\vtab
 EingenValues - Molecule A     \\
\begin{tabular}{|c c c|}
1331.25	 & 	1331.27	 & 	1331.28	 \\
\end{tabular}
\columnbreak

Molecule B \
lj\_08\_Cu

\includegraphics[width=6cm]{../Comparisons/ImagesFromVMD/lj_08_Cu.png}

Inertia Tensor - Molecule B \\
\begin{tabular}{|c c c|}
1617.21	 & 	-91.1093	 & 	-437.808	 \\
-91.1093	 & 	1798.78	 & 	-179.618	 \\
-437.808	 & 	-179.618	 & 	957.828
\end{tabular}

\vtab
 EingenVectors - Molecule B     \\
\begin{tabular}{|c c c|}
0.43919	 & 	0.180681	 & 	0.880038	 \\
-0.601213	 & 	-0.668782	 & 	0.437348	 \\
-0.667574	 & 	0.721169	 & 	0.185095
\end{tabular}

\vtab
 EingenValues - Molecule B     \\
\begin{tabular}{|c c c|}
702.459	 & 	1834.34	 & 	1837.02	 \\
\end{tabular}

\end{center}
\end{multicols}

\vtab[-5mm]
\begin{tabular}{*{2}{m{0.38\textwidth}}}
\begin{center}
\textcolor{NavyBlue}{\Large Different}
\end{center}
&
\begin{center}
\includegraphics[height=6.5cm]{../Comparisons/Vectors/inertia_tensor_of_SC_08_Cu_and_lj_08_Cu.png}
\end{center}
\end{tabular}

 \newpage

\vtab[-3cm]
\begin{center}
{\large ClustersCu \tab Número 71}
\end{center}
\begin{multicols}{2}
\begin{center}
Molecule A \\ 
SC\_08\_Cu
\includegraphics[width=8cm]{../Comparisons/ImagesFromVMD/SC_08_Cu.png}
\\
\vtab

\columnbreak
Molecule B \\ 
lj\_09\_Cu\_AFTER\_DFT
\includegraphics[width=8cm]{../Comparisons/ImagesFromVMD/lj_09_Cu_AFTER_DFT.png}
\\
\vtab


\end{center}
\end{multicols}
\begin{center}
\textcolor{NavyBlue}{\Large Different}
\end{center}

 \newpage

\vtab[-3cm]
\begin{center}
{\large ClustersCu \tab Número 72}
\end{center}
\begin{multicols}{2}
\begin{center}
Molecule A \\ 
SC\_08\_Cu\_AFTER\_DFT
\includegraphics[width=8cm]{../Comparisons/ImagesFromVMD/SC_08_Cu_AFTER_DFT.png}
\\
\vtab

\columnbreak
Molecule B \\ 
SC\_09\_Cu
\includegraphics[width=8cm]{../Comparisons/ImagesFromVMD/SC_09_Cu.png}
\\
\vtab


\end{center}
\end{multicols}
\begin{center}
\textcolor{NavyBlue}{\Large Different}
\end{center}

 \newpage

\vtab[-3cm]
\begin{center}
{\large ClustersCu \tab Número 73}
\end{center}
\begin{multicols}{2}
\begin{center}
Molecule A \\ 
SC\_08\_Cu\_AFTER\_DFT
\includegraphics[width=8cm]{../Comparisons/ImagesFromVMD/SC_08_Cu_AFTER_DFT.png}
\\
\vtab

\columnbreak
Molecule B \\ 
SC\_09\_Cu\_AFTER\_DFT
\includegraphics[width=8cm]{../Comparisons/ImagesFromVMD/SC_09_Cu_AFTER_DFT.png}
\\
\vtab


\end{center}
\end{multicols}
\begin{center}
\textcolor{NavyBlue}{\Large Different}
\end{center}

 \newpage

\vtab[-3cm]
\begin{center}
{\large ClustersCu \tab Número 74}
\end{center}
\begin{multicols}{2}
\begin{center}

Molecule A \
SC\_08\_Cu\_AFTER\_DFT

\includegraphics[width=6cm]{../Comparisons/ImagesFromVMD/SC_08_Cu_AFTER_DFT.png}

Inertia Tensor - Molecule A \\
\begin{tabular}{|c c c|}
1394.63	 & 	1.12138	 & 	-5.89124	 \\
1.12138	 & 	1385.89	 & 	0.251188	 \\
-5.89124	 & 	0.251188	 & 	1398.86
\end{tabular}

\vtab
 EingenVectors - Molecule A     \\
\begin{tabular}{|c c c|}
0.189307	 & 	-0.97651	 & 	0.102909	 \\
0.794609	 & 	0.213922	 & 	0.568184	 \\
-0.576853	 & 	-0.0257885	 & 	0.816441
\end{tabular}

\vtab
 EingenValues - Molecule A     \\
\begin{tabular}{|c c c|}
1385.64	 & 	1390.72	 & 	1403.02	 \\
\end{tabular}
\columnbreak

Molecule B \
lj\_08\_Cu

\includegraphics[width=6cm]{../Comparisons/ImagesFromVMD/lj_08_Cu.png}

Inertia Tensor - Molecule B \\
\begin{tabular}{|c c c|}
1617.21	 & 	-91.1093	 & 	-437.808	 \\
-91.1093	 & 	1798.78	 & 	-179.618	 \\
-437.808	 & 	-179.618	 & 	957.828
\end{tabular}

\vtab
 EingenVectors - Molecule B     \\
\begin{tabular}{|c c c|}
0.43919	 & 	0.180681	 & 	0.880038	 \\
-0.601213	 & 	-0.668782	 & 	0.437348	 \\
-0.667574	 & 	0.721169	 & 	0.185095
\end{tabular}

\vtab
 EingenValues - Molecule B     \\
\begin{tabular}{|c c c|}
702.459	 & 	1834.34	 & 	1837.02	 \\
\end{tabular}

\end{center}
\end{multicols}

\vtab[-5mm]
\begin{tabular}{*{2}{m{0.38\textwidth}}}
\begin{center}
\textcolor{NavyBlue}{\Large Different}
\end{center}
&
\begin{center}
\includegraphics[height=6.5cm]{../Comparisons/Vectors/inertia_tensor_of_SC_08_Cu_AFTER_DFT_and_lj_08_Cu.png}
\end{center}
\end{tabular}

 \newpage

\vtab[-3cm]
\begin{center}
{\large ClustersCu \tab Número 75}
\end{center}
\begin{multicols}{2}
\begin{center}
Molecule A \\ 
SC\_08\_Cu\_AFTER\_DFT
\includegraphics[width=8cm]{../Comparisons/ImagesFromVMD/SC_08_Cu_AFTER_DFT.png}
\\
\vtab

\columnbreak
Molecule B \\ 
lj\_09\_Cu\_AFTER\_DFT
\includegraphics[width=8cm]{../Comparisons/ImagesFromVMD/lj_09_Cu_AFTER_DFT.png}
\\
\vtab


\end{center}
\end{multicols}
\begin{center}
\textcolor{NavyBlue}{\Large Different}
\end{center}

 \newpage

\vtab[-3cm]
\begin{center}
{\large ClustersCu \tab Número 76}
\end{center}
\begin{multicols}{2}
\begin{center}

Molecule A \
SC\_09\_Cu

\includegraphics[width=6cm]{../Comparisons/ImagesFromVMD/SC_09_Cu.png}

Inertia Tensor - Molecule A \\
\begin{tabular}{|c c c|}
1697.74	 & 	-70.4883	 & 	5.56477	 \\
-70.4883	 & 	1438.99	 & 	-78.8974	 \\
5.56477	 & 	-78.8974	 & 	2188.92
\end{tabular}

\vtab
 EingenVectors - Molecule A     \\
\begin{tabular}{|c c c|}
0.23798	 & 	0.966449	 & 	0.0966559	 \\
0.970922	 & 	-0.234051	 & 	-0.0503052	 \\
0.025995	 & 	-0.105817	 & 	0.994046
\end{tabular}

\vtab
 EingenValues - Molecule A     \\
\begin{tabular}{|c c c|}
1413.74	 & 	1714.44	 & 	2197.47	 \\
\end{tabular}
\columnbreak

Molecule B \
SC\_09\_Cu\_AFTER\_DFT

\includegraphics[width=6cm]{../Comparisons/ImagesFromVMD/SC_09_Cu_AFTER_DFT.png}

Inertia Tensor - Molecule B \\
\begin{tabular}{|c c c|}
1766.59	 & 	-64.0178	 & 	3.81884	 \\
-64.0178	 & 	1532.55	 & 	-76.7432	 \\
3.81884	 & 	-76.7432	 & 	2248.22
\end{tabular}

\vtab
 EingenVectors - Molecule B     \\
\begin{tabular}{|c c c|}
0.238514	 & 	0.966075	 & 	0.0990463	 \\
0.970895	 & 	-0.234926	 & 	-0.046613	 \\
0.0217631	 & 	-0.107281	 & 	0.99399
\end{tabular}

\vtab
 EingenValues - Molecule B     \\
\begin{tabular}{|c c c|}
1508.88	 & 	1781.9	 & 	2256.58	 \\
\end{tabular}

\end{center}
\end{multicols}

\vtab[-5mm]
\begin{tabular}{*{2}{m{0.38\textwidth}}}
\begin{center}
\textcolor{NavyBlue}{\Large Different}
\end{center}
&
\begin{center}
\includegraphics[height=6.5cm]{../Comparisons/Vectors/inertia_tensor_of_SC_09_Cu_and_SC_09_Cu_AFTER_DFT.png}
\end{center}
\end{tabular}

 \newpage

\vtab[-3cm]
\begin{center}
{\large ClustersCu \tab Número 77}
\end{center}
\begin{multicols}{2}
\begin{center}
Molecule A \\ 
SC\_09\_Cu
\includegraphics[width=8cm]{../Comparisons/ImagesFromVMD/SC_09_Cu.png}
\\
\vtab

\columnbreak
Molecule B \\ 
lj\_08\_Cu
\includegraphics[width=8cm]{../Comparisons/ImagesFromVMD/lj_08_Cu.png}
\\
\vtab


\end{center}
\end{multicols}
\begin{center}
\textcolor{NavyBlue}{\Large Different}
\end{center}

 \newpage

\vtab[-3cm]
\begin{center}
{\large ClustersCu \tab Número 78}
\end{center}
\begin{multicols}{2}
\begin{center}

Molecule A \
SC\_09\_Cu

\includegraphics[width=6cm]{../Comparisons/ImagesFromVMD/SC_09_Cu.png}

Inertia Tensor - Molecule A \\
\begin{tabular}{|c c c|}
1697.74	 & 	-70.4883	 & 	5.56477	 \\
-70.4883	 & 	1438.99	 & 	-78.8974	 \\
5.56477	 & 	-78.8974	 & 	2188.92
\end{tabular}

\vtab
 EingenVectors - Molecule A     \\
\begin{tabular}{|c c c|}
0.23798	 & 	0.966449	 & 	0.0966559	 \\
0.970922	 & 	-0.234051	 & 	-0.0503052	 \\
0.025995	 & 	-0.105817	 & 	0.994046
\end{tabular}

\vtab
 EingenValues - Molecule A     \\
\begin{tabular}{|c c c|}
1413.74	 & 	1714.44	 & 	2197.47	 \\
\end{tabular}
\columnbreak

Molecule B \
lj\_09\_Cu\_AFTER\_DFT

\includegraphics[width=6cm]{../Comparisons/ImagesFromVMD/lj_09_Cu_AFTER_DFT.png}

Inertia Tensor - Molecule B \\
\begin{tabular}{|c c c|}
2617.21	 & 	-111.803	 & 	97.8221	 \\
-111.803	 & 	1523.61	 & 	773.326	 \\
97.8221	 & 	773.326	 & 	1971.07
\end{tabular}

\vtab
 EingenVectors - Molecule B     \\
\begin{tabular}{|c c c|}
-0.0874373	 & 	-0.795928	 & 	0.599044	 \\
0.136388	 & 	-0.60525	 & 	-0.784265	 \\
-0.986789	 & 	-0.0131283	 & 	-0.161476
\end{tabular}

\vtab
 EingenValues - Molecule B     \\
\begin{tabular}{|c c c|}
929.299	 & 	2550.86	 & 	2631.73	 \\
\end{tabular}

\end{center}
\end{multicols}

\vtab[-5mm]
\begin{tabular}{*{2}{m{0.38\textwidth}}}
\begin{center}
\textcolor{NavyBlue}{\Large Different}
\end{center}
&
\begin{center}
\includegraphics[height=6.5cm]{../Comparisons/Vectors/inertia_tensor_of_SC_09_Cu_and_lj_09_Cu_AFTER_DFT.png}
\end{center}
\end{tabular}

 \newpage

\vtab[-3cm]
\begin{center}
{\large ClustersCu \tab Número 79}
\end{center}
\begin{multicols}{2}
\begin{center}
Molecule A \\ 
SC\_09\_Cu\_AFTER\_DFT
\includegraphics[width=8cm]{../Comparisons/ImagesFromVMD/SC_09_Cu_AFTER_DFT.png}
\\
\vtab

\columnbreak
Molecule B \\ 
lj\_08\_Cu
\includegraphics[width=8cm]{../Comparisons/ImagesFromVMD/lj_08_Cu.png}
\\
\vtab


\end{center}
\end{multicols}
\begin{center}
\textcolor{NavyBlue}{\Large Different}
\end{center}

 \newpage

\vtab[-3cm]
\begin{center}
{\large ClustersCu \tab Número 80}
\end{center}
\begin{multicols}{2}
\begin{center}

Molecule A \
SC\_09\_Cu\_AFTER\_DFT

\includegraphics[width=6cm]{../Comparisons/ImagesFromVMD/SC_09_Cu_AFTER_DFT.png}

Inertia Tensor - Molecule A \\
\begin{tabular}{|c c c|}
1766.59	 & 	-64.0178	 & 	3.81884	 \\
-64.0178	 & 	1532.55	 & 	-76.7432	 \\
3.81884	 & 	-76.7432	 & 	2248.22
\end{tabular}

\vtab
 EingenVectors - Molecule A     \\
\begin{tabular}{|c c c|}
0.238514	 & 	0.966075	 & 	0.0990463	 \\
0.970895	 & 	-0.234926	 & 	-0.046613	 \\
0.0217631	 & 	-0.107281	 & 	0.99399
\end{tabular}

\vtab
 EingenValues - Molecule A     \\
\begin{tabular}{|c c c|}
1508.88	 & 	1781.9	 & 	2256.58	 \\
\end{tabular}
\columnbreak

Molecule B \
lj\_09\_Cu\_AFTER\_DFT

\includegraphics[width=6cm]{../Comparisons/ImagesFromVMD/lj_09_Cu_AFTER_DFT.png}

Inertia Tensor - Molecule B \\
\begin{tabular}{|c c c|}
2617.21	 & 	-111.803	 & 	97.8221	 \\
-111.803	 & 	1523.61	 & 	773.326	 \\
97.8221	 & 	773.326	 & 	1971.07
\end{tabular}

\vtab
 EingenVectors - Molecule B     \\
\begin{tabular}{|c c c|}
-0.0874373	 & 	-0.795928	 & 	0.599044	 \\
0.136388	 & 	-0.60525	 & 	-0.784265	 \\
-0.986789	 & 	-0.0131283	 & 	-0.161476
\end{tabular}

\vtab
 EingenValues - Molecule B     \\
\begin{tabular}{|c c c|}
929.299	 & 	2550.86	 & 	2631.73	 \\
\end{tabular}

\end{center}
\end{multicols}

\vtab[-5mm]
\begin{tabular}{*{2}{m{0.38\textwidth}}}
\begin{center}
\textcolor{NavyBlue}{\Large Different}
\end{center}
&
\begin{center}
\includegraphics[height=6.5cm]{../Comparisons/Vectors/inertia_tensor_of_SC_09_Cu_AFTER_DFT_and_lj_09_Cu_AFTER_DFT.png}
\end{center}
\end{tabular}

 \newpage

\vtab[-3cm]
\begin{center}
{\large ClustersCu \tab Número 81}
\end{center}
\begin{multicols}{2}
\begin{center}
Molecule A \\ 
lj\_08\_Cu
\includegraphics[width=8cm]{../Comparisons/ImagesFromVMD/lj_08_Cu.png}
\\
\vtab

\columnbreak
Molecule B \\ 
lj\_09\_Cu\_AFTER\_DFT
\includegraphics[width=8cm]{../Comparisons/ImagesFromVMD/lj_09_Cu_AFTER_DFT.png}
\\
\vtab


\end{center}
\end{multicols}
\begin{center}
\textcolor{NavyBlue}{\Large Different}
\end{center}

 \newpage

\vtab[-3cm]
\begin{center}
{\large ClustersNi \tab Número 82}
\end{center}
\begin{multicols}{2}
\begin{center}

Molecule A \
G\_08\_Ni

\includegraphics[width=6cm]{../Comparisons/ImagesFromVMD/G_08_Ni.png}

Inertia Tensor - Molecule A \\
\begin{tabular}{|c c c|}
1351.65	 & 	-0.00419552	 & 	0.0117528	 \\
-0.00419552	 & 	1351.64	 & 	-0.0031736	 \\
0.0117528	 & 	-0.0031736	 & 	1351.66
\end{tabular}

\vtab
 EingenVectors - Molecule A     \\
\begin{tabular}{|c c c|}
-0.508502	 & 	-0.847012	 & 	0.154908	 \\
0.668029	 & 	-0.501581	 & 	-0.549685	 \\
0.543288	 & 	-0.176033	 & 	0.820884
\end{tabular}

\vtab
 EingenValues - Molecule A     \\
\begin{tabular}{|c c c|}
1351.64	 & 	1351.64	 & 	1351.67	 \\
\end{tabular}
\columnbreak

Molecule B \
G\_08\_Ni\_AFTER\_DFT

\includegraphics[width=6cm]{../Comparisons/ImagesFromVMD/G_08_Ni_AFTER_DFT.png}

Inertia Tensor - Molecule B \\
\begin{tabular}{|c c c|}
1190.24	 & 	6.60641	 & 	-0.89759	 \\
6.60641	 & 	1168.04	 & 	3.23374	 \\
-0.89759	 & 	3.23374	 & 	1191.18
\end{tabular}

\vtab
 EingenVectors - Molecule B     \\
\begin{tabular}{|c c c|}
-0.262972	 & 	0.955871	 & 	-0.130979	 \\
-0.0175286	 & 	0.131002	 & 	0.991227	 \\
-0.964644	 & 	-0.262961	 & 	0.0176947
\end{tabular}

\vtab
 EingenValues - Molecule B     \\
\begin{tabular}{|c c c|}
1165.78	 & 	1191.63	 & 	1192.06	 \\
\end{tabular}

\end{center}
\end{multicols}

\vtab[-5mm]
\begin{tabular}{*{2}{m{0.38\textwidth}}}
\begin{center}
\textcolor{NavyBlue}{\Large Different}
\end{center}
&
\begin{center}
\includegraphics[height=6.5cm]{../Comparisons/Vectors/inertia_tensor_of_G_08_Ni_and_G_08_Ni_AFTER_DFT.png}
\end{center}
\end{tabular}

 \newpage

\vtab[-3cm]
\begin{center}
{\large ClustersNi \tab Número 83}
\end{center}
\begin{multicols}{2}
\begin{center}
Molecule A \\ 
G\_08\_Ni
\includegraphics[width=8cm]{../Comparisons/ImagesFromVMD/G_08_Ni.png}
\\
\vtab

\columnbreak
Molecule B \\ 
G\_09\_Ni
\includegraphics[width=8cm]{../Comparisons/ImagesFromVMD/G_09_Ni.png}
\\
\vtab


\end{center}
\end{multicols}
\begin{center}
\textcolor{NavyBlue}{\Large Different}
\end{center}

 \newpage

\vtab[-3cm]
\begin{center}
{\large ClustersNi \tab Número 84}
\end{center}
\begin{multicols}{2}
\begin{center}
Molecule A \\ 
G\_08\_Ni
\includegraphics[width=8cm]{../Comparisons/ImagesFromVMD/G_08_Ni.png}
\\
\vtab

\columnbreak
Molecule B \\ 
G\_09\_Ni\_AFTER\_DFT
\includegraphics[width=8cm]{../Comparisons/ImagesFromVMD/G_09_Ni_AFTER_DFT.png}
\\
\vtab


\end{center}
\end{multicols}
\begin{center}
\textcolor{NavyBlue}{\Large Different}
\end{center}

 \newpage

\vtab[-3cm]
\begin{center}
{\large ClustersNi \tab Número 85}
\end{center}
\begin{multicols}{2}
\begin{center}

Molecule A \
G\_08\_Ni

\includegraphics[width=6cm]{../Comparisons/ImagesFromVMD/G_08_Ni.png}

Inertia Tensor - Molecule A \\
\begin{tabular}{|c c c|}
1351.65	 & 	-0.00419552	 & 	0.0117528	 \\
-0.00419552	 & 	1351.64	 & 	-0.0031736	 \\
0.0117528	 & 	-0.0031736	 & 	1351.66
\end{tabular}

\vtab
 EingenVectors - Molecule A     \\
\begin{tabular}{|c c c|}
-0.508502	 & 	-0.847012	 & 	0.154908	 \\
0.668029	 & 	-0.501581	 & 	-0.549685	 \\
0.543288	 & 	-0.176033	 & 	0.820884
\end{tabular}

\vtab
 EingenValues - Molecule A     \\
\begin{tabular}{|c c c|}
1351.64	 & 	1351.64	 & 	1351.67	 \\
\end{tabular}
\columnbreak

Molecule B \
SC\_08\_Ni

\includegraphics[width=6cm]{../Comparisons/ImagesFromVMD/SC_08_Ni.png}

Inertia Tensor - Molecule B \\
\begin{tabular}{|c c c|}
1167.79	 & 	-0.00280809	 & 	0.01009	 \\
-0.00280809	 & 	1167.78	 & 	0.00700655	 \\
0.01009	 & 	0.00700655	 & 	1167.79
\end{tabular}

\vtab
 EingenVectors - Molecule B     \\
\begin{tabular}{|c c c|}
0.395943	 & 	0.735527	 & 	-0.549753	 \\
-0.57735	 & 	0.664947	 & 	0.473828	 \\
0.71407	 & 	0.129791	 & 	0.687938
\end{tabular}

\vtab
 EingenValues - Molecule B     \\
\begin{tabular}{|c c c|}
1167.77	 & 	1167.79	 & 	1167.8	 \\
\end{tabular}

\end{center}
\end{multicols}

\vtab[-5mm]
\begin{tabular}{*{2}{m{0.38\textwidth}}}
\begin{center}
\textcolor{NavyBlue}{\Large Different}
\end{center}
&
\begin{center}
\includegraphics[height=6.5cm]{../Comparisons/Vectors/inertia_tensor_of_G_08_Ni_and_SC_08_Ni.png}
\end{center}
\end{tabular}

 \newpage

\vtab[-3cm]
\begin{center}
{\large ClustersNi \tab Número 86}
\end{center}
\begin{multicols}{2}
\begin{center}

Molecule A \
G\_08\_Ni

\includegraphics[width=6cm]{../Comparisons/ImagesFromVMD/G_08_Ni.png}

Inertia Tensor - Molecule A \\
\begin{tabular}{|c c c|}
1351.65	 & 	-0.00419552	 & 	0.0117528	 \\
-0.00419552	 & 	1351.64	 & 	-0.0031736	 \\
0.0117528	 & 	-0.0031736	 & 	1351.66
\end{tabular}

\vtab
 EingenVectors - Molecule A     \\
\begin{tabular}{|c c c|}
-0.508502	 & 	-0.847012	 & 	0.154908	 \\
0.668029	 & 	-0.501581	 & 	-0.549685	 \\
0.543288	 & 	-0.176033	 & 	0.820884
\end{tabular}

\vtab
 EingenValues - Molecule A     \\
\begin{tabular}{|c c c|}
1351.64	 & 	1351.64	 & 	1351.67	 \\
\end{tabular}
\columnbreak

Molecule B \
SC\_08\_Ni\_AFTER\_DFT

\includegraphics[width=6cm]{../Comparisons/ImagesFromVMD/SC_08_Ni_AFTER_DFT.png}

Inertia Tensor - Molecule B \\
\begin{tabular}{|c c c|}
1202.23	 & 	-22.8163	 & 	-12.5979	 \\
-22.8163	 & 	1197.22	 & 	1.92415	 \\
-12.5979	 & 	1.92415	 & 	1155.53
\end{tabular}

\vtab
 EingenVectors - Molecule B     \\
\begin{tabular}{|c c c|}
0.285716	 & 	0.10352	 & 	0.952707	 \\
0.595734	 & 	0.759527	 & 	-0.261189	 \\
-0.750645	 & 	0.642185	 & 	0.155339
\end{tabular}

\vtab
 EingenValues - Molecule B     \\
\begin{tabular}{|c c c|}
1151.96	 & 	1178.67	 & 	1224.36	 \\
\end{tabular}

\end{center}
\end{multicols}

\vtab[-5mm]
\begin{tabular}{*{2}{m{0.38\textwidth}}}
\begin{center}
\textcolor{NavyBlue}{\Large Different}
\end{center}
&
\begin{center}
\includegraphics[height=6.5cm]{../Comparisons/Vectors/inertia_tensor_of_G_08_Ni_and_SC_08_Ni_AFTER_DFT.png}
\end{center}
\end{tabular}

 \newpage

\vtab[-3cm]
\begin{center}
{\large ClustersNi \tab Número 87}
\end{center}
\begin{multicols}{2}
\begin{center}
Molecule A \\ 
G\_08\_Ni
\includegraphics[width=8cm]{../Comparisons/ImagesFromVMD/G_08_Ni.png}
\\
\vtab

\columnbreak
Molecule B \\ 
SC\_09\_Ni
\includegraphics[width=8cm]{../Comparisons/ImagesFromVMD/SC_09_Ni.png}
\\
\vtab


\end{center}
\end{multicols}
\begin{center}
\textcolor{NavyBlue}{\Large Different}
\end{center}

 \newpage

\vtab[-3cm]
\begin{center}
{\large ClustersNi \tab Número 88}
\end{center}
\begin{multicols}{2}
\begin{center}
Molecule A \\ 
G\_08\_Ni
\includegraphics[width=8cm]{../Comparisons/ImagesFromVMD/G_08_Ni.png}
\\
\vtab

\columnbreak
Molecule B \\ 
SC\_09\_Ni\_AFTER\_DFT
\includegraphics[width=8cm]{../Comparisons/ImagesFromVMD/SC_09_Ni_AFTER_DFT.png}
\\
\vtab


\end{center}
\end{multicols}
\begin{center}
\textcolor{NavyBlue}{\Large Different}
\end{center}

 \newpage

\vtab[-3cm]
\begin{center}
{\large ClustersNi \tab Número 89}
\end{center}
\begin{multicols}{2}
\begin{center}
Molecule A \\ 
G\_08\_Ni
\includegraphics[width=8cm]{../Comparisons/ImagesFromVMD/G_08_Ni.png}
\\
\vtab

\columnbreak
Molecule B \\ 
lj\_08
\includegraphics[width=8cm]{../Comparisons/ImagesFromVMD/lj_08.png}
\\
\vtab


\end{center}
\end{multicols}
\begin{center}
\textcolor{NavyBlue}{\Large Different}
\end{center}

 \newpage

\vtab[-3cm]
\begin{center}
{\large ClustersNi \tab Número 90}
\end{center}
\begin{multicols}{2}
\begin{center}
Molecule A \\ 
G\_08\_Ni
\includegraphics[width=8cm]{../Comparisons/ImagesFromVMD/G_08_Ni.png}
\\
\vtab

\columnbreak
Molecule B \\ 
lj\_08\_Ni
\includegraphics[width=8cm]{../Comparisons/ImagesFromVMD/lj_08_Ni.png}
\\
\vtab


\end{center}
\end{multicols}
\begin{center}
\textcolor{NavyBlue}{\Large Different}
\end{center}

 \newpage

\vtab[-3cm]
\begin{center}
{\large ClustersNi \tab Número 91}
\end{center}
\begin{multicols}{2}
\begin{center}

Molecule A \
G\_08\_Ni

\includegraphics[width=6cm]{../Comparisons/ImagesFromVMD/G_08_Ni.png}

Inertia Tensor - Molecule A \\
\begin{tabular}{|c c c|}
1351.65	 & 	-0.00419552	 & 	0.0117528	 \\
-0.00419552	 & 	1351.64	 & 	-0.0031736	 \\
0.0117528	 & 	-0.0031736	 & 	1351.66
\end{tabular}

\vtab
 EingenVectors - Molecule A     \\
\begin{tabular}{|c c c|}
-0.508502	 & 	-0.847012	 & 	0.154908	 \\
0.668029	 & 	-0.501581	 & 	-0.549685	 \\
0.543288	 & 	-0.176033	 & 	0.820884
\end{tabular}

\vtab
 EingenValues - Molecule A     \\
\begin{tabular}{|c c c|}
1351.64	 & 	1351.64	 & 	1351.67	 \\
\end{tabular}
\columnbreak

Molecule B \
lj\_08\_Ni\_AFTER\_DFT

\includegraphics[width=6cm]{../Comparisons/ImagesFromVMD/lj_08_Ni_AFTER_DFT.png}

Inertia Tensor - Molecule B \\
\begin{tabular}{|c c c|}
1461.29	 & 	-67.9001	 & 	-380.896	 \\
-67.9001	 & 	1626.58	 & 	-132.788	 \\
-380.896	 & 	-132.788	 & 	885.164
\end{tabular}

\vtab
 EingenVectors - Molecule B     \\
\begin{tabular}{|c c c|}
0.440314	 & 	0.154412	 & 	0.884466	 \\
-0.488448	 & 	-0.785372	 & 	0.380276	 \\
-0.753354	 & 	0.599456	 & 	0.270388
\end{tabular}

\vtab
 EingenValues - Molecule B     \\
\begin{tabular}{|c c c|}
672.361	 & 	1648.65	 & 	1652.02	 \\
\end{tabular}

\end{center}
\end{multicols}

\vtab[-5mm]
\begin{tabular}{*{2}{m{0.38\textwidth}}}
\begin{center}
\textcolor{NavyBlue}{\Large Different}
\end{center}
&
\begin{center}
\includegraphics[height=6.5cm]{../Comparisons/Vectors/inertia_tensor_of_G_08_Ni_and_lj_08_Ni_AFTER_DFT.png}
\end{center}
\end{tabular}

 \newpage

\vtab[-3cm]
\begin{center}
{\large ClustersNi \tab Número 92}
\end{center}
\begin{multicols}{2}
\begin{center}
Molecule A \\ 
G\_08\_Ni
\includegraphics[width=8cm]{../Comparisons/ImagesFromVMD/G_08_Ni.png}
\\
\vtab

\columnbreak
Molecule B \\ 
lj\_09
\includegraphics[width=8cm]{../Comparisons/ImagesFromVMD/lj_09.png}
\\
\vtab


\end{center}
\end{multicols}
\begin{center}
\textcolor{NavyBlue}{\Large Different}
\end{center}

 \newpage

\vtab[-3cm]
\begin{center}
{\large ClustersNi \tab Número 93}
\end{center}
\begin{multicols}{2}
\begin{center}
Molecule A \\ 
G\_08\_Ni
\includegraphics[width=8cm]{../Comparisons/ImagesFromVMD/G_08_Ni.png}
\\
\vtab

\columnbreak
Molecule B \\ 
lj\_09\_Ni
\includegraphics[width=8cm]{../Comparisons/ImagesFromVMD/lj_09_Ni.png}
\\
\vtab


\end{center}
\end{multicols}
\begin{center}
\textcolor{NavyBlue}{\Large Different}
\end{center}

 \newpage

\vtab[-3cm]
\begin{center}
{\large ClustersNi \tab Número 94}
\end{center}
\begin{multicols}{2}
\begin{center}
Molecule A \\ 
G\_08\_Ni
\includegraphics[width=8cm]{../Comparisons/ImagesFromVMD/G_08_Ni.png}
\\
\vtab

\columnbreak
Molecule B \\ 
lj\_09\_Ni\_AFTER\_DFT
\includegraphics[width=8cm]{../Comparisons/ImagesFromVMD/lj_09_Ni_AFTER_DFT.png}
\\
\vtab


\end{center}
\end{multicols}
\begin{center}
\textcolor{NavyBlue}{\Large Different}
\end{center}

 \newpage

\vtab[-3cm]
\begin{center}
{\large ClustersNi \tab Número 95}
\end{center}
\begin{multicols}{2}
\begin{center}
Molecule A \\ 
G\_08\_Ni
\includegraphics[width=8cm]{../Comparisons/ImagesFromVMD/G_08_Ni.png}
\\
\vtab

\columnbreak
Molecule B \\ 
lj\_10
\includegraphics[width=8cm]{../Comparisons/ImagesFromVMD/lj_10.png}
\\
\vtab


\end{center}
\end{multicols}
\begin{center}
\textcolor{NavyBlue}{\Large Different}
\end{center}

 \newpage

\vtab[-3cm]
\begin{center}
{\large ClustersNi \tab Número 96}
\end{center}
\begin{multicols}{2}
\begin{center}
Molecule A \\ 
G\_08\_Ni
\includegraphics[width=8cm]{../Comparisons/ImagesFromVMD/G_08_Ni.png}
\\
\vtab

\columnbreak
Molecule B \\ 
lj\_11
\includegraphics[width=8cm]{../Comparisons/ImagesFromVMD/lj_11.png}
\\
\vtab


\end{center}
\end{multicols}
\begin{center}
\textcolor{NavyBlue}{\Large Different}
\end{center}

 \newpage

\vtab[-3cm]
\begin{center}
{\large ClustersNi \tab Número 97}
\end{center}
\begin{multicols}{2}
\begin{center}
Molecule A \\ 
G\_08\_Ni\_AFTER\_DFT
\includegraphics[width=8cm]{../Comparisons/ImagesFromVMD/G_08_Ni_AFTER_DFT.png}
\\
\vtab

\columnbreak
Molecule B \\ 
G\_09\_Ni
\includegraphics[width=8cm]{../Comparisons/ImagesFromVMD/G_09_Ni.png}
\\
\vtab


\end{center}
\end{multicols}
\begin{center}
\textcolor{NavyBlue}{\Large Different}
\end{center}

 \newpage

\vtab[-3cm]
\begin{center}
{\large ClustersNi \tab Número 98}
\end{center}
\begin{multicols}{2}
\begin{center}
Molecule A \\ 
G\_08\_Ni\_AFTER\_DFT
\includegraphics[width=8cm]{../Comparisons/ImagesFromVMD/G_08_Ni_AFTER_DFT.png}
\\
\vtab

\columnbreak
Molecule B \\ 
G\_09\_Ni\_AFTER\_DFT
\includegraphics[width=8cm]{../Comparisons/ImagesFromVMD/G_09_Ni_AFTER_DFT.png}
\\
\vtab


\end{center}
\end{multicols}
\begin{center}
\textcolor{NavyBlue}{\Large Different}
\end{center}

 \newpage

\vtab[-3cm]
\begin{center}
{\large ClustersNi \tab Número 99}
\end{center}
\begin{multicols}{2}
\begin{center}

Molecule A \
G\_08\_Ni\_AFTER\_DFT

\includegraphics[width=6cm]{../Comparisons/ImagesFromVMD/G_08_Ni_AFTER_DFT.png}

Inertia Tensor - Molecule A \\
\begin{tabular}{|c c c|}
1190.24	 & 	6.60641	 & 	-0.89759	 \\
6.60641	 & 	1168.04	 & 	3.23374	 \\
-0.89759	 & 	3.23374	 & 	1191.18
\end{tabular}

\vtab
 EingenVectors - Molecule A     \\
\begin{tabular}{|c c c|}
-0.262972	 & 	0.955871	 & 	-0.130979	 \\
-0.0175286	 & 	0.131002	 & 	0.991227	 \\
-0.964644	 & 	-0.262961	 & 	0.0176947
\end{tabular}

\vtab
 EingenValues - Molecule A     \\
\begin{tabular}{|c c c|}
1165.78	 & 	1191.63	 & 	1192.06	 \\
\end{tabular}
\columnbreak

Molecule B \
SC\_08\_Ni

\includegraphics[width=6cm]{../Comparisons/ImagesFromVMD/SC_08_Ni.png}

Inertia Tensor - Molecule B \\
\begin{tabular}{|c c c|}
1167.79	 & 	-0.00280809	 & 	0.01009	 \\
-0.00280809	 & 	1167.78	 & 	0.00700655	 \\
0.01009	 & 	0.00700655	 & 	1167.79
\end{tabular}

\vtab
 EingenVectors - Molecule B     \\
\begin{tabular}{|c c c|}
0.395943	 & 	0.735527	 & 	-0.549753	 \\
-0.57735	 & 	0.664947	 & 	0.473828	 \\
0.71407	 & 	0.129791	 & 	0.687938
\end{tabular}

\vtab
 EingenValues - Molecule B     \\
\begin{tabular}{|c c c|}
1167.77	 & 	1167.79	 & 	1167.8	 \\
\end{tabular}

\end{center}
\end{multicols}

\vtab[-5mm]
\begin{tabular}{*{2}{m{0.38\textwidth}}}
\begin{center}
\textcolor{NavyBlue}{\Large Different}
\end{center}
&
\begin{center}
\includegraphics[height=6.5cm]{../Comparisons/Vectors/inertia_tensor_of_G_08_Ni_AFTER_DFT_and_SC_08_Ni.png}
\end{center}
\end{tabular}

 \newpage

\vtab[-3cm]
\begin{center}
{\large ClustersNi \tab Número 100}
\end{center}
\begin{multicols}{2}
\begin{center}

Molecule A \
G\_08\_Ni\_AFTER\_DFT

\includegraphics[width=6cm]{../Comparisons/ImagesFromVMD/G_08_Ni_AFTER_DFT.png}

Inertia Tensor - Molecule A \\
\begin{tabular}{|c c c|}
1190.24	 & 	6.60641	 & 	-0.89759	 \\
6.60641	 & 	1168.04	 & 	3.23374	 \\
-0.89759	 & 	3.23374	 & 	1191.18
\end{tabular}

\vtab
 EingenVectors - Molecule A     \\
\begin{tabular}{|c c c|}
-0.262972	 & 	0.955871	 & 	-0.130979	 \\
-0.0175286	 & 	0.131002	 & 	0.991227	 \\
-0.964644	 & 	-0.262961	 & 	0.0176947
\end{tabular}

\vtab
 EingenValues - Molecule A     \\
\begin{tabular}{|c c c|}
1165.78	 & 	1191.63	 & 	1192.06	 \\
\end{tabular}
\columnbreak

Molecule B \
SC\_08\_Ni\_AFTER\_DFT

\includegraphics[width=6cm]{../Comparisons/ImagesFromVMD/SC_08_Ni_AFTER_DFT.png}

Inertia Tensor - Molecule B \\
\begin{tabular}{|c c c|}
1202.23	 & 	-22.8163	 & 	-12.5979	 \\
-22.8163	 & 	1197.22	 & 	1.92415	 \\
-12.5979	 & 	1.92415	 & 	1155.53
\end{tabular}

\vtab
 EingenVectors - Molecule B     \\
\begin{tabular}{|c c c|}
0.285716	 & 	0.10352	 & 	0.952707	 \\
0.595734	 & 	0.759527	 & 	-0.261189	 \\
-0.750645	 & 	0.642185	 & 	0.155339
\end{tabular}

\vtab
 EingenValues - Molecule B     \\
\begin{tabular}{|c c c|}
1151.96	 & 	1178.67	 & 	1224.36	 \\
\end{tabular}

\end{center}
\end{multicols}

\vtab[-5mm]
\begin{tabular}{*{2}{m{0.38\textwidth}}}
\begin{center}
\textcolor{NavyBlue}{\Large Different}
\end{center}
&
\begin{center}
\includegraphics[height=6.5cm]{../Comparisons/Vectors/inertia_tensor_of_G_08_Ni_AFTER_DFT_and_SC_08_Ni_AFTER_DFT.png}
\end{center}
\end{tabular}

 \newpage

\vtab[-3cm]
\begin{center}
{\large ClustersNi \tab Número 101}
\end{center}
\begin{multicols}{2}
\begin{center}
Molecule A \\ 
G\_08\_Ni\_AFTER\_DFT
\includegraphics[width=8cm]{../Comparisons/ImagesFromVMD/G_08_Ni_AFTER_DFT.png}
\\
\vtab

\columnbreak
Molecule B \\ 
SC\_09\_Ni
\includegraphics[width=8cm]{../Comparisons/ImagesFromVMD/SC_09_Ni.png}
\\
\vtab


\end{center}
\end{multicols}
\begin{center}
\textcolor{NavyBlue}{\Large Different}
\end{center}

 \newpage

\vtab[-3cm]
\begin{center}
{\large ClustersNi \tab Número 102}
\end{center}
\begin{multicols}{2}
\begin{center}
Molecule A \\ 
G\_08\_Ni\_AFTER\_DFT
\includegraphics[width=8cm]{../Comparisons/ImagesFromVMD/G_08_Ni_AFTER_DFT.png}
\\
\vtab

\columnbreak
Molecule B \\ 
SC\_09\_Ni\_AFTER\_DFT
\includegraphics[width=8cm]{../Comparisons/ImagesFromVMD/SC_09_Ni_AFTER_DFT.png}
\\
\vtab


\end{center}
\end{multicols}
\begin{center}
\textcolor{NavyBlue}{\Large Different}
\end{center}

 \newpage

\vtab[-3cm]
\begin{center}
{\large ClustersNi \tab Número 103}
\end{center}
\begin{multicols}{2}
\begin{center}
Molecule A \\ 
G\_08\_Ni\_AFTER\_DFT
\includegraphics[width=8cm]{../Comparisons/ImagesFromVMD/G_08_Ni_AFTER_DFT.png}
\\
\vtab

\columnbreak
Molecule B \\ 
lj\_08
\includegraphics[width=8cm]{../Comparisons/ImagesFromVMD/lj_08.png}
\\
\vtab


\end{center}
\end{multicols}
\begin{center}
\textcolor{NavyBlue}{\Large Different}
\end{center}

 \newpage

\vtab[-3cm]
\begin{center}
{\large ClustersNi \tab Número 104}
\end{center}
\begin{multicols}{2}
\begin{center}
Molecule A \\ 
G\_08\_Ni\_AFTER\_DFT
\includegraphics[width=8cm]{../Comparisons/ImagesFromVMD/G_08_Ni_AFTER_DFT.png}
\\
\vtab

\columnbreak
Molecule B \\ 
lj\_08\_Ni
\includegraphics[width=8cm]{../Comparisons/ImagesFromVMD/lj_08_Ni.png}
\\
\vtab


\end{center}
\end{multicols}
\begin{center}
\textcolor{NavyBlue}{\Large Different}
\end{center}

 \newpage

\vtab[-3cm]
\begin{center}
{\large ClustersNi \tab Número 105}
\end{center}
\begin{multicols}{2}
\begin{center}

Molecule A \
G\_08\_Ni\_AFTER\_DFT

\includegraphics[width=6cm]{../Comparisons/ImagesFromVMD/G_08_Ni_AFTER_DFT.png}

Inertia Tensor - Molecule A \\
\begin{tabular}{|c c c|}
1190.24	 & 	6.60641	 & 	-0.89759	 \\
6.60641	 & 	1168.04	 & 	3.23374	 \\
-0.89759	 & 	3.23374	 & 	1191.18
\end{tabular}

\vtab
 EingenVectors - Molecule A     \\
\begin{tabular}{|c c c|}
-0.262972	 & 	0.955871	 & 	-0.130979	 \\
-0.0175286	 & 	0.131002	 & 	0.991227	 \\
-0.964644	 & 	-0.262961	 & 	0.0176947
\end{tabular}

\vtab
 EingenValues - Molecule A     \\
\begin{tabular}{|c c c|}
1165.78	 & 	1191.63	 & 	1192.06	 \\
\end{tabular}
\columnbreak

Molecule B \
lj\_08\_Ni\_AFTER\_DFT

\includegraphics[width=6cm]{../Comparisons/ImagesFromVMD/lj_08_Ni_AFTER_DFT.png}

Inertia Tensor - Molecule B \\
\begin{tabular}{|c c c|}
1461.29	 & 	-67.9001	 & 	-380.896	 \\
-67.9001	 & 	1626.58	 & 	-132.788	 \\
-380.896	 & 	-132.788	 & 	885.164
\end{tabular}

\vtab
 EingenVectors - Molecule B     \\
\begin{tabular}{|c c c|}
0.440314	 & 	0.154412	 & 	0.884466	 \\
-0.488448	 & 	-0.785372	 & 	0.380276	 \\
-0.753354	 & 	0.599456	 & 	0.270388
\end{tabular}

\vtab
 EingenValues - Molecule B     \\
\begin{tabular}{|c c c|}
672.361	 & 	1648.65	 & 	1652.02	 \\
\end{tabular}

\end{center}
\end{multicols}

\vtab[-5mm]
\begin{tabular}{*{2}{m{0.38\textwidth}}}
\begin{center}
\textcolor{NavyBlue}{\Large Different}
\end{center}
&
\begin{center}
\includegraphics[height=6.5cm]{../Comparisons/Vectors/inertia_tensor_of_G_08_Ni_AFTER_DFT_and_lj_08_Ni_AFTER_DFT.png}
\end{center}
\end{tabular}

 \newpage

\vtab[-3cm]
\begin{center}
{\large ClustersNi \tab Número 106}
\end{center}
\begin{multicols}{2}
\begin{center}
Molecule A \\ 
G\_08\_Ni\_AFTER\_DFT
\includegraphics[width=8cm]{../Comparisons/ImagesFromVMD/G_08_Ni_AFTER_DFT.png}
\\
\vtab

\columnbreak
Molecule B \\ 
lj\_09
\includegraphics[width=8cm]{../Comparisons/ImagesFromVMD/lj_09.png}
\\
\vtab


\end{center}
\end{multicols}
\begin{center}
\textcolor{NavyBlue}{\Large Different}
\end{center}

 \newpage

\vtab[-3cm]
\begin{center}
{\large ClustersNi \tab Número 107}
\end{center}
\begin{multicols}{2}
\begin{center}
Molecule A \\ 
G\_08\_Ni\_AFTER\_DFT
\includegraphics[width=8cm]{../Comparisons/ImagesFromVMD/G_08_Ni_AFTER_DFT.png}
\\
\vtab

\columnbreak
Molecule B \\ 
lj\_09\_Ni
\includegraphics[width=8cm]{../Comparisons/ImagesFromVMD/lj_09_Ni.png}
\\
\vtab


\end{center}
\end{multicols}
\begin{center}
\textcolor{NavyBlue}{\Large Different}
\end{center}

 \newpage

\vtab[-3cm]
\begin{center}
{\large ClustersNi \tab Número 108}
\end{center}
\begin{multicols}{2}
\begin{center}
Molecule A \\ 
G\_08\_Ni\_AFTER\_DFT
\includegraphics[width=8cm]{../Comparisons/ImagesFromVMD/G_08_Ni_AFTER_DFT.png}
\\
\vtab

\columnbreak
Molecule B \\ 
lj\_09\_Ni\_AFTER\_DFT
\includegraphics[width=8cm]{../Comparisons/ImagesFromVMD/lj_09_Ni_AFTER_DFT.png}
\\
\vtab


\end{center}
\end{multicols}
\begin{center}
\textcolor{NavyBlue}{\Large Different}
\end{center}

 \newpage

\vtab[-3cm]
\begin{center}
{\large ClustersNi \tab Número 109}
\end{center}
\begin{multicols}{2}
\begin{center}
Molecule A \\ 
G\_08\_Ni\_AFTER\_DFT
\includegraphics[width=8cm]{../Comparisons/ImagesFromVMD/G_08_Ni_AFTER_DFT.png}
\\
\vtab

\columnbreak
Molecule B \\ 
lj\_10
\includegraphics[width=8cm]{../Comparisons/ImagesFromVMD/lj_10.png}
\\
\vtab


\end{center}
\end{multicols}
\begin{center}
\textcolor{NavyBlue}{\Large Different}
\end{center}

 \newpage

\vtab[-3cm]
\begin{center}
{\large ClustersNi \tab Número 110}
\end{center}
\begin{multicols}{2}
\begin{center}
Molecule A \\ 
G\_08\_Ni\_AFTER\_DFT
\includegraphics[width=8cm]{../Comparisons/ImagesFromVMD/G_08_Ni_AFTER_DFT.png}
\\
\vtab

\columnbreak
Molecule B \\ 
lj\_11
\includegraphics[width=8cm]{../Comparisons/ImagesFromVMD/lj_11.png}
\\
\vtab


\end{center}
\end{multicols}
\begin{center}
\textcolor{NavyBlue}{\Large Different}
\end{center}

 \newpage

\vtab[-3cm]
\begin{center}
{\large ClustersNi \tab Número 111}
\end{center}
\begin{multicols}{2}
\begin{center}

Molecule A \
G\_09\_Ni

\includegraphics[width=6cm]{../Comparisons/ImagesFromVMD/G_09_Ni.png}

Inertia Tensor - Molecule A \\
\begin{tabular}{|c c c|}
2295.28	 & 	127.678	 & 	34.3162	 \\
127.678	 & 	2047.43	 & 	-608.198	 \\
34.3162	 & 	-608.198	 & 	1341.02
\end{tabular}

\vtab
 EingenVectors - Molecule A     \\
\begin{tabular}{|c c c|}
0.0714239	 & 	-0.501716	 & 	-0.862078	 \\
-0.850908	 & 	0.420309	 & 	-0.315112	 \\
0.520436	 & 	0.756056	 & 	-0.396895
\end{tabular}

\vtab
 EingenValues - Molecule A     \\
\begin{tabular}{|c c c|}
984.213	 & 	2244.92	 & 	2454.59	 \\
\end{tabular}
\columnbreak

Molecule B \
G\_09\_Ni\_AFTER\_DFT

\includegraphics[width=6cm]{../Comparisons/ImagesFromVMD/G_09_Ni_AFTER_DFT.png}

Inertia Tensor - Molecule B \\
\begin{tabular}{|c c c|}
2034.24	 & 	115.071	 & 	30.1504	 \\
115.071	 & 	1837.6	 & 	-545.965	 \\
30.1504	 & 	-545.965	 & 	1191.04
\end{tabular}

\vtab
 EingenVectors - Molecule B     \\
\begin{tabular}{|c c c|}
0.0718467	 & 	-0.498105	 & 	-0.864135	 \\
-0.874868	 & 	0.3846	 & 	-0.29443	 \\
0.479003	 & 	0.777158	 & 	-0.408144
\end{tabular}

\vtab
 EingenValues - Molecule B     \\
\begin{tabular}{|c c c|}
873.833	 & 	1993.8	 & 	2195.25	 \\
\end{tabular}

\end{center}
\end{multicols}

\vtab[-5mm]
\begin{tabular}{*{2}{m{0.38\textwidth}}}
\begin{center}
\textcolor{NavyBlue}{\Large Different}
\end{center}
&
\begin{center}
\includegraphics[height=6.5cm]{../Comparisons/Vectors/inertia_tensor_of_G_09_Ni_and_G_09_Ni_AFTER_DFT.png}
\end{center}
\end{tabular}

 \newpage

\vtab[-3cm]
\begin{center}
{\large ClustersNi \tab Número 112}
\end{center}
\begin{multicols}{2}
\begin{center}
Molecule A \\ 
G\_09\_Ni
\includegraphics[width=8cm]{../Comparisons/ImagesFromVMD/G_09_Ni.png}
\\
\vtab

\columnbreak
Molecule B \\ 
SC\_08\_Ni
\includegraphics[width=8cm]{../Comparisons/ImagesFromVMD/SC_08_Ni.png}
\\
\vtab


\end{center}
\end{multicols}
\begin{center}
\textcolor{NavyBlue}{\Large Different}
\end{center}

 \newpage

\vtab[-3cm]
\begin{center}
{\large ClustersNi \tab Número 113}
\end{center}
\begin{multicols}{2}
\begin{center}
Molecule A \\ 
G\_09\_Ni
\includegraphics[width=8cm]{../Comparisons/ImagesFromVMD/G_09_Ni.png}
\\
\vtab

\columnbreak
Molecule B \\ 
SC\_08\_Ni\_AFTER\_DFT
\includegraphics[width=8cm]{../Comparisons/ImagesFromVMD/SC_08_Ni_AFTER_DFT.png}
\\
\vtab


\end{center}
\end{multicols}
\begin{center}
\textcolor{NavyBlue}{\Large Different}
\end{center}

 \newpage

\vtab[-3cm]
\begin{center}
{\large ClustersNi \tab Número 114}
\end{center}
\begin{multicols}{2}
\begin{center}

Molecule A \
G\_09\_Ni

\includegraphics[width=6cm]{../Comparisons/ImagesFromVMD/G_09_Ni.png}

Inertia Tensor - Molecule A \\
\begin{tabular}{|c c c|}
2295.28	 & 	127.678	 & 	34.3162	 \\
127.678	 & 	2047.43	 & 	-608.198	 \\
34.3162	 & 	-608.198	 & 	1341.02
\end{tabular}

\vtab
 EingenVectors - Molecule A     \\
\begin{tabular}{|c c c|}
0.0714239	 & 	-0.501716	 & 	-0.862078	 \\
-0.850908	 & 	0.420309	 & 	-0.315112	 \\
0.520436	 & 	0.756056	 & 	-0.396895
\end{tabular}

\vtab
 EingenValues - Molecule A     \\
\begin{tabular}{|c c c|}
984.213	 & 	2244.92	 & 	2454.59	 \\
\end{tabular}
\columnbreak

Molecule B \
SC\_09\_Ni

\includegraphics[width=6cm]{../Comparisons/ImagesFromVMD/SC_09_Ni.png}

Inertia Tensor - Molecule B \\
\begin{tabular}{|c c c|}
1536.98	 & 	56.225	 & 	-101.457	 \\
56.225	 & 	1328.77	 & 	-222.686	 \\
-101.457	 & 	-222.686	 & 	1805.96
\end{tabular}

\vtab
 EingenVectors - Molecule B     \\
\begin{tabular}{|c c c|}
0.0545275	 & 	-0.93243	 & 	-0.357212	 \\
-0.957676	 & 	0.0524255	 & 	-0.283033	 \\
-0.282636	 & 	-0.357527	 & 	0.890108
\end{tabular}

\vtab
 EingenValues - Molecule B     \\
\begin{tabular}{|c c c|}
1240.17	 & 	1503.92	 & 	1927.62	 \\
\end{tabular}

\end{center}
\end{multicols}

\vtab[-5mm]
\begin{tabular}{*{2}{m{0.38\textwidth}}}
\begin{center}
\textcolor{NavyBlue}{\Large Different}
\end{center}
&
\begin{center}
\includegraphics[height=6.5cm]{../Comparisons/Vectors/inertia_tensor_of_G_09_Ni_and_SC_09_Ni.png}
\end{center}
\end{tabular}

 \newpage

\vtab[-3cm]
\begin{center}
{\large ClustersNi \tab Número 115}
\end{center}
\begin{multicols}{2}
\begin{center}

Molecule A \
G\_09\_Ni

\includegraphics[width=6cm]{../Comparisons/ImagesFromVMD/G_09_Ni.png}

Inertia Tensor - Molecule A \\
\begin{tabular}{|c c c|}
2295.28	 & 	127.678	 & 	34.3162	 \\
127.678	 & 	2047.43	 & 	-608.198	 \\
34.3162	 & 	-608.198	 & 	1341.02
\end{tabular}

\vtab
 EingenVectors - Molecule A     \\
\begin{tabular}{|c c c|}
0.0714239	 & 	-0.501716	 & 	-0.862078	 \\
-0.850908	 & 	0.420309	 & 	-0.315112	 \\
0.520436	 & 	0.756056	 & 	-0.396895
\end{tabular}

\vtab
 EingenValues - Molecule A     \\
\begin{tabular}{|c c c|}
984.213	 & 	2244.92	 & 	2454.59	 \\
\end{tabular}
\columnbreak

Molecule B \
SC\_09\_Ni\_AFTER\_DFT

\includegraphics[width=6cm]{../Comparisons/ImagesFromVMD/SC_09_Ni_AFTER_DFT.png}

Inertia Tensor - Molecule B \\
\begin{tabular}{|c c c|}
1526.11	 & 	61.3141	 & 	-104.189	 \\
61.3141	 & 	1367.3	 & 	-214.394	 \\
-104.189	 & 	-214.394	 & 	1801.23
\end{tabular}

\vtab
 EingenVectors - Molecule B     \\
\begin{tabular}{|c c c|}
0.0759984	 & 	-0.927891	 & 	-0.365024	 \\
-0.954757	 & 	0.0378395	 & 	-0.294969	 \\
-0.287511	 & 	-0.370927	 & 	0.883035
\end{tabular}

\vtab
 EingenValues - Molecule B     \\
\begin{tabular}{|c c c|}
1277.93	 & 	1491.49	 & 	1925.21	 \\
\end{tabular}

\end{center}
\end{multicols}

\vtab[-5mm]
\begin{tabular}{*{2}{m{0.38\textwidth}}}
\begin{center}
\textcolor{NavyBlue}{\Large Different}
\end{center}
&
\begin{center}
\includegraphics[height=6.5cm]{../Comparisons/Vectors/inertia_tensor_of_G_09_Ni_and_SC_09_Ni_AFTER_DFT.png}
\end{center}
\end{tabular}

 \newpage

\vtab[-3cm]
\begin{center}
{\large ClustersNi \tab Número 116}
\end{center}
\begin{multicols}{2}
\begin{center}
Molecule A \\ 
G\_09\_Ni
\includegraphics[width=8cm]{../Comparisons/ImagesFromVMD/G_09_Ni.png}
\\
\vtab

\columnbreak
Molecule B \\ 
lj\_08
\includegraphics[width=8cm]{../Comparisons/ImagesFromVMD/lj_08.png}
\\
\vtab


\end{center}
\end{multicols}
\begin{center}
\textcolor{NavyBlue}{\Large Different}
\end{center}

 \newpage

\vtab[-3cm]
\begin{center}
{\large ClustersNi \tab Número 117}
\end{center}
\begin{multicols}{2}
\begin{center}
Molecule A \\ 
G\_09\_Ni
\includegraphics[width=8cm]{../Comparisons/ImagesFromVMD/G_09_Ni.png}
\\
\vtab

\columnbreak
Molecule B \\ 
lj\_08\_Ni
\includegraphics[width=8cm]{../Comparisons/ImagesFromVMD/lj_08_Ni.png}
\\
\vtab


\end{center}
\end{multicols}
\begin{center}
\textcolor{NavyBlue}{\Large Different}
\end{center}

 \newpage

\vtab[-3cm]
\begin{center}
{\large ClustersNi \tab Número 118}
\end{center}
\begin{multicols}{2}
\begin{center}
Molecule A \\ 
G\_09\_Ni
\includegraphics[width=8cm]{../Comparisons/ImagesFromVMD/G_09_Ni.png}
\\
\vtab

\columnbreak
Molecule B \\ 
lj\_08\_Ni\_AFTER\_DFT
\includegraphics[width=8cm]{../Comparisons/ImagesFromVMD/lj_08_Ni_AFTER_DFT.png}
\\
\vtab


\end{center}
\end{multicols}
\begin{center}
\textcolor{NavyBlue}{\Large Different}
\end{center}

 \newpage

\vtab[-3cm]
\begin{center}
{\large ClustersNi \tab Número 119}
\end{center}
\begin{multicols}{2}
\begin{center}
Molecule A \\ 
G\_09\_Ni
\includegraphics[width=8cm]{../Comparisons/ImagesFromVMD/G_09_Ni.png}
\\
\vtab

\columnbreak
Molecule B \\ 
lj\_09
\includegraphics[width=8cm]{../Comparisons/ImagesFromVMD/lj_09.png}
\\
\vtab


\end{center}
\end{multicols}
\begin{center}
\textcolor{NavyBlue}{\Large Different}
\end{center}

 \newpage

\vtab[-3cm]
\begin{center}
{\large ClustersNi \tab Número 120}
\end{center}
\begin{multicols}{2}
\begin{center}

Molecule A \
G\_09\_Ni

\includegraphics[width=6cm]{../Comparisons/ImagesFromVMD/G_09_Ni.png}

Inertia Tensor - Molecule A \\
\begin{tabular}{|c c c|}
2295.28	 & 	127.678	 & 	34.3162	 \\
127.678	 & 	2047.43	 & 	-608.198	 \\
34.3162	 & 	-608.198	 & 	1341.02
\end{tabular}

\vtab
 EingenVectors - Molecule A     \\
\begin{tabular}{|c c c|}
0.0714239	 & 	-0.501716	 & 	-0.862078	 \\
-0.850908	 & 	0.420309	 & 	-0.315112	 \\
0.520436	 & 	0.756056	 & 	-0.396895
\end{tabular}

\vtab
 EingenValues - Molecule A     \\
\begin{tabular}{|c c c|}
984.213	 & 	2244.92	 & 	2454.59	 \\
\end{tabular}
\columnbreak

Molecule B \
lj\_09\_Ni

\includegraphics[width=6cm]{../Comparisons/ImagesFromVMD/lj_09_Ni.png}

Inertia Tensor - Molecule B \\
\begin{tabular}{|c c c|}
2099.87	 & 	-92.0868	 & 	87.5491	 \\
-92.0868	 & 	1185.59	 & 	628.538	 \\
87.5491	 & 	628.538	 & 	1533.16
\end{tabular}

\vtab
 EingenVectors - Molecule B     \\
\begin{tabular}{|c c c|}
-0.0896058	 & 	-0.791993	 & 	0.603919	 \\
0.134366	 & 	-0.610428	 & 	-0.780592	 \\
-0.986872	 & 	-0.0112006	 & 	-0.161115
\end{tabular}

\vtab
 EingenValues - Molecule B     \\
\begin{tabular}{|c c c|}
695.893	 & 	2009.61	 & 	2113.12	 \\
\end{tabular}

\end{center}
\end{multicols}

\vtab[-5mm]
\begin{tabular}{*{2}{m{0.38\textwidth}}}
\begin{center}
\textcolor{NavyBlue}{\Large Different}
\end{center}
&
\begin{center}
\includegraphics[height=6.5cm]{../Comparisons/Vectors/inertia_tensor_of_G_09_Ni_and_lj_09_Ni.png}
\end{center}
\end{tabular}

 \newpage

\vtab[-3cm]
\begin{center}
{\large ClustersNi \tab Número 121}
\end{center}
\begin{multicols}{2}
\begin{center}

Molecule A \
G\_09\_Ni

\includegraphics[width=6cm]{../Comparisons/ImagesFromVMD/G_09_Ni.png}

Inertia Tensor - Molecule A \\
\begin{tabular}{|c c c|}
2295.28	 & 	127.678	 & 	34.3162	 \\
127.678	 & 	2047.43	 & 	-608.198	 \\
34.3162	 & 	-608.198	 & 	1341.02
\end{tabular}

\vtab
 EingenVectors - Molecule A     \\
\begin{tabular}{|c c c|}
0.0714239	 & 	-0.501716	 & 	-0.862078	 \\
-0.850908	 & 	0.420309	 & 	-0.315112	 \\
0.520436	 & 	0.756056	 & 	-0.396895
\end{tabular}

\vtab
 EingenValues - Molecule A     \\
\begin{tabular}{|c c c|}
984.213	 & 	2244.92	 & 	2454.59	 \\
\end{tabular}
\columnbreak

Molecule B \
lj\_09\_Ni\_AFTER\_DFT

\includegraphics[width=6cm]{../Comparisons/ImagesFromVMD/lj_09_Ni_AFTER_DFT.png}

Inertia Tensor - Molecule B \\
\begin{tabular}{|c c c|}
2192.28	 & 	-89.3917	 & 	85.1517	 \\
-89.3917	 & 	1303.55	 & 	619.274	 \\
85.1517	 & 	619.274	 & 	1643.51
\end{tabular}

\vtab
 EingenVectors - Molecule B     \\
\begin{tabular}{|c c c|}
-0.0891023	 & 	-0.791474	 & 	0.604673	 \\
0.147893	 & 	-0.610872	 & 	-0.777794	 \\
-0.984981	 & 	-0.0201236	 & 	-0.171483
\end{tabular}

\vtab
 EingenValues - Molecule B     \\
\begin{tabular}{|c c c|}
820.373	 & 	2113.69	 & 	2205.28	 \\
\end{tabular}

\end{center}
\end{multicols}

\vtab[-5mm]
\begin{tabular}{*{2}{m{0.38\textwidth}}}
\begin{center}
\textcolor{NavyBlue}{\Large Different}
\end{center}
&
\begin{center}
\includegraphics[height=6.5cm]{../Comparisons/Vectors/inertia_tensor_of_G_09_Ni_and_lj_09_Ni_AFTER_DFT.png}
\end{center}
\end{tabular}

 \newpage

\vtab[-3cm]
\begin{center}
{\large ClustersNi \tab Número 122}
\end{center}
\begin{multicols}{2}
\begin{center}
Molecule A \\ 
G\_09\_Ni
\includegraphics[width=8cm]{../Comparisons/ImagesFromVMD/G_09_Ni.png}
\\
\vtab

\columnbreak
Molecule B \\ 
lj\_10
\includegraphics[width=8cm]{../Comparisons/ImagesFromVMD/lj_10.png}
\\
\vtab


\end{center}
\end{multicols}
\begin{center}
\textcolor{NavyBlue}{\Large Different}
\end{center}

 \newpage

\vtab[-3cm]
\begin{center}
{\large ClustersNi \tab Número 123}
\end{center}
\begin{multicols}{2}
\begin{center}
Molecule A \\ 
G\_09\_Ni
\includegraphics[width=8cm]{../Comparisons/ImagesFromVMD/G_09_Ni.png}
\\
\vtab

\columnbreak
Molecule B \\ 
lj\_11
\includegraphics[width=8cm]{../Comparisons/ImagesFromVMD/lj_11.png}
\\
\vtab


\end{center}
\end{multicols}
\begin{center}
\textcolor{NavyBlue}{\Large Different}
\end{center}

 \newpage

\vtab[-3cm]
\begin{center}
{\large ClustersNi \tab Número 124}
\end{center}
\begin{multicols}{2}
\begin{center}
Molecule A \\ 
G\_09\_Ni\_AFTER\_DFT
\includegraphics[width=8cm]{../Comparisons/ImagesFromVMD/G_09_Ni_AFTER_DFT.png}
\\
\vtab

\columnbreak
Molecule B \\ 
SC\_08\_Ni
\includegraphics[width=8cm]{../Comparisons/ImagesFromVMD/SC_08_Ni.png}
\\
\vtab


\end{center}
\end{multicols}
\begin{center}
\textcolor{NavyBlue}{\Large Different}
\end{center}

 \newpage

\vtab[-3cm]
\begin{center}
{\large ClustersNi \tab Número 125}
\end{center}
\begin{multicols}{2}
\begin{center}
Molecule A \\ 
G\_09\_Ni\_AFTER\_DFT
\includegraphics[width=8cm]{../Comparisons/ImagesFromVMD/G_09_Ni_AFTER_DFT.png}
\\
\vtab

\columnbreak
Molecule B \\ 
SC\_08\_Ni\_AFTER\_DFT
\includegraphics[width=8cm]{../Comparisons/ImagesFromVMD/SC_08_Ni_AFTER_DFT.png}
\\
\vtab


\end{center}
\end{multicols}
\begin{center}
\textcolor{NavyBlue}{\Large Different}
\end{center}

 \newpage

\vtab[-3cm]
\begin{center}
{\large ClustersNi \tab Número 126}
\end{center}
\begin{multicols}{2}
\begin{center}

Molecule A \
G\_09\_Ni\_AFTER\_DFT

\includegraphics[width=6cm]{../Comparisons/ImagesFromVMD/G_09_Ni_AFTER_DFT.png}

Inertia Tensor - Molecule A \\
\begin{tabular}{|c c c|}
2034.24	 & 	115.071	 & 	30.1504	 \\
115.071	 & 	1837.6	 & 	-545.965	 \\
30.1504	 & 	-545.965	 & 	1191.04
\end{tabular}

\vtab
 EingenVectors - Molecule A     \\
\begin{tabular}{|c c c|}
0.0718467	 & 	-0.498105	 & 	-0.864135	 \\
-0.874868	 & 	0.3846	 & 	-0.29443	 \\
0.479003	 & 	0.777158	 & 	-0.408144
\end{tabular}

\vtab
 EingenValues - Molecule A     \\
\begin{tabular}{|c c c|}
873.833	 & 	1993.8	 & 	2195.25	 \\
\end{tabular}
\columnbreak

Molecule B \
SC\_09\_Ni

\includegraphics[width=6cm]{../Comparisons/ImagesFromVMD/SC_09_Ni.png}

Inertia Tensor - Molecule B \\
\begin{tabular}{|c c c|}
1536.98	 & 	56.225	 & 	-101.457	 \\
56.225	 & 	1328.77	 & 	-222.686	 \\
-101.457	 & 	-222.686	 & 	1805.96
\end{tabular}

\vtab
 EingenVectors - Molecule B     \\
\begin{tabular}{|c c c|}
0.0545275	 & 	-0.93243	 & 	-0.357212	 \\
-0.957676	 & 	0.0524255	 & 	-0.283033	 \\
-0.282636	 & 	-0.357527	 & 	0.890108
\end{tabular}

\vtab
 EingenValues - Molecule B     \\
\begin{tabular}{|c c c|}
1240.17	 & 	1503.92	 & 	1927.62	 \\
\end{tabular}

\end{center}
\end{multicols}

\vtab[-5mm]
\begin{tabular}{*{2}{m{0.38\textwidth}}}
\begin{center}
\textcolor{NavyBlue}{\Large Different}
\end{center}
&
\begin{center}
\includegraphics[height=6.5cm]{../Comparisons/Vectors/inertia_tensor_of_G_09_Ni_AFTER_DFT_and_SC_09_Ni.png}
\end{center}
\end{tabular}

 \newpage

\vtab[-3cm]
\begin{center}
{\large ClustersNi \tab Número 127}
\end{center}
\begin{multicols}{2}
\begin{center}

Molecule A \
G\_09\_Ni\_AFTER\_DFT

\includegraphics[width=6cm]{../Comparisons/ImagesFromVMD/G_09_Ni_AFTER_DFT.png}

Inertia Tensor - Molecule A \\
\begin{tabular}{|c c c|}
2034.24	 & 	115.071	 & 	30.1504	 \\
115.071	 & 	1837.6	 & 	-545.965	 \\
30.1504	 & 	-545.965	 & 	1191.04
\end{tabular}

\vtab
 EingenVectors - Molecule A     \\
\begin{tabular}{|c c c|}
0.0718467	 & 	-0.498105	 & 	-0.864135	 \\
-0.874868	 & 	0.3846	 & 	-0.29443	 \\
0.479003	 & 	0.777158	 & 	-0.408144
\end{tabular}

\vtab
 EingenValues - Molecule A     \\
\begin{tabular}{|c c c|}
873.833	 & 	1993.8	 & 	2195.25	 \\
\end{tabular}
\columnbreak

Molecule B \
SC\_09\_Ni\_AFTER\_DFT

\includegraphics[width=6cm]{../Comparisons/ImagesFromVMD/SC_09_Ni_AFTER_DFT.png}

Inertia Tensor - Molecule B \\
\begin{tabular}{|c c c|}
1526.11	 & 	61.3141	 & 	-104.189	 \\
61.3141	 & 	1367.3	 & 	-214.394	 \\
-104.189	 & 	-214.394	 & 	1801.23
\end{tabular}

\vtab
 EingenVectors - Molecule B     \\
\begin{tabular}{|c c c|}
0.0759984	 & 	-0.927891	 & 	-0.365024	 \\
-0.954757	 & 	0.0378395	 & 	-0.294969	 \\
-0.287511	 & 	-0.370927	 & 	0.883035
\end{tabular}

\vtab
 EingenValues - Molecule B     \\
\begin{tabular}{|c c c|}
1277.93	 & 	1491.49	 & 	1925.21	 \\
\end{tabular}

\end{center}
\end{multicols}

\vtab[-5mm]
\begin{tabular}{*{2}{m{0.38\textwidth}}}
\begin{center}
\textcolor{NavyBlue}{\Large Different}
\end{center}
&
\begin{center}
\includegraphics[height=6.5cm]{../Comparisons/Vectors/inertia_tensor_of_G_09_Ni_AFTER_DFT_and_SC_09_Ni_AFTER_DFT.png}
\end{center}
\end{tabular}

 \newpage

\vtab[-3cm]
\begin{center}
{\large ClustersNi \tab Número 128}
\end{center}
\begin{multicols}{2}
\begin{center}
Molecule A \\ 
G\_09\_Ni\_AFTER\_DFT
\includegraphics[width=8cm]{../Comparisons/ImagesFromVMD/G_09_Ni_AFTER_DFT.png}
\\
\vtab

\columnbreak
Molecule B \\ 
lj\_08
\includegraphics[width=8cm]{../Comparisons/ImagesFromVMD/lj_08.png}
\\
\vtab


\end{center}
\end{multicols}
\begin{center}
\textcolor{NavyBlue}{\Large Different}
\end{center}

 \newpage

\vtab[-3cm]
\begin{center}
{\large ClustersNi \tab Número 129}
\end{center}
\begin{multicols}{2}
\begin{center}
Molecule A \\ 
G\_09\_Ni\_AFTER\_DFT
\includegraphics[width=8cm]{../Comparisons/ImagesFromVMD/G_09_Ni_AFTER_DFT.png}
\\
\vtab

\columnbreak
Molecule B \\ 
lj\_08\_Ni
\includegraphics[width=8cm]{../Comparisons/ImagesFromVMD/lj_08_Ni.png}
\\
\vtab


\end{center}
\end{multicols}
\begin{center}
\textcolor{NavyBlue}{\Large Different}
\end{center}

 \newpage

\vtab[-3cm]
\begin{center}
{\large ClustersNi \tab Número 130}
\end{center}
\begin{multicols}{2}
\begin{center}
Molecule A \\ 
G\_09\_Ni\_AFTER\_DFT
\includegraphics[width=8cm]{../Comparisons/ImagesFromVMD/G_09_Ni_AFTER_DFT.png}
\\
\vtab

\columnbreak
Molecule B \\ 
lj\_08\_Ni\_AFTER\_DFT
\includegraphics[width=8cm]{../Comparisons/ImagesFromVMD/lj_08_Ni_AFTER_DFT.png}
\\
\vtab


\end{center}
\end{multicols}
\begin{center}
\textcolor{NavyBlue}{\Large Different}
\end{center}

 \newpage

\vtab[-3cm]
\begin{center}
{\large ClustersNi \tab Número 131}
\end{center}
\begin{multicols}{2}
\begin{center}
Molecule A \\ 
G\_09\_Ni\_AFTER\_DFT
\includegraphics[width=8cm]{../Comparisons/ImagesFromVMD/G_09_Ni_AFTER_DFT.png}
\\
\vtab

\columnbreak
Molecule B \\ 
lj\_09
\includegraphics[width=8cm]{../Comparisons/ImagesFromVMD/lj_09.png}
\\
\vtab


\end{center}
\end{multicols}
\begin{center}
\textcolor{NavyBlue}{\Large Different}
\end{center}

 \newpage

\vtab[-3cm]
\begin{center}
{\large ClustersNi \tab Número 132}
\end{center}
\begin{multicols}{2}
\begin{center}

Molecule A \
G\_09\_Ni\_AFTER\_DFT

\includegraphics[width=6cm]{../Comparisons/ImagesFromVMD/G_09_Ni_AFTER_DFT.png}

Inertia Tensor - Molecule A \\
\begin{tabular}{|c c c|}
2034.24	 & 	115.071	 & 	30.1504	 \\
115.071	 & 	1837.6	 & 	-545.965	 \\
30.1504	 & 	-545.965	 & 	1191.04
\end{tabular}

\vtab
 EingenVectors - Molecule A     \\
\begin{tabular}{|c c c|}
0.0718467	 & 	-0.498105	 & 	-0.864135	 \\
-0.874868	 & 	0.3846	 & 	-0.29443	 \\
0.479003	 & 	0.777158	 & 	-0.408144
\end{tabular}

\vtab
 EingenValues - Molecule A     \\
\begin{tabular}{|c c c|}
873.833	 & 	1993.8	 & 	2195.25	 \\
\end{tabular}
\columnbreak

Molecule B \
lj\_09\_Ni

\includegraphics[width=6cm]{../Comparisons/ImagesFromVMD/lj_09_Ni.png}

Inertia Tensor - Molecule B \\
\begin{tabular}{|c c c|}
2099.87	 & 	-92.0868	 & 	87.5491	 \\
-92.0868	 & 	1185.59	 & 	628.538	 \\
87.5491	 & 	628.538	 & 	1533.16
\end{tabular}

\vtab
 EingenVectors - Molecule B     \\
\begin{tabular}{|c c c|}
-0.0896058	 & 	-0.791993	 & 	0.603919	 \\
0.134366	 & 	-0.610428	 & 	-0.780592	 \\
-0.986872	 & 	-0.0112006	 & 	-0.161115
\end{tabular}

\vtab
 EingenValues - Molecule B     \\
\begin{tabular}{|c c c|}
695.893	 & 	2009.61	 & 	2113.12	 \\
\end{tabular}

\end{center}
\end{multicols}

\vtab[-5mm]
\begin{tabular}{*{2}{m{0.38\textwidth}}}
\begin{center}
\textcolor{NavyBlue}{\Large Different}
\end{center}
&
\begin{center}
\includegraphics[height=6.5cm]{../Comparisons/Vectors/inertia_tensor_of_G_09_Ni_AFTER_DFT_and_lj_09_Ni.png}
\end{center}
\end{tabular}

 \newpage

\vtab[-3cm]
\begin{center}
{\large ClustersNi \tab Número 133}
\end{center}
\begin{multicols}{2}
\begin{center}

Molecule A \
G\_09\_Ni\_AFTER\_DFT

\includegraphics[width=6cm]{../Comparisons/ImagesFromVMD/G_09_Ni_AFTER_DFT.png}

Inertia Tensor - Molecule A \\
\begin{tabular}{|c c c|}
2034.24	 & 	115.071	 & 	30.1504	 \\
115.071	 & 	1837.6	 & 	-545.965	 \\
30.1504	 & 	-545.965	 & 	1191.04
\end{tabular}

\vtab
 EingenVectors - Molecule A     \\
\begin{tabular}{|c c c|}
0.0718467	 & 	-0.498105	 & 	-0.864135	 \\
-0.874868	 & 	0.3846	 & 	-0.29443	 \\
0.479003	 & 	0.777158	 & 	-0.408144
\end{tabular}

\vtab
 EingenValues - Molecule A     \\
\begin{tabular}{|c c c|}
873.833	 & 	1993.8	 & 	2195.25	 \\
\end{tabular}
\columnbreak

Molecule B \
lj\_09\_Ni\_AFTER\_DFT

\includegraphics[width=6cm]{../Comparisons/ImagesFromVMD/lj_09_Ni_AFTER_DFT.png}

Inertia Tensor - Molecule B \\
\begin{tabular}{|c c c|}
2192.28	 & 	-89.3917	 & 	85.1517	 \\
-89.3917	 & 	1303.55	 & 	619.274	 \\
85.1517	 & 	619.274	 & 	1643.51
\end{tabular}

\vtab
 EingenVectors - Molecule B     \\
\begin{tabular}{|c c c|}
-0.0891023	 & 	-0.791474	 & 	0.604673	 \\
0.147893	 & 	-0.610872	 & 	-0.777794	 \\
-0.984981	 & 	-0.0201236	 & 	-0.171483
\end{tabular}

\vtab
 EingenValues - Molecule B     \\
\begin{tabular}{|c c c|}
820.373	 & 	2113.69	 & 	2205.28	 \\
\end{tabular}

\end{center}
\end{multicols}

\vtab[-5mm]
\begin{tabular}{*{2}{m{0.38\textwidth}}}
\begin{center}
\textcolor{NavyBlue}{\Large Different}
\end{center}
&
\begin{center}
\includegraphics[height=6.5cm]{../Comparisons/Vectors/inertia_tensor_of_G_09_Ni_AFTER_DFT_and_lj_09_Ni_AFTER_DFT.png}
\end{center}
\end{tabular}

 \newpage

\vtab[-3cm]
\begin{center}
{\large ClustersNi \tab Número 134}
\end{center}
\begin{multicols}{2}
\begin{center}
Molecule A \\ 
G\_09\_Ni\_AFTER\_DFT
\includegraphics[width=8cm]{../Comparisons/ImagesFromVMD/G_09_Ni_AFTER_DFT.png}
\\
\vtab

\columnbreak
Molecule B \\ 
lj\_10
\includegraphics[width=8cm]{../Comparisons/ImagesFromVMD/lj_10.png}
\\
\vtab


\end{center}
\end{multicols}
\begin{center}
\textcolor{NavyBlue}{\Large Different}
\end{center}

 \newpage

\vtab[-3cm]
\begin{center}
{\large ClustersNi \tab Número 135}
\end{center}
\begin{multicols}{2}
\begin{center}
Molecule A \\ 
G\_09\_Ni\_AFTER\_DFT
\includegraphics[width=8cm]{../Comparisons/ImagesFromVMD/G_09_Ni_AFTER_DFT.png}
\\
\vtab

\columnbreak
Molecule B \\ 
lj\_11
\includegraphics[width=8cm]{../Comparisons/ImagesFromVMD/lj_11.png}
\\
\vtab


\end{center}
\end{multicols}
\begin{center}
\textcolor{NavyBlue}{\Large Different}
\end{center}

 \newpage

\vtab[-3cm]
\begin{center}
{\large ClustersNi \tab Número 136}
\end{center}
\begin{multicols}{2}
\begin{center}

Molecule A \
SC\_08\_Ni

\includegraphics[width=6cm]{../Comparisons/ImagesFromVMD/SC_08_Ni.png}

Inertia Tensor - Molecule A \\
\begin{tabular}{|c c c|}
1167.79	 & 	-0.00280809	 & 	0.01009	 \\
-0.00280809	 & 	1167.78	 & 	0.00700655	 \\
0.01009	 & 	0.00700655	 & 	1167.79
\end{tabular}

\vtab
 EingenVectors - Molecule A     \\
\begin{tabular}{|c c c|}
0.395943	 & 	0.735527	 & 	-0.549753	 \\
-0.57735	 & 	0.664947	 & 	0.473828	 \\
0.71407	 & 	0.129791	 & 	0.687938
\end{tabular}

\vtab
 EingenValues - Molecule A     \\
\begin{tabular}{|c c c|}
1167.77	 & 	1167.79	 & 	1167.8	 \\
\end{tabular}
\columnbreak

Molecule B \
SC\_08\_Ni\_AFTER\_DFT

\includegraphics[width=6cm]{../Comparisons/ImagesFromVMD/SC_08_Ni_AFTER_DFT.png}

Inertia Tensor - Molecule B \\
\begin{tabular}{|c c c|}
1202.23	 & 	-22.8163	 & 	-12.5979	 \\
-22.8163	 & 	1197.22	 & 	1.92415	 \\
-12.5979	 & 	1.92415	 & 	1155.53
\end{tabular}

\vtab
 EingenVectors - Molecule B     \\
\begin{tabular}{|c c c|}
0.285716	 & 	0.10352	 & 	0.952707	 \\
0.595734	 & 	0.759527	 & 	-0.261189	 \\
-0.750645	 & 	0.642185	 & 	0.155339
\end{tabular}

\vtab
 EingenValues - Molecule B     \\
\begin{tabular}{|c c c|}
1151.96	 & 	1178.67	 & 	1224.36	 \\
\end{tabular}

\end{center}
\end{multicols}

\vtab[-5mm]
\begin{tabular}{*{2}{m{0.38\textwidth}}}
\begin{center}
\textcolor{NavyBlue}{\Large Different}
\end{center}
&
\begin{center}
\includegraphics[height=6.5cm]{../Comparisons/Vectors/inertia_tensor_of_SC_08_Ni_and_SC_08_Ni_AFTER_DFT.png}
\end{center}
\end{tabular}

 \newpage

\vtab[-3cm]
\begin{center}
{\large ClustersNi \tab Número 137}
\end{center}
\begin{multicols}{2}
\begin{center}
Molecule A \\ 
SC\_08\_Ni
\includegraphics[width=8cm]{../Comparisons/ImagesFromVMD/SC_08_Ni.png}
\\
\vtab

\columnbreak
Molecule B \\ 
SC\_09\_Ni
\includegraphics[width=8cm]{../Comparisons/ImagesFromVMD/SC_09_Ni.png}
\\
\vtab


\end{center}
\end{multicols}
\begin{center}
\textcolor{NavyBlue}{\Large Different}
\end{center}

 \newpage

\vtab[-3cm]
\begin{center}
{\large ClustersNi \tab Número 138}
\end{center}
\begin{multicols}{2}
\begin{center}
Molecule A \\ 
SC\_08\_Ni
\includegraphics[width=8cm]{../Comparisons/ImagesFromVMD/SC_08_Ni.png}
\\
\vtab

\columnbreak
Molecule B \\ 
SC\_09\_Ni\_AFTER\_DFT
\includegraphics[width=8cm]{../Comparisons/ImagesFromVMD/SC_09_Ni_AFTER_DFT.png}
\\
\vtab


\end{center}
\end{multicols}
\begin{center}
\textcolor{NavyBlue}{\Large Different}
\end{center}

 \newpage

\vtab[-3cm]
\begin{center}
{\large ClustersNi \tab Número 139}
\end{center}
\begin{multicols}{2}
\begin{center}
Molecule A \\ 
SC\_08\_Ni
\includegraphics[width=8cm]{../Comparisons/ImagesFromVMD/SC_08_Ni.png}
\\
\vtab

\columnbreak
Molecule B \\ 
lj\_08
\includegraphics[width=8cm]{../Comparisons/ImagesFromVMD/lj_08.png}
\\
\vtab


\end{center}
\end{multicols}
\begin{center}
\textcolor{NavyBlue}{\Large Different}
\end{center}

 \newpage

\vtab[-3cm]
\begin{center}
{\large ClustersNi \tab Número 140}
\end{center}
\begin{multicols}{2}
\begin{center}
Molecule A \\ 
SC\_08\_Ni
\includegraphics[width=8cm]{../Comparisons/ImagesFromVMD/SC_08_Ni.png}
\\
\vtab

\columnbreak
Molecule B \\ 
lj\_08\_Ni
\includegraphics[width=8cm]{../Comparisons/ImagesFromVMD/lj_08_Ni.png}
\\
\vtab


\end{center}
\end{multicols}
\begin{center}
\textcolor{NavyBlue}{\Large Different}
\end{center}

 \newpage

\vtab[-3cm]
\begin{center}
{\large ClustersNi \tab Número 141}
\end{center}
\begin{multicols}{2}
\begin{center}

Molecule A \
SC\_08\_Ni

\includegraphics[width=6cm]{../Comparisons/ImagesFromVMD/SC_08_Ni.png}

Inertia Tensor - Molecule A \\
\begin{tabular}{|c c c|}
1167.79	 & 	-0.00280809	 & 	0.01009	 \\
-0.00280809	 & 	1167.78	 & 	0.00700655	 \\
0.01009	 & 	0.00700655	 & 	1167.79
\end{tabular}

\vtab
 EingenVectors - Molecule A     \\
\begin{tabular}{|c c c|}
0.395943	 & 	0.735527	 & 	-0.549753	 \\
-0.57735	 & 	0.664947	 & 	0.473828	 \\
0.71407	 & 	0.129791	 & 	0.687938
\end{tabular}

\vtab
 EingenValues - Molecule A     \\
\begin{tabular}{|c c c|}
1167.77	 & 	1167.79	 & 	1167.8	 \\
\end{tabular}
\columnbreak

Molecule B \
lj\_08\_Ni\_AFTER\_DFT

\includegraphics[width=6cm]{../Comparisons/ImagesFromVMD/lj_08_Ni_AFTER_DFT.png}

Inertia Tensor - Molecule B \\
\begin{tabular}{|c c c|}
1461.29	 & 	-67.9001	 & 	-380.896	 \\
-67.9001	 & 	1626.58	 & 	-132.788	 \\
-380.896	 & 	-132.788	 & 	885.164
\end{tabular}

\vtab
 EingenVectors - Molecule B     \\
\begin{tabular}{|c c c|}
0.440314	 & 	0.154412	 & 	0.884466	 \\
-0.488448	 & 	-0.785372	 & 	0.380276	 \\
-0.753354	 & 	0.599456	 & 	0.270388
\end{tabular}

\vtab
 EingenValues - Molecule B     \\
\begin{tabular}{|c c c|}
672.361	 & 	1648.65	 & 	1652.02	 \\
\end{tabular}

\end{center}
\end{multicols}

\vtab[-5mm]
\begin{tabular}{*{2}{m{0.38\textwidth}}}
\begin{center}
\textcolor{NavyBlue}{\Large Different}
\end{center}
&
\begin{center}
\includegraphics[height=6.5cm]{../Comparisons/Vectors/inertia_tensor_of_SC_08_Ni_and_lj_08_Ni_AFTER_DFT.png}
\end{center}
\end{tabular}

 \newpage

\vtab[-3cm]
\begin{center}
{\large ClustersNi \tab Número 142}
\end{center}
\begin{multicols}{2}
\begin{center}
Molecule A \\ 
SC\_08\_Ni
\includegraphics[width=8cm]{../Comparisons/ImagesFromVMD/SC_08_Ni.png}
\\
\vtab

\columnbreak
Molecule B \\ 
lj\_09
\includegraphics[width=8cm]{../Comparisons/ImagesFromVMD/lj_09.png}
\\
\vtab


\end{center}
\end{multicols}
\begin{center}
\textcolor{NavyBlue}{\Large Different}
\end{center}

 \newpage

\vtab[-3cm]
\begin{center}
{\large ClustersNi \tab Número 143}
\end{center}
\begin{multicols}{2}
\begin{center}
Molecule A \\ 
SC\_08\_Ni
\includegraphics[width=8cm]{../Comparisons/ImagesFromVMD/SC_08_Ni.png}
\\
\vtab

\columnbreak
Molecule B \\ 
lj\_09\_Ni
\includegraphics[width=8cm]{../Comparisons/ImagesFromVMD/lj_09_Ni.png}
\\
\vtab


\end{center}
\end{multicols}
\begin{center}
\textcolor{NavyBlue}{\Large Different}
\end{center}

 \newpage

\vtab[-3cm]
\begin{center}
{\large ClustersNi \tab Número 144}
\end{center}
\begin{multicols}{2}
\begin{center}
Molecule A \\ 
SC\_08\_Ni
\includegraphics[width=8cm]{../Comparisons/ImagesFromVMD/SC_08_Ni.png}
\\
\vtab

\columnbreak
Molecule B \\ 
lj\_09\_Ni\_AFTER\_DFT
\includegraphics[width=8cm]{../Comparisons/ImagesFromVMD/lj_09_Ni_AFTER_DFT.png}
\\
\vtab


\end{center}
\end{multicols}
\begin{center}
\textcolor{NavyBlue}{\Large Different}
\end{center}

 \newpage

\vtab[-3cm]
\begin{center}
{\large ClustersNi \tab Número 145}
\end{center}
\begin{multicols}{2}
\begin{center}
Molecule A \\ 
SC\_08\_Ni
\includegraphics[width=8cm]{../Comparisons/ImagesFromVMD/SC_08_Ni.png}
\\
\vtab

\columnbreak
Molecule B \\ 
lj\_10
\includegraphics[width=8cm]{../Comparisons/ImagesFromVMD/lj_10.png}
\\
\vtab


\end{center}
\end{multicols}
\begin{center}
\textcolor{NavyBlue}{\Large Different}
\end{center}

 \newpage

\vtab[-3cm]
\begin{center}
{\large ClustersNi \tab Número 146}
\end{center}
\begin{multicols}{2}
\begin{center}
Molecule A \\ 
SC\_08\_Ni
\includegraphics[width=8cm]{../Comparisons/ImagesFromVMD/SC_08_Ni.png}
\\
\vtab

\columnbreak
Molecule B \\ 
lj\_11
\includegraphics[width=8cm]{../Comparisons/ImagesFromVMD/lj_11.png}
\\
\vtab


\end{center}
\end{multicols}
\begin{center}
\textcolor{NavyBlue}{\Large Different}
\end{center}

 \newpage

\vtab[-3cm]
\begin{center}
{\large ClustersNi \tab Número 147}
\end{center}
\begin{multicols}{2}
\begin{center}
Molecule A \\ 
SC\_08\_Ni\_AFTER\_DFT
\includegraphics[width=8cm]{../Comparisons/ImagesFromVMD/SC_08_Ni_AFTER_DFT.png}
\\
\vtab

\columnbreak
Molecule B \\ 
SC\_09\_Ni
\includegraphics[width=8cm]{../Comparisons/ImagesFromVMD/SC_09_Ni.png}
\\
\vtab


\end{center}
\end{multicols}
\begin{center}
\textcolor{NavyBlue}{\Large Different}
\end{center}

 \newpage

\vtab[-3cm]
\begin{center}
{\large ClustersNi \tab Número 148}
\end{center}
\begin{multicols}{2}
\begin{center}
Molecule A \\ 
SC\_08\_Ni\_AFTER\_DFT
\includegraphics[width=8cm]{../Comparisons/ImagesFromVMD/SC_08_Ni_AFTER_DFT.png}
\\
\vtab

\columnbreak
Molecule B \\ 
SC\_09\_Ni\_AFTER\_DFT
\includegraphics[width=8cm]{../Comparisons/ImagesFromVMD/SC_09_Ni_AFTER_DFT.png}
\\
\vtab


\end{center}
\end{multicols}
\begin{center}
\textcolor{NavyBlue}{\Large Different}
\end{center}

 \newpage

\vtab[-3cm]
\begin{center}
{\large ClustersNi \tab Número 149}
\end{center}
\begin{multicols}{2}
\begin{center}
Molecule A \\ 
SC\_08\_Ni\_AFTER\_DFT
\includegraphics[width=8cm]{../Comparisons/ImagesFromVMD/SC_08_Ni_AFTER_DFT.png}
\\
\vtab

\columnbreak
Molecule B \\ 
lj\_08
\includegraphics[width=8cm]{../Comparisons/ImagesFromVMD/lj_08.png}
\\
\vtab


\end{center}
\end{multicols}
\begin{center}
\textcolor{NavyBlue}{\Large Different}
\end{center}

 \newpage

\vtab[-3cm]
\begin{center}
{\large ClustersNi \tab Número 150}
\end{center}
\begin{multicols}{2}
\begin{center}
Molecule A \\ 
SC\_08\_Ni\_AFTER\_DFT
\includegraphics[width=8cm]{../Comparisons/ImagesFromVMD/SC_08_Ni_AFTER_DFT.png}
\\
\vtab

\columnbreak
Molecule B \\ 
lj\_08\_Ni
\includegraphics[width=8cm]{../Comparisons/ImagesFromVMD/lj_08_Ni.png}
\\
\vtab


\end{center}
\end{multicols}
\begin{center}
\textcolor{NavyBlue}{\Large Different}
\end{center}

 \newpage

\vtab[-3cm]
\begin{center}
{\large ClustersNi \tab Número 151}
\end{center}
\begin{multicols}{2}
\begin{center}

Molecule A \
SC\_08\_Ni\_AFTER\_DFT

\includegraphics[width=6cm]{../Comparisons/ImagesFromVMD/SC_08_Ni_AFTER_DFT.png}

Inertia Tensor - Molecule A \\
\begin{tabular}{|c c c|}
1202.23	 & 	-22.8163	 & 	-12.5979	 \\
-22.8163	 & 	1197.22	 & 	1.92415	 \\
-12.5979	 & 	1.92415	 & 	1155.53
\end{tabular}

\vtab
 EingenVectors - Molecule A     \\
\begin{tabular}{|c c c|}
0.285716	 & 	0.10352	 & 	0.952707	 \\
0.595734	 & 	0.759527	 & 	-0.261189	 \\
-0.750645	 & 	0.642185	 & 	0.155339
\end{tabular}

\vtab
 EingenValues - Molecule A     \\
\begin{tabular}{|c c c|}
1151.96	 & 	1178.67	 & 	1224.36	 \\
\end{tabular}
\columnbreak

Molecule B \
lj\_08\_Ni\_AFTER\_DFT

\includegraphics[width=6cm]{../Comparisons/ImagesFromVMD/lj_08_Ni_AFTER_DFT.png}

Inertia Tensor - Molecule B \\
\begin{tabular}{|c c c|}
1461.29	 & 	-67.9001	 & 	-380.896	 \\
-67.9001	 & 	1626.58	 & 	-132.788	 \\
-380.896	 & 	-132.788	 & 	885.164
\end{tabular}

\vtab
 EingenVectors - Molecule B     \\
\begin{tabular}{|c c c|}
0.440314	 & 	0.154412	 & 	0.884466	 \\
-0.488448	 & 	-0.785372	 & 	0.380276	 \\
-0.753354	 & 	0.599456	 & 	0.270388
\end{tabular}

\vtab
 EingenValues - Molecule B     \\
\begin{tabular}{|c c c|}
672.361	 & 	1648.65	 & 	1652.02	 \\
\end{tabular}

\end{center}
\end{multicols}

\vtab[-5mm]
\begin{tabular}{*{2}{m{0.38\textwidth}}}
\begin{center}
\textcolor{NavyBlue}{\Large Different}
\end{center}
&
\begin{center}
\includegraphics[height=6.5cm]{../Comparisons/Vectors/inertia_tensor_of_SC_08_Ni_AFTER_DFT_and_lj_08_Ni_AFTER_DFT.png}
\end{center}
\end{tabular}

 \newpage

\vtab[-3cm]
\begin{center}
{\large ClustersNi \tab Número 152}
\end{center}
\begin{multicols}{2}
\begin{center}
Molecule A \\ 
SC\_08\_Ni\_AFTER\_DFT
\includegraphics[width=8cm]{../Comparisons/ImagesFromVMD/SC_08_Ni_AFTER_DFT.png}
\\
\vtab

\columnbreak
Molecule B \\ 
lj\_09
\includegraphics[width=8cm]{../Comparisons/ImagesFromVMD/lj_09.png}
\\
\vtab


\end{center}
\end{multicols}
\begin{center}
\textcolor{NavyBlue}{\Large Different}
\end{center}

 \newpage

\vtab[-3cm]
\begin{center}
{\large ClustersNi \tab Número 153}
\end{center}
\begin{multicols}{2}
\begin{center}
Molecule A \\ 
SC\_08\_Ni\_AFTER\_DFT
\includegraphics[width=8cm]{../Comparisons/ImagesFromVMD/SC_08_Ni_AFTER_DFT.png}
\\
\vtab

\columnbreak
Molecule B \\ 
lj\_09\_Ni
\includegraphics[width=8cm]{../Comparisons/ImagesFromVMD/lj_09_Ni.png}
\\
\vtab


\end{center}
\end{multicols}
\begin{center}
\textcolor{NavyBlue}{\Large Different}
\end{center}

 \newpage

\vtab[-3cm]
\begin{center}
{\large ClustersNi \tab Número 154}
\end{center}
\begin{multicols}{2}
\begin{center}
Molecule A \\ 
SC\_08\_Ni\_AFTER\_DFT
\includegraphics[width=8cm]{../Comparisons/ImagesFromVMD/SC_08_Ni_AFTER_DFT.png}
\\
\vtab

\columnbreak
Molecule B \\ 
lj\_09\_Ni\_AFTER\_DFT
\includegraphics[width=8cm]{../Comparisons/ImagesFromVMD/lj_09_Ni_AFTER_DFT.png}
\\
\vtab


\end{center}
\end{multicols}
\begin{center}
\textcolor{NavyBlue}{\Large Different}
\end{center}

 \newpage

\vtab[-3cm]
\begin{center}
{\large ClustersNi \tab Número 155}
\end{center}
\begin{multicols}{2}
\begin{center}
Molecule A \\ 
SC\_08\_Ni\_AFTER\_DFT
\includegraphics[width=8cm]{../Comparisons/ImagesFromVMD/SC_08_Ni_AFTER_DFT.png}
\\
\vtab

\columnbreak
Molecule B \\ 
lj\_10
\includegraphics[width=8cm]{../Comparisons/ImagesFromVMD/lj_10.png}
\\
\vtab


\end{center}
\end{multicols}
\begin{center}
\textcolor{NavyBlue}{\Large Different}
\end{center}

 \newpage

\vtab[-3cm]
\begin{center}
{\large ClustersNi \tab Número 156}
\end{center}
\begin{multicols}{2}
\begin{center}
Molecule A \\ 
SC\_08\_Ni\_AFTER\_DFT
\includegraphics[width=8cm]{../Comparisons/ImagesFromVMD/SC_08_Ni_AFTER_DFT.png}
\\
\vtab

\columnbreak
Molecule B \\ 
lj\_11
\includegraphics[width=8cm]{../Comparisons/ImagesFromVMD/lj_11.png}
\\
\vtab


\end{center}
\end{multicols}
\begin{center}
\textcolor{NavyBlue}{\Large Different}
\end{center}

 \newpage

\vtab[-3cm]
\begin{center}
{\large ClustersNi \tab Número 157}
\end{center}
\begin{multicols}{2}
\begin{center}

Molecule A \
SC\_09\_Ni

\includegraphics[width=6cm]{../Comparisons/ImagesFromVMD/SC_09_Ni.png}

Inertia Tensor - Molecule A \\
\begin{tabular}{|c c c|}
1536.98	 & 	56.225	 & 	-101.457	 \\
56.225	 & 	1328.77	 & 	-222.686	 \\
-101.457	 & 	-222.686	 & 	1805.96
\end{tabular}

\vtab
 EingenVectors - Molecule A     \\
\begin{tabular}{|c c c|}
0.0545275	 & 	-0.93243	 & 	-0.357212	 \\
-0.957676	 & 	0.0524255	 & 	-0.283033	 \\
-0.282636	 & 	-0.357527	 & 	0.890108
\end{tabular}

\vtab
 EingenValues - Molecule A     \\
\begin{tabular}{|c c c|}
1240.17	 & 	1503.92	 & 	1927.62	 \\
\end{tabular}
\columnbreak

Molecule B \
SC\_09\_Ni\_AFTER\_DFT

\includegraphics[width=6cm]{../Comparisons/ImagesFromVMD/SC_09_Ni_AFTER_DFT.png}

Inertia Tensor - Molecule B \\
\begin{tabular}{|c c c|}
1526.11	 & 	61.3141	 & 	-104.189	 \\
61.3141	 & 	1367.3	 & 	-214.394	 \\
-104.189	 & 	-214.394	 & 	1801.23
\end{tabular}

\vtab
 EingenVectors - Molecule B     \\
\begin{tabular}{|c c c|}
0.0759984	 & 	-0.927891	 & 	-0.365024	 \\
-0.954757	 & 	0.0378395	 & 	-0.294969	 \\
-0.287511	 & 	-0.370927	 & 	0.883035
\end{tabular}

\vtab
 EingenValues - Molecule B     \\
\begin{tabular}{|c c c|}
1277.93	 & 	1491.49	 & 	1925.21	 \\
\end{tabular}

\end{center}
\end{multicols}

\vtab[-5mm]
\begin{tabular}{*{2}{m{0.38\textwidth}}}
\begin{center}
\textcolor{NavyBlue}{\Large Different}
\end{center}
&
\begin{center}
\includegraphics[height=6.5cm]{../Comparisons/Vectors/inertia_tensor_of_SC_09_Ni_and_SC_09_Ni_AFTER_DFT.png}
\end{center}
\end{tabular}

 \newpage

\vtab[-3cm]
\begin{center}
{\large ClustersNi \tab Número 158}
\end{center}
\begin{multicols}{2}
\begin{center}
Molecule A \\ 
SC\_09\_Ni
\includegraphics[width=8cm]{../Comparisons/ImagesFromVMD/SC_09_Ni.png}
\\
\vtab

\columnbreak
Molecule B \\ 
lj\_08
\includegraphics[width=8cm]{../Comparisons/ImagesFromVMD/lj_08.png}
\\
\vtab


\end{center}
\end{multicols}
\begin{center}
\textcolor{NavyBlue}{\Large Different}
\end{center}

 \newpage

\vtab[-3cm]
\begin{center}
{\large ClustersNi \tab Número 159}
\end{center}
\begin{multicols}{2}
\begin{center}
Molecule A \\ 
SC\_09\_Ni
\includegraphics[width=8cm]{../Comparisons/ImagesFromVMD/SC_09_Ni.png}
\\
\vtab

\columnbreak
Molecule B \\ 
lj\_08\_Ni
\includegraphics[width=8cm]{../Comparisons/ImagesFromVMD/lj_08_Ni.png}
\\
\vtab


\end{center}
\end{multicols}
\begin{center}
\textcolor{NavyBlue}{\Large Different}
\end{center}

 \newpage

\vtab[-3cm]
\begin{center}
{\large ClustersNi \tab Número 160}
\end{center}
\begin{multicols}{2}
\begin{center}
Molecule A \\ 
SC\_09\_Ni
\includegraphics[width=8cm]{../Comparisons/ImagesFromVMD/SC_09_Ni.png}
\\
\vtab

\columnbreak
Molecule B \\ 
lj\_08\_Ni\_AFTER\_DFT
\includegraphics[width=8cm]{../Comparisons/ImagesFromVMD/lj_08_Ni_AFTER_DFT.png}
\\
\vtab


\end{center}
\end{multicols}
\begin{center}
\textcolor{NavyBlue}{\Large Different}
\end{center}

 \newpage

\vtab[-3cm]
\begin{center}
{\large ClustersNi \tab Número 161}
\end{center}
\begin{multicols}{2}
\begin{center}
Molecule A \\ 
SC\_09\_Ni
\includegraphics[width=8cm]{../Comparisons/ImagesFromVMD/SC_09_Ni.png}
\\
\vtab

\columnbreak
Molecule B \\ 
lj\_09
\includegraphics[width=8cm]{../Comparisons/ImagesFromVMD/lj_09.png}
\\
\vtab


\end{center}
\end{multicols}
\begin{center}
\textcolor{NavyBlue}{\Large Different}
\end{center}

 \newpage

\vtab[-3cm]
\begin{center}
{\large ClustersNi \tab Número 162}
\end{center}
\begin{multicols}{2}
\begin{center}

Molecule A \
SC\_09\_Ni

\includegraphics[width=6cm]{../Comparisons/ImagesFromVMD/SC_09_Ni.png}

Inertia Tensor - Molecule A \\
\begin{tabular}{|c c c|}
1536.98	 & 	56.225	 & 	-101.457	 \\
56.225	 & 	1328.77	 & 	-222.686	 \\
-101.457	 & 	-222.686	 & 	1805.96
\end{tabular}

\vtab
 EingenVectors - Molecule A     \\
\begin{tabular}{|c c c|}
0.0545275	 & 	-0.93243	 & 	-0.357212	 \\
-0.957676	 & 	0.0524255	 & 	-0.283033	 \\
-0.282636	 & 	-0.357527	 & 	0.890108
\end{tabular}

\vtab
 EingenValues - Molecule A     \\
\begin{tabular}{|c c c|}
1240.17	 & 	1503.92	 & 	1927.62	 \\
\end{tabular}
\columnbreak

Molecule B \
lj\_09\_Ni

\includegraphics[width=6cm]{../Comparisons/ImagesFromVMD/lj_09_Ni.png}

Inertia Tensor - Molecule B \\
\begin{tabular}{|c c c|}
2099.87	 & 	-92.0868	 & 	87.5491	 \\
-92.0868	 & 	1185.59	 & 	628.538	 \\
87.5491	 & 	628.538	 & 	1533.16
\end{tabular}

\vtab
 EingenVectors - Molecule B     \\
\begin{tabular}{|c c c|}
-0.0896058	 & 	-0.791993	 & 	0.603919	 \\
0.134366	 & 	-0.610428	 & 	-0.780592	 \\
-0.986872	 & 	-0.0112006	 & 	-0.161115
\end{tabular}

\vtab
 EingenValues - Molecule B     \\
\begin{tabular}{|c c c|}
695.893	 & 	2009.61	 & 	2113.12	 \\
\end{tabular}

\end{center}
\end{multicols}

\vtab[-5mm]
\begin{tabular}{*{2}{m{0.38\textwidth}}}
\begin{center}
\textcolor{NavyBlue}{\Large Different}
\end{center}
&
\begin{center}
\includegraphics[height=6.5cm]{../Comparisons/Vectors/inertia_tensor_of_SC_09_Ni_and_lj_09_Ni.png}
\end{center}
\end{tabular}

 \newpage

\vtab[-3cm]
\begin{center}
{\large ClustersNi \tab Número 163}
\end{center}
\begin{multicols}{2}
\begin{center}

Molecule A \
SC\_09\_Ni

\includegraphics[width=6cm]{../Comparisons/ImagesFromVMD/SC_09_Ni.png}

Inertia Tensor - Molecule A \\
\begin{tabular}{|c c c|}
1536.98	 & 	56.225	 & 	-101.457	 \\
56.225	 & 	1328.77	 & 	-222.686	 \\
-101.457	 & 	-222.686	 & 	1805.96
\end{tabular}

\vtab
 EingenVectors - Molecule A     \\
\begin{tabular}{|c c c|}
0.0545275	 & 	-0.93243	 & 	-0.357212	 \\
-0.957676	 & 	0.0524255	 & 	-0.283033	 \\
-0.282636	 & 	-0.357527	 & 	0.890108
\end{tabular}

\vtab
 EingenValues - Molecule A     \\
\begin{tabular}{|c c c|}
1240.17	 & 	1503.92	 & 	1927.62	 \\
\end{tabular}
\columnbreak

Molecule B \
lj\_09\_Ni\_AFTER\_DFT

\includegraphics[width=6cm]{../Comparisons/ImagesFromVMD/lj_09_Ni_AFTER_DFT.png}

Inertia Tensor - Molecule B \\
\begin{tabular}{|c c c|}
2192.28	 & 	-89.3917	 & 	85.1517	 \\
-89.3917	 & 	1303.55	 & 	619.274	 \\
85.1517	 & 	619.274	 & 	1643.51
\end{tabular}

\vtab
 EingenVectors - Molecule B     \\
\begin{tabular}{|c c c|}
-0.0891023	 & 	-0.791474	 & 	0.604673	 \\
0.147893	 & 	-0.610872	 & 	-0.777794	 \\
-0.984981	 & 	-0.0201236	 & 	-0.171483
\end{tabular}

\vtab
 EingenValues - Molecule B     \\
\begin{tabular}{|c c c|}
820.373	 & 	2113.69	 & 	2205.28	 \\
\end{tabular}

\end{center}
\end{multicols}

\vtab[-5mm]
\begin{tabular}{*{2}{m{0.38\textwidth}}}
\begin{center}
\textcolor{NavyBlue}{\Large Different}
\end{center}
&
\begin{center}
\includegraphics[height=6.5cm]{../Comparisons/Vectors/inertia_tensor_of_SC_09_Ni_and_lj_09_Ni_AFTER_DFT.png}
\end{center}
\end{tabular}

 \newpage

\vtab[-3cm]
\begin{center}
{\large ClustersNi \tab Número 164}
\end{center}
\begin{multicols}{2}
\begin{center}
Molecule A \\ 
SC\_09\_Ni
\includegraphics[width=8cm]{../Comparisons/ImagesFromVMD/SC_09_Ni.png}
\\
\vtab

\columnbreak
Molecule B \\ 
lj\_10
\includegraphics[width=8cm]{../Comparisons/ImagesFromVMD/lj_10.png}
\\
\vtab


\end{center}
\end{multicols}
\begin{center}
\textcolor{NavyBlue}{\Large Different}
\end{center}

 \newpage

\vtab[-3cm]
\begin{center}
{\large ClustersNi \tab Número 165}
\end{center}
\begin{multicols}{2}
\begin{center}
Molecule A \\ 
SC\_09\_Ni
\includegraphics[width=8cm]{../Comparisons/ImagesFromVMD/SC_09_Ni.png}
\\
\vtab

\columnbreak
Molecule B \\ 
lj\_11
\includegraphics[width=8cm]{../Comparisons/ImagesFromVMD/lj_11.png}
\\
\vtab


\end{center}
\end{multicols}
\begin{center}
\textcolor{NavyBlue}{\Large Different}
\end{center}

 \newpage

\vtab[-3cm]
\begin{center}
{\large ClustersNi \tab Número 166}
\end{center}
\begin{multicols}{2}
\begin{center}
Molecule A \\ 
SC\_09\_Ni\_AFTER\_DFT
\includegraphics[width=8cm]{../Comparisons/ImagesFromVMD/SC_09_Ni_AFTER_DFT.png}
\\
\vtab

\columnbreak
Molecule B \\ 
lj\_08
\includegraphics[width=8cm]{../Comparisons/ImagesFromVMD/lj_08.png}
\\
\vtab


\end{center}
\end{multicols}
\begin{center}
\textcolor{NavyBlue}{\Large Different}
\end{center}

 \newpage

\vtab[-3cm]
\begin{center}
{\large ClustersNi \tab Número 167}
\end{center}
\begin{multicols}{2}
\begin{center}
Molecule A \\ 
SC\_09\_Ni\_AFTER\_DFT
\includegraphics[width=8cm]{../Comparisons/ImagesFromVMD/SC_09_Ni_AFTER_DFT.png}
\\
\vtab

\columnbreak
Molecule B \\ 
lj\_08\_Ni
\includegraphics[width=8cm]{../Comparisons/ImagesFromVMD/lj_08_Ni.png}
\\
\vtab


\end{center}
\end{multicols}
\begin{center}
\textcolor{NavyBlue}{\Large Different}
\end{center}

 \newpage

\vtab[-3cm]
\begin{center}
{\large ClustersNi \tab Número 168}
\end{center}
\begin{multicols}{2}
\begin{center}
Molecule A \\ 
SC\_09\_Ni\_AFTER\_DFT
\includegraphics[width=8cm]{../Comparisons/ImagesFromVMD/SC_09_Ni_AFTER_DFT.png}
\\
\vtab

\columnbreak
Molecule B \\ 
lj\_08\_Ni\_AFTER\_DFT
\includegraphics[width=8cm]{../Comparisons/ImagesFromVMD/lj_08_Ni_AFTER_DFT.png}
\\
\vtab


\end{center}
\end{multicols}
\begin{center}
\textcolor{NavyBlue}{\Large Different}
\end{center}

 \newpage

\vtab[-3cm]
\begin{center}
{\large ClustersNi \tab Número 169}
\end{center}
\begin{multicols}{2}
\begin{center}
Molecule A \\ 
SC\_09\_Ni\_AFTER\_DFT
\includegraphics[width=8cm]{../Comparisons/ImagesFromVMD/SC_09_Ni_AFTER_DFT.png}
\\
\vtab

\columnbreak
Molecule B \\ 
lj\_09
\includegraphics[width=8cm]{../Comparisons/ImagesFromVMD/lj_09.png}
\\
\vtab


\end{center}
\end{multicols}
\begin{center}
\textcolor{NavyBlue}{\Large Different}
\end{center}

 \newpage

\vtab[-3cm]
\begin{center}
{\large ClustersNi \tab Número 170}
\end{center}
\begin{multicols}{2}
\begin{center}

Molecule A \
SC\_09\_Ni\_AFTER\_DFT

\includegraphics[width=6cm]{../Comparisons/ImagesFromVMD/SC_09_Ni_AFTER_DFT.png}

Inertia Tensor - Molecule A \\
\begin{tabular}{|c c c|}
1526.11	 & 	61.3141	 & 	-104.189	 \\
61.3141	 & 	1367.3	 & 	-214.394	 \\
-104.189	 & 	-214.394	 & 	1801.23
\end{tabular}

\vtab
 EingenVectors - Molecule A     \\
\begin{tabular}{|c c c|}
0.0759984	 & 	-0.927891	 & 	-0.365024	 \\
-0.954757	 & 	0.0378395	 & 	-0.294969	 \\
-0.287511	 & 	-0.370927	 & 	0.883035
\end{tabular}

\vtab
 EingenValues - Molecule A     \\
\begin{tabular}{|c c c|}
1277.93	 & 	1491.49	 & 	1925.21	 \\
\end{tabular}
\columnbreak

Molecule B \
lj\_09\_Ni

\includegraphics[width=6cm]{../Comparisons/ImagesFromVMD/lj_09_Ni.png}

Inertia Tensor - Molecule B \\
\begin{tabular}{|c c c|}
2099.87	 & 	-92.0868	 & 	87.5491	 \\
-92.0868	 & 	1185.59	 & 	628.538	 \\
87.5491	 & 	628.538	 & 	1533.16
\end{tabular}

\vtab
 EingenVectors - Molecule B     \\
\begin{tabular}{|c c c|}
-0.0896058	 & 	-0.791993	 & 	0.603919	 \\
0.134366	 & 	-0.610428	 & 	-0.780592	 \\
-0.986872	 & 	-0.0112006	 & 	-0.161115
\end{tabular}

\vtab
 EingenValues - Molecule B     \\
\begin{tabular}{|c c c|}
695.893	 & 	2009.61	 & 	2113.12	 \\
\end{tabular}

\end{center}
\end{multicols}

\vtab[-5mm]
\begin{tabular}{*{2}{m{0.38\textwidth}}}
\begin{center}
\textcolor{NavyBlue}{\Large Different}
\end{center}
&
\begin{center}
\includegraphics[height=6.5cm]{../Comparisons/Vectors/inertia_tensor_of_SC_09_Ni_AFTER_DFT_and_lj_09_Ni.png}
\end{center}
\end{tabular}

 \newpage

\vtab[-3cm]
\begin{center}
{\large ClustersNi \tab Número 171}
\end{center}
\begin{multicols}{2}
\begin{center}

Molecule A \
SC\_09\_Ni\_AFTER\_DFT

\includegraphics[width=6cm]{../Comparisons/ImagesFromVMD/SC_09_Ni_AFTER_DFT.png}

Inertia Tensor - Molecule A \\
\begin{tabular}{|c c c|}
1526.11	 & 	61.3141	 & 	-104.189	 \\
61.3141	 & 	1367.3	 & 	-214.394	 \\
-104.189	 & 	-214.394	 & 	1801.23
\end{tabular}

\vtab
 EingenVectors - Molecule A     \\
\begin{tabular}{|c c c|}
0.0759984	 & 	-0.927891	 & 	-0.365024	 \\
-0.954757	 & 	0.0378395	 & 	-0.294969	 \\
-0.287511	 & 	-0.370927	 & 	0.883035
\end{tabular}

\vtab
 EingenValues - Molecule A     \\
\begin{tabular}{|c c c|}
1277.93	 & 	1491.49	 & 	1925.21	 \\
\end{tabular}
\columnbreak

Molecule B \
lj\_09\_Ni\_AFTER\_DFT

\includegraphics[width=6cm]{../Comparisons/ImagesFromVMD/lj_09_Ni_AFTER_DFT.png}

Inertia Tensor - Molecule B \\
\begin{tabular}{|c c c|}
2192.28	 & 	-89.3917	 & 	85.1517	 \\
-89.3917	 & 	1303.55	 & 	619.274	 \\
85.1517	 & 	619.274	 & 	1643.51
\end{tabular}

\vtab
 EingenVectors - Molecule B     \\
\begin{tabular}{|c c c|}
-0.0891023	 & 	-0.791474	 & 	0.604673	 \\
0.147893	 & 	-0.610872	 & 	-0.777794	 \\
-0.984981	 & 	-0.0201236	 & 	-0.171483
\end{tabular}

\vtab
 EingenValues - Molecule B     \\
\begin{tabular}{|c c c|}
820.373	 & 	2113.69	 & 	2205.28	 \\
\end{tabular}

\end{center}
\end{multicols}

\vtab[-5mm]
\begin{tabular}{*{2}{m{0.38\textwidth}}}
\begin{center}
\textcolor{NavyBlue}{\Large Different}
\end{center}
&
\begin{center}
\includegraphics[height=6.5cm]{../Comparisons/Vectors/inertia_tensor_of_SC_09_Ni_AFTER_DFT_and_lj_09_Ni_AFTER_DFT.png}
\end{center}
\end{tabular}

 \newpage

\vtab[-3cm]
\begin{center}
{\large ClustersNi \tab Número 172}
\end{center}
\begin{multicols}{2}
\begin{center}
Molecule A \\ 
SC\_09\_Ni\_AFTER\_DFT
\includegraphics[width=8cm]{../Comparisons/ImagesFromVMD/SC_09_Ni_AFTER_DFT.png}
\\
\vtab

\columnbreak
Molecule B \\ 
lj\_10
\includegraphics[width=8cm]{../Comparisons/ImagesFromVMD/lj_10.png}
\\
\vtab


\end{center}
\end{multicols}
\begin{center}
\textcolor{NavyBlue}{\Large Different}
\end{center}

 \newpage

\vtab[-3cm]
\begin{center}
{\large ClustersNi \tab Número 173}
\end{center}
\begin{multicols}{2}
\begin{center}
Molecule A \\ 
SC\_09\_Ni\_AFTER\_DFT
\includegraphics[width=8cm]{../Comparisons/ImagesFromVMD/SC_09_Ni_AFTER_DFT.png}
\\
\vtab

\columnbreak
Molecule B \\ 
lj\_11
\includegraphics[width=8cm]{../Comparisons/ImagesFromVMD/lj_11.png}
\\
\vtab


\end{center}
\end{multicols}
\begin{center}
\textcolor{NavyBlue}{\Large Different}
\end{center}

 \newpage

\vtab[-3cm]
\begin{center}
{\large ClustersNi \tab Número 174}
\end{center}
\begin{multicols}{2}
\begin{center}

Molecule A \
lj\_08

\includegraphics[width=6cm]{../Comparisons/ImagesFromVMD/lj_08.png}

Inertia Tensor - Molecule A \\
\begin{tabular}{|c c c|}
261.999	 & 	-18.7437	 & 	-6.34405	 \\
-18.7437	 & 	120.16	 & 	-48.8007	 \\
-6.34405	 & 	-48.8007	 & 	248.357
\end{tabular}

\vtab
 EingenVectors - Molecule A     \\
\begin{tabular}{|c c c|}
-0.122187	 & 	-0.940431	 & 	-0.31727	 \\
-0.991225	 & 	0.131869	 & 	-0.00913745	 \\
0.0504311	 & 	0.313369	 & 	-0.948291
\end{tabular}

\vtab
 EingenValues - Molecule A     \\
\begin{tabular}{|c c c|}
101.261	 & 	264.434	 & 	264.82	 \\
\end{tabular}
\columnbreak

Molecule B \
lj\_08\_Ni

\includegraphics[width=6cm]{../Comparisons/ImagesFromVMD/lj_08_Ni.png}

Inertia Tensor - Molecule B \\
\begin{tabular}{|c c c|}
938.857	 & 	-52.894	 & 	-254.175	 \\
-52.894	 & 	1044.27	 & 	-104.28	 \\
-254.175	 & 	-104.28	 & 	556.076
\end{tabular}

\vtab
 EingenVectors - Molecule B     \\
\begin{tabular}{|c c c|}
0.439206	 & 	0.180689	 & 	0.880028	 \\
-0.598991	 & 	-0.671163	 & 	0.43675	 \\
-0.669558	 & 	0.718952	 & 	0.186547
\end{tabular}

\vtab
 EingenValues - Molecule B     \\
\begin{tabular}{|c c c|}
407.811	 & 	1064.92	 & 	1066.47	 \\
\end{tabular}

\end{center}
\end{multicols}

\vtab[-5mm]
\begin{tabular}{*{2}{m{0.38\textwidth}}}
\begin{center}
\textcolor{NavyBlue}{\Large Different}
\end{center}
&
\begin{center}
\includegraphics[height=6.5cm]{../Comparisons/Vectors/inertia_tensor_of_lj_08_and_lj_08_Ni.png}
\end{center}
\end{tabular}

 \newpage

\vtab[-3cm]
\begin{center}
{\large ClustersNi \tab Número 175}
\end{center}
\begin{multicols}{2}
\begin{center}
Molecule A \\ 
lj\_08
\includegraphics[width=8cm]{../Comparisons/ImagesFromVMD/lj_08.png}
\\
\vtab

\columnbreak
Molecule B \\ 
lj\_08\_Ni\_AFTER\_DFT
\includegraphics[width=8cm]{../Comparisons/ImagesFromVMD/lj_08_Ni_AFTER_DFT.png}
\\
\vtab


\end{center}
\end{multicols}
\begin{center}
\textcolor{NavyBlue}{\Large Different}
\end{center}

 \newpage

\vtab[-3cm]
\begin{center}
{\large ClustersNi \tab Número 176}
\end{center}
\begin{multicols}{2}
\begin{center}
Molecule A \\ 
lj\_08
\includegraphics[width=8cm]{../Comparisons/ImagesFromVMD/lj_08.png}
\\
\vtab

\columnbreak
Molecule B \\ 
lj\_09
\includegraphics[width=8cm]{../Comparisons/ImagesFromVMD/lj_09.png}
\\
\vtab


\end{center}
\end{multicols}
\begin{center}
\textcolor{NavyBlue}{\Large Different}
\end{center}

 \newpage

\vtab[-3cm]
\begin{center}
{\large ClustersNi \tab Número 177}
\end{center}
\begin{multicols}{2}
\begin{center}
Molecule A \\ 
lj\_08
\includegraphics[width=8cm]{../Comparisons/ImagesFromVMD/lj_08.png}
\\
\vtab

\columnbreak
Molecule B \\ 
lj\_09\_Ni
\includegraphics[width=8cm]{../Comparisons/ImagesFromVMD/lj_09_Ni.png}
\\
\vtab


\end{center}
\end{multicols}
\begin{center}
\textcolor{NavyBlue}{\Large Different}
\end{center}

 \newpage

\vtab[-3cm]
\begin{center}
{\large ClustersNi \tab Número 178}
\end{center}
\begin{multicols}{2}
\begin{center}
Molecule A \\ 
lj\_08
\includegraphics[width=8cm]{../Comparisons/ImagesFromVMD/lj_08.png}
\\
\vtab

\columnbreak
Molecule B \\ 
lj\_09\_Ni\_AFTER\_DFT
\includegraphics[width=8cm]{../Comparisons/ImagesFromVMD/lj_09_Ni_AFTER_DFT.png}
\\
\vtab


\end{center}
\end{multicols}
\begin{center}
\textcolor{NavyBlue}{\Large Different}
\end{center}

 \newpage

\vtab[-3cm]
\begin{center}
{\large ClustersNi \tab Número 179}
\end{center}
\begin{multicols}{2}
\begin{center}
Molecule A \\ 
lj\_08
\includegraphics[width=8cm]{../Comparisons/ImagesFromVMD/lj_08.png}
\\
\vtab

\columnbreak
Molecule B \\ 
lj\_10
\includegraphics[width=8cm]{../Comparisons/ImagesFromVMD/lj_10.png}
\\
\vtab


\end{center}
\end{multicols}
\begin{center}
\textcolor{NavyBlue}{\Large Different}
\end{center}

 \newpage

\vtab[-3cm]
\begin{center}
{\large ClustersNi \tab Número 180}
\end{center}
\begin{multicols}{2}
\begin{center}
Molecule A \\ 
lj\_08
\includegraphics[width=8cm]{../Comparisons/ImagesFromVMD/lj_08.png}
\\
\vtab

\columnbreak
Molecule B \\ 
lj\_11
\includegraphics[width=8cm]{../Comparisons/ImagesFromVMD/lj_11.png}
\\
\vtab


\end{center}
\end{multicols}
\begin{center}
\textcolor{NavyBlue}{\Large Different}
\end{center}

 \newpage

\vtab[-3cm]
\begin{center}
{\large ClustersNi \tab Número 181}
\end{center}
\begin{multicols}{2}
\begin{center}
Molecule A \\ 
lj\_08\_Ni
\includegraphics[width=8cm]{../Comparisons/ImagesFromVMD/lj_08_Ni.png}
\\
\vtab

\columnbreak
Molecule B \\ 
lj\_08\_Ni\_AFTER\_DFT
\includegraphics[width=8cm]{../Comparisons/ImagesFromVMD/lj_08_Ni_AFTER_DFT.png}
\\
\vtab


\end{center}
\end{multicols}
\begin{center}
\textcolor{NavyBlue}{\Large Different}
\end{center}

 \newpage

\vtab[-3cm]
\begin{center}
{\large ClustersNi \tab Número 182}
\end{center}
\begin{multicols}{2}
\begin{center}
Molecule A \\ 
lj\_08\_Ni
\includegraphics[width=8cm]{../Comparisons/ImagesFromVMD/lj_08_Ni.png}
\\
\vtab

\columnbreak
Molecule B \\ 
lj\_09
\includegraphics[width=8cm]{../Comparisons/ImagesFromVMD/lj_09.png}
\\
\vtab


\end{center}
\end{multicols}
\begin{center}
\textcolor{NavyBlue}{\Large Different}
\end{center}

 \newpage

\vtab[-3cm]
\begin{center}
{\large ClustersNi \tab Número 183}
\end{center}
\begin{multicols}{2}
\begin{center}
Molecule A \\ 
lj\_08\_Ni
\includegraphics[width=8cm]{../Comparisons/ImagesFromVMD/lj_08_Ni.png}
\\
\vtab

\columnbreak
Molecule B \\ 
lj\_09\_Ni
\includegraphics[width=8cm]{../Comparisons/ImagesFromVMD/lj_09_Ni.png}
\\
\vtab


\end{center}
\end{multicols}
\begin{center}
\textcolor{NavyBlue}{\Large Different}
\end{center}

 \newpage

\vtab[-3cm]
\begin{center}
{\large ClustersNi \tab Número 184}
\end{center}
\begin{multicols}{2}
\begin{center}
Molecule A \\ 
lj\_08\_Ni
\includegraphics[width=8cm]{../Comparisons/ImagesFromVMD/lj_08_Ni.png}
\\
\vtab

\columnbreak
Molecule B \\ 
lj\_09\_Ni\_AFTER\_DFT
\includegraphics[width=8cm]{../Comparisons/ImagesFromVMD/lj_09_Ni_AFTER_DFT.png}
\\
\vtab


\end{center}
\end{multicols}
\begin{center}
\textcolor{NavyBlue}{\Large Different}
\end{center}

 \newpage

\vtab[-3cm]
\begin{center}
{\large ClustersNi \tab Número 185}
\end{center}
\begin{multicols}{2}
\begin{center}
Molecule A \\ 
lj\_08\_Ni
\includegraphics[width=8cm]{../Comparisons/ImagesFromVMD/lj_08_Ni.png}
\\
\vtab

\columnbreak
Molecule B \\ 
lj\_10
\includegraphics[width=8cm]{../Comparisons/ImagesFromVMD/lj_10.png}
\\
\vtab


\end{center}
\end{multicols}
\begin{center}
\textcolor{NavyBlue}{\Large Different}
\end{center}

 \newpage

\vtab[-3cm]
\begin{center}
{\large ClustersNi \tab Número 186}
\end{center}
\begin{multicols}{2}
\begin{center}
Molecule A \\ 
lj\_08\_Ni
\includegraphics[width=8cm]{../Comparisons/ImagesFromVMD/lj_08_Ni.png}
\\
\vtab

\columnbreak
Molecule B \\ 
lj\_11
\includegraphics[width=8cm]{../Comparisons/ImagesFromVMD/lj_11.png}
\\
\vtab


\end{center}
\end{multicols}
\begin{center}
\textcolor{NavyBlue}{\Large Different}
\end{center}

 \newpage

\vtab[-3cm]
\begin{center}
{\large ClustersNi \tab Número 187}
\end{center}
\begin{multicols}{2}
\begin{center}
Molecule A \\ 
lj\_08\_Ni\_AFTER\_DFT
\includegraphics[width=8cm]{../Comparisons/ImagesFromVMD/lj_08_Ni_AFTER_DFT.png}
\\
\vtab

\columnbreak
Molecule B \\ 
lj\_09
\includegraphics[width=8cm]{../Comparisons/ImagesFromVMD/lj_09.png}
\\
\vtab


\end{center}
\end{multicols}
\begin{center}
\textcolor{NavyBlue}{\Large Different}
\end{center}

 \newpage

\vtab[-3cm]
\begin{center}
{\large ClustersNi \tab Número 188}
\end{center}
\begin{multicols}{2}
\begin{center}
Molecule A \\ 
lj\_08\_Ni\_AFTER\_DFT
\includegraphics[width=8cm]{../Comparisons/ImagesFromVMD/lj_08_Ni_AFTER_DFT.png}
\\
\vtab

\columnbreak
Molecule B \\ 
lj\_09\_Ni
\includegraphics[width=8cm]{../Comparisons/ImagesFromVMD/lj_09_Ni.png}
\\
\vtab


\end{center}
\end{multicols}
\begin{center}
\textcolor{NavyBlue}{\Large Different}
\end{center}

 \newpage

\vtab[-3cm]
\begin{center}
{\large ClustersNi \tab Número 189}
\end{center}
\begin{multicols}{2}
\begin{center}
Molecule A \\ 
lj\_08\_Ni\_AFTER\_DFT
\includegraphics[width=8cm]{../Comparisons/ImagesFromVMD/lj_08_Ni_AFTER_DFT.png}
\\
\vtab

\columnbreak
Molecule B \\ 
lj\_09\_Ni\_AFTER\_DFT
\includegraphics[width=8cm]{../Comparisons/ImagesFromVMD/lj_09_Ni_AFTER_DFT.png}
\\
\vtab


\end{center}
\end{multicols}
\begin{center}
\textcolor{NavyBlue}{\Large Different}
\end{center}

 \newpage

\vtab[-3cm]
\begin{center}
{\large ClustersNi \tab Número 190}
\end{center}
\begin{multicols}{2}
\begin{center}
Molecule A \\ 
lj\_08\_Ni\_AFTER\_DFT
\includegraphics[width=8cm]{../Comparisons/ImagesFromVMD/lj_08_Ni_AFTER_DFT.png}
\\
\vtab

\columnbreak
Molecule B \\ 
lj\_10
\includegraphics[width=8cm]{../Comparisons/ImagesFromVMD/lj_10.png}
\\
\vtab


\end{center}
\end{multicols}
\begin{center}
\textcolor{NavyBlue}{\Large Different}
\end{center}

 \newpage

\vtab[-3cm]
\begin{center}
{\large ClustersNi \tab Número 191}
\end{center}
\begin{multicols}{2}
\begin{center}
Molecule A \\ 
lj\_08\_Ni\_AFTER\_DFT
\includegraphics[width=8cm]{../Comparisons/ImagesFromVMD/lj_08_Ni_AFTER_DFT.png}
\\
\vtab

\columnbreak
Molecule B \\ 
lj\_11
\includegraphics[width=8cm]{../Comparisons/ImagesFromVMD/lj_11.png}
\\
\vtab


\end{center}
\end{multicols}
\begin{center}
\textcolor{NavyBlue}{\Large Different}
\end{center}

 \newpage

\vtab[-3cm]
\begin{center}
{\large ClustersNi \tab Número 192}
\end{center}
\begin{multicols}{2}
\begin{center}
Molecule A \\ 
lj\_09
\includegraphics[width=8cm]{../Comparisons/ImagesFromVMD/lj_09.png}
\\
\vtab

\columnbreak
Molecule B \\ 
lj\_09\_Ni
\includegraphics[width=8cm]{../Comparisons/ImagesFromVMD/lj_09_Ni.png}
\\
\vtab


\end{center}
\end{multicols}
\begin{center}
\textcolor{NavyBlue}{\Large Different}
\end{center}

 \newpage

\vtab[-3cm]
\begin{center}
{\large ClustersNi \tab Número 193}
\end{center}
\begin{multicols}{2}
\begin{center}
Molecule A \\ 
lj\_09
\includegraphics[width=8cm]{../Comparisons/ImagesFromVMD/lj_09.png}
\\
\vtab

\columnbreak
Molecule B \\ 
lj\_09\_Ni\_AFTER\_DFT
\includegraphics[width=8cm]{../Comparisons/ImagesFromVMD/lj_09_Ni_AFTER_DFT.png}
\\
\vtab


\end{center}
\end{multicols}
\begin{center}
\textcolor{NavyBlue}{\Large Different}
\end{center}

 \newpage

\vtab[-3cm]
\begin{center}
{\large ClustersNi \tab Número 194}
\end{center}
\begin{multicols}{2}
\begin{center}
Molecule A \\ 
lj\_09
\includegraphics[width=8cm]{../Comparisons/ImagesFromVMD/lj_09.png}
\\
\vtab

\columnbreak
Molecule B \\ 
lj\_10
\includegraphics[width=8cm]{../Comparisons/ImagesFromVMD/lj_10.png}
\\
\vtab


\end{center}
\end{multicols}
\begin{center}
\textcolor{NavyBlue}{\Large Different}
\end{center}

 \newpage

\vtab[-3cm]
\begin{center}
{\large ClustersNi \tab Número 195}
\end{center}
\begin{multicols}{2}
\begin{center}
Molecule A \\ 
lj\_09
\includegraphics[width=8cm]{../Comparisons/ImagesFromVMD/lj_09.png}
\\
\vtab

\columnbreak
Molecule B \\ 
lj\_11
\includegraphics[width=8cm]{../Comparisons/ImagesFromVMD/lj_11.png}
\\
\vtab


\end{center}
\end{multicols}
\begin{center}
\textcolor{NavyBlue}{\Large Different}
\end{center}

 \newpage

\vtab[-3cm]
\begin{center}
{\large ClustersNi \tab Número 196}
\end{center}
\begin{multicols}{2}
\begin{center}

Molecule A \
lj\_09\_Ni

\includegraphics[width=6cm]{../Comparisons/ImagesFromVMD/lj_09_Ni.png}

Inertia Tensor - Molecule A \\
\begin{tabular}{|c c c|}
2099.87	 & 	-92.0868	 & 	87.5491	 \\
-92.0868	 & 	1185.59	 & 	628.538	 \\
87.5491	 & 	628.538	 & 	1533.16
\end{tabular}

\vtab
 EingenVectors - Molecule A     \\
\begin{tabular}{|c c c|}
-0.0896058	 & 	-0.791993	 & 	0.603919	 \\
0.134366	 & 	-0.610428	 & 	-0.780592	 \\
-0.986872	 & 	-0.0112006	 & 	-0.161115
\end{tabular}

\vtab
 EingenValues - Molecule A     \\
\begin{tabular}{|c c c|}
695.893	 & 	2009.61	 & 	2113.12	 \\
\end{tabular}
\columnbreak

Molecule B \
lj\_09\_Ni\_AFTER\_DFT

\includegraphics[width=6cm]{../Comparisons/ImagesFromVMD/lj_09_Ni_AFTER_DFT.png}

Inertia Tensor - Molecule B \\
\begin{tabular}{|c c c|}
2192.28	 & 	-89.3917	 & 	85.1517	 \\
-89.3917	 & 	1303.55	 & 	619.274	 \\
85.1517	 & 	619.274	 & 	1643.51
\end{tabular}

\vtab
 EingenVectors - Molecule B     \\
\begin{tabular}{|c c c|}
-0.0891023	 & 	-0.791474	 & 	0.604673	 \\
0.147893	 & 	-0.610872	 & 	-0.777794	 \\
-0.984981	 & 	-0.0201236	 & 	-0.171483
\end{tabular}

\vtab
 EingenValues - Molecule B     \\
\begin{tabular}{|c c c|}
820.373	 & 	2113.69	 & 	2205.28	 \\
\end{tabular}

\end{center}
\end{multicols}

\vtab[-5mm]
\begin{tabular}{*{2}{m{0.38\textwidth}}}
\begin{center}
\textcolor{NavyBlue}{\Large Different}
\end{center}
&
\begin{center}
\includegraphics[height=6.5cm]{../Comparisons/Vectors/inertia_tensor_of_lj_09_Ni_and_lj_09_Ni_AFTER_DFT.png}
\end{center}
\end{tabular}

 \newpage

\vtab[-3cm]
\begin{center}
{\large ClustersNi \tab Número 197}
\end{center}
\begin{multicols}{2}
\begin{center}
Molecule A \\ 
lj\_09\_Ni
\includegraphics[width=8cm]{../Comparisons/ImagesFromVMD/lj_09_Ni.png}
\\
\vtab

\columnbreak
Molecule B \\ 
lj\_10
\includegraphics[width=8cm]{../Comparisons/ImagesFromVMD/lj_10.png}
\\
\vtab


\end{center}
\end{multicols}
\begin{center}
\textcolor{NavyBlue}{\Large Different}
\end{center}

 \newpage

\vtab[-3cm]
\begin{center}
{\large ClustersNi \tab Número 198}
\end{center}
\begin{multicols}{2}
\begin{center}
Molecule A \\ 
lj\_09\_Ni
\includegraphics[width=8cm]{../Comparisons/ImagesFromVMD/lj_09_Ni.png}
\\
\vtab

\columnbreak
Molecule B \\ 
lj\_11
\includegraphics[width=8cm]{../Comparisons/ImagesFromVMD/lj_11.png}
\\
\vtab


\end{center}
\end{multicols}
\begin{center}
\textcolor{NavyBlue}{\Large Different}
\end{center}

 \newpage

\vtab[-3cm]
\begin{center}
{\large ClustersNi \tab Número 199}
\end{center}
\begin{multicols}{2}
\begin{center}
Molecule A \\ 
lj\_09\_Ni\_AFTER\_DFT
\includegraphics[width=8cm]{../Comparisons/ImagesFromVMD/lj_09_Ni_AFTER_DFT.png}
\\
\vtab

\columnbreak
Molecule B \\ 
lj\_10
\includegraphics[width=8cm]{../Comparisons/ImagesFromVMD/lj_10.png}
\\
\vtab


\end{center}
\end{multicols}
\begin{center}
\textcolor{NavyBlue}{\Large Different}
\end{center}

 \newpage

\vtab[-3cm]
\begin{center}
{\large ClustersNi \tab Número 200}
\end{center}
\begin{multicols}{2}
\begin{center}
Molecule A \\ 
lj\_09\_Ni\_AFTER\_DFT
\includegraphics[width=8cm]{../Comparisons/ImagesFromVMD/lj_09_Ni_AFTER_DFT.png}
\\
\vtab

\columnbreak
Molecule B \\ 
lj\_11
\includegraphics[width=8cm]{../Comparisons/ImagesFromVMD/lj_11.png}
\\
\vtab


\end{center}
\end{multicols}
\begin{center}
\textcolor{NavyBlue}{\Large Different}
\end{center}

 \newpage

\vtab[-3cm]
\begin{center}
{\large ClustersNi \tab Número 201}
\end{center}
\begin{multicols}{2}
\begin{center}
Molecule A \\ 
lj\_10
\includegraphics[width=8cm]{../Comparisons/ImagesFromVMD/lj_10.png}
\\
\vtab

\columnbreak
Molecule B \\ 
lj\_11
\includegraphics[width=8cm]{../Comparisons/ImagesFromVMD/lj_11.png}
\\
\vtab


\end{center}
\end{multicols}
\begin{center}
\textcolor{NavyBlue}{\Large Different}
\end{center}

 \newpage

\vtab[-3cm]
\begin{center}
{\large FireTest \tab Número 202}
\end{center}
\begin{multicols}{2}
\begin{center}

Molecule A \
3NFAACa

\includegraphics[width=6cm]{../Comparisons/ImagesFromVMD/3NFAACa.png}

Inertia Tensor - Molecule A \\
\begin{tabular}{|c c c|}
530.125	 & 	-0.358211	 & 	-0.0456723	 \\
-0.358211	 & 	915.82	 & 	2.40736	 \\
-0.0456723	 & 	2.40736	 & 	1361.13
\end{tabular}

\vtab
 EingenVectors - Molecule A     \\
\begin{tabular}{|c c c|}
-1	 & 	-0.000928415	 & 	-5.22706e-05	 \\
0.000928119	 & 	-0.999985	 & 	0.00540587	 \\
-5.72887e-05	 & 	0.00540582	 & 	0.999985
\end{tabular}

\vtab
 EingenValues - Molecule A     \\
\begin{tabular}{|c c c|}
530.124	 & 	915.807	 & 	1361.14	 \\
\end{tabular}
\columnbreak

Molecule B \
3NFAACb

\includegraphics[width=6cm]{../Comparisons/ImagesFromVMD/3NFAACb.png}

Inertia Tensor - Molecule B \\
\begin{tabular}{|c c c|}
530.265	 & 	-1.20883	 & 	-7.84653	 \\
-1.20883	 & 	1361.05	 & 	-3.78865	 \\
-7.84653	 & 	-3.78865	 & 	915.61
\end{tabular}

\vtab
 EingenVectors - Molecule B     \\
\begin{tabular}{|c c c|}
-0.999791	 & 	-0.00154731	 & 	-0.0203648	 \\
-0.0203771	 & 	0.00845054	 & 	0.999757	 \\
-0.00137484	 & 	0.999963	 & 	-0.00848031
\end{tabular}

\vtab
 EingenValues - Molecule B     \\
\begin{tabular}{|c c c|}
530.103	 & 	915.738	 & 	1361.08	 \\
\end{tabular}

\end{center}
\end{multicols}

\vtab[-5mm]
\begin{tabular}{*{2}{m{0.38\textwidth}}}
\begin{center}
\textcolor{NavyBlue}{\Large Equal}
\end{center}
&
\begin{center}
\includegraphics[height=6.5cm]{../Comparisons/Vectors/inertia_tensor_of_3NFAACa_and_3NFAACb.png}
\end{center}
\end{tabular}

 \newpage

\vtab[-3cm]
\begin{center}
{\large FireTest \tab Número 203}
\end{center}
\begin{multicols}{2}
\begin{center}

Molecule A \
3NFAACa

\includegraphics[width=6cm]{../Comparisons/ImagesFromVMD/3NFAACa.png}

Inertia Tensor - Molecule A \\
\begin{tabular}{|c c c|}
530.125	 & 	-0.358211	 & 	-0.0456723	 \\
-0.358211	 & 	915.82	 & 	2.40736	 \\
-0.0456723	 & 	2.40736	 & 	1361.13
\end{tabular}

\vtab
 EingenVectors - Molecule A     \\
\begin{tabular}{|c c c|}
-1	 & 	-0.000928415	 & 	-5.22706e-05	 \\
0.000928119	 & 	-0.999985	 & 	0.00540587	 \\
-5.72887e-05	 & 	0.00540582	 & 	0.999985
\end{tabular}

\vtab
 EingenValues - Molecule A     \\
\begin{tabular}{|c c c|}
530.124	 & 	915.807	 & 	1361.14	 \\
\end{tabular}
\columnbreak

Molecule B \
3NFAACc

\includegraphics[width=6cm]{../Comparisons/ImagesFromVMD/3NFAACc.png}

Inertia Tensor - Molecule B \\
\begin{tabular}{|c c c|}
531.482	 & 	-5.48557	 & 	-0.978638	 \\
-5.48557	 & 	913.233	 & 	2.89201	 \\
-0.978638	 & 	2.89201	 & 	1353.14
\end{tabular}

\vtab
 EingenVectors - Molecule B     \\
\begin{tabular}{|c c c|}
-0.999896	 & 	-0.0143563	 & 	-0.00114029	 \\
0.0143485	 & 	-0.999875	 & 	0.00660611	 \\
-0.00123498	 & 	0.00658906	 & 	0.999978
\end{tabular}

\vtab
 EingenValues - Molecule B     \\
\begin{tabular}{|c c c|}
531.402	 & 	913.293	 & 	1353.16	 \\
\end{tabular}

\end{center}
\end{multicols}

\vtab[-5mm]
\begin{tabular}{*{2}{m{0.38\textwidth}}}
\begin{center}
\textcolor{NavyBlue}{\Large Different}
\end{center}
&
\begin{center}
\includegraphics[height=6.5cm]{../Comparisons/Vectors/inertia_tensor_of_3NFAACa_and_3NFAACc.png}
\end{center}
\end{tabular}

 \newpage

\vtab[-3cm]
\begin{center}
{\large FireTest \tab Número 204}
\end{center}
\begin{multicols}{2}
\begin{center}

Molecule A \
3NFAACa

\includegraphics[width=6cm]{../Comparisons/ImagesFromVMD/3NFAACa.png}

Inertia Tensor - Molecule A \\
\begin{tabular}{|c c c|}
530.125	 & 	-0.358211	 & 	-0.0456723	 \\
-0.358211	 & 	915.82	 & 	2.40736	 \\
-0.0456723	 & 	2.40736	 & 	1361.13
\end{tabular}

\vtab
 EingenVectors - Molecule A     \\
\begin{tabular}{|c c c|}
-1	 & 	-0.000928415	 & 	-5.22706e-05	 \\
0.000928119	 & 	-0.999985	 & 	0.00540587	 \\
-5.72887e-05	 & 	0.00540582	 & 	0.999985
\end{tabular}

\vtab
 EingenValues - Molecule A     \\
\begin{tabular}{|c c c|}
530.124	 & 	915.807	 & 	1361.14	 \\
\end{tabular}
\columnbreak

Molecule B \
3NFAACd

\includegraphics[width=6cm]{../Comparisons/ImagesFromVMD/3NFAACd.png}

Inertia Tensor - Molecule B \\
\begin{tabular}{|c c c|}
524.186	 & 	-1.32648	 & 	-2.36411	 \\
-1.32648	 & 	935.358	 & 	-2.91274	 \\
-2.36411	 & 	-2.91274	 & 	1359.16
\end{tabular}

\vtab
 EingenVectors - Molecule B     \\
\begin{tabular}{|c c c|}
-0.999991	 & 	-0.00324612	 & 	-0.00284261	 \\
0.00326554	 & 	-0.999971	 & 	-0.0068542	 \\
-0.00282028	 & 	-0.00686342	 & 	0.999972
\end{tabular}

\vtab
 EingenValues - Molecule B     \\
\begin{tabular}{|c c c|}
524.175	 & 	935.342	 & 	1359.19	 \\
\end{tabular}

\end{center}
\end{multicols}

\vtab[-5mm]
\begin{tabular}{*{2}{m{0.38\textwidth}}}
\begin{center}
\textcolor{NavyBlue}{\Large Different}
\end{center}
&
\begin{center}
\includegraphics[height=6.5cm]{../Comparisons/Vectors/inertia_tensor_of_3NFAACa_and_3NFAACd.png}
\end{center}
\end{tabular}

 \newpage

\vtab[-3cm]
\begin{center}
{\large FireTest \tab Número 205}
\end{center}
\begin{multicols}{2}
\begin{center}

Molecule A \
3NFAACa

\includegraphics[width=6cm]{../Comparisons/ImagesFromVMD/3NFAACa.png}

Inertia Tensor - Molecule A \\
\begin{tabular}{|c c c|}
530.125	 & 	-0.358211	 & 	-0.0456723	 \\
-0.358211	 & 	915.82	 & 	2.40736	 \\
-0.0456723	 & 	2.40736	 & 	1361.13
\end{tabular}

\vtab
 EingenVectors - Molecule A     \\
\begin{tabular}{|c c c|}
-1	 & 	-0.000928415	 & 	-5.22706e-05	 \\
0.000928119	 & 	-0.999985	 & 	0.00540587	 \\
-5.72887e-05	 & 	0.00540582	 & 	0.999985
\end{tabular}

\vtab
 EingenValues - Molecule A     \\
\begin{tabular}{|c c c|}
530.124	 & 	915.807	 & 	1361.14	 \\
\end{tabular}
\columnbreak

Molecule B \
3NFAACe

\includegraphics[width=6cm]{../Comparisons/ImagesFromVMD/3NFAACe.png}

Inertia Tensor - Molecule B \\
\begin{tabular}{|c c c|}
524.508	 & 	15.8785	 & 	-4.02001	 \\
15.8785	 & 	1358.65	 & 	10.8723	 \\
-4.02001	 & 	10.8723	 & 	935.54
\end{tabular}

\vtab
 EingenVectors - Molecule B     \\
\begin{tabular}{|c c c|}
-0.999764	 & 	0.0191573	 & 	-0.0102761	 \\
0.0107588	 & 	0.025269	 & 	-0.999623	 \\
0.0188904	 & 	0.999497	 & 	0.0254692
\end{tabular}

\vtab
 EingenValues - Molecule B     \\
\begin{tabular}{|c c c|}
524.162	 & 	935.308	 & 	1359.22	 \\
\end{tabular}

\end{center}
\end{multicols}

\vtab[-5mm]
\begin{tabular}{*{2}{m{0.38\textwidth}}}
\begin{center}
\textcolor{NavyBlue}{\Large Different}
\end{center}
&
\begin{center}
\includegraphics[height=6.5cm]{../Comparisons/Vectors/inertia_tensor_of_3NFAACa_and_3NFAACe.png}
\end{center}
\end{tabular}

 \newpage

\vtab[-3cm]
\begin{center}
{\large FireTest \tab Número 206}
\end{center}
\begin{multicols}{2}
\begin{center}

Molecule A \
3NFAACa

\includegraphics[width=6cm]{../Comparisons/ImagesFromVMD/3NFAACa.png}

Inertia Tensor - Molecule A \\
\begin{tabular}{|c c c|}
530.125	 & 	-0.358211	 & 	-0.0456723	 \\
-0.358211	 & 	915.82	 & 	2.40736	 \\
-0.0456723	 & 	2.40736	 & 	1361.13
\end{tabular}

\vtab
 EingenVectors - Molecule A     \\
\begin{tabular}{|c c c|}
-1	 & 	-0.000928415	 & 	-5.22706e-05	 \\
0.000928119	 & 	-0.999985	 & 	0.00540587	 \\
-5.72887e-05	 & 	0.00540582	 & 	0.999985
\end{tabular}

\vtab
 EingenValues - Molecule A     \\
\begin{tabular}{|c c c|}
530.124	 & 	915.807	 & 	1361.14	 \\
\end{tabular}
\columnbreak

Molecule B \
3NFAACf

\includegraphics[width=6cm]{../Comparisons/ImagesFromVMD/3NFAACf.png}

Inertia Tensor - Molecule B \\
\begin{tabular}{|c c c|}
530.208	 & 	-1.63273	 & 	-7.84857	 \\
-1.63273	 & 	1360.76	 & 	-3.6411	 \\
-7.84857	 & 	-3.6411	 & 	915.396
\end{tabular}

\vtab
 EingenVectors - Molecule B     \\
\begin{tabular}{|c c c|}
-0.99979	 & 	-0.00205437	 & 	-0.0203824	 \\
-0.0203985	 & 	0.00810116	 & 	0.999759	 \\
-0.00188875	 & 	0.999965	 & 	-0.00814137
\end{tabular}

\vtab
 EingenValues - Molecule B     \\
\begin{tabular}{|c c c|}
530.045	 & 	915.527	 & 	1360.79	 \\
\end{tabular}

\end{center}
\end{multicols}

\vtab[-5mm]
\begin{tabular}{*{2}{m{0.38\textwidth}}}
\begin{center}
\textcolor{NavyBlue}{\Large Equal}
\end{center}
&
\begin{center}
\includegraphics[height=6.5cm]{../Comparisons/Vectors/inertia_tensor_of_3NFAACa_and_3NFAACf.png}
\end{center}
\end{tabular}

 \newpage

\vtab[-3cm]
\begin{center}
{\large FireTest \tab Número 207}
\end{center}
\begin{multicols}{2}
\begin{center}

Molecule A \
3NFAACa

\includegraphics[width=6cm]{../Comparisons/ImagesFromVMD/3NFAACa.png}

Inertia Tensor - Molecule A \\
\begin{tabular}{|c c c|}
530.125	 & 	-0.358211	 & 	-0.0456723	 \\
-0.358211	 & 	915.82	 & 	2.40736	 \\
-0.0456723	 & 	2.40736	 & 	1361.13
\end{tabular}

\vtab
 EingenVectors - Molecule A     \\
\begin{tabular}{|c c c|}
-1	 & 	-0.000928415	 & 	-5.22706e-05	 \\
0.000928119	 & 	-0.999985	 & 	0.00540587	 \\
-5.72887e-05	 & 	0.00540582	 & 	0.999985
\end{tabular}

\vtab
 EingenValues - Molecule A     \\
\begin{tabular}{|c c c|}
530.124	 & 	915.807	 & 	1361.14	 \\
\end{tabular}
\columnbreak

Molecule B \
3NFAACg

\includegraphics[width=6cm]{../Comparisons/ImagesFromVMD/3NFAACg.png}

Inertia Tensor - Molecule B \\
\begin{tabular}{|c c c|}
532.891	 & 	-0.526938	 & 	-0.504713	 \\
-0.526938	 & 	918.454	 & 	2.35809	 \\
-0.504713	 & 	2.35809	 & 	1356.91
\end{tabular}

\vtab
 EingenVectors - Molecule B     \\
\begin{tabular}{|c c c|}
-0.999999	 & 	-0.00136295	 & 	-0.000608598	 \\
0.00135965	 & 	-0.999985	 & 	0.00537946	 \\
-0.000615921	 & 	0.00537863	 & 	0.999985
\end{tabular}

\vtab
 EingenValues - Molecule B     \\
\begin{tabular}{|c c c|}
532.89	 & 	918.442	 & 	1356.93	 \\
\end{tabular}

\end{center}
\end{multicols}

\vtab[-5mm]
\begin{tabular}{*{2}{m{0.38\textwidth}}}
\begin{center}
\textcolor{NavyBlue}{\Large Different}
\end{center}
&
\begin{center}
\includegraphics[height=6.5cm]{../Comparisons/Vectors/inertia_tensor_of_3NFAACa_and_3NFAACg.png}
\end{center}
\end{tabular}

 \newpage

\vtab[-3cm]
\begin{center}
{\large FireTest \tab Número 208}
\end{center}
\begin{multicols}{2}
\begin{center}

Molecule A \
3NFAACa

\includegraphics[width=6cm]{../Comparisons/ImagesFromVMD/3NFAACa.png}

Inertia Tensor - Molecule A \\
\begin{tabular}{|c c c|}
530.125	 & 	-0.358211	 & 	-0.0456723	 \\
-0.358211	 & 	915.82	 & 	2.40736	 \\
-0.0456723	 & 	2.40736	 & 	1361.13
\end{tabular}

\vtab
 EingenVectors - Molecule A     \\
\begin{tabular}{|c c c|}
-1	 & 	-0.000928415	 & 	-5.22706e-05	 \\
0.000928119	 & 	-0.999985	 & 	0.00540587	 \\
-5.72887e-05	 & 	0.00540582	 & 	0.999985
\end{tabular}

\vtab
 EingenValues - Molecule A     \\
\begin{tabular}{|c c c|}
530.124	 & 	915.807	 & 	1361.14	 \\
\end{tabular}
\columnbreak

Molecule B \
3NFAACh

\includegraphics[width=6cm]{../Comparisons/ImagesFromVMD/3NFAACh.png}

Inertia Tensor - Molecule B \\
\begin{tabular}{|c c c|}
528.718	 & 	-3.89066	 & 	-3.66882	 \\
-3.89066	 & 	924.911	 & 	2.93817	 \\
-3.66882	 & 	2.93817	 & 	1336.08
\end{tabular}

\vtab
 EingenVectors - Molecule B     \\
\begin{tabular}{|c c c|}
0.999942	 & 	0.00978475	 & 	0.00450803	 \\
0.00975199	 & 	-0.999926	 & 	0.00723269	 \\
-0.00457847	 & 	0.0071883	 & 	0.999964
\end{tabular}

\vtab
 EingenValues - Molecule B     \\
\begin{tabular}{|c c c|}
528.663	 & 	924.928	 & 	1336.12	 \\
\end{tabular}

\end{center}
\end{multicols}

\vtab[-5mm]
\begin{tabular}{*{2}{m{0.38\textwidth}}}
\begin{center}
\textcolor{NavyBlue}{\Large Different}
\end{center}
&
\begin{center}
\includegraphics[height=6.5cm]{../Comparisons/Vectors/inertia_tensor_of_3NFAACa_and_3NFAACh.png}
\end{center}
\end{tabular}

 \newpage

\vtab[-3cm]
\begin{center}
{\large FireTest \tab Número 209}
\end{center}
\begin{multicols}{2}
\begin{center}

Molecule A \
3NFAACa

\includegraphics[width=6cm]{../Comparisons/ImagesFromVMD/3NFAACa.png}

Inertia Tensor - Molecule A \\
\begin{tabular}{|c c c|}
530.125	 & 	-0.358211	 & 	-0.0456723	 \\
-0.358211	 & 	915.82	 & 	2.40736	 \\
-0.0456723	 & 	2.40736	 & 	1361.13
\end{tabular}

\vtab
 EingenVectors - Molecule A     \\
\begin{tabular}{|c c c|}
-1	 & 	-0.000928415	 & 	-5.22706e-05	 \\
0.000928119	 & 	-0.999985	 & 	0.00540587	 \\
-5.72887e-05	 & 	0.00540582	 & 	0.999985
\end{tabular}

\vtab
 EingenValues - Molecule A     \\
\begin{tabular}{|c c c|}
530.124	 & 	915.807	 & 	1361.14	 \\
\end{tabular}
\columnbreak

Molecule B \
3NFAACi

\includegraphics[width=6cm]{../Comparisons/ImagesFromVMD/3NFAACi.png}

Inertia Tensor - Molecule B \\
\begin{tabular}{|c c c|}
521.487	 & 	-1.22943	 & 	-2.71042	 \\
-1.22943	 & 	949.663	 & 	1.67006	 \\
-2.71042	 & 	1.67006	 & 	1346.16
\end{tabular}

\vtab
 EingenVectors - Molecule B     \\
\begin{tabular}{|c c c|}
0.999991	 & 	0.00285841	 & 	0.00328077	 \\
0.00284452	 & 	-0.999987	 & 	0.00423133	 \\
-0.00329282	 & 	0.00422196	 & 	0.999986
\end{tabular}

\vtab
 EingenValues - Molecule B     \\
\begin{tabular}{|c c c|}
521.475	 & 	949.659	 & 	1346.18	 \\
\end{tabular}

\end{center}
\end{multicols}

\vtab[-5mm]
\begin{tabular}{*{2}{m{0.38\textwidth}}}
\begin{center}
\textcolor{NavyBlue}{\Large Different}
\end{center}
&
\begin{center}
\includegraphics[height=6.5cm]{../Comparisons/Vectors/inertia_tensor_of_3NFAACa_and_3NFAACi.png}
\end{center}
\end{tabular}

 \newpage

\vtab[-3cm]
\begin{center}
{\large FireTest \tab Número 210}
\end{center}
\begin{multicols}{2}
\begin{center}

Molecule A \
3NFAACa

\includegraphics[width=6cm]{../Comparisons/ImagesFromVMD/3NFAACa.png}

Inertia Tensor - Molecule A \\
\begin{tabular}{|c c c|}
530.125	 & 	-0.358211	 & 	-0.0456723	 \\
-0.358211	 & 	915.82	 & 	2.40736	 \\
-0.0456723	 & 	2.40736	 & 	1361.13
\end{tabular}

\vtab
 EingenVectors - Molecule A     \\
\begin{tabular}{|c c c|}
-1	 & 	-0.000928415	 & 	-5.22706e-05	 \\
0.000928119	 & 	-0.999985	 & 	0.00540587	 \\
-5.72887e-05	 & 	0.00540582	 & 	0.999985
\end{tabular}

\vtab
 EingenValues - Molecule A     \\
\begin{tabular}{|c c c|}
530.124	 & 	915.807	 & 	1361.14	 \\
\end{tabular}
\columnbreak

Molecule B \
3NFAACj

\includegraphics[width=6cm]{../Comparisons/ImagesFromVMD/3NFAACj.png}

Inertia Tensor - Molecule B \\
\begin{tabular}{|c c c|}
533.789	 & 	-4.75521	 & 	-1.91525	 \\
-4.75521	 & 	920.091	 & 	2.28449	 \\
-1.91525	 & 	2.28449	 & 	1348.28
\end{tabular}

\vtab
 EingenVectors - Molecule B     \\
\begin{tabular}{|c c c|}
-0.999922	 & 	-0.0122929	 & 	-0.00231663	 \\
0.0122803	 & 	-0.99991	 & 	0.00539026	 \\
-0.00238268	 & 	0.00536139	 & 	0.999983
\end{tabular}

\vtab
 EingenValues - Molecule B     \\
\begin{tabular}{|c c c|}
533.726	 & 	920.137	 & 	1348.3	 \\
\end{tabular}

\end{center}
\end{multicols}

\vtab[-5mm]
\begin{tabular}{*{2}{m{0.38\textwidth}}}
\begin{center}
\textcolor{NavyBlue}{\Large Different}
\end{center}
&
\begin{center}
\includegraphics[height=6.5cm]{../Comparisons/Vectors/inertia_tensor_of_3NFAACa_and_3NFAACj.png}
\end{center}
\end{tabular}

 \newpage

\vtab[-3cm]
\begin{center}
{\large FireTest \tab Número 211}
\end{center}
\begin{multicols}{2}
\begin{center}

Molecule A \
3NFAACa

\includegraphics[width=6cm]{../Comparisons/ImagesFromVMD/3NFAACa.png}

Inertia Tensor - Molecule A \\
\begin{tabular}{|c c c|}
530.125	 & 	-0.358211	 & 	-0.0456723	 \\
-0.358211	 & 	915.82	 & 	2.40736	 \\
-0.0456723	 & 	2.40736	 & 	1361.13
\end{tabular}

\vtab
 EingenVectors - Molecule A     \\
\begin{tabular}{|c c c|}
-1	 & 	-0.000928415	 & 	-5.22706e-05	 \\
0.000928119	 & 	-0.999985	 & 	0.00540587	 \\
-5.72887e-05	 & 	0.00540582	 & 	0.999985
\end{tabular}

\vtab
 EingenValues - Molecule A     \\
\begin{tabular}{|c c c|}
530.124	 & 	915.807	 & 	1361.14	 \\
\end{tabular}
\columnbreak

Molecule B \
3NFAACk

\includegraphics[width=6cm]{../Comparisons/ImagesFromVMD/3NFAACk.png}

Inertia Tensor - Molecule B \\
\begin{tabular}{|c c c|}
534.899	 & 	-5.81418	 & 	-0.270063	 \\
-5.81418	 & 	913.263	 & 	2.4519	 \\
-0.270063	 & 	2.4519	 & 	1353.77
\end{tabular}

\vtab
 EingenVectors - Molecule B     \\
\begin{tabular}{|c c c|}
-0.999882	 & 	-0.0153593	 & 	-0.00028374	 \\
0.0153575	 & 	-0.999867	 & 	0.00557573	 \\
-0.000369342	 & 	0.00557072	 & 	0.999984
\end{tabular}

\vtab
 EingenValues - Molecule B     \\
\begin{tabular}{|c c c|}
534.81	 & 	913.339	 & 	1353.78	 \\
\end{tabular}

\end{center}
\end{multicols}

\vtab[-5mm]
\begin{tabular}{*{2}{m{0.38\textwidth}}}
\begin{center}
\textcolor{NavyBlue}{\Large Different}
\end{center}
&
\begin{center}
\includegraphics[height=6.5cm]{../Comparisons/Vectors/inertia_tensor_of_3NFAACa_and_3NFAACk.png}
\end{center}
\end{tabular}

 \newpage

\vtab[-3cm]
\begin{center}
{\large FireTest \tab Número 212}
\end{center}
\begin{multicols}{2}
\begin{center}

Molecule A \
3NFAACa

\includegraphics[width=6cm]{../Comparisons/ImagesFromVMD/3NFAACa.png}

Inertia Tensor - Molecule A \\
\begin{tabular}{|c c c|}
530.125	 & 	-0.358211	 & 	-0.0456723	 \\
-0.358211	 & 	915.82	 & 	2.40736	 \\
-0.0456723	 & 	2.40736	 & 	1361.13
\end{tabular}

\vtab
 EingenVectors - Molecule A     \\
\begin{tabular}{|c c c|}
-1	 & 	-0.000928415	 & 	-5.22706e-05	 \\
0.000928119	 & 	-0.999985	 & 	0.00540587	 \\
-5.72887e-05	 & 	0.00540582	 & 	0.999985
\end{tabular}

\vtab
 EingenValues - Molecule A     \\
\begin{tabular}{|c c c|}
530.124	 & 	915.807	 & 	1361.14	 \\
\end{tabular}
\columnbreak

Molecule B \
3NFAACl

\includegraphics[width=6cm]{../Comparisons/ImagesFromVMD/3NFAACl.png}

Inertia Tensor - Molecule B \\
\begin{tabular}{|c c c|}
531.723	 & 	3.03424	 & 	2.73426	 \\
3.03424	 & 	929.418	 & 	-1.84284	 \\
2.73426	 & 	-1.84284	 & 	1355.39
\end{tabular}

\vtab
 EingenVectors - Molecule B     \\
\begin{tabular}{|c c c|}
0.999965	 & 	-0.00764413	 & 	-0.00333648	 \\
-0.00765838	 & 	-0.999962	 & 	-0.00427708	 \\
0.00330366	 & 	-0.00430248	 & 	0.999985
\end{tabular}

\vtab
 EingenValues - Molecule B     \\
\begin{tabular}{|c c c|}
531.691	 & 	929.434	 & 	1355.4	 \\
\end{tabular}

\end{center}
\end{multicols}

\vtab[-5mm]
\begin{tabular}{*{2}{m{0.38\textwidth}}}
\begin{center}
\textcolor{NavyBlue}{\Large Different}
\end{center}
&
\begin{center}
\includegraphics[height=6.5cm]{../Comparisons/Vectors/inertia_tensor_of_3NFAACa_and_3NFAACl.png}
\end{center}
\end{tabular}

 \newpage

\vtab[-3cm]
\begin{center}
{\large FireTest \tab Número 213}
\end{center}
\begin{multicols}{2}
\begin{center}

Molecule A \
3NFAACa

\includegraphics[width=6cm]{../Comparisons/ImagesFromVMD/3NFAACa.png}

Inertia Tensor - Molecule A \\
\begin{tabular}{|c c c|}
530.125	 & 	-0.358211	 & 	-0.0456723	 \\
-0.358211	 & 	915.82	 & 	2.40736	 \\
-0.0456723	 & 	2.40736	 & 	1361.13
\end{tabular}

\vtab
 EingenVectors - Molecule A     \\
\begin{tabular}{|c c c|}
-1	 & 	-0.000928415	 & 	-5.22706e-05	 \\
0.000928119	 & 	-0.999985	 & 	0.00540587	 \\
-5.72887e-05	 & 	0.00540582	 & 	0.999985
\end{tabular}

\vtab
 EingenValues - Molecule A     \\
\begin{tabular}{|c c c|}
530.124	 & 	915.807	 & 	1361.14	 \\
\end{tabular}
\columnbreak

Molecule B \
3NFAACm

\includegraphics[width=6cm]{../Comparisons/ImagesFromVMD/3NFAACm.png}

Inertia Tensor - Molecule B \\
\begin{tabular}{|c c c|}
532.546	 & 	13.7854	 & 	-15.4626	 \\
13.7854	 & 	1354.87	 & 	11.5786	 \\
-15.4626	 & 	11.5786	 & 	929.101
\end{tabular}

\vtab
 EingenVectors - Molecule B     \\
\begin{tabular}{|c c c|}
-0.999075	 & 	0.0172851	 & 	-0.0393769	 \\
0.0398168	 & 	0.0258942	 & 	-0.998871	 \\
-0.016246	 & 	-0.999515	 & 	-0.0265584
\end{tabular}

\vtab
 EingenValues - Molecule B     \\
\begin{tabular}{|c c c|}
531.698	 & 	929.417	 & 	1355.4	 \\
\end{tabular}

\end{center}
\end{multicols}

\vtab[-5mm]
\begin{tabular}{*{2}{m{0.38\textwidth}}}
\begin{center}
\textcolor{NavyBlue}{\Large Different}
\end{center}
&
\begin{center}
\includegraphics[height=6.5cm]{../Comparisons/Vectors/inertia_tensor_of_3NFAACa_and_3NFAACm.png}
\end{center}
\end{tabular}

 \newpage

\vtab[-3cm]
\begin{center}
{\large FireTest \tab Número 214}
\end{center}
\begin{multicols}{2}
\begin{center}

Molecule A \
3NFAACa

\includegraphics[width=6cm]{../Comparisons/ImagesFromVMD/3NFAACa.png}

Inertia Tensor - Molecule A \\
\begin{tabular}{|c c c|}
530.125	 & 	-0.358211	 & 	-0.0456723	 \\
-0.358211	 & 	915.82	 & 	2.40736	 \\
-0.0456723	 & 	2.40736	 & 	1361.13
\end{tabular}

\vtab
 EingenVectors - Molecule A     \\
\begin{tabular}{|c c c|}
-1	 & 	-0.000928415	 & 	-5.22706e-05	 \\
0.000928119	 & 	-0.999985	 & 	0.00540587	 \\
-5.72887e-05	 & 	0.00540582	 & 	0.999985
\end{tabular}

\vtab
 EingenValues - Molecule A     \\
\begin{tabular}{|c c c|}
530.124	 & 	915.807	 & 	1361.14	 \\
\end{tabular}
\columnbreak

Molecule B \
3NFAACn

\includegraphics[width=6cm]{../Comparisons/ImagesFromVMD/3NFAACn.png}

Inertia Tensor - Molecule B \\
\begin{tabular}{|c c c|}
531.896	 & 	3.78027	 & 	-13.1151	 \\
3.78027	 & 	1353.2	 & 	-7.47403	 \\
-13.1151	 & 	-7.47403	 & 	912.989
\end{tabular}

\vtab
 EingenVectors - Molecule B     \\
\begin{tabular}{|c c c|}
0.999403	 & 	-0.00428573	 & 	0.0342679	 \\
0.0341891	 & 	-0.0172718	 & 	-0.999266	 \\
0.00487445	 & 	0.999842	 & 	-0.017115
\end{tabular}

\vtab
 EingenValues - Molecule B     \\
\begin{tabular}{|c c c|}
531.43	 & 	913.309	 & 	1353.35	 \\
\end{tabular}

\end{center}
\end{multicols}

\vtab[-5mm]
\begin{tabular}{*{2}{m{0.38\textwidth}}}
\begin{center}
\textcolor{NavyBlue}{\Large Different}
\end{center}
&
\begin{center}
\includegraphics[height=6.5cm]{../Comparisons/Vectors/inertia_tensor_of_3NFAACa_and_3NFAACn.png}
\end{center}
\end{tabular}

 \newpage

\vtab[-3cm]
\begin{center}
{\large FireTest \tab Número 215}
\end{center}
\begin{multicols}{2}
\begin{center}

Molecule A \
3NFAACa

\includegraphics[width=6cm]{../Comparisons/ImagesFromVMD/3NFAACa.png}

Inertia Tensor - Molecule A \\
\begin{tabular}{|c c c|}
530.125	 & 	-0.358211	 & 	-0.0456723	 \\
-0.358211	 & 	915.82	 & 	2.40736	 \\
-0.0456723	 & 	2.40736	 & 	1361.13
\end{tabular}

\vtab
 EingenVectors - Molecule A     \\
\begin{tabular}{|c c c|}
-1	 & 	-0.000928415	 & 	-5.22706e-05	 \\
0.000928119	 & 	-0.999985	 & 	0.00540587	 \\
-5.72887e-05	 & 	0.00540582	 & 	0.999985
\end{tabular}

\vtab
 EingenValues - Molecule A     \\
\begin{tabular}{|c c c|}
530.124	 & 	915.807	 & 	1361.14	 \\
\end{tabular}
\columnbreak

Molecule B \
4NFAACa

\includegraphics[width=6cm]{../Comparisons/ImagesFromVMD/4NFAACa.png}

Inertia Tensor - Molecule B \\
\begin{tabular}{|c c c|}
479.392	 & 	3.27131	 & 	4.22557	 \\
3.27131	 & 	1242.39	 & 	-0.852684	 \\
4.22557	 & 	-0.852684	 & 	1647.37
\end{tabular}

\vtab
 EingenVectors - Molecule B     \\
\begin{tabular}{|c c c|}
0.999984	 & 	-0.00429123	 & 	-0.00362083	 \\
-0.00429871	 & 	-0.999989	 & 	-0.0020607	 \\
0.00361195	 & 	-0.00207623	 & 	0.999991
\end{tabular}

\vtab
 EingenValues - Molecule B     \\
\begin{tabular}{|c c c|}
479.363	 & 	1242.41	 & 	1647.39	 \\
\end{tabular}

\end{center}
\end{multicols}

\vtab[-5mm]
\begin{tabular}{*{2}{m{0.38\textwidth}}}
\begin{center}
\textcolor{NavyBlue}{\Large Different}
\end{center}
&
\begin{center}
\includegraphics[height=6.5cm]{../Comparisons/Vectors/inertia_tensor_of_3NFAACa_and_4NFAACa.png}
\end{center}
\end{tabular}

 \newpage

\vtab[-3cm]
\begin{center}
{\large FireTest \tab Número 216}
\end{center}
\begin{multicols}{2}
\begin{center}

Molecule A \
3NFAACa

\includegraphics[width=6cm]{../Comparisons/ImagesFromVMD/3NFAACa.png}

Inertia Tensor - Molecule A \\
\begin{tabular}{|c c c|}
530.125	 & 	-0.358211	 & 	-0.0456723	 \\
-0.358211	 & 	915.82	 & 	2.40736	 \\
-0.0456723	 & 	2.40736	 & 	1361.13
\end{tabular}

\vtab
 EingenVectors - Molecule A     \\
\begin{tabular}{|c c c|}
-1	 & 	-0.000928415	 & 	-5.22706e-05	 \\
0.000928119	 & 	-0.999985	 & 	0.00540587	 \\
-5.72887e-05	 & 	0.00540582	 & 	0.999985
\end{tabular}

\vtab
 EingenValues - Molecule A     \\
\begin{tabular}{|c c c|}
530.124	 & 	915.807	 & 	1361.14	 \\
\end{tabular}
\columnbreak

Molecule B \
4NFAACb

\includegraphics[width=6cm]{../Comparisons/ImagesFromVMD/4NFAACb.png}

Inertia Tensor - Molecule B \\
\begin{tabular}{|c c c|}
479.338	 & 	3.27331	 & 	-4.22553	 \\
3.27331	 & 	1242.4	 & 	0.852083	 \\
-4.22553	 & 	0.852083	 & 	1647.3
\end{tabular}

\vtab
 EingenVectors - Molecule B     \\
\begin{tabular}{|c c c|}
0.999984	 & 	-0.00429353	 & 	0.00362086	 \\
-0.004301	 & 	-0.999989	 & 	0.00205959	 \\
-0.00361198	 & 	0.00207513	 & 	0.999991
\end{tabular}

\vtab
 EingenValues - Molecule B     \\
\begin{tabular}{|c c c|}
479.308	 & 	1242.41	 & 	1647.31	 \\
\end{tabular}

\end{center}
\end{multicols}

\vtab[-5mm]
\begin{tabular}{*{2}{m{0.38\textwidth}}}
\begin{center}
\textcolor{NavyBlue}{\Large Different}
\end{center}
&
\begin{center}
\includegraphics[height=6.5cm]{../Comparisons/Vectors/inertia_tensor_of_3NFAACa_and_4NFAACb.png}
\end{center}
\end{tabular}

 \newpage

\vtab[-3cm]
\begin{center}
{\large FireTest \tab Número 217}
\end{center}
\begin{multicols}{2}
\begin{center}

Molecule A \
3NFAACa

\includegraphics[width=6cm]{../Comparisons/ImagesFromVMD/3NFAACa.png}

Inertia Tensor - Molecule A \\
\begin{tabular}{|c c c|}
530.125	 & 	-0.358211	 & 	-0.0456723	 \\
-0.358211	 & 	915.82	 & 	2.40736	 \\
-0.0456723	 & 	2.40736	 & 	1361.13
\end{tabular}

\vtab
 EingenVectors - Molecule A     \\
\begin{tabular}{|c c c|}
-1	 & 	-0.000928415	 & 	-5.22706e-05	 \\
0.000928119	 & 	-0.999985	 & 	0.00540587	 \\
-5.72887e-05	 & 	0.00540582	 & 	0.999985
\end{tabular}

\vtab
 EingenValues - Molecule A     \\
\begin{tabular}{|c c c|}
530.124	 & 	915.807	 & 	1361.14	 \\
\end{tabular}
\columnbreak

Molecule B \
4NFAACc

\includegraphics[width=6cm]{../Comparisons/ImagesFromVMD/4NFAACc.png}

Inertia Tensor - Molecule B \\
\begin{tabular}{|c c c|}
482.067	 & 	-5.39474	 & 	-1.35857	 \\
-5.39474	 & 	1240.3	 & 	-2.54035	 \\
-1.35857	 & 	-2.54035	 & 	1647.06
\end{tabular}

\vtab
 EingenVectors - Molecule B     \\
\begin{tabular}{|c c c|}
-0.999974	 & 	-0.00711826	 & 	-0.0011816	 \\
0.00712547	 & 	-0.999955	 & 	-0.00622156	 \\
-0.00113726	 & 	-0.00622982	 & 	0.99998
\end{tabular}

\vtab
 EingenValues - Molecule B     \\
\begin{tabular}{|c c c|}
482.027	 & 	1240.32	 & 	1647.08	 \\
\end{tabular}

\end{center}
\end{multicols}

\vtab[-5mm]
\begin{tabular}{*{2}{m{0.38\textwidth}}}
\begin{center}
\textcolor{NavyBlue}{\Large Different}
\end{center}
&
\begin{center}
\includegraphics[height=6.5cm]{../Comparisons/Vectors/inertia_tensor_of_3NFAACa_and_4NFAACc.png}
\end{center}
\end{tabular}

 \newpage

\vtab[-3cm]
\begin{center}
{\large FireTest \tab Número 218}
\end{center}
\begin{multicols}{2}
\begin{center}

Molecule A \
3NFAACa

\includegraphics[width=6cm]{../Comparisons/ImagesFromVMD/3NFAACa.png}

Inertia Tensor - Molecule A \\
\begin{tabular}{|c c c|}
530.125	 & 	-0.358211	 & 	-0.0456723	 \\
-0.358211	 & 	915.82	 & 	2.40736	 \\
-0.0456723	 & 	2.40736	 & 	1361.13
\end{tabular}

\vtab
 EingenVectors - Molecule A     \\
\begin{tabular}{|c c c|}
-1	 & 	-0.000928415	 & 	-5.22706e-05	 \\
0.000928119	 & 	-0.999985	 & 	0.00540587	 \\
-5.72887e-05	 & 	0.00540582	 & 	0.999985
\end{tabular}

\vtab
 EingenValues - Molecule A     \\
\begin{tabular}{|c c c|}
530.124	 & 	915.807	 & 	1361.14	 \\
\end{tabular}
\columnbreak

Molecule B \
4NFAACd

\includegraphics[width=6cm]{../Comparisons/ImagesFromVMD/4NFAACd.png}

Inertia Tensor - Molecule B \\
\begin{tabular}{|c c c|}
491.672	 & 	0.24486	 & 	-3.10016	 \\
0.24486	 & 	1231.15	 & 	2.19965	 \\
-3.10016	 & 	2.19965	 & 	1650.11
\end{tabular}

\vtab
 EingenVectors - Molecule B     \\
\begin{tabular}{|c c c|}
0.999996	 & 	-0.000339081	 & 	0.00267677	 \\
-0.000353124	 & 	-0.999986	 & 	0.00524747	 \\
-0.00267495	 & 	0.0052484	 & 	0.999983
\end{tabular}

\vtab
 EingenValues - Molecule B     \\
\begin{tabular}{|c c c|}
491.663	 & 	1231.14	 & 	1650.13	 \\
\end{tabular}

\end{center}
\end{multicols}

\vtab[-5mm]
\begin{tabular}{*{2}{m{0.38\textwidth}}}
\begin{center}
\textcolor{NavyBlue}{\Large Different}
\end{center}
&
\begin{center}
\includegraphics[height=6.5cm]{../Comparisons/Vectors/inertia_tensor_of_3NFAACa_and_4NFAACd.png}
\end{center}
\end{tabular}

 \newpage

\vtab[-3cm]
\begin{center}
{\large FireTest \tab Número 219}
\end{center}
\begin{multicols}{2}
\begin{center}

Molecule A \
3NFAACa

\includegraphics[width=6cm]{../Comparisons/ImagesFromVMD/3NFAACa.png}

Inertia Tensor - Molecule A \\
\begin{tabular}{|c c c|}
530.125	 & 	-0.358211	 & 	-0.0456723	 \\
-0.358211	 & 	915.82	 & 	2.40736	 \\
-0.0456723	 & 	2.40736	 & 	1361.13
\end{tabular}

\vtab
 EingenVectors - Molecule A     \\
\begin{tabular}{|c c c|}
-1	 & 	-0.000928415	 & 	-5.22706e-05	 \\
0.000928119	 & 	-0.999985	 & 	0.00540587	 \\
-5.72887e-05	 & 	0.00540582	 & 	0.999985
\end{tabular}

\vtab
 EingenValues - Molecule A     \\
\begin{tabular}{|c c c|}
530.124	 & 	915.807	 & 	1361.14	 \\
\end{tabular}
\columnbreak

Molecule B \
4NFAACe

\includegraphics[width=6cm]{../Comparisons/ImagesFromVMD/4NFAACe.png}

Inertia Tensor - Molecule B \\
\begin{tabular}{|c c c|}
489.025	 & 	-0.430035	 & 	3.98876	 \\
-0.430035	 & 	1233.71	 & 	-2.06505	 \\
3.98876	 & 	-2.06505	 & 	1641.79
\end{tabular}

\vtab
 EingenVectors - Molecule B     \\
\begin{tabular}{|c c c|}
0.999994	 & 	0.000567863	 & 	-0.00345908	 \\
0.000550336	 & 	-0.999987	 & 	-0.00506565	 \\
0.00346192	 & 	-0.00506372	 & 	0.999981
\end{tabular}

\vtab
 EingenValues - Molecule B     \\
\begin{tabular}{|c c c|}
489.011	 & 	1233.7	 & 	1641.81	 \\
\end{tabular}

\end{center}
\end{multicols}

\vtab[-5mm]
\begin{tabular}{*{2}{m{0.38\textwidth}}}
\begin{center}
\textcolor{NavyBlue}{\Large Different}
\end{center}
&
\begin{center}
\includegraphics[height=6.5cm]{../Comparisons/Vectors/inertia_tensor_of_3NFAACa_and_4NFAACe.png}
\end{center}
\end{tabular}

 \newpage

\vtab[-3cm]
\begin{center}
{\large FireTest \tab Número 220}
\end{center}
\begin{multicols}{2}
\begin{center}

Molecule A \
3NFAACa

\includegraphics[width=6cm]{../Comparisons/ImagesFromVMD/3NFAACa.png}

Inertia Tensor - Molecule A \\
\begin{tabular}{|c c c|}
530.125	 & 	-0.358211	 & 	-0.0456723	 \\
-0.358211	 & 	915.82	 & 	2.40736	 \\
-0.0456723	 & 	2.40736	 & 	1361.13
\end{tabular}

\vtab
 EingenVectors - Molecule A     \\
\begin{tabular}{|c c c|}
-1	 & 	-0.000928415	 & 	-5.22706e-05	 \\
0.000928119	 & 	-0.999985	 & 	0.00540587	 \\
-5.72887e-05	 & 	0.00540582	 & 	0.999985
\end{tabular}

\vtab
 EingenValues - Molecule A     \\
\begin{tabular}{|c c c|}
530.124	 & 	915.807	 & 	1361.14	 \\
\end{tabular}
\columnbreak

Molecule B \
4NFAACf

\includegraphics[width=6cm]{../Comparisons/ImagesFromVMD/4NFAACf.png}

Inertia Tensor - Molecule B \\
\begin{tabular}{|c c c|}
509.683	 & 	2.80651	 & 	-1.91422	 \\
2.80651	 & 	1219.11	 & 	2.66132	 \\
-1.91422	 & 	2.66132	 & 	1681.17
\end{tabular}

\vtab
 EingenVectors - Molecule B     \\
\begin{tabular}{|c c c|}
-0.999991	 & 	0.00396206	 & 	-0.00164298	 \\
-0.00397143	 & 	-0.999976	 & 	0.0057431	 \\
-0.00162019	 & 	0.00574957	 & 	0.999982
\end{tabular}

\vtab
 EingenValues - Molecule B     \\
\begin{tabular}{|c c c|}
509.668	 & 	1219.11	 & 	1681.18	 \\
\end{tabular}

\end{center}
\end{multicols}

\vtab[-5mm]
\begin{tabular}{*{2}{m{0.38\textwidth}}}
\begin{center}
\textcolor{NavyBlue}{\Large Different}
\end{center}
&
\begin{center}
\includegraphics[height=6.5cm]{../Comparisons/Vectors/inertia_tensor_of_3NFAACa_and_4NFAACf.png}
\end{center}
\end{tabular}

 \newpage

\vtab[-3cm]
\begin{center}
{\large FireTest \tab Número 221}
\end{center}
\begin{multicols}{2}
\begin{center}

Molecule A \
3NFAACa

\includegraphics[width=6cm]{../Comparisons/ImagesFromVMD/3NFAACa.png}

Inertia Tensor - Molecule A \\
\begin{tabular}{|c c c|}
530.125	 & 	-0.358211	 & 	-0.0456723	 \\
-0.358211	 & 	915.82	 & 	2.40736	 \\
-0.0456723	 & 	2.40736	 & 	1361.13
\end{tabular}

\vtab
 EingenVectors - Molecule A     \\
\begin{tabular}{|c c c|}
-1	 & 	-0.000928415	 & 	-5.22706e-05	 \\
0.000928119	 & 	-0.999985	 & 	0.00540587	 \\
-5.72887e-05	 & 	0.00540582	 & 	0.999985
\end{tabular}

\vtab
 EingenValues - Molecule A     \\
\begin{tabular}{|c c c|}
530.124	 & 	915.807	 & 	1361.14	 \\
\end{tabular}
\columnbreak

Molecule B \
4NFAACg

\includegraphics[width=6cm]{../Comparisons/ImagesFromVMD/4NFAACg.png}

Inertia Tensor - Molecule B \\
\begin{tabular}{|c c c|}
513.78	 & 	4.51917	 & 	0.266555	 \\
4.51917	 & 	1208.04	 & 	-1.18628	 \\
0.266555	 & 	-1.18628	 & 	1700.9
\end{tabular}

\vtab
 EingenVectors - Molecule B     \\
\begin{tabular}{|c c c|}
-0.999979	 & 	0.00650929	 & 	0.000231034	 \\
-0.00650983	 & 	-0.999976	 & 	-0.00240351	 \\
0.000215383	 & 	-0.00240496	 & 	0.999997
\end{tabular}

\vtab
 EingenValues - Molecule B     \\
\begin{tabular}{|c c c|}
513.751	 & 	1208.07	 & 	1700.9	 \\
\end{tabular}

\end{center}
\end{multicols}

\vtab[-5mm]
\begin{tabular}{*{2}{m{0.38\textwidth}}}
\begin{center}
\textcolor{NavyBlue}{\Large Different}
\end{center}
&
\begin{center}
\includegraphics[height=6.5cm]{../Comparisons/Vectors/inertia_tensor_of_3NFAACa_and_4NFAACg.png}
\end{center}
\end{tabular}

 \newpage

\vtab[-3cm]
\begin{center}
{\large FireTest \tab Número 222}
\end{center}
\begin{multicols}{2}
\begin{center}

Molecule A \
3NFAACa

\includegraphics[width=6cm]{../Comparisons/ImagesFromVMD/3NFAACa.png}

Inertia Tensor - Molecule A \\
\begin{tabular}{|c c c|}
530.125	 & 	-0.358211	 & 	-0.0456723	 \\
-0.358211	 & 	915.82	 & 	2.40736	 \\
-0.0456723	 & 	2.40736	 & 	1361.13
\end{tabular}

\vtab
 EingenVectors - Molecule A     \\
\begin{tabular}{|c c c|}
-1	 & 	-0.000928415	 & 	-5.22706e-05	 \\
0.000928119	 & 	-0.999985	 & 	0.00540587	 \\
-5.72887e-05	 & 	0.00540582	 & 	0.999985
\end{tabular}

\vtab
 EingenValues - Molecule A     \\
\begin{tabular}{|c c c|}
530.124	 & 	915.807	 & 	1361.14	 \\
\end{tabular}
\columnbreak

Molecule B \
4NFAACi

\includegraphics[width=6cm]{../Comparisons/ImagesFromVMD/4NFAACi.png}

Inertia Tensor - Molecule B \\
\begin{tabular}{|c c c|}
502.43	 & 	-0.602691	 & 	-4.86988	 \\
-0.602691	 & 	1232.26	 & 	0.407295	 \\
-4.86988	 & 	0.407295	 & 	1676
\end{tabular}

\vtab
 EingenVectors - Molecule B     \\
\begin{tabular}{|c c c|}
0.999991	 & 	0.000823447	 & 	0.00414923	 \\
0.000819608	 & 	-0.999999	 & 	0.00092687	 \\
-0.00414999	 & 	0.000923461	 & 	0.999991
\end{tabular}

\vtab
 EingenValues - Molecule B     \\
\begin{tabular}{|c c c|}
502.409	 & 	1232.26	 & 	1676.02	 \\
\end{tabular}

\end{center}
\end{multicols}

\vtab[-5mm]
\begin{tabular}{*{2}{m{0.38\textwidth}}}
\begin{center}
\textcolor{NavyBlue}{\Large Different}
\end{center}
&
\begin{center}
\includegraphics[height=6.5cm]{../Comparisons/Vectors/inertia_tensor_of_3NFAACa_and_4NFAACi.png}
\end{center}
\end{tabular}

 \newpage

\vtab[-3cm]
\begin{center}
{\large FireTest \tab Número 223}
\end{center}
\begin{multicols}{2}
\begin{center}

Molecule A \
3NFAACa

\includegraphics[width=6cm]{../Comparisons/ImagesFromVMD/3NFAACa.png}

Inertia Tensor - Molecule A \\
\begin{tabular}{|c c c|}
530.125	 & 	-0.358211	 & 	-0.0456723	 \\
-0.358211	 & 	915.82	 & 	2.40736	 \\
-0.0456723	 & 	2.40736	 & 	1361.13
\end{tabular}

\vtab
 EingenVectors - Molecule A     \\
\begin{tabular}{|c c c|}
-1	 & 	-0.000928415	 & 	-5.22706e-05	 \\
0.000928119	 & 	-0.999985	 & 	0.00540587	 \\
-5.72887e-05	 & 	0.00540582	 & 	0.999985
\end{tabular}

\vtab
 EingenValues - Molecule A     \\
\begin{tabular}{|c c c|}
530.124	 & 	915.807	 & 	1361.14	 \\
\end{tabular}
\columnbreak

Molecule B \
4NFAACj

\includegraphics[width=6cm]{../Comparisons/ImagesFromVMD/4NFAACj.png}

Inertia Tensor - Molecule B \\
\begin{tabular}{|c c c|}
510.047	 & 	9.97005	 & 	-3.6306	 \\
9.97005	 & 	1225.52	 & 	-0.981092	 \\
-3.6306	 & 	-0.981092	 & 	1680.82
\end{tabular}

\vtab
 EingenVectors - Molecule B     \\
\begin{tabular}{|c c c|}
-0.999898	 & 	0.0139264	 & 	-0.00308865	 \\
0.0139195	 & 	0.999901	 & 	0.00226627	 \\
-0.00311991	 & 	-0.00222305	 & 	0.999993
\end{tabular}

\vtab
 EingenValues - Molecule B     \\
\begin{tabular}{|c c c|}
509.897	 & 	1225.65	 & 	1680.83	 \\
\end{tabular}

\end{center}
\end{multicols}

\vtab[-5mm]
\begin{tabular}{*{2}{m{0.38\textwidth}}}
\begin{center}
\textcolor{NavyBlue}{\Large Different}
\end{center}
&
\begin{center}
\includegraphics[height=6.5cm]{../Comparisons/Vectors/inertia_tensor_of_3NFAACa_and_4NFAACj.png}
\end{center}
\end{tabular}

 \newpage

\vtab[-3cm]
\begin{center}
{\large FireTest \tab Número 224}
\end{center}
\begin{multicols}{2}
\begin{center}

Molecule A \
3NFAACa

\includegraphics[width=6cm]{../Comparisons/ImagesFromVMD/3NFAACa.png}

Inertia Tensor - Molecule A \\
\begin{tabular}{|c c c|}
530.125	 & 	-0.358211	 & 	-0.0456723	 \\
-0.358211	 & 	915.82	 & 	2.40736	 \\
-0.0456723	 & 	2.40736	 & 	1361.13
\end{tabular}

\vtab
 EingenVectors - Molecule A     \\
\begin{tabular}{|c c c|}
-1	 & 	-0.000928415	 & 	-5.22706e-05	 \\
0.000928119	 & 	-0.999985	 & 	0.00540587	 \\
-5.72887e-05	 & 	0.00540582	 & 	0.999985
\end{tabular}

\vtab
 EingenValues - Molecule A     \\
\begin{tabular}{|c c c|}
530.124	 & 	915.807	 & 	1361.14	 \\
\end{tabular}
\columnbreak

Molecule B \
4NFAACl-3

\includegraphics[width=6cm]{../Comparisons/ImagesFromVMD/4NFAACl-3.png}

Inertia Tensor - Molecule B \\
\begin{tabular}{|c c c|}
506.608	 & 	0.709539	 & 	-0.555426	 \\
0.709539	 & 	1222.37	 & 	-2.84005	 \\
-0.555426	 & 	-2.84005	 & 	1678.41
\end{tabular}

\vtab
 EingenVectors - Molecule B     \\
\begin{tabular}{|c c c|}
-0.999999	 & 	0.000989428	 & 	-0.000471595	 \\
-0.000986471	 & 	-0.99998	 & 	-0.00622856	 \\
-0.000477748	 & 	-0.00622809	 & 	0.99998
\end{tabular}

\vtab
 EingenValues - Molecule B     \\
\begin{tabular}{|c c c|}
506.607	 & 	1222.36	 & 	1678.43	 \\
\end{tabular}

\end{center}
\end{multicols}

\vtab[-5mm]
\begin{tabular}{*{2}{m{0.38\textwidth}}}
\begin{center}
\textcolor{NavyBlue}{\Large Different}
\end{center}
&
\begin{center}
\includegraphics[height=6.5cm]{../Comparisons/Vectors/inertia_tensor_of_3NFAACa_and_4NFAACl-3.png}
\end{center}
\end{tabular}

 \newpage

\vtab[-3cm]
\begin{center}
{\large FireTest \tab Número 225}
\end{center}
\begin{multicols}{2}
\begin{center}

Molecule A \
3NFAACb

\includegraphics[width=6cm]{../Comparisons/ImagesFromVMD/3NFAACb.png}

Inertia Tensor - Molecule A \\
\begin{tabular}{|c c c|}
530.265	 & 	-1.20883	 & 	-7.84653	 \\
-1.20883	 & 	1361.05	 & 	-3.78865	 \\
-7.84653	 & 	-3.78865	 & 	915.61
\end{tabular}

\vtab
 EingenVectors - Molecule A     \\
\begin{tabular}{|c c c|}
-0.999791	 & 	-0.00154731	 & 	-0.0203648	 \\
-0.0203771	 & 	0.00845054	 & 	0.999757	 \\
-0.00137484	 & 	0.999963	 & 	-0.00848031
\end{tabular}

\vtab
 EingenValues - Molecule A     \\
\begin{tabular}{|c c c|}
530.103	 & 	915.738	 & 	1361.08	 \\
\end{tabular}
\columnbreak

Molecule B \
3NFAACc

\includegraphics[width=6cm]{../Comparisons/ImagesFromVMD/3NFAACc.png}

Inertia Tensor - Molecule B \\
\begin{tabular}{|c c c|}
531.482	 & 	-5.48557	 & 	-0.978638	 \\
-5.48557	 & 	913.233	 & 	2.89201	 \\
-0.978638	 & 	2.89201	 & 	1353.14
\end{tabular}

\vtab
 EingenVectors - Molecule B     \\
\begin{tabular}{|c c c|}
-0.999896	 & 	-0.0143563	 & 	-0.00114029	 \\
0.0143485	 & 	-0.999875	 & 	0.00660611	 \\
-0.00123498	 & 	0.00658906	 & 	0.999978
\end{tabular}

\vtab
 EingenValues - Molecule B     \\
\begin{tabular}{|c c c|}
531.402	 & 	913.293	 & 	1353.16	 \\
\end{tabular}

\end{center}
\end{multicols}

\vtab[-5mm]
\begin{tabular}{*{2}{m{0.38\textwidth}}}
\begin{center}
\textcolor{NavyBlue}{\Large Different}
\end{center}
&
\begin{center}
\includegraphics[height=6.5cm]{../Comparisons/Vectors/inertia_tensor_of_3NFAACb_and_3NFAACc.png}
\end{center}
\end{tabular}

 \newpage

\vtab[-3cm]
\begin{center}
{\large FireTest \tab Número 226}
\end{center}
\begin{multicols}{2}
\begin{center}

Molecule A \
3NFAACb

\includegraphics[width=6cm]{../Comparisons/ImagesFromVMD/3NFAACb.png}

Inertia Tensor - Molecule A \\
\begin{tabular}{|c c c|}
530.265	 & 	-1.20883	 & 	-7.84653	 \\
-1.20883	 & 	1361.05	 & 	-3.78865	 \\
-7.84653	 & 	-3.78865	 & 	915.61
\end{tabular}

\vtab
 EingenVectors - Molecule A     \\
\begin{tabular}{|c c c|}
-0.999791	 & 	-0.00154731	 & 	-0.0203648	 \\
-0.0203771	 & 	0.00845054	 & 	0.999757	 \\
-0.00137484	 & 	0.999963	 & 	-0.00848031
\end{tabular}

\vtab
 EingenValues - Molecule A     \\
\begin{tabular}{|c c c|}
530.103	 & 	915.738	 & 	1361.08	 \\
\end{tabular}
\columnbreak

Molecule B \
3NFAACd

\includegraphics[width=6cm]{../Comparisons/ImagesFromVMD/3NFAACd.png}

Inertia Tensor - Molecule B \\
\begin{tabular}{|c c c|}
524.186	 & 	-1.32648	 & 	-2.36411	 \\
-1.32648	 & 	935.358	 & 	-2.91274	 \\
-2.36411	 & 	-2.91274	 & 	1359.16
\end{tabular}

\vtab
 EingenVectors - Molecule B     \\
\begin{tabular}{|c c c|}
-0.999991	 & 	-0.00324612	 & 	-0.00284261	 \\
0.00326554	 & 	-0.999971	 & 	-0.0068542	 \\
-0.00282028	 & 	-0.00686342	 & 	0.999972
\end{tabular}

\vtab
 EingenValues - Molecule B     \\
\begin{tabular}{|c c c|}
524.175	 & 	935.342	 & 	1359.19	 \\
\end{tabular}

\end{center}
\end{multicols}

\vtab[-5mm]
\begin{tabular}{*{2}{m{0.38\textwidth}}}
\begin{center}
\textcolor{NavyBlue}{\Large Different}
\end{center}
&
\begin{center}
\includegraphics[height=6.5cm]{../Comparisons/Vectors/inertia_tensor_of_3NFAACb_and_3NFAACd.png}
\end{center}
\end{tabular}

 \newpage

\vtab[-3cm]
\begin{center}
{\large FireTest \tab Número 227}
\end{center}
\begin{multicols}{2}
\begin{center}

Molecule A \
3NFAACb

\includegraphics[width=6cm]{../Comparisons/ImagesFromVMD/3NFAACb.png}

Inertia Tensor - Molecule A \\
\begin{tabular}{|c c c|}
530.265	 & 	-1.20883	 & 	-7.84653	 \\
-1.20883	 & 	1361.05	 & 	-3.78865	 \\
-7.84653	 & 	-3.78865	 & 	915.61
\end{tabular}

\vtab
 EingenVectors - Molecule A     \\
\begin{tabular}{|c c c|}
-0.999791	 & 	-0.00154731	 & 	-0.0203648	 \\
-0.0203771	 & 	0.00845054	 & 	0.999757	 \\
-0.00137484	 & 	0.999963	 & 	-0.00848031
\end{tabular}

\vtab
 EingenValues - Molecule A     \\
\begin{tabular}{|c c c|}
530.103	 & 	915.738	 & 	1361.08	 \\
\end{tabular}
\columnbreak

Molecule B \
3NFAACe

\includegraphics[width=6cm]{../Comparisons/ImagesFromVMD/3NFAACe.png}

Inertia Tensor - Molecule B \\
\begin{tabular}{|c c c|}
524.508	 & 	15.8785	 & 	-4.02001	 \\
15.8785	 & 	1358.65	 & 	10.8723	 \\
-4.02001	 & 	10.8723	 & 	935.54
\end{tabular}

\vtab
 EingenVectors - Molecule B     \\
\begin{tabular}{|c c c|}
-0.999764	 & 	0.0191573	 & 	-0.0102761	 \\
0.0107588	 & 	0.025269	 & 	-0.999623	 \\
0.0188904	 & 	0.999497	 & 	0.0254692
\end{tabular}

\vtab
 EingenValues - Molecule B     \\
\begin{tabular}{|c c c|}
524.162	 & 	935.308	 & 	1359.22	 \\
\end{tabular}

\end{center}
\end{multicols}

\vtab[-5mm]
\begin{tabular}{*{2}{m{0.38\textwidth}}}
\begin{center}
\textcolor{NavyBlue}{\Large Different}
\end{center}
&
\begin{center}
\includegraphics[height=6.5cm]{../Comparisons/Vectors/inertia_tensor_of_3NFAACb_and_3NFAACe.png}
\end{center}
\end{tabular}

 \newpage

\vtab[-3cm]
\begin{center}
{\large FireTest \tab Número 228}
\end{center}
\begin{multicols}{2}
\begin{center}

Molecule A \
3NFAACb

\includegraphics[width=6cm]{../Comparisons/ImagesFromVMD/3NFAACb.png}

Inertia Tensor - Molecule A \\
\begin{tabular}{|c c c|}
530.265	 & 	-1.20883	 & 	-7.84653	 \\
-1.20883	 & 	1361.05	 & 	-3.78865	 \\
-7.84653	 & 	-3.78865	 & 	915.61
\end{tabular}

\vtab
 EingenVectors - Molecule A     \\
\begin{tabular}{|c c c|}
-0.999791	 & 	-0.00154731	 & 	-0.0203648	 \\
-0.0203771	 & 	0.00845054	 & 	0.999757	 \\
-0.00137484	 & 	0.999963	 & 	-0.00848031
\end{tabular}

\vtab
 EingenValues - Molecule A     \\
\begin{tabular}{|c c c|}
530.103	 & 	915.738	 & 	1361.08	 \\
\end{tabular}
\columnbreak

Molecule B \
3NFAACf

\includegraphics[width=6cm]{../Comparisons/ImagesFromVMD/3NFAACf.png}

Inertia Tensor - Molecule B \\
\begin{tabular}{|c c c|}
530.208	 & 	-1.63273	 & 	-7.84857	 \\
-1.63273	 & 	1360.76	 & 	-3.6411	 \\
-7.84857	 & 	-3.6411	 & 	915.396
\end{tabular}

\vtab
 EingenVectors - Molecule B     \\
\begin{tabular}{|c c c|}
-0.99979	 & 	-0.00205437	 & 	-0.0203824	 \\
-0.0203985	 & 	0.00810116	 & 	0.999759	 \\
-0.00188875	 & 	0.999965	 & 	-0.00814137
\end{tabular}

\vtab
 EingenValues - Molecule B     \\
\begin{tabular}{|c c c|}
530.045	 & 	915.527	 & 	1360.79	 \\
\end{tabular}

\end{center}
\end{multicols}

\vtab[-5mm]
\begin{tabular}{*{2}{m{0.38\textwidth}}}
\begin{center}
\textcolor{NavyBlue}{\Large Equal}
\end{center}
&
\begin{center}
\includegraphics[height=6.5cm]{../Comparisons/Vectors/inertia_tensor_of_3NFAACb_and_3NFAACf.png}
\end{center}
\end{tabular}

 \newpage

\vtab[-3cm]
\begin{center}
{\large FireTest \tab Número 229}
\end{center}
\begin{multicols}{2}
\begin{center}

Molecule A \
3NFAACb

\includegraphics[width=6cm]{../Comparisons/ImagesFromVMD/3NFAACb.png}

Inertia Tensor - Molecule A \\
\begin{tabular}{|c c c|}
530.265	 & 	-1.20883	 & 	-7.84653	 \\
-1.20883	 & 	1361.05	 & 	-3.78865	 \\
-7.84653	 & 	-3.78865	 & 	915.61
\end{tabular}

\vtab
 EingenVectors - Molecule A     \\
\begin{tabular}{|c c c|}
-0.999791	 & 	-0.00154731	 & 	-0.0203648	 \\
-0.0203771	 & 	0.00845054	 & 	0.999757	 \\
-0.00137484	 & 	0.999963	 & 	-0.00848031
\end{tabular}

\vtab
 EingenValues - Molecule A     \\
\begin{tabular}{|c c c|}
530.103	 & 	915.738	 & 	1361.08	 \\
\end{tabular}
\columnbreak

Molecule B \
3NFAACg

\includegraphics[width=6cm]{../Comparisons/ImagesFromVMD/3NFAACg.png}

Inertia Tensor - Molecule B \\
\begin{tabular}{|c c c|}
532.891	 & 	-0.526938	 & 	-0.504713	 \\
-0.526938	 & 	918.454	 & 	2.35809	 \\
-0.504713	 & 	2.35809	 & 	1356.91
\end{tabular}

\vtab
 EingenVectors - Molecule B     \\
\begin{tabular}{|c c c|}
-0.999999	 & 	-0.00136295	 & 	-0.000608598	 \\
0.00135965	 & 	-0.999985	 & 	0.00537946	 \\
-0.000615921	 & 	0.00537863	 & 	0.999985
\end{tabular}

\vtab
 EingenValues - Molecule B     \\
\begin{tabular}{|c c c|}
532.89	 & 	918.442	 & 	1356.93	 \\
\end{tabular}

\end{center}
\end{multicols}

\vtab[-5mm]
\begin{tabular}{*{2}{m{0.38\textwidth}}}
\begin{center}
\textcolor{NavyBlue}{\Large Different}
\end{center}
&
\begin{center}
\includegraphics[height=6.5cm]{../Comparisons/Vectors/inertia_tensor_of_3NFAACb_and_3NFAACg.png}
\end{center}
\end{tabular}

 \newpage

\vtab[-3cm]
\begin{center}
{\large FireTest \tab Número 230}
\end{center}
\begin{multicols}{2}
\begin{center}

Molecule A \
3NFAACb

\includegraphics[width=6cm]{../Comparisons/ImagesFromVMD/3NFAACb.png}

Inertia Tensor - Molecule A \\
\begin{tabular}{|c c c|}
530.265	 & 	-1.20883	 & 	-7.84653	 \\
-1.20883	 & 	1361.05	 & 	-3.78865	 \\
-7.84653	 & 	-3.78865	 & 	915.61
\end{tabular}

\vtab
 EingenVectors - Molecule A     \\
\begin{tabular}{|c c c|}
-0.999791	 & 	-0.00154731	 & 	-0.0203648	 \\
-0.0203771	 & 	0.00845054	 & 	0.999757	 \\
-0.00137484	 & 	0.999963	 & 	-0.00848031
\end{tabular}

\vtab
 EingenValues - Molecule A     \\
\begin{tabular}{|c c c|}
530.103	 & 	915.738	 & 	1361.08	 \\
\end{tabular}
\columnbreak

Molecule B \
3NFAACh

\includegraphics[width=6cm]{../Comparisons/ImagesFromVMD/3NFAACh.png}

Inertia Tensor - Molecule B \\
\begin{tabular}{|c c c|}
528.718	 & 	-3.89066	 & 	-3.66882	 \\
-3.89066	 & 	924.911	 & 	2.93817	 \\
-3.66882	 & 	2.93817	 & 	1336.08
\end{tabular}

\vtab
 EingenVectors - Molecule B     \\
\begin{tabular}{|c c c|}
0.999942	 & 	0.00978475	 & 	0.00450803	 \\
0.00975199	 & 	-0.999926	 & 	0.00723269	 \\
-0.00457847	 & 	0.0071883	 & 	0.999964
\end{tabular}

\vtab
 EingenValues - Molecule B     \\
\begin{tabular}{|c c c|}
528.663	 & 	924.928	 & 	1336.12	 \\
\end{tabular}

\end{center}
\end{multicols}

\vtab[-5mm]
\begin{tabular}{*{2}{m{0.38\textwidth}}}
\begin{center}
\textcolor{NavyBlue}{\Large Different}
\end{center}
&
\begin{center}
\includegraphics[height=6.5cm]{../Comparisons/Vectors/inertia_tensor_of_3NFAACb_and_3NFAACh.png}
\end{center}
\end{tabular}

 \newpage

\vtab[-3cm]
\begin{center}
{\large FireTest \tab Número 231}
\end{center}
\begin{multicols}{2}
\begin{center}

Molecule A \
3NFAACb

\includegraphics[width=6cm]{../Comparisons/ImagesFromVMD/3NFAACb.png}

Inertia Tensor - Molecule A \\
\begin{tabular}{|c c c|}
530.265	 & 	-1.20883	 & 	-7.84653	 \\
-1.20883	 & 	1361.05	 & 	-3.78865	 \\
-7.84653	 & 	-3.78865	 & 	915.61
\end{tabular}

\vtab
 EingenVectors - Molecule A     \\
\begin{tabular}{|c c c|}
-0.999791	 & 	-0.00154731	 & 	-0.0203648	 \\
-0.0203771	 & 	0.00845054	 & 	0.999757	 \\
-0.00137484	 & 	0.999963	 & 	-0.00848031
\end{tabular}

\vtab
 EingenValues - Molecule A     \\
\begin{tabular}{|c c c|}
530.103	 & 	915.738	 & 	1361.08	 \\
\end{tabular}
\columnbreak

Molecule B \
3NFAACi

\includegraphics[width=6cm]{../Comparisons/ImagesFromVMD/3NFAACi.png}

Inertia Tensor - Molecule B \\
\begin{tabular}{|c c c|}
521.487	 & 	-1.22943	 & 	-2.71042	 \\
-1.22943	 & 	949.663	 & 	1.67006	 \\
-2.71042	 & 	1.67006	 & 	1346.16
\end{tabular}

\vtab
 EingenVectors - Molecule B     \\
\begin{tabular}{|c c c|}
0.999991	 & 	0.00285841	 & 	0.00328077	 \\
0.00284452	 & 	-0.999987	 & 	0.00423133	 \\
-0.00329282	 & 	0.00422196	 & 	0.999986
\end{tabular}

\vtab
 EingenValues - Molecule B     \\
\begin{tabular}{|c c c|}
521.475	 & 	949.659	 & 	1346.18	 \\
\end{tabular}

\end{center}
\end{multicols}

\vtab[-5mm]
\begin{tabular}{*{2}{m{0.38\textwidth}}}
\begin{center}
\textcolor{NavyBlue}{\Large Different}
\end{center}
&
\begin{center}
\includegraphics[height=6.5cm]{../Comparisons/Vectors/inertia_tensor_of_3NFAACb_and_3NFAACi.png}
\end{center}
\end{tabular}

 \newpage

\vtab[-3cm]
\begin{center}
{\large FireTest \tab Número 232}
\end{center}
\begin{multicols}{2}
\begin{center}

Molecule A \
3NFAACb

\includegraphics[width=6cm]{../Comparisons/ImagesFromVMD/3NFAACb.png}

Inertia Tensor - Molecule A \\
\begin{tabular}{|c c c|}
530.265	 & 	-1.20883	 & 	-7.84653	 \\
-1.20883	 & 	1361.05	 & 	-3.78865	 \\
-7.84653	 & 	-3.78865	 & 	915.61
\end{tabular}

\vtab
 EingenVectors - Molecule A     \\
\begin{tabular}{|c c c|}
-0.999791	 & 	-0.00154731	 & 	-0.0203648	 \\
-0.0203771	 & 	0.00845054	 & 	0.999757	 \\
-0.00137484	 & 	0.999963	 & 	-0.00848031
\end{tabular}

\vtab
 EingenValues - Molecule A     \\
\begin{tabular}{|c c c|}
530.103	 & 	915.738	 & 	1361.08	 \\
\end{tabular}
\columnbreak

Molecule B \
3NFAACj

\includegraphics[width=6cm]{../Comparisons/ImagesFromVMD/3NFAACj.png}

Inertia Tensor - Molecule B \\
\begin{tabular}{|c c c|}
533.789	 & 	-4.75521	 & 	-1.91525	 \\
-4.75521	 & 	920.091	 & 	2.28449	 \\
-1.91525	 & 	2.28449	 & 	1348.28
\end{tabular}

\vtab
 EingenVectors - Molecule B     \\
\begin{tabular}{|c c c|}
-0.999922	 & 	-0.0122929	 & 	-0.00231663	 \\
0.0122803	 & 	-0.99991	 & 	0.00539026	 \\
-0.00238268	 & 	0.00536139	 & 	0.999983
\end{tabular}

\vtab
 EingenValues - Molecule B     \\
\begin{tabular}{|c c c|}
533.726	 & 	920.137	 & 	1348.3	 \\
\end{tabular}

\end{center}
\end{multicols}

\vtab[-5mm]
\begin{tabular}{*{2}{m{0.38\textwidth}}}
\begin{center}
\textcolor{NavyBlue}{\Large Different}
\end{center}
&
\begin{center}
\includegraphics[height=6.5cm]{../Comparisons/Vectors/inertia_tensor_of_3NFAACb_and_3NFAACj.png}
\end{center}
\end{tabular}

 \newpage

\vtab[-3cm]
\begin{center}
{\large FireTest \tab Número 233}
\end{center}
\begin{multicols}{2}
\begin{center}

Molecule A \
3NFAACb

\includegraphics[width=6cm]{../Comparisons/ImagesFromVMD/3NFAACb.png}

Inertia Tensor - Molecule A \\
\begin{tabular}{|c c c|}
530.265	 & 	-1.20883	 & 	-7.84653	 \\
-1.20883	 & 	1361.05	 & 	-3.78865	 \\
-7.84653	 & 	-3.78865	 & 	915.61
\end{tabular}

\vtab
 EingenVectors - Molecule A     \\
\begin{tabular}{|c c c|}
-0.999791	 & 	-0.00154731	 & 	-0.0203648	 \\
-0.0203771	 & 	0.00845054	 & 	0.999757	 \\
-0.00137484	 & 	0.999963	 & 	-0.00848031
\end{tabular}

\vtab
 EingenValues - Molecule A     \\
\begin{tabular}{|c c c|}
530.103	 & 	915.738	 & 	1361.08	 \\
\end{tabular}
\columnbreak

Molecule B \
3NFAACk

\includegraphics[width=6cm]{../Comparisons/ImagesFromVMD/3NFAACk.png}

Inertia Tensor - Molecule B \\
\begin{tabular}{|c c c|}
534.899	 & 	-5.81418	 & 	-0.270063	 \\
-5.81418	 & 	913.263	 & 	2.4519	 \\
-0.270063	 & 	2.4519	 & 	1353.77
\end{tabular}

\vtab
 EingenVectors - Molecule B     \\
\begin{tabular}{|c c c|}
-0.999882	 & 	-0.0153593	 & 	-0.00028374	 \\
0.0153575	 & 	-0.999867	 & 	0.00557573	 \\
-0.000369342	 & 	0.00557072	 & 	0.999984
\end{tabular}

\vtab
 EingenValues - Molecule B     \\
\begin{tabular}{|c c c|}
534.81	 & 	913.339	 & 	1353.78	 \\
\end{tabular}

\end{center}
\end{multicols}

\vtab[-5mm]
\begin{tabular}{*{2}{m{0.38\textwidth}}}
\begin{center}
\textcolor{NavyBlue}{\Large Different}
\end{center}
&
\begin{center}
\includegraphics[height=6.5cm]{../Comparisons/Vectors/inertia_tensor_of_3NFAACb_and_3NFAACk.png}
\end{center}
\end{tabular}

 \newpage

\vtab[-3cm]
\begin{center}
{\large FireTest \tab Número 234}
\end{center}
\begin{multicols}{2}
\begin{center}

Molecule A \
3NFAACb

\includegraphics[width=6cm]{../Comparisons/ImagesFromVMD/3NFAACb.png}

Inertia Tensor - Molecule A \\
\begin{tabular}{|c c c|}
530.265	 & 	-1.20883	 & 	-7.84653	 \\
-1.20883	 & 	1361.05	 & 	-3.78865	 \\
-7.84653	 & 	-3.78865	 & 	915.61
\end{tabular}

\vtab
 EingenVectors - Molecule A     \\
\begin{tabular}{|c c c|}
-0.999791	 & 	-0.00154731	 & 	-0.0203648	 \\
-0.0203771	 & 	0.00845054	 & 	0.999757	 \\
-0.00137484	 & 	0.999963	 & 	-0.00848031
\end{tabular}

\vtab
 EingenValues - Molecule A     \\
\begin{tabular}{|c c c|}
530.103	 & 	915.738	 & 	1361.08	 \\
\end{tabular}
\columnbreak

Molecule B \
3NFAACl

\includegraphics[width=6cm]{../Comparisons/ImagesFromVMD/3NFAACl.png}

Inertia Tensor - Molecule B \\
\begin{tabular}{|c c c|}
531.723	 & 	3.03424	 & 	2.73426	 \\
3.03424	 & 	929.418	 & 	-1.84284	 \\
2.73426	 & 	-1.84284	 & 	1355.39
\end{tabular}

\vtab
 EingenVectors - Molecule B     \\
\begin{tabular}{|c c c|}
0.999965	 & 	-0.00764413	 & 	-0.00333648	 \\
-0.00765838	 & 	-0.999962	 & 	-0.00427708	 \\
0.00330366	 & 	-0.00430248	 & 	0.999985
\end{tabular}

\vtab
 EingenValues - Molecule B     \\
\begin{tabular}{|c c c|}
531.691	 & 	929.434	 & 	1355.4	 \\
\end{tabular}

\end{center}
\end{multicols}

\vtab[-5mm]
\begin{tabular}{*{2}{m{0.38\textwidth}}}
\begin{center}
\textcolor{NavyBlue}{\Large Different}
\end{center}
&
\begin{center}
\includegraphics[height=6.5cm]{../Comparisons/Vectors/inertia_tensor_of_3NFAACb_and_3NFAACl.png}
\end{center}
\end{tabular}

 \newpage

\vtab[-3cm]
\begin{center}
{\large FireTest \tab Número 235}
\end{center}
\begin{multicols}{2}
\begin{center}

Molecule A \
3NFAACb

\includegraphics[width=6cm]{../Comparisons/ImagesFromVMD/3NFAACb.png}

Inertia Tensor - Molecule A \\
\begin{tabular}{|c c c|}
530.265	 & 	-1.20883	 & 	-7.84653	 \\
-1.20883	 & 	1361.05	 & 	-3.78865	 \\
-7.84653	 & 	-3.78865	 & 	915.61
\end{tabular}

\vtab
 EingenVectors - Molecule A     \\
\begin{tabular}{|c c c|}
-0.999791	 & 	-0.00154731	 & 	-0.0203648	 \\
-0.0203771	 & 	0.00845054	 & 	0.999757	 \\
-0.00137484	 & 	0.999963	 & 	-0.00848031
\end{tabular}

\vtab
 EingenValues - Molecule A     \\
\begin{tabular}{|c c c|}
530.103	 & 	915.738	 & 	1361.08	 \\
\end{tabular}
\columnbreak

Molecule B \
3NFAACm

\includegraphics[width=6cm]{../Comparisons/ImagesFromVMD/3NFAACm.png}

Inertia Tensor - Molecule B \\
\begin{tabular}{|c c c|}
532.546	 & 	13.7854	 & 	-15.4626	 \\
13.7854	 & 	1354.87	 & 	11.5786	 \\
-15.4626	 & 	11.5786	 & 	929.101
\end{tabular}

\vtab
 EingenVectors - Molecule B     \\
\begin{tabular}{|c c c|}
-0.999075	 & 	0.0172851	 & 	-0.0393769	 \\
0.0398168	 & 	0.0258942	 & 	-0.998871	 \\
-0.016246	 & 	-0.999515	 & 	-0.0265584
\end{tabular}

\vtab
 EingenValues - Molecule B     \\
\begin{tabular}{|c c c|}
531.698	 & 	929.417	 & 	1355.4	 \\
\end{tabular}

\end{center}
\end{multicols}

\vtab[-5mm]
\begin{tabular}{*{2}{m{0.38\textwidth}}}
\begin{center}
\textcolor{NavyBlue}{\Large Different}
\end{center}
&
\begin{center}
\includegraphics[height=6.5cm]{../Comparisons/Vectors/inertia_tensor_of_3NFAACb_and_3NFAACm.png}
\end{center}
\end{tabular}

 \newpage

\vtab[-3cm]
\begin{center}
{\large FireTest \tab Número 236}
\end{center}
\begin{multicols}{2}
\begin{center}

Molecule A \
3NFAACb

\includegraphics[width=6cm]{../Comparisons/ImagesFromVMD/3NFAACb.png}

Inertia Tensor - Molecule A \\
\begin{tabular}{|c c c|}
530.265	 & 	-1.20883	 & 	-7.84653	 \\
-1.20883	 & 	1361.05	 & 	-3.78865	 \\
-7.84653	 & 	-3.78865	 & 	915.61
\end{tabular}

\vtab
 EingenVectors - Molecule A     \\
\begin{tabular}{|c c c|}
-0.999791	 & 	-0.00154731	 & 	-0.0203648	 \\
-0.0203771	 & 	0.00845054	 & 	0.999757	 \\
-0.00137484	 & 	0.999963	 & 	-0.00848031
\end{tabular}

\vtab
 EingenValues - Molecule A     \\
\begin{tabular}{|c c c|}
530.103	 & 	915.738	 & 	1361.08	 \\
\end{tabular}
\columnbreak

Molecule B \
3NFAACn

\includegraphics[width=6cm]{../Comparisons/ImagesFromVMD/3NFAACn.png}

Inertia Tensor - Molecule B \\
\begin{tabular}{|c c c|}
531.896	 & 	3.78027	 & 	-13.1151	 \\
3.78027	 & 	1353.2	 & 	-7.47403	 \\
-13.1151	 & 	-7.47403	 & 	912.989
\end{tabular}

\vtab
 EingenVectors - Molecule B     \\
\begin{tabular}{|c c c|}
0.999403	 & 	-0.00428573	 & 	0.0342679	 \\
0.0341891	 & 	-0.0172718	 & 	-0.999266	 \\
0.00487445	 & 	0.999842	 & 	-0.017115
\end{tabular}

\vtab
 EingenValues - Molecule B     \\
\begin{tabular}{|c c c|}
531.43	 & 	913.309	 & 	1353.35	 \\
\end{tabular}

\end{center}
\end{multicols}

\vtab[-5mm]
\begin{tabular}{*{2}{m{0.38\textwidth}}}
\begin{center}
\textcolor{NavyBlue}{\Large Different}
\end{center}
&
\begin{center}
\includegraphics[height=6.5cm]{../Comparisons/Vectors/inertia_tensor_of_3NFAACb_and_3NFAACn.png}
\end{center}
\end{tabular}

 \newpage

\vtab[-3cm]
\begin{center}
{\large FireTest \tab Número 237}
\end{center}
\begin{multicols}{2}
\begin{center}

Molecule A \
3NFAACb

\includegraphics[width=6cm]{../Comparisons/ImagesFromVMD/3NFAACb.png}

Inertia Tensor - Molecule A \\
\begin{tabular}{|c c c|}
530.265	 & 	-1.20883	 & 	-7.84653	 \\
-1.20883	 & 	1361.05	 & 	-3.78865	 \\
-7.84653	 & 	-3.78865	 & 	915.61
\end{tabular}

\vtab
 EingenVectors - Molecule A     \\
\begin{tabular}{|c c c|}
-0.999791	 & 	-0.00154731	 & 	-0.0203648	 \\
-0.0203771	 & 	0.00845054	 & 	0.999757	 \\
-0.00137484	 & 	0.999963	 & 	-0.00848031
\end{tabular}

\vtab
 EingenValues - Molecule A     \\
\begin{tabular}{|c c c|}
530.103	 & 	915.738	 & 	1361.08	 \\
\end{tabular}
\columnbreak

Molecule B \
4NFAACa

\includegraphics[width=6cm]{../Comparisons/ImagesFromVMD/4NFAACa.png}

Inertia Tensor - Molecule B \\
\begin{tabular}{|c c c|}
479.392	 & 	3.27131	 & 	4.22557	 \\
3.27131	 & 	1242.39	 & 	-0.852684	 \\
4.22557	 & 	-0.852684	 & 	1647.37
\end{tabular}

\vtab
 EingenVectors - Molecule B     \\
\begin{tabular}{|c c c|}
0.999984	 & 	-0.00429123	 & 	-0.00362083	 \\
-0.00429871	 & 	-0.999989	 & 	-0.0020607	 \\
0.00361195	 & 	-0.00207623	 & 	0.999991
\end{tabular}

\vtab
 EingenValues - Molecule B     \\
\begin{tabular}{|c c c|}
479.363	 & 	1242.41	 & 	1647.39	 \\
\end{tabular}

\end{center}
\end{multicols}

\vtab[-5mm]
\begin{tabular}{*{2}{m{0.38\textwidth}}}
\begin{center}
\textcolor{NavyBlue}{\Large Different}
\end{center}
&
\begin{center}
\includegraphics[height=6.5cm]{../Comparisons/Vectors/inertia_tensor_of_3NFAACb_and_4NFAACa.png}
\end{center}
\end{tabular}

 \newpage

\vtab[-3cm]
\begin{center}
{\large FireTest \tab Número 238}
\end{center}
\begin{multicols}{2}
\begin{center}

Molecule A \
3NFAACb

\includegraphics[width=6cm]{../Comparisons/ImagesFromVMD/3NFAACb.png}

Inertia Tensor - Molecule A \\
\begin{tabular}{|c c c|}
530.265	 & 	-1.20883	 & 	-7.84653	 \\
-1.20883	 & 	1361.05	 & 	-3.78865	 \\
-7.84653	 & 	-3.78865	 & 	915.61
\end{tabular}

\vtab
 EingenVectors - Molecule A     \\
\begin{tabular}{|c c c|}
-0.999791	 & 	-0.00154731	 & 	-0.0203648	 \\
-0.0203771	 & 	0.00845054	 & 	0.999757	 \\
-0.00137484	 & 	0.999963	 & 	-0.00848031
\end{tabular}

\vtab
 EingenValues - Molecule A     \\
\begin{tabular}{|c c c|}
530.103	 & 	915.738	 & 	1361.08	 \\
\end{tabular}
\columnbreak

Molecule B \
4NFAACb

\includegraphics[width=6cm]{../Comparisons/ImagesFromVMD/4NFAACb.png}

Inertia Tensor - Molecule B \\
\begin{tabular}{|c c c|}
479.338	 & 	3.27331	 & 	-4.22553	 \\
3.27331	 & 	1242.4	 & 	0.852083	 \\
-4.22553	 & 	0.852083	 & 	1647.3
\end{tabular}

\vtab
 EingenVectors - Molecule B     \\
\begin{tabular}{|c c c|}
0.999984	 & 	-0.00429353	 & 	0.00362086	 \\
-0.004301	 & 	-0.999989	 & 	0.00205959	 \\
-0.00361198	 & 	0.00207513	 & 	0.999991
\end{tabular}

\vtab
 EingenValues - Molecule B     \\
\begin{tabular}{|c c c|}
479.308	 & 	1242.41	 & 	1647.31	 \\
\end{tabular}

\end{center}
\end{multicols}

\vtab[-5mm]
\begin{tabular}{*{2}{m{0.38\textwidth}}}
\begin{center}
\textcolor{NavyBlue}{\Large Different}
\end{center}
&
\begin{center}
\includegraphics[height=6.5cm]{../Comparisons/Vectors/inertia_tensor_of_3NFAACb_and_4NFAACb.png}
\end{center}
\end{tabular}

 \newpage

\vtab[-3cm]
\begin{center}
{\large FireTest \tab Número 239}
\end{center}
\begin{multicols}{2}
\begin{center}

Molecule A \
3NFAACb

\includegraphics[width=6cm]{../Comparisons/ImagesFromVMD/3NFAACb.png}

Inertia Tensor - Molecule A \\
\begin{tabular}{|c c c|}
530.265	 & 	-1.20883	 & 	-7.84653	 \\
-1.20883	 & 	1361.05	 & 	-3.78865	 \\
-7.84653	 & 	-3.78865	 & 	915.61
\end{tabular}

\vtab
 EingenVectors - Molecule A     \\
\begin{tabular}{|c c c|}
-0.999791	 & 	-0.00154731	 & 	-0.0203648	 \\
-0.0203771	 & 	0.00845054	 & 	0.999757	 \\
-0.00137484	 & 	0.999963	 & 	-0.00848031
\end{tabular}

\vtab
 EingenValues - Molecule A     \\
\begin{tabular}{|c c c|}
530.103	 & 	915.738	 & 	1361.08	 \\
\end{tabular}
\columnbreak

Molecule B \
4NFAACc

\includegraphics[width=6cm]{../Comparisons/ImagesFromVMD/4NFAACc.png}

Inertia Tensor - Molecule B \\
\begin{tabular}{|c c c|}
482.067	 & 	-5.39474	 & 	-1.35857	 \\
-5.39474	 & 	1240.3	 & 	-2.54035	 \\
-1.35857	 & 	-2.54035	 & 	1647.06
\end{tabular}

\vtab
 EingenVectors - Molecule B     \\
\begin{tabular}{|c c c|}
-0.999974	 & 	-0.00711826	 & 	-0.0011816	 \\
0.00712547	 & 	-0.999955	 & 	-0.00622156	 \\
-0.00113726	 & 	-0.00622982	 & 	0.99998
\end{tabular}

\vtab
 EingenValues - Molecule B     \\
\begin{tabular}{|c c c|}
482.027	 & 	1240.32	 & 	1647.08	 \\
\end{tabular}

\end{center}
\end{multicols}

\vtab[-5mm]
\begin{tabular}{*{2}{m{0.38\textwidth}}}
\begin{center}
\textcolor{NavyBlue}{\Large Different}
\end{center}
&
\begin{center}
\includegraphics[height=6.5cm]{../Comparisons/Vectors/inertia_tensor_of_3NFAACb_and_4NFAACc.png}
\end{center}
\end{tabular}

 \newpage

\vtab[-3cm]
\begin{center}
{\large FireTest \tab Número 240}
\end{center}
\begin{multicols}{2}
\begin{center}

Molecule A \
3NFAACb

\includegraphics[width=6cm]{../Comparisons/ImagesFromVMD/3NFAACb.png}

Inertia Tensor - Molecule A \\
\begin{tabular}{|c c c|}
530.265	 & 	-1.20883	 & 	-7.84653	 \\
-1.20883	 & 	1361.05	 & 	-3.78865	 \\
-7.84653	 & 	-3.78865	 & 	915.61
\end{tabular}

\vtab
 EingenVectors - Molecule A     \\
\begin{tabular}{|c c c|}
-0.999791	 & 	-0.00154731	 & 	-0.0203648	 \\
-0.0203771	 & 	0.00845054	 & 	0.999757	 \\
-0.00137484	 & 	0.999963	 & 	-0.00848031
\end{tabular}

\vtab
 EingenValues - Molecule A     \\
\begin{tabular}{|c c c|}
530.103	 & 	915.738	 & 	1361.08	 \\
\end{tabular}
\columnbreak

Molecule B \
4NFAACd

\includegraphics[width=6cm]{../Comparisons/ImagesFromVMD/4NFAACd.png}

Inertia Tensor - Molecule B \\
\begin{tabular}{|c c c|}
491.672	 & 	0.24486	 & 	-3.10016	 \\
0.24486	 & 	1231.15	 & 	2.19965	 \\
-3.10016	 & 	2.19965	 & 	1650.11
\end{tabular}

\vtab
 EingenVectors - Molecule B     \\
\begin{tabular}{|c c c|}
0.999996	 & 	-0.000339081	 & 	0.00267677	 \\
-0.000353124	 & 	-0.999986	 & 	0.00524747	 \\
-0.00267495	 & 	0.0052484	 & 	0.999983
\end{tabular}

\vtab
 EingenValues - Molecule B     \\
\begin{tabular}{|c c c|}
491.663	 & 	1231.14	 & 	1650.13	 \\
\end{tabular}

\end{center}
\end{multicols}

\vtab[-5mm]
\begin{tabular}{*{2}{m{0.38\textwidth}}}
\begin{center}
\textcolor{NavyBlue}{\Large Different}
\end{center}
&
\begin{center}
\includegraphics[height=6.5cm]{../Comparisons/Vectors/inertia_tensor_of_3NFAACb_and_4NFAACd.png}
\end{center}
\end{tabular}

 \newpage

\vtab[-3cm]
\begin{center}
{\large FireTest \tab Número 241}
\end{center}
\begin{multicols}{2}
\begin{center}

Molecule A \
3NFAACb

\includegraphics[width=6cm]{../Comparisons/ImagesFromVMD/3NFAACb.png}

Inertia Tensor - Molecule A \\
\begin{tabular}{|c c c|}
530.265	 & 	-1.20883	 & 	-7.84653	 \\
-1.20883	 & 	1361.05	 & 	-3.78865	 \\
-7.84653	 & 	-3.78865	 & 	915.61
\end{tabular}

\vtab
 EingenVectors - Molecule A     \\
\begin{tabular}{|c c c|}
-0.999791	 & 	-0.00154731	 & 	-0.0203648	 \\
-0.0203771	 & 	0.00845054	 & 	0.999757	 \\
-0.00137484	 & 	0.999963	 & 	-0.00848031
\end{tabular}

\vtab
 EingenValues - Molecule A     \\
\begin{tabular}{|c c c|}
530.103	 & 	915.738	 & 	1361.08	 \\
\end{tabular}
\columnbreak

Molecule B \
4NFAACe

\includegraphics[width=6cm]{../Comparisons/ImagesFromVMD/4NFAACe.png}

Inertia Tensor - Molecule B \\
\begin{tabular}{|c c c|}
489.025	 & 	-0.430035	 & 	3.98876	 \\
-0.430035	 & 	1233.71	 & 	-2.06505	 \\
3.98876	 & 	-2.06505	 & 	1641.79
\end{tabular}

\vtab
 EingenVectors - Molecule B     \\
\begin{tabular}{|c c c|}
0.999994	 & 	0.000567863	 & 	-0.00345908	 \\
0.000550336	 & 	-0.999987	 & 	-0.00506565	 \\
0.00346192	 & 	-0.00506372	 & 	0.999981
\end{tabular}

\vtab
 EingenValues - Molecule B     \\
\begin{tabular}{|c c c|}
489.011	 & 	1233.7	 & 	1641.81	 \\
\end{tabular}

\end{center}
\end{multicols}

\vtab[-5mm]
\begin{tabular}{*{2}{m{0.38\textwidth}}}
\begin{center}
\textcolor{NavyBlue}{\Large Different}
\end{center}
&
\begin{center}
\includegraphics[height=6.5cm]{../Comparisons/Vectors/inertia_tensor_of_3NFAACb_and_4NFAACe.png}
\end{center}
\end{tabular}

 \newpage

\vtab[-3cm]
\begin{center}
{\large FireTest \tab Número 242}
\end{center}
\begin{multicols}{2}
\begin{center}

Molecule A \
3NFAACb

\includegraphics[width=6cm]{../Comparisons/ImagesFromVMD/3NFAACb.png}

Inertia Tensor - Molecule A \\
\begin{tabular}{|c c c|}
530.265	 & 	-1.20883	 & 	-7.84653	 \\
-1.20883	 & 	1361.05	 & 	-3.78865	 \\
-7.84653	 & 	-3.78865	 & 	915.61
\end{tabular}

\vtab
 EingenVectors - Molecule A     \\
\begin{tabular}{|c c c|}
-0.999791	 & 	-0.00154731	 & 	-0.0203648	 \\
-0.0203771	 & 	0.00845054	 & 	0.999757	 \\
-0.00137484	 & 	0.999963	 & 	-0.00848031
\end{tabular}

\vtab
 EingenValues - Molecule A     \\
\begin{tabular}{|c c c|}
530.103	 & 	915.738	 & 	1361.08	 \\
\end{tabular}
\columnbreak

Molecule B \
4NFAACf

\includegraphics[width=6cm]{../Comparisons/ImagesFromVMD/4NFAACf.png}

Inertia Tensor - Molecule B \\
\begin{tabular}{|c c c|}
509.683	 & 	2.80651	 & 	-1.91422	 \\
2.80651	 & 	1219.11	 & 	2.66132	 \\
-1.91422	 & 	2.66132	 & 	1681.17
\end{tabular}

\vtab
 EingenVectors - Molecule B     \\
\begin{tabular}{|c c c|}
-0.999991	 & 	0.00396206	 & 	-0.00164298	 \\
-0.00397143	 & 	-0.999976	 & 	0.0057431	 \\
-0.00162019	 & 	0.00574957	 & 	0.999982
\end{tabular}

\vtab
 EingenValues - Molecule B     \\
\begin{tabular}{|c c c|}
509.668	 & 	1219.11	 & 	1681.18	 \\
\end{tabular}

\end{center}
\end{multicols}

\vtab[-5mm]
\begin{tabular}{*{2}{m{0.38\textwidth}}}
\begin{center}
\textcolor{NavyBlue}{\Large Different}
\end{center}
&
\begin{center}
\includegraphics[height=6.5cm]{../Comparisons/Vectors/inertia_tensor_of_3NFAACb_and_4NFAACf.png}
\end{center}
\end{tabular}

 \newpage

\vtab[-3cm]
\begin{center}
{\large FireTest \tab Número 243}
\end{center}
\begin{multicols}{2}
\begin{center}

Molecule A \
3NFAACb

\includegraphics[width=6cm]{../Comparisons/ImagesFromVMD/3NFAACb.png}

Inertia Tensor - Molecule A \\
\begin{tabular}{|c c c|}
530.265	 & 	-1.20883	 & 	-7.84653	 \\
-1.20883	 & 	1361.05	 & 	-3.78865	 \\
-7.84653	 & 	-3.78865	 & 	915.61
\end{tabular}

\vtab
 EingenVectors - Molecule A     \\
\begin{tabular}{|c c c|}
-0.999791	 & 	-0.00154731	 & 	-0.0203648	 \\
-0.0203771	 & 	0.00845054	 & 	0.999757	 \\
-0.00137484	 & 	0.999963	 & 	-0.00848031
\end{tabular}

\vtab
 EingenValues - Molecule A     \\
\begin{tabular}{|c c c|}
530.103	 & 	915.738	 & 	1361.08	 \\
\end{tabular}
\columnbreak

Molecule B \
4NFAACg

\includegraphics[width=6cm]{../Comparisons/ImagesFromVMD/4NFAACg.png}

Inertia Tensor - Molecule B \\
\begin{tabular}{|c c c|}
513.78	 & 	4.51917	 & 	0.266555	 \\
4.51917	 & 	1208.04	 & 	-1.18628	 \\
0.266555	 & 	-1.18628	 & 	1700.9
\end{tabular}

\vtab
 EingenVectors - Molecule B     \\
\begin{tabular}{|c c c|}
-0.999979	 & 	0.00650929	 & 	0.000231034	 \\
-0.00650983	 & 	-0.999976	 & 	-0.00240351	 \\
0.000215383	 & 	-0.00240496	 & 	0.999997
\end{tabular}

\vtab
 EingenValues - Molecule B     \\
\begin{tabular}{|c c c|}
513.751	 & 	1208.07	 & 	1700.9	 \\
\end{tabular}

\end{center}
\end{multicols}

\vtab[-5mm]
\begin{tabular}{*{2}{m{0.38\textwidth}}}
\begin{center}
\textcolor{NavyBlue}{\Large Different}
\end{center}
&
\begin{center}
\includegraphics[height=6.5cm]{../Comparisons/Vectors/inertia_tensor_of_3NFAACb_and_4NFAACg.png}
\end{center}
\end{tabular}

 \newpage

\vtab[-3cm]
\begin{center}
{\large FireTest \tab Número 244}
\end{center}
\begin{multicols}{2}
\begin{center}

Molecule A \
3NFAACb

\includegraphics[width=6cm]{../Comparisons/ImagesFromVMD/3NFAACb.png}

Inertia Tensor - Molecule A \\
\begin{tabular}{|c c c|}
530.265	 & 	-1.20883	 & 	-7.84653	 \\
-1.20883	 & 	1361.05	 & 	-3.78865	 \\
-7.84653	 & 	-3.78865	 & 	915.61
\end{tabular}

\vtab
 EingenVectors - Molecule A     \\
\begin{tabular}{|c c c|}
-0.999791	 & 	-0.00154731	 & 	-0.0203648	 \\
-0.0203771	 & 	0.00845054	 & 	0.999757	 \\
-0.00137484	 & 	0.999963	 & 	-0.00848031
\end{tabular}

\vtab
 EingenValues - Molecule A     \\
\begin{tabular}{|c c c|}
530.103	 & 	915.738	 & 	1361.08	 \\
\end{tabular}
\columnbreak

Molecule B \
4NFAACi

\includegraphics[width=6cm]{../Comparisons/ImagesFromVMD/4NFAACi.png}

Inertia Tensor - Molecule B \\
\begin{tabular}{|c c c|}
502.43	 & 	-0.602691	 & 	-4.86988	 \\
-0.602691	 & 	1232.26	 & 	0.407295	 \\
-4.86988	 & 	0.407295	 & 	1676
\end{tabular}

\vtab
 EingenVectors - Molecule B     \\
\begin{tabular}{|c c c|}
0.999991	 & 	0.000823447	 & 	0.00414923	 \\
0.000819608	 & 	-0.999999	 & 	0.00092687	 \\
-0.00414999	 & 	0.000923461	 & 	0.999991
\end{tabular}

\vtab
 EingenValues - Molecule B     \\
\begin{tabular}{|c c c|}
502.409	 & 	1232.26	 & 	1676.02	 \\
\end{tabular}

\end{center}
\end{multicols}

\vtab[-5mm]
\begin{tabular}{*{2}{m{0.38\textwidth}}}
\begin{center}
\textcolor{NavyBlue}{\Large Different}
\end{center}
&
\begin{center}
\includegraphics[height=6.5cm]{../Comparisons/Vectors/inertia_tensor_of_3NFAACb_and_4NFAACi.png}
\end{center}
\end{tabular}

 \newpage

\vtab[-3cm]
\begin{center}
{\large FireTest \tab Número 245}
\end{center}
\begin{multicols}{2}
\begin{center}

Molecule A \
3NFAACb

\includegraphics[width=6cm]{../Comparisons/ImagesFromVMD/3NFAACb.png}

Inertia Tensor - Molecule A \\
\begin{tabular}{|c c c|}
530.265	 & 	-1.20883	 & 	-7.84653	 \\
-1.20883	 & 	1361.05	 & 	-3.78865	 \\
-7.84653	 & 	-3.78865	 & 	915.61
\end{tabular}

\vtab
 EingenVectors - Molecule A     \\
\begin{tabular}{|c c c|}
-0.999791	 & 	-0.00154731	 & 	-0.0203648	 \\
-0.0203771	 & 	0.00845054	 & 	0.999757	 \\
-0.00137484	 & 	0.999963	 & 	-0.00848031
\end{tabular}

\vtab
 EingenValues - Molecule A     \\
\begin{tabular}{|c c c|}
530.103	 & 	915.738	 & 	1361.08	 \\
\end{tabular}
\columnbreak

Molecule B \
4NFAACj

\includegraphics[width=6cm]{../Comparisons/ImagesFromVMD/4NFAACj.png}

Inertia Tensor - Molecule B \\
\begin{tabular}{|c c c|}
510.047	 & 	9.97005	 & 	-3.6306	 \\
9.97005	 & 	1225.52	 & 	-0.981092	 \\
-3.6306	 & 	-0.981092	 & 	1680.82
\end{tabular}

\vtab
 EingenVectors - Molecule B     \\
\begin{tabular}{|c c c|}
-0.999898	 & 	0.0139264	 & 	-0.00308865	 \\
0.0139195	 & 	0.999901	 & 	0.00226627	 \\
-0.00311991	 & 	-0.00222305	 & 	0.999993
\end{tabular}

\vtab
 EingenValues - Molecule B     \\
\begin{tabular}{|c c c|}
509.897	 & 	1225.65	 & 	1680.83	 \\
\end{tabular}

\end{center}
\end{multicols}

\vtab[-5mm]
\begin{tabular}{*{2}{m{0.38\textwidth}}}
\begin{center}
\textcolor{NavyBlue}{\Large Different}
\end{center}
&
\begin{center}
\includegraphics[height=6.5cm]{../Comparisons/Vectors/inertia_tensor_of_3NFAACb_and_4NFAACj.png}
\end{center}
\end{tabular}

 \newpage

\vtab[-3cm]
\begin{center}
{\large FireTest \tab Número 246}
\end{center}
\begin{multicols}{2}
\begin{center}

Molecule A \
3NFAACb

\includegraphics[width=6cm]{../Comparisons/ImagesFromVMD/3NFAACb.png}

Inertia Tensor - Molecule A \\
\begin{tabular}{|c c c|}
530.265	 & 	-1.20883	 & 	-7.84653	 \\
-1.20883	 & 	1361.05	 & 	-3.78865	 \\
-7.84653	 & 	-3.78865	 & 	915.61
\end{tabular}

\vtab
 EingenVectors - Molecule A     \\
\begin{tabular}{|c c c|}
-0.999791	 & 	-0.00154731	 & 	-0.0203648	 \\
-0.0203771	 & 	0.00845054	 & 	0.999757	 \\
-0.00137484	 & 	0.999963	 & 	-0.00848031
\end{tabular}

\vtab
 EingenValues - Molecule A     \\
\begin{tabular}{|c c c|}
530.103	 & 	915.738	 & 	1361.08	 \\
\end{tabular}
\columnbreak

Molecule B \
4NFAACl-3

\includegraphics[width=6cm]{../Comparisons/ImagesFromVMD/4NFAACl-3.png}

Inertia Tensor - Molecule B \\
\begin{tabular}{|c c c|}
506.608	 & 	0.709539	 & 	-0.555426	 \\
0.709539	 & 	1222.37	 & 	-2.84005	 \\
-0.555426	 & 	-2.84005	 & 	1678.41
\end{tabular}

\vtab
 EingenVectors - Molecule B     \\
\begin{tabular}{|c c c|}
-0.999999	 & 	0.000989428	 & 	-0.000471595	 \\
-0.000986471	 & 	-0.99998	 & 	-0.00622856	 \\
-0.000477748	 & 	-0.00622809	 & 	0.99998
\end{tabular}

\vtab
 EingenValues - Molecule B     \\
\begin{tabular}{|c c c|}
506.607	 & 	1222.36	 & 	1678.43	 \\
\end{tabular}

\end{center}
\end{multicols}

\vtab[-5mm]
\begin{tabular}{*{2}{m{0.38\textwidth}}}
\begin{center}
\textcolor{NavyBlue}{\Large Different}
\end{center}
&
\begin{center}
\includegraphics[height=6.5cm]{../Comparisons/Vectors/inertia_tensor_of_3NFAACb_and_4NFAACl-3.png}
\end{center}
\end{tabular}

 \newpage

\vtab[-3cm]
\begin{center}
{\large FireTest \tab Número 247}
\end{center}
\begin{multicols}{2}
\begin{center}

Molecule A \
3NFAACc

\includegraphics[width=6cm]{../Comparisons/ImagesFromVMD/3NFAACc.png}

Inertia Tensor - Molecule A \\
\begin{tabular}{|c c c|}
531.482	 & 	-5.48557	 & 	-0.978638	 \\
-5.48557	 & 	913.233	 & 	2.89201	 \\
-0.978638	 & 	2.89201	 & 	1353.14
\end{tabular}

\vtab
 EingenVectors - Molecule A     \\
\begin{tabular}{|c c c|}
-0.999896	 & 	-0.0143563	 & 	-0.00114029	 \\
0.0143485	 & 	-0.999875	 & 	0.00660611	 \\
-0.00123498	 & 	0.00658906	 & 	0.999978
\end{tabular}

\vtab
 EingenValues - Molecule A     \\
\begin{tabular}{|c c c|}
531.402	 & 	913.293	 & 	1353.16	 \\
\end{tabular}
\columnbreak

Molecule B \
3NFAACd

\includegraphics[width=6cm]{../Comparisons/ImagesFromVMD/3NFAACd.png}

Inertia Tensor - Molecule B \\
\begin{tabular}{|c c c|}
524.186	 & 	-1.32648	 & 	-2.36411	 \\
-1.32648	 & 	935.358	 & 	-2.91274	 \\
-2.36411	 & 	-2.91274	 & 	1359.16
\end{tabular}

\vtab
 EingenVectors - Molecule B     \\
\begin{tabular}{|c c c|}
-0.999991	 & 	-0.00324612	 & 	-0.00284261	 \\
0.00326554	 & 	-0.999971	 & 	-0.0068542	 \\
-0.00282028	 & 	-0.00686342	 & 	0.999972
\end{tabular}

\vtab
 EingenValues - Molecule B     \\
\begin{tabular}{|c c c|}
524.175	 & 	935.342	 & 	1359.19	 \\
\end{tabular}

\end{center}
\end{multicols}

\vtab[-5mm]
\begin{tabular}{*{2}{m{0.38\textwidth}}}
\begin{center}
\textcolor{NavyBlue}{\Large Different}
\end{center}
&
\begin{center}
\includegraphics[height=6.5cm]{../Comparisons/Vectors/inertia_tensor_of_3NFAACc_and_3NFAACd.png}
\end{center}
\end{tabular}

 \newpage

\vtab[-3cm]
\begin{center}
{\large FireTest \tab Número 248}
\end{center}
\begin{multicols}{2}
\begin{center}

Molecule A \
3NFAACc

\includegraphics[width=6cm]{../Comparisons/ImagesFromVMD/3NFAACc.png}

Inertia Tensor - Molecule A \\
\begin{tabular}{|c c c|}
531.482	 & 	-5.48557	 & 	-0.978638	 \\
-5.48557	 & 	913.233	 & 	2.89201	 \\
-0.978638	 & 	2.89201	 & 	1353.14
\end{tabular}

\vtab
 EingenVectors - Molecule A     \\
\begin{tabular}{|c c c|}
-0.999896	 & 	-0.0143563	 & 	-0.00114029	 \\
0.0143485	 & 	-0.999875	 & 	0.00660611	 \\
-0.00123498	 & 	0.00658906	 & 	0.999978
\end{tabular}

\vtab
 EingenValues - Molecule A     \\
\begin{tabular}{|c c c|}
531.402	 & 	913.293	 & 	1353.16	 \\
\end{tabular}
\columnbreak

Molecule B \
3NFAACe

\includegraphics[width=6cm]{../Comparisons/ImagesFromVMD/3NFAACe.png}

Inertia Tensor - Molecule B \\
\begin{tabular}{|c c c|}
524.508	 & 	15.8785	 & 	-4.02001	 \\
15.8785	 & 	1358.65	 & 	10.8723	 \\
-4.02001	 & 	10.8723	 & 	935.54
\end{tabular}

\vtab
 EingenVectors - Molecule B     \\
\begin{tabular}{|c c c|}
-0.999764	 & 	0.0191573	 & 	-0.0102761	 \\
0.0107588	 & 	0.025269	 & 	-0.999623	 \\
0.0188904	 & 	0.999497	 & 	0.0254692
\end{tabular}

\vtab
 EingenValues - Molecule B     \\
\begin{tabular}{|c c c|}
524.162	 & 	935.308	 & 	1359.22	 \\
\end{tabular}

\end{center}
\end{multicols}

\vtab[-5mm]
\begin{tabular}{*{2}{m{0.38\textwidth}}}
\begin{center}
\textcolor{NavyBlue}{\Large Different}
\end{center}
&
\begin{center}
\includegraphics[height=6.5cm]{../Comparisons/Vectors/inertia_tensor_of_3NFAACc_and_3NFAACe.png}
\end{center}
\end{tabular}

 \newpage

\vtab[-3cm]
\begin{center}
{\large FireTest \tab Número 249}
\end{center}
\begin{multicols}{2}
\begin{center}

Molecule A \
3NFAACc

\includegraphics[width=6cm]{../Comparisons/ImagesFromVMD/3NFAACc.png}

Inertia Tensor - Molecule A \\
\begin{tabular}{|c c c|}
531.482	 & 	-5.48557	 & 	-0.978638	 \\
-5.48557	 & 	913.233	 & 	2.89201	 \\
-0.978638	 & 	2.89201	 & 	1353.14
\end{tabular}

\vtab
 EingenVectors - Molecule A     \\
\begin{tabular}{|c c c|}
-0.999896	 & 	-0.0143563	 & 	-0.00114029	 \\
0.0143485	 & 	-0.999875	 & 	0.00660611	 \\
-0.00123498	 & 	0.00658906	 & 	0.999978
\end{tabular}

\vtab
 EingenValues - Molecule A     \\
\begin{tabular}{|c c c|}
531.402	 & 	913.293	 & 	1353.16	 \\
\end{tabular}
\columnbreak

Molecule B \
3NFAACf

\includegraphics[width=6cm]{../Comparisons/ImagesFromVMD/3NFAACf.png}

Inertia Tensor - Molecule B \\
\begin{tabular}{|c c c|}
530.208	 & 	-1.63273	 & 	-7.84857	 \\
-1.63273	 & 	1360.76	 & 	-3.6411	 \\
-7.84857	 & 	-3.6411	 & 	915.396
\end{tabular}

\vtab
 EingenVectors - Molecule B     \\
\begin{tabular}{|c c c|}
-0.99979	 & 	-0.00205437	 & 	-0.0203824	 \\
-0.0203985	 & 	0.00810116	 & 	0.999759	 \\
-0.00188875	 & 	0.999965	 & 	-0.00814137
\end{tabular}

\vtab
 EingenValues - Molecule B     \\
\begin{tabular}{|c c c|}
530.045	 & 	915.527	 & 	1360.79	 \\
\end{tabular}

\end{center}
\end{multicols}

\vtab[-5mm]
\begin{tabular}{*{2}{m{0.38\textwidth}}}
\begin{center}
\textcolor{NavyBlue}{\Large Different}
\end{center}
&
\begin{center}
\includegraphics[height=6.5cm]{../Comparisons/Vectors/inertia_tensor_of_3NFAACc_and_3NFAACf.png}
\end{center}
\end{tabular}

 \newpage

\vtab[-3cm]
\begin{center}
{\large FireTest \tab Número 250}
\end{center}
\begin{multicols}{2}
\begin{center}

Molecule A \
3NFAACc

\includegraphics[width=6cm]{../Comparisons/ImagesFromVMD/3NFAACc.png}

Inertia Tensor - Molecule A \\
\begin{tabular}{|c c c|}
531.482	 & 	-5.48557	 & 	-0.978638	 \\
-5.48557	 & 	913.233	 & 	2.89201	 \\
-0.978638	 & 	2.89201	 & 	1353.14
\end{tabular}

\vtab
 EingenVectors - Molecule A     \\
\begin{tabular}{|c c c|}
-0.999896	 & 	-0.0143563	 & 	-0.00114029	 \\
0.0143485	 & 	-0.999875	 & 	0.00660611	 \\
-0.00123498	 & 	0.00658906	 & 	0.999978
\end{tabular}

\vtab
 EingenValues - Molecule A     \\
\begin{tabular}{|c c c|}
531.402	 & 	913.293	 & 	1353.16	 \\
\end{tabular}
\columnbreak

Molecule B \
3NFAACg

\includegraphics[width=6cm]{../Comparisons/ImagesFromVMD/3NFAACg.png}

Inertia Tensor - Molecule B \\
\begin{tabular}{|c c c|}
532.891	 & 	-0.526938	 & 	-0.504713	 \\
-0.526938	 & 	918.454	 & 	2.35809	 \\
-0.504713	 & 	2.35809	 & 	1356.91
\end{tabular}

\vtab
 EingenVectors - Molecule B     \\
\begin{tabular}{|c c c|}
-0.999999	 & 	-0.00136295	 & 	-0.000608598	 \\
0.00135965	 & 	-0.999985	 & 	0.00537946	 \\
-0.000615921	 & 	0.00537863	 & 	0.999985
\end{tabular}

\vtab
 EingenValues - Molecule B     \\
\begin{tabular}{|c c c|}
532.89	 & 	918.442	 & 	1356.93	 \\
\end{tabular}

\end{center}
\end{multicols}

\vtab[-5mm]
\begin{tabular}{*{2}{m{0.38\textwidth}}}
\begin{center}
\textcolor{NavyBlue}{\Large Different}
\end{center}
&
\begin{center}
\includegraphics[height=6.5cm]{../Comparisons/Vectors/inertia_tensor_of_3NFAACc_and_3NFAACg.png}
\end{center}
\end{tabular}

 \newpage

\vtab[-3cm]
\begin{center}
{\large FireTest \tab Número 251}
\end{center}
\begin{multicols}{2}
\begin{center}

Molecule A \
3NFAACc

\includegraphics[width=6cm]{../Comparisons/ImagesFromVMD/3NFAACc.png}

Inertia Tensor - Molecule A \\
\begin{tabular}{|c c c|}
531.482	 & 	-5.48557	 & 	-0.978638	 \\
-5.48557	 & 	913.233	 & 	2.89201	 \\
-0.978638	 & 	2.89201	 & 	1353.14
\end{tabular}

\vtab
 EingenVectors - Molecule A     \\
\begin{tabular}{|c c c|}
-0.999896	 & 	-0.0143563	 & 	-0.00114029	 \\
0.0143485	 & 	-0.999875	 & 	0.00660611	 \\
-0.00123498	 & 	0.00658906	 & 	0.999978
\end{tabular}

\vtab
 EingenValues - Molecule A     \\
\begin{tabular}{|c c c|}
531.402	 & 	913.293	 & 	1353.16	 \\
\end{tabular}
\columnbreak

Molecule B \
3NFAACh

\includegraphics[width=6cm]{../Comparisons/ImagesFromVMD/3NFAACh.png}

Inertia Tensor - Molecule B \\
\begin{tabular}{|c c c|}
528.718	 & 	-3.89066	 & 	-3.66882	 \\
-3.89066	 & 	924.911	 & 	2.93817	 \\
-3.66882	 & 	2.93817	 & 	1336.08
\end{tabular}

\vtab
 EingenVectors - Molecule B     \\
\begin{tabular}{|c c c|}
0.999942	 & 	0.00978475	 & 	0.00450803	 \\
0.00975199	 & 	-0.999926	 & 	0.00723269	 \\
-0.00457847	 & 	0.0071883	 & 	0.999964
\end{tabular}

\vtab
 EingenValues - Molecule B     \\
\begin{tabular}{|c c c|}
528.663	 & 	924.928	 & 	1336.12	 \\
\end{tabular}

\end{center}
\end{multicols}

\vtab[-5mm]
\begin{tabular}{*{2}{m{0.38\textwidth}}}
\begin{center}
\textcolor{NavyBlue}{\Large Different}
\end{center}
&
\begin{center}
\includegraphics[height=6.5cm]{../Comparisons/Vectors/inertia_tensor_of_3NFAACc_and_3NFAACh.png}
\end{center}
\end{tabular}

 \newpage

\vtab[-3cm]
\begin{center}
{\large FireTest \tab Número 252}
\end{center}
\begin{multicols}{2}
\begin{center}

Molecule A \
3NFAACc

\includegraphics[width=6cm]{../Comparisons/ImagesFromVMD/3NFAACc.png}

Inertia Tensor - Molecule A \\
\begin{tabular}{|c c c|}
531.482	 & 	-5.48557	 & 	-0.978638	 \\
-5.48557	 & 	913.233	 & 	2.89201	 \\
-0.978638	 & 	2.89201	 & 	1353.14
\end{tabular}

\vtab
 EingenVectors - Molecule A     \\
\begin{tabular}{|c c c|}
-0.999896	 & 	-0.0143563	 & 	-0.00114029	 \\
0.0143485	 & 	-0.999875	 & 	0.00660611	 \\
-0.00123498	 & 	0.00658906	 & 	0.999978
\end{tabular}

\vtab
 EingenValues - Molecule A     \\
\begin{tabular}{|c c c|}
531.402	 & 	913.293	 & 	1353.16	 \\
\end{tabular}
\columnbreak

Molecule B \
3NFAACi

\includegraphics[width=6cm]{../Comparisons/ImagesFromVMD/3NFAACi.png}

Inertia Tensor - Molecule B \\
\begin{tabular}{|c c c|}
521.487	 & 	-1.22943	 & 	-2.71042	 \\
-1.22943	 & 	949.663	 & 	1.67006	 \\
-2.71042	 & 	1.67006	 & 	1346.16
\end{tabular}

\vtab
 EingenVectors - Molecule B     \\
\begin{tabular}{|c c c|}
0.999991	 & 	0.00285841	 & 	0.00328077	 \\
0.00284452	 & 	-0.999987	 & 	0.00423133	 \\
-0.00329282	 & 	0.00422196	 & 	0.999986
\end{tabular}

\vtab
 EingenValues - Molecule B     \\
\begin{tabular}{|c c c|}
521.475	 & 	949.659	 & 	1346.18	 \\
\end{tabular}

\end{center}
\end{multicols}

\vtab[-5mm]
\begin{tabular}{*{2}{m{0.38\textwidth}}}
\begin{center}
\textcolor{NavyBlue}{\Large Different}
\end{center}
&
\begin{center}
\includegraphics[height=6.5cm]{../Comparisons/Vectors/inertia_tensor_of_3NFAACc_and_3NFAACi.png}
\end{center}
\end{tabular}

 \newpage

\vtab[-3cm]
\begin{center}
{\large FireTest \tab Número 253}
\end{center}
\begin{multicols}{2}
\begin{center}

Molecule A \
3NFAACc

\includegraphics[width=6cm]{../Comparisons/ImagesFromVMD/3NFAACc.png}

Inertia Tensor - Molecule A \\
\begin{tabular}{|c c c|}
531.482	 & 	-5.48557	 & 	-0.978638	 \\
-5.48557	 & 	913.233	 & 	2.89201	 \\
-0.978638	 & 	2.89201	 & 	1353.14
\end{tabular}

\vtab
 EingenVectors - Molecule A     \\
\begin{tabular}{|c c c|}
-0.999896	 & 	-0.0143563	 & 	-0.00114029	 \\
0.0143485	 & 	-0.999875	 & 	0.00660611	 \\
-0.00123498	 & 	0.00658906	 & 	0.999978
\end{tabular}

\vtab
 EingenValues - Molecule A     \\
\begin{tabular}{|c c c|}
531.402	 & 	913.293	 & 	1353.16	 \\
\end{tabular}
\columnbreak

Molecule B \
3NFAACj

\includegraphics[width=6cm]{../Comparisons/ImagesFromVMD/3NFAACj.png}

Inertia Tensor - Molecule B \\
\begin{tabular}{|c c c|}
533.789	 & 	-4.75521	 & 	-1.91525	 \\
-4.75521	 & 	920.091	 & 	2.28449	 \\
-1.91525	 & 	2.28449	 & 	1348.28
\end{tabular}

\vtab
 EingenVectors - Molecule B     \\
\begin{tabular}{|c c c|}
-0.999922	 & 	-0.0122929	 & 	-0.00231663	 \\
0.0122803	 & 	-0.99991	 & 	0.00539026	 \\
-0.00238268	 & 	0.00536139	 & 	0.999983
\end{tabular}

\vtab
 EingenValues - Molecule B     \\
\begin{tabular}{|c c c|}
533.726	 & 	920.137	 & 	1348.3	 \\
\end{tabular}

\end{center}
\end{multicols}

\vtab[-5mm]
\begin{tabular}{*{2}{m{0.38\textwidth}}}
\begin{center}
\textcolor{NavyBlue}{\Large Different}
\end{center}
&
\begin{center}
\includegraphics[height=6.5cm]{../Comparisons/Vectors/inertia_tensor_of_3NFAACc_and_3NFAACj.png}
\end{center}
\end{tabular}

 \newpage

\vtab[-3cm]
\begin{center}
{\large FireTest \tab Número 254}
\end{center}
\begin{multicols}{2}
\begin{center}

Molecule A \
3NFAACc

\includegraphics[width=6cm]{../Comparisons/ImagesFromVMD/3NFAACc.png}

Inertia Tensor - Molecule A \\
\begin{tabular}{|c c c|}
531.482	 & 	-5.48557	 & 	-0.978638	 \\
-5.48557	 & 	913.233	 & 	2.89201	 \\
-0.978638	 & 	2.89201	 & 	1353.14
\end{tabular}

\vtab
 EingenVectors - Molecule A     \\
\begin{tabular}{|c c c|}
-0.999896	 & 	-0.0143563	 & 	-0.00114029	 \\
0.0143485	 & 	-0.999875	 & 	0.00660611	 \\
-0.00123498	 & 	0.00658906	 & 	0.999978
\end{tabular}

\vtab
 EingenValues - Molecule A     \\
\begin{tabular}{|c c c|}
531.402	 & 	913.293	 & 	1353.16	 \\
\end{tabular}
\columnbreak

Molecule B \
3NFAACk

\includegraphics[width=6cm]{../Comparisons/ImagesFromVMD/3NFAACk.png}

Inertia Tensor - Molecule B \\
\begin{tabular}{|c c c|}
534.899	 & 	-5.81418	 & 	-0.270063	 \\
-5.81418	 & 	913.263	 & 	2.4519	 \\
-0.270063	 & 	2.4519	 & 	1353.77
\end{tabular}

\vtab
 EingenVectors - Molecule B     \\
\begin{tabular}{|c c c|}
-0.999882	 & 	-0.0153593	 & 	-0.00028374	 \\
0.0153575	 & 	-0.999867	 & 	0.00557573	 \\
-0.000369342	 & 	0.00557072	 & 	0.999984
\end{tabular}

\vtab
 EingenValues - Molecule B     \\
\begin{tabular}{|c c c|}
534.81	 & 	913.339	 & 	1353.78	 \\
\end{tabular}

\end{center}
\end{multicols}

\vtab[-5mm]
\begin{tabular}{*{2}{m{0.38\textwidth}}}
\begin{center}
\textcolor{NavyBlue}{\Large Different}
\end{center}
&
\begin{center}
\includegraphics[height=6.5cm]{../Comparisons/Vectors/inertia_tensor_of_3NFAACc_and_3NFAACk.png}
\end{center}
\end{tabular}

 \newpage

\vtab[-3cm]
\begin{center}
{\large FireTest \tab Número 255}
\end{center}
\begin{multicols}{2}
\begin{center}

Molecule A \
3NFAACc

\includegraphics[width=6cm]{../Comparisons/ImagesFromVMD/3NFAACc.png}

Inertia Tensor - Molecule A \\
\begin{tabular}{|c c c|}
531.482	 & 	-5.48557	 & 	-0.978638	 \\
-5.48557	 & 	913.233	 & 	2.89201	 \\
-0.978638	 & 	2.89201	 & 	1353.14
\end{tabular}

\vtab
 EingenVectors - Molecule A     \\
\begin{tabular}{|c c c|}
-0.999896	 & 	-0.0143563	 & 	-0.00114029	 \\
0.0143485	 & 	-0.999875	 & 	0.00660611	 \\
-0.00123498	 & 	0.00658906	 & 	0.999978
\end{tabular}

\vtab
 EingenValues - Molecule A     \\
\begin{tabular}{|c c c|}
531.402	 & 	913.293	 & 	1353.16	 \\
\end{tabular}
\columnbreak

Molecule B \
3NFAACl

\includegraphics[width=6cm]{../Comparisons/ImagesFromVMD/3NFAACl.png}

Inertia Tensor - Molecule B \\
\begin{tabular}{|c c c|}
531.723	 & 	3.03424	 & 	2.73426	 \\
3.03424	 & 	929.418	 & 	-1.84284	 \\
2.73426	 & 	-1.84284	 & 	1355.39
\end{tabular}

\vtab
 EingenVectors - Molecule B     \\
\begin{tabular}{|c c c|}
0.999965	 & 	-0.00764413	 & 	-0.00333648	 \\
-0.00765838	 & 	-0.999962	 & 	-0.00427708	 \\
0.00330366	 & 	-0.00430248	 & 	0.999985
\end{tabular}

\vtab
 EingenValues - Molecule B     \\
\begin{tabular}{|c c c|}
531.691	 & 	929.434	 & 	1355.4	 \\
\end{tabular}

\end{center}
\end{multicols}

\vtab[-5mm]
\begin{tabular}{*{2}{m{0.38\textwidth}}}
\begin{center}
\textcolor{NavyBlue}{\Large Different}
\end{center}
&
\begin{center}
\includegraphics[height=6.5cm]{../Comparisons/Vectors/inertia_tensor_of_3NFAACc_and_3NFAACl.png}
\end{center}
\end{tabular}

 \newpage

\vtab[-3cm]
\begin{center}
{\large FireTest \tab Número 256}
\end{center}
\begin{multicols}{2}
\begin{center}

Molecule A \
3NFAACc

\includegraphics[width=6cm]{../Comparisons/ImagesFromVMD/3NFAACc.png}

Inertia Tensor - Molecule A \\
\begin{tabular}{|c c c|}
531.482	 & 	-5.48557	 & 	-0.978638	 \\
-5.48557	 & 	913.233	 & 	2.89201	 \\
-0.978638	 & 	2.89201	 & 	1353.14
\end{tabular}

\vtab
 EingenVectors - Molecule A     \\
\begin{tabular}{|c c c|}
-0.999896	 & 	-0.0143563	 & 	-0.00114029	 \\
0.0143485	 & 	-0.999875	 & 	0.00660611	 \\
-0.00123498	 & 	0.00658906	 & 	0.999978
\end{tabular}

\vtab
 EingenValues - Molecule A     \\
\begin{tabular}{|c c c|}
531.402	 & 	913.293	 & 	1353.16	 \\
\end{tabular}
\columnbreak

Molecule B \
3NFAACm

\includegraphics[width=6cm]{../Comparisons/ImagesFromVMD/3NFAACm.png}

Inertia Tensor - Molecule B \\
\begin{tabular}{|c c c|}
532.546	 & 	13.7854	 & 	-15.4626	 \\
13.7854	 & 	1354.87	 & 	11.5786	 \\
-15.4626	 & 	11.5786	 & 	929.101
\end{tabular}

\vtab
 EingenVectors - Molecule B     \\
\begin{tabular}{|c c c|}
-0.999075	 & 	0.0172851	 & 	-0.0393769	 \\
0.0398168	 & 	0.0258942	 & 	-0.998871	 \\
-0.016246	 & 	-0.999515	 & 	-0.0265584
\end{tabular}

\vtab
 EingenValues - Molecule B     \\
\begin{tabular}{|c c c|}
531.698	 & 	929.417	 & 	1355.4	 \\
\end{tabular}

\end{center}
\end{multicols}

\vtab[-5mm]
\begin{tabular}{*{2}{m{0.38\textwidth}}}
\begin{center}
\textcolor{NavyBlue}{\Large Different}
\end{center}
&
\begin{center}
\includegraphics[height=6.5cm]{../Comparisons/Vectors/inertia_tensor_of_3NFAACc_and_3NFAACm.png}
\end{center}
\end{tabular}

 \newpage

\vtab[-3cm]
\begin{center}
{\large FireTest \tab Número 257}
\end{center}
\begin{multicols}{2}
\begin{center}

Molecule A \
3NFAACc

\includegraphics[width=6cm]{../Comparisons/ImagesFromVMD/3NFAACc.png}

Inertia Tensor - Molecule A \\
\begin{tabular}{|c c c|}
531.482	 & 	-5.48557	 & 	-0.978638	 \\
-5.48557	 & 	913.233	 & 	2.89201	 \\
-0.978638	 & 	2.89201	 & 	1353.14
\end{tabular}

\vtab
 EingenVectors - Molecule A     \\
\begin{tabular}{|c c c|}
-0.999896	 & 	-0.0143563	 & 	-0.00114029	 \\
0.0143485	 & 	-0.999875	 & 	0.00660611	 \\
-0.00123498	 & 	0.00658906	 & 	0.999978
\end{tabular}

\vtab
 EingenValues - Molecule A     \\
\begin{tabular}{|c c c|}
531.402	 & 	913.293	 & 	1353.16	 \\
\end{tabular}
\columnbreak

Molecule B \
3NFAACn

\includegraphics[width=6cm]{../Comparisons/ImagesFromVMD/3NFAACn.png}

Inertia Tensor - Molecule B \\
\begin{tabular}{|c c c|}
531.896	 & 	3.78027	 & 	-13.1151	 \\
3.78027	 & 	1353.2	 & 	-7.47403	 \\
-13.1151	 & 	-7.47403	 & 	912.989
\end{tabular}

\vtab
 EingenVectors - Molecule B     \\
\begin{tabular}{|c c c|}
0.999403	 & 	-0.00428573	 & 	0.0342679	 \\
0.0341891	 & 	-0.0172718	 & 	-0.999266	 \\
0.00487445	 & 	0.999842	 & 	-0.017115
\end{tabular}

\vtab
 EingenValues - Molecule B     \\
\begin{tabular}{|c c c|}
531.43	 & 	913.309	 & 	1353.35	 \\
\end{tabular}

\end{center}
\end{multicols}

\vtab[-5mm]
\begin{tabular}{*{2}{m{0.38\textwidth}}}
\begin{center}
\textcolor{NavyBlue}{\Large Enantiomers}
\end{center}
&
\begin{center}
\includegraphics[height=6.5cm]{../Comparisons/Vectors/inertia_tensor_of_3NFAACc_and_3NFAACn.png}
\end{center}
\end{tabular}

 \newpage

\vtab[-3cm]
\begin{center}
{\large FireTest \tab Número 258}
\end{center}
\begin{multicols}{2}
\begin{center}

Molecule A \
3NFAACc

\includegraphics[width=6cm]{../Comparisons/ImagesFromVMD/3NFAACc.png}

Inertia Tensor - Molecule A \\
\begin{tabular}{|c c c|}
531.482	 & 	-5.48557	 & 	-0.978638	 \\
-5.48557	 & 	913.233	 & 	2.89201	 \\
-0.978638	 & 	2.89201	 & 	1353.14
\end{tabular}

\vtab
 EingenVectors - Molecule A     \\
\begin{tabular}{|c c c|}
-0.999896	 & 	-0.0143563	 & 	-0.00114029	 \\
0.0143485	 & 	-0.999875	 & 	0.00660611	 \\
-0.00123498	 & 	0.00658906	 & 	0.999978
\end{tabular}

\vtab
 EingenValues - Molecule A     \\
\begin{tabular}{|c c c|}
531.402	 & 	913.293	 & 	1353.16	 \\
\end{tabular}
\columnbreak

Molecule B \
4NFAACa

\includegraphics[width=6cm]{../Comparisons/ImagesFromVMD/4NFAACa.png}

Inertia Tensor - Molecule B \\
\begin{tabular}{|c c c|}
479.392	 & 	3.27131	 & 	4.22557	 \\
3.27131	 & 	1242.39	 & 	-0.852684	 \\
4.22557	 & 	-0.852684	 & 	1647.37
\end{tabular}

\vtab
 EingenVectors - Molecule B     \\
\begin{tabular}{|c c c|}
0.999984	 & 	-0.00429123	 & 	-0.00362083	 \\
-0.00429871	 & 	-0.999989	 & 	-0.0020607	 \\
0.00361195	 & 	-0.00207623	 & 	0.999991
\end{tabular}

\vtab
 EingenValues - Molecule B     \\
\begin{tabular}{|c c c|}
479.363	 & 	1242.41	 & 	1647.39	 \\
\end{tabular}

\end{center}
\end{multicols}

\vtab[-5mm]
\begin{tabular}{*{2}{m{0.38\textwidth}}}
\begin{center}
\textcolor{NavyBlue}{\Large Different}
\end{center}
&
\begin{center}
\includegraphics[height=6.5cm]{../Comparisons/Vectors/inertia_tensor_of_3NFAACc_and_4NFAACa.png}
\end{center}
\end{tabular}

 \newpage

\vtab[-3cm]
\begin{center}
{\large FireTest \tab Número 259}
\end{center}
\begin{multicols}{2}
\begin{center}

Molecule A \
3NFAACc

\includegraphics[width=6cm]{../Comparisons/ImagesFromVMD/3NFAACc.png}

Inertia Tensor - Molecule A \\
\begin{tabular}{|c c c|}
531.482	 & 	-5.48557	 & 	-0.978638	 \\
-5.48557	 & 	913.233	 & 	2.89201	 \\
-0.978638	 & 	2.89201	 & 	1353.14
\end{tabular}

\vtab
 EingenVectors - Molecule A     \\
\begin{tabular}{|c c c|}
-0.999896	 & 	-0.0143563	 & 	-0.00114029	 \\
0.0143485	 & 	-0.999875	 & 	0.00660611	 \\
-0.00123498	 & 	0.00658906	 & 	0.999978
\end{tabular}

\vtab
 EingenValues - Molecule A     \\
\begin{tabular}{|c c c|}
531.402	 & 	913.293	 & 	1353.16	 \\
\end{tabular}
\columnbreak

Molecule B \
4NFAACb

\includegraphics[width=6cm]{../Comparisons/ImagesFromVMD/4NFAACb.png}

Inertia Tensor - Molecule B \\
\begin{tabular}{|c c c|}
479.338	 & 	3.27331	 & 	-4.22553	 \\
3.27331	 & 	1242.4	 & 	0.852083	 \\
-4.22553	 & 	0.852083	 & 	1647.3
\end{tabular}

\vtab
 EingenVectors - Molecule B     \\
\begin{tabular}{|c c c|}
0.999984	 & 	-0.00429353	 & 	0.00362086	 \\
-0.004301	 & 	-0.999989	 & 	0.00205959	 \\
-0.00361198	 & 	0.00207513	 & 	0.999991
\end{tabular}

\vtab
 EingenValues - Molecule B     \\
\begin{tabular}{|c c c|}
479.308	 & 	1242.41	 & 	1647.31	 \\
\end{tabular}

\end{center}
\end{multicols}

\vtab[-5mm]
\begin{tabular}{*{2}{m{0.38\textwidth}}}
\begin{center}
\textcolor{NavyBlue}{\Large Different}
\end{center}
&
\begin{center}
\includegraphics[height=6.5cm]{../Comparisons/Vectors/inertia_tensor_of_3NFAACc_and_4NFAACb.png}
\end{center}
\end{tabular}

 \newpage

\vtab[-3cm]
\begin{center}
{\large FireTest \tab Número 260}
\end{center}
\begin{multicols}{2}
\begin{center}

Molecule A \
3NFAACc

\includegraphics[width=6cm]{../Comparisons/ImagesFromVMD/3NFAACc.png}

Inertia Tensor - Molecule A \\
\begin{tabular}{|c c c|}
531.482	 & 	-5.48557	 & 	-0.978638	 \\
-5.48557	 & 	913.233	 & 	2.89201	 \\
-0.978638	 & 	2.89201	 & 	1353.14
\end{tabular}

\vtab
 EingenVectors - Molecule A     \\
\begin{tabular}{|c c c|}
-0.999896	 & 	-0.0143563	 & 	-0.00114029	 \\
0.0143485	 & 	-0.999875	 & 	0.00660611	 \\
-0.00123498	 & 	0.00658906	 & 	0.999978
\end{tabular}

\vtab
 EingenValues - Molecule A     \\
\begin{tabular}{|c c c|}
531.402	 & 	913.293	 & 	1353.16	 \\
\end{tabular}
\columnbreak

Molecule B \
4NFAACc

\includegraphics[width=6cm]{../Comparisons/ImagesFromVMD/4NFAACc.png}

Inertia Tensor - Molecule B \\
\begin{tabular}{|c c c|}
482.067	 & 	-5.39474	 & 	-1.35857	 \\
-5.39474	 & 	1240.3	 & 	-2.54035	 \\
-1.35857	 & 	-2.54035	 & 	1647.06
\end{tabular}

\vtab
 EingenVectors - Molecule B     \\
\begin{tabular}{|c c c|}
-0.999974	 & 	-0.00711826	 & 	-0.0011816	 \\
0.00712547	 & 	-0.999955	 & 	-0.00622156	 \\
-0.00113726	 & 	-0.00622982	 & 	0.99998
\end{tabular}

\vtab
 EingenValues - Molecule B     \\
\begin{tabular}{|c c c|}
482.027	 & 	1240.32	 & 	1647.08	 \\
\end{tabular}

\end{center}
\end{multicols}

\vtab[-5mm]
\begin{tabular}{*{2}{m{0.38\textwidth}}}
\begin{center}
\textcolor{NavyBlue}{\Large Different}
\end{center}
&
\begin{center}
\includegraphics[height=6.5cm]{../Comparisons/Vectors/inertia_tensor_of_3NFAACc_and_4NFAACc.png}
\end{center}
\end{tabular}

 \newpage

\vtab[-3cm]
\begin{center}
{\large FireTest \tab Número 261}
\end{center}
\begin{multicols}{2}
\begin{center}

Molecule A \
3NFAACc

\includegraphics[width=6cm]{../Comparisons/ImagesFromVMD/3NFAACc.png}

Inertia Tensor - Molecule A \\
\begin{tabular}{|c c c|}
531.482	 & 	-5.48557	 & 	-0.978638	 \\
-5.48557	 & 	913.233	 & 	2.89201	 \\
-0.978638	 & 	2.89201	 & 	1353.14
\end{tabular}

\vtab
 EingenVectors - Molecule A     \\
\begin{tabular}{|c c c|}
-0.999896	 & 	-0.0143563	 & 	-0.00114029	 \\
0.0143485	 & 	-0.999875	 & 	0.00660611	 \\
-0.00123498	 & 	0.00658906	 & 	0.999978
\end{tabular}

\vtab
 EingenValues - Molecule A     \\
\begin{tabular}{|c c c|}
531.402	 & 	913.293	 & 	1353.16	 \\
\end{tabular}
\columnbreak

Molecule B \
4NFAACd

\includegraphics[width=6cm]{../Comparisons/ImagesFromVMD/4NFAACd.png}

Inertia Tensor - Molecule B \\
\begin{tabular}{|c c c|}
491.672	 & 	0.24486	 & 	-3.10016	 \\
0.24486	 & 	1231.15	 & 	2.19965	 \\
-3.10016	 & 	2.19965	 & 	1650.11
\end{tabular}

\vtab
 EingenVectors - Molecule B     \\
\begin{tabular}{|c c c|}
0.999996	 & 	-0.000339081	 & 	0.00267677	 \\
-0.000353124	 & 	-0.999986	 & 	0.00524747	 \\
-0.00267495	 & 	0.0052484	 & 	0.999983
\end{tabular}

\vtab
 EingenValues - Molecule B     \\
\begin{tabular}{|c c c|}
491.663	 & 	1231.14	 & 	1650.13	 \\
\end{tabular}

\end{center}
\end{multicols}

\vtab[-5mm]
\begin{tabular}{*{2}{m{0.38\textwidth}}}
\begin{center}
\textcolor{NavyBlue}{\Large Different}
\end{center}
&
\begin{center}
\includegraphics[height=6.5cm]{../Comparisons/Vectors/inertia_tensor_of_3NFAACc_and_4NFAACd.png}
\end{center}
\end{tabular}

 \newpage

\vtab[-3cm]
\begin{center}
{\large FireTest \tab Número 262}
\end{center}
\begin{multicols}{2}
\begin{center}

Molecule A \
3NFAACc

\includegraphics[width=6cm]{../Comparisons/ImagesFromVMD/3NFAACc.png}

Inertia Tensor - Molecule A \\
\begin{tabular}{|c c c|}
531.482	 & 	-5.48557	 & 	-0.978638	 \\
-5.48557	 & 	913.233	 & 	2.89201	 \\
-0.978638	 & 	2.89201	 & 	1353.14
\end{tabular}

\vtab
 EingenVectors - Molecule A     \\
\begin{tabular}{|c c c|}
-0.999896	 & 	-0.0143563	 & 	-0.00114029	 \\
0.0143485	 & 	-0.999875	 & 	0.00660611	 \\
-0.00123498	 & 	0.00658906	 & 	0.999978
\end{tabular}

\vtab
 EingenValues - Molecule A     \\
\begin{tabular}{|c c c|}
531.402	 & 	913.293	 & 	1353.16	 \\
\end{tabular}
\columnbreak

Molecule B \
4NFAACe

\includegraphics[width=6cm]{../Comparisons/ImagesFromVMD/4NFAACe.png}

Inertia Tensor - Molecule B \\
\begin{tabular}{|c c c|}
489.025	 & 	-0.430035	 & 	3.98876	 \\
-0.430035	 & 	1233.71	 & 	-2.06505	 \\
3.98876	 & 	-2.06505	 & 	1641.79
\end{tabular}

\vtab
 EingenVectors - Molecule B     \\
\begin{tabular}{|c c c|}
0.999994	 & 	0.000567863	 & 	-0.00345908	 \\
0.000550336	 & 	-0.999987	 & 	-0.00506565	 \\
0.00346192	 & 	-0.00506372	 & 	0.999981
\end{tabular}

\vtab
 EingenValues - Molecule B     \\
\begin{tabular}{|c c c|}
489.011	 & 	1233.7	 & 	1641.81	 \\
\end{tabular}

\end{center}
\end{multicols}

\vtab[-5mm]
\begin{tabular}{*{2}{m{0.38\textwidth}}}
\begin{center}
\textcolor{NavyBlue}{\Large Different}
\end{center}
&
\begin{center}
\includegraphics[height=6.5cm]{../Comparisons/Vectors/inertia_tensor_of_3NFAACc_and_4NFAACe.png}
\end{center}
\end{tabular}

 \newpage

\vtab[-3cm]
\begin{center}
{\large FireTest \tab Número 263}
\end{center}
\begin{multicols}{2}
\begin{center}

Molecule A \
3NFAACc

\includegraphics[width=6cm]{../Comparisons/ImagesFromVMD/3NFAACc.png}

Inertia Tensor - Molecule A \\
\begin{tabular}{|c c c|}
531.482	 & 	-5.48557	 & 	-0.978638	 \\
-5.48557	 & 	913.233	 & 	2.89201	 \\
-0.978638	 & 	2.89201	 & 	1353.14
\end{tabular}

\vtab
 EingenVectors - Molecule A     \\
\begin{tabular}{|c c c|}
-0.999896	 & 	-0.0143563	 & 	-0.00114029	 \\
0.0143485	 & 	-0.999875	 & 	0.00660611	 \\
-0.00123498	 & 	0.00658906	 & 	0.999978
\end{tabular}

\vtab
 EingenValues - Molecule A     \\
\begin{tabular}{|c c c|}
531.402	 & 	913.293	 & 	1353.16	 \\
\end{tabular}
\columnbreak

Molecule B \
4NFAACf

\includegraphics[width=6cm]{../Comparisons/ImagesFromVMD/4NFAACf.png}

Inertia Tensor - Molecule B \\
\begin{tabular}{|c c c|}
509.683	 & 	2.80651	 & 	-1.91422	 \\
2.80651	 & 	1219.11	 & 	2.66132	 \\
-1.91422	 & 	2.66132	 & 	1681.17
\end{tabular}

\vtab
 EingenVectors - Molecule B     \\
\begin{tabular}{|c c c|}
-0.999991	 & 	0.00396206	 & 	-0.00164298	 \\
-0.00397143	 & 	-0.999976	 & 	0.0057431	 \\
-0.00162019	 & 	0.00574957	 & 	0.999982
\end{tabular}

\vtab
 EingenValues - Molecule B     \\
\begin{tabular}{|c c c|}
509.668	 & 	1219.11	 & 	1681.18	 \\
\end{tabular}

\end{center}
\end{multicols}

\vtab[-5mm]
\begin{tabular}{*{2}{m{0.38\textwidth}}}
\begin{center}
\textcolor{NavyBlue}{\Large Different}
\end{center}
&
\begin{center}
\includegraphics[height=6.5cm]{../Comparisons/Vectors/inertia_tensor_of_3NFAACc_and_4NFAACf.png}
\end{center}
\end{tabular}

 \newpage

\vtab[-3cm]
\begin{center}
{\large FireTest \tab Número 264}
\end{center}
\begin{multicols}{2}
\begin{center}

Molecule A \
3NFAACc

\includegraphics[width=6cm]{../Comparisons/ImagesFromVMD/3NFAACc.png}

Inertia Tensor - Molecule A \\
\begin{tabular}{|c c c|}
531.482	 & 	-5.48557	 & 	-0.978638	 \\
-5.48557	 & 	913.233	 & 	2.89201	 \\
-0.978638	 & 	2.89201	 & 	1353.14
\end{tabular}

\vtab
 EingenVectors - Molecule A     \\
\begin{tabular}{|c c c|}
-0.999896	 & 	-0.0143563	 & 	-0.00114029	 \\
0.0143485	 & 	-0.999875	 & 	0.00660611	 \\
-0.00123498	 & 	0.00658906	 & 	0.999978
\end{tabular}

\vtab
 EingenValues - Molecule A     \\
\begin{tabular}{|c c c|}
531.402	 & 	913.293	 & 	1353.16	 \\
\end{tabular}
\columnbreak

Molecule B \
4NFAACg

\includegraphics[width=6cm]{../Comparisons/ImagesFromVMD/4NFAACg.png}

Inertia Tensor - Molecule B \\
\begin{tabular}{|c c c|}
513.78	 & 	4.51917	 & 	0.266555	 \\
4.51917	 & 	1208.04	 & 	-1.18628	 \\
0.266555	 & 	-1.18628	 & 	1700.9
\end{tabular}

\vtab
 EingenVectors - Molecule B     \\
\begin{tabular}{|c c c|}
-0.999979	 & 	0.00650929	 & 	0.000231034	 \\
-0.00650983	 & 	-0.999976	 & 	-0.00240351	 \\
0.000215383	 & 	-0.00240496	 & 	0.999997
\end{tabular}

\vtab
 EingenValues - Molecule B     \\
\begin{tabular}{|c c c|}
513.751	 & 	1208.07	 & 	1700.9	 \\
\end{tabular}

\end{center}
\end{multicols}

\vtab[-5mm]
\begin{tabular}{*{2}{m{0.38\textwidth}}}
\begin{center}
\textcolor{NavyBlue}{\Large Different}
\end{center}
&
\begin{center}
\includegraphics[height=6.5cm]{../Comparisons/Vectors/inertia_tensor_of_3NFAACc_and_4NFAACg.png}
\end{center}
\end{tabular}

 \newpage

\vtab[-3cm]
\begin{center}
{\large FireTest \tab Número 265}
\end{center}
\begin{multicols}{2}
\begin{center}

Molecule A \
3NFAACc

\includegraphics[width=6cm]{../Comparisons/ImagesFromVMD/3NFAACc.png}

Inertia Tensor - Molecule A \\
\begin{tabular}{|c c c|}
531.482	 & 	-5.48557	 & 	-0.978638	 \\
-5.48557	 & 	913.233	 & 	2.89201	 \\
-0.978638	 & 	2.89201	 & 	1353.14
\end{tabular}

\vtab
 EingenVectors - Molecule A     \\
\begin{tabular}{|c c c|}
-0.999896	 & 	-0.0143563	 & 	-0.00114029	 \\
0.0143485	 & 	-0.999875	 & 	0.00660611	 \\
-0.00123498	 & 	0.00658906	 & 	0.999978
\end{tabular}

\vtab
 EingenValues - Molecule A     \\
\begin{tabular}{|c c c|}
531.402	 & 	913.293	 & 	1353.16	 \\
\end{tabular}
\columnbreak

Molecule B \
4NFAACi

\includegraphics[width=6cm]{../Comparisons/ImagesFromVMD/4NFAACi.png}

Inertia Tensor - Molecule B \\
\begin{tabular}{|c c c|}
502.43	 & 	-0.602691	 & 	-4.86988	 \\
-0.602691	 & 	1232.26	 & 	0.407295	 \\
-4.86988	 & 	0.407295	 & 	1676
\end{tabular}

\vtab
 EingenVectors - Molecule B     \\
\begin{tabular}{|c c c|}
0.999991	 & 	0.000823447	 & 	0.00414923	 \\
0.000819608	 & 	-0.999999	 & 	0.00092687	 \\
-0.00414999	 & 	0.000923461	 & 	0.999991
\end{tabular}

\vtab
 EingenValues - Molecule B     \\
\begin{tabular}{|c c c|}
502.409	 & 	1232.26	 & 	1676.02	 \\
\end{tabular}

\end{center}
\end{multicols}

\vtab[-5mm]
\begin{tabular}{*{2}{m{0.38\textwidth}}}
\begin{center}
\textcolor{NavyBlue}{\Large Different}
\end{center}
&
\begin{center}
\includegraphics[height=6.5cm]{../Comparisons/Vectors/inertia_tensor_of_3NFAACc_and_4NFAACi.png}
\end{center}
\end{tabular}

 \newpage

\vtab[-3cm]
\begin{center}
{\large FireTest \tab Número 266}
\end{center}
\begin{multicols}{2}
\begin{center}

Molecule A \
3NFAACc

\includegraphics[width=6cm]{../Comparisons/ImagesFromVMD/3NFAACc.png}

Inertia Tensor - Molecule A \\
\begin{tabular}{|c c c|}
531.482	 & 	-5.48557	 & 	-0.978638	 \\
-5.48557	 & 	913.233	 & 	2.89201	 \\
-0.978638	 & 	2.89201	 & 	1353.14
\end{tabular}

\vtab
 EingenVectors - Molecule A     \\
\begin{tabular}{|c c c|}
-0.999896	 & 	-0.0143563	 & 	-0.00114029	 \\
0.0143485	 & 	-0.999875	 & 	0.00660611	 \\
-0.00123498	 & 	0.00658906	 & 	0.999978
\end{tabular}

\vtab
 EingenValues - Molecule A     \\
\begin{tabular}{|c c c|}
531.402	 & 	913.293	 & 	1353.16	 \\
\end{tabular}
\columnbreak

Molecule B \
4NFAACj

\includegraphics[width=6cm]{../Comparisons/ImagesFromVMD/4NFAACj.png}

Inertia Tensor - Molecule B \\
\begin{tabular}{|c c c|}
510.047	 & 	9.97005	 & 	-3.6306	 \\
9.97005	 & 	1225.52	 & 	-0.981092	 \\
-3.6306	 & 	-0.981092	 & 	1680.82
\end{tabular}

\vtab
 EingenVectors - Molecule B     \\
\begin{tabular}{|c c c|}
-0.999898	 & 	0.0139264	 & 	-0.00308865	 \\
0.0139195	 & 	0.999901	 & 	0.00226627	 \\
-0.00311991	 & 	-0.00222305	 & 	0.999993
\end{tabular}

\vtab
 EingenValues - Molecule B     \\
\begin{tabular}{|c c c|}
509.897	 & 	1225.65	 & 	1680.83	 \\
\end{tabular}

\end{center}
\end{multicols}

\vtab[-5mm]
\begin{tabular}{*{2}{m{0.38\textwidth}}}
\begin{center}
\textcolor{NavyBlue}{\Large Different}
\end{center}
&
\begin{center}
\includegraphics[height=6.5cm]{../Comparisons/Vectors/inertia_tensor_of_3NFAACc_and_4NFAACj.png}
\end{center}
\end{tabular}

 \newpage

\vtab[-3cm]
\begin{center}
{\large FireTest \tab Número 267}
\end{center}
\begin{multicols}{2}
\begin{center}

Molecule A \
3NFAACc

\includegraphics[width=6cm]{../Comparisons/ImagesFromVMD/3NFAACc.png}

Inertia Tensor - Molecule A \\
\begin{tabular}{|c c c|}
531.482	 & 	-5.48557	 & 	-0.978638	 \\
-5.48557	 & 	913.233	 & 	2.89201	 \\
-0.978638	 & 	2.89201	 & 	1353.14
\end{tabular}

\vtab
 EingenVectors - Molecule A     \\
\begin{tabular}{|c c c|}
-0.999896	 & 	-0.0143563	 & 	-0.00114029	 \\
0.0143485	 & 	-0.999875	 & 	0.00660611	 \\
-0.00123498	 & 	0.00658906	 & 	0.999978
\end{tabular}

\vtab
 EingenValues - Molecule A     \\
\begin{tabular}{|c c c|}
531.402	 & 	913.293	 & 	1353.16	 \\
\end{tabular}
\columnbreak

Molecule B \
4NFAACl-3

\includegraphics[width=6cm]{../Comparisons/ImagesFromVMD/4NFAACl-3.png}

Inertia Tensor - Molecule B \\
\begin{tabular}{|c c c|}
506.608	 & 	0.709539	 & 	-0.555426	 \\
0.709539	 & 	1222.37	 & 	-2.84005	 \\
-0.555426	 & 	-2.84005	 & 	1678.41
\end{tabular}

\vtab
 EingenVectors - Molecule B     \\
\begin{tabular}{|c c c|}
-0.999999	 & 	0.000989428	 & 	-0.000471595	 \\
-0.000986471	 & 	-0.99998	 & 	-0.00622856	 \\
-0.000477748	 & 	-0.00622809	 & 	0.99998
\end{tabular}

\vtab
 EingenValues - Molecule B     \\
\begin{tabular}{|c c c|}
506.607	 & 	1222.36	 & 	1678.43	 \\
\end{tabular}

\end{center}
\end{multicols}

\vtab[-5mm]
\begin{tabular}{*{2}{m{0.38\textwidth}}}
\begin{center}
\textcolor{NavyBlue}{\Large Different}
\end{center}
&
\begin{center}
\includegraphics[height=6.5cm]{../Comparisons/Vectors/inertia_tensor_of_3NFAACc_and_4NFAACl-3.png}
\end{center}
\end{tabular}

 \newpage

\vtab[-3cm]
\begin{center}
{\large FireTest \tab Número 268}
\end{center}
\begin{multicols}{2}
\begin{center}

Molecule A \
3NFAACd

\includegraphics[width=6cm]{../Comparisons/ImagesFromVMD/3NFAACd.png}

Inertia Tensor - Molecule A \\
\begin{tabular}{|c c c|}
524.186	 & 	-1.32648	 & 	-2.36411	 \\
-1.32648	 & 	935.358	 & 	-2.91274	 \\
-2.36411	 & 	-2.91274	 & 	1359.16
\end{tabular}

\vtab
 EingenVectors - Molecule A     \\
\begin{tabular}{|c c c|}
-0.999991	 & 	-0.00324612	 & 	-0.00284261	 \\
0.00326554	 & 	-0.999971	 & 	-0.0068542	 \\
-0.00282028	 & 	-0.00686342	 & 	0.999972
\end{tabular}

\vtab
 EingenValues - Molecule A     \\
\begin{tabular}{|c c c|}
524.175	 & 	935.342	 & 	1359.19	 \\
\end{tabular}
\columnbreak

Molecule B \
3NFAACe

\includegraphics[width=6cm]{../Comparisons/ImagesFromVMD/3NFAACe.png}

Inertia Tensor - Molecule B \\
\begin{tabular}{|c c c|}
524.508	 & 	15.8785	 & 	-4.02001	 \\
15.8785	 & 	1358.65	 & 	10.8723	 \\
-4.02001	 & 	10.8723	 & 	935.54
\end{tabular}

\vtab
 EingenVectors - Molecule B     \\
\begin{tabular}{|c c c|}
-0.999764	 & 	0.0191573	 & 	-0.0102761	 \\
0.0107588	 & 	0.025269	 & 	-0.999623	 \\
0.0188904	 & 	0.999497	 & 	0.0254692
\end{tabular}

\vtab
 EingenValues - Molecule B     \\
\begin{tabular}{|c c c|}
524.162	 & 	935.308	 & 	1359.22	 \\
\end{tabular}

\end{center}
\end{multicols}

\vtab[-5mm]
\begin{tabular}{*{2}{m{0.38\textwidth}}}
\begin{center}
\textcolor{NavyBlue}{\Large Enantiomers}
\end{center}
&
\begin{center}
\includegraphics[height=6.5cm]{../Comparisons/Vectors/inertia_tensor_of_3NFAACd_and_3NFAACe.png}
\end{center}
\end{tabular}

 \newpage

\vtab[-3cm]
\begin{center}
{\large FireTest \tab Número 269}
\end{center}
\begin{multicols}{2}
\begin{center}

Molecule A \
3NFAACd

\includegraphics[width=6cm]{../Comparisons/ImagesFromVMD/3NFAACd.png}

Inertia Tensor - Molecule A \\
\begin{tabular}{|c c c|}
524.186	 & 	-1.32648	 & 	-2.36411	 \\
-1.32648	 & 	935.358	 & 	-2.91274	 \\
-2.36411	 & 	-2.91274	 & 	1359.16
\end{tabular}

\vtab
 EingenVectors - Molecule A     \\
\begin{tabular}{|c c c|}
-0.999991	 & 	-0.00324612	 & 	-0.00284261	 \\
0.00326554	 & 	-0.999971	 & 	-0.0068542	 \\
-0.00282028	 & 	-0.00686342	 & 	0.999972
\end{tabular}

\vtab
 EingenValues - Molecule A     \\
\begin{tabular}{|c c c|}
524.175	 & 	935.342	 & 	1359.19	 \\
\end{tabular}
\columnbreak

Molecule B \
3NFAACf

\includegraphics[width=6cm]{../Comparisons/ImagesFromVMD/3NFAACf.png}

Inertia Tensor - Molecule B \\
\begin{tabular}{|c c c|}
530.208	 & 	-1.63273	 & 	-7.84857	 \\
-1.63273	 & 	1360.76	 & 	-3.6411	 \\
-7.84857	 & 	-3.6411	 & 	915.396
\end{tabular}

\vtab
 EingenVectors - Molecule B     \\
\begin{tabular}{|c c c|}
-0.99979	 & 	-0.00205437	 & 	-0.0203824	 \\
-0.0203985	 & 	0.00810116	 & 	0.999759	 \\
-0.00188875	 & 	0.999965	 & 	-0.00814137
\end{tabular}

\vtab
 EingenValues - Molecule B     \\
\begin{tabular}{|c c c|}
530.045	 & 	915.527	 & 	1360.79	 \\
\end{tabular}

\end{center}
\end{multicols}

\vtab[-5mm]
\begin{tabular}{*{2}{m{0.38\textwidth}}}
\begin{center}
\textcolor{NavyBlue}{\Large Different}
\end{center}
&
\begin{center}
\includegraphics[height=6.5cm]{../Comparisons/Vectors/inertia_tensor_of_3NFAACd_and_3NFAACf.png}
\end{center}
\end{tabular}

 \newpage

\vtab[-3cm]
\begin{center}
{\large FireTest \tab Número 270}
\end{center}
\begin{multicols}{2}
\begin{center}

Molecule A \
3NFAACd

\includegraphics[width=6cm]{../Comparisons/ImagesFromVMD/3NFAACd.png}

Inertia Tensor - Molecule A \\
\begin{tabular}{|c c c|}
524.186	 & 	-1.32648	 & 	-2.36411	 \\
-1.32648	 & 	935.358	 & 	-2.91274	 \\
-2.36411	 & 	-2.91274	 & 	1359.16
\end{tabular}

\vtab
 EingenVectors - Molecule A     \\
\begin{tabular}{|c c c|}
-0.999991	 & 	-0.00324612	 & 	-0.00284261	 \\
0.00326554	 & 	-0.999971	 & 	-0.0068542	 \\
-0.00282028	 & 	-0.00686342	 & 	0.999972
\end{tabular}

\vtab
 EingenValues - Molecule A     \\
\begin{tabular}{|c c c|}
524.175	 & 	935.342	 & 	1359.19	 \\
\end{tabular}
\columnbreak

Molecule B \
3NFAACg

\includegraphics[width=6cm]{../Comparisons/ImagesFromVMD/3NFAACg.png}

Inertia Tensor - Molecule B \\
\begin{tabular}{|c c c|}
532.891	 & 	-0.526938	 & 	-0.504713	 \\
-0.526938	 & 	918.454	 & 	2.35809	 \\
-0.504713	 & 	2.35809	 & 	1356.91
\end{tabular}

\vtab
 EingenVectors - Molecule B     \\
\begin{tabular}{|c c c|}
-0.999999	 & 	-0.00136295	 & 	-0.000608598	 \\
0.00135965	 & 	-0.999985	 & 	0.00537946	 \\
-0.000615921	 & 	0.00537863	 & 	0.999985
\end{tabular}

\vtab
 EingenValues - Molecule B     \\
\begin{tabular}{|c c c|}
532.89	 & 	918.442	 & 	1356.93	 \\
\end{tabular}

\end{center}
\end{multicols}

\vtab[-5mm]
\begin{tabular}{*{2}{m{0.38\textwidth}}}
\begin{center}
\textcolor{NavyBlue}{\Large Different}
\end{center}
&
\begin{center}
\includegraphics[height=6.5cm]{../Comparisons/Vectors/inertia_tensor_of_3NFAACd_and_3NFAACg.png}
\end{center}
\end{tabular}

 \newpage

\vtab[-3cm]
\begin{center}
{\large FireTest \tab Número 271}
\end{center}
\begin{multicols}{2}
\begin{center}

Molecule A \
3NFAACd

\includegraphics[width=6cm]{../Comparisons/ImagesFromVMD/3NFAACd.png}

Inertia Tensor - Molecule A \\
\begin{tabular}{|c c c|}
524.186	 & 	-1.32648	 & 	-2.36411	 \\
-1.32648	 & 	935.358	 & 	-2.91274	 \\
-2.36411	 & 	-2.91274	 & 	1359.16
\end{tabular}

\vtab
 EingenVectors - Molecule A     \\
\begin{tabular}{|c c c|}
-0.999991	 & 	-0.00324612	 & 	-0.00284261	 \\
0.00326554	 & 	-0.999971	 & 	-0.0068542	 \\
-0.00282028	 & 	-0.00686342	 & 	0.999972
\end{tabular}

\vtab
 EingenValues - Molecule A     \\
\begin{tabular}{|c c c|}
524.175	 & 	935.342	 & 	1359.19	 \\
\end{tabular}
\columnbreak

Molecule B \
3NFAACh

\includegraphics[width=6cm]{../Comparisons/ImagesFromVMD/3NFAACh.png}

Inertia Tensor - Molecule B \\
\begin{tabular}{|c c c|}
528.718	 & 	-3.89066	 & 	-3.66882	 \\
-3.89066	 & 	924.911	 & 	2.93817	 \\
-3.66882	 & 	2.93817	 & 	1336.08
\end{tabular}

\vtab
 EingenVectors - Molecule B     \\
\begin{tabular}{|c c c|}
0.999942	 & 	0.00978475	 & 	0.00450803	 \\
0.00975199	 & 	-0.999926	 & 	0.00723269	 \\
-0.00457847	 & 	0.0071883	 & 	0.999964
\end{tabular}

\vtab
 EingenValues - Molecule B     \\
\begin{tabular}{|c c c|}
528.663	 & 	924.928	 & 	1336.12	 \\
\end{tabular}

\end{center}
\end{multicols}

\vtab[-5mm]
\begin{tabular}{*{2}{m{0.38\textwidth}}}
\begin{center}
\textcolor{NavyBlue}{\Large Different}
\end{center}
&
\begin{center}
\includegraphics[height=6.5cm]{../Comparisons/Vectors/inertia_tensor_of_3NFAACd_and_3NFAACh.png}
\end{center}
\end{tabular}

 \newpage

\vtab[-3cm]
\begin{center}
{\large FireTest \tab Número 272}
\end{center}
\begin{multicols}{2}
\begin{center}

Molecule A \
3NFAACd

\includegraphics[width=6cm]{../Comparisons/ImagesFromVMD/3NFAACd.png}

Inertia Tensor - Molecule A \\
\begin{tabular}{|c c c|}
524.186	 & 	-1.32648	 & 	-2.36411	 \\
-1.32648	 & 	935.358	 & 	-2.91274	 \\
-2.36411	 & 	-2.91274	 & 	1359.16
\end{tabular}

\vtab
 EingenVectors - Molecule A     \\
\begin{tabular}{|c c c|}
-0.999991	 & 	-0.00324612	 & 	-0.00284261	 \\
0.00326554	 & 	-0.999971	 & 	-0.0068542	 \\
-0.00282028	 & 	-0.00686342	 & 	0.999972
\end{tabular}

\vtab
 EingenValues - Molecule A     \\
\begin{tabular}{|c c c|}
524.175	 & 	935.342	 & 	1359.19	 \\
\end{tabular}
\columnbreak

Molecule B \
3NFAACi

\includegraphics[width=6cm]{../Comparisons/ImagesFromVMD/3NFAACi.png}

Inertia Tensor - Molecule B \\
\begin{tabular}{|c c c|}
521.487	 & 	-1.22943	 & 	-2.71042	 \\
-1.22943	 & 	949.663	 & 	1.67006	 \\
-2.71042	 & 	1.67006	 & 	1346.16
\end{tabular}

\vtab
 EingenVectors - Molecule B     \\
\begin{tabular}{|c c c|}
0.999991	 & 	0.00285841	 & 	0.00328077	 \\
0.00284452	 & 	-0.999987	 & 	0.00423133	 \\
-0.00329282	 & 	0.00422196	 & 	0.999986
\end{tabular}

\vtab
 EingenValues - Molecule B     \\
\begin{tabular}{|c c c|}
521.475	 & 	949.659	 & 	1346.18	 \\
\end{tabular}

\end{center}
\end{multicols}

\vtab[-5mm]
\begin{tabular}{*{2}{m{0.38\textwidth}}}
\begin{center}
\textcolor{NavyBlue}{\Large Different}
\end{center}
&
\begin{center}
\includegraphics[height=6.5cm]{../Comparisons/Vectors/inertia_tensor_of_3NFAACd_and_3NFAACi.png}
\end{center}
\end{tabular}

 \newpage

\vtab[-3cm]
\begin{center}
{\large FireTest \tab Número 273}
\end{center}
\begin{multicols}{2}
\begin{center}

Molecule A \
3NFAACd

\includegraphics[width=6cm]{../Comparisons/ImagesFromVMD/3NFAACd.png}

Inertia Tensor - Molecule A \\
\begin{tabular}{|c c c|}
524.186	 & 	-1.32648	 & 	-2.36411	 \\
-1.32648	 & 	935.358	 & 	-2.91274	 \\
-2.36411	 & 	-2.91274	 & 	1359.16
\end{tabular}

\vtab
 EingenVectors - Molecule A     \\
\begin{tabular}{|c c c|}
-0.999991	 & 	-0.00324612	 & 	-0.00284261	 \\
0.00326554	 & 	-0.999971	 & 	-0.0068542	 \\
-0.00282028	 & 	-0.00686342	 & 	0.999972
\end{tabular}

\vtab
 EingenValues - Molecule A     \\
\begin{tabular}{|c c c|}
524.175	 & 	935.342	 & 	1359.19	 \\
\end{tabular}
\columnbreak

Molecule B \
3NFAACj

\includegraphics[width=6cm]{../Comparisons/ImagesFromVMD/3NFAACj.png}

Inertia Tensor - Molecule B \\
\begin{tabular}{|c c c|}
533.789	 & 	-4.75521	 & 	-1.91525	 \\
-4.75521	 & 	920.091	 & 	2.28449	 \\
-1.91525	 & 	2.28449	 & 	1348.28
\end{tabular}

\vtab
 EingenVectors - Molecule B     \\
\begin{tabular}{|c c c|}
-0.999922	 & 	-0.0122929	 & 	-0.00231663	 \\
0.0122803	 & 	-0.99991	 & 	0.00539026	 \\
-0.00238268	 & 	0.00536139	 & 	0.999983
\end{tabular}

\vtab
 EingenValues - Molecule B     \\
\begin{tabular}{|c c c|}
533.726	 & 	920.137	 & 	1348.3	 \\
\end{tabular}

\end{center}
\end{multicols}

\vtab[-5mm]
\begin{tabular}{*{2}{m{0.38\textwidth}}}
\begin{center}
\textcolor{NavyBlue}{\Large Different}
\end{center}
&
\begin{center}
\includegraphics[height=6.5cm]{../Comparisons/Vectors/inertia_tensor_of_3NFAACd_and_3NFAACj.png}
\end{center}
\end{tabular}

 \newpage

\vtab[-3cm]
\begin{center}
{\large FireTest \tab Número 274}
\end{center}
\begin{multicols}{2}
\begin{center}

Molecule A \
3NFAACd

\includegraphics[width=6cm]{../Comparisons/ImagesFromVMD/3NFAACd.png}

Inertia Tensor - Molecule A \\
\begin{tabular}{|c c c|}
524.186	 & 	-1.32648	 & 	-2.36411	 \\
-1.32648	 & 	935.358	 & 	-2.91274	 \\
-2.36411	 & 	-2.91274	 & 	1359.16
\end{tabular}

\vtab
 EingenVectors - Molecule A     \\
\begin{tabular}{|c c c|}
-0.999991	 & 	-0.00324612	 & 	-0.00284261	 \\
0.00326554	 & 	-0.999971	 & 	-0.0068542	 \\
-0.00282028	 & 	-0.00686342	 & 	0.999972
\end{tabular}

\vtab
 EingenValues - Molecule A     \\
\begin{tabular}{|c c c|}
524.175	 & 	935.342	 & 	1359.19	 \\
\end{tabular}
\columnbreak

Molecule B \
3NFAACk

\includegraphics[width=6cm]{../Comparisons/ImagesFromVMD/3NFAACk.png}

Inertia Tensor - Molecule B \\
\begin{tabular}{|c c c|}
534.899	 & 	-5.81418	 & 	-0.270063	 \\
-5.81418	 & 	913.263	 & 	2.4519	 \\
-0.270063	 & 	2.4519	 & 	1353.77
\end{tabular}

\vtab
 EingenVectors - Molecule B     \\
\begin{tabular}{|c c c|}
-0.999882	 & 	-0.0153593	 & 	-0.00028374	 \\
0.0153575	 & 	-0.999867	 & 	0.00557573	 \\
-0.000369342	 & 	0.00557072	 & 	0.999984
\end{tabular}

\vtab
 EingenValues - Molecule B     \\
\begin{tabular}{|c c c|}
534.81	 & 	913.339	 & 	1353.78	 \\
\end{tabular}

\end{center}
\end{multicols}

\vtab[-5mm]
\begin{tabular}{*{2}{m{0.38\textwidth}}}
\begin{center}
\textcolor{NavyBlue}{\Large Different}
\end{center}
&
\begin{center}
\includegraphics[height=6.5cm]{../Comparisons/Vectors/inertia_tensor_of_3NFAACd_and_3NFAACk.png}
\end{center}
\end{tabular}

 \newpage

\vtab[-3cm]
\begin{center}
{\large FireTest \tab Número 275}
\end{center}
\begin{multicols}{2}
\begin{center}

Molecule A \
3NFAACd

\includegraphics[width=6cm]{../Comparisons/ImagesFromVMD/3NFAACd.png}

Inertia Tensor - Molecule A \\
\begin{tabular}{|c c c|}
524.186	 & 	-1.32648	 & 	-2.36411	 \\
-1.32648	 & 	935.358	 & 	-2.91274	 \\
-2.36411	 & 	-2.91274	 & 	1359.16
\end{tabular}

\vtab
 EingenVectors - Molecule A     \\
\begin{tabular}{|c c c|}
-0.999991	 & 	-0.00324612	 & 	-0.00284261	 \\
0.00326554	 & 	-0.999971	 & 	-0.0068542	 \\
-0.00282028	 & 	-0.00686342	 & 	0.999972
\end{tabular}

\vtab
 EingenValues - Molecule A     \\
\begin{tabular}{|c c c|}
524.175	 & 	935.342	 & 	1359.19	 \\
\end{tabular}
\columnbreak

Molecule B \
3NFAACl

\includegraphics[width=6cm]{../Comparisons/ImagesFromVMD/3NFAACl.png}

Inertia Tensor - Molecule B \\
\begin{tabular}{|c c c|}
531.723	 & 	3.03424	 & 	2.73426	 \\
3.03424	 & 	929.418	 & 	-1.84284	 \\
2.73426	 & 	-1.84284	 & 	1355.39
\end{tabular}

\vtab
 EingenVectors - Molecule B     \\
\begin{tabular}{|c c c|}
0.999965	 & 	-0.00764413	 & 	-0.00333648	 \\
-0.00765838	 & 	-0.999962	 & 	-0.00427708	 \\
0.00330366	 & 	-0.00430248	 & 	0.999985
\end{tabular}

\vtab
 EingenValues - Molecule B     \\
\begin{tabular}{|c c c|}
531.691	 & 	929.434	 & 	1355.4	 \\
\end{tabular}

\end{center}
\end{multicols}

\vtab[-5mm]
\begin{tabular}{*{2}{m{0.38\textwidth}}}
\begin{center}
\textcolor{NavyBlue}{\Large Different}
\end{center}
&
\begin{center}
\includegraphics[height=6.5cm]{../Comparisons/Vectors/inertia_tensor_of_3NFAACd_and_3NFAACl.png}
\end{center}
\end{tabular}

 \newpage

\vtab[-3cm]
\begin{center}
{\large FireTest \tab Número 276}
\end{center}
\begin{multicols}{2}
\begin{center}

Molecule A \
3NFAACd

\includegraphics[width=6cm]{../Comparisons/ImagesFromVMD/3NFAACd.png}

Inertia Tensor - Molecule A \\
\begin{tabular}{|c c c|}
524.186	 & 	-1.32648	 & 	-2.36411	 \\
-1.32648	 & 	935.358	 & 	-2.91274	 \\
-2.36411	 & 	-2.91274	 & 	1359.16
\end{tabular}

\vtab
 EingenVectors - Molecule A     \\
\begin{tabular}{|c c c|}
-0.999991	 & 	-0.00324612	 & 	-0.00284261	 \\
0.00326554	 & 	-0.999971	 & 	-0.0068542	 \\
-0.00282028	 & 	-0.00686342	 & 	0.999972
\end{tabular}

\vtab
 EingenValues - Molecule A     \\
\begin{tabular}{|c c c|}
524.175	 & 	935.342	 & 	1359.19	 \\
\end{tabular}
\columnbreak

Molecule B \
3NFAACm

\includegraphics[width=6cm]{../Comparisons/ImagesFromVMD/3NFAACm.png}

Inertia Tensor - Molecule B \\
\begin{tabular}{|c c c|}
532.546	 & 	13.7854	 & 	-15.4626	 \\
13.7854	 & 	1354.87	 & 	11.5786	 \\
-15.4626	 & 	11.5786	 & 	929.101
\end{tabular}

\vtab
 EingenVectors - Molecule B     \\
\begin{tabular}{|c c c|}
-0.999075	 & 	0.0172851	 & 	-0.0393769	 \\
0.0398168	 & 	0.0258942	 & 	-0.998871	 \\
-0.016246	 & 	-0.999515	 & 	-0.0265584
\end{tabular}

\vtab
 EingenValues - Molecule B     \\
\begin{tabular}{|c c c|}
531.698	 & 	929.417	 & 	1355.4	 \\
\end{tabular}

\end{center}
\end{multicols}

\vtab[-5mm]
\begin{tabular}{*{2}{m{0.38\textwidth}}}
\begin{center}
\textcolor{NavyBlue}{\Large Different}
\end{center}
&
\begin{center}
\includegraphics[height=6.5cm]{../Comparisons/Vectors/inertia_tensor_of_3NFAACd_and_3NFAACm.png}
\end{center}
\end{tabular}

 \newpage

\vtab[-3cm]
\begin{center}
{\large FireTest \tab Número 277}
\end{center}
\begin{multicols}{2}
\begin{center}

Molecule A \
3NFAACd

\includegraphics[width=6cm]{../Comparisons/ImagesFromVMD/3NFAACd.png}

Inertia Tensor - Molecule A \\
\begin{tabular}{|c c c|}
524.186	 & 	-1.32648	 & 	-2.36411	 \\
-1.32648	 & 	935.358	 & 	-2.91274	 \\
-2.36411	 & 	-2.91274	 & 	1359.16
\end{tabular}

\vtab
 EingenVectors - Molecule A     \\
\begin{tabular}{|c c c|}
-0.999991	 & 	-0.00324612	 & 	-0.00284261	 \\
0.00326554	 & 	-0.999971	 & 	-0.0068542	 \\
-0.00282028	 & 	-0.00686342	 & 	0.999972
\end{tabular}

\vtab
 EingenValues - Molecule A     \\
\begin{tabular}{|c c c|}
524.175	 & 	935.342	 & 	1359.19	 \\
\end{tabular}
\columnbreak

Molecule B \
3NFAACn

\includegraphics[width=6cm]{../Comparisons/ImagesFromVMD/3NFAACn.png}

Inertia Tensor - Molecule B \\
\begin{tabular}{|c c c|}
531.896	 & 	3.78027	 & 	-13.1151	 \\
3.78027	 & 	1353.2	 & 	-7.47403	 \\
-13.1151	 & 	-7.47403	 & 	912.989
\end{tabular}

\vtab
 EingenVectors - Molecule B     \\
\begin{tabular}{|c c c|}
0.999403	 & 	-0.00428573	 & 	0.0342679	 \\
0.0341891	 & 	-0.0172718	 & 	-0.999266	 \\
0.00487445	 & 	0.999842	 & 	-0.017115
\end{tabular}

\vtab
 EingenValues - Molecule B     \\
\begin{tabular}{|c c c|}
531.43	 & 	913.309	 & 	1353.35	 \\
\end{tabular}

\end{center}
\end{multicols}

\vtab[-5mm]
\begin{tabular}{*{2}{m{0.38\textwidth}}}
\begin{center}
\textcolor{NavyBlue}{\Large Different}
\end{center}
&
\begin{center}
\includegraphics[height=6.5cm]{../Comparisons/Vectors/inertia_tensor_of_3NFAACd_and_3NFAACn.png}
\end{center}
\end{tabular}

 \newpage

\vtab[-3cm]
\begin{center}
{\large FireTest \tab Número 278}
\end{center}
\begin{multicols}{2}
\begin{center}

Molecule A \
3NFAACd

\includegraphics[width=6cm]{../Comparisons/ImagesFromVMD/3NFAACd.png}

Inertia Tensor - Molecule A \\
\begin{tabular}{|c c c|}
524.186	 & 	-1.32648	 & 	-2.36411	 \\
-1.32648	 & 	935.358	 & 	-2.91274	 \\
-2.36411	 & 	-2.91274	 & 	1359.16
\end{tabular}

\vtab
 EingenVectors - Molecule A     \\
\begin{tabular}{|c c c|}
-0.999991	 & 	-0.00324612	 & 	-0.00284261	 \\
0.00326554	 & 	-0.999971	 & 	-0.0068542	 \\
-0.00282028	 & 	-0.00686342	 & 	0.999972
\end{tabular}

\vtab
 EingenValues - Molecule A     \\
\begin{tabular}{|c c c|}
524.175	 & 	935.342	 & 	1359.19	 \\
\end{tabular}
\columnbreak

Molecule B \
4NFAACa

\includegraphics[width=6cm]{../Comparisons/ImagesFromVMD/4NFAACa.png}

Inertia Tensor - Molecule B \\
\begin{tabular}{|c c c|}
479.392	 & 	3.27131	 & 	4.22557	 \\
3.27131	 & 	1242.39	 & 	-0.852684	 \\
4.22557	 & 	-0.852684	 & 	1647.37
\end{tabular}

\vtab
 EingenVectors - Molecule B     \\
\begin{tabular}{|c c c|}
0.999984	 & 	-0.00429123	 & 	-0.00362083	 \\
-0.00429871	 & 	-0.999989	 & 	-0.0020607	 \\
0.00361195	 & 	-0.00207623	 & 	0.999991
\end{tabular}

\vtab
 EingenValues - Molecule B     \\
\begin{tabular}{|c c c|}
479.363	 & 	1242.41	 & 	1647.39	 \\
\end{tabular}

\end{center}
\end{multicols}

\vtab[-5mm]
\begin{tabular}{*{2}{m{0.38\textwidth}}}
\begin{center}
\textcolor{NavyBlue}{\Large Different}
\end{center}
&
\begin{center}
\includegraphics[height=6.5cm]{../Comparisons/Vectors/inertia_tensor_of_3NFAACd_and_4NFAACa.png}
\end{center}
\end{tabular}

 \newpage

\vtab[-3cm]
\begin{center}
{\large FireTest \tab Número 279}
\end{center}
\begin{multicols}{2}
\begin{center}

Molecule A \
3NFAACd

\includegraphics[width=6cm]{../Comparisons/ImagesFromVMD/3NFAACd.png}

Inertia Tensor - Molecule A \\
\begin{tabular}{|c c c|}
524.186	 & 	-1.32648	 & 	-2.36411	 \\
-1.32648	 & 	935.358	 & 	-2.91274	 \\
-2.36411	 & 	-2.91274	 & 	1359.16
\end{tabular}

\vtab
 EingenVectors - Molecule A     \\
\begin{tabular}{|c c c|}
-0.999991	 & 	-0.00324612	 & 	-0.00284261	 \\
0.00326554	 & 	-0.999971	 & 	-0.0068542	 \\
-0.00282028	 & 	-0.00686342	 & 	0.999972
\end{tabular}

\vtab
 EingenValues - Molecule A     \\
\begin{tabular}{|c c c|}
524.175	 & 	935.342	 & 	1359.19	 \\
\end{tabular}
\columnbreak

Molecule B \
4NFAACb

\includegraphics[width=6cm]{../Comparisons/ImagesFromVMD/4NFAACb.png}

Inertia Tensor - Molecule B \\
\begin{tabular}{|c c c|}
479.338	 & 	3.27331	 & 	-4.22553	 \\
3.27331	 & 	1242.4	 & 	0.852083	 \\
-4.22553	 & 	0.852083	 & 	1647.3
\end{tabular}

\vtab
 EingenVectors - Molecule B     \\
\begin{tabular}{|c c c|}
0.999984	 & 	-0.00429353	 & 	0.00362086	 \\
-0.004301	 & 	-0.999989	 & 	0.00205959	 \\
-0.00361198	 & 	0.00207513	 & 	0.999991
\end{tabular}

\vtab
 EingenValues - Molecule B     \\
\begin{tabular}{|c c c|}
479.308	 & 	1242.41	 & 	1647.31	 \\
\end{tabular}

\end{center}
\end{multicols}

\vtab[-5mm]
\begin{tabular}{*{2}{m{0.38\textwidth}}}
\begin{center}
\textcolor{NavyBlue}{\Large Different}
\end{center}
&
\begin{center}
\includegraphics[height=6.5cm]{../Comparisons/Vectors/inertia_tensor_of_3NFAACd_and_4NFAACb.png}
\end{center}
\end{tabular}

 \newpage

\vtab[-3cm]
\begin{center}
{\large FireTest \tab Número 280}
\end{center}
\begin{multicols}{2}
\begin{center}

Molecule A \
3NFAACd

\includegraphics[width=6cm]{../Comparisons/ImagesFromVMD/3NFAACd.png}

Inertia Tensor - Molecule A \\
\begin{tabular}{|c c c|}
524.186	 & 	-1.32648	 & 	-2.36411	 \\
-1.32648	 & 	935.358	 & 	-2.91274	 \\
-2.36411	 & 	-2.91274	 & 	1359.16
\end{tabular}

\vtab
 EingenVectors - Molecule A     \\
\begin{tabular}{|c c c|}
-0.999991	 & 	-0.00324612	 & 	-0.00284261	 \\
0.00326554	 & 	-0.999971	 & 	-0.0068542	 \\
-0.00282028	 & 	-0.00686342	 & 	0.999972
\end{tabular}

\vtab
 EingenValues - Molecule A     \\
\begin{tabular}{|c c c|}
524.175	 & 	935.342	 & 	1359.19	 \\
\end{tabular}
\columnbreak

Molecule B \
4NFAACc

\includegraphics[width=6cm]{../Comparisons/ImagesFromVMD/4NFAACc.png}

Inertia Tensor - Molecule B \\
\begin{tabular}{|c c c|}
482.067	 & 	-5.39474	 & 	-1.35857	 \\
-5.39474	 & 	1240.3	 & 	-2.54035	 \\
-1.35857	 & 	-2.54035	 & 	1647.06
\end{tabular}

\vtab
 EingenVectors - Molecule B     \\
\begin{tabular}{|c c c|}
-0.999974	 & 	-0.00711826	 & 	-0.0011816	 \\
0.00712547	 & 	-0.999955	 & 	-0.00622156	 \\
-0.00113726	 & 	-0.00622982	 & 	0.99998
\end{tabular}

\vtab
 EingenValues - Molecule B     \\
\begin{tabular}{|c c c|}
482.027	 & 	1240.32	 & 	1647.08	 \\
\end{tabular}

\end{center}
\end{multicols}

\vtab[-5mm]
\begin{tabular}{*{2}{m{0.38\textwidth}}}
\begin{center}
\textcolor{NavyBlue}{\Large Different}
\end{center}
&
\begin{center}
\includegraphics[height=6.5cm]{../Comparisons/Vectors/inertia_tensor_of_3NFAACd_and_4NFAACc.png}
\end{center}
\end{tabular}

 \newpage

\vtab[-3cm]
\begin{center}
{\large FireTest \tab Número 281}
\end{center}
\begin{multicols}{2}
\begin{center}

Molecule A \
3NFAACd

\includegraphics[width=6cm]{../Comparisons/ImagesFromVMD/3NFAACd.png}

Inertia Tensor - Molecule A \\
\begin{tabular}{|c c c|}
524.186	 & 	-1.32648	 & 	-2.36411	 \\
-1.32648	 & 	935.358	 & 	-2.91274	 \\
-2.36411	 & 	-2.91274	 & 	1359.16
\end{tabular}

\vtab
 EingenVectors - Molecule A     \\
\begin{tabular}{|c c c|}
-0.999991	 & 	-0.00324612	 & 	-0.00284261	 \\
0.00326554	 & 	-0.999971	 & 	-0.0068542	 \\
-0.00282028	 & 	-0.00686342	 & 	0.999972
\end{tabular}

\vtab
 EingenValues - Molecule A     \\
\begin{tabular}{|c c c|}
524.175	 & 	935.342	 & 	1359.19	 \\
\end{tabular}
\columnbreak

Molecule B \
4NFAACd

\includegraphics[width=6cm]{../Comparisons/ImagesFromVMD/4NFAACd.png}

Inertia Tensor - Molecule B \\
\begin{tabular}{|c c c|}
491.672	 & 	0.24486	 & 	-3.10016	 \\
0.24486	 & 	1231.15	 & 	2.19965	 \\
-3.10016	 & 	2.19965	 & 	1650.11
\end{tabular}

\vtab
 EingenVectors - Molecule B     \\
\begin{tabular}{|c c c|}
0.999996	 & 	-0.000339081	 & 	0.00267677	 \\
-0.000353124	 & 	-0.999986	 & 	0.00524747	 \\
-0.00267495	 & 	0.0052484	 & 	0.999983
\end{tabular}

\vtab
 EingenValues - Molecule B     \\
\begin{tabular}{|c c c|}
491.663	 & 	1231.14	 & 	1650.13	 \\
\end{tabular}

\end{center}
\end{multicols}

\vtab[-5mm]
\begin{tabular}{*{2}{m{0.38\textwidth}}}
\begin{center}
\textcolor{NavyBlue}{\Large Different}
\end{center}
&
\begin{center}
\includegraphics[height=6.5cm]{../Comparisons/Vectors/inertia_tensor_of_3NFAACd_and_4NFAACd.png}
\end{center}
\end{tabular}

 \newpage

\vtab[-3cm]
\begin{center}
{\large FireTest \tab Número 282}
\end{center}
\begin{multicols}{2}
\begin{center}

Molecule A \
3NFAACd

\includegraphics[width=6cm]{../Comparisons/ImagesFromVMD/3NFAACd.png}

Inertia Tensor - Molecule A \\
\begin{tabular}{|c c c|}
524.186	 & 	-1.32648	 & 	-2.36411	 \\
-1.32648	 & 	935.358	 & 	-2.91274	 \\
-2.36411	 & 	-2.91274	 & 	1359.16
\end{tabular}

\vtab
 EingenVectors - Molecule A     \\
\begin{tabular}{|c c c|}
-0.999991	 & 	-0.00324612	 & 	-0.00284261	 \\
0.00326554	 & 	-0.999971	 & 	-0.0068542	 \\
-0.00282028	 & 	-0.00686342	 & 	0.999972
\end{tabular}

\vtab
 EingenValues - Molecule A     \\
\begin{tabular}{|c c c|}
524.175	 & 	935.342	 & 	1359.19	 \\
\end{tabular}
\columnbreak

Molecule B \
4NFAACe

\includegraphics[width=6cm]{../Comparisons/ImagesFromVMD/4NFAACe.png}

Inertia Tensor - Molecule B \\
\begin{tabular}{|c c c|}
489.025	 & 	-0.430035	 & 	3.98876	 \\
-0.430035	 & 	1233.71	 & 	-2.06505	 \\
3.98876	 & 	-2.06505	 & 	1641.79
\end{tabular}

\vtab
 EingenVectors - Molecule B     \\
\begin{tabular}{|c c c|}
0.999994	 & 	0.000567863	 & 	-0.00345908	 \\
0.000550336	 & 	-0.999987	 & 	-0.00506565	 \\
0.00346192	 & 	-0.00506372	 & 	0.999981
\end{tabular}

\vtab
 EingenValues - Molecule B     \\
\begin{tabular}{|c c c|}
489.011	 & 	1233.7	 & 	1641.81	 \\
\end{tabular}

\end{center}
\end{multicols}

\vtab[-5mm]
\begin{tabular}{*{2}{m{0.38\textwidth}}}
\begin{center}
\textcolor{NavyBlue}{\Large Different}
\end{center}
&
\begin{center}
\includegraphics[height=6.5cm]{../Comparisons/Vectors/inertia_tensor_of_3NFAACd_and_4NFAACe.png}
\end{center}
\end{tabular}

 \newpage

\vtab[-3cm]
\begin{center}
{\large FireTest \tab Número 283}
\end{center}
\begin{multicols}{2}
\begin{center}

Molecule A \
3NFAACd

\includegraphics[width=6cm]{../Comparisons/ImagesFromVMD/3NFAACd.png}

Inertia Tensor - Molecule A \\
\begin{tabular}{|c c c|}
524.186	 & 	-1.32648	 & 	-2.36411	 \\
-1.32648	 & 	935.358	 & 	-2.91274	 \\
-2.36411	 & 	-2.91274	 & 	1359.16
\end{tabular}

\vtab
 EingenVectors - Molecule A     \\
\begin{tabular}{|c c c|}
-0.999991	 & 	-0.00324612	 & 	-0.00284261	 \\
0.00326554	 & 	-0.999971	 & 	-0.0068542	 \\
-0.00282028	 & 	-0.00686342	 & 	0.999972
\end{tabular}

\vtab
 EingenValues - Molecule A     \\
\begin{tabular}{|c c c|}
524.175	 & 	935.342	 & 	1359.19	 \\
\end{tabular}
\columnbreak

Molecule B \
4NFAACf

\includegraphics[width=6cm]{../Comparisons/ImagesFromVMD/4NFAACf.png}

Inertia Tensor - Molecule B \\
\begin{tabular}{|c c c|}
509.683	 & 	2.80651	 & 	-1.91422	 \\
2.80651	 & 	1219.11	 & 	2.66132	 \\
-1.91422	 & 	2.66132	 & 	1681.17
\end{tabular}

\vtab
 EingenVectors - Molecule B     \\
\begin{tabular}{|c c c|}
-0.999991	 & 	0.00396206	 & 	-0.00164298	 \\
-0.00397143	 & 	-0.999976	 & 	0.0057431	 \\
-0.00162019	 & 	0.00574957	 & 	0.999982
\end{tabular}

\vtab
 EingenValues - Molecule B     \\
\begin{tabular}{|c c c|}
509.668	 & 	1219.11	 & 	1681.18	 \\
\end{tabular}

\end{center}
\end{multicols}

\vtab[-5mm]
\begin{tabular}{*{2}{m{0.38\textwidth}}}
\begin{center}
\textcolor{NavyBlue}{\Large Different}
\end{center}
&
\begin{center}
\includegraphics[height=6.5cm]{../Comparisons/Vectors/inertia_tensor_of_3NFAACd_and_4NFAACf.png}
\end{center}
\end{tabular}

 \newpage

\vtab[-3cm]
\begin{center}
{\large FireTest \tab Número 284}
\end{center}
\begin{multicols}{2}
\begin{center}

Molecule A \
3NFAACd

\includegraphics[width=6cm]{../Comparisons/ImagesFromVMD/3NFAACd.png}

Inertia Tensor - Molecule A \\
\begin{tabular}{|c c c|}
524.186	 & 	-1.32648	 & 	-2.36411	 \\
-1.32648	 & 	935.358	 & 	-2.91274	 \\
-2.36411	 & 	-2.91274	 & 	1359.16
\end{tabular}

\vtab
 EingenVectors - Molecule A     \\
\begin{tabular}{|c c c|}
-0.999991	 & 	-0.00324612	 & 	-0.00284261	 \\
0.00326554	 & 	-0.999971	 & 	-0.0068542	 \\
-0.00282028	 & 	-0.00686342	 & 	0.999972
\end{tabular}

\vtab
 EingenValues - Molecule A     \\
\begin{tabular}{|c c c|}
524.175	 & 	935.342	 & 	1359.19	 \\
\end{tabular}
\columnbreak

Molecule B \
4NFAACg

\includegraphics[width=6cm]{../Comparisons/ImagesFromVMD/4NFAACg.png}

Inertia Tensor - Molecule B \\
\begin{tabular}{|c c c|}
513.78	 & 	4.51917	 & 	0.266555	 \\
4.51917	 & 	1208.04	 & 	-1.18628	 \\
0.266555	 & 	-1.18628	 & 	1700.9
\end{tabular}

\vtab
 EingenVectors - Molecule B     \\
\begin{tabular}{|c c c|}
-0.999979	 & 	0.00650929	 & 	0.000231034	 \\
-0.00650983	 & 	-0.999976	 & 	-0.00240351	 \\
0.000215383	 & 	-0.00240496	 & 	0.999997
\end{tabular}

\vtab
 EingenValues - Molecule B     \\
\begin{tabular}{|c c c|}
513.751	 & 	1208.07	 & 	1700.9	 \\
\end{tabular}

\end{center}
\end{multicols}

\vtab[-5mm]
\begin{tabular}{*{2}{m{0.38\textwidth}}}
\begin{center}
\textcolor{NavyBlue}{\Large Different}
\end{center}
&
\begin{center}
\includegraphics[height=6.5cm]{../Comparisons/Vectors/inertia_tensor_of_3NFAACd_and_4NFAACg.png}
\end{center}
\end{tabular}

 \newpage

\vtab[-3cm]
\begin{center}
{\large FireTest \tab Número 285}
\end{center}
\begin{multicols}{2}
\begin{center}

Molecule A \
3NFAACd

\includegraphics[width=6cm]{../Comparisons/ImagesFromVMD/3NFAACd.png}

Inertia Tensor - Molecule A \\
\begin{tabular}{|c c c|}
524.186	 & 	-1.32648	 & 	-2.36411	 \\
-1.32648	 & 	935.358	 & 	-2.91274	 \\
-2.36411	 & 	-2.91274	 & 	1359.16
\end{tabular}

\vtab
 EingenVectors - Molecule A     \\
\begin{tabular}{|c c c|}
-0.999991	 & 	-0.00324612	 & 	-0.00284261	 \\
0.00326554	 & 	-0.999971	 & 	-0.0068542	 \\
-0.00282028	 & 	-0.00686342	 & 	0.999972
\end{tabular}

\vtab
 EingenValues - Molecule A     \\
\begin{tabular}{|c c c|}
524.175	 & 	935.342	 & 	1359.19	 \\
\end{tabular}
\columnbreak

Molecule B \
4NFAACi

\includegraphics[width=6cm]{../Comparisons/ImagesFromVMD/4NFAACi.png}

Inertia Tensor - Molecule B \\
\begin{tabular}{|c c c|}
502.43	 & 	-0.602691	 & 	-4.86988	 \\
-0.602691	 & 	1232.26	 & 	0.407295	 \\
-4.86988	 & 	0.407295	 & 	1676
\end{tabular}

\vtab
 EingenVectors - Molecule B     \\
\begin{tabular}{|c c c|}
0.999991	 & 	0.000823447	 & 	0.00414923	 \\
0.000819608	 & 	-0.999999	 & 	0.00092687	 \\
-0.00414999	 & 	0.000923461	 & 	0.999991
\end{tabular}

\vtab
 EingenValues - Molecule B     \\
\begin{tabular}{|c c c|}
502.409	 & 	1232.26	 & 	1676.02	 \\
\end{tabular}

\end{center}
\end{multicols}

\vtab[-5mm]
\begin{tabular}{*{2}{m{0.38\textwidth}}}
\begin{center}
\textcolor{NavyBlue}{\Large Different}
\end{center}
&
\begin{center}
\includegraphics[height=6.5cm]{../Comparisons/Vectors/inertia_tensor_of_3NFAACd_and_4NFAACi.png}
\end{center}
\end{tabular}

 \newpage

\vtab[-3cm]
\begin{center}
{\large FireTest \tab Número 286}
\end{center}
\begin{multicols}{2}
\begin{center}

Molecule A \
3NFAACd

\includegraphics[width=6cm]{../Comparisons/ImagesFromVMD/3NFAACd.png}

Inertia Tensor - Molecule A \\
\begin{tabular}{|c c c|}
524.186	 & 	-1.32648	 & 	-2.36411	 \\
-1.32648	 & 	935.358	 & 	-2.91274	 \\
-2.36411	 & 	-2.91274	 & 	1359.16
\end{tabular}

\vtab
 EingenVectors - Molecule A     \\
\begin{tabular}{|c c c|}
-0.999991	 & 	-0.00324612	 & 	-0.00284261	 \\
0.00326554	 & 	-0.999971	 & 	-0.0068542	 \\
-0.00282028	 & 	-0.00686342	 & 	0.999972
\end{tabular}

\vtab
 EingenValues - Molecule A     \\
\begin{tabular}{|c c c|}
524.175	 & 	935.342	 & 	1359.19	 \\
\end{tabular}
\columnbreak

Molecule B \
4NFAACj

\includegraphics[width=6cm]{../Comparisons/ImagesFromVMD/4NFAACj.png}

Inertia Tensor - Molecule B \\
\begin{tabular}{|c c c|}
510.047	 & 	9.97005	 & 	-3.6306	 \\
9.97005	 & 	1225.52	 & 	-0.981092	 \\
-3.6306	 & 	-0.981092	 & 	1680.82
\end{tabular}

\vtab
 EingenVectors - Molecule B     \\
\begin{tabular}{|c c c|}
-0.999898	 & 	0.0139264	 & 	-0.00308865	 \\
0.0139195	 & 	0.999901	 & 	0.00226627	 \\
-0.00311991	 & 	-0.00222305	 & 	0.999993
\end{tabular}

\vtab
 EingenValues - Molecule B     \\
\begin{tabular}{|c c c|}
509.897	 & 	1225.65	 & 	1680.83	 \\
\end{tabular}

\end{center}
\end{multicols}

\vtab[-5mm]
\begin{tabular}{*{2}{m{0.38\textwidth}}}
\begin{center}
\textcolor{NavyBlue}{\Large Different}
\end{center}
&
\begin{center}
\includegraphics[height=6.5cm]{../Comparisons/Vectors/inertia_tensor_of_3NFAACd_and_4NFAACj.png}
\end{center}
\end{tabular}

 \newpage

\vtab[-3cm]
\begin{center}
{\large FireTest \tab Número 287}
\end{center}
\begin{multicols}{2}
\begin{center}

Molecule A \
3NFAACd

\includegraphics[width=6cm]{../Comparisons/ImagesFromVMD/3NFAACd.png}

Inertia Tensor - Molecule A \\
\begin{tabular}{|c c c|}
524.186	 & 	-1.32648	 & 	-2.36411	 \\
-1.32648	 & 	935.358	 & 	-2.91274	 \\
-2.36411	 & 	-2.91274	 & 	1359.16
\end{tabular}

\vtab
 EingenVectors - Molecule A     \\
\begin{tabular}{|c c c|}
-0.999991	 & 	-0.00324612	 & 	-0.00284261	 \\
0.00326554	 & 	-0.999971	 & 	-0.0068542	 \\
-0.00282028	 & 	-0.00686342	 & 	0.999972
\end{tabular}

\vtab
 EingenValues - Molecule A     \\
\begin{tabular}{|c c c|}
524.175	 & 	935.342	 & 	1359.19	 \\
\end{tabular}
\columnbreak

Molecule B \
4NFAACl-3

\includegraphics[width=6cm]{../Comparisons/ImagesFromVMD/4NFAACl-3.png}

Inertia Tensor - Molecule B \\
\begin{tabular}{|c c c|}
506.608	 & 	0.709539	 & 	-0.555426	 \\
0.709539	 & 	1222.37	 & 	-2.84005	 \\
-0.555426	 & 	-2.84005	 & 	1678.41
\end{tabular}

\vtab
 EingenVectors - Molecule B     \\
\begin{tabular}{|c c c|}
-0.999999	 & 	0.000989428	 & 	-0.000471595	 \\
-0.000986471	 & 	-0.99998	 & 	-0.00622856	 \\
-0.000477748	 & 	-0.00622809	 & 	0.99998
\end{tabular}

\vtab
 EingenValues - Molecule B     \\
\begin{tabular}{|c c c|}
506.607	 & 	1222.36	 & 	1678.43	 \\
\end{tabular}

\end{center}
\end{multicols}

\vtab[-5mm]
\begin{tabular}{*{2}{m{0.38\textwidth}}}
\begin{center}
\textcolor{NavyBlue}{\Large Different}
\end{center}
&
\begin{center}
\includegraphics[height=6.5cm]{../Comparisons/Vectors/inertia_tensor_of_3NFAACd_and_4NFAACl-3.png}
\end{center}
\end{tabular}

 \newpage

\vtab[-3cm]
\begin{center}
{\large FireTest \tab Número 288}
\end{center}
\begin{multicols}{2}
\begin{center}

Molecule A \
3NFAACe

\includegraphics[width=6cm]{../Comparisons/ImagesFromVMD/3NFAACe.png}

Inertia Tensor - Molecule A \\
\begin{tabular}{|c c c|}
524.508	 & 	15.8785	 & 	-4.02001	 \\
15.8785	 & 	1358.65	 & 	10.8723	 \\
-4.02001	 & 	10.8723	 & 	935.54
\end{tabular}

\vtab
 EingenVectors - Molecule A     \\
\begin{tabular}{|c c c|}
-0.999764	 & 	0.0191573	 & 	-0.0102761	 \\
0.0107588	 & 	0.025269	 & 	-0.999623	 \\
0.0188904	 & 	0.999497	 & 	0.0254692
\end{tabular}

\vtab
 EingenValues - Molecule A     \\
\begin{tabular}{|c c c|}
524.162	 & 	935.308	 & 	1359.22	 \\
\end{tabular}
\columnbreak

Molecule B \
3NFAACf

\includegraphics[width=6cm]{../Comparisons/ImagesFromVMD/3NFAACf.png}

Inertia Tensor - Molecule B \\
\begin{tabular}{|c c c|}
530.208	 & 	-1.63273	 & 	-7.84857	 \\
-1.63273	 & 	1360.76	 & 	-3.6411	 \\
-7.84857	 & 	-3.6411	 & 	915.396
\end{tabular}

\vtab
 EingenVectors - Molecule B     \\
\begin{tabular}{|c c c|}
-0.99979	 & 	-0.00205437	 & 	-0.0203824	 \\
-0.0203985	 & 	0.00810116	 & 	0.999759	 \\
-0.00188875	 & 	0.999965	 & 	-0.00814137
\end{tabular}

\vtab
 EingenValues - Molecule B     \\
\begin{tabular}{|c c c|}
530.045	 & 	915.527	 & 	1360.79	 \\
\end{tabular}

\end{center}
\end{multicols}

\vtab[-5mm]
\begin{tabular}{*{2}{m{0.38\textwidth}}}
\begin{center}
\textcolor{NavyBlue}{\Large Different}
\end{center}
&
\begin{center}
\includegraphics[height=6.5cm]{../Comparisons/Vectors/inertia_tensor_of_3NFAACe_and_3NFAACf.png}
\end{center}
\end{tabular}

 \newpage

\vtab[-3cm]
\begin{center}
{\large FireTest \tab Número 289}
\end{center}
\begin{multicols}{2}
\begin{center}

Molecule A \
3NFAACe

\includegraphics[width=6cm]{../Comparisons/ImagesFromVMD/3NFAACe.png}

Inertia Tensor - Molecule A \\
\begin{tabular}{|c c c|}
524.508	 & 	15.8785	 & 	-4.02001	 \\
15.8785	 & 	1358.65	 & 	10.8723	 \\
-4.02001	 & 	10.8723	 & 	935.54
\end{tabular}

\vtab
 EingenVectors - Molecule A     \\
\begin{tabular}{|c c c|}
-0.999764	 & 	0.0191573	 & 	-0.0102761	 \\
0.0107588	 & 	0.025269	 & 	-0.999623	 \\
0.0188904	 & 	0.999497	 & 	0.0254692
\end{tabular}

\vtab
 EingenValues - Molecule A     \\
\begin{tabular}{|c c c|}
524.162	 & 	935.308	 & 	1359.22	 \\
\end{tabular}
\columnbreak

Molecule B \
3NFAACg

\includegraphics[width=6cm]{../Comparisons/ImagesFromVMD/3NFAACg.png}

Inertia Tensor - Molecule B \\
\begin{tabular}{|c c c|}
532.891	 & 	-0.526938	 & 	-0.504713	 \\
-0.526938	 & 	918.454	 & 	2.35809	 \\
-0.504713	 & 	2.35809	 & 	1356.91
\end{tabular}

\vtab
 EingenVectors - Molecule B     \\
\begin{tabular}{|c c c|}
-0.999999	 & 	-0.00136295	 & 	-0.000608598	 \\
0.00135965	 & 	-0.999985	 & 	0.00537946	 \\
-0.000615921	 & 	0.00537863	 & 	0.999985
\end{tabular}

\vtab
 EingenValues - Molecule B     \\
\begin{tabular}{|c c c|}
532.89	 & 	918.442	 & 	1356.93	 \\
\end{tabular}

\end{center}
\end{multicols}

\vtab[-5mm]
\begin{tabular}{*{2}{m{0.38\textwidth}}}
\begin{center}
\textcolor{NavyBlue}{\Large Different}
\end{center}
&
\begin{center}
\includegraphics[height=6.5cm]{../Comparisons/Vectors/inertia_tensor_of_3NFAACe_and_3NFAACg.png}
\end{center}
\end{tabular}

 \newpage

\vtab[-3cm]
\begin{center}
{\large FireTest \tab Número 290}
\end{center}
\begin{multicols}{2}
\begin{center}

Molecule A \
3NFAACe

\includegraphics[width=6cm]{../Comparisons/ImagesFromVMD/3NFAACe.png}

Inertia Tensor - Molecule A \\
\begin{tabular}{|c c c|}
524.508	 & 	15.8785	 & 	-4.02001	 \\
15.8785	 & 	1358.65	 & 	10.8723	 \\
-4.02001	 & 	10.8723	 & 	935.54
\end{tabular}

\vtab
 EingenVectors - Molecule A     \\
\begin{tabular}{|c c c|}
-0.999764	 & 	0.0191573	 & 	-0.0102761	 \\
0.0107588	 & 	0.025269	 & 	-0.999623	 \\
0.0188904	 & 	0.999497	 & 	0.0254692
\end{tabular}

\vtab
 EingenValues - Molecule A     \\
\begin{tabular}{|c c c|}
524.162	 & 	935.308	 & 	1359.22	 \\
\end{tabular}
\columnbreak

Molecule B \
3NFAACh

\includegraphics[width=6cm]{../Comparisons/ImagesFromVMD/3NFAACh.png}

Inertia Tensor - Molecule B \\
\begin{tabular}{|c c c|}
528.718	 & 	-3.89066	 & 	-3.66882	 \\
-3.89066	 & 	924.911	 & 	2.93817	 \\
-3.66882	 & 	2.93817	 & 	1336.08
\end{tabular}

\vtab
 EingenVectors - Molecule B     \\
\begin{tabular}{|c c c|}
0.999942	 & 	0.00978475	 & 	0.00450803	 \\
0.00975199	 & 	-0.999926	 & 	0.00723269	 \\
-0.00457847	 & 	0.0071883	 & 	0.999964
\end{tabular}

\vtab
 EingenValues - Molecule B     \\
\begin{tabular}{|c c c|}
528.663	 & 	924.928	 & 	1336.12	 \\
\end{tabular}

\end{center}
\end{multicols}

\vtab[-5mm]
\begin{tabular}{*{2}{m{0.38\textwidth}}}
\begin{center}
\textcolor{NavyBlue}{\Large Different}
\end{center}
&
\begin{center}
\includegraphics[height=6.5cm]{../Comparisons/Vectors/inertia_tensor_of_3NFAACe_and_3NFAACh.png}
\end{center}
\end{tabular}

 \newpage

\vtab[-3cm]
\begin{center}
{\large FireTest \tab Número 291}
\end{center}
\begin{multicols}{2}
\begin{center}

Molecule A \
3NFAACe

\includegraphics[width=6cm]{../Comparisons/ImagesFromVMD/3NFAACe.png}

Inertia Tensor - Molecule A \\
\begin{tabular}{|c c c|}
524.508	 & 	15.8785	 & 	-4.02001	 \\
15.8785	 & 	1358.65	 & 	10.8723	 \\
-4.02001	 & 	10.8723	 & 	935.54
\end{tabular}

\vtab
 EingenVectors - Molecule A     \\
\begin{tabular}{|c c c|}
-0.999764	 & 	0.0191573	 & 	-0.0102761	 \\
0.0107588	 & 	0.025269	 & 	-0.999623	 \\
0.0188904	 & 	0.999497	 & 	0.0254692
\end{tabular}

\vtab
 EingenValues - Molecule A     \\
\begin{tabular}{|c c c|}
524.162	 & 	935.308	 & 	1359.22	 \\
\end{tabular}
\columnbreak

Molecule B \
3NFAACi

\includegraphics[width=6cm]{../Comparisons/ImagesFromVMD/3NFAACi.png}

Inertia Tensor - Molecule B \\
\begin{tabular}{|c c c|}
521.487	 & 	-1.22943	 & 	-2.71042	 \\
-1.22943	 & 	949.663	 & 	1.67006	 \\
-2.71042	 & 	1.67006	 & 	1346.16
\end{tabular}

\vtab
 EingenVectors - Molecule B     \\
\begin{tabular}{|c c c|}
0.999991	 & 	0.00285841	 & 	0.00328077	 \\
0.00284452	 & 	-0.999987	 & 	0.00423133	 \\
-0.00329282	 & 	0.00422196	 & 	0.999986
\end{tabular}

\vtab
 EingenValues - Molecule B     \\
\begin{tabular}{|c c c|}
521.475	 & 	949.659	 & 	1346.18	 \\
\end{tabular}

\end{center}
\end{multicols}

\vtab[-5mm]
\begin{tabular}{*{2}{m{0.38\textwidth}}}
\begin{center}
\textcolor{NavyBlue}{\Large Different}
\end{center}
&
\begin{center}
\includegraphics[height=6.5cm]{../Comparisons/Vectors/inertia_tensor_of_3NFAACe_and_3NFAACi.png}
\end{center}
\end{tabular}

 \newpage

\vtab[-3cm]
\begin{center}
{\large FireTest \tab Número 292}
\end{center}
\begin{multicols}{2}
\begin{center}

Molecule A \
3NFAACe

\includegraphics[width=6cm]{../Comparisons/ImagesFromVMD/3NFAACe.png}

Inertia Tensor - Molecule A \\
\begin{tabular}{|c c c|}
524.508	 & 	15.8785	 & 	-4.02001	 \\
15.8785	 & 	1358.65	 & 	10.8723	 \\
-4.02001	 & 	10.8723	 & 	935.54
\end{tabular}

\vtab
 EingenVectors - Molecule A     \\
\begin{tabular}{|c c c|}
-0.999764	 & 	0.0191573	 & 	-0.0102761	 \\
0.0107588	 & 	0.025269	 & 	-0.999623	 \\
0.0188904	 & 	0.999497	 & 	0.0254692
\end{tabular}

\vtab
 EingenValues - Molecule A     \\
\begin{tabular}{|c c c|}
524.162	 & 	935.308	 & 	1359.22	 \\
\end{tabular}
\columnbreak

Molecule B \
3NFAACj

\includegraphics[width=6cm]{../Comparisons/ImagesFromVMD/3NFAACj.png}

Inertia Tensor - Molecule B \\
\begin{tabular}{|c c c|}
533.789	 & 	-4.75521	 & 	-1.91525	 \\
-4.75521	 & 	920.091	 & 	2.28449	 \\
-1.91525	 & 	2.28449	 & 	1348.28
\end{tabular}

\vtab
 EingenVectors - Molecule B     \\
\begin{tabular}{|c c c|}
-0.999922	 & 	-0.0122929	 & 	-0.00231663	 \\
0.0122803	 & 	-0.99991	 & 	0.00539026	 \\
-0.00238268	 & 	0.00536139	 & 	0.999983
\end{tabular}

\vtab
 EingenValues - Molecule B     \\
\begin{tabular}{|c c c|}
533.726	 & 	920.137	 & 	1348.3	 \\
\end{tabular}

\end{center}
\end{multicols}

\vtab[-5mm]
\begin{tabular}{*{2}{m{0.38\textwidth}}}
\begin{center}
\textcolor{NavyBlue}{\Large Different}
\end{center}
&
\begin{center}
\includegraphics[height=6.5cm]{../Comparisons/Vectors/inertia_tensor_of_3NFAACe_and_3NFAACj.png}
\end{center}
\end{tabular}

 \newpage

\vtab[-3cm]
\begin{center}
{\large FireTest \tab Número 293}
\end{center}
\begin{multicols}{2}
\begin{center}

Molecule A \
3NFAACe

\includegraphics[width=6cm]{../Comparisons/ImagesFromVMD/3NFAACe.png}

Inertia Tensor - Molecule A \\
\begin{tabular}{|c c c|}
524.508	 & 	15.8785	 & 	-4.02001	 \\
15.8785	 & 	1358.65	 & 	10.8723	 \\
-4.02001	 & 	10.8723	 & 	935.54
\end{tabular}

\vtab
 EingenVectors - Molecule A     \\
\begin{tabular}{|c c c|}
-0.999764	 & 	0.0191573	 & 	-0.0102761	 \\
0.0107588	 & 	0.025269	 & 	-0.999623	 \\
0.0188904	 & 	0.999497	 & 	0.0254692
\end{tabular}

\vtab
 EingenValues - Molecule A     \\
\begin{tabular}{|c c c|}
524.162	 & 	935.308	 & 	1359.22	 \\
\end{tabular}
\columnbreak

Molecule B \
3NFAACk

\includegraphics[width=6cm]{../Comparisons/ImagesFromVMD/3NFAACk.png}

Inertia Tensor - Molecule B \\
\begin{tabular}{|c c c|}
534.899	 & 	-5.81418	 & 	-0.270063	 \\
-5.81418	 & 	913.263	 & 	2.4519	 \\
-0.270063	 & 	2.4519	 & 	1353.77
\end{tabular}

\vtab
 EingenVectors - Molecule B     \\
\begin{tabular}{|c c c|}
-0.999882	 & 	-0.0153593	 & 	-0.00028374	 \\
0.0153575	 & 	-0.999867	 & 	0.00557573	 \\
-0.000369342	 & 	0.00557072	 & 	0.999984
\end{tabular}

\vtab
 EingenValues - Molecule B     \\
\begin{tabular}{|c c c|}
534.81	 & 	913.339	 & 	1353.78	 \\
\end{tabular}

\end{center}
\end{multicols}

\vtab[-5mm]
\begin{tabular}{*{2}{m{0.38\textwidth}}}
\begin{center}
\textcolor{NavyBlue}{\Large Different}
\end{center}
&
\begin{center}
\includegraphics[height=6.5cm]{../Comparisons/Vectors/inertia_tensor_of_3NFAACe_and_3NFAACk.png}
\end{center}
\end{tabular}

 \newpage

\vtab[-3cm]
\begin{center}
{\large FireTest \tab Número 294}
\end{center}
\begin{multicols}{2}
\begin{center}

Molecule A \
3NFAACe

\includegraphics[width=6cm]{../Comparisons/ImagesFromVMD/3NFAACe.png}

Inertia Tensor - Molecule A \\
\begin{tabular}{|c c c|}
524.508	 & 	15.8785	 & 	-4.02001	 \\
15.8785	 & 	1358.65	 & 	10.8723	 \\
-4.02001	 & 	10.8723	 & 	935.54
\end{tabular}

\vtab
 EingenVectors - Molecule A     \\
\begin{tabular}{|c c c|}
-0.999764	 & 	0.0191573	 & 	-0.0102761	 \\
0.0107588	 & 	0.025269	 & 	-0.999623	 \\
0.0188904	 & 	0.999497	 & 	0.0254692
\end{tabular}

\vtab
 EingenValues - Molecule A     \\
\begin{tabular}{|c c c|}
524.162	 & 	935.308	 & 	1359.22	 \\
\end{tabular}
\columnbreak

Molecule B \
3NFAACl

\includegraphics[width=6cm]{../Comparisons/ImagesFromVMD/3NFAACl.png}

Inertia Tensor - Molecule B \\
\begin{tabular}{|c c c|}
531.723	 & 	3.03424	 & 	2.73426	 \\
3.03424	 & 	929.418	 & 	-1.84284	 \\
2.73426	 & 	-1.84284	 & 	1355.39
\end{tabular}

\vtab
 EingenVectors - Molecule B     \\
\begin{tabular}{|c c c|}
0.999965	 & 	-0.00764413	 & 	-0.00333648	 \\
-0.00765838	 & 	-0.999962	 & 	-0.00427708	 \\
0.00330366	 & 	-0.00430248	 & 	0.999985
\end{tabular}

\vtab
 EingenValues - Molecule B     \\
\begin{tabular}{|c c c|}
531.691	 & 	929.434	 & 	1355.4	 \\
\end{tabular}

\end{center}
\end{multicols}

\vtab[-5mm]
\begin{tabular}{*{2}{m{0.38\textwidth}}}
\begin{center}
\textcolor{NavyBlue}{\Large Different}
\end{center}
&
\begin{center}
\includegraphics[height=6.5cm]{../Comparisons/Vectors/inertia_tensor_of_3NFAACe_and_3NFAACl.png}
\end{center}
\end{tabular}

 \newpage

\vtab[-3cm]
\begin{center}
{\large FireTest \tab Número 295}
\end{center}
\begin{multicols}{2}
\begin{center}

Molecule A \
3NFAACe

\includegraphics[width=6cm]{../Comparisons/ImagesFromVMD/3NFAACe.png}

Inertia Tensor - Molecule A \\
\begin{tabular}{|c c c|}
524.508	 & 	15.8785	 & 	-4.02001	 \\
15.8785	 & 	1358.65	 & 	10.8723	 \\
-4.02001	 & 	10.8723	 & 	935.54
\end{tabular}

\vtab
 EingenVectors - Molecule A     \\
\begin{tabular}{|c c c|}
-0.999764	 & 	0.0191573	 & 	-0.0102761	 \\
0.0107588	 & 	0.025269	 & 	-0.999623	 \\
0.0188904	 & 	0.999497	 & 	0.0254692
\end{tabular}

\vtab
 EingenValues - Molecule A     \\
\begin{tabular}{|c c c|}
524.162	 & 	935.308	 & 	1359.22	 \\
\end{tabular}
\columnbreak

Molecule B \
3NFAACm

\includegraphics[width=6cm]{../Comparisons/ImagesFromVMD/3NFAACm.png}

Inertia Tensor - Molecule B \\
\begin{tabular}{|c c c|}
532.546	 & 	13.7854	 & 	-15.4626	 \\
13.7854	 & 	1354.87	 & 	11.5786	 \\
-15.4626	 & 	11.5786	 & 	929.101
\end{tabular}

\vtab
 EingenVectors - Molecule B     \\
\begin{tabular}{|c c c|}
-0.999075	 & 	0.0172851	 & 	-0.0393769	 \\
0.0398168	 & 	0.0258942	 & 	-0.998871	 \\
-0.016246	 & 	-0.999515	 & 	-0.0265584
\end{tabular}

\vtab
 EingenValues - Molecule B     \\
\begin{tabular}{|c c c|}
531.698	 & 	929.417	 & 	1355.4	 \\
\end{tabular}

\end{center}
\end{multicols}

\vtab[-5mm]
\begin{tabular}{*{2}{m{0.38\textwidth}}}
\begin{center}
\textcolor{NavyBlue}{\Large Different}
\end{center}
&
\begin{center}
\includegraphics[height=6.5cm]{../Comparisons/Vectors/inertia_tensor_of_3NFAACe_and_3NFAACm.png}
\end{center}
\end{tabular}

 \newpage

\vtab[-3cm]
\begin{center}
{\large FireTest \tab Número 296}
\end{center}
\begin{multicols}{2}
\begin{center}

Molecule A \
3NFAACe

\includegraphics[width=6cm]{../Comparisons/ImagesFromVMD/3NFAACe.png}

Inertia Tensor - Molecule A \\
\begin{tabular}{|c c c|}
524.508	 & 	15.8785	 & 	-4.02001	 \\
15.8785	 & 	1358.65	 & 	10.8723	 \\
-4.02001	 & 	10.8723	 & 	935.54
\end{tabular}

\vtab
 EingenVectors - Molecule A     \\
\begin{tabular}{|c c c|}
-0.999764	 & 	0.0191573	 & 	-0.0102761	 \\
0.0107588	 & 	0.025269	 & 	-0.999623	 \\
0.0188904	 & 	0.999497	 & 	0.0254692
\end{tabular}

\vtab
 EingenValues - Molecule A     \\
\begin{tabular}{|c c c|}
524.162	 & 	935.308	 & 	1359.22	 \\
\end{tabular}
\columnbreak

Molecule B \
3NFAACn

\includegraphics[width=6cm]{../Comparisons/ImagesFromVMD/3NFAACn.png}

Inertia Tensor - Molecule B \\
\begin{tabular}{|c c c|}
531.896	 & 	3.78027	 & 	-13.1151	 \\
3.78027	 & 	1353.2	 & 	-7.47403	 \\
-13.1151	 & 	-7.47403	 & 	912.989
\end{tabular}

\vtab
 EingenVectors - Molecule B     \\
\begin{tabular}{|c c c|}
0.999403	 & 	-0.00428573	 & 	0.0342679	 \\
0.0341891	 & 	-0.0172718	 & 	-0.999266	 \\
0.00487445	 & 	0.999842	 & 	-0.017115
\end{tabular}

\vtab
 EingenValues - Molecule B     \\
\begin{tabular}{|c c c|}
531.43	 & 	913.309	 & 	1353.35	 \\
\end{tabular}

\end{center}
\end{multicols}

\vtab[-5mm]
\begin{tabular}{*{2}{m{0.38\textwidth}}}
\begin{center}
\textcolor{NavyBlue}{\Large Different}
\end{center}
&
\begin{center}
\includegraphics[height=6.5cm]{../Comparisons/Vectors/inertia_tensor_of_3NFAACe_and_3NFAACn.png}
\end{center}
\end{tabular}

 \newpage

\vtab[-3cm]
\begin{center}
{\large FireTest \tab Número 297}
\end{center}
\begin{multicols}{2}
\begin{center}

Molecule A \
3NFAACe

\includegraphics[width=6cm]{../Comparisons/ImagesFromVMD/3NFAACe.png}

Inertia Tensor - Molecule A \\
\begin{tabular}{|c c c|}
524.508	 & 	15.8785	 & 	-4.02001	 \\
15.8785	 & 	1358.65	 & 	10.8723	 \\
-4.02001	 & 	10.8723	 & 	935.54
\end{tabular}

\vtab
 EingenVectors - Molecule A     \\
\begin{tabular}{|c c c|}
-0.999764	 & 	0.0191573	 & 	-0.0102761	 \\
0.0107588	 & 	0.025269	 & 	-0.999623	 \\
0.0188904	 & 	0.999497	 & 	0.0254692
\end{tabular}

\vtab
 EingenValues - Molecule A     \\
\begin{tabular}{|c c c|}
524.162	 & 	935.308	 & 	1359.22	 \\
\end{tabular}
\columnbreak

Molecule B \
4NFAACa

\includegraphics[width=6cm]{../Comparisons/ImagesFromVMD/4NFAACa.png}

Inertia Tensor - Molecule B \\
\begin{tabular}{|c c c|}
479.392	 & 	3.27131	 & 	4.22557	 \\
3.27131	 & 	1242.39	 & 	-0.852684	 \\
4.22557	 & 	-0.852684	 & 	1647.37
\end{tabular}

\vtab
 EingenVectors - Molecule B     \\
\begin{tabular}{|c c c|}
0.999984	 & 	-0.00429123	 & 	-0.00362083	 \\
-0.00429871	 & 	-0.999989	 & 	-0.0020607	 \\
0.00361195	 & 	-0.00207623	 & 	0.999991
\end{tabular}

\vtab
 EingenValues - Molecule B     \\
\begin{tabular}{|c c c|}
479.363	 & 	1242.41	 & 	1647.39	 \\
\end{tabular}

\end{center}
\end{multicols}

\vtab[-5mm]
\begin{tabular}{*{2}{m{0.38\textwidth}}}
\begin{center}
\textcolor{NavyBlue}{\Large Different}
\end{center}
&
\begin{center}
\includegraphics[height=6.5cm]{../Comparisons/Vectors/inertia_tensor_of_3NFAACe_and_4NFAACa.png}
\end{center}
\end{tabular}

 \newpage

\vtab[-3cm]
\begin{center}
{\large FireTest \tab Número 298}
\end{center}
\begin{multicols}{2}
\begin{center}

Molecule A \
3NFAACe

\includegraphics[width=6cm]{../Comparisons/ImagesFromVMD/3NFAACe.png}

Inertia Tensor - Molecule A \\
\begin{tabular}{|c c c|}
524.508	 & 	15.8785	 & 	-4.02001	 \\
15.8785	 & 	1358.65	 & 	10.8723	 \\
-4.02001	 & 	10.8723	 & 	935.54
\end{tabular}

\vtab
 EingenVectors - Molecule A     \\
\begin{tabular}{|c c c|}
-0.999764	 & 	0.0191573	 & 	-0.0102761	 \\
0.0107588	 & 	0.025269	 & 	-0.999623	 \\
0.0188904	 & 	0.999497	 & 	0.0254692
\end{tabular}

\vtab
 EingenValues - Molecule A     \\
\begin{tabular}{|c c c|}
524.162	 & 	935.308	 & 	1359.22	 \\
\end{tabular}
\columnbreak

Molecule B \
4NFAACb

\includegraphics[width=6cm]{../Comparisons/ImagesFromVMD/4NFAACb.png}

Inertia Tensor - Molecule B \\
\begin{tabular}{|c c c|}
479.338	 & 	3.27331	 & 	-4.22553	 \\
3.27331	 & 	1242.4	 & 	0.852083	 \\
-4.22553	 & 	0.852083	 & 	1647.3
\end{tabular}

\vtab
 EingenVectors - Molecule B     \\
\begin{tabular}{|c c c|}
0.999984	 & 	-0.00429353	 & 	0.00362086	 \\
-0.004301	 & 	-0.999989	 & 	0.00205959	 \\
-0.00361198	 & 	0.00207513	 & 	0.999991
\end{tabular}

\vtab
 EingenValues - Molecule B     \\
\begin{tabular}{|c c c|}
479.308	 & 	1242.41	 & 	1647.31	 \\
\end{tabular}

\end{center}
\end{multicols}

\vtab[-5mm]
\begin{tabular}{*{2}{m{0.38\textwidth}}}
\begin{center}
\textcolor{NavyBlue}{\Large Different}
\end{center}
&
\begin{center}
\includegraphics[height=6.5cm]{../Comparisons/Vectors/inertia_tensor_of_3NFAACe_and_4NFAACb.png}
\end{center}
\end{tabular}

 \newpage

\vtab[-3cm]
\begin{center}
{\large FireTest \tab Número 299}
\end{center}
\begin{multicols}{2}
\begin{center}

Molecule A \
3NFAACe

\includegraphics[width=6cm]{../Comparisons/ImagesFromVMD/3NFAACe.png}

Inertia Tensor - Molecule A \\
\begin{tabular}{|c c c|}
524.508	 & 	15.8785	 & 	-4.02001	 \\
15.8785	 & 	1358.65	 & 	10.8723	 \\
-4.02001	 & 	10.8723	 & 	935.54
\end{tabular}

\vtab
 EingenVectors - Molecule A     \\
\begin{tabular}{|c c c|}
-0.999764	 & 	0.0191573	 & 	-0.0102761	 \\
0.0107588	 & 	0.025269	 & 	-0.999623	 \\
0.0188904	 & 	0.999497	 & 	0.0254692
\end{tabular}

\vtab
 EingenValues - Molecule A     \\
\begin{tabular}{|c c c|}
524.162	 & 	935.308	 & 	1359.22	 \\
\end{tabular}
\columnbreak

Molecule B \
4NFAACc

\includegraphics[width=6cm]{../Comparisons/ImagesFromVMD/4NFAACc.png}

Inertia Tensor - Molecule B \\
\begin{tabular}{|c c c|}
482.067	 & 	-5.39474	 & 	-1.35857	 \\
-5.39474	 & 	1240.3	 & 	-2.54035	 \\
-1.35857	 & 	-2.54035	 & 	1647.06
\end{tabular}

\vtab
 EingenVectors - Molecule B     \\
\begin{tabular}{|c c c|}
-0.999974	 & 	-0.00711826	 & 	-0.0011816	 \\
0.00712547	 & 	-0.999955	 & 	-0.00622156	 \\
-0.00113726	 & 	-0.00622982	 & 	0.99998
\end{tabular}

\vtab
 EingenValues - Molecule B     \\
\begin{tabular}{|c c c|}
482.027	 & 	1240.32	 & 	1647.08	 \\
\end{tabular}

\end{center}
\end{multicols}

\vtab[-5mm]
\begin{tabular}{*{2}{m{0.38\textwidth}}}
\begin{center}
\textcolor{NavyBlue}{\Large Different}
\end{center}
&
\begin{center}
\includegraphics[height=6.5cm]{../Comparisons/Vectors/inertia_tensor_of_3NFAACe_and_4NFAACc.png}
\end{center}
\end{tabular}

 \newpage

\vtab[-3cm]
\begin{center}
{\large FireTest \tab Número 300}
\end{center}
\begin{multicols}{2}
\begin{center}

Molecule A \
3NFAACe

\includegraphics[width=6cm]{../Comparisons/ImagesFromVMD/3NFAACe.png}

Inertia Tensor - Molecule A \\
\begin{tabular}{|c c c|}
524.508	 & 	15.8785	 & 	-4.02001	 \\
15.8785	 & 	1358.65	 & 	10.8723	 \\
-4.02001	 & 	10.8723	 & 	935.54
\end{tabular}

\vtab
 EingenVectors - Molecule A     \\
\begin{tabular}{|c c c|}
-0.999764	 & 	0.0191573	 & 	-0.0102761	 \\
0.0107588	 & 	0.025269	 & 	-0.999623	 \\
0.0188904	 & 	0.999497	 & 	0.0254692
\end{tabular}

\vtab
 EingenValues - Molecule A     \\
\begin{tabular}{|c c c|}
524.162	 & 	935.308	 & 	1359.22	 \\
\end{tabular}
\columnbreak

Molecule B \
4NFAACd

\includegraphics[width=6cm]{../Comparisons/ImagesFromVMD/4NFAACd.png}

Inertia Tensor - Molecule B \\
\begin{tabular}{|c c c|}
491.672	 & 	0.24486	 & 	-3.10016	 \\
0.24486	 & 	1231.15	 & 	2.19965	 \\
-3.10016	 & 	2.19965	 & 	1650.11
\end{tabular}

\vtab
 EingenVectors - Molecule B     \\
\begin{tabular}{|c c c|}
0.999996	 & 	-0.000339081	 & 	0.00267677	 \\
-0.000353124	 & 	-0.999986	 & 	0.00524747	 \\
-0.00267495	 & 	0.0052484	 & 	0.999983
\end{tabular}

\vtab
 EingenValues - Molecule B     \\
\begin{tabular}{|c c c|}
491.663	 & 	1231.14	 & 	1650.13	 \\
\end{tabular}

\end{center}
\end{multicols}

\vtab[-5mm]
\begin{tabular}{*{2}{m{0.38\textwidth}}}
\begin{center}
\textcolor{NavyBlue}{\Large Different}
\end{center}
&
\begin{center}
\includegraphics[height=6.5cm]{../Comparisons/Vectors/inertia_tensor_of_3NFAACe_and_4NFAACd.png}
\end{center}
\end{tabular}

 \newpage

\vtab[-3cm]
\begin{center}
{\large FireTest \tab Número 301}
\end{center}
\begin{multicols}{2}
\begin{center}

Molecule A \
3NFAACe

\includegraphics[width=6cm]{../Comparisons/ImagesFromVMD/3NFAACe.png}

Inertia Tensor - Molecule A \\
\begin{tabular}{|c c c|}
524.508	 & 	15.8785	 & 	-4.02001	 \\
15.8785	 & 	1358.65	 & 	10.8723	 \\
-4.02001	 & 	10.8723	 & 	935.54
\end{tabular}

\vtab
 EingenVectors - Molecule A     \\
\begin{tabular}{|c c c|}
-0.999764	 & 	0.0191573	 & 	-0.0102761	 \\
0.0107588	 & 	0.025269	 & 	-0.999623	 \\
0.0188904	 & 	0.999497	 & 	0.0254692
\end{tabular}

\vtab
 EingenValues - Molecule A     \\
\begin{tabular}{|c c c|}
524.162	 & 	935.308	 & 	1359.22	 \\
\end{tabular}
\columnbreak

Molecule B \
4NFAACe

\includegraphics[width=6cm]{../Comparisons/ImagesFromVMD/4NFAACe.png}

Inertia Tensor - Molecule B \\
\begin{tabular}{|c c c|}
489.025	 & 	-0.430035	 & 	3.98876	 \\
-0.430035	 & 	1233.71	 & 	-2.06505	 \\
3.98876	 & 	-2.06505	 & 	1641.79
\end{tabular}

\vtab
 EingenVectors - Molecule B     \\
\begin{tabular}{|c c c|}
0.999994	 & 	0.000567863	 & 	-0.00345908	 \\
0.000550336	 & 	-0.999987	 & 	-0.00506565	 \\
0.00346192	 & 	-0.00506372	 & 	0.999981
\end{tabular}

\vtab
 EingenValues - Molecule B     \\
\begin{tabular}{|c c c|}
489.011	 & 	1233.7	 & 	1641.81	 \\
\end{tabular}

\end{center}
\end{multicols}

\vtab[-5mm]
\begin{tabular}{*{2}{m{0.38\textwidth}}}
\begin{center}
\textcolor{NavyBlue}{\Large Different}
\end{center}
&
\begin{center}
\includegraphics[height=6.5cm]{../Comparisons/Vectors/inertia_tensor_of_3NFAACe_and_4NFAACe.png}
\end{center}
\end{tabular}

 \newpage

\vtab[-3cm]
\begin{center}
{\large FireTest \tab Número 302}
\end{center}
\begin{multicols}{2}
\begin{center}

Molecule A \
3NFAACe

\includegraphics[width=6cm]{../Comparisons/ImagesFromVMD/3NFAACe.png}

Inertia Tensor - Molecule A \\
\begin{tabular}{|c c c|}
524.508	 & 	15.8785	 & 	-4.02001	 \\
15.8785	 & 	1358.65	 & 	10.8723	 \\
-4.02001	 & 	10.8723	 & 	935.54
\end{tabular}

\vtab
 EingenVectors - Molecule A     \\
\begin{tabular}{|c c c|}
-0.999764	 & 	0.0191573	 & 	-0.0102761	 \\
0.0107588	 & 	0.025269	 & 	-0.999623	 \\
0.0188904	 & 	0.999497	 & 	0.0254692
\end{tabular}

\vtab
 EingenValues - Molecule A     \\
\begin{tabular}{|c c c|}
524.162	 & 	935.308	 & 	1359.22	 \\
\end{tabular}
\columnbreak

Molecule B \
4NFAACf

\includegraphics[width=6cm]{../Comparisons/ImagesFromVMD/4NFAACf.png}

Inertia Tensor - Molecule B \\
\begin{tabular}{|c c c|}
509.683	 & 	2.80651	 & 	-1.91422	 \\
2.80651	 & 	1219.11	 & 	2.66132	 \\
-1.91422	 & 	2.66132	 & 	1681.17
\end{tabular}

\vtab
 EingenVectors - Molecule B     \\
\begin{tabular}{|c c c|}
-0.999991	 & 	0.00396206	 & 	-0.00164298	 \\
-0.00397143	 & 	-0.999976	 & 	0.0057431	 \\
-0.00162019	 & 	0.00574957	 & 	0.999982
\end{tabular}

\vtab
 EingenValues - Molecule B     \\
\begin{tabular}{|c c c|}
509.668	 & 	1219.11	 & 	1681.18	 \\
\end{tabular}

\end{center}
\end{multicols}

\vtab[-5mm]
\begin{tabular}{*{2}{m{0.38\textwidth}}}
\begin{center}
\textcolor{NavyBlue}{\Large Different}
\end{center}
&
\begin{center}
\includegraphics[height=6.5cm]{../Comparisons/Vectors/inertia_tensor_of_3NFAACe_and_4NFAACf.png}
\end{center}
\end{tabular}

 \newpage

\vtab[-3cm]
\begin{center}
{\large FireTest \tab Número 303}
\end{center}
\begin{multicols}{2}
\begin{center}

Molecule A \
3NFAACe

\includegraphics[width=6cm]{../Comparisons/ImagesFromVMD/3NFAACe.png}

Inertia Tensor - Molecule A \\
\begin{tabular}{|c c c|}
524.508	 & 	15.8785	 & 	-4.02001	 \\
15.8785	 & 	1358.65	 & 	10.8723	 \\
-4.02001	 & 	10.8723	 & 	935.54
\end{tabular}

\vtab
 EingenVectors - Molecule A     \\
\begin{tabular}{|c c c|}
-0.999764	 & 	0.0191573	 & 	-0.0102761	 \\
0.0107588	 & 	0.025269	 & 	-0.999623	 \\
0.0188904	 & 	0.999497	 & 	0.0254692
\end{tabular}

\vtab
 EingenValues - Molecule A     \\
\begin{tabular}{|c c c|}
524.162	 & 	935.308	 & 	1359.22	 \\
\end{tabular}
\columnbreak

Molecule B \
4NFAACg

\includegraphics[width=6cm]{../Comparisons/ImagesFromVMD/4NFAACg.png}

Inertia Tensor - Molecule B \\
\begin{tabular}{|c c c|}
513.78	 & 	4.51917	 & 	0.266555	 \\
4.51917	 & 	1208.04	 & 	-1.18628	 \\
0.266555	 & 	-1.18628	 & 	1700.9
\end{tabular}

\vtab
 EingenVectors - Molecule B     \\
\begin{tabular}{|c c c|}
-0.999979	 & 	0.00650929	 & 	0.000231034	 \\
-0.00650983	 & 	-0.999976	 & 	-0.00240351	 \\
0.000215383	 & 	-0.00240496	 & 	0.999997
\end{tabular}

\vtab
 EingenValues - Molecule B     \\
\begin{tabular}{|c c c|}
513.751	 & 	1208.07	 & 	1700.9	 \\
\end{tabular}

\end{center}
\end{multicols}

\vtab[-5mm]
\begin{tabular}{*{2}{m{0.38\textwidth}}}
\begin{center}
\textcolor{NavyBlue}{\Large Different}
\end{center}
&
\begin{center}
\includegraphics[height=6.5cm]{../Comparisons/Vectors/inertia_tensor_of_3NFAACe_and_4NFAACg.png}
\end{center}
\end{tabular}

 \newpage

\vtab[-3cm]
\begin{center}
{\large FireTest \tab Número 304}
\end{center}
\begin{multicols}{2}
\begin{center}

Molecule A \
3NFAACe

\includegraphics[width=6cm]{../Comparisons/ImagesFromVMD/3NFAACe.png}

Inertia Tensor - Molecule A \\
\begin{tabular}{|c c c|}
524.508	 & 	15.8785	 & 	-4.02001	 \\
15.8785	 & 	1358.65	 & 	10.8723	 \\
-4.02001	 & 	10.8723	 & 	935.54
\end{tabular}

\vtab
 EingenVectors - Molecule A     \\
\begin{tabular}{|c c c|}
-0.999764	 & 	0.0191573	 & 	-0.0102761	 \\
0.0107588	 & 	0.025269	 & 	-0.999623	 \\
0.0188904	 & 	0.999497	 & 	0.0254692
\end{tabular}

\vtab
 EingenValues - Molecule A     \\
\begin{tabular}{|c c c|}
524.162	 & 	935.308	 & 	1359.22	 \\
\end{tabular}
\columnbreak

Molecule B \
4NFAACi

\includegraphics[width=6cm]{../Comparisons/ImagesFromVMD/4NFAACi.png}

Inertia Tensor - Molecule B \\
\begin{tabular}{|c c c|}
502.43	 & 	-0.602691	 & 	-4.86988	 \\
-0.602691	 & 	1232.26	 & 	0.407295	 \\
-4.86988	 & 	0.407295	 & 	1676
\end{tabular}

\vtab
 EingenVectors - Molecule B     \\
\begin{tabular}{|c c c|}
0.999991	 & 	0.000823447	 & 	0.00414923	 \\
0.000819608	 & 	-0.999999	 & 	0.00092687	 \\
-0.00414999	 & 	0.000923461	 & 	0.999991
\end{tabular}

\vtab
 EingenValues - Molecule B     \\
\begin{tabular}{|c c c|}
502.409	 & 	1232.26	 & 	1676.02	 \\
\end{tabular}

\end{center}
\end{multicols}

\vtab[-5mm]
\begin{tabular}{*{2}{m{0.38\textwidth}}}
\begin{center}
\textcolor{NavyBlue}{\Large Different}
\end{center}
&
\begin{center}
\includegraphics[height=6.5cm]{../Comparisons/Vectors/inertia_tensor_of_3NFAACe_and_4NFAACi.png}
\end{center}
\end{tabular}

 \newpage

\vtab[-3cm]
\begin{center}
{\large FireTest \tab Número 305}
\end{center}
\begin{multicols}{2}
\begin{center}

Molecule A \
3NFAACe

\includegraphics[width=6cm]{../Comparisons/ImagesFromVMD/3NFAACe.png}

Inertia Tensor - Molecule A \\
\begin{tabular}{|c c c|}
524.508	 & 	15.8785	 & 	-4.02001	 \\
15.8785	 & 	1358.65	 & 	10.8723	 \\
-4.02001	 & 	10.8723	 & 	935.54
\end{tabular}

\vtab
 EingenVectors - Molecule A     \\
\begin{tabular}{|c c c|}
-0.999764	 & 	0.0191573	 & 	-0.0102761	 \\
0.0107588	 & 	0.025269	 & 	-0.999623	 \\
0.0188904	 & 	0.999497	 & 	0.0254692
\end{tabular}

\vtab
 EingenValues - Molecule A     \\
\begin{tabular}{|c c c|}
524.162	 & 	935.308	 & 	1359.22	 \\
\end{tabular}
\columnbreak

Molecule B \
4NFAACj

\includegraphics[width=6cm]{../Comparisons/ImagesFromVMD/4NFAACj.png}

Inertia Tensor - Molecule B \\
\begin{tabular}{|c c c|}
510.047	 & 	9.97005	 & 	-3.6306	 \\
9.97005	 & 	1225.52	 & 	-0.981092	 \\
-3.6306	 & 	-0.981092	 & 	1680.82
\end{tabular}

\vtab
 EingenVectors - Molecule B     \\
\begin{tabular}{|c c c|}
-0.999898	 & 	0.0139264	 & 	-0.00308865	 \\
0.0139195	 & 	0.999901	 & 	0.00226627	 \\
-0.00311991	 & 	-0.00222305	 & 	0.999993
\end{tabular}

\vtab
 EingenValues - Molecule B     \\
\begin{tabular}{|c c c|}
509.897	 & 	1225.65	 & 	1680.83	 \\
\end{tabular}

\end{center}
\end{multicols}

\vtab[-5mm]
\begin{tabular}{*{2}{m{0.38\textwidth}}}
\begin{center}
\textcolor{NavyBlue}{\Large Different}
\end{center}
&
\begin{center}
\includegraphics[height=6.5cm]{../Comparisons/Vectors/inertia_tensor_of_3NFAACe_and_4NFAACj.png}
\end{center}
\end{tabular}

 \newpage

\vtab[-3cm]
\begin{center}
{\large FireTest \tab Número 306}
\end{center}
\begin{multicols}{2}
\begin{center}

Molecule A \
3NFAACe

\includegraphics[width=6cm]{../Comparisons/ImagesFromVMD/3NFAACe.png}

Inertia Tensor - Molecule A \\
\begin{tabular}{|c c c|}
524.508	 & 	15.8785	 & 	-4.02001	 \\
15.8785	 & 	1358.65	 & 	10.8723	 \\
-4.02001	 & 	10.8723	 & 	935.54
\end{tabular}

\vtab
 EingenVectors - Molecule A     \\
\begin{tabular}{|c c c|}
-0.999764	 & 	0.0191573	 & 	-0.0102761	 \\
0.0107588	 & 	0.025269	 & 	-0.999623	 \\
0.0188904	 & 	0.999497	 & 	0.0254692
\end{tabular}

\vtab
 EingenValues - Molecule A     \\
\begin{tabular}{|c c c|}
524.162	 & 	935.308	 & 	1359.22	 \\
\end{tabular}
\columnbreak

Molecule B \
4NFAACl-3

\includegraphics[width=6cm]{../Comparisons/ImagesFromVMD/4NFAACl-3.png}

Inertia Tensor - Molecule B \\
\begin{tabular}{|c c c|}
506.608	 & 	0.709539	 & 	-0.555426	 \\
0.709539	 & 	1222.37	 & 	-2.84005	 \\
-0.555426	 & 	-2.84005	 & 	1678.41
\end{tabular}

\vtab
 EingenVectors - Molecule B     \\
\begin{tabular}{|c c c|}
-0.999999	 & 	0.000989428	 & 	-0.000471595	 \\
-0.000986471	 & 	-0.99998	 & 	-0.00622856	 \\
-0.000477748	 & 	-0.00622809	 & 	0.99998
\end{tabular}

\vtab
 EingenValues - Molecule B     \\
\begin{tabular}{|c c c|}
506.607	 & 	1222.36	 & 	1678.43	 \\
\end{tabular}

\end{center}
\end{multicols}

\vtab[-5mm]
\begin{tabular}{*{2}{m{0.38\textwidth}}}
\begin{center}
\textcolor{NavyBlue}{\Large Different}
\end{center}
&
\begin{center}
\includegraphics[height=6.5cm]{../Comparisons/Vectors/inertia_tensor_of_3NFAACe_and_4NFAACl-3.png}
\end{center}
\end{tabular}

 \newpage

\vtab[-3cm]
\begin{center}
{\large FireTest \tab Número 307}
\end{center}
\begin{multicols}{2}
\begin{center}

Molecule A \
3NFAACf

\includegraphics[width=6cm]{../Comparisons/ImagesFromVMD/3NFAACf.png}

Inertia Tensor - Molecule A \\
\begin{tabular}{|c c c|}
530.208	 & 	-1.63273	 & 	-7.84857	 \\
-1.63273	 & 	1360.76	 & 	-3.6411	 \\
-7.84857	 & 	-3.6411	 & 	915.396
\end{tabular}

\vtab
 EingenVectors - Molecule A     \\
\begin{tabular}{|c c c|}
-0.99979	 & 	-0.00205437	 & 	-0.0203824	 \\
-0.0203985	 & 	0.00810116	 & 	0.999759	 \\
-0.00188875	 & 	0.999965	 & 	-0.00814137
\end{tabular}

\vtab
 EingenValues - Molecule A     \\
\begin{tabular}{|c c c|}
530.045	 & 	915.527	 & 	1360.79	 \\
\end{tabular}
\columnbreak

Molecule B \
3NFAACg

\includegraphics[width=6cm]{../Comparisons/ImagesFromVMD/3NFAACg.png}

Inertia Tensor - Molecule B \\
\begin{tabular}{|c c c|}
532.891	 & 	-0.526938	 & 	-0.504713	 \\
-0.526938	 & 	918.454	 & 	2.35809	 \\
-0.504713	 & 	2.35809	 & 	1356.91
\end{tabular}

\vtab
 EingenVectors - Molecule B     \\
\begin{tabular}{|c c c|}
-0.999999	 & 	-0.00136295	 & 	-0.000608598	 \\
0.00135965	 & 	-0.999985	 & 	0.00537946	 \\
-0.000615921	 & 	0.00537863	 & 	0.999985
\end{tabular}

\vtab
 EingenValues - Molecule B     \\
\begin{tabular}{|c c c|}
532.89	 & 	918.442	 & 	1356.93	 \\
\end{tabular}

\end{center}
\end{multicols}

\vtab[-5mm]
\begin{tabular}{*{2}{m{0.38\textwidth}}}
\begin{center}
\textcolor{NavyBlue}{\Large Different}
\end{center}
&
\begin{center}
\includegraphics[height=6.5cm]{../Comparisons/Vectors/inertia_tensor_of_3NFAACf_and_3NFAACg.png}
\end{center}
\end{tabular}

 \newpage

\vtab[-3cm]
\begin{center}
{\large FireTest \tab Número 308}
\end{center}
\begin{multicols}{2}
\begin{center}

Molecule A \
3NFAACf

\includegraphics[width=6cm]{../Comparisons/ImagesFromVMD/3NFAACf.png}

Inertia Tensor - Molecule A \\
\begin{tabular}{|c c c|}
530.208	 & 	-1.63273	 & 	-7.84857	 \\
-1.63273	 & 	1360.76	 & 	-3.6411	 \\
-7.84857	 & 	-3.6411	 & 	915.396
\end{tabular}

\vtab
 EingenVectors - Molecule A     \\
\begin{tabular}{|c c c|}
-0.99979	 & 	-0.00205437	 & 	-0.0203824	 \\
-0.0203985	 & 	0.00810116	 & 	0.999759	 \\
-0.00188875	 & 	0.999965	 & 	-0.00814137
\end{tabular}

\vtab
 EingenValues - Molecule A     \\
\begin{tabular}{|c c c|}
530.045	 & 	915.527	 & 	1360.79	 \\
\end{tabular}
\columnbreak

Molecule B \
3NFAACh

\includegraphics[width=6cm]{../Comparisons/ImagesFromVMD/3NFAACh.png}

Inertia Tensor - Molecule B \\
\begin{tabular}{|c c c|}
528.718	 & 	-3.89066	 & 	-3.66882	 \\
-3.89066	 & 	924.911	 & 	2.93817	 \\
-3.66882	 & 	2.93817	 & 	1336.08
\end{tabular}

\vtab
 EingenVectors - Molecule B     \\
\begin{tabular}{|c c c|}
0.999942	 & 	0.00978475	 & 	0.00450803	 \\
0.00975199	 & 	-0.999926	 & 	0.00723269	 \\
-0.00457847	 & 	0.0071883	 & 	0.999964
\end{tabular}

\vtab
 EingenValues - Molecule B     \\
\begin{tabular}{|c c c|}
528.663	 & 	924.928	 & 	1336.12	 \\
\end{tabular}

\end{center}
\end{multicols}

\vtab[-5mm]
\begin{tabular}{*{2}{m{0.38\textwidth}}}
\begin{center}
\textcolor{NavyBlue}{\Large Different}
\end{center}
&
\begin{center}
\includegraphics[height=6.5cm]{../Comparisons/Vectors/inertia_tensor_of_3NFAACf_and_3NFAACh.png}
\end{center}
\end{tabular}

 \newpage

\vtab[-3cm]
\begin{center}
{\large FireTest \tab Número 309}
\end{center}
\begin{multicols}{2}
\begin{center}

Molecule A \
3NFAACf

\includegraphics[width=6cm]{../Comparisons/ImagesFromVMD/3NFAACf.png}

Inertia Tensor - Molecule A \\
\begin{tabular}{|c c c|}
530.208	 & 	-1.63273	 & 	-7.84857	 \\
-1.63273	 & 	1360.76	 & 	-3.6411	 \\
-7.84857	 & 	-3.6411	 & 	915.396
\end{tabular}

\vtab
 EingenVectors - Molecule A     \\
\begin{tabular}{|c c c|}
-0.99979	 & 	-0.00205437	 & 	-0.0203824	 \\
-0.0203985	 & 	0.00810116	 & 	0.999759	 \\
-0.00188875	 & 	0.999965	 & 	-0.00814137
\end{tabular}

\vtab
 EingenValues - Molecule A     \\
\begin{tabular}{|c c c|}
530.045	 & 	915.527	 & 	1360.79	 \\
\end{tabular}
\columnbreak

Molecule B \
3NFAACi

\includegraphics[width=6cm]{../Comparisons/ImagesFromVMD/3NFAACi.png}

Inertia Tensor - Molecule B \\
\begin{tabular}{|c c c|}
521.487	 & 	-1.22943	 & 	-2.71042	 \\
-1.22943	 & 	949.663	 & 	1.67006	 \\
-2.71042	 & 	1.67006	 & 	1346.16
\end{tabular}

\vtab
 EingenVectors - Molecule B     \\
\begin{tabular}{|c c c|}
0.999991	 & 	0.00285841	 & 	0.00328077	 \\
0.00284452	 & 	-0.999987	 & 	0.00423133	 \\
-0.00329282	 & 	0.00422196	 & 	0.999986
\end{tabular}

\vtab
 EingenValues - Molecule B     \\
\begin{tabular}{|c c c|}
521.475	 & 	949.659	 & 	1346.18	 \\
\end{tabular}

\end{center}
\end{multicols}

\vtab[-5mm]
\begin{tabular}{*{2}{m{0.38\textwidth}}}
\begin{center}
\textcolor{NavyBlue}{\Large Different}
\end{center}
&
\begin{center}
\includegraphics[height=6.5cm]{../Comparisons/Vectors/inertia_tensor_of_3NFAACf_and_3NFAACi.png}
\end{center}
\end{tabular}

 \newpage

\vtab[-3cm]
\begin{center}
{\large FireTest \tab Número 310}
\end{center}
\begin{multicols}{2}
\begin{center}

Molecule A \
3NFAACf

\includegraphics[width=6cm]{../Comparisons/ImagesFromVMD/3NFAACf.png}

Inertia Tensor - Molecule A \\
\begin{tabular}{|c c c|}
530.208	 & 	-1.63273	 & 	-7.84857	 \\
-1.63273	 & 	1360.76	 & 	-3.6411	 \\
-7.84857	 & 	-3.6411	 & 	915.396
\end{tabular}

\vtab
 EingenVectors - Molecule A     \\
\begin{tabular}{|c c c|}
-0.99979	 & 	-0.00205437	 & 	-0.0203824	 \\
-0.0203985	 & 	0.00810116	 & 	0.999759	 \\
-0.00188875	 & 	0.999965	 & 	-0.00814137
\end{tabular}

\vtab
 EingenValues - Molecule A     \\
\begin{tabular}{|c c c|}
530.045	 & 	915.527	 & 	1360.79	 \\
\end{tabular}
\columnbreak

Molecule B \
3NFAACj

\includegraphics[width=6cm]{../Comparisons/ImagesFromVMD/3NFAACj.png}

Inertia Tensor - Molecule B \\
\begin{tabular}{|c c c|}
533.789	 & 	-4.75521	 & 	-1.91525	 \\
-4.75521	 & 	920.091	 & 	2.28449	 \\
-1.91525	 & 	2.28449	 & 	1348.28
\end{tabular}

\vtab
 EingenVectors - Molecule B     \\
\begin{tabular}{|c c c|}
-0.999922	 & 	-0.0122929	 & 	-0.00231663	 \\
0.0122803	 & 	-0.99991	 & 	0.00539026	 \\
-0.00238268	 & 	0.00536139	 & 	0.999983
\end{tabular}

\vtab
 EingenValues - Molecule B     \\
\begin{tabular}{|c c c|}
533.726	 & 	920.137	 & 	1348.3	 \\
\end{tabular}

\end{center}
\end{multicols}

\vtab[-5mm]
\begin{tabular}{*{2}{m{0.38\textwidth}}}
\begin{center}
\textcolor{NavyBlue}{\Large Different}
\end{center}
&
\begin{center}
\includegraphics[height=6.5cm]{../Comparisons/Vectors/inertia_tensor_of_3NFAACf_and_3NFAACj.png}
\end{center}
\end{tabular}

 \newpage

\vtab[-3cm]
\begin{center}
{\large FireTest \tab Número 311}
\end{center}
\begin{multicols}{2}
\begin{center}

Molecule A \
3NFAACf

\includegraphics[width=6cm]{../Comparisons/ImagesFromVMD/3NFAACf.png}

Inertia Tensor - Molecule A \\
\begin{tabular}{|c c c|}
530.208	 & 	-1.63273	 & 	-7.84857	 \\
-1.63273	 & 	1360.76	 & 	-3.6411	 \\
-7.84857	 & 	-3.6411	 & 	915.396
\end{tabular}

\vtab
 EingenVectors - Molecule A     \\
\begin{tabular}{|c c c|}
-0.99979	 & 	-0.00205437	 & 	-0.0203824	 \\
-0.0203985	 & 	0.00810116	 & 	0.999759	 \\
-0.00188875	 & 	0.999965	 & 	-0.00814137
\end{tabular}

\vtab
 EingenValues - Molecule A     \\
\begin{tabular}{|c c c|}
530.045	 & 	915.527	 & 	1360.79	 \\
\end{tabular}
\columnbreak

Molecule B \
3NFAACk

\includegraphics[width=6cm]{../Comparisons/ImagesFromVMD/3NFAACk.png}

Inertia Tensor - Molecule B \\
\begin{tabular}{|c c c|}
534.899	 & 	-5.81418	 & 	-0.270063	 \\
-5.81418	 & 	913.263	 & 	2.4519	 \\
-0.270063	 & 	2.4519	 & 	1353.77
\end{tabular}

\vtab
 EingenVectors - Molecule B     \\
\begin{tabular}{|c c c|}
-0.999882	 & 	-0.0153593	 & 	-0.00028374	 \\
0.0153575	 & 	-0.999867	 & 	0.00557573	 \\
-0.000369342	 & 	0.00557072	 & 	0.999984
\end{tabular}

\vtab
 EingenValues - Molecule B     \\
\begin{tabular}{|c c c|}
534.81	 & 	913.339	 & 	1353.78	 \\
\end{tabular}

\end{center}
\end{multicols}

\vtab[-5mm]
\begin{tabular}{*{2}{m{0.38\textwidth}}}
\begin{center}
\textcolor{NavyBlue}{\Large Different}
\end{center}
&
\begin{center}
\includegraphics[height=6.5cm]{../Comparisons/Vectors/inertia_tensor_of_3NFAACf_and_3NFAACk.png}
\end{center}
\end{tabular}

 \newpage

\vtab[-3cm]
\begin{center}
{\large FireTest \tab Número 312}
\end{center}
\begin{multicols}{2}
\begin{center}

Molecule A \
3NFAACf

\includegraphics[width=6cm]{../Comparisons/ImagesFromVMD/3NFAACf.png}

Inertia Tensor - Molecule A \\
\begin{tabular}{|c c c|}
530.208	 & 	-1.63273	 & 	-7.84857	 \\
-1.63273	 & 	1360.76	 & 	-3.6411	 \\
-7.84857	 & 	-3.6411	 & 	915.396
\end{tabular}

\vtab
 EingenVectors - Molecule A     \\
\begin{tabular}{|c c c|}
-0.99979	 & 	-0.00205437	 & 	-0.0203824	 \\
-0.0203985	 & 	0.00810116	 & 	0.999759	 \\
-0.00188875	 & 	0.999965	 & 	-0.00814137
\end{tabular}

\vtab
 EingenValues - Molecule A     \\
\begin{tabular}{|c c c|}
530.045	 & 	915.527	 & 	1360.79	 \\
\end{tabular}
\columnbreak

Molecule B \
3NFAACl

\includegraphics[width=6cm]{../Comparisons/ImagesFromVMD/3NFAACl.png}

Inertia Tensor - Molecule B \\
\begin{tabular}{|c c c|}
531.723	 & 	3.03424	 & 	2.73426	 \\
3.03424	 & 	929.418	 & 	-1.84284	 \\
2.73426	 & 	-1.84284	 & 	1355.39
\end{tabular}

\vtab
 EingenVectors - Molecule B     \\
\begin{tabular}{|c c c|}
0.999965	 & 	-0.00764413	 & 	-0.00333648	 \\
-0.00765838	 & 	-0.999962	 & 	-0.00427708	 \\
0.00330366	 & 	-0.00430248	 & 	0.999985
\end{tabular}

\vtab
 EingenValues - Molecule B     \\
\begin{tabular}{|c c c|}
531.691	 & 	929.434	 & 	1355.4	 \\
\end{tabular}

\end{center}
\end{multicols}

\vtab[-5mm]
\begin{tabular}{*{2}{m{0.38\textwidth}}}
\begin{center}
\textcolor{NavyBlue}{\Large Different}
\end{center}
&
\begin{center}
\includegraphics[height=6.5cm]{../Comparisons/Vectors/inertia_tensor_of_3NFAACf_and_3NFAACl.png}
\end{center}
\end{tabular}

 \newpage

\vtab[-3cm]
\begin{center}
{\large FireTest \tab Número 313}
\end{center}
\begin{multicols}{2}
\begin{center}

Molecule A \
3NFAACf

\includegraphics[width=6cm]{../Comparisons/ImagesFromVMD/3NFAACf.png}

Inertia Tensor - Molecule A \\
\begin{tabular}{|c c c|}
530.208	 & 	-1.63273	 & 	-7.84857	 \\
-1.63273	 & 	1360.76	 & 	-3.6411	 \\
-7.84857	 & 	-3.6411	 & 	915.396
\end{tabular}

\vtab
 EingenVectors - Molecule A     \\
\begin{tabular}{|c c c|}
-0.99979	 & 	-0.00205437	 & 	-0.0203824	 \\
-0.0203985	 & 	0.00810116	 & 	0.999759	 \\
-0.00188875	 & 	0.999965	 & 	-0.00814137
\end{tabular}

\vtab
 EingenValues - Molecule A     \\
\begin{tabular}{|c c c|}
530.045	 & 	915.527	 & 	1360.79	 \\
\end{tabular}
\columnbreak

Molecule B \
3NFAACm

\includegraphics[width=6cm]{../Comparisons/ImagesFromVMD/3NFAACm.png}

Inertia Tensor - Molecule B \\
\begin{tabular}{|c c c|}
532.546	 & 	13.7854	 & 	-15.4626	 \\
13.7854	 & 	1354.87	 & 	11.5786	 \\
-15.4626	 & 	11.5786	 & 	929.101
\end{tabular}

\vtab
 EingenVectors - Molecule B     \\
\begin{tabular}{|c c c|}
-0.999075	 & 	0.0172851	 & 	-0.0393769	 \\
0.0398168	 & 	0.0258942	 & 	-0.998871	 \\
-0.016246	 & 	-0.999515	 & 	-0.0265584
\end{tabular}

\vtab
 EingenValues - Molecule B     \\
\begin{tabular}{|c c c|}
531.698	 & 	929.417	 & 	1355.4	 \\
\end{tabular}

\end{center}
\end{multicols}

\vtab[-5mm]
\begin{tabular}{*{2}{m{0.38\textwidth}}}
\begin{center}
\textcolor{NavyBlue}{\Large Different}
\end{center}
&
\begin{center}
\includegraphics[height=6.5cm]{../Comparisons/Vectors/inertia_tensor_of_3NFAACf_and_3NFAACm.png}
\end{center}
\end{tabular}

 \newpage

\vtab[-3cm]
\begin{center}
{\large FireTest \tab Número 314}
\end{center}
\begin{multicols}{2}
\begin{center}

Molecule A \
3NFAACf

\includegraphics[width=6cm]{../Comparisons/ImagesFromVMD/3NFAACf.png}

Inertia Tensor - Molecule A \\
\begin{tabular}{|c c c|}
530.208	 & 	-1.63273	 & 	-7.84857	 \\
-1.63273	 & 	1360.76	 & 	-3.6411	 \\
-7.84857	 & 	-3.6411	 & 	915.396
\end{tabular}

\vtab
 EingenVectors - Molecule A     \\
\begin{tabular}{|c c c|}
-0.99979	 & 	-0.00205437	 & 	-0.0203824	 \\
-0.0203985	 & 	0.00810116	 & 	0.999759	 \\
-0.00188875	 & 	0.999965	 & 	-0.00814137
\end{tabular}

\vtab
 EingenValues - Molecule A     \\
\begin{tabular}{|c c c|}
530.045	 & 	915.527	 & 	1360.79	 \\
\end{tabular}
\columnbreak

Molecule B \
3NFAACn

\includegraphics[width=6cm]{../Comparisons/ImagesFromVMD/3NFAACn.png}

Inertia Tensor - Molecule B \\
\begin{tabular}{|c c c|}
531.896	 & 	3.78027	 & 	-13.1151	 \\
3.78027	 & 	1353.2	 & 	-7.47403	 \\
-13.1151	 & 	-7.47403	 & 	912.989
\end{tabular}

\vtab
 EingenVectors - Molecule B     \\
\begin{tabular}{|c c c|}
0.999403	 & 	-0.00428573	 & 	0.0342679	 \\
0.0341891	 & 	-0.0172718	 & 	-0.999266	 \\
0.00487445	 & 	0.999842	 & 	-0.017115
\end{tabular}

\vtab
 EingenValues - Molecule B     \\
\begin{tabular}{|c c c|}
531.43	 & 	913.309	 & 	1353.35	 \\
\end{tabular}

\end{center}
\end{multicols}

\vtab[-5mm]
\begin{tabular}{*{2}{m{0.38\textwidth}}}
\begin{center}
\textcolor{NavyBlue}{\Large Different}
\end{center}
&
\begin{center}
\includegraphics[height=6.5cm]{../Comparisons/Vectors/inertia_tensor_of_3NFAACf_and_3NFAACn.png}
\end{center}
\end{tabular}

 \newpage

\vtab[-3cm]
\begin{center}
{\large FireTest \tab Número 315}
\end{center}
\begin{multicols}{2}
\begin{center}

Molecule A \
3NFAACf

\includegraphics[width=6cm]{../Comparisons/ImagesFromVMD/3NFAACf.png}

Inertia Tensor - Molecule A \\
\begin{tabular}{|c c c|}
530.208	 & 	-1.63273	 & 	-7.84857	 \\
-1.63273	 & 	1360.76	 & 	-3.6411	 \\
-7.84857	 & 	-3.6411	 & 	915.396
\end{tabular}

\vtab
 EingenVectors - Molecule A     \\
\begin{tabular}{|c c c|}
-0.99979	 & 	-0.00205437	 & 	-0.0203824	 \\
-0.0203985	 & 	0.00810116	 & 	0.999759	 \\
-0.00188875	 & 	0.999965	 & 	-0.00814137
\end{tabular}

\vtab
 EingenValues - Molecule A     \\
\begin{tabular}{|c c c|}
530.045	 & 	915.527	 & 	1360.79	 \\
\end{tabular}
\columnbreak

Molecule B \
4NFAACa

\includegraphics[width=6cm]{../Comparisons/ImagesFromVMD/4NFAACa.png}

Inertia Tensor - Molecule B \\
\begin{tabular}{|c c c|}
479.392	 & 	3.27131	 & 	4.22557	 \\
3.27131	 & 	1242.39	 & 	-0.852684	 \\
4.22557	 & 	-0.852684	 & 	1647.37
\end{tabular}

\vtab
 EingenVectors - Molecule B     \\
\begin{tabular}{|c c c|}
0.999984	 & 	-0.00429123	 & 	-0.00362083	 \\
-0.00429871	 & 	-0.999989	 & 	-0.0020607	 \\
0.00361195	 & 	-0.00207623	 & 	0.999991
\end{tabular}

\vtab
 EingenValues - Molecule B     \\
\begin{tabular}{|c c c|}
479.363	 & 	1242.41	 & 	1647.39	 \\
\end{tabular}

\end{center}
\end{multicols}

\vtab[-5mm]
\begin{tabular}{*{2}{m{0.38\textwidth}}}
\begin{center}
\textcolor{NavyBlue}{\Large Different}
\end{center}
&
\begin{center}
\includegraphics[height=6.5cm]{../Comparisons/Vectors/inertia_tensor_of_3NFAACf_and_4NFAACa.png}
\end{center}
\end{tabular}

 \newpage

\vtab[-3cm]
\begin{center}
{\large FireTest \tab Número 316}
\end{center}
\begin{multicols}{2}
\begin{center}

Molecule A \
3NFAACf

\includegraphics[width=6cm]{../Comparisons/ImagesFromVMD/3NFAACf.png}

Inertia Tensor - Molecule A \\
\begin{tabular}{|c c c|}
530.208	 & 	-1.63273	 & 	-7.84857	 \\
-1.63273	 & 	1360.76	 & 	-3.6411	 \\
-7.84857	 & 	-3.6411	 & 	915.396
\end{tabular}

\vtab
 EingenVectors - Molecule A     \\
\begin{tabular}{|c c c|}
-0.99979	 & 	-0.00205437	 & 	-0.0203824	 \\
-0.0203985	 & 	0.00810116	 & 	0.999759	 \\
-0.00188875	 & 	0.999965	 & 	-0.00814137
\end{tabular}

\vtab
 EingenValues - Molecule A     \\
\begin{tabular}{|c c c|}
530.045	 & 	915.527	 & 	1360.79	 \\
\end{tabular}
\columnbreak

Molecule B \
4NFAACb

\includegraphics[width=6cm]{../Comparisons/ImagesFromVMD/4NFAACb.png}

Inertia Tensor - Molecule B \\
\begin{tabular}{|c c c|}
479.338	 & 	3.27331	 & 	-4.22553	 \\
3.27331	 & 	1242.4	 & 	0.852083	 \\
-4.22553	 & 	0.852083	 & 	1647.3
\end{tabular}

\vtab
 EingenVectors - Molecule B     \\
\begin{tabular}{|c c c|}
0.999984	 & 	-0.00429353	 & 	0.00362086	 \\
-0.004301	 & 	-0.999989	 & 	0.00205959	 \\
-0.00361198	 & 	0.00207513	 & 	0.999991
\end{tabular}

\vtab
 EingenValues - Molecule B     \\
\begin{tabular}{|c c c|}
479.308	 & 	1242.41	 & 	1647.31	 \\
\end{tabular}

\end{center}
\end{multicols}

\vtab[-5mm]
\begin{tabular}{*{2}{m{0.38\textwidth}}}
\begin{center}
\textcolor{NavyBlue}{\Large Different}
\end{center}
&
\begin{center}
\includegraphics[height=6.5cm]{../Comparisons/Vectors/inertia_tensor_of_3NFAACf_and_4NFAACb.png}
\end{center}
\end{tabular}

 \newpage

\vtab[-3cm]
\begin{center}
{\large FireTest \tab Número 317}
\end{center}
\begin{multicols}{2}
\begin{center}

Molecule A \
3NFAACf

\includegraphics[width=6cm]{../Comparisons/ImagesFromVMD/3NFAACf.png}

Inertia Tensor - Molecule A \\
\begin{tabular}{|c c c|}
530.208	 & 	-1.63273	 & 	-7.84857	 \\
-1.63273	 & 	1360.76	 & 	-3.6411	 \\
-7.84857	 & 	-3.6411	 & 	915.396
\end{tabular}

\vtab
 EingenVectors - Molecule A     \\
\begin{tabular}{|c c c|}
-0.99979	 & 	-0.00205437	 & 	-0.0203824	 \\
-0.0203985	 & 	0.00810116	 & 	0.999759	 \\
-0.00188875	 & 	0.999965	 & 	-0.00814137
\end{tabular}

\vtab
 EingenValues - Molecule A     \\
\begin{tabular}{|c c c|}
530.045	 & 	915.527	 & 	1360.79	 \\
\end{tabular}
\columnbreak

Molecule B \
4NFAACc

\includegraphics[width=6cm]{../Comparisons/ImagesFromVMD/4NFAACc.png}

Inertia Tensor - Molecule B \\
\begin{tabular}{|c c c|}
482.067	 & 	-5.39474	 & 	-1.35857	 \\
-5.39474	 & 	1240.3	 & 	-2.54035	 \\
-1.35857	 & 	-2.54035	 & 	1647.06
\end{tabular}

\vtab
 EingenVectors - Molecule B     \\
\begin{tabular}{|c c c|}
-0.999974	 & 	-0.00711826	 & 	-0.0011816	 \\
0.00712547	 & 	-0.999955	 & 	-0.00622156	 \\
-0.00113726	 & 	-0.00622982	 & 	0.99998
\end{tabular}

\vtab
 EingenValues - Molecule B     \\
\begin{tabular}{|c c c|}
482.027	 & 	1240.32	 & 	1647.08	 \\
\end{tabular}

\end{center}
\end{multicols}

\vtab[-5mm]
\begin{tabular}{*{2}{m{0.38\textwidth}}}
\begin{center}
\textcolor{NavyBlue}{\Large Different}
\end{center}
&
\begin{center}
\includegraphics[height=6.5cm]{../Comparisons/Vectors/inertia_tensor_of_3NFAACf_and_4NFAACc.png}
\end{center}
\end{tabular}

 \newpage

\vtab[-3cm]
\begin{center}
{\large FireTest \tab Número 318}
\end{center}
\begin{multicols}{2}
\begin{center}

Molecule A \
3NFAACf

\includegraphics[width=6cm]{../Comparisons/ImagesFromVMD/3NFAACf.png}

Inertia Tensor - Molecule A \\
\begin{tabular}{|c c c|}
530.208	 & 	-1.63273	 & 	-7.84857	 \\
-1.63273	 & 	1360.76	 & 	-3.6411	 \\
-7.84857	 & 	-3.6411	 & 	915.396
\end{tabular}

\vtab
 EingenVectors - Molecule A     \\
\begin{tabular}{|c c c|}
-0.99979	 & 	-0.00205437	 & 	-0.0203824	 \\
-0.0203985	 & 	0.00810116	 & 	0.999759	 \\
-0.00188875	 & 	0.999965	 & 	-0.00814137
\end{tabular}

\vtab
 EingenValues - Molecule A     \\
\begin{tabular}{|c c c|}
530.045	 & 	915.527	 & 	1360.79	 \\
\end{tabular}
\columnbreak

Molecule B \
4NFAACd

\includegraphics[width=6cm]{../Comparisons/ImagesFromVMD/4NFAACd.png}

Inertia Tensor - Molecule B \\
\begin{tabular}{|c c c|}
491.672	 & 	0.24486	 & 	-3.10016	 \\
0.24486	 & 	1231.15	 & 	2.19965	 \\
-3.10016	 & 	2.19965	 & 	1650.11
\end{tabular}

\vtab
 EingenVectors - Molecule B     \\
\begin{tabular}{|c c c|}
0.999996	 & 	-0.000339081	 & 	0.00267677	 \\
-0.000353124	 & 	-0.999986	 & 	0.00524747	 \\
-0.00267495	 & 	0.0052484	 & 	0.999983
\end{tabular}

\vtab
 EingenValues - Molecule B     \\
\begin{tabular}{|c c c|}
491.663	 & 	1231.14	 & 	1650.13	 \\
\end{tabular}

\end{center}
\end{multicols}

\vtab[-5mm]
\begin{tabular}{*{2}{m{0.38\textwidth}}}
\begin{center}
\textcolor{NavyBlue}{\Large Different}
\end{center}
&
\begin{center}
\includegraphics[height=6.5cm]{../Comparisons/Vectors/inertia_tensor_of_3NFAACf_and_4NFAACd.png}
\end{center}
\end{tabular}

 \newpage

\vtab[-3cm]
\begin{center}
{\large FireTest \tab Número 319}
\end{center}
\begin{multicols}{2}
\begin{center}

Molecule A \
3NFAACf

\includegraphics[width=6cm]{../Comparisons/ImagesFromVMD/3NFAACf.png}

Inertia Tensor - Molecule A \\
\begin{tabular}{|c c c|}
530.208	 & 	-1.63273	 & 	-7.84857	 \\
-1.63273	 & 	1360.76	 & 	-3.6411	 \\
-7.84857	 & 	-3.6411	 & 	915.396
\end{tabular}

\vtab
 EingenVectors - Molecule A     \\
\begin{tabular}{|c c c|}
-0.99979	 & 	-0.00205437	 & 	-0.0203824	 \\
-0.0203985	 & 	0.00810116	 & 	0.999759	 \\
-0.00188875	 & 	0.999965	 & 	-0.00814137
\end{tabular}

\vtab
 EingenValues - Molecule A     \\
\begin{tabular}{|c c c|}
530.045	 & 	915.527	 & 	1360.79	 \\
\end{tabular}
\columnbreak

Molecule B \
4NFAACe

\includegraphics[width=6cm]{../Comparisons/ImagesFromVMD/4NFAACe.png}

Inertia Tensor - Molecule B \\
\begin{tabular}{|c c c|}
489.025	 & 	-0.430035	 & 	3.98876	 \\
-0.430035	 & 	1233.71	 & 	-2.06505	 \\
3.98876	 & 	-2.06505	 & 	1641.79
\end{tabular}

\vtab
 EingenVectors - Molecule B     \\
\begin{tabular}{|c c c|}
0.999994	 & 	0.000567863	 & 	-0.00345908	 \\
0.000550336	 & 	-0.999987	 & 	-0.00506565	 \\
0.00346192	 & 	-0.00506372	 & 	0.999981
\end{tabular}

\vtab
 EingenValues - Molecule B     \\
\begin{tabular}{|c c c|}
489.011	 & 	1233.7	 & 	1641.81	 \\
\end{tabular}

\end{center}
\end{multicols}

\vtab[-5mm]
\begin{tabular}{*{2}{m{0.38\textwidth}}}
\begin{center}
\textcolor{NavyBlue}{\Large Different}
\end{center}
&
\begin{center}
\includegraphics[height=6.5cm]{../Comparisons/Vectors/inertia_tensor_of_3NFAACf_and_4NFAACe.png}
\end{center}
\end{tabular}

 \newpage

\vtab[-3cm]
\begin{center}
{\large FireTest \tab Número 320}
\end{center}
\begin{multicols}{2}
\begin{center}

Molecule A \
3NFAACf

\includegraphics[width=6cm]{../Comparisons/ImagesFromVMD/3NFAACf.png}

Inertia Tensor - Molecule A \\
\begin{tabular}{|c c c|}
530.208	 & 	-1.63273	 & 	-7.84857	 \\
-1.63273	 & 	1360.76	 & 	-3.6411	 \\
-7.84857	 & 	-3.6411	 & 	915.396
\end{tabular}

\vtab
 EingenVectors - Molecule A     \\
\begin{tabular}{|c c c|}
-0.99979	 & 	-0.00205437	 & 	-0.0203824	 \\
-0.0203985	 & 	0.00810116	 & 	0.999759	 \\
-0.00188875	 & 	0.999965	 & 	-0.00814137
\end{tabular}

\vtab
 EingenValues - Molecule A     \\
\begin{tabular}{|c c c|}
530.045	 & 	915.527	 & 	1360.79	 \\
\end{tabular}
\columnbreak

Molecule B \
4NFAACf

\includegraphics[width=6cm]{../Comparisons/ImagesFromVMD/4NFAACf.png}

Inertia Tensor - Molecule B \\
\begin{tabular}{|c c c|}
509.683	 & 	2.80651	 & 	-1.91422	 \\
2.80651	 & 	1219.11	 & 	2.66132	 \\
-1.91422	 & 	2.66132	 & 	1681.17
\end{tabular}

\vtab
 EingenVectors - Molecule B     \\
\begin{tabular}{|c c c|}
-0.999991	 & 	0.00396206	 & 	-0.00164298	 \\
-0.00397143	 & 	-0.999976	 & 	0.0057431	 \\
-0.00162019	 & 	0.00574957	 & 	0.999982
\end{tabular}

\vtab
 EingenValues - Molecule B     \\
\begin{tabular}{|c c c|}
509.668	 & 	1219.11	 & 	1681.18	 \\
\end{tabular}

\end{center}
\end{multicols}

\vtab[-5mm]
\begin{tabular}{*{2}{m{0.38\textwidth}}}
\begin{center}
\textcolor{NavyBlue}{\Large Different}
\end{center}
&
\begin{center}
\includegraphics[height=6.5cm]{../Comparisons/Vectors/inertia_tensor_of_3NFAACf_and_4NFAACf.png}
\end{center}
\end{tabular}

 \newpage

\vtab[-3cm]
\begin{center}
{\large FireTest \tab Número 321}
\end{center}
\begin{multicols}{2}
\begin{center}

Molecule A \
3NFAACf

\includegraphics[width=6cm]{../Comparisons/ImagesFromVMD/3NFAACf.png}

Inertia Tensor - Molecule A \\
\begin{tabular}{|c c c|}
530.208	 & 	-1.63273	 & 	-7.84857	 \\
-1.63273	 & 	1360.76	 & 	-3.6411	 \\
-7.84857	 & 	-3.6411	 & 	915.396
\end{tabular}

\vtab
 EingenVectors - Molecule A     \\
\begin{tabular}{|c c c|}
-0.99979	 & 	-0.00205437	 & 	-0.0203824	 \\
-0.0203985	 & 	0.00810116	 & 	0.999759	 \\
-0.00188875	 & 	0.999965	 & 	-0.00814137
\end{tabular}

\vtab
 EingenValues - Molecule A     \\
\begin{tabular}{|c c c|}
530.045	 & 	915.527	 & 	1360.79	 \\
\end{tabular}
\columnbreak

Molecule B \
4NFAACg

\includegraphics[width=6cm]{../Comparisons/ImagesFromVMD/4NFAACg.png}

Inertia Tensor - Molecule B \\
\begin{tabular}{|c c c|}
513.78	 & 	4.51917	 & 	0.266555	 \\
4.51917	 & 	1208.04	 & 	-1.18628	 \\
0.266555	 & 	-1.18628	 & 	1700.9
\end{tabular}

\vtab
 EingenVectors - Molecule B     \\
\begin{tabular}{|c c c|}
-0.999979	 & 	0.00650929	 & 	0.000231034	 \\
-0.00650983	 & 	-0.999976	 & 	-0.00240351	 \\
0.000215383	 & 	-0.00240496	 & 	0.999997
\end{tabular}

\vtab
 EingenValues - Molecule B     \\
\begin{tabular}{|c c c|}
513.751	 & 	1208.07	 & 	1700.9	 \\
\end{tabular}

\end{center}
\end{multicols}

\vtab[-5mm]
\begin{tabular}{*{2}{m{0.38\textwidth}}}
\begin{center}
\textcolor{NavyBlue}{\Large Different}
\end{center}
&
\begin{center}
\includegraphics[height=6.5cm]{../Comparisons/Vectors/inertia_tensor_of_3NFAACf_and_4NFAACg.png}
\end{center}
\end{tabular}

 \newpage

\vtab[-3cm]
\begin{center}
{\large FireTest \tab Número 322}
\end{center}
\begin{multicols}{2}
\begin{center}

Molecule A \
3NFAACf

\includegraphics[width=6cm]{../Comparisons/ImagesFromVMD/3NFAACf.png}

Inertia Tensor - Molecule A \\
\begin{tabular}{|c c c|}
530.208	 & 	-1.63273	 & 	-7.84857	 \\
-1.63273	 & 	1360.76	 & 	-3.6411	 \\
-7.84857	 & 	-3.6411	 & 	915.396
\end{tabular}

\vtab
 EingenVectors - Molecule A     \\
\begin{tabular}{|c c c|}
-0.99979	 & 	-0.00205437	 & 	-0.0203824	 \\
-0.0203985	 & 	0.00810116	 & 	0.999759	 \\
-0.00188875	 & 	0.999965	 & 	-0.00814137
\end{tabular}

\vtab
 EingenValues - Molecule A     \\
\begin{tabular}{|c c c|}
530.045	 & 	915.527	 & 	1360.79	 \\
\end{tabular}
\columnbreak

Molecule B \
4NFAACi

\includegraphics[width=6cm]{../Comparisons/ImagesFromVMD/4NFAACi.png}

Inertia Tensor - Molecule B \\
\begin{tabular}{|c c c|}
502.43	 & 	-0.602691	 & 	-4.86988	 \\
-0.602691	 & 	1232.26	 & 	0.407295	 \\
-4.86988	 & 	0.407295	 & 	1676
\end{tabular}

\vtab
 EingenVectors - Molecule B     \\
\begin{tabular}{|c c c|}
0.999991	 & 	0.000823447	 & 	0.00414923	 \\
0.000819608	 & 	-0.999999	 & 	0.00092687	 \\
-0.00414999	 & 	0.000923461	 & 	0.999991
\end{tabular}

\vtab
 EingenValues - Molecule B     \\
\begin{tabular}{|c c c|}
502.409	 & 	1232.26	 & 	1676.02	 \\
\end{tabular}

\end{center}
\end{multicols}

\vtab[-5mm]
\begin{tabular}{*{2}{m{0.38\textwidth}}}
\begin{center}
\textcolor{NavyBlue}{\Large Different}
\end{center}
&
\begin{center}
\includegraphics[height=6.5cm]{../Comparisons/Vectors/inertia_tensor_of_3NFAACf_and_4NFAACi.png}
\end{center}
\end{tabular}

 \newpage

\vtab[-3cm]
\begin{center}
{\large FireTest \tab Número 323}
\end{center}
\begin{multicols}{2}
\begin{center}

Molecule A \
3NFAACf

\includegraphics[width=6cm]{../Comparisons/ImagesFromVMD/3NFAACf.png}

Inertia Tensor - Molecule A \\
\begin{tabular}{|c c c|}
530.208	 & 	-1.63273	 & 	-7.84857	 \\
-1.63273	 & 	1360.76	 & 	-3.6411	 \\
-7.84857	 & 	-3.6411	 & 	915.396
\end{tabular}

\vtab
 EingenVectors - Molecule A     \\
\begin{tabular}{|c c c|}
-0.99979	 & 	-0.00205437	 & 	-0.0203824	 \\
-0.0203985	 & 	0.00810116	 & 	0.999759	 \\
-0.00188875	 & 	0.999965	 & 	-0.00814137
\end{tabular}

\vtab
 EingenValues - Molecule A     \\
\begin{tabular}{|c c c|}
530.045	 & 	915.527	 & 	1360.79	 \\
\end{tabular}
\columnbreak

Molecule B \
4NFAACj

\includegraphics[width=6cm]{../Comparisons/ImagesFromVMD/4NFAACj.png}

Inertia Tensor - Molecule B \\
\begin{tabular}{|c c c|}
510.047	 & 	9.97005	 & 	-3.6306	 \\
9.97005	 & 	1225.52	 & 	-0.981092	 \\
-3.6306	 & 	-0.981092	 & 	1680.82
\end{tabular}

\vtab
 EingenVectors - Molecule B     \\
\begin{tabular}{|c c c|}
-0.999898	 & 	0.0139264	 & 	-0.00308865	 \\
0.0139195	 & 	0.999901	 & 	0.00226627	 \\
-0.00311991	 & 	-0.00222305	 & 	0.999993
\end{tabular}

\vtab
 EingenValues - Molecule B     \\
\begin{tabular}{|c c c|}
509.897	 & 	1225.65	 & 	1680.83	 \\
\end{tabular}

\end{center}
\end{multicols}

\vtab[-5mm]
\begin{tabular}{*{2}{m{0.38\textwidth}}}
\begin{center}
\textcolor{NavyBlue}{\Large Different}
\end{center}
&
\begin{center}
\includegraphics[height=6.5cm]{../Comparisons/Vectors/inertia_tensor_of_3NFAACf_and_4NFAACj.png}
\end{center}
\end{tabular}

 \newpage

\vtab[-3cm]
\begin{center}
{\large FireTest \tab Número 324}
\end{center}
\begin{multicols}{2}
\begin{center}

Molecule A \
3NFAACf

\includegraphics[width=6cm]{../Comparisons/ImagesFromVMD/3NFAACf.png}

Inertia Tensor - Molecule A \\
\begin{tabular}{|c c c|}
530.208	 & 	-1.63273	 & 	-7.84857	 \\
-1.63273	 & 	1360.76	 & 	-3.6411	 \\
-7.84857	 & 	-3.6411	 & 	915.396
\end{tabular}

\vtab
 EingenVectors - Molecule A     \\
\begin{tabular}{|c c c|}
-0.99979	 & 	-0.00205437	 & 	-0.0203824	 \\
-0.0203985	 & 	0.00810116	 & 	0.999759	 \\
-0.00188875	 & 	0.999965	 & 	-0.00814137
\end{tabular}

\vtab
 EingenValues - Molecule A     \\
\begin{tabular}{|c c c|}
530.045	 & 	915.527	 & 	1360.79	 \\
\end{tabular}
\columnbreak

Molecule B \
4NFAACl-3

\includegraphics[width=6cm]{../Comparisons/ImagesFromVMD/4NFAACl-3.png}

Inertia Tensor - Molecule B \\
\begin{tabular}{|c c c|}
506.608	 & 	0.709539	 & 	-0.555426	 \\
0.709539	 & 	1222.37	 & 	-2.84005	 \\
-0.555426	 & 	-2.84005	 & 	1678.41
\end{tabular}

\vtab
 EingenVectors - Molecule B     \\
\begin{tabular}{|c c c|}
-0.999999	 & 	0.000989428	 & 	-0.000471595	 \\
-0.000986471	 & 	-0.99998	 & 	-0.00622856	 \\
-0.000477748	 & 	-0.00622809	 & 	0.99998
\end{tabular}

\vtab
 EingenValues - Molecule B     \\
\begin{tabular}{|c c c|}
506.607	 & 	1222.36	 & 	1678.43	 \\
\end{tabular}

\end{center}
\end{multicols}

\vtab[-5mm]
\begin{tabular}{*{2}{m{0.38\textwidth}}}
\begin{center}
\textcolor{NavyBlue}{\Large Different}
\end{center}
&
\begin{center}
\includegraphics[height=6.5cm]{../Comparisons/Vectors/inertia_tensor_of_3NFAACf_and_4NFAACl-3.png}
\end{center}
\end{tabular}

 \newpage

\vtab[-3cm]
\begin{center}
{\large FireTest \tab Número 325}
\end{center}
\begin{multicols}{2}
\begin{center}

Molecule A \
3NFAACg

\includegraphics[width=6cm]{../Comparisons/ImagesFromVMD/3NFAACg.png}

Inertia Tensor - Molecule A \\
\begin{tabular}{|c c c|}
532.891	 & 	-0.526938	 & 	-0.504713	 \\
-0.526938	 & 	918.454	 & 	2.35809	 \\
-0.504713	 & 	2.35809	 & 	1356.91
\end{tabular}

\vtab
 EingenVectors - Molecule A     \\
\begin{tabular}{|c c c|}
-0.999999	 & 	-0.00136295	 & 	-0.000608598	 \\
0.00135965	 & 	-0.999985	 & 	0.00537946	 \\
-0.000615921	 & 	0.00537863	 & 	0.999985
\end{tabular}

\vtab
 EingenValues - Molecule A     \\
\begin{tabular}{|c c c|}
532.89	 & 	918.442	 & 	1356.93	 \\
\end{tabular}
\columnbreak

Molecule B \
3NFAACh

\includegraphics[width=6cm]{../Comparisons/ImagesFromVMD/3NFAACh.png}

Inertia Tensor - Molecule B \\
\begin{tabular}{|c c c|}
528.718	 & 	-3.89066	 & 	-3.66882	 \\
-3.89066	 & 	924.911	 & 	2.93817	 \\
-3.66882	 & 	2.93817	 & 	1336.08
\end{tabular}

\vtab
 EingenVectors - Molecule B     \\
\begin{tabular}{|c c c|}
0.999942	 & 	0.00978475	 & 	0.00450803	 \\
0.00975199	 & 	-0.999926	 & 	0.00723269	 \\
-0.00457847	 & 	0.0071883	 & 	0.999964
\end{tabular}

\vtab
 EingenValues - Molecule B     \\
\begin{tabular}{|c c c|}
528.663	 & 	924.928	 & 	1336.12	 \\
\end{tabular}

\end{center}
\end{multicols}

\vtab[-5mm]
\begin{tabular}{*{2}{m{0.38\textwidth}}}
\begin{center}
\textcolor{NavyBlue}{\Large Different}
\end{center}
&
\begin{center}
\includegraphics[height=6.5cm]{../Comparisons/Vectors/inertia_tensor_of_3NFAACg_and_3NFAACh.png}
\end{center}
\end{tabular}

 \newpage

\vtab[-3cm]
\begin{center}
{\large FireTest \tab Número 326}
\end{center}
\begin{multicols}{2}
\begin{center}

Molecule A \
3NFAACg

\includegraphics[width=6cm]{../Comparisons/ImagesFromVMD/3NFAACg.png}

Inertia Tensor - Molecule A \\
\begin{tabular}{|c c c|}
532.891	 & 	-0.526938	 & 	-0.504713	 \\
-0.526938	 & 	918.454	 & 	2.35809	 \\
-0.504713	 & 	2.35809	 & 	1356.91
\end{tabular}

\vtab
 EingenVectors - Molecule A     \\
\begin{tabular}{|c c c|}
-0.999999	 & 	-0.00136295	 & 	-0.000608598	 \\
0.00135965	 & 	-0.999985	 & 	0.00537946	 \\
-0.000615921	 & 	0.00537863	 & 	0.999985
\end{tabular}

\vtab
 EingenValues - Molecule A     \\
\begin{tabular}{|c c c|}
532.89	 & 	918.442	 & 	1356.93	 \\
\end{tabular}
\columnbreak

Molecule B \
3NFAACi

\includegraphics[width=6cm]{../Comparisons/ImagesFromVMD/3NFAACi.png}

Inertia Tensor - Molecule B \\
\begin{tabular}{|c c c|}
521.487	 & 	-1.22943	 & 	-2.71042	 \\
-1.22943	 & 	949.663	 & 	1.67006	 \\
-2.71042	 & 	1.67006	 & 	1346.16
\end{tabular}

\vtab
 EingenVectors - Molecule B     \\
\begin{tabular}{|c c c|}
0.999991	 & 	0.00285841	 & 	0.00328077	 \\
0.00284452	 & 	-0.999987	 & 	0.00423133	 \\
-0.00329282	 & 	0.00422196	 & 	0.999986
\end{tabular}

\vtab
 EingenValues - Molecule B     \\
\begin{tabular}{|c c c|}
521.475	 & 	949.659	 & 	1346.18	 \\
\end{tabular}

\end{center}
\end{multicols}

\vtab[-5mm]
\begin{tabular}{*{2}{m{0.38\textwidth}}}
\begin{center}
\textcolor{NavyBlue}{\Large Different}
\end{center}
&
\begin{center}
\includegraphics[height=6.5cm]{../Comparisons/Vectors/inertia_tensor_of_3NFAACg_and_3NFAACi.png}
\end{center}
\end{tabular}

 \newpage

\vtab[-3cm]
\begin{center}
{\large FireTest \tab Número 327}
\end{center}
\begin{multicols}{2}
\begin{center}

Molecule A \
3NFAACg

\includegraphics[width=6cm]{../Comparisons/ImagesFromVMD/3NFAACg.png}

Inertia Tensor - Molecule A \\
\begin{tabular}{|c c c|}
532.891	 & 	-0.526938	 & 	-0.504713	 \\
-0.526938	 & 	918.454	 & 	2.35809	 \\
-0.504713	 & 	2.35809	 & 	1356.91
\end{tabular}

\vtab
 EingenVectors - Molecule A     \\
\begin{tabular}{|c c c|}
-0.999999	 & 	-0.00136295	 & 	-0.000608598	 \\
0.00135965	 & 	-0.999985	 & 	0.00537946	 \\
-0.000615921	 & 	0.00537863	 & 	0.999985
\end{tabular}

\vtab
 EingenValues - Molecule A     \\
\begin{tabular}{|c c c|}
532.89	 & 	918.442	 & 	1356.93	 \\
\end{tabular}
\columnbreak

Molecule B \
3NFAACj

\includegraphics[width=6cm]{../Comparisons/ImagesFromVMD/3NFAACj.png}

Inertia Tensor - Molecule B \\
\begin{tabular}{|c c c|}
533.789	 & 	-4.75521	 & 	-1.91525	 \\
-4.75521	 & 	920.091	 & 	2.28449	 \\
-1.91525	 & 	2.28449	 & 	1348.28
\end{tabular}

\vtab
 EingenVectors - Molecule B     \\
\begin{tabular}{|c c c|}
-0.999922	 & 	-0.0122929	 & 	-0.00231663	 \\
0.0122803	 & 	-0.99991	 & 	0.00539026	 \\
-0.00238268	 & 	0.00536139	 & 	0.999983
\end{tabular}

\vtab
 EingenValues - Molecule B     \\
\begin{tabular}{|c c c|}
533.726	 & 	920.137	 & 	1348.3	 \\
\end{tabular}

\end{center}
\end{multicols}

\vtab[-5mm]
\begin{tabular}{*{2}{m{0.38\textwidth}}}
\begin{center}
\textcolor{NavyBlue}{\Large Different}
\end{center}
&
\begin{center}
\includegraphics[height=6.5cm]{../Comparisons/Vectors/inertia_tensor_of_3NFAACg_and_3NFAACj.png}
\end{center}
\end{tabular}

 \newpage

\vtab[-3cm]
\begin{center}
{\large FireTest \tab Número 328}
\end{center}
\begin{multicols}{2}
\begin{center}

Molecule A \
3NFAACg

\includegraphics[width=6cm]{../Comparisons/ImagesFromVMD/3NFAACg.png}

Inertia Tensor - Molecule A \\
\begin{tabular}{|c c c|}
532.891	 & 	-0.526938	 & 	-0.504713	 \\
-0.526938	 & 	918.454	 & 	2.35809	 \\
-0.504713	 & 	2.35809	 & 	1356.91
\end{tabular}

\vtab
 EingenVectors - Molecule A     \\
\begin{tabular}{|c c c|}
-0.999999	 & 	-0.00136295	 & 	-0.000608598	 \\
0.00135965	 & 	-0.999985	 & 	0.00537946	 \\
-0.000615921	 & 	0.00537863	 & 	0.999985
\end{tabular}

\vtab
 EingenValues - Molecule A     \\
\begin{tabular}{|c c c|}
532.89	 & 	918.442	 & 	1356.93	 \\
\end{tabular}
\columnbreak

Molecule B \
3NFAACk

\includegraphics[width=6cm]{../Comparisons/ImagesFromVMD/3NFAACk.png}

Inertia Tensor - Molecule B \\
\begin{tabular}{|c c c|}
534.899	 & 	-5.81418	 & 	-0.270063	 \\
-5.81418	 & 	913.263	 & 	2.4519	 \\
-0.270063	 & 	2.4519	 & 	1353.77
\end{tabular}

\vtab
 EingenVectors - Molecule B     \\
\begin{tabular}{|c c c|}
-0.999882	 & 	-0.0153593	 & 	-0.00028374	 \\
0.0153575	 & 	-0.999867	 & 	0.00557573	 \\
-0.000369342	 & 	0.00557072	 & 	0.999984
\end{tabular}

\vtab
 EingenValues - Molecule B     \\
\begin{tabular}{|c c c|}
534.81	 & 	913.339	 & 	1353.78	 \\
\end{tabular}

\end{center}
\end{multicols}

\vtab[-5mm]
\begin{tabular}{*{2}{m{0.38\textwidth}}}
\begin{center}
\textcolor{NavyBlue}{\Large Different}
\end{center}
&
\begin{center}
\includegraphics[height=6.5cm]{../Comparisons/Vectors/inertia_tensor_of_3NFAACg_and_3NFAACk.png}
\end{center}
\end{tabular}

 \newpage

\vtab[-3cm]
\begin{center}
{\large FireTest \tab Número 329}
\end{center}
\begin{multicols}{2}
\begin{center}

Molecule A \
3NFAACg

\includegraphics[width=6cm]{../Comparisons/ImagesFromVMD/3NFAACg.png}

Inertia Tensor - Molecule A \\
\begin{tabular}{|c c c|}
532.891	 & 	-0.526938	 & 	-0.504713	 \\
-0.526938	 & 	918.454	 & 	2.35809	 \\
-0.504713	 & 	2.35809	 & 	1356.91
\end{tabular}

\vtab
 EingenVectors - Molecule A     \\
\begin{tabular}{|c c c|}
-0.999999	 & 	-0.00136295	 & 	-0.000608598	 \\
0.00135965	 & 	-0.999985	 & 	0.00537946	 \\
-0.000615921	 & 	0.00537863	 & 	0.999985
\end{tabular}

\vtab
 EingenValues - Molecule A     \\
\begin{tabular}{|c c c|}
532.89	 & 	918.442	 & 	1356.93	 \\
\end{tabular}
\columnbreak

Molecule B \
3NFAACl

\includegraphics[width=6cm]{../Comparisons/ImagesFromVMD/3NFAACl.png}

Inertia Tensor - Molecule B \\
\begin{tabular}{|c c c|}
531.723	 & 	3.03424	 & 	2.73426	 \\
3.03424	 & 	929.418	 & 	-1.84284	 \\
2.73426	 & 	-1.84284	 & 	1355.39
\end{tabular}

\vtab
 EingenVectors - Molecule B     \\
\begin{tabular}{|c c c|}
0.999965	 & 	-0.00764413	 & 	-0.00333648	 \\
-0.00765838	 & 	-0.999962	 & 	-0.00427708	 \\
0.00330366	 & 	-0.00430248	 & 	0.999985
\end{tabular}

\vtab
 EingenValues - Molecule B     \\
\begin{tabular}{|c c c|}
531.691	 & 	929.434	 & 	1355.4	 \\
\end{tabular}

\end{center}
\end{multicols}

\vtab[-5mm]
\begin{tabular}{*{2}{m{0.38\textwidth}}}
\begin{center}
\textcolor{NavyBlue}{\Large Different}
\end{center}
&
\begin{center}
\includegraphics[height=6.5cm]{../Comparisons/Vectors/inertia_tensor_of_3NFAACg_and_3NFAACl.png}
\end{center}
\end{tabular}

 \newpage

\vtab[-3cm]
\begin{center}
{\large FireTest \tab Número 330}
\end{center}
\begin{multicols}{2}
\begin{center}

Molecule A \
3NFAACg

\includegraphics[width=6cm]{../Comparisons/ImagesFromVMD/3NFAACg.png}

Inertia Tensor - Molecule A \\
\begin{tabular}{|c c c|}
532.891	 & 	-0.526938	 & 	-0.504713	 \\
-0.526938	 & 	918.454	 & 	2.35809	 \\
-0.504713	 & 	2.35809	 & 	1356.91
\end{tabular}

\vtab
 EingenVectors - Molecule A     \\
\begin{tabular}{|c c c|}
-0.999999	 & 	-0.00136295	 & 	-0.000608598	 \\
0.00135965	 & 	-0.999985	 & 	0.00537946	 \\
-0.000615921	 & 	0.00537863	 & 	0.999985
\end{tabular}

\vtab
 EingenValues - Molecule A     \\
\begin{tabular}{|c c c|}
532.89	 & 	918.442	 & 	1356.93	 \\
\end{tabular}
\columnbreak

Molecule B \
3NFAACm

\includegraphics[width=6cm]{../Comparisons/ImagesFromVMD/3NFAACm.png}

Inertia Tensor - Molecule B \\
\begin{tabular}{|c c c|}
532.546	 & 	13.7854	 & 	-15.4626	 \\
13.7854	 & 	1354.87	 & 	11.5786	 \\
-15.4626	 & 	11.5786	 & 	929.101
\end{tabular}

\vtab
 EingenVectors - Molecule B     \\
\begin{tabular}{|c c c|}
-0.999075	 & 	0.0172851	 & 	-0.0393769	 \\
0.0398168	 & 	0.0258942	 & 	-0.998871	 \\
-0.016246	 & 	-0.999515	 & 	-0.0265584
\end{tabular}

\vtab
 EingenValues - Molecule B     \\
\begin{tabular}{|c c c|}
531.698	 & 	929.417	 & 	1355.4	 \\
\end{tabular}

\end{center}
\end{multicols}

\vtab[-5mm]
\begin{tabular}{*{2}{m{0.38\textwidth}}}
\begin{center}
\textcolor{NavyBlue}{\Large Different}
\end{center}
&
\begin{center}
\includegraphics[height=6.5cm]{../Comparisons/Vectors/inertia_tensor_of_3NFAACg_and_3NFAACm.png}
\end{center}
\end{tabular}

 \newpage

\vtab[-3cm]
\begin{center}
{\large FireTest \tab Número 331}
\end{center}
\begin{multicols}{2}
\begin{center}

Molecule A \
3NFAACg

\includegraphics[width=6cm]{../Comparisons/ImagesFromVMD/3NFAACg.png}

Inertia Tensor - Molecule A \\
\begin{tabular}{|c c c|}
532.891	 & 	-0.526938	 & 	-0.504713	 \\
-0.526938	 & 	918.454	 & 	2.35809	 \\
-0.504713	 & 	2.35809	 & 	1356.91
\end{tabular}

\vtab
 EingenVectors - Molecule A     \\
\begin{tabular}{|c c c|}
-0.999999	 & 	-0.00136295	 & 	-0.000608598	 \\
0.00135965	 & 	-0.999985	 & 	0.00537946	 \\
-0.000615921	 & 	0.00537863	 & 	0.999985
\end{tabular}

\vtab
 EingenValues - Molecule A     \\
\begin{tabular}{|c c c|}
532.89	 & 	918.442	 & 	1356.93	 \\
\end{tabular}
\columnbreak

Molecule B \
3NFAACn

\includegraphics[width=6cm]{../Comparisons/ImagesFromVMD/3NFAACn.png}

Inertia Tensor - Molecule B \\
\begin{tabular}{|c c c|}
531.896	 & 	3.78027	 & 	-13.1151	 \\
3.78027	 & 	1353.2	 & 	-7.47403	 \\
-13.1151	 & 	-7.47403	 & 	912.989
\end{tabular}

\vtab
 EingenVectors - Molecule B     \\
\begin{tabular}{|c c c|}
0.999403	 & 	-0.00428573	 & 	0.0342679	 \\
0.0341891	 & 	-0.0172718	 & 	-0.999266	 \\
0.00487445	 & 	0.999842	 & 	-0.017115
\end{tabular}

\vtab
 EingenValues - Molecule B     \\
\begin{tabular}{|c c c|}
531.43	 & 	913.309	 & 	1353.35	 \\
\end{tabular}

\end{center}
\end{multicols}

\vtab[-5mm]
\begin{tabular}{*{2}{m{0.38\textwidth}}}
\begin{center}
\textcolor{NavyBlue}{\Large Different}
\end{center}
&
\begin{center}
\includegraphics[height=6.5cm]{../Comparisons/Vectors/inertia_tensor_of_3NFAACg_and_3NFAACn.png}
\end{center}
\end{tabular}

 \newpage

\vtab[-3cm]
\begin{center}
{\large FireTest \tab Número 332}
\end{center}
\begin{multicols}{2}
\begin{center}

Molecule A \
3NFAACg

\includegraphics[width=6cm]{../Comparisons/ImagesFromVMD/3NFAACg.png}

Inertia Tensor - Molecule A \\
\begin{tabular}{|c c c|}
532.891	 & 	-0.526938	 & 	-0.504713	 \\
-0.526938	 & 	918.454	 & 	2.35809	 \\
-0.504713	 & 	2.35809	 & 	1356.91
\end{tabular}

\vtab
 EingenVectors - Molecule A     \\
\begin{tabular}{|c c c|}
-0.999999	 & 	-0.00136295	 & 	-0.000608598	 \\
0.00135965	 & 	-0.999985	 & 	0.00537946	 \\
-0.000615921	 & 	0.00537863	 & 	0.999985
\end{tabular}

\vtab
 EingenValues - Molecule A     \\
\begin{tabular}{|c c c|}
532.89	 & 	918.442	 & 	1356.93	 \\
\end{tabular}
\columnbreak

Molecule B \
4NFAACa

\includegraphics[width=6cm]{../Comparisons/ImagesFromVMD/4NFAACa.png}

Inertia Tensor - Molecule B \\
\begin{tabular}{|c c c|}
479.392	 & 	3.27131	 & 	4.22557	 \\
3.27131	 & 	1242.39	 & 	-0.852684	 \\
4.22557	 & 	-0.852684	 & 	1647.37
\end{tabular}

\vtab
 EingenVectors - Molecule B     \\
\begin{tabular}{|c c c|}
0.999984	 & 	-0.00429123	 & 	-0.00362083	 \\
-0.00429871	 & 	-0.999989	 & 	-0.0020607	 \\
0.00361195	 & 	-0.00207623	 & 	0.999991
\end{tabular}

\vtab
 EingenValues - Molecule B     \\
\begin{tabular}{|c c c|}
479.363	 & 	1242.41	 & 	1647.39	 \\
\end{tabular}

\end{center}
\end{multicols}

\vtab[-5mm]
\begin{tabular}{*{2}{m{0.38\textwidth}}}
\begin{center}
\textcolor{NavyBlue}{\Large Different}
\end{center}
&
\begin{center}
\includegraphics[height=6.5cm]{../Comparisons/Vectors/inertia_tensor_of_3NFAACg_and_4NFAACa.png}
\end{center}
\end{tabular}

 \newpage

\vtab[-3cm]
\begin{center}
{\large FireTest \tab Número 333}
\end{center}
\begin{multicols}{2}
\begin{center}

Molecule A \
3NFAACg

\includegraphics[width=6cm]{../Comparisons/ImagesFromVMD/3NFAACg.png}

Inertia Tensor - Molecule A \\
\begin{tabular}{|c c c|}
532.891	 & 	-0.526938	 & 	-0.504713	 \\
-0.526938	 & 	918.454	 & 	2.35809	 \\
-0.504713	 & 	2.35809	 & 	1356.91
\end{tabular}

\vtab
 EingenVectors - Molecule A     \\
\begin{tabular}{|c c c|}
-0.999999	 & 	-0.00136295	 & 	-0.000608598	 \\
0.00135965	 & 	-0.999985	 & 	0.00537946	 \\
-0.000615921	 & 	0.00537863	 & 	0.999985
\end{tabular}

\vtab
 EingenValues - Molecule A     \\
\begin{tabular}{|c c c|}
532.89	 & 	918.442	 & 	1356.93	 \\
\end{tabular}
\columnbreak

Molecule B \
4NFAACb

\includegraphics[width=6cm]{../Comparisons/ImagesFromVMD/4NFAACb.png}

Inertia Tensor - Molecule B \\
\begin{tabular}{|c c c|}
479.338	 & 	3.27331	 & 	-4.22553	 \\
3.27331	 & 	1242.4	 & 	0.852083	 \\
-4.22553	 & 	0.852083	 & 	1647.3
\end{tabular}

\vtab
 EingenVectors - Molecule B     \\
\begin{tabular}{|c c c|}
0.999984	 & 	-0.00429353	 & 	0.00362086	 \\
-0.004301	 & 	-0.999989	 & 	0.00205959	 \\
-0.00361198	 & 	0.00207513	 & 	0.999991
\end{tabular}

\vtab
 EingenValues - Molecule B     \\
\begin{tabular}{|c c c|}
479.308	 & 	1242.41	 & 	1647.31	 \\
\end{tabular}

\end{center}
\end{multicols}

\vtab[-5mm]
\begin{tabular}{*{2}{m{0.38\textwidth}}}
\begin{center}
\textcolor{NavyBlue}{\Large Different}
\end{center}
&
\begin{center}
\includegraphics[height=6.5cm]{../Comparisons/Vectors/inertia_tensor_of_3NFAACg_and_4NFAACb.png}
\end{center}
\end{tabular}

 \newpage

\vtab[-3cm]
\begin{center}
{\large FireTest \tab Número 334}
\end{center}
\begin{multicols}{2}
\begin{center}

Molecule A \
3NFAACg

\includegraphics[width=6cm]{../Comparisons/ImagesFromVMD/3NFAACg.png}

Inertia Tensor - Molecule A \\
\begin{tabular}{|c c c|}
532.891	 & 	-0.526938	 & 	-0.504713	 \\
-0.526938	 & 	918.454	 & 	2.35809	 \\
-0.504713	 & 	2.35809	 & 	1356.91
\end{tabular}

\vtab
 EingenVectors - Molecule A     \\
\begin{tabular}{|c c c|}
-0.999999	 & 	-0.00136295	 & 	-0.000608598	 \\
0.00135965	 & 	-0.999985	 & 	0.00537946	 \\
-0.000615921	 & 	0.00537863	 & 	0.999985
\end{tabular}

\vtab
 EingenValues - Molecule A     \\
\begin{tabular}{|c c c|}
532.89	 & 	918.442	 & 	1356.93	 \\
\end{tabular}
\columnbreak

Molecule B \
4NFAACc

\includegraphics[width=6cm]{../Comparisons/ImagesFromVMD/4NFAACc.png}

Inertia Tensor - Molecule B \\
\begin{tabular}{|c c c|}
482.067	 & 	-5.39474	 & 	-1.35857	 \\
-5.39474	 & 	1240.3	 & 	-2.54035	 \\
-1.35857	 & 	-2.54035	 & 	1647.06
\end{tabular}

\vtab
 EingenVectors - Molecule B     \\
\begin{tabular}{|c c c|}
-0.999974	 & 	-0.00711826	 & 	-0.0011816	 \\
0.00712547	 & 	-0.999955	 & 	-0.00622156	 \\
-0.00113726	 & 	-0.00622982	 & 	0.99998
\end{tabular}

\vtab
 EingenValues - Molecule B     \\
\begin{tabular}{|c c c|}
482.027	 & 	1240.32	 & 	1647.08	 \\
\end{tabular}

\end{center}
\end{multicols}

\vtab[-5mm]
\begin{tabular}{*{2}{m{0.38\textwidth}}}
\begin{center}
\textcolor{NavyBlue}{\Large Different}
\end{center}
&
\begin{center}
\includegraphics[height=6.5cm]{../Comparisons/Vectors/inertia_tensor_of_3NFAACg_and_4NFAACc.png}
\end{center}
\end{tabular}

 \newpage

\vtab[-3cm]
\begin{center}
{\large FireTest \tab Número 335}
\end{center}
\begin{multicols}{2}
\begin{center}

Molecule A \
3NFAACg

\includegraphics[width=6cm]{../Comparisons/ImagesFromVMD/3NFAACg.png}

Inertia Tensor - Molecule A \\
\begin{tabular}{|c c c|}
532.891	 & 	-0.526938	 & 	-0.504713	 \\
-0.526938	 & 	918.454	 & 	2.35809	 \\
-0.504713	 & 	2.35809	 & 	1356.91
\end{tabular}

\vtab
 EingenVectors - Molecule A     \\
\begin{tabular}{|c c c|}
-0.999999	 & 	-0.00136295	 & 	-0.000608598	 \\
0.00135965	 & 	-0.999985	 & 	0.00537946	 \\
-0.000615921	 & 	0.00537863	 & 	0.999985
\end{tabular}

\vtab
 EingenValues - Molecule A     \\
\begin{tabular}{|c c c|}
532.89	 & 	918.442	 & 	1356.93	 \\
\end{tabular}
\columnbreak

Molecule B \
4NFAACd

\includegraphics[width=6cm]{../Comparisons/ImagesFromVMD/4NFAACd.png}

Inertia Tensor - Molecule B \\
\begin{tabular}{|c c c|}
491.672	 & 	0.24486	 & 	-3.10016	 \\
0.24486	 & 	1231.15	 & 	2.19965	 \\
-3.10016	 & 	2.19965	 & 	1650.11
\end{tabular}

\vtab
 EingenVectors - Molecule B     \\
\begin{tabular}{|c c c|}
0.999996	 & 	-0.000339081	 & 	0.00267677	 \\
-0.000353124	 & 	-0.999986	 & 	0.00524747	 \\
-0.00267495	 & 	0.0052484	 & 	0.999983
\end{tabular}

\vtab
 EingenValues - Molecule B     \\
\begin{tabular}{|c c c|}
491.663	 & 	1231.14	 & 	1650.13	 \\
\end{tabular}

\end{center}
\end{multicols}

\vtab[-5mm]
\begin{tabular}{*{2}{m{0.38\textwidth}}}
\begin{center}
\textcolor{NavyBlue}{\Large Different}
\end{center}
&
\begin{center}
\includegraphics[height=6.5cm]{../Comparisons/Vectors/inertia_tensor_of_3NFAACg_and_4NFAACd.png}
\end{center}
\end{tabular}

 \newpage

\vtab[-3cm]
\begin{center}
{\large FireTest \tab Número 336}
\end{center}
\begin{multicols}{2}
\begin{center}

Molecule A \
3NFAACg

\includegraphics[width=6cm]{../Comparisons/ImagesFromVMD/3NFAACg.png}

Inertia Tensor - Molecule A \\
\begin{tabular}{|c c c|}
532.891	 & 	-0.526938	 & 	-0.504713	 \\
-0.526938	 & 	918.454	 & 	2.35809	 \\
-0.504713	 & 	2.35809	 & 	1356.91
\end{tabular}

\vtab
 EingenVectors - Molecule A     \\
\begin{tabular}{|c c c|}
-0.999999	 & 	-0.00136295	 & 	-0.000608598	 \\
0.00135965	 & 	-0.999985	 & 	0.00537946	 \\
-0.000615921	 & 	0.00537863	 & 	0.999985
\end{tabular}

\vtab
 EingenValues - Molecule A     \\
\begin{tabular}{|c c c|}
532.89	 & 	918.442	 & 	1356.93	 \\
\end{tabular}
\columnbreak

Molecule B \
4NFAACe

\includegraphics[width=6cm]{../Comparisons/ImagesFromVMD/4NFAACe.png}

Inertia Tensor - Molecule B \\
\begin{tabular}{|c c c|}
489.025	 & 	-0.430035	 & 	3.98876	 \\
-0.430035	 & 	1233.71	 & 	-2.06505	 \\
3.98876	 & 	-2.06505	 & 	1641.79
\end{tabular}

\vtab
 EingenVectors - Molecule B     \\
\begin{tabular}{|c c c|}
0.999994	 & 	0.000567863	 & 	-0.00345908	 \\
0.000550336	 & 	-0.999987	 & 	-0.00506565	 \\
0.00346192	 & 	-0.00506372	 & 	0.999981
\end{tabular}

\vtab
 EingenValues - Molecule B     \\
\begin{tabular}{|c c c|}
489.011	 & 	1233.7	 & 	1641.81	 \\
\end{tabular}

\end{center}
\end{multicols}

\vtab[-5mm]
\begin{tabular}{*{2}{m{0.38\textwidth}}}
\begin{center}
\textcolor{NavyBlue}{\Large Different}
\end{center}
&
\begin{center}
\includegraphics[height=6.5cm]{../Comparisons/Vectors/inertia_tensor_of_3NFAACg_and_4NFAACe.png}
\end{center}
\end{tabular}

 \newpage

\vtab[-3cm]
\begin{center}
{\large FireTest \tab Número 337}
\end{center}
\begin{multicols}{2}
\begin{center}

Molecule A \
3NFAACg

\includegraphics[width=6cm]{../Comparisons/ImagesFromVMD/3NFAACg.png}

Inertia Tensor - Molecule A \\
\begin{tabular}{|c c c|}
532.891	 & 	-0.526938	 & 	-0.504713	 \\
-0.526938	 & 	918.454	 & 	2.35809	 \\
-0.504713	 & 	2.35809	 & 	1356.91
\end{tabular}

\vtab
 EingenVectors - Molecule A     \\
\begin{tabular}{|c c c|}
-0.999999	 & 	-0.00136295	 & 	-0.000608598	 \\
0.00135965	 & 	-0.999985	 & 	0.00537946	 \\
-0.000615921	 & 	0.00537863	 & 	0.999985
\end{tabular}

\vtab
 EingenValues - Molecule A     \\
\begin{tabular}{|c c c|}
532.89	 & 	918.442	 & 	1356.93	 \\
\end{tabular}
\columnbreak

Molecule B \
4NFAACf

\includegraphics[width=6cm]{../Comparisons/ImagesFromVMD/4NFAACf.png}

Inertia Tensor - Molecule B \\
\begin{tabular}{|c c c|}
509.683	 & 	2.80651	 & 	-1.91422	 \\
2.80651	 & 	1219.11	 & 	2.66132	 \\
-1.91422	 & 	2.66132	 & 	1681.17
\end{tabular}

\vtab
 EingenVectors - Molecule B     \\
\begin{tabular}{|c c c|}
-0.999991	 & 	0.00396206	 & 	-0.00164298	 \\
-0.00397143	 & 	-0.999976	 & 	0.0057431	 \\
-0.00162019	 & 	0.00574957	 & 	0.999982
\end{tabular}

\vtab
 EingenValues - Molecule B     \\
\begin{tabular}{|c c c|}
509.668	 & 	1219.11	 & 	1681.18	 \\
\end{tabular}

\end{center}
\end{multicols}

\vtab[-5mm]
\begin{tabular}{*{2}{m{0.38\textwidth}}}
\begin{center}
\textcolor{NavyBlue}{\Large Different}
\end{center}
&
\begin{center}
\includegraphics[height=6.5cm]{../Comparisons/Vectors/inertia_tensor_of_3NFAACg_and_4NFAACf.png}
\end{center}
\end{tabular}

 \newpage

\vtab[-3cm]
\begin{center}
{\large FireTest \tab Número 338}
\end{center}
\begin{multicols}{2}
\begin{center}

Molecule A \
3NFAACg

\includegraphics[width=6cm]{../Comparisons/ImagesFromVMD/3NFAACg.png}

Inertia Tensor - Molecule A \\
\begin{tabular}{|c c c|}
532.891	 & 	-0.526938	 & 	-0.504713	 \\
-0.526938	 & 	918.454	 & 	2.35809	 \\
-0.504713	 & 	2.35809	 & 	1356.91
\end{tabular}

\vtab
 EingenVectors - Molecule A     \\
\begin{tabular}{|c c c|}
-0.999999	 & 	-0.00136295	 & 	-0.000608598	 \\
0.00135965	 & 	-0.999985	 & 	0.00537946	 \\
-0.000615921	 & 	0.00537863	 & 	0.999985
\end{tabular}

\vtab
 EingenValues - Molecule A     \\
\begin{tabular}{|c c c|}
532.89	 & 	918.442	 & 	1356.93	 \\
\end{tabular}
\columnbreak

Molecule B \
4NFAACg

\includegraphics[width=6cm]{../Comparisons/ImagesFromVMD/4NFAACg.png}

Inertia Tensor - Molecule B \\
\begin{tabular}{|c c c|}
513.78	 & 	4.51917	 & 	0.266555	 \\
4.51917	 & 	1208.04	 & 	-1.18628	 \\
0.266555	 & 	-1.18628	 & 	1700.9
\end{tabular}

\vtab
 EingenVectors - Molecule B     \\
\begin{tabular}{|c c c|}
-0.999979	 & 	0.00650929	 & 	0.000231034	 \\
-0.00650983	 & 	-0.999976	 & 	-0.00240351	 \\
0.000215383	 & 	-0.00240496	 & 	0.999997
\end{tabular}

\vtab
 EingenValues - Molecule B     \\
\begin{tabular}{|c c c|}
513.751	 & 	1208.07	 & 	1700.9	 \\
\end{tabular}

\end{center}
\end{multicols}

\vtab[-5mm]
\begin{tabular}{*{2}{m{0.38\textwidth}}}
\begin{center}
\textcolor{NavyBlue}{\Large Different}
\end{center}
&
\begin{center}
\includegraphics[height=6.5cm]{../Comparisons/Vectors/inertia_tensor_of_3NFAACg_and_4NFAACg.png}
\end{center}
\end{tabular}

 \newpage

\vtab[-3cm]
\begin{center}
{\large FireTest \tab Número 339}
\end{center}
\begin{multicols}{2}
\begin{center}

Molecule A \
3NFAACg

\includegraphics[width=6cm]{../Comparisons/ImagesFromVMD/3NFAACg.png}

Inertia Tensor - Molecule A \\
\begin{tabular}{|c c c|}
532.891	 & 	-0.526938	 & 	-0.504713	 \\
-0.526938	 & 	918.454	 & 	2.35809	 \\
-0.504713	 & 	2.35809	 & 	1356.91
\end{tabular}

\vtab
 EingenVectors - Molecule A     \\
\begin{tabular}{|c c c|}
-0.999999	 & 	-0.00136295	 & 	-0.000608598	 \\
0.00135965	 & 	-0.999985	 & 	0.00537946	 \\
-0.000615921	 & 	0.00537863	 & 	0.999985
\end{tabular}

\vtab
 EingenValues - Molecule A     \\
\begin{tabular}{|c c c|}
532.89	 & 	918.442	 & 	1356.93	 \\
\end{tabular}
\columnbreak

Molecule B \
4NFAACi

\includegraphics[width=6cm]{../Comparisons/ImagesFromVMD/4NFAACi.png}

Inertia Tensor - Molecule B \\
\begin{tabular}{|c c c|}
502.43	 & 	-0.602691	 & 	-4.86988	 \\
-0.602691	 & 	1232.26	 & 	0.407295	 \\
-4.86988	 & 	0.407295	 & 	1676
\end{tabular}

\vtab
 EingenVectors - Molecule B     \\
\begin{tabular}{|c c c|}
0.999991	 & 	0.000823447	 & 	0.00414923	 \\
0.000819608	 & 	-0.999999	 & 	0.00092687	 \\
-0.00414999	 & 	0.000923461	 & 	0.999991
\end{tabular}

\vtab
 EingenValues - Molecule B     \\
\begin{tabular}{|c c c|}
502.409	 & 	1232.26	 & 	1676.02	 \\
\end{tabular}

\end{center}
\end{multicols}

\vtab[-5mm]
\begin{tabular}{*{2}{m{0.38\textwidth}}}
\begin{center}
\textcolor{NavyBlue}{\Large Different}
\end{center}
&
\begin{center}
\includegraphics[height=6.5cm]{../Comparisons/Vectors/inertia_tensor_of_3NFAACg_and_4NFAACi.png}
\end{center}
\end{tabular}

 \newpage

\vtab[-3cm]
\begin{center}
{\large FireTest \tab Número 340}
\end{center}
\begin{multicols}{2}
\begin{center}

Molecule A \
3NFAACg

\includegraphics[width=6cm]{../Comparisons/ImagesFromVMD/3NFAACg.png}

Inertia Tensor - Molecule A \\
\begin{tabular}{|c c c|}
532.891	 & 	-0.526938	 & 	-0.504713	 \\
-0.526938	 & 	918.454	 & 	2.35809	 \\
-0.504713	 & 	2.35809	 & 	1356.91
\end{tabular}

\vtab
 EingenVectors - Molecule A     \\
\begin{tabular}{|c c c|}
-0.999999	 & 	-0.00136295	 & 	-0.000608598	 \\
0.00135965	 & 	-0.999985	 & 	0.00537946	 \\
-0.000615921	 & 	0.00537863	 & 	0.999985
\end{tabular}

\vtab
 EingenValues - Molecule A     \\
\begin{tabular}{|c c c|}
532.89	 & 	918.442	 & 	1356.93	 \\
\end{tabular}
\columnbreak

Molecule B \
4NFAACj

\includegraphics[width=6cm]{../Comparisons/ImagesFromVMD/4NFAACj.png}

Inertia Tensor - Molecule B \\
\begin{tabular}{|c c c|}
510.047	 & 	9.97005	 & 	-3.6306	 \\
9.97005	 & 	1225.52	 & 	-0.981092	 \\
-3.6306	 & 	-0.981092	 & 	1680.82
\end{tabular}

\vtab
 EingenVectors - Molecule B     \\
\begin{tabular}{|c c c|}
-0.999898	 & 	0.0139264	 & 	-0.00308865	 \\
0.0139195	 & 	0.999901	 & 	0.00226627	 \\
-0.00311991	 & 	-0.00222305	 & 	0.999993
\end{tabular}

\vtab
 EingenValues - Molecule B     \\
\begin{tabular}{|c c c|}
509.897	 & 	1225.65	 & 	1680.83	 \\
\end{tabular}

\end{center}
\end{multicols}

\vtab[-5mm]
\begin{tabular}{*{2}{m{0.38\textwidth}}}
\begin{center}
\textcolor{NavyBlue}{\Large Different}
\end{center}
&
\begin{center}
\includegraphics[height=6.5cm]{../Comparisons/Vectors/inertia_tensor_of_3NFAACg_and_4NFAACj.png}
\end{center}
\end{tabular}

 \newpage

\vtab[-3cm]
\begin{center}
{\large FireTest \tab Número 341}
\end{center}
\begin{multicols}{2}
\begin{center}

Molecule A \
3NFAACg

\includegraphics[width=6cm]{../Comparisons/ImagesFromVMD/3NFAACg.png}

Inertia Tensor - Molecule A \\
\begin{tabular}{|c c c|}
532.891	 & 	-0.526938	 & 	-0.504713	 \\
-0.526938	 & 	918.454	 & 	2.35809	 \\
-0.504713	 & 	2.35809	 & 	1356.91
\end{tabular}

\vtab
 EingenVectors - Molecule A     \\
\begin{tabular}{|c c c|}
-0.999999	 & 	-0.00136295	 & 	-0.000608598	 \\
0.00135965	 & 	-0.999985	 & 	0.00537946	 \\
-0.000615921	 & 	0.00537863	 & 	0.999985
\end{tabular}

\vtab
 EingenValues - Molecule A     \\
\begin{tabular}{|c c c|}
532.89	 & 	918.442	 & 	1356.93	 \\
\end{tabular}
\columnbreak

Molecule B \
4NFAACl-3

\includegraphics[width=6cm]{../Comparisons/ImagesFromVMD/4NFAACl-3.png}

Inertia Tensor - Molecule B \\
\begin{tabular}{|c c c|}
506.608	 & 	0.709539	 & 	-0.555426	 \\
0.709539	 & 	1222.37	 & 	-2.84005	 \\
-0.555426	 & 	-2.84005	 & 	1678.41
\end{tabular}

\vtab
 EingenVectors - Molecule B     \\
\begin{tabular}{|c c c|}
-0.999999	 & 	0.000989428	 & 	-0.000471595	 \\
-0.000986471	 & 	-0.99998	 & 	-0.00622856	 \\
-0.000477748	 & 	-0.00622809	 & 	0.99998
\end{tabular}

\vtab
 EingenValues - Molecule B     \\
\begin{tabular}{|c c c|}
506.607	 & 	1222.36	 & 	1678.43	 \\
\end{tabular}

\end{center}
\end{multicols}

\vtab[-5mm]
\begin{tabular}{*{2}{m{0.38\textwidth}}}
\begin{center}
\textcolor{NavyBlue}{\Large Different}
\end{center}
&
\begin{center}
\includegraphics[height=6.5cm]{../Comparisons/Vectors/inertia_tensor_of_3NFAACg_and_4NFAACl-3.png}
\end{center}
\end{tabular}

 \newpage

\vtab[-3cm]
\begin{center}
{\large FireTest \tab Número 342}
\end{center}
\begin{multicols}{2}
\begin{center}

Molecule A \
3NFAACh

\includegraphics[width=6cm]{../Comparisons/ImagesFromVMD/3NFAACh.png}

Inertia Tensor - Molecule A \\
\begin{tabular}{|c c c|}
528.718	 & 	-3.89066	 & 	-3.66882	 \\
-3.89066	 & 	924.911	 & 	2.93817	 \\
-3.66882	 & 	2.93817	 & 	1336.08
\end{tabular}

\vtab
 EingenVectors - Molecule A     \\
\begin{tabular}{|c c c|}
0.999942	 & 	0.00978475	 & 	0.00450803	 \\
0.00975199	 & 	-0.999926	 & 	0.00723269	 \\
-0.00457847	 & 	0.0071883	 & 	0.999964
\end{tabular}

\vtab
 EingenValues - Molecule A     \\
\begin{tabular}{|c c c|}
528.663	 & 	924.928	 & 	1336.12	 \\
\end{tabular}
\columnbreak

Molecule B \
3NFAACi

\includegraphics[width=6cm]{../Comparisons/ImagesFromVMD/3NFAACi.png}

Inertia Tensor - Molecule B \\
\begin{tabular}{|c c c|}
521.487	 & 	-1.22943	 & 	-2.71042	 \\
-1.22943	 & 	949.663	 & 	1.67006	 \\
-2.71042	 & 	1.67006	 & 	1346.16
\end{tabular}

\vtab
 EingenVectors - Molecule B     \\
\begin{tabular}{|c c c|}
0.999991	 & 	0.00285841	 & 	0.00328077	 \\
0.00284452	 & 	-0.999987	 & 	0.00423133	 \\
-0.00329282	 & 	0.00422196	 & 	0.999986
\end{tabular}

\vtab
 EingenValues - Molecule B     \\
\begin{tabular}{|c c c|}
521.475	 & 	949.659	 & 	1346.18	 \\
\end{tabular}

\end{center}
\end{multicols}

\vtab[-5mm]
\begin{tabular}{*{2}{m{0.38\textwidth}}}
\begin{center}
\textcolor{NavyBlue}{\Large Different}
\end{center}
&
\begin{center}
\includegraphics[height=6.5cm]{../Comparisons/Vectors/inertia_tensor_of_3NFAACh_and_3NFAACi.png}
\end{center}
\end{tabular}

 \newpage

\vtab[-3cm]
\begin{center}
{\large FireTest \tab Número 343}
\end{center}
\begin{multicols}{2}
\begin{center}

Molecule A \
3NFAACh

\includegraphics[width=6cm]{../Comparisons/ImagesFromVMD/3NFAACh.png}

Inertia Tensor - Molecule A \\
\begin{tabular}{|c c c|}
528.718	 & 	-3.89066	 & 	-3.66882	 \\
-3.89066	 & 	924.911	 & 	2.93817	 \\
-3.66882	 & 	2.93817	 & 	1336.08
\end{tabular}

\vtab
 EingenVectors - Molecule A     \\
\begin{tabular}{|c c c|}
0.999942	 & 	0.00978475	 & 	0.00450803	 \\
0.00975199	 & 	-0.999926	 & 	0.00723269	 \\
-0.00457847	 & 	0.0071883	 & 	0.999964
\end{tabular}

\vtab
 EingenValues - Molecule A     \\
\begin{tabular}{|c c c|}
528.663	 & 	924.928	 & 	1336.12	 \\
\end{tabular}
\columnbreak

Molecule B \
3NFAACj

\includegraphics[width=6cm]{../Comparisons/ImagesFromVMD/3NFAACj.png}

Inertia Tensor - Molecule B \\
\begin{tabular}{|c c c|}
533.789	 & 	-4.75521	 & 	-1.91525	 \\
-4.75521	 & 	920.091	 & 	2.28449	 \\
-1.91525	 & 	2.28449	 & 	1348.28
\end{tabular}

\vtab
 EingenVectors - Molecule B     \\
\begin{tabular}{|c c c|}
-0.999922	 & 	-0.0122929	 & 	-0.00231663	 \\
0.0122803	 & 	-0.99991	 & 	0.00539026	 \\
-0.00238268	 & 	0.00536139	 & 	0.999983
\end{tabular}

\vtab
 EingenValues - Molecule B     \\
\begin{tabular}{|c c c|}
533.726	 & 	920.137	 & 	1348.3	 \\
\end{tabular}

\end{center}
\end{multicols}

\vtab[-5mm]
\begin{tabular}{*{2}{m{0.38\textwidth}}}
\begin{center}
\textcolor{NavyBlue}{\Large Different}
\end{center}
&
\begin{center}
\includegraphics[height=6.5cm]{../Comparisons/Vectors/inertia_tensor_of_3NFAACh_and_3NFAACj.png}
\end{center}
\end{tabular}

 \newpage

\vtab[-3cm]
\begin{center}
{\large FireTest \tab Número 344}
\end{center}
\begin{multicols}{2}
\begin{center}

Molecule A \
3NFAACh

\includegraphics[width=6cm]{../Comparisons/ImagesFromVMD/3NFAACh.png}

Inertia Tensor - Molecule A \\
\begin{tabular}{|c c c|}
528.718	 & 	-3.89066	 & 	-3.66882	 \\
-3.89066	 & 	924.911	 & 	2.93817	 \\
-3.66882	 & 	2.93817	 & 	1336.08
\end{tabular}

\vtab
 EingenVectors - Molecule A     \\
\begin{tabular}{|c c c|}
0.999942	 & 	0.00978475	 & 	0.00450803	 \\
0.00975199	 & 	-0.999926	 & 	0.00723269	 \\
-0.00457847	 & 	0.0071883	 & 	0.999964
\end{tabular}

\vtab
 EingenValues - Molecule A     \\
\begin{tabular}{|c c c|}
528.663	 & 	924.928	 & 	1336.12	 \\
\end{tabular}
\columnbreak

Molecule B \
3NFAACk

\includegraphics[width=6cm]{../Comparisons/ImagesFromVMD/3NFAACk.png}

Inertia Tensor - Molecule B \\
\begin{tabular}{|c c c|}
534.899	 & 	-5.81418	 & 	-0.270063	 \\
-5.81418	 & 	913.263	 & 	2.4519	 \\
-0.270063	 & 	2.4519	 & 	1353.77
\end{tabular}

\vtab
 EingenVectors - Molecule B     \\
\begin{tabular}{|c c c|}
-0.999882	 & 	-0.0153593	 & 	-0.00028374	 \\
0.0153575	 & 	-0.999867	 & 	0.00557573	 \\
-0.000369342	 & 	0.00557072	 & 	0.999984
\end{tabular}

\vtab
 EingenValues - Molecule B     \\
\begin{tabular}{|c c c|}
534.81	 & 	913.339	 & 	1353.78	 \\
\end{tabular}

\end{center}
\end{multicols}

\vtab[-5mm]
\begin{tabular}{*{2}{m{0.38\textwidth}}}
\begin{center}
\textcolor{NavyBlue}{\Large Different}
\end{center}
&
\begin{center}
\includegraphics[height=6.5cm]{../Comparisons/Vectors/inertia_tensor_of_3NFAACh_and_3NFAACk.png}
\end{center}
\end{tabular}

 \newpage

\vtab[-3cm]
\begin{center}
{\large FireTest \tab Número 345}
\end{center}
\begin{multicols}{2}
\begin{center}

Molecule A \
3NFAACh

\includegraphics[width=6cm]{../Comparisons/ImagesFromVMD/3NFAACh.png}

Inertia Tensor - Molecule A \\
\begin{tabular}{|c c c|}
528.718	 & 	-3.89066	 & 	-3.66882	 \\
-3.89066	 & 	924.911	 & 	2.93817	 \\
-3.66882	 & 	2.93817	 & 	1336.08
\end{tabular}

\vtab
 EingenVectors - Molecule A     \\
\begin{tabular}{|c c c|}
0.999942	 & 	0.00978475	 & 	0.00450803	 \\
0.00975199	 & 	-0.999926	 & 	0.00723269	 \\
-0.00457847	 & 	0.0071883	 & 	0.999964
\end{tabular}

\vtab
 EingenValues - Molecule A     \\
\begin{tabular}{|c c c|}
528.663	 & 	924.928	 & 	1336.12	 \\
\end{tabular}
\columnbreak

Molecule B \
3NFAACl

\includegraphics[width=6cm]{../Comparisons/ImagesFromVMD/3NFAACl.png}

Inertia Tensor - Molecule B \\
\begin{tabular}{|c c c|}
531.723	 & 	3.03424	 & 	2.73426	 \\
3.03424	 & 	929.418	 & 	-1.84284	 \\
2.73426	 & 	-1.84284	 & 	1355.39
\end{tabular}

\vtab
 EingenVectors - Molecule B     \\
\begin{tabular}{|c c c|}
0.999965	 & 	-0.00764413	 & 	-0.00333648	 \\
-0.00765838	 & 	-0.999962	 & 	-0.00427708	 \\
0.00330366	 & 	-0.00430248	 & 	0.999985
\end{tabular}

\vtab
 EingenValues - Molecule B     \\
\begin{tabular}{|c c c|}
531.691	 & 	929.434	 & 	1355.4	 \\
\end{tabular}

\end{center}
\end{multicols}

\vtab[-5mm]
\begin{tabular}{*{2}{m{0.38\textwidth}}}
\begin{center}
\textcolor{NavyBlue}{\Large Different}
\end{center}
&
\begin{center}
\includegraphics[height=6.5cm]{../Comparisons/Vectors/inertia_tensor_of_3NFAACh_and_3NFAACl.png}
\end{center}
\end{tabular}

 \newpage

\vtab[-3cm]
\begin{center}
{\large FireTest \tab Número 346}
\end{center}
\begin{multicols}{2}
\begin{center}

Molecule A \
3NFAACh

\includegraphics[width=6cm]{../Comparisons/ImagesFromVMD/3NFAACh.png}

Inertia Tensor - Molecule A \\
\begin{tabular}{|c c c|}
528.718	 & 	-3.89066	 & 	-3.66882	 \\
-3.89066	 & 	924.911	 & 	2.93817	 \\
-3.66882	 & 	2.93817	 & 	1336.08
\end{tabular}

\vtab
 EingenVectors - Molecule A     \\
\begin{tabular}{|c c c|}
0.999942	 & 	0.00978475	 & 	0.00450803	 \\
0.00975199	 & 	-0.999926	 & 	0.00723269	 \\
-0.00457847	 & 	0.0071883	 & 	0.999964
\end{tabular}

\vtab
 EingenValues - Molecule A     \\
\begin{tabular}{|c c c|}
528.663	 & 	924.928	 & 	1336.12	 \\
\end{tabular}
\columnbreak

Molecule B \
3NFAACm

\includegraphics[width=6cm]{../Comparisons/ImagesFromVMD/3NFAACm.png}

Inertia Tensor - Molecule B \\
\begin{tabular}{|c c c|}
532.546	 & 	13.7854	 & 	-15.4626	 \\
13.7854	 & 	1354.87	 & 	11.5786	 \\
-15.4626	 & 	11.5786	 & 	929.101
\end{tabular}

\vtab
 EingenVectors - Molecule B     \\
\begin{tabular}{|c c c|}
-0.999075	 & 	0.0172851	 & 	-0.0393769	 \\
0.0398168	 & 	0.0258942	 & 	-0.998871	 \\
-0.016246	 & 	-0.999515	 & 	-0.0265584
\end{tabular}

\vtab
 EingenValues - Molecule B     \\
\begin{tabular}{|c c c|}
531.698	 & 	929.417	 & 	1355.4	 \\
\end{tabular}

\end{center}
\end{multicols}

\vtab[-5mm]
\begin{tabular}{*{2}{m{0.38\textwidth}}}
\begin{center}
\textcolor{NavyBlue}{\Large Different}
\end{center}
&
\begin{center}
\includegraphics[height=6.5cm]{../Comparisons/Vectors/inertia_tensor_of_3NFAACh_and_3NFAACm.png}
\end{center}
\end{tabular}

 \newpage

\vtab[-3cm]
\begin{center}
{\large FireTest \tab Número 347}
\end{center}
\begin{multicols}{2}
\begin{center}

Molecule A \
3NFAACh

\includegraphics[width=6cm]{../Comparisons/ImagesFromVMD/3NFAACh.png}

Inertia Tensor - Molecule A \\
\begin{tabular}{|c c c|}
528.718	 & 	-3.89066	 & 	-3.66882	 \\
-3.89066	 & 	924.911	 & 	2.93817	 \\
-3.66882	 & 	2.93817	 & 	1336.08
\end{tabular}

\vtab
 EingenVectors - Molecule A     \\
\begin{tabular}{|c c c|}
0.999942	 & 	0.00978475	 & 	0.00450803	 \\
0.00975199	 & 	-0.999926	 & 	0.00723269	 \\
-0.00457847	 & 	0.0071883	 & 	0.999964
\end{tabular}

\vtab
 EingenValues - Molecule A     \\
\begin{tabular}{|c c c|}
528.663	 & 	924.928	 & 	1336.12	 \\
\end{tabular}
\columnbreak

Molecule B \
3NFAACn

\includegraphics[width=6cm]{../Comparisons/ImagesFromVMD/3NFAACn.png}

Inertia Tensor - Molecule B \\
\begin{tabular}{|c c c|}
531.896	 & 	3.78027	 & 	-13.1151	 \\
3.78027	 & 	1353.2	 & 	-7.47403	 \\
-13.1151	 & 	-7.47403	 & 	912.989
\end{tabular}

\vtab
 EingenVectors - Molecule B     \\
\begin{tabular}{|c c c|}
0.999403	 & 	-0.00428573	 & 	0.0342679	 \\
0.0341891	 & 	-0.0172718	 & 	-0.999266	 \\
0.00487445	 & 	0.999842	 & 	-0.017115
\end{tabular}

\vtab
 EingenValues - Molecule B     \\
\begin{tabular}{|c c c|}
531.43	 & 	913.309	 & 	1353.35	 \\
\end{tabular}

\end{center}
\end{multicols}

\vtab[-5mm]
\begin{tabular}{*{2}{m{0.38\textwidth}}}
\begin{center}
\textcolor{NavyBlue}{\Large Different}
\end{center}
&
\begin{center}
\includegraphics[height=6.5cm]{../Comparisons/Vectors/inertia_tensor_of_3NFAACh_and_3NFAACn.png}
\end{center}
\end{tabular}

 \newpage

\vtab[-3cm]
\begin{center}
{\large FireTest \tab Número 348}
\end{center}
\begin{multicols}{2}
\begin{center}

Molecule A \
3NFAACh

\includegraphics[width=6cm]{../Comparisons/ImagesFromVMD/3NFAACh.png}

Inertia Tensor - Molecule A \\
\begin{tabular}{|c c c|}
528.718	 & 	-3.89066	 & 	-3.66882	 \\
-3.89066	 & 	924.911	 & 	2.93817	 \\
-3.66882	 & 	2.93817	 & 	1336.08
\end{tabular}

\vtab
 EingenVectors - Molecule A     \\
\begin{tabular}{|c c c|}
0.999942	 & 	0.00978475	 & 	0.00450803	 \\
0.00975199	 & 	-0.999926	 & 	0.00723269	 \\
-0.00457847	 & 	0.0071883	 & 	0.999964
\end{tabular}

\vtab
 EingenValues - Molecule A     \\
\begin{tabular}{|c c c|}
528.663	 & 	924.928	 & 	1336.12	 \\
\end{tabular}
\columnbreak

Molecule B \
4NFAACa

\includegraphics[width=6cm]{../Comparisons/ImagesFromVMD/4NFAACa.png}

Inertia Tensor - Molecule B \\
\begin{tabular}{|c c c|}
479.392	 & 	3.27131	 & 	4.22557	 \\
3.27131	 & 	1242.39	 & 	-0.852684	 \\
4.22557	 & 	-0.852684	 & 	1647.37
\end{tabular}

\vtab
 EingenVectors - Molecule B     \\
\begin{tabular}{|c c c|}
0.999984	 & 	-0.00429123	 & 	-0.00362083	 \\
-0.00429871	 & 	-0.999989	 & 	-0.0020607	 \\
0.00361195	 & 	-0.00207623	 & 	0.999991
\end{tabular}

\vtab
 EingenValues - Molecule B     \\
\begin{tabular}{|c c c|}
479.363	 & 	1242.41	 & 	1647.39	 \\
\end{tabular}

\end{center}
\end{multicols}

\vtab[-5mm]
\begin{tabular}{*{2}{m{0.38\textwidth}}}
\begin{center}
\textcolor{NavyBlue}{\Large Different}
\end{center}
&
\begin{center}
\includegraphics[height=6.5cm]{../Comparisons/Vectors/inertia_tensor_of_3NFAACh_and_4NFAACa.png}
\end{center}
\end{tabular}

 \newpage

\vtab[-3cm]
\begin{center}
{\large FireTest \tab Número 349}
\end{center}
\begin{multicols}{2}
\begin{center}

Molecule A \
3NFAACh

\includegraphics[width=6cm]{../Comparisons/ImagesFromVMD/3NFAACh.png}

Inertia Tensor - Molecule A \\
\begin{tabular}{|c c c|}
528.718	 & 	-3.89066	 & 	-3.66882	 \\
-3.89066	 & 	924.911	 & 	2.93817	 \\
-3.66882	 & 	2.93817	 & 	1336.08
\end{tabular}

\vtab
 EingenVectors - Molecule A     \\
\begin{tabular}{|c c c|}
0.999942	 & 	0.00978475	 & 	0.00450803	 \\
0.00975199	 & 	-0.999926	 & 	0.00723269	 \\
-0.00457847	 & 	0.0071883	 & 	0.999964
\end{tabular}

\vtab
 EingenValues - Molecule A     \\
\begin{tabular}{|c c c|}
528.663	 & 	924.928	 & 	1336.12	 \\
\end{tabular}
\columnbreak

Molecule B \
4NFAACb

\includegraphics[width=6cm]{../Comparisons/ImagesFromVMD/4NFAACb.png}

Inertia Tensor - Molecule B \\
\begin{tabular}{|c c c|}
479.338	 & 	3.27331	 & 	-4.22553	 \\
3.27331	 & 	1242.4	 & 	0.852083	 \\
-4.22553	 & 	0.852083	 & 	1647.3
\end{tabular}

\vtab
 EingenVectors - Molecule B     \\
\begin{tabular}{|c c c|}
0.999984	 & 	-0.00429353	 & 	0.00362086	 \\
-0.004301	 & 	-0.999989	 & 	0.00205959	 \\
-0.00361198	 & 	0.00207513	 & 	0.999991
\end{tabular}

\vtab
 EingenValues - Molecule B     \\
\begin{tabular}{|c c c|}
479.308	 & 	1242.41	 & 	1647.31	 \\
\end{tabular}

\end{center}
\end{multicols}

\vtab[-5mm]
\begin{tabular}{*{2}{m{0.38\textwidth}}}
\begin{center}
\textcolor{NavyBlue}{\Large Different}
\end{center}
&
\begin{center}
\includegraphics[height=6.5cm]{../Comparisons/Vectors/inertia_tensor_of_3NFAACh_and_4NFAACb.png}
\end{center}
\end{tabular}

 \newpage

\vtab[-3cm]
\begin{center}
{\large FireTest \tab Número 350}
\end{center}
\begin{multicols}{2}
\begin{center}

Molecule A \
3NFAACh

\includegraphics[width=6cm]{../Comparisons/ImagesFromVMD/3NFAACh.png}

Inertia Tensor - Molecule A \\
\begin{tabular}{|c c c|}
528.718	 & 	-3.89066	 & 	-3.66882	 \\
-3.89066	 & 	924.911	 & 	2.93817	 \\
-3.66882	 & 	2.93817	 & 	1336.08
\end{tabular}

\vtab
 EingenVectors - Molecule A     \\
\begin{tabular}{|c c c|}
0.999942	 & 	0.00978475	 & 	0.00450803	 \\
0.00975199	 & 	-0.999926	 & 	0.00723269	 \\
-0.00457847	 & 	0.0071883	 & 	0.999964
\end{tabular}

\vtab
 EingenValues - Molecule A     \\
\begin{tabular}{|c c c|}
528.663	 & 	924.928	 & 	1336.12	 \\
\end{tabular}
\columnbreak

Molecule B \
4NFAACc

\includegraphics[width=6cm]{../Comparisons/ImagesFromVMD/4NFAACc.png}

Inertia Tensor - Molecule B \\
\begin{tabular}{|c c c|}
482.067	 & 	-5.39474	 & 	-1.35857	 \\
-5.39474	 & 	1240.3	 & 	-2.54035	 \\
-1.35857	 & 	-2.54035	 & 	1647.06
\end{tabular}

\vtab
 EingenVectors - Molecule B     \\
\begin{tabular}{|c c c|}
-0.999974	 & 	-0.00711826	 & 	-0.0011816	 \\
0.00712547	 & 	-0.999955	 & 	-0.00622156	 \\
-0.00113726	 & 	-0.00622982	 & 	0.99998
\end{tabular}

\vtab
 EingenValues - Molecule B     \\
\begin{tabular}{|c c c|}
482.027	 & 	1240.32	 & 	1647.08	 \\
\end{tabular}

\end{center}
\end{multicols}

\vtab[-5mm]
\begin{tabular}{*{2}{m{0.38\textwidth}}}
\begin{center}
\textcolor{NavyBlue}{\Large Different}
\end{center}
&
\begin{center}
\includegraphics[height=6.5cm]{../Comparisons/Vectors/inertia_tensor_of_3NFAACh_and_4NFAACc.png}
\end{center}
\end{tabular}

 \newpage

\vtab[-3cm]
\begin{center}
{\large FireTest \tab Número 351}
\end{center}
\begin{multicols}{2}
\begin{center}

Molecule A \
3NFAACh

\includegraphics[width=6cm]{../Comparisons/ImagesFromVMD/3NFAACh.png}

Inertia Tensor - Molecule A \\
\begin{tabular}{|c c c|}
528.718	 & 	-3.89066	 & 	-3.66882	 \\
-3.89066	 & 	924.911	 & 	2.93817	 \\
-3.66882	 & 	2.93817	 & 	1336.08
\end{tabular}

\vtab
 EingenVectors - Molecule A     \\
\begin{tabular}{|c c c|}
0.999942	 & 	0.00978475	 & 	0.00450803	 \\
0.00975199	 & 	-0.999926	 & 	0.00723269	 \\
-0.00457847	 & 	0.0071883	 & 	0.999964
\end{tabular}

\vtab
 EingenValues - Molecule A     \\
\begin{tabular}{|c c c|}
528.663	 & 	924.928	 & 	1336.12	 \\
\end{tabular}
\columnbreak

Molecule B \
4NFAACd

\includegraphics[width=6cm]{../Comparisons/ImagesFromVMD/4NFAACd.png}

Inertia Tensor - Molecule B \\
\begin{tabular}{|c c c|}
491.672	 & 	0.24486	 & 	-3.10016	 \\
0.24486	 & 	1231.15	 & 	2.19965	 \\
-3.10016	 & 	2.19965	 & 	1650.11
\end{tabular}

\vtab
 EingenVectors - Molecule B     \\
\begin{tabular}{|c c c|}
0.999996	 & 	-0.000339081	 & 	0.00267677	 \\
-0.000353124	 & 	-0.999986	 & 	0.00524747	 \\
-0.00267495	 & 	0.0052484	 & 	0.999983
\end{tabular}

\vtab
 EingenValues - Molecule B     \\
\begin{tabular}{|c c c|}
491.663	 & 	1231.14	 & 	1650.13	 \\
\end{tabular}

\end{center}
\end{multicols}

\vtab[-5mm]
\begin{tabular}{*{2}{m{0.38\textwidth}}}
\begin{center}
\textcolor{NavyBlue}{\Large Different}
\end{center}
&
\begin{center}
\includegraphics[height=6.5cm]{../Comparisons/Vectors/inertia_tensor_of_3NFAACh_and_4NFAACd.png}
\end{center}
\end{tabular}

 \newpage

\vtab[-3cm]
\begin{center}
{\large FireTest \tab Número 352}
\end{center}
\begin{multicols}{2}
\begin{center}

Molecule A \
3NFAACh

\includegraphics[width=6cm]{../Comparisons/ImagesFromVMD/3NFAACh.png}

Inertia Tensor - Molecule A \\
\begin{tabular}{|c c c|}
528.718	 & 	-3.89066	 & 	-3.66882	 \\
-3.89066	 & 	924.911	 & 	2.93817	 \\
-3.66882	 & 	2.93817	 & 	1336.08
\end{tabular}

\vtab
 EingenVectors - Molecule A     \\
\begin{tabular}{|c c c|}
0.999942	 & 	0.00978475	 & 	0.00450803	 \\
0.00975199	 & 	-0.999926	 & 	0.00723269	 \\
-0.00457847	 & 	0.0071883	 & 	0.999964
\end{tabular}

\vtab
 EingenValues - Molecule A     \\
\begin{tabular}{|c c c|}
528.663	 & 	924.928	 & 	1336.12	 \\
\end{tabular}
\columnbreak

Molecule B \
4NFAACe

\includegraphics[width=6cm]{../Comparisons/ImagesFromVMD/4NFAACe.png}

Inertia Tensor - Molecule B \\
\begin{tabular}{|c c c|}
489.025	 & 	-0.430035	 & 	3.98876	 \\
-0.430035	 & 	1233.71	 & 	-2.06505	 \\
3.98876	 & 	-2.06505	 & 	1641.79
\end{tabular}

\vtab
 EingenVectors - Molecule B     \\
\begin{tabular}{|c c c|}
0.999994	 & 	0.000567863	 & 	-0.00345908	 \\
0.000550336	 & 	-0.999987	 & 	-0.00506565	 \\
0.00346192	 & 	-0.00506372	 & 	0.999981
\end{tabular}

\vtab
 EingenValues - Molecule B     \\
\begin{tabular}{|c c c|}
489.011	 & 	1233.7	 & 	1641.81	 \\
\end{tabular}

\end{center}
\end{multicols}

\vtab[-5mm]
\begin{tabular}{*{2}{m{0.38\textwidth}}}
\begin{center}
\textcolor{NavyBlue}{\Large Different}
\end{center}
&
\begin{center}
\includegraphics[height=6.5cm]{../Comparisons/Vectors/inertia_tensor_of_3NFAACh_and_4NFAACe.png}
\end{center}
\end{tabular}

 \newpage

\vtab[-3cm]
\begin{center}
{\large FireTest \tab Número 353}
\end{center}
\begin{multicols}{2}
\begin{center}

Molecule A \
3NFAACh

\includegraphics[width=6cm]{../Comparisons/ImagesFromVMD/3NFAACh.png}

Inertia Tensor - Molecule A \\
\begin{tabular}{|c c c|}
528.718	 & 	-3.89066	 & 	-3.66882	 \\
-3.89066	 & 	924.911	 & 	2.93817	 \\
-3.66882	 & 	2.93817	 & 	1336.08
\end{tabular}

\vtab
 EingenVectors - Molecule A     \\
\begin{tabular}{|c c c|}
0.999942	 & 	0.00978475	 & 	0.00450803	 \\
0.00975199	 & 	-0.999926	 & 	0.00723269	 \\
-0.00457847	 & 	0.0071883	 & 	0.999964
\end{tabular}

\vtab
 EingenValues - Molecule A     \\
\begin{tabular}{|c c c|}
528.663	 & 	924.928	 & 	1336.12	 \\
\end{tabular}
\columnbreak

Molecule B \
4NFAACf

\includegraphics[width=6cm]{../Comparisons/ImagesFromVMD/4NFAACf.png}

Inertia Tensor - Molecule B \\
\begin{tabular}{|c c c|}
509.683	 & 	2.80651	 & 	-1.91422	 \\
2.80651	 & 	1219.11	 & 	2.66132	 \\
-1.91422	 & 	2.66132	 & 	1681.17
\end{tabular}

\vtab
 EingenVectors - Molecule B     \\
\begin{tabular}{|c c c|}
-0.999991	 & 	0.00396206	 & 	-0.00164298	 \\
-0.00397143	 & 	-0.999976	 & 	0.0057431	 \\
-0.00162019	 & 	0.00574957	 & 	0.999982
\end{tabular}

\vtab
 EingenValues - Molecule B     \\
\begin{tabular}{|c c c|}
509.668	 & 	1219.11	 & 	1681.18	 \\
\end{tabular}

\end{center}
\end{multicols}

\vtab[-5mm]
\begin{tabular}{*{2}{m{0.38\textwidth}}}
\begin{center}
\textcolor{NavyBlue}{\Large Different}
\end{center}
&
\begin{center}
\includegraphics[height=6.5cm]{../Comparisons/Vectors/inertia_tensor_of_3NFAACh_and_4NFAACf.png}
\end{center}
\end{tabular}

 \newpage

\vtab[-3cm]
\begin{center}
{\large FireTest \tab Número 354}
\end{center}
\begin{multicols}{2}
\begin{center}

Molecule A \
3NFAACh

\includegraphics[width=6cm]{../Comparisons/ImagesFromVMD/3NFAACh.png}

Inertia Tensor - Molecule A \\
\begin{tabular}{|c c c|}
528.718	 & 	-3.89066	 & 	-3.66882	 \\
-3.89066	 & 	924.911	 & 	2.93817	 \\
-3.66882	 & 	2.93817	 & 	1336.08
\end{tabular}

\vtab
 EingenVectors - Molecule A     \\
\begin{tabular}{|c c c|}
0.999942	 & 	0.00978475	 & 	0.00450803	 \\
0.00975199	 & 	-0.999926	 & 	0.00723269	 \\
-0.00457847	 & 	0.0071883	 & 	0.999964
\end{tabular}

\vtab
 EingenValues - Molecule A     \\
\begin{tabular}{|c c c|}
528.663	 & 	924.928	 & 	1336.12	 \\
\end{tabular}
\columnbreak

Molecule B \
4NFAACg

\includegraphics[width=6cm]{../Comparisons/ImagesFromVMD/4NFAACg.png}

Inertia Tensor - Molecule B \\
\begin{tabular}{|c c c|}
513.78	 & 	4.51917	 & 	0.266555	 \\
4.51917	 & 	1208.04	 & 	-1.18628	 \\
0.266555	 & 	-1.18628	 & 	1700.9
\end{tabular}

\vtab
 EingenVectors - Molecule B     \\
\begin{tabular}{|c c c|}
-0.999979	 & 	0.00650929	 & 	0.000231034	 \\
-0.00650983	 & 	-0.999976	 & 	-0.00240351	 \\
0.000215383	 & 	-0.00240496	 & 	0.999997
\end{tabular}

\vtab
 EingenValues - Molecule B     \\
\begin{tabular}{|c c c|}
513.751	 & 	1208.07	 & 	1700.9	 \\
\end{tabular}

\end{center}
\end{multicols}

\vtab[-5mm]
\begin{tabular}{*{2}{m{0.38\textwidth}}}
\begin{center}
\textcolor{NavyBlue}{\Large Different}
\end{center}
&
\begin{center}
\includegraphics[height=6.5cm]{../Comparisons/Vectors/inertia_tensor_of_3NFAACh_and_4NFAACg.png}
\end{center}
\end{tabular}

 \newpage

\vtab[-3cm]
\begin{center}
{\large FireTest \tab Número 355}
\end{center}
\begin{multicols}{2}
\begin{center}

Molecule A \
3NFAACh

\includegraphics[width=6cm]{../Comparisons/ImagesFromVMD/3NFAACh.png}

Inertia Tensor - Molecule A \\
\begin{tabular}{|c c c|}
528.718	 & 	-3.89066	 & 	-3.66882	 \\
-3.89066	 & 	924.911	 & 	2.93817	 \\
-3.66882	 & 	2.93817	 & 	1336.08
\end{tabular}

\vtab
 EingenVectors - Molecule A     \\
\begin{tabular}{|c c c|}
0.999942	 & 	0.00978475	 & 	0.00450803	 \\
0.00975199	 & 	-0.999926	 & 	0.00723269	 \\
-0.00457847	 & 	0.0071883	 & 	0.999964
\end{tabular}

\vtab
 EingenValues - Molecule A     \\
\begin{tabular}{|c c c|}
528.663	 & 	924.928	 & 	1336.12	 \\
\end{tabular}
\columnbreak

Molecule B \
4NFAACi

\includegraphics[width=6cm]{../Comparisons/ImagesFromVMD/4NFAACi.png}

Inertia Tensor - Molecule B \\
\begin{tabular}{|c c c|}
502.43	 & 	-0.602691	 & 	-4.86988	 \\
-0.602691	 & 	1232.26	 & 	0.407295	 \\
-4.86988	 & 	0.407295	 & 	1676
\end{tabular}

\vtab
 EingenVectors - Molecule B     \\
\begin{tabular}{|c c c|}
0.999991	 & 	0.000823447	 & 	0.00414923	 \\
0.000819608	 & 	-0.999999	 & 	0.00092687	 \\
-0.00414999	 & 	0.000923461	 & 	0.999991
\end{tabular}

\vtab
 EingenValues - Molecule B     \\
\begin{tabular}{|c c c|}
502.409	 & 	1232.26	 & 	1676.02	 \\
\end{tabular}

\end{center}
\end{multicols}

\vtab[-5mm]
\begin{tabular}{*{2}{m{0.38\textwidth}}}
\begin{center}
\textcolor{NavyBlue}{\Large Different}
\end{center}
&
\begin{center}
\includegraphics[height=6.5cm]{../Comparisons/Vectors/inertia_tensor_of_3NFAACh_and_4NFAACi.png}
\end{center}
\end{tabular}

 \newpage

\vtab[-3cm]
\begin{center}
{\large FireTest \tab Número 356}
\end{center}
\begin{multicols}{2}
\begin{center}

Molecule A \
3NFAACh

\includegraphics[width=6cm]{../Comparisons/ImagesFromVMD/3NFAACh.png}

Inertia Tensor - Molecule A \\
\begin{tabular}{|c c c|}
528.718	 & 	-3.89066	 & 	-3.66882	 \\
-3.89066	 & 	924.911	 & 	2.93817	 \\
-3.66882	 & 	2.93817	 & 	1336.08
\end{tabular}

\vtab
 EingenVectors - Molecule A     \\
\begin{tabular}{|c c c|}
0.999942	 & 	0.00978475	 & 	0.00450803	 \\
0.00975199	 & 	-0.999926	 & 	0.00723269	 \\
-0.00457847	 & 	0.0071883	 & 	0.999964
\end{tabular}

\vtab
 EingenValues - Molecule A     \\
\begin{tabular}{|c c c|}
528.663	 & 	924.928	 & 	1336.12	 \\
\end{tabular}
\columnbreak

Molecule B \
4NFAACj

\includegraphics[width=6cm]{../Comparisons/ImagesFromVMD/4NFAACj.png}

Inertia Tensor - Molecule B \\
\begin{tabular}{|c c c|}
510.047	 & 	9.97005	 & 	-3.6306	 \\
9.97005	 & 	1225.52	 & 	-0.981092	 \\
-3.6306	 & 	-0.981092	 & 	1680.82
\end{tabular}

\vtab
 EingenVectors - Molecule B     \\
\begin{tabular}{|c c c|}
-0.999898	 & 	0.0139264	 & 	-0.00308865	 \\
0.0139195	 & 	0.999901	 & 	0.00226627	 \\
-0.00311991	 & 	-0.00222305	 & 	0.999993
\end{tabular}

\vtab
 EingenValues - Molecule B     \\
\begin{tabular}{|c c c|}
509.897	 & 	1225.65	 & 	1680.83	 \\
\end{tabular}

\end{center}
\end{multicols}

\vtab[-5mm]
\begin{tabular}{*{2}{m{0.38\textwidth}}}
\begin{center}
\textcolor{NavyBlue}{\Large Different}
\end{center}
&
\begin{center}
\includegraphics[height=6.5cm]{../Comparisons/Vectors/inertia_tensor_of_3NFAACh_and_4NFAACj.png}
\end{center}
\end{tabular}

 \newpage

\vtab[-3cm]
\begin{center}
{\large FireTest \tab Número 357}
\end{center}
\begin{multicols}{2}
\begin{center}

Molecule A \
3NFAACh

\includegraphics[width=6cm]{../Comparisons/ImagesFromVMD/3NFAACh.png}

Inertia Tensor - Molecule A \\
\begin{tabular}{|c c c|}
528.718	 & 	-3.89066	 & 	-3.66882	 \\
-3.89066	 & 	924.911	 & 	2.93817	 \\
-3.66882	 & 	2.93817	 & 	1336.08
\end{tabular}

\vtab
 EingenVectors - Molecule A     \\
\begin{tabular}{|c c c|}
0.999942	 & 	0.00978475	 & 	0.00450803	 \\
0.00975199	 & 	-0.999926	 & 	0.00723269	 \\
-0.00457847	 & 	0.0071883	 & 	0.999964
\end{tabular}

\vtab
 EingenValues - Molecule A     \\
\begin{tabular}{|c c c|}
528.663	 & 	924.928	 & 	1336.12	 \\
\end{tabular}
\columnbreak

Molecule B \
4NFAACl-3

\includegraphics[width=6cm]{../Comparisons/ImagesFromVMD/4NFAACl-3.png}

Inertia Tensor - Molecule B \\
\begin{tabular}{|c c c|}
506.608	 & 	0.709539	 & 	-0.555426	 \\
0.709539	 & 	1222.37	 & 	-2.84005	 \\
-0.555426	 & 	-2.84005	 & 	1678.41
\end{tabular}

\vtab
 EingenVectors - Molecule B     \\
\begin{tabular}{|c c c|}
-0.999999	 & 	0.000989428	 & 	-0.000471595	 \\
-0.000986471	 & 	-0.99998	 & 	-0.00622856	 \\
-0.000477748	 & 	-0.00622809	 & 	0.99998
\end{tabular}

\vtab
 EingenValues - Molecule B     \\
\begin{tabular}{|c c c|}
506.607	 & 	1222.36	 & 	1678.43	 \\
\end{tabular}

\end{center}
\end{multicols}

\vtab[-5mm]
\begin{tabular}{*{2}{m{0.38\textwidth}}}
\begin{center}
\textcolor{NavyBlue}{\Large Different}
\end{center}
&
\begin{center}
\includegraphics[height=6.5cm]{../Comparisons/Vectors/inertia_tensor_of_3NFAACh_and_4NFAACl-3.png}
\end{center}
\end{tabular}

 \newpage

\vtab[-3cm]
\begin{center}
{\large FireTest \tab Número 358}
\end{center}
\begin{multicols}{2}
\begin{center}

Molecule A \
3NFAACi

\includegraphics[width=6cm]{../Comparisons/ImagesFromVMD/3NFAACi.png}

Inertia Tensor - Molecule A \\
\begin{tabular}{|c c c|}
521.487	 & 	-1.22943	 & 	-2.71042	 \\
-1.22943	 & 	949.663	 & 	1.67006	 \\
-2.71042	 & 	1.67006	 & 	1346.16
\end{tabular}

\vtab
 EingenVectors - Molecule A     \\
\begin{tabular}{|c c c|}
0.999991	 & 	0.00285841	 & 	0.00328077	 \\
0.00284452	 & 	-0.999987	 & 	0.00423133	 \\
-0.00329282	 & 	0.00422196	 & 	0.999986
\end{tabular}

\vtab
 EingenValues - Molecule A     \\
\begin{tabular}{|c c c|}
521.475	 & 	949.659	 & 	1346.18	 \\
\end{tabular}
\columnbreak

Molecule B \
3NFAACj

\includegraphics[width=6cm]{../Comparisons/ImagesFromVMD/3NFAACj.png}

Inertia Tensor - Molecule B \\
\begin{tabular}{|c c c|}
533.789	 & 	-4.75521	 & 	-1.91525	 \\
-4.75521	 & 	920.091	 & 	2.28449	 \\
-1.91525	 & 	2.28449	 & 	1348.28
\end{tabular}

\vtab
 EingenVectors - Molecule B     \\
\begin{tabular}{|c c c|}
-0.999922	 & 	-0.0122929	 & 	-0.00231663	 \\
0.0122803	 & 	-0.99991	 & 	0.00539026	 \\
-0.00238268	 & 	0.00536139	 & 	0.999983
\end{tabular}

\vtab
 EingenValues - Molecule B     \\
\begin{tabular}{|c c c|}
533.726	 & 	920.137	 & 	1348.3	 \\
\end{tabular}

\end{center}
\end{multicols}

\vtab[-5mm]
\begin{tabular}{*{2}{m{0.38\textwidth}}}
\begin{center}
\textcolor{NavyBlue}{\Large Different}
\end{center}
&
\begin{center}
\includegraphics[height=6.5cm]{../Comparisons/Vectors/inertia_tensor_of_3NFAACi_and_3NFAACj.png}
\end{center}
\end{tabular}

 \newpage

\vtab[-3cm]
\begin{center}
{\large FireTest \tab Número 359}
\end{center}
\begin{multicols}{2}
\begin{center}

Molecule A \
3NFAACi

\includegraphics[width=6cm]{../Comparisons/ImagesFromVMD/3NFAACi.png}

Inertia Tensor - Molecule A \\
\begin{tabular}{|c c c|}
521.487	 & 	-1.22943	 & 	-2.71042	 \\
-1.22943	 & 	949.663	 & 	1.67006	 \\
-2.71042	 & 	1.67006	 & 	1346.16
\end{tabular}

\vtab
 EingenVectors - Molecule A     \\
\begin{tabular}{|c c c|}
0.999991	 & 	0.00285841	 & 	0.00328077	 \\
0.00284452	 & 	-0.999987	 & 	0.00423133	 \\
-0.00329282	 & 	0.00422196	 & 	0.999986
\end{tabular}

\vtab
 EingenValues - Molecule A     \\
\begin{tabular}{|c c c|}
521.475	 & 	949.659	 & 	1346.18	 \\
\end{tabular}
\columnbreak

Molecule B \
3NFAACk

\includegraphics[width=6cm]{../Comparisons/ImagesFromVMD/3NFAACk.png}

Inertia Tensor - Molecule B \\
\begin{tabular}{|c c c|}
534.899	 & 	-5.81418	 & 	-0.270063	 \\
-5.81418	 & 	913.263	 & 	2.4519	 \\
-0.270063	 & 	2.4519	 & 	1353.77
\end{tabular}

\vtab
 EingenVectors - Molecule B     \\
\begin{tabular}{|c c c|}
-0.999882	 & 	-0.0153593	 & 	-0.00028374	 \\
0.0153575	 & 	-0.999867	 & 	0.00557573	 \\
-0.000369342	 & 	0.00557072	 & 	0.999984
\end{tabular}

\vtab
 EingenValues - Molecule B     \\
\begin{tabular}{|c c c|}
534.81	 & 	913.339	 & 	1353.78	 \\
\end{tabular}

\end{center}
\end{multicols}

\vtab[-5mm]
\begin{tabular}{*{2}{m{0.38\textwidth}}}
\begin{center}
\textcolor{NavyBlue}{\Large Different}
\end{center}
&
\begin{center}
\includegraphics[height=6.5cm]{../Comparisons/Vectors/inertia_tensor_of_3NFAACi_and_3NFAACk.png}
\end{center}
\end{tabular}

 \newpage

\vtab[-3cm]
\begin{center}
{\large FireTest \tab Número 360}
\end{center}
\begin{multicols}{2}
\begin{center}

Molecule A \
3NFAACi

\includegraphics[width=6cm]{../Comparisons/ImagesFromVMD/3NFAACi.png}

Inertia Tensor - Molecule A \\
\begin{tabular}{|c c c|}
521.487	 & 	-1.22943	 & 	-2.71042	 \\
-1.22943	 & 	949.663	 & 	1.67006	 \\
-2.71042	 & 	1.67006	 & 	1346.16
\end{tabular}

\vtab
 EingenVectors - Molecule A     \\
\begin{tabular}{|c c c|}
0.999991	 & 	0.00285841	 & 	0.00328077	 \\
0.00284452	 & 	-0.999987	 & 	0.00423133	 \\
-0.00329282	 & 	0.00422196	 & 	0.999986
\end{tabular}

\vtab
 EingenValues - Molecule A     \\
\begin{tabular}{|c c c|}
521.475	 & 	949.659	 & 	1346.18	 \\
\end{tabular}
\columnbreak

Molecule B \
3NFAACl

\includegraphics[width=6cm]{../Comparisons/ImagesFromVMD/3NFAACl.png}

Inertia Tensor - Molecule B \\
\begin{tabular}{|c c c|}
531.723	 & 	3.03424	 & 	2.73426	 \\
3.03424	 & 	929.418	 & 	-1.84284	 \\
2.73426	 & 	-1.84284	 & 	1355.39
\end{tabular}

\vtab
 EingenVectors - Molecule B     \\
\begin{tabular}{|c c c|}
0.999965	 & 	-0.00764413	 & 	-0.00333648	 \\
-0.00765838	 & 	-0.999962	 & 	-0.00427708	 \\
0.00330366	 & 	-0.00430248	 & 	0.999985
\end{tabular}

\vtab
 EingenValues - Molecule B     \\
\begin{tabular}{|c c c|}
531.691	 & 	929.434	 & 	1355.4	 \\
\end{tabular}

\end{center}
\end{multicols}

\vtab[-5mm]
\begin{tabular}{*{2}{m{0.38\textwidth}}}
\begin{center}
\textcolor{NavyBlue}{\Large Different}
\end{center}
&
\begin{center}
\includegraphics[height=6.5cm]{../Comparisons/Vectors/inertia_tensor_of_3NFAACi_and_3NFAACl.png}
\end{center}
\end{tabular}

 \newpage

\vtab[-3cm]
\begin{center}
{\large FireTest \tab Número 361}
\end{center}
\begin{multicols}{2}
\begin{center}

Molecule A \
3NFAACi

\includegraphics[width=6cm]{../Comparisons/ImagesFromVMD/3NFAACi.png}

Inertia Tensor - Molecule A \\
\begin{tabular}{|c c c|}
521.487	 & 	-1.22943	 & 	-2.71042	 \\
-1.22943	 & 	949.663	 & 	1.67006	 \\
-2.71042	 & 	1.67006	 & 	1346.16
\end{tabular}

\vtab
 EingenVectors - Molecule A     \\
\begin{tabular}{|c c c|}
0.999991	 & 	0.00285841	 & 	0.00328077	 \\
0.00284452	 & 	-0.999987	 & 	0.00423133	 \\
-0.00329282	 & 	0.00422196	 & 	0.999986
\end{tabular}

\vtab
 EingenValues - Molecule A     \\
\begin{tabular}{|c c c|}
521.475	 & 	949.659	 & 	1346.18	 \\
\end{tabular}
\columnbreak

Molecule B \
3NFAACm

\includegraphics[width=6cm]{../Comparisons/ImagesFromVMD/3NFAACm.png}

Inertia Tensor - Molecule B \\
\begin{tabular}{|c c c|}
532.546	 & 	13.7854	 & 	-15.4626	 \\
13.7854	 & 	1354.87	 & 	11.5786	 \\
-15.4626	 & 	11.5786	 & 	929.101
\end{tabular}

\vtab
 EingenVectors - Molecule B     \\
\begin{tabular}{|c c c|}
-0.999075	 & 	0.0172851	 & 	-0.0393769	 \\
0.0398168	 & 	0.0258942	 & 	-0.998871	 \\
-0.016246	 & 	-0.999515	 & 	-0.0265584
\end{tabular}

\vtab
 EingenValues - Molecule B     \\
\begin{tabular}{|c c c|}
531.698	 & 	929.417	 & 	1355.4	 \\
\end{tabular}

\end{center}
\end{multicols}

\vtab[-5mm]
\begin{tabular}{*{2}{m{0.38\textwidth}}}
\begin{center}
\textcolor{NavyBlue}{\Large Different}
\end{center}
&
\begin{center}
\includegraphics[height=6.5cm]{../Comparisons/Vectors/inertia_tensor_of_3NFAACi_and_3NFAACm.png}
\end{center}
\end{tabular}

 \newpage

\vtab[-3cm]
\begin{center}
{\large FireTest \tab Número 362}
\end{center}
\begin{multicols}{2}
\begin{center}

Molecule A \
3NFAACi

\includegraphics[width=6cm]{../Comparisons/ImagesFromVMD/3NFAACi.png}

Inertia Tensor - Molecule A \\
\begin{tabular}{|c c c|}
521.487	 & 	-1.22943	 & 	-2.71042	 \\
-1.22943	 & 	949.663	 & 	1.67006	 \\
-2.71042	 & 	1.67006	 & 	1346.16
\end{tabular}

\vtab
 EingenVectors - Molecule A     \\
\begin{tabular}{|c c c|}
0.999991	 & 	0.00285841	 & 	0.00328077	 \\
0.00284452	 & 	-0.999987	 & 	0.00423133	 \\
-0.00329282	 & 	0.00422196	 & 	0.999986
\end{tabular}

\vtab
 EingenValues - Molecule A     \\
\begin{tabular}{|c c c|}
521.475	 & 	949.659	 & 	1346.18	 \\
\end{tabular}
\columnbreak

Molecule B \
3NFAACn

\includegraphics[width=6cm]{../Comparisons/ImagesFromVMD/3NFAACn.png}

Inertia Tensor - Molecule B \\
\begin{tabular}{|c c c|}
531.896	 & 	3.78027	 & 	-13.1151	 \\
3.78027	 & 	1353.2	 & 	-7.47403	 \\
-13.1151	 & 	-7.47403	 & 	912.989
\end{tabular}

\vtab
 EingenVectors - Molecule B     \\
\begin{tabular}{|c c c|}
0.999403	 & 	-0.00428573	 & 	0.0342679	 \\
0.0341891	 & 	-0.0172718	 & 	-0.999266	 \\
0.00487445	 & 	0.999842	 & 	-0.017115
\end{tabular}

\vtab
 EingenValues - Molecule B     \\
\begin{tabular}{|c c c|}
531.43	 & 	913.309	 & 	1353.35	 \\
\end{tabular}

\end{center}
\end{multicols}

\vtab[-5mm]
\begin{tabular}{*{2}{m{0.38\textwidth}}}
\begin{center}
\textcolor{NavyBlue}{\Large Different}
\end{center}
&
\begin{center}
\includegraphics[height=6.5cm]{../Comparisons/Vectors/inertia_tensor_of_3NFAACi_and_3NFAACn.png}
\end{center}
\end{tabular}

 \newpage

\vtab[-3cm]
\begin{center}
{\large FireTest \tab Número 363}
\end{center}
\begin{multicols}{2}
\begin{center}

Molecule A \
3NFAACi

\includegraphics[width=6cm]{../Comparisons/ImagesFromVMD/3NFAACi.png}

Inertia Tensor - Molecule A \\
\begin{tabular}{|c c c|}
521.487	 & 	-1.22943	 & 	-2.71042	 \\
-1.22943	 & 	949.663	 & 	1.67006	 \\
-2.71042	 & 	1.67006	 & 	1346.16
\end{tabular}

\vtab
 EingenVectors - Molecule A     \\
\begin{tabular}{|c c c|}
0.999991	 & 	0.00285841	 & 	0.00328077	 \\
0.00284452	 & 	-0.999987	 & 	0.00423133	 \\
-0.00329282	 & 	0.00422196	 & 	0.999986
\end{tabular}

\vtab
 EingenValues - Molecule A     \\
\begin{tabular}{|c c c|}
521.475	 & 	949.659	 & 	1346.18	 \\
\end{tabular}
\columnbreak

Molecule B \
4NFAACa

\includegraphics[width=6cm]{../Comparisons/ImagesFromVMD/4NFAACa.png}

Inertia Tensor - Molecule B \\
\begin{tabular}{|c c c|}
479.392	 & 	3.27131	 & 	4.22557	 \\
3.27131	 & 	1242.39	 & 	-0.852684	 \\
4.22557	 & 	-0.852684	 & 	1647.37
\end{tabular}

\vtab
 EingenVectors - Molecule B     \\
\begin{tabular}{|c c c|}
0.999984	 & 	-0.00429123	 & 	-0.00362083	 \\
-0.00429871	 & 	-0.999989	 & 	-0.0020607	 \\
0.00361195	 & 	-0.00207623	 & 	0.999991
\end{tabular}

\vtab
 EingenValues - Molecule B     \\
\begin{tabular}{|c c c|}
479.363	 & 	1242.41	 & 	1647.39	 \\
\end{tabular}

\end{center}
\end{multicols}

\vtab[-5mm]
\begin{tabular}{*{2}{m{0.38\textwidth}}}
\begin{center}
\textcolor{NavyBlue}{\Large Different}
\end{center}
&
\begin{center}
\includegraphics[height=6.5cm]{../Comparisons/Vectors/inertia_tensor_of_3NFAACi_and_4NFAACa.png}
\end{center}
\end{tabular}

 \newpage

\vtab[-3cm]
\begin{center}
{\large FireTest \tab Número 364}
\end{center}
\begin{multicols}{2}
\begin{center}

Molecule A \
3NFAACi

\includegraphics[width=6cm]{../Comparisons/ImagesFromVMD/3NFAACi.png}

Inertia Tensor - Molecule A \\
\begin{tabular}{|c c c|}
521.487	 & 	-1.22943	 & 	-2.71042	 \\
-1.22943	 & 	949.663	 & 	1.67006	 \\
-2.71042	 & 	1.67006	 & 	1346.16
\end{tabular}

\vtab
 EingenVectors - Molecule A     \\
\begin{tabular}{|c c c|}
0.999991	 & 	0.00285841	 & 	0.00328077	 \\
0.00284452	 & 	-0.999987	 & 	0.00423133	 \\
-0.00329282	 & 	0.00422196	 & 	0.999986
\end{tabular}

\vtab
 EingenValues - Molecule A     \\
\begin{tabular}{|c c c|}
521.475	 & 	949.659	 & 	1346.18	 \\
\end{tabular}
\columnbreak

Molecule B \
4NFAACb

\includegraphics[width=6cm]{../Comparisons/ImagesFromVMD/4NFAACb.png}

Inertia Tensor - Molecule B \\
\begin{tabular}{|c c c|}
479.338	 & 	3.27331	 & 	-4.22553	 \\
3.27331	 & 	1242.4	 & 	0.852083	 \\
-4.22553	 & 	0.852083	 & 	1647.3
\end{tabular}

\vtab
 EingenVectors - Molecule B     \\
\begin{tabular}{|c c c|}
0.999984	 & 	-0.00429353	 & 	0.00362086	 \\
-0.004301	 & 	-0.999989	 & 	0.00205959	 \\
-0.00361198	 & 	0.00207513	 & 	0.999991
\end{tabular}

\vtab
 EingenValues - Molecule B     \\
\begin{tabular}{|c c c|}
479.308	 & 	1242.41	 & 	1647.31	 \\
\end{tabular}

\end{center}
\end{multicols}

\vtab[-5mm]
\begin{tabular}{*{2}{m{0.38\textwidth}}}
\begin{center}
\textcolor{NavyBlue}{\Large Different}
\end{center}
&
\begin{center}
\includegraphics[height=6.5cm]{../Comparisons/Vectors/inertia_tensor_of_3NFAACi_and_4NFAACb.png}
\end{center}
\end{tabular}

 \newpage

\vtab[-3cm]
\begin{center}
{\large FireTest \tab Número 365}
\end{center}
\begin{multicols}{2}
\begin{center}

Molecule A \
3NFAACi

\includegraphics[width=6cm]{../Comparisons/ImagesFromVMD/3NFAACi.png}

Inertia Tensor - Molecule A \\
\begin{tabular}{|c c c|}
521.487	 & 	-1.22943	 & 	-2.71042	 \\
-1.22943	 & 	949.663	 & 	1.67006	 \\
-2.71042	 & 	1.67006	 & 	1346.16
\end{tabular}

\vtab
 EingenVectors - Molecule A     \\
\begin{tabular}{|c c c|}
0.999991	 & 	0.00285841	 & 	0.00328077	 \\
0.00284452	 & 	-0.999987	 & 	0.00423133	 \\
-0.00329282	 & 	0.00422196	 & 	0.999986
\end{tabular}

\vtab
 EingenValues - Molecule A     \\
\begin{tabular}{|c c c|}
521.475	 & 	949.659	 & 	1346.18	 \\
\end{tabular}
\columnbreak

Molecule B \
4NFAACc

\includegraphics[width=6cm]{../Comparisons/ImagesFromVMD/4NFAACc.png}

Inertia Tensor - Molecule B \\
\begin{tabular}{|c c c|}
482.067	 & 	-5.39474	 & 	-1.35857	 \\
-5.39474	 & 	1240.3	 & 	-2.54035	 \\
-1.35857	 & 	-2.54035	 & 	1647.06
\end{tabular}

\vtab
 EingenVectors - Molecule B     \\
\begin{tabular}{|c c c|}
-0.999974	 & 	-0.00711826	 & 	-0.0011816	 \\
0.00712547	 & 	-0.999955	 & 	-0.00622156	 \\
-0.00113726	 & 	-0.00622982	 & 	0.99998
\end{tabular}

\vtab
 EingenValues - Molecule B     \\
\begin{tabular}{|c c c|}
482.027	 & 	1240.32	 & 	1647.08	 \\
\end{tabular}

\end{center}
\end{multicols}

\vtab[-5mm]
\begin{tabular}{*{2}{m{0.38\textwidth}}}
\begin{center}
\textcolor{NavyBlue}{\Large Different}
\end{center}
&
\begin{center}
\includegraphics[height=6.5cm]{../Comparisons/Vectors/inertia_tensor_of_3NFAACi_and_4NFAACc.png}
\end{center}
\end{tabular}

 \newpage

\vtab[-3cm]
\begin{center}
{\large FireTest \tab Número 366}
\end{center}
\begin{multicols}{2}
\begin{center}

Molecule A \
3NFAACi

\includegraphics[width=6cm]{../Comparisons/ImagesFromVMD/3NFAACi.png}

Inertia Tensor - Molecule A \\
\begin{tabular}{|c c c|}
521.487	 & 	-1.22943	 & 	-2.71042	 \\
-1.22943	 & 	949.663	 & 	1.67006	 \\
-2.71042	 & 	1.67006	 & 	1346.16
\end{tabular}

\vtab
 EingenVectors - Molecule A     \\
\begin{tabular}{|c c c|}
0.999991	 & 	0.00285841	 & 	0.00328077	 \\
0.00284452	 & 	-0.999987	 & 	0.00423133	 \\
-0.00329282	 & 	0.00422196	 & 	0.999986
\end{tabular}

\vtab
 EingenValues - Molecule A     \\
\begin{tabular}{|c c c|}
521.475	 & 	949.659	 & 	1346.18	 \\
\end{tabular}
\columnbreak

Molecule B \
4NFAACd

\includegraphics[width=6cm]{../Comparisons/ImagesFromVMD/4NFAACd.png}

Inertia Tensor - Molecule B \\
\begin{tabular}{|c c c|}
491.672	 & 	0.24486	 & 	-3.10016	 \\
0.24486	 & 	1231.15	 & 	2.19965	 \\
-3.10016	 & 	2.19965	 & 	1650.11
\end{tabular}

\vtab
 EingenVectors - Molecule B     \\
\begin{tabular}{|c c c|}
0.999996	 & 	-0.000339081	 & 	0.00267677	 \\
-0.000353124	 & 	-0.999986	 & 	0.00524747	 \\
-0.00267495	 & 	0.0052484	 & 	0.999983
\end{tabular}

\vtab
 EingenValues - Molecule B     \\
\begin{tabular}{|c c c|}
491.663	 & 	1231.14	 & 	1650.13	 \\
\end{tabular}

\end{center}
\end{multicols}

\vtab[-5mm]
\begin{tabular}{*{2}{m{0.38\textwidth}}}
\begin{center}
\textcolor{NavyBlue}{\Large Different}
\end{center}
&
\begin{center}
\includegraphics[height=6.5cm]{../Comparisons/Vectors/inertia_tensor_of_3NFAACi_and_4NFAACd.png}
\end{center}
\end{tabular}

 \newpage

\vtab[-3cm]
\begin{center}
{\large FireTest \tab Número 367}
\end{center}
\begin{multicols}{2}
\begin{center}

Molecule A \
3NFAACi

\includegraphics[width=6cm]{../Comparisons/ImagesFromVMD/3NFAACi.png}

Inertia Tensor - Molecule A \\
\begin{tabular}{|c c c|}
521.487	 & 	-1.22943	 & 	-2.71042	 \\
-1.22943	 & 	949.663	 & 	1.67006	 \\
-2.71042	 & 	1.67006	 & 	1346.16
\end{tabular}

\vtab
 EingenVectors - Molecule A     \\
\begin{tabular}{|c c c|}
0.999991	 & 	0.00285841	 & 	0.00328077	 \\
0.00284452	 & 	-0.999987	 & 	0.00423133	 \\
-0.00329282	 & 	0.00422196	 & 	0.999986
\end{tabular}

\vtab
 EingenValues - Molecule A     \\
\begin{tabular}{|c c c|}
521.475	 & 	949.659	 & 	1346.18	 \\
\end{tabular}
\columnbreak

Molecule B \
4NFAACe

\includegraphics[width=6cm]{../Comparisons/ImagesFromVMD/4NFAACe.png}

Inertia Tensor - Molecule B \\
\begin{tabular}{|c c c|}
489.025	 & 	-0.430035	 & 	3.98876	 \\
-0.430035	 & 	1233.71	 & 	-2.06505	 \\
3.98876	 & 	-2.06505	 & 	1641.79
\end{tabular}

\vtab
 EingenVectors - Molecule B     \\
\begin{tabular}{|c c c|}
0.999994	 & 	0.000567863	 & 	-0.00345908	 \\
0.000550336	 & 	-0.999987	 & 	-0.00506565	 \\
0.00346192	 & 	-0.00506372	 & 	0.999981
\end{tabular}

\vtab
 EingenValues - Molecule B     \\
\begin{tabular}{|c c c|}
489.011	 & 	1233.7	 & 	1641.81	 \\
\end{tabular}

\end{center}
\end{multicols}

\vtab[-5mm]
\begin{tabular}{*{2}{m{0.38\textwidth}}}
\begin{center}
\textcolor{NavyBlue}{\Large Different}
\end{center}
&
\begin{center}
\includegraphics[height=6.5cm]{../Comparisons/Vectors/inertia_tensor_of_3NFAACi_and_4NFAACe.png}
\end{center}
\end{tabular}

 \newpage

\vtab[-3cm]
\begin{center}
{\large FireTest \tab Número 368}
\end{center}
\begin{multicols}{2}
\begin{center}

Molecule A \
3NFAACi

\includegraphics[width=6cm]{../Comparisons/ImagesFromVMD/3NFAACi.png}

Inertia Tensor - Molecule A \\
\begin{tabular}{|c c c|}
521.487	 & 	-1.22943	 & 	-2.71042	 \\
-1.22943	 & 	949.663	 & 	1.67006	 \\
-2.71042	 & 	1.67006	 & 	1346.16
\end{tabular}

\vtab
 EingenVectors - Molecule A     \\
\begin{tabular}{|c c c|}
0.999991	 & 	0.00285841	 & 	0.00328077	 \\
0.00284452	 & 	-0.999987	 & 	0.00423133	 \\
-0.00329282	 & 	0.00422196	 & 	0.999986
\end{tabular}

\vtab
 EingenValues - Molecule A     \\
\begin{tabular}{|c c c|}
521.475	 & 	949.659	 & 	1346.18	 \\
\end{tabular}
\columnbreak

Molecule B \
4NFAACf

\includegraphics[width=6cm]{../Comparisons/ImagesFromVMD/4NFAACf.png}

Inertia Tensor - Molecule B \\
\begin{tabular}{|c c c|}
509.683	 & 	2.80651	 & 	-1.91422	 \\
2.80651	 & 	1219.11	 & 	2.66132	 \\
-1.91422	 & 	2.66132	 & 	1681.17
\end{tabular}

\vtab
 EingenVectors - Molecule B     \\
\begin{tabular}{|c c c|}
-0.999991	 & 	0.00396206	 & 	-0.00164298	 \\
-0.00397143	 & 	-0.999976	 & 	0.0057431	 \\
-0.00162019	 & 	0.00574957	 & 	0.999982
\end{tabular}

\vtab
 EingenValues - Molecule B     \\
\begin{tabular}{|c c c|}
509.668	 & 	1219.11	 & 	1681.18	 \\
\end{tabular}

\end{center}
\end{multicols}

\vtab[-5mm]
\begin{tabular}{*{2}{m{0.38\textwidth}}}
\begin{center}
\textcolor{NavyBlue}{\Large Different}
\end{center}
&
\begin{center}
\includegraphics[height=6.5cm]{../Comparisons/Vectors/inertia_tensor_of_3NFAACi_and_4NFAACf.png}
\end{center}
\end{tabular}

 \newpage

\vtab[-3cm]
\begin{center}
{\large FireTest \tab Número 369}
\end{center}
\begin{multicols}{2}
\begin{center}

Molecule A \
3NFAACi

\includegraphics[width=6cm]{../Comparisons/ImagesFromVMD/3NFAACi.png}

Inertia Tensor - Molecule A \\
\begin{tabular}{|c c c|}
521.487	 & 	-1.22943	 & 	-2.71042	 \\
-1.22943	 & 	949.663	 & 	1.67006	 \\
-2.71042	 & 	1.67006	 & 	1346.16
\end{tabular}

\vtab
 EingenVectors - Molecule A     \\
\begin{tabular}{|c c c|}
0.999991	 & 	0.00285841	 & 	0.00328077	 \\
0.00284452	 & 	-0.999987	 & 	0.00423133	 \\
-0.00329282	 & 	0.00422196	 & 	0.999986
\end{tabular}

\vtab
 EingenValues - Molecule A     \\
\begin{tabular}{|c c c|}
521.475	 & 	949.659	 & 	1346.18	 \\
\end{tabular}
\columnbreak

Molecule B \
4NFAACg

\includegraphics[width=6cm]{../Comparisons/ImagesFromVMD/4NFAACg.png}

Inertia Tensor - Molecule B \\
\begin{tabular}{|c c c|}
513.78	 & 	4.51917	 & 	0.266555	 \\
4.51917	 & 	1208.04	 & 	-1.18628	 \\
0.266555	 & 	-1.18628	 & 	1700.9
\end{tabular}

\vtab
 EingenVectors - Molecule B     \\
\begin{tabular}{|c c c|}
-0.999979	 & 	0.00650929	 & 	0.000231034	 \\
-0.00650983	 & 	-0.999976	 & 	-0.00240351	 \\
0.000215383	 & 	-0.00240496	 & 	0.999997
\end{tabular}

\vtab
 EingenValues - Molecule B     \\
\begin{tabular}{|c c c|}
513.751	 & 	1208.07	 & 	1700.9	 \\
\end{tabular}

\end{center}
\end{multicols}

\vtab[-5mm]
\begin{tabular}{*{2}{m{0.38\textwidth}}}
\begin{center}
\textcolor{NavyBlue}{\Large Different}
\end{center}
&
\begin{center}
\includegraphics[height=6.5cm]{../Comparisons/Vectors/inertia_tensor_of_3NFAACi_and_4NFAACg.png}
\end{center}
\end{tabular}

 \newpage

\vtab[-3cm]
\begin{center}
{\large FireTest \tab Número 370}
\end{center}
\begin{multicols}{2}
\begin{center}

Molecule A \
3NFAACi

\includegraphics[width=6cm]{../Comparisons/ImagesFromVMD/3NFAACi.png}

Inertia Tensor - Molecule A \\
\begin{tabular}{|c c c|}
521.487	 & 	-1.22943	 & 	-2.71042	 \\
-1.22943	 & 	949.663	 & 	1.67006	 \\
-2.71042	 & 	1.67006	 & 	1346.16
\end{tabular}

\vtab
 EingenVectors - Molecule A     \\
\begin{tabular}{|c c c|}
0.999991	 & 	0.00285841	 & 	0.00328077	 \\
0.00284452	 & 	-0.999987	 & 	0.00423133	 \\
-0.00329282	 & 	0.00422196	 & 	0.999986
\end{tabular}

\vtab
 EingenValues - Molecule A     \\
\begin{tabular}{|c c c|}
521.475	 & 	949.659	 & 	1346.18	 \\
\end{tabular}
\columnbreak

Molecule B \
4NFAACi

\includegraphics[width=6cm]{../Comparisons/ImagesFromVMD/4NFAACi.png}

Inertia Tensor - Molecule B \\
\begin{tabular}{|c c c|}
502.43	 & 	-0.602691	 & 	-4.86988	 \\
-0.602691	 & 	1232.26	 & 	0.407295	 \\
-4.86988	 & 	0.407295	 & 	1676
\end{tabular}

\vtab
 EingenVectors - Molecule B     \\
\begin{tabular}{|c c c|}
0.999991	 & 	0.000823447	 & 	0.00414923	 \\
0.000819608	 & 	-0.999999	 & 	0.00092687	 \\
-0.00414999	 & 	0.000923461	 & 	0.999991
\end{tabular}

\vtab
 EingenValues - Molecule B     \\
\begin{tabular}{|c c c|}
502.409	 & 	1232.26	 & 	1676.02	 \\
\end{tabular}

\end{center}
\end{multicols}

\vtab[-5mm]
\begin{tabular}{*{2}{m{0.38\textwidth}}}
\begin{center}
\textcolor{NavyBlue}{\Large Different}
\end{center}
&
\begin{center}
\includegraphics[height=6.5cm]{../Comparisons/Vectors/inertia_tensor_of_3NFAACi_and_4NFAACi.png}
\end{center}
\end{tabular}

 \newpage

\vtab[-3cm]
\begin{center}
{\large FireTest \tab Número 371}
\end{center}
\begin{multicols}{2}
\begin{center}

Molecule A \
3NFAACi

\includegraphics[width=6cm]{../Comparisons/ImagesFromVMD/3NFAACi.png}

Inertia Tensor - Molecule A \\
\begin{tabular}{|c c c|}
521.487	 & 	-1.22943	 & 	-2.71042	 \\
-1.22943	 & 	949.663	 & 	1.67006	 \\
-2.71042	 & 	1.67006	 & 	1346.16
\end{tabular}

\vtab
 EingenVectors - Molecule A     \\
\begin{tabular}{|c c c|}
0.999991	 & 	0.00285841	 & 	0.00328077	 \\
0.00284452	 & 	-0.999987	 & 	0.00423133	 \\
-0.00329282	 & 	0.00422196	 & 	0.999986
\end{tabular}

\vtab
 EingenValues - Molecule A     \\
\begin{tabular}{|c c c|}
521.475	 & 	949.659	 & 	1346.18	 \\
\end{tabular}
\columnbreak

Molecule B \
4NFAACj

\includegraphics[width=6cm]{../Comparisons/ImagesFromVMD/4NFAACj.png}

Inertia Tensor - Molecule B \\
\begin{tabular}{|c c c|}
510.047	 & 	9.97005	 & 	-3.6306	 \\
9.97005	 & 	1225.52	 & 	-0.981092	 \\
-3.6306	 & 	-0.981092	 & 	1680.82
\end{tabular}

\vtab
 EingenVectors - Molecule B     \\
\begin{tabular}{|c c c|}
-0.999898	 & 	0.0139264	 & 	-0.00308865	 \\
0.0139195	 & 	0.999901	 & 	0.00226627	 \\
-0.00311991	 & 	-0.00222305	 & 	0.999993
\end{tabular}

\vtab
 EingenValues - Molecule B     \\
\begin{tabular}{|c c c|}
509.897	 & 	1225.65	 & 	1680.83	 \\
\end{tabular}

\end{center}
\end{multicols}

\vtab[-5mm]
\begin{tabular}{*{2}{m{0.38\textwidth}}}
\begin{center}
\textcolor{NavyBlue}{\Large Different}
\end{center}
&
\begin{center}
\includegraphics[height=6.5cm]{../Comparisons/Vectors/inertia_tensor_of_3NFAACi_and_4NFAACj.png}
\end{center}
\end{tabular}

 \newpage

\vtab[-3cm]
\begin{center}
{\large FireTest \tab Número 372}
\end{center}
\begin{multicols}{2}
\begin{center}

Molecule A \
3NFAACi

\includegraphics[width=6cm]{../Comparisons/ImagesFromVMD/3NFAACi.png}

Inertia Tensor - Molecule A \\
\begin{tabular}{|c c c|}
521.487	 & 	-1.22943	 & 	-2.71042	 \\
-1.22943	 & 	949.663	 & 	1.67006	 \\
-2.71042	 & 	1.67006	 & 	1346.16
\end{tabular}

\vtab
 EingenVectors - Molecule A     \\
\begin{tabular}{|c c c|}
0.999991	 & 	0.00285841	 & 	0.00328077	 \\
0.00284452	 & 	-0.999987	 & 	0.00423133	 \\
-0.00329282	 & 	0.00422196	 & 	0.999986
\end{tabular}

\vtab
 EingenValues - Molecule A     \\
\begin{tabular}{|c c c|}
521.475	 & 	949.659	 & 	1346.18	 \\
\end{tabular}
\columnbreak

Molecule B \
4NFAACl-3

\includegraphics[width=6cm]{../Comparisons/ImagesFromVMD/4NFAACl-3.png}

Inertia Tensor - Molecule B \\
\begin{tabular}{|c c c|}
506.608	 & 	0.709539	 & 	-0.555426	 \\
0.709539	 & 	1222.37	 & 	-2.84005	 \\
-0.555426	 & 	-2.84005	 & 	1678.41
\end{tabular}

\vtab
 EingenVectors - Molecule B     \\
\begin{tabular}{|c c c|}
-0.999999	 & 	0.000989428	 & 	-0.000471595	 \\
-0.000986471	 & 	-0.99998	 & 	-0.00622856	 \\
-0.000477748	 & 	-0.00622809	 & 	0.99998
\end{tabular}

\vtab
 EingenValues - Molecule B     \\
\begin{tabular}{|c c c|}
506.607	 & 	1222.36	 & 	1678.43	 \\
\end{tabular}

\end{center}
\end{multicols}

\vtab[-5mm]
\begin{tabular}{*{2}{m{0.38\textwidth}}}
\begin{center}
\textcolor{NavyBlue}{\Large Different}
\end{center}
&
\begin{center}
\includegraphics[height=6.5cm]{../Comparisons/Vectors/inertia_tensor_of_3NFAACi_and_4NFAACl-3.png}
\end{center}
\end{tabular}

 \newpage

\vtab[-3cm]
\begin{center}
{\large FireTest \tab Número 373}
\end{center}
\begin{multicols}{2}
\begin{center}

Molecule A \
3NFAACj

\includegraphics[width=6cm]{../Comparisons/ImagesFromVMD/3NFAACj.png}

Inertia Tensor - Molecule A \\
\begin{tabular}{|c c c|}
533.789	 & 	-4.75521	 & 	-1.91525	 \\
-4.75521	 & 	920.091	 & 	2.28449	 \\
-1.91525	 & 	2.28449	 & 	1348.28
\end{tabular}

\vtab
 EingenVectors - Molecule A     \\
\begin{tabular}{|c c c|}
-0.999922	 & 	-0.0122929	 & 	-0.00231663	 \\
0.0122803	 & 	-0.99991	 & 	0.00539026	 \\
-0.00238268	 & 	0.00536139	 & 	0.999983
\end{tabular}

\vtab
 EingenValues - Molecule A     \\
\begin{tabular}{|c c c|}
533.726	 & 	920.137	 & 	1348.3	 \\
\end{tabular}
\columnbreak

Molecule B \
3NFAACk

\includegraphics[width=6cm]{../Comparisons/ImagesFromVMD/3NFAACk.png}

Inertia Tensor - Molecule B \\
\begin{tabular}{|c c c|}
534.899	 & 	-5.81418	 & 	-0.270063	 \\
-5.81418	 & 	913.263	 & 	2.4519	 \\
-0.270063	 & 	2.4519	 & 	1353.77
\end{tabular}

\vtab
 EingenVectors - Molecule B     \\
\begin{tabular}{|c c c|}
-0.999882	 & 	-0.0153593	 & 	-0.00028374	 \\
0.0153575	 & 	-0.999867	 & 	0.00557573	 \\
-0.000369342	 & 	0.00557072	 & 	0.999984
\end{tabular}

\vtab
 EingenValues - Molecule B     \\
\begin{tabular}{|c c c|}
534.81	 & 	913.339	 & 	1353.78	 \\
\end{tabular}

\end{center}
\end{multicols}

\vtab[-5mm]
\begin{tabular}{*{2}{m{0.38\textwidth}}}
\begin{center}
\textcolor{NavyBlue}{\Large Different}
\end{center}
&
\begin{center}
\includegraphics[height=6.5cm]{../Comparisons/Vectors/inertia_tensor_of_3NFAACj_and_3NFAACk.png}
\end{center}
\end{tabular}

 \newpage

\vtab[-3cm]
\begin{center}
{\large FireTest \tab Número 374}
\end{center}
\begin{multicols}{2}
\begin{center}

Molecule A \
3NFAACj

\includegraphics[width=6cm]{../Comparisons/ImagesFromVMD/3NFAACj.png}

Inertia Tensor - Molecule A \\
\begin{tabular}{|c c c|}
533.789	 & 	-4.75521	 & 	-1.91525	 \\
-4.75521	 & 	920.091	 & 	2.28449	 \\
-1.91525	 & 	2.28449	 & 	1348.28
\end{tabular}

\vtab
 EingenVectors - Molecule A     \\
\begin{tabular}{|c c c|}
-0.999922	 & 	-0.0122929	 & 	-0.00231663	 \\
0.0122803	 & 	-0.99991	 & 	0.00539026	 \\
-0.00238268	 & 	0.00536139	 & 	0.999983
\end{tabular}

\vtab
 EingenValues - Molecule A     \\
\begin{tabular}{|c c c|}
533.726	 & 	920.137	 & 	1348.3	 \\
\end{tabular}
\columnbreak

Molecule B \
3NFAACl

\includegraphics[width=6cm]{../Comparisons/ImagesFromVMD/3NFAACl.png}

Inertia Tensor - Molecule B \\
\begin{tabular}{|c c c|}
531.723	 & 	3.03424	 & 	2.73426	 \\
3.03424	 & 	929.418	 & 	-1.84284	 \\
2.73426	 & 	-1.84284	 & 	1355.39
\end{tabular}

\vtab
 EingenVectors - Molecule B     \\
\begin{tabular}{|c c c|}
0.999965	 & 	-0.00764413	 & 	-0.00333648	 \\
-0.00765838	 & 	-0.999962	 & 	-0.00427708	 \\
0.00330366	 & 	-0.00430248	 & 	0.999985
\end{tabular}

\vtab
 EingenValues - Molecule B     \\
\begin{tabular}{|c c c|}
531.691	 & 	929.434	 & 	1355.4	 \\
\end{tabular}

\end{center}
\end{multicols}

\vtab[-5mm]
\begin{tabular}{*{2}{m{0.38\textwidth}}}
\begin{center}
\textcolor{NavyBlue}{\Large Different}
\end{center}
&
\begin{center}
\includegraphics[height=6.5cm]{../Comparisons/Vectors/inertia_tensor_of_3NFAACj_and_3NFAACl.png}
\end{center}
\end{tabular}

 \newpage

\vtab[-3cm]
\begin{center}
{\large FireTest \tab Número 375}
\end{center}
\begin{multicols}{2}
\begin{center}

Molecule A \
3NFAACj

\includegraphics[width=6cm]{../Comparisons/ImagesFromVMD/3NFAACj.png}

Inertia Tensor - Molecule A \\
\begin{tabular}{|c c c|}
533.789	 & 	-4.75521	 & 	-1.91525	 \\
-4.75521	 & 	920.091	 & 	2.28449	 \\
-1.91525	 & 	2.28449	 & 	1348.28
\end{tabular}

\vtab
 EingenVectors - Molecule A     \\
\begin{tabular}{|c c c|}
-0.999922	 & 	-0.0122929	 & 	-0.00231663	 \\
0.0122803	 & 	-0.99991	 & 	0.00539026	 \\
-0.00238268	 & 	0.00536139	 & 	0.999983
\end{tabular}

\vtab
 EingenValues - Molecule A     \\
\begin{tabular}{|c c c|}
533.726	 & 	920.137	 & 	1348.3	 \\
\end{tabular}
\columnbreak

Molecule B \
3NFAACm

\includegraphics[width=6cm]{../Comparisons/ImagesFromVMD/3NFAACm.png}

Inertia Tensor - Molecule B \\
\begin{tabular}{|c c c|}
532.546	 & 	13.7854	 & 	-15.4626	 \\
13.7854	 & 	1354.87	 & 	11.5786	 \\
-15.4626	 & 	11.5786	 & 	929.101
\end{tabular}

\vtab
 EingenVectors - Molecule B     \\
\begin{tabular}{|c c c|}
-0.999075	 & 	0.0172851	 & 	-0.0393769	 \\
0.0398168	 & 	0.0258942	 & 	-0.998871	 \\
-0.016246	 & 	-0.999515	 & 	-0.0265584
\end{tabular}

\vtab
 EingenValues - Molecule B     \\
\begin{tabular}{|c c c|}
531.698	 & 	929.417	 & 	1355.4	 \\
\end{tabular}

\end{center}
\end{multicols}

\vtab[-5mm]
\begin{tabular}{*{2}{m{0.38\textwidth}}}
\begin{center}
\textcolor{NavyBlue}{\Large Different}
\end{center}
&
\begin{center}
\includegraphics[height=6.5cm]{../Comparisons/Vectors/inertia_tensor_of_3NFAACj_and_3NFAACm.png}
\end{center}
\end{tabular}

 \newpage

\vtab[-3cm]
\begin{center}
{\large FireTest \tab Número 376}
\end{center}
\begin{multicols}{2}
\begin{center}

Molecule A \
3NFAACj

\includegraphics[width=6cm]{../Comparisons/ImagesFromVMD/3NFAACj.png}

Inertia Tensor - Molecule A \\
\begin{tabular}{|c c c|}
533.789	 & 	-4.75521	 & 	-1.91525	 \\
-4.75521	 & 	920.091	 & 	2.28449	 \\
-1.91525	 & 	2.28449	 & 	1348.28
\end{tabular}

\vtab
 EingenVectors - Molecule A     \\
\begin{tabular}{|c c c|}
-0.999922	 & 	-0.0122929	 & 	-0.00231663	 \\
0.0122803	 & 	-0.99991	 & 	0.00539026	 \\
-0.00238268	 & 	0.00536139	 & 	0.999983
\end{tabular}

\vtab
 EingenValues - Molecule A     \\
\begin{tabular}{|c c c|}
533.726	 & 	920.137	 & 	1348.3	 \\
\end{tabular}
\columnbreak

Molecule B \
3NFAACn

\includegraphics[width=6cm]{../Comparisons/ImagesFromVMD/3NFAACn.png}

Inertia Tensor - Molecule B \\
\begin{tabular}{|c c c|}
531.896	 & 	3.78027	 & 	-13.1151	 \\
3.78027	 & 	1353.2	 & 	-7.47403	 \\
-13.1151	 & 	-7.47403	 & 	912.989
\end{tabular}

\vtab
 EingenVectors - Molecule B     \\
\begin{tabular}{|c c c|}
0.999403	 & 	-0.00428573	 & 	0.0342679	 \\
0.0341891	 & 	-0.0172718	 & 	-0.999266	 \\
0.00487445	 & 	0.999842	 & 	-0.017115
\end{tabular}

\vtab
 EingenValues - Molecule B     \\
\begin{tabular}{|c c c|}
531.43	 & 	913.309	 & 	1353.35	 \\
\end{tabular}

\end{center}
\end{multicols}

\vtab[-5mm]
\begin{tabular}{*{2}{m{0.38\textwidth}}}
\begin{center}
\textcolor{NavyBlue}{\Large Different}
\end{center}
&
\begin{center}
\includegraphics[height=6.5cm]{../Comparisons/Vectors/inertia_tensor_of_3NFAACj_and_3NFAACn.png}
\end{center}
\end{tabular}

 \newpage

\vtab[-3cm]
\begin{center}
{\large FireTest \tab Número 377}
\end{center}
\begin{multicols}{2}
\begin{center}

Molecule A \
3NFAACj

\includegraphics[width=6cm]{../Comparisons/ImagesFromVMD/3NFAACj.png}

Inertia Tensor - Molecule A \\
\begin{tabular}{|c c c|}
533.789	 & 	-4.75521	 & 	-1.91525	 \\
-4.75521	 & 	920.091	 & 	2.28449	 \\
-1.91525	 & 	2.28449	 & 	1348.28
\end{tabular}

\vtab
 EingenVectors - Molecule A     \\
\begin{tabular}{|c c c|}
-0.999922	 & 	-0.0122929	 & 	-0.00231663	 \\
0.0122803	 & 	-0.99991	 & 	0.00539026	 \\
-0.00238268	 & 	0.00536139	 & 	0.999983
\end{tabular}

\vtab
 EingenValues - Molecule A     \\
\begin{tabular}{|c c c|}
533.726	 & 	920.137	 & 	1348.3	 \\
\end{tabular}
\columnbreak

Molecule B \
4NFAACa

\includegraphics[width=6cm]{../Comparisons/ImagesFromVMD/4NFAACa.png}

Inertia Tensor - Molecule B \\
\begin{tabular}{|c c c|}
479.392	 & 	3.27131	 & 	4.22557	 \\
3.27131	 & 	1242.39	 & 	-0.852684	 \\
4.22557	 & 	-0.852684	 & 	1647.37
\end{tabular}

\vtab
 EingenVectors - Molecule B     \\
\begin{tabular}{|c c c|}
0.999984	 & 	-0.00429123	 & 	-0.00362083	 \\
-0.00429871	 & 	-0.999989	 & 	-0.0020607	 \\
0.00361195	 & 	-0.00207623	 & 	0.999991
\end{tabular}

\vtab
 EingenValues - Molecule B     \\
\begin{tabular}{|c c c|}
479.363	 & 	1242.41	 & 	1647.39	 \\
\end{tabular}

\end{center}
\end{multicols}

\vtab[-5mm]
\begin{tabular}{*{2}{m{0.38\textwidth}}}
\begin{center}
\textcolor{NavyBlue}{\Large Different}
\end{center}
&
\begin{center}
\includegraphics[height=6.5cm]{../Comparisons/Vectors/inertia_tensor_of_3NFAACj_and_4NFAACa.png}
\end{center}
\end{tabular}

 \newpage

\vtab[-3cm]
\begin{center}
{\large FireTest \tab Número 378}
\end{center}
\begin{multicols}{2}
\begin{center}

Molecule A \
3NFAACj

\includegraphics[width=6cm]{../Comparisons/ImagesFromVMD/3NFAACj.png}

Inertia Tensor - Molecule A \\
\begin{tabular}{|c c c|}
533.789	 & 	-4.75521	 & 	-1.91525	 \\
-4.75521	 & 	920.091	 & 	2.28449	 \\
-1.91525	 & 	2.28449	 & 	1348.28
\end{tabular}

\vtab
 EingenVectors - Molecule A     \\
\begin{tabular}{|c c c|}
-0.999922	 & 	-0.0122929	 & 	-0.00231663	 \\
0.0122803	 & 	-0.99991	 & 	0.00539026	 \\
-0.00238268	 & 	0.00536139	 & 	0.999983
\end{tabular}

\vtab
 EingenValues - Molecule A     \\
\begin{tabular}{|c c c|}
533.726	 & 	920.137	 & 	1348.3	 \\
\end{tabular}
\columnbreak

Molecule B \
4NFAACb

\includegraphics[width=6cm]{../Comparisons/ImagesFromVMD/4NFAACb.png}

Inertia Tensor - Molecule B \\
\begin{tabular}{|c c c|}
479.338	 & 	3.27331	 & 	-4.22553	 \\
3.27331	 & 	1242.4	 & 	0.852083	 \\
-4.22553	 & 	0.852083	 & 	1647.3
\end{tabular}

\vtab
 EingenVectors - Molecule B     \\
\begin{tabular}{|c c c|}
0.999984	 & 	-0.00429353	 & 	0.00362086	 \\
-0.004301	 & 	-0.999989	 & 	0.00205959	 \\
-0.00361198	 & 	0.00207513	 & 	0.999991
\end{tabular}

\vtab
 EingenValues - Molecule B     \\
\begin{tabular}{|c c c|}
479.308	 & 	1242.41	 & 	1647.31	 \\
\end{tabular}

\end{center}
\end{multicols}

\vtab[-5mm]
\begin{tabular}{*{2}{m{0.38\textwidth}}}
\begin{center}
\textcolor{NavyBlue}{\Large Different}
\end{center}
&
\begin{center}
\includegraphics[height=6.5cm]{../Comparisons/Vectors/inertia_tensor_of_3NFAACj_and_4NFAACb.png}
\end{center}
\end{tabular}

 \newpage

\vtab[-3cm]
\begin{center}
{\large FireTest \tab Número 379}
\end{center}
\begin{multicols}{2}
\begin{center}

Molecule A \
3NFAACj

\includegraphics[width=6cm]{../Comparisons/ImagesFromVMD/3NFAACj.png}

Inertia Tensor - Molecule A \\
\begin{tabular}{|c c c|}
533.789	 & 	-4.75521	 & 	-1.91525	 \\
-4.75521	 & 	920.091	 & 	2.28449	 \\
-1.91525	 & 	2.28449	 & 	1348.28
\end{tabular}

\vtab
 EingenVectors - Molecule A     \\
\begin{tabular}{|c c c|}
-0.999922	 & 	-0.0122929	 & 	-0.00231663	 \\
0.0122803	 & 	-0.99991	 & 	0.00539026	 \\
-0.00238268	 & 	0.00536139	 & 	0.999983
\end{tabular}

\vtab
 EingenValues - Molecule A     \\
\begin{tabular}{|c c c|}
533.726	 & 	920.137	 & 	1348.3	 \\
\end{tabular}
\columnbreak

Molecule B \
4NFAACc

\includegraphics[width=6cm]{../Comparisons/ImagesFromVMD/4NFAACc.png}

Inertia Tensor - Molecule B \\
\begin{tabular}{|c c c|}
482.067	 & 	-5.39474	 & 	-1.35857	 \\
-5.39474	 & 	1240.3	 & 	-2.54035	 \\
-1.35857	 & 	-2.54035	 & 	1647.06
\end{tabular}

\vtab
 EingenVectors - Molecule B     \\
\begin{tabular}{|c c c|}
-0.999974	 & 	-0.00711826	 & 	-0.0011816	 \\
0.00712547	 & 	-0.999955	 & 	-0.00622156	 \\
-0.00113726	 & 	-0.00622982	 & 	0.99998
\end{tabular}

\vtab
 EingenValues - Molecule B     \\
\begin{tabular}{|c c c|}
482.027	 & 	1240.32	 & 	1647.08	 \\
\end{tabular}

\end{center}
\end{multicols}

\vtab[-5mm]
\begin{tabular}{*{2}{m{0.38\textwidth}}}
\begin{center}
\textcolor{NavyBlue}{\Large Different}
\end{center}
&
\begin{center}
\includegraphics[height=6.5cm]{../Comparisons/Vectors/inertia_tensor_of_3NFAACj_and_4NFAACc.png}
\end{center}
\end{tabular}

 \newpage

\vtab[-3cm]
\begin{center}
{\large FireTest \tab Número 380}
\end{center}
\begin{multicols}{2}
\begin{center}

Molecule A \
3NFAACj

\includegraphics[width=6cm]{../Comparisons/ImagesFromVMD/3NFAACj.png}

Inertia Tensor - Molecule A \\
\begin{tabular}{|c c c|}
533.789	 & 	-4.75521	 & 	-1.91525	 \\
-4.75521	 & 	920.091	 & 	2.28449	 \\
-1.91525	 & 	2.28449	 & 	1348.28
\end{tabular}

\vtab
 EingenVectors - Molecule A     \\
\begin{tabular}{|c c c|}
-0.999922	 & 	-0.0122929	 & 	-0.00231663	 \\
0.0122803	 & 	-0.99991	 & 	0.00539026	 \\
-0.00238268	 & 	0.00536139	 & 	0.999983
\end{tabular}

\vtab
 EingenValues - Molecule A     \\
\begin{tabular}{|c c c|}
533.726	 & 	920.137	 & 	1348.3	 \\
\end{tabular}
\columnbreak

Molecule B \
4NFAACd

\includegraphics[width=6cm]{../Comparisons/ImagesFromVMD/4NFAACd.png}

Inertia Tensor - Molecule B \\
\begin{tabular}{|c c c|}
491.672	 & 	0.24486	 & 	-3.10016	 \\
0.24486	 & 	1231.15	 & 	2.19965	 \\
-3.10016	 & 	2.19965	 & 	1650.11
\end{tabular}

\vtab
 EingenVectors - Molecule B     \\
\begin{tabular}{|c c c|}
0.999996	 & 	-0.000339081	 & 	0.00267677	 \\
-0.000353124	 & 	-0.999986	 & 	0.00524747	 \\
-0.00267495	 & 	0.0052484	 & 	0.999983
\end{tabular}

\vtab
 EingenValues - Molecule B     \\
\begin{tabular}{|c c c|}
491.663	 & 	1231.14	 & 	1650.13	 \\
\end{tabular}

\end{center}
\end{multicols}

\vtab[-5mm]
\begin{tabular}{*{2}{m{0.38\textwidth}}}
\begin{center}
\textcolor{NavyBlue}{\Large Different}
\end{center}
&
\begin{center}
\includegraphics[height=6.5cm]{../Comparisons/Vectors/inertia_tensor_of_3NFAACj_and_4NFAACd.png}
\end{center}
\end{tabular}

 \newpage

\vtab[-3cm]
\begin{center}
{\large FireTest \tab Número 381}
\end{center}
\begin{multicols}{2}
\begin{center}

Molecule A \
3NFAACj

\includegraphics[width=6cm]{../Comparisons/ImagesFromVMD/3NFAACj.png}

Inertia Tensor - Molecule A \\
\begin{tabular}{|c c c|}
533.789	 & 	-4.75521	 & 	-1.91525	 \\
-4.75521	 & 	920.091	 & 	2.28449	 \\
-1.91525	 & 	2.28449	 & 	1348.28
\end{tabular}

\vtab
 EingenVectors - Molecule A     \\
\begin{tabular}{|c c c|}
-0.999922	 & 	-0.0122929	 & 	-0.00231663	 \\
0.0122803	 & 	-0.99991	 & 	0.00539026	 \\
-0.00238268	 & 	0.00536139	 & 	0.999983
\end{tabular}

\vtab
 EingenValues - Molecule A     \\
\begin{tabular}{|c c c|}
533.726	 & 	920.137	 & 	1348.3	 \\
\end{tabular}
\columnbreak

Molecule B \
4NFAACe

\includegraphics[width=6cm]{../Comparisons/ImagesFromVMD/4NFAACe.png}

Inertia Tensor - Molecule B \\
\begin{tabular}{|c c c|}
489.025	 & 	-0.430035	 & 	3.98876	 \\
-0.430035	 & 	1233.71	 & 	-2.06505	 \\
3.98876	 & 	-2.06505	 & 	1641.79
\end{tabular}

\vtab
 EingenVectors - Molecule B     \\
\begin{tabular}{|c c c|}
0.999994	 & 	0.000567863	 & 	-0.00345908	 \\
0.000550336	 & 	-0.999987	 & 	-0.00506565	 \\
0.00346192	 & 	-0.00506372	 & 	0.999981
\end{tabular}

\vtab
 EingenValues - Molecule B     \\
\begin{tabular}{|c c c|}
489.011	 & 	1233.7	 & 	1641.81	 \\
\end{tabular}

\end{center}
\end{multicols}

\vtab[-5mm]
\begin{tabular}{*{2}{m{0.38\textwidth}}}
\begin{center}
\textcolor{NavyBlue}{\Large Different}
\end{center}
&
\begin{center}
\includegraphics[height=6.5cm]{../Comparisons/Vectors/inertia_tensor_of_3NFAACj_and_4NFAACe.png}
\end{center}
\end{tabular}

 \newpage

\vtab[-3cm]
\begin{center}
{\large FireTest \tab Número 382}
\end{center}
\begin{multicols}{2}
\begin{center}

Molecule A \
3NFAACj

\includegraphics[width=6cm]{../Comparisons/ImagesFromVMD/3NFAACj.png}

Inertia Tensor - Molecule A \\
\begin{tabular}{|c c c|}
533.789	 & 	-4.75521	 & 	-1.91525	 \\
-4.75521	 & 	920.091	 & 	2.28449	 \\
-1.91525	 & 	2.28449	 & 	1348.28
\end{tabular}

\vtab
 EingenVectors - Molecule A     \\
\begin{tabular}{|c c c|}
-0.999922	 & 	-0.0122929	 & 	-0.00231663	 \\
0.0122803	 & 	-0.99991	 & 	0.00539026	 \\
-0.00238268	 & 	0.00536139	 & 	0.999983
\end{tabular}

\vtab
 EingenValues - Molecule A     \\
\begin{tabular}{|c c c|}
533.726	 & 	920.137	 & 	1348.3	 \\
\end{tabular}
\columnbreak

Molecule B \
4NFAACf

\includegraphics[width=6cm]{../Comparisons/ImagesFromVMD/4NFAACf.png}

Inertia Tensor - Molecule B \\
\begin{tabular}{|c c c|}
509.683	 & 	2.80651	 & 	-1.91422	 \\
2.80651	 & 	1219.11	 & 	2.66132	 \\
-1.91422	 & 	2.66132	 & 	1681.17
\end{tabular}

\vtab
 EingenVectors - Molecule B     \\
\begin{tabular}{|c c c|}
-0.999991	 & 	0.00396206	 & 	-0.00164298	 \\
-0.00397143	 & 	-0.999976	 & 	0.0057431	 \\
-0.00162019	 & 	0.00574957	 & 	0.999982
\end{tabular}

\vtab
 EingenValues - Molecule B     \\
\begin{tabular}{|c c c|}
509.668	 & 	1219.11	 & 	1681.18	 \\
\end{tabular}

\end{center}
\end{multicols}

\vtab[-5mm]
\begin{tabular}{*{2}{m{0.38\textwidth}}}
\begin{center}
\textcolor{NavyBlue}{\Large Different}
\end{center}
&
\begin{center}
\includegraphics[height=6.5cm]{../Comparisons/Vectors/inertia_tensor_of_3NFAACj_and_4NFAACf.png}
\end{center}
\end{tabular}

 \newpage

\vtab[-3cm]
\begin{center}
{\large FireTest \tab Número 383}
\end{center}
\begin{multicols}{2}
\begin{center}

Molecule A \
3NFAACj

\includegraphics[width=6cm]{../Comparisons/ImagesFromVMD/3NFAACj.png}

Inertia Tensor - Molecule A \\
\begin{tabular}{|c c c|}
533.789	 & 	-4.75521	 & 	-1.91525	 \\
-4.75521	 & 	920.091	 & 	2.28449	 \\
-1.91525	 & 	2.28449	 & 	1348.28
\end{tabular}

\vtab
 EingenVectors - Molecule A     \\
\begin{tabular}{|c c c|}
-0.999922	 & 	-0.0122929	 & 	-0.00231663	 \\
0.0122803	 & 	-0.99991	 & 	0.00539026	 \\
-0.00238268	 & 	0.00536139	 & 	0.999983
\end{tabular}

\vtab
 EingenValues - Molecule A     \\
\begin{tabular}{|c c c|}
533.726	 & 	920.137	 & 	1348.3	 \\
\end{tabular}
\columnbreak

Molecule B \
4NFAACg

\includegraphics[width=6cm]{../Comparisons/ImagesFromVMD/4NFAACg.png}

Inertia Tensor - Molecule B \\
\begin{tabular}{|c c c|}
513.78	 & 	4.51917	 & 	0.266555	 \\
4.51917	 & 	1208.04	 & 	-1.18628	 \\
0.266555	 & 	-1.18628	 & 	1700.9
\end{tabular}

\vtab
 EingenVectors - Molecule B     \\
\begin{tabular}{|c c c|}
-0.999979	 & 	0.00650929	 & 	0.000231034	 \\
-0.00650983	 & 	-0.999976	 & 	-0.00240351	 \\
0.000215383	 & 	-0.00240496	 & 	0.999997
\end{tabular}

\vtab
 EingenValues - Molecule B     \\
\begin{tabular}{|c c c|}
513.751	 & 	1208.07	 & 	1700.9	 \\
\end{tabular}

\end{center}
\end{multicols}

\vtab[-5mm]
\begin{tabular}{*{2}{m{0.38\textwidth}}}
\begin{center}
\textcolor{NavyBlue}{\Large Different}
\end{center}
&
\begin{center}
\includegraphics[height=6.5cm]{../Comparisons/Vectors/inertia_tensor_of_3NFAACj_and_4NFAACg.png}
\end{center}
\end{tabular}

 \newpage

\vtab[-3cm]
\begin{center}
{\large FireTest \tab Número 384}
\end{center}
\begin{multicols}{2}
\begin{center}

Molecule A \
3NFAACj

\includegraphics[width=6cm]{../Comparisons/ImagesFromVMD/3NFAACj.png}

Inertia Tensor - Molecule A \\
\begin{tabular}{|c c c|}
533.789	 & 	-4.75521	 & 	-1.91525	 \\
-4.75521	 & 	920.091	 & 	2.28449	 \\
-1.91525	 & 	2.28449	 & 	1348.28
\end{tabular}

\vtab
 EingenVectors - Molecule A     \\
\begin{tabular}{|c c c|}
-0.999922	 & 	-0.0122929	 & 	-0.00231663	 \\
0.0122803	 & 	-0.99991	 & 	0.00539026	 \\
-0.00238268	 & 	0.00536139	 & 	0.999983
\end{tabular}

\vtab
 EingenValues - Molecule A     \\
\begin{tabular}{|c c c|}
533.726	 & 	920.137	 & 	1348.3	 \\
\end{tabular}
\columnbreak

Molecule B \
4NFAACi

\includegraphics[width=6cm]{../Comparisons/ImagesFromVMD/4NFAACi.png}

Inertia Tensor - Molecule B \\
\begin{tabular}{|c c c|}
502.43	 & 	-0.602691	 & 	-4.86988	 \\
-0.602691	 & 	1232.26	 & 	0.407295	 \\
-4.86988	 & 	0.407295	 & 	1676
\end{tabular}

\vtab
 EingenVectors - Molecule B     \\
\begin{tabular}{|c c c|}
0.999991	 & 	0.000823447	 & 	0.00414923	 \\
0.000819608	 & 	-0.999999	 & 	0.00092687	 \\
-0.00414999	 & 	0.000923461	 & 	0.999991
\end{tabular}

\vtab
 EingenValues - Molecule B     \\
\begin{tabular}{|c c c|}
502.409	 & 	1232.26	 & 	1676.02	 \\
\end{tabular}

\end{center}
\end{multicols}

\vtab[-5mm]
\begin{tabular}{*{2}{m{0.38\textwidth}}}
\begin{center}
\textcolor{NavyBlue}{\Large Different}
\end{center}
&
\begin{center}
\includegraphics[height=6.5cm]{../Comparisons/Vectors/inertia_tensor_of_3NFAACj_and_4NFAACi.png}
\end{center}
\end{tabular}

 \newpage

\vtab[-3cm]
\begin{center}
{\large FireTest \tab Número 385}
\end{center}
\begin{multicols}{2}
\begin{center}

Molecule A \
3NFAACj

\includegraphics[width=6cm]{../Comparisons/ImagesFromVMD/3NFAACj.png}

Inertia Tensor - Molecule A \\
\begin{tabular}{|c c c|}
533.789	 & 	-4.75521	 & 	-1.91525	 \\
-4.75521	 & 	920.091	 & 	2.28449	 \\
-1.91525	 & 	2.28449	 & 	1348.28
\end{tabular}

\vtab
 EingenVectors - Molecule A     \\
\begin{tabular}{|c c c|}
-0.999922	 & 	-0.0122929	 & 	-0.00231663	 \\
0.0122803	 & 	-0.99991	 & 	0.00539026	 \\
-0.00238268	 & 	0.00536139	 & 	0.999983
\end{tabular}

\vtab
 EingenValues - Molecule A     \\
\begin{tabular}{|c c c|}
533.726	 & 	920.137	 & 	1348.3	 \\
\end{tabular}
\columnbreak

Molecule B \
4NFAACj

\includegraphics[width=6cm]{../Comparisons/ImagesFromVMD/4NFAACj.png}

Inertia Tensor - Molecule B \\
\begin{tabular}{|c c c|}
510.047	 & 	9.97005	 & 	-3.6306	 \\
9.97005	 & 	1225.52	 & 	-0.981092	 \\
-3.6306	 & 	-0.981092	 & 	1680.82
\end{tabular}

\vtab
 EingenVectors - Molecule B     \\
\begin{tabular}{|c c c|}
-0.999898	 & 	0.0139264	 & 	-0.00308865	 \\
0.0139195	 & 	0.999901	 & 	0.00226627	 \\
-0.00311991	 & 	-0.00222305	 & 	0.999993
\end{tabular}

\vtab
 EingenValues - Molecule B     \\
\begin{tabular}{|c c c|}
509.897	 & 	1225.65	 & 	1680.83	 \\
\end{tabular}

\end{center}
\end{multicols}

\vtab[-5mm]
\begin{tabular}{*{2}{m{0.38\textwidth}}}
\begin{center}
\textcolor{NavyBlue}{\Large Different}
\end{center}
&
\begin{center}
\includegraphics[height=6.5cm]{../Comparisons/Vectors/inertia_tensor_of_3NFAACj_and_4NFAACj.png}
\end{center}
\end{tabular}

 \newpage

\vtab[-3cm]
\begin{center}
{\large FireTest \tab Número 386}
\end{center}
\begin{multicols}{2}
\begin{center}

Molecule A \
3NFAACj

\includegraphics[width=6cm]{../Comparisons/ImagesFromVMD/3NFAACj.png}

Inertia Tensor - Molecule A \\
\begin{tabular}{|c c c|}
533.789	 & 	-4.75521	 & 	-1.91525	 \\
-4.75521	 & 	920.091	 & 	2.28449	 \\
-1.91525	 & 	2.28449	 & 	1348.28
\end{tabular}

\vtab
 EingenVectors - Molecule A     \\
\begin{tabular}{|c c c|}
-0.999922	 & 	-0.0122929	 & 	-0.00231663	 \\
0.0122803	 & 	-0.99991	 & 	0.00539026	 \\
-0.00238268	 & 	0.00536139	 & 	0.999983
\end{tabular}

\vtab
 EingenValues - Molecule A     \\
\begin{tabular}{|c c c|}
533.726	 & 	920.137	 & 	1348.3	 \\
\end{tabular}
\columnbreak

Molecule B \
4NFAACl-3

\includegraphics[width=6cm]{../Comparisons/ImagesFromVMD/4NFAACl-3.png}

Inertia Tensor - Molecule B \\
\begin{tabular}{|c c c|}
506.608	 & 	0.709539	 & 	-0.555426	 \\
0.709539	 & 	1222.37	 & 	-2.84005	 \\
-0.555426	 & 	-2.84005	 & 	1678.41
\end{tabular}

\vtab
 EingenVectors - Molecule B     \\
\begin{tabular}{|c c c|}
-0.999999	 & 	0.000989428	 & 	-0.000471595	 \\
-0.000986471	 & 	-0.99998	 & 	-0.00622856	 \\
-0.000477748	 & 	-0.00622809	 & 	0.99998
\end{tabular}

\vtab
 EingenValues - Molecule B     \\
\begin{tabular}{|c c c|}
506.607	 & 	1222.36	 & 	1678.43	 \\
\end{tabular}

\end{center}
\end{multicols}

\vtab[-5mm]
\begin{tabular}{*{2}{m{0.38\textwidth}}}
\begin{center}
\textcolor{NavyBlue}{\Large Different}
\end{center}
&
\begin{center}
\includegraphics[height=6.5cm]{../Comparisons/Vectors/inertia_tensor_of_3NFAACj_and_4NFAACl-3.png}
\end{center}
\end{tabular}

 \newpage

\vtab[-3cm]
\begin{center}
{\large FireTest \tab Número 387}
\end{center}
\begin{multicols}{2}
\begin{center}

Molecule A \
3NFAACk

\includegraphics[width=6cm]{../Comparisons/ImagesFromVMD/3NFAACk.png}

Inertia Tensor - Molecule A \\
\begin{tabular}{|c c c|}
534.899	 & 	-5.81418	 & 	-0.270063	 \\
-5.81418	 & 	913.263	 & 	2.4519	 \\
-0.270063	 & 	2.4519	 & 	1353.77
\end{tabular}

\vtab
 EingenVectors - Molecule A     \\
\begin{tabular}{|c c c|}
-0.999882	 & 	-0.0153593	 & 	-0.00028374	 \\
0.0153575	 & 	-0.999867	 & 	0.00557573	 \\
-0.000369342	 & 	0.00557072	 & 	0.999984
\end{tabular}

\vtab
 EingenValues - Molecule A     \\
\begin{tabular}{|c c c|}
534.81	 & 	913.339	 & 	1353.78	 \\
\end{tabular}
\columnbreak

Molecule B \
3NFAACl

\includegraphics[width=6cm]{../Comparisons/ImagesFromVMD/3NFAACl.png}

Inertia Tensor - Molecule B \\
\begin{tabular}{|c c c|}
531.723	 & 	3.03424	 & 	2.73426	 \\
3.03424	 & 	929.418	 & 	-1.84284	 \\
2.73426	 & 	-1.84284	 & 	1355.39
\end{tabular}

\vtab
 EingenVectors - Molecule B     \\
\begin{tabular}{|c c c|}
0.999965	 & 	-0.00764413	 & 	-0.00333648	 \\
-0.00765838	 & 	-0.999962	 & 	-0.00427708	 \\
0.00330366	 & 	-0.00430248	 & 	0.999985
\end{tabular}

\vtab
 EingenValues - Molecule B     \\
\begin{tabular}{|c c c|}
531.691	 & 	929.434	 & 	1355.4	 \\
\end{tabular}

\end{center}
\end{multicols}

\vtab[-5mm]
\begin{tabular}{*{2}{m{0.38\textwidth}}}
\begin{center}
\textcolor{NavyBlue}{\Large Different}
\end{center}
&
\begin{center}
\includegraphics[height=6.5cm]{../Comparisons/Vectors/inertia_tensor_of_3NFAACk_and_3NFAACl.png}
\end{center}
\end{tabular}

 \newpage

\vtab[-3cm]
\begin{center}
{\large FireTest \tab Número 388}
\end{center}
\begin{multicols}{2}
\begin{center}

Molecule A \
3NFAACk

\includegraphics[width=6cm]{../Comparisons/ImagesFromVMD/3NFAACk.png}

Inertia Tensor - Molecule A \\
\begin{tabular}{|c c c|}
534.899	 & 	-5.81418	 & 	-0.270063	 \\
-5.81418	 & 	913.263	 & 	2.4519	 \\
-0.270063	 & 	2.4519	 & 	1353.77
\end{tabular}

\vtab
 EingenVectors - Molecule A     \\
\begin{tabular}{|c c c|}
-0.999882	 & 	-0.0153593	 & 	-0.00028374	 \\
0.0153575	 & 	-0.999867	 & 	0.00557573	 \\
-0.000369342	 & 	0.00557072	 & 	0.999984
\end{tabular}

\vtab
 EingenValues - Molecule A     \\
\begin{tabular}{|c c c|}
534.81	 & 	913.339	 & 	1353.78	 \\
\end{tabular}
\columnbreak

Molecule B \
3NFAACm

\includegraphics[width=6cm]{../Comparisons/ImagesFromVMD/3NFAACm.png}

Inertia Tensor - Molecule B \\
\begin{tabular}{|c c c|}
532.546	 & 	13.7854	 & 	-15.4626	 \\
13.7854	 & 	1354.87	 & 	11.5786	 \\
-15.4626	 & 	11.5786	 & 	929.101
\end{tabular}

\vtab
 EingenVectors - Molecule B     \\
\begin{tabular}{|c c c|}
-0.999075	 & 	0.0172851	 & 	-0.0393769	 \\
0.0398168	 & 	0.0258942	 & 	-0.998871	 \\
-0.016246	 & 	-0.999515	 & 	-0.0265584
\end{tabular}

\vtab
 EingenValues - Molecule B     \\
\begin{tabular}{|c c c|}
531.698	 & 	929.417	 & 	1355.4	 \\
\end{tabular}

\end{center}
\end{multicols}

\vtab[-5mm]
\begin{tabular}{*{2}{m{0.38\textwidth}}}
\begin{center}
\textcolor{NavyBlue}{\Large Different}
\end{center}
&
\begin{center}
\includegraphics[height=6.5cm]{../Comparisons/Vectors/inertia_tensor_of_3NFAACk_and_3NFAACm.png}
\end{center}
\end{tabular}

 \newpage

\vtab[-3cm]
\begin{center}
{\large FireTest \tab Número 389}
\end{center}
\begin{multicols}{2}
\begin{center}

Molecule A \
3NFAACk

\includegraphics[width=6cm]{../Comparisons/ImagesFromVMD/3NFAACk.png}

Inertia Tensor - Molecule A \\
\begin{tabular}{|c c c|}
534.899	 & 	-5.81418	 & 	-0.270063	 \\
-5.81418	 & 	913.263	 & 	2.4519	 \\
-0.270063	 & 	2.4519	 & 	1353.77
\end{tabular}

\vtab
 EingenVectors - Molecule A     \\
\begin{tabular}{|c c c|}
-0.999882	 & 	-0.0153593	 & 	-0.00028374	 \\
0.0153575	 & 	-0.999867	 & 	0.00557573	 \\
-0.000369342	 & 	0.00557072	 & 	0.999984
\end{tabular}

\vtab
 EingenValues - Molecule A     \\
\begin{tabular}{|c c c|}
534.81	 & 	913.339	 & 	1353.78	 \\
\end{tabular}
\columnbreak

Molecule B \
3NFAACn

\includegraphics[width=6cm]{../Comparisons/ImagesFromVMD/3NFAACn.png}

Inertia Tensor - Molecule B \\
\begin{tabular}{|c c c|}
531.896	 & 	3.78027	 & 	-13.1151	 \\
3.78027	 & 	1353.2	 & 	-7.47403	 \\
-13.1151	 & 	-7.47403	 & 	912.989
\end{tabular}

\vtab
 EingenVectors - Molecule B     \\
\begin{tabular}{|c c c|}
0.999403	 & 	-0.00428573	 & 	0.0342679	 \\
0.0341891	 & 	-0.0172718	 & 	-0.999266	 \\
0.00487445	 & 	0.999842	 & 	-0.017115
\end{tabular}

\vtab
 EingenValues - Molecule B     \\
\begin{tabular}{|c c c|}
531.43	 & 	913.309	 & 	1353.35	 \\
\end{tabular}

\end{center}
\end{multicols}

\vtab[-5mm]
\begin{tabular}{*{2}{m{0.38\textwidth}}}
\begin{center}
\textcolor{NavyBlue}{\Large Different}
\end{center}
&
\begin{center}
\includegraphics[height=6.5cm]{../Comparisons/Vectors/inertia_tensor_of_3NFAACk_and_3NFAACn.png}
\end{center}
\end{tabular}

 \newpage

\vtab[-3cm]
\begin{center}
{\large FireTest \tab Número 390}
\end{center}
\begin{multicols}{2}
\begin{center}

Molecule A \
3NFAACk

\includegraphics[width=6cm]{../Comparisons/ImagesFromVMD/3NFAACk.png}

Inertia Tensor - Molecule A \\
\begin{tabular}{|c c c|}
534.899	 & 	-5.81418	 & 	-0.270063	 \\
-5.81418	 & 	913.263	 & 	2.4519	 \\
-0.270063	 & 	2.4519	 & 	1353.77
\end{tabular}

\vtab
 EingenVectors - Molecule A     \\
\begin{tabular}{|c c c|}
-0.999882	 & 	-0.0153593	 & 	-0.00028374	 \\
0.0153575	 & 	-0.999867	 & 	0.00557573	 \\
-0.000369342	 & 	0.00557072	 & 	0.999984
\end{tabular}

\vtab
 EingenValues - Molecule A     \\
\begin{tabular}{|c c c|}
534.81	 & 	913.339	 & 	1353.78	 \\
\end{tabular}
\columnbreak

Molecule B \
4NFAACa

\includegraphics[width=6cm]{../Comparisons/ImagesFromVMD/4NFAACa.png}

Inertia Tensor - Molecule B \\
\begin{tabular}{|c c c|}
479.392	 & 	3.27131	 & 	4.22557	 \\
3.27131	 & 	1242.39	 & 	-0.852684	 \\
4.22557	 & 	-0.852684	 & 	1647.37
\end{tabular}

\vtab
 EingenVectors - Molecule B     \\
\begin{tabular}{|c c c|}
0.999984	 & 	-0.00429123	 & 	-0.00362083	 \\
-0.00429871	 & 	-0.999989	 & 	-0.0020607	 \\
0.00361195	 & 	-0.00207623	 & 	0.999991
\end{tabular}

\vtab
 EingenValues - Molecule B     \\
\begin{tabular}{|c c c|}
479.363	 & 	1242.41	 & 	1647.39	 \\
\end{tabular}

\end{center}
\end{multicols}

\vtab[-5mm]
\begin{tabular}{*{2}{m{0.38\textwidth}}}
\begin{center}
\textcolor{NavyBlue}{\Large Different}
\end{center}
&
\begin{center}
\includegraphics[height=6.5cm]{../Comparisons/Vectors/inertia_tensor_of_3NFAACk_and_4NFAACa.png}
\end{center}
\end{tabular}

 \newpage

\vtab[-3cm]
\begin{center}
{\large FireTest \tab Número 391}
\end{center}
\begin{multicols}{2}
\begin{center}

Molecule A \
3NFAACk

\includegraphics[width=6cm]{../Comparisons/ImagesFromVMD/3NFAACk.png}

Inertia Tensor - Molecule A \\
\begin{tabular}{|c c c|}
534.899	 & 	-5.81418	 & 	-0.270063	 \\
-5.81418	 & 	913.263	 & 	2.4519	 \\
-0.270063	 & 	2.4519	 & 	1353.77
\end{tabular}

\vtab
 EingenVectors - Molecule A     \\
\begin{tabular}{|c c c|}
-0.999882	 & 	-0.0153593	 & 	-0.00028374	 \\
0.0153575	 & 	-0.999867	 & 	0.00557573	 \\
-0.000369342	 & 	0.00557072	 & 	0.999984
\end{tabular}

\vtab
 EingenValues - Molecule A     \\
\begin{tabular}{|c c c|}
534.81	 & 	913.339	 & 	1353.78	 \\
\end{tabular}
\columnbreak

Molecule B \
4NFAACb

\includegraphics[width=6cm]{../Comparisons/ImagesFromVMD/4NFAACb.png}

Inertia Tensor - Molecule B \\
\begin{tabular}{|c c c|}
479.338	 & 	3.27331	 & 	-4.22553	 \\
3.27331	 & 	1242.4	 & 	0.852083	 \\
-4.22553	 & 	0.852083	 & 	1647.3
\end{tabular}

\vtab
 EingenVectors - Molecule B     \\
\begin{tabular}{|c c c|}
0.999984	 & 	-0.00429353	 & 	0.00362086	 \\
-0.004301	 & 	-0.999989	 & 	0.00205959	 \\
-0.00361198	 & 	0.00207513	 & 	0.999991
\end{tabular}

\vtab
 EingenValues - Molecule B     \\
\begin{tabular}{|c c c|}
479.308	 & 	1242.41	 & 	1647.31	 \\
\end{tabular}

\end{center}
\end{multicols}

\vtab[-5mm]
\begin{tabular}{*{2}{m{0.38\textwidth}}}
\begin{center}
\textcolor{NavyBlue}{\Large Different}
\end{center}
&
\begin{center}
\includegraphics[height=6.5cm]{../Comparisons/Vectors/inertia_tensor_of_3NFAACk_and_4NFAACb.png}
\end{center}
\end{tabular}

 \newpage

\vtab[-3cm]
\begin{center}
{\large FireTest \tab Número 392}
\end{center}
\begin{multicols}{2}
\begin{center}

Molecule A \
3NFAACk

\includegraphics[width=6cm]{../Comparisons/ImagesFromVMD/3NFAACk.png}

Inertia Tensor - Molecule A \\
\begin{tabular}{|c c c|}
534.899	 & 	-5.81418	 & 	-0.270063	 \\
-5.81418	 & 	913.263	 & 	2.4519	 \\
-0.270063	 & 	2.4519	 & 	1353.77
\end{tabular}

\vtab
 EingenVectors - Molecule A     \\
\begin{tabular}{|c c c|}
-0.999882	 & 	-0.0153593	 & 	-0.00028374	 \\
0.0153575	 & 	-0.999867	 & 	0.00557573	 \\
-0.000369342	 & 	0.00557072	 & 	0.999984
\end{tabular}

\vtab
 EingenValues - Molecule A     \\
\begin{tabular}{|c c c|}
534.81	 & 	913.339	 & 	1353.78	 \\
\end{tabular}
\columnbreak

Molecule B \
4NFAACc

\includegraphics[width=6cm]{../Comparisons/ImagesFromVMD/4NFAACc.png}

Inertia Tensor - Molecule B \\
\begin{tabular}{|c c c|}
482.067	 & 	-5.39474	 & 	-1.35857	 \\
-5.39474	 & 	1240.3	 & 	-2.54035	 \\
-1.35857	 & 	-2.54035	 & 	1647.06
\end{tabular}

\vtab
 EingenVectors - Molecule B     \\
\begin{tabular}{|c c c|}
-0.999974	 & 	-0.00711826	 & 	-0.0011816	 \\
0.00712547	 & 	-0.999955	 & 	-0.00622156	 \\
-0.00113726	 & 	-0.00622982	 & 	0.99998
\end{tabular}

\vtab
 EingenValues - Molecule B     \\
\begin{tabular}{|c c c|}
482.027	 & 	1240.32	 & 	1647.08	 \\
\end{tabular}

\end{center}
\end{multicols}

\vtab[-5mm]
\begin{tabular}{*{2}{m{0.38\textwidth}}}
\begin{center}
\textcolor{NavyBlue}{\Large Different}
\end{center}
&
\begin{center}
\includegraphics[height=6.5cm]{../Comparisons/Vectors/inertia_tensor_of_3NFAACk_and_4NFAACc.png}
\end{center}
\end{tabular}

 \newpage

\vtab[-3cm]
\begin{center}
{\large FireTest \tab Número 393}
\end{center}
\begin{multicols}{2}
\begin{center}

Molecule A \
3NFAACk

\includegraphics[width=6cm]{../Comparisons/ImagesFromVMD/3NFAACk.png}

Inertia Tensor - Molecule A \\
\begin{tabular}{|c c c|}
534.899	 & 	-5.81418	 & 	-0.270063	 \\
-5.81418	 & 	913.263	 & 	2.4519	 \\
-0.270063	 & 	2.4519	 & 	1353.77
\end{tabular}

\vtab
 EingenVectors - Molecule A     \\
\begin{tabular}{|c c c|}
-0.999882	 & 	-0.0153593	 & 	-0.00028374	 \\
0.0153575	 & 	-0.999867	 & 	0.00557573	 \\
-0.000369342	 & 	0.00557072	 & 	0.999984
\end{tabular}

\vtab
 EingenValues - Molecule A     \\
\begin{tabular}{|c c c|}
534.81	 & 	913.339	 & 	1353.78	 \\
\end{tabular}
\columnbreak

Molecule B \
4NFAACd

\includegraphics[width=6cm]{../Comparisons/ImagesFromVMD/4NFAACd.png}

Inertia Tensor - Molecule B \\
\begin{tabular}{|c c c|}
491.672	 & 	0.24486	 & 	-3.10016	 \\
0.24486	 & 	1231.15	 & 	2.19965	 \\
-3.10016	 & 	2.19965	 & 	1650.11
\end{tabular}

\vtab
 EingenVectors - Molecule B     \\
\begin{tabular}{|c c c|}
0.999996	 & 	-0.000339081	 & 	0.00267677	 \\
-0.000353124	 & 	-0.999986	 & 	0.00524747	 \\
-0.00267495	 & 	0.0052484	 & 	0.999983
\end{tabular}

\vtab
 EingenValues - Molecule B     \\
\begin{tabular}{|c c c|}
491.663	 & 	1231.14	 & 	1650.13	 \\
\end{tabular}

\end{center}
\end{multicols}

\vtab[-5mm]
\begin{tabular}{*{2}{m{0.38\textwidth}}}
\begin{center}
\textcolor{NavyBlue}{\Large Different}
\end{center}
&
\begin{center}
\includegraphics[height=6.5cm]{../Comparisons/Vectors/inertia_tensor_of_3NFAACk_and_4NFAACd.png}
\end{center}
\end{tabular}

 \newpage

\vtab[-3cm]
\begin{center}
{\large FireTest \tab Número 394}
\end{center}
\begin{multicols}{2}
\begin{center}

Molecule A \
3NFAACk

\includegraphics[width=6cm]{../Comparisons/ImagesFromVMD/3NFAACk.png}

Inertia Tensor - Molecule A \\
\begin{tabular}{|c c c|}
534.899	 & 	-5.81418	 & 	-0.270063	 \\
-5.81418	 & 	913.263	 & 	2.4519	 \\
-0.270063	 & 	2.4519	 & 	1353.77
\end{tabular}

\vtab
 EingenVectors - Molecule A     \\
\begin{tabular}{|c c c|}
-0.999882	 & 	-0.0153593	 & 	-0.00028374	 \\
0.0153575	 & 	-0.999867	 & 	0.00557573	 \\
-0.000369342	 & 	0.00557072	 & 	0.999984
\end{tabular}

\vtab
 EingenValues - Molecule A     \\
\begin{tabular}{|c c c|}
534.81	 & 	913.339	 & 	1353.78	 \\
\end{tabular}
\columnbreak

Molecule B \
4NFAACe

\includegraphics[width=6cm]{../Comparisons/ImagesFromVMD/4NFAACe.png}

Inertia Tensor - Molecule B \\
\begin{tabular}{|c c c|}
489.025	 & 	-0.430035	 & 	3.98876	 \\
-0.430035	 & 	1233.71	 & 	-2.06505	 \\
3.98876	 & 	-2.06505	 & 	1641.79
\end{tabular}

\vtab
 EingenVectors - Molecule B     \\
\begin{tabular}{|c c c|}
0.999994	 & 	0.000567863	 & 	-0.00345908	 \\
0.000550336	 & 	-0.999987	 & 	-0.00506565	 \\
0.00346192	 & 	-0.00506372	 & 	0.999981
\end{tabular}

\vtab
 EingenValues - Molecule B     \\
\begin{tabular}{|c c c|}
489.011	 & 	1233.7	 & 	1641.81	 \\
\end{tabular}

\end{center}
\end{multicols}

\vtab[-5mm]
\begin{tabular}{*{2}{m{0.38\textwidth}}}
\begin{center}
\textcolor{NavyBlue}{\Large Different}
\end{center}
&
\begin{center}
\includegraphics[height=6.5cm]{../Comparisons/Vectors/inertia_tensor_of_3NFAACk_and_4NFAACe.png}
\end{center}
\end{tabular}

 \newpage

\vtab[-3cm]
\begin{center}
{\large FireTest \tab Número 395}
\end{center}
\begin{multicols}{2}
\begin{center}

Molecule A \
3NFAACk

\includegraphics[width=6cm]{../Comparisons/ImagesFromVMD/3NFAACk.png}

Inertia Tensor - Molecule A \\
\begin{tabular}{|c c c|}
534.899	 & 	-5.81418	 & 	-0.270063	 \\
-5.81418	 & 	913.263	 & 	2.4519	 \\
-0.270063	 & 	2.4519	 & 	1353.77
\end{tabular}

\vtab
 EingenVectors - Molecule A     \\
\begin{tabular}{|c c c|}
-0.999882	 & 	-0.0153593	 & 	-0.00028374	 \\
0.0153575	 & 	-0.999867	 & 	0.00557573	 \\
-0.000369342	 & 	0.00557072	 & 	0.999984
\end{tabular}

\vtab
 EingenValues - Molecule A     \\
\begin{tabular}{|c c c|}
534.81	 & 	913.339	 & 	1353.78	 \\
\end{tabular}
\columnbreak

Molecule B \
4NFAACf

\includegraphics[width=6cm]{../Comparisons/ImagesFromVMD/4NFAACf.png}

Inertia Tensor - Molecule B \\
\begin{tabular}{|c c c|}
509.683	 & 	2.80651	 & 	-1.91422	 \\
2.80651	 & 	1219.11	 & 	2.66132	 \\
-1.91422	 & 	2.66132	 & 	1681.17
\end{tabular}

\vtab
 EingenVectors - Molecule B     \\
\begin{tabular}{|c c c|}
-0.999991	 & 	0.00396206	 & 	-0.00164298	 \\
-0.00397143	 & 	-0.999976	 & 	0.0057431	 \\
-0.00162019	 & 	0.00574957	 & 	0.999982
\end{tabular}

\vtab
 EingenValues - Molecule B     \\
\begin{tabular}{|c c c|}
509.668	 & 	1219.11	 & 	1681.18	 \\
\end{tabular}

\end{center}
\end{multicols}

\vtab[-5mm]
\begin{tabular}{*{2}{m{0.38\textwidth}}}
\begin{center}
\textcolor{NavyBlue}{\Large Different}
\end{center}
&
\begin{center}
\includegraphics[height=6.5cm]{../Comparisons/Vectors/inertia_tensor_of_3NFAACk_and_4NFAACf.png}
\end{center}
\end{tabular}

 \newpage

\vtab[-3cm]
\begin{center}
{\large FireTest \tab Número 396}
\end{center}
\begin{multicols}{2}
\begin{center}

Molecule A \
3NFAACk

\includegraphics[width=6cm]{../Comparisons/ImagesFromVMD/3NFAACk.png}

Inertia Tensor - Molecule A \\
\begin{tabular}{|c c c|}
534.899	 & 	-5.81418	 & 	-0.270063	 \\
-5.81418	 & 	913.263	 & 	2.4519	 \\
-0.270063	 & 	2.4519	 & 	1353.77
\end{tabular}

\vtab
 EingenVectors - Molecule A     \\
\begin{tabular}{|c c c|}
-0.999882	 & 	-0.0153593	 & 	-0.00028374	 \\
0.0153575	 & 	-0.999867	 & 	0.00557573	 \\
-0.000369342	 & 	0.00557072	 & 	0.999984
\end{tabular}

\vtab
 EingenValues - Molecule A     \\
\begin{tabular}{|c c c|}
534.81	 & 	913.339	 & 	1353.78	 \\
\end{tabular}
\columnbreak

Molecule B \
4NFAACg

\includegraphics[width=6cm]{../Comparisons/ImagesFromVMD/4NFAACg.png}

Inertia Tensor - Molecule B \\
\begin{tabular}{|c c c|}
513.78	 & 	4.51917	 & 	0.266555	 \\
4.51917	 & 	1208.04	 & 	-1.18628	 \\
0.266555	 & 	-1.18628	 & 	1700.9
\end{tabular}

\vtab
 EingenVectors - Molecule B     \\
\begin{tabular}{|c c c|}
-0.999979	 & 	0.00650929	 & 	0.000231034	 \\
-0.00650983	 & 	-0.999976	 & 	-0.00240351	 \\
0.000215383	 & 	-0.00240496	 & 	0.999997
\end{tabular}

\vtab
 EingenValues - Molecule B     \\
\begin{tabular}{|c c c|}
513.751	 & 	1208.07	 & 	1700.9	 \\
\end{tabular}

\end{center}
\end{multicols}

\vtab[-5mm]
\begin{tabular}{*{2}{m{0.38\textwidth}}}
\begin{center}
\textcolor{NavyBlue}{\Large Different}
\end{center}
&
\begin{center}
\includegraphics[height=6.5cm]{../Comparisons/Vectors/inertia_tensor_of_3NFAACk_and_4NFAACg.png}
\end{center}
\end{tabular}

 \newpage

\vtab[-3cm]
\begin{center}
{\large FireTest \tab Número 397}
\end{center}
\begin{multicols}{2}
\begin{center}

Molecule A \
3NFAACk

\includegraphics[width=6cm]{../Comparisons/ImagesFromVMD/3NFAACk.png}

Inertia Tensor - Molecule A \\
\begin{tabular}{|c c c|}
534.899	 & 	-5.81418	 & 	-0.270063	 \\
-5.81418	 & 	913.263	 & 	2.4519	 \\
-0.270063	 & 	2.4519	 & 	1353.77
\end{tabular}

\vtab
 EingenVectors - Molecule A     \\
\begin{tabular}{|c c c|}
-0.999882	 & 	-0.0153593	 & 	-0.00028374	 \\
0.0153575	 & 	-0.999867	 & 	0.00557573	 \\
-0.000369342	 & 	0.00557072	 & 	0.999984
\end{tabular}

\vtab
 EingenValues - Molecule A     \\
\begin{tabular}{|c c c|}
534.81	 & 	913.339	 & 	1353.78	 \\
\end{tabular}
\columnbreak

Molecule B \
4NFAACi

\includegraphics[width=6cm]{../Comparisons/ImagesFromVMD/4NFAACi.png}

Inertia Tensor - Molecule B \\
\begin{tabular}{|c c c|}
502.43	 & 	-0.602691	 & 	-4.86988	 \\
-0.602691	 & 	1232.26	 & 	0.407295	 \\
-4.86988	 & 	0.407295	 & 	1676
\end{tabular}

\vtab
 EingenVectors - Molecule B     \\
\begin{tabular}{|c c c|}
0.999991	 & 	0.000823447	 & 	0.00414923	 \\
0.000819608	 & 	-0.999999	 & 	0.00092687	 \\
-0.00414999	 & 	0.000923461	 & 	0.999991
\end{tabular}

\vtab
 EingenValues - Molecule B     \\
\begin{tabular}{|c c c|}
502.409	 & 	1232.26	 & 	1676.02	 \\
\end{tabular}

\end{center}
\end{multicols}

\vtab[-5mm]
\begin{tabular}{*{2}{m{0.38\textwidth}}}
\begin{center}
\textcolor{NavyBlue}{\Large Different}
\end{center}
&
\begin{center}
\includegraphics[height=6.5cm]{../Comparisons/Vectors/inertia_tensor_of_3NFAACk_and_4NFAACi.png}
\end{center}
\end{tabular}

 \newpage

\vtab[-3cm]
\begin{center}
{\large FireTest \tab Número 398}
\end{center}
\begin{multicols}{2}
\begin{center}

Molecule A \
3NFAACk

\includegraphics[width=6cm]{../Comparisons/ImagesFromVMD/3NFAACk.png}

Inertia Tensor - Molecule A \\
\begin{tabular}{|c c c|}
534.899	 & 	-5.81418	 & 	-0.270063	 \\
-5.81418	 & 	913.263	 & 	2.4519	 \\
-0.270063	 & 	2.4519	 & 	1353.77
\end{tabular}

\vtab
 EingenVectors - Molecule A     \\
\begin{tabular}{|c c c|}
-0.999882	 & 	-0.0153593	 & 	-0.00028374	 \\
0.0153575	 & 	-0.999867	 & 	0.00557573	 \\
-0.000369342	 & 	0.00557072	 & 	0.999984
\end{tabular}

\vtab
 EingenValues - Molecule A     \\
\begin{tabular}{|c c c|}
534.81	 & 	913.339	 & 	1353.78	 \\
\end{tabular}
\columnbreak

Molecule B \
4NFAACj

\includegraphics[width=6cm]{../Comparisons/ImagesFromVMD/4NFAACj.png}

Inertia Tensor - Molecule B \\
\begin{tabular}{|c c c|}
510.047	 & 	9.97005	 & 	-3.6306	 \\
9.97005	 & 	1225.52	 & 	-0.981092	 \\
-3.6306	 & 	-0.981092	 & 	1680.82
\end{tabular}

\vtab
 EingenVectors - Molecule B     \\
\begin{tabular}{|c c c|}
-0.999898	 & 	0.0139264	 & 	-0.00308865	 \\
0.0139195	 & 	0.999901	 & 	0.00226627	 \\
-0.00311991	 & 	-0.00222305	 & 	0.999993
\end{tabular}

\vtab
 EingenValues - Molecule B     \\
\begin{tabular}{|c c c|}
509.897	 & 	1225.65	 & 	1680.83	 \\
\end{tabular}

\end{center}
\end{multicols}

\vtab[-5mm]
\begin{tabular}{*{2}{m{0.38\textwidth}}}
\begin{center}
\textcolor{NavyBlue}{\Large Different}
\end{center}
&
\begin{center}
\includegraphics[height=6.5cm]{../Comparisons/Vectors/inertia_tensor_of_3NFAACk_and_4NFAACj.png}
\end{center}
\end{tabular}

 \newpage

\vtab[-3cm]
\begin{center}
{\large FireTest \tab Número 399}
\end{center}
\begin{multicols}{2}
\begin{center}

Molecule A \
3NFAACk

\includegraphics[width=6cm]{../Comparisons/ImagesFromVMD/3NFAACk.png}

Inertia Tensor - Molecule A \\
\begin{tabular}{|c c c|}
534.899	 & 	-5.81418	 & 	-0.270063	 \\
-5.81418	 & 	913.263	 & 	2.4519	 \\
-0.270063	 & 	2.4519	 & 	1353.77
\end{tabular}

\vtab
 EingenVectors - Molecule A     \\
\begin{tabular}{|c c c|}
-0.999882	 & 	-0.0153593	 & 	-0.00028374	 \\
0.0153575	 & 	-0.999867	 & 	0.00557573	 \\
-0.000369342	 & 	0.00557072	 & 	0.999984
\end{tabular}

\vtab
 EingenValues - Molecule A     \\
\begin{tabular}{|c c c|}
534.81	 & 	913.339	 & 	1353.78	 \\
\end{tabular}
\columnbreak

Molecule B \
4NFAACl-3

\includegraphics[width=6cm]{../Comparisons/ImagesFromVMD/4NFAACl-3.png}

Inertia Tensor - Molecule B \\
\begin{tabular}{|c c c|}
506.608	 & 	0.709539	 & 	-0.555426	 \\
0.709539	 & 	1222.37	 & 	-2.84005	 \\
-0.555426	 & 	-2.84005	 & 	1678.41
\end{tabular}

\vtab
 EingenVectors - Molecule B     \\
\begin{tabular}{|c c c|}
-0.999999	 & 	0.000989428	 & 	-0.000471595	 \\
-0.000986471	 & 	-0.99998	 & 	-0.00622856	 \\
-0.000477748	 & 	-0.00622809	 & 	0.99998
\end{tabular}

\vtab
 EingenValues - Molecule B     \\
\begin{tabular}{|c c c|}
506.607	 & 	1222.36	 & 	1678.43	 \\
\end{tabular}

\end{center}
\end{multicols}

\vtab[-5mm]
\begin{tabular}{*{2}{m{0.38\textwidth}}}
\begin{center}
\textcolor{NavyBlue}{\Large Different}
\end{center}
&
\begin{center}
\includegraphics[height=6.5cm]{../Comparisons/Vectors/inertia_tensor_of_3NFAACk_and_4NFAACl-3.png}
\end{center}
\end{tabular}

 \newpage

\vtab[-3cm]
\begin{center}
{\large FireTest \tab Número 400}
\end{center}
\begin{multicols}{2}
\begin{center}

Molecule A \
3NFAACl

\includegraphics[width=6cm]{../Comparisons/ImagesFromVMD/3NFAACl.png}

Inertia Tensor - Molecule A \\
\begin{tabular}{|c c c|}
531.723	 & 	3.03424	 & 	2.73426	 \\
3.03424	 & 	929.418	 & 	-1.84284	 \\
2.73426	 & 	-1.84284	 & 	1355.39
\end{tabular}

\vtab
 EingenVectors - Molecule A     \\
\begin{tabular}{|c c c|}
0.999965	 & 	-0.00764413	 & 	-0.00333648	 \\
-0.00765838	 & 	-0.999962	 & 	-0.00427708	 \\
0.00330366	 & 	-0.00430248	 & 	0.999985
\end{tabular}

\vtab
 EingenValues - Molecule A     \\
\begin{tabular}{|c c c|}
531.691	 & 	929.434	 & 	1355.4	 \\
\end{tabular}
\columnbreak

Molecule B \
3NFAACm

\includegraphics[width=6cm]{../Comparisons/ImagesFromVMD/3NFAACm.png}

Inertia Tensor - Molecule B \\
\begin{tabular}{|c c c|}
532.546	 & 	13.7854	 & 	-15.4626	 \\
13.7854	 & 	1354.87	 & 	11.5786	 \\
-15.4626	 & 	11.5786	 & 	929.101
\end{tabular}

\vtab
 EingenVectors - Molecule B     \\
\begin{tabular}{|c c c|}
-0.999075	 & 	0.0172851	 & 	-0.0393769	 \\
0.0398168	 & 	0.0258942	 & 	-0.998871	 \\
-0.016246	 & 	-0.999515	 & 	-0.0265584
\end{tabular}

\vtab
 EingenValues - Molecule B     \\
\begin{tabular}{|c c c|}
531.698	 & 	929.417	 & 	1355.4	 \\
\end{tabular}

\end{center}
\end{multicols}

\vtab[-5mm]
\begin{tabular}{*{2}{m{0.38\textwidth}}}
\begin{center}
\textcolor{NavyBlue}{\Large Equal}
\end{center}
&
\begin{center}
\includegraphics[height=6.5cm]{../Comparisons/Vectors/inertia_tensor_of_3NFAACl_and_3NFAACm.png}
\end{center}
\end{tabular}

 \newpage

\vtab[-3cm]
\begin{center}
{\large FireTest \tab Número 401}
\end{center}
\begin{multicols}{2}
\begin{center}

Molecule A \
3NFAACl

\includegraphics[width=6cm]{../Comparisons/ImagesFromVMD/3NFAACl.png}

Inertia Tensor - Molecule A \\
\begin{tabular}{|c c c|}
531.723	 & 	3.03424	 & 	2.73426	 \\
3.03424	 & 	929.418	 & 	-1.84284	 \\
2.73426	 & 	-1.84284	 & 	1355.39
\end{tabular}

\vtab
 EingenVectors - Molecule A     \\
\begin{tabular}{|c c c|}
0.999965	 & 	-0.00764413	 & 	-0.00333648	 \\
-0.00765838	 & 	-0.999962	 & 	-0.00427708	 \\
0.00330366	 & 	-0.00430248	 & 	0.999985
\end{tabular}

\vtab
 EingenValues - Molecule A     \\
\begin{tabular}{|c c c|}
531.691	 & 	929.434	 & 	1355.4	 \\
\end{tabular}
\columnbreak

Molecule B \
3NFAACn

\includegraphics[width=6cm]{../Comparisons/ImagesFromVMD/3NFAACn.png}

Inertia Tensor - Molecule B \\
\begin{tabular}{|c c c|}
531.896	 & 	3.78027	 & 	-13.1151	 \\
3.78027	 & 	1353.2	 & 	-7.47403	 \\
-13.1151	 & 	-7.47403	 & 	912.989
\end{tabular}

\vtab
 EingenVectors - Molecule B     \\
\begin{tabular}{|c c c|}
0.999403	 & 	-0.00428573	 & 	0.0342679	 \\
0.0341891	 & 	-0.0172718	 & 	-0.999266	 \\
0.00487445	 & 	0.999842	 & 	-0.017115
\end{tabular}

\vtab
 EingenValues - Molecule B     \\
\begin{tabular}{|c c c|}
531.43	 & 	913.309	 & 	1353.35	 \\
\end{tabular}

\end{center}
\end{multicols}

\vtab[-5mm]
\begin{tabular}{*{2}{m{0.38\textwidth}}}
\begin{center}
\textcolor{NavyBlue}{\Large Different}
\end{center}
&
\begin{center}
\includegraphics[height=6.5cm]{../Comparisons/Vectors/inertia_tensor_of_3NFAACl_and_3NFAACn.png}
\end{center}
\end{tabular}

 \newpage

\vtab[-3cm]
\begin{center}
{\large FireTest \tab Número 402}
\end{center}
\begin{multicols}{2}
\begin{center}

Molecule A \
3NFAACl

\includegraphics[width=6cm]{../Comparisons/ImagesFromVMD/3NFAACl.png}

Inertia Tensor - Molecule A \\
\begin{tabular}{|c c c|}
531.723	 & 	3.03424	 & 	2.73426	 \\
3.03424	 & 	929.418	 & 	-1.84284	 \\
2.73426	 & 	-1.84284	 & 	1355.39
\end{tabular}

\vtab
 EingenVectors - Molecule A     \\
\begin{tabular}{|c c c|}
0.999965	 & 	-0.00764413	 & 	-0.00333648	 \\
-0.00765838	 & 	-0.999962	 & 	-0.00427708	 \\
0.00330366	 & 	-0.00430248	 & 	0.999985
\end{tabular}

\vtab
 EingenValues - Molecule A     \\
\begin{tabular}{|c c c|}
531.691	 & 	929.434	 & 	1355.4	 \\
\end{tabular}
\columnbreak

Molecule B \
4NFAACa

\includegraphics[width=6cm]{../Comparisons/ImagesFromVMD/4NFAACa.png}

Inertia Tensor - Molecule B \\
\begin{tabular}{|c c c|}
479.392	 & 	3.27131	 & 	4.22557	 \\
3.27131	 & 	1242.39	 & 	-0.852684	 \\
4.22557	 & 	-0.852684	 & 	1647.37
\end{tabular}

\vtab
 EingenVectors - Molecule B     \\
\begin{tabular}{|c c c|}
0.999984	 & 	-0.00429123	 & 	-0.00362083	 \\
-0.00429871	 & 	-0.999989	 & 	-0.0020607	 \\
0.00361195	 & 	-0.00207623	 & 	0.999991
\end{tabular}

\vtab
 EingenValues - Molecule B     \\
\begin{tabular}{|c c c|}
479.363	 & 	1242.41	 & 	1647.39	 \\
\end{tabular}

\end{center}
\end{multicols}

\vtab[-5mm]
\begin{tabular}{*{2}{m{0.38\textwidth}}}
\begin{center}
\textcolor{NavyBlue}{\Large Different}
\end{center}
&
\begin{center}
\includegraphics[height=6.5cm]{../Comparisons/Vectors/inertia_tensor_of_3NFAACl_and_4NFAACa.png}
\end{center}
\end{tabular}

 \newpage

\vtab[-3cm]
\begin{center}
{\large FireTest \tab Número 403}
\end{center}
\begin{multicols}{2}
\begin{center}

Molecule A \
3NFAACl

\includegraphics[width=6cm]{../Comparisons/ImagesFromVMD/3NFAACl.png}

Inertia Tensor - Molecule A \\
\begin{tabular}{|c c c|}
531.723	 & 	3.03424	 & 	2.73426	 \\
3.03424	 & 	929.418	 & 	-1.84284	 \\
2.73426	 & 	-1.84284	 & 	1355.39
\end{tabular}

\vtab
 EingenVectors - Molecule A     \\
\begin{tabular}{|c c c|}
0.999965	 & 	-0.00764413	 & 	-0.00333648	 \\
-0.00765838	 & 	-0.999962	 & 	-0.00427708	 \\
0.00330366	 & 	-0.00430248	 & 	0.999985
\end{tabular}

\vtab
 EingenValues - Molecule A     \\
\begin{tabular}{|c c c|}
531.691	 & 	929.434	 & 	1355.4	 \\
\end{tabular}
\columnbreak

Molecule B \
4NFAACb

\includegraphics[width=6cm]{../Comparisons/ImagesFromVMD/4NFAACb.png}

Inertia Tensor - Molecule B \\
\begin{tabular}{|c c c|}
479.338	 & 	3.27331	 & 	-4.22553	 \\
3.27331	 & 	1242.4	 & 	0.852083	 \\
-4.22553	 & 	0.852083	 & 	1647.3
\end{tabular}

\vtab
 EingenVectors - Molecule B     \\
\begin{tabular}{|c c c|}
0.999984	 & 	-0.00429353	 & 	0.00362086	 \\
-0.004301	 & 	-0.999989	 & 	0.00205959	 \\
-0.00361198	 & 	0.00207513	 & 	0.999991
\end{tabular}

\vtab
 EingenValues - Molecule B     \\
\begin{tabular}{|c c c|}
479.308	 & 	1242.41	 & 	1647.31	 \\
\end{tabular}

\end{center}
\end{multicols}

\vtab[-5mm]
\begin{tabular}{*{2}{m{0.38\textwidth}}}
\begin{center}
\textcolor{NavyBlue}{\Large Different}
\end{center}
&
\begin{center}
\includegraphics[height=6.5cm]{../Comparisons/Vectors/inertia_tensor_of_3NFAACl_and_4NFAACb.png}
\end{center}
\end{tabular}

 \newpage

\vtab[-3cm]
\begin{center}
{\large FireTest \tab Número 404}
\end{center}
\begin{multicols}{2}
\begin{center}

Molecule A \
3NFAACl

\includegraphics[width=6cm]{../Comparisons/ImagesFromVMD/3NFAACl.png}

Inertia Tensor - Molecule A \\
\begin{tabular}{|c c c|}
531.723	 & 	3.03424	 & 	2.73426	 \\
3.03424	 & 	929.418	 & 	-1.84284	 \\
2.73426	 & 	-1.84284	 & 	1355.39
\end{tabular}

\vtab
 EingenVectors - Molecule A     \\
\begin{tabular}{|c c c|}
0.999965	 & 	-0.00764413	 & 	-0.00333648	 \\
-0.00765838	 & 	-0.999962	 & 	-0.00427708	 \\
0.00330366	 & 	-0.00430248	 & 	0.999985
\end{tabular}

\vtab
 EingenValues - Molecule A     \\
\begin{tabular}{|c c c|}
531.691	 & 	929.434	 & 	1355.4	 \\
\end{tabular}
\columnbreak

Molecule B \
4NFAACc

\includegraphics[width=6cm]{../Comparisons/ImagesFromVMD/4NFAACc.png}

Inertia Tensor - Molecule B \\
\begin{tabular}{|c c c|}
482.067	 & 	-5.39474	 & 	-1.35857	 \\
-5.39474	 & 	1240.3	 & 	-2.54035	 \\
-1.35857	 & 	-2.54035	 & 	1647.06
\end{tabular}

\vtab
 EingenVectors - Molecule B     \\
\begin{tabular}{|c c c|}
-0.999974	 & 	-0.00711826	 & 	-0.0011816	 \\
0.00712547	 & 	-0.999955	 & 	-0.00622156	 \\
-0.00113726	 & 	-0.00622982	 & 	0.99998
\end{tabular}

\vtab
 EingenValues - Molecule B     \\
\begin{tabular}{|c c c|}
482.027	 & 	1240.32	 & 	1647.08	 \\
\end{tabular}

\end{center}
\end{multicols}

\vtab[-5mm]
\begin{tabular}{*{2}{m{0.38\textwidth}}}
\begin{center}
\textcolor{NavyBlue}{\Large Different}
\end{center}
&
\begin{center}
\includegraphics[height=6.5cm]{../Comparisons/Vectors/inertia_tensor_of_3NFAACl_and_4NFAACc.png}
\end{center}
\end{tabular}

 \newpage

\vtab[-3cm]
\begin{center}
{\large FireTest \tab Número 405}
\end{center}
\begin{multicols}{2}
\begin{center}

Molecule A \
3NFAACl

\includegraphics[width=6cm]{../Comparisons/ImagesFromVMD/3NFAACl.png}

Inertia Tensor - Molecule A \\
\begin{tabular}{|c c c|}
531.723	 & 	3.03424	 & 	2.73426	 \\
3.03424	 & 	929.418	 & 	-1.84284	 \\
2.73426	 & 	-1.84284	 & 	1355.39
\end{tabular}

\vtab
 EingenVectors - Molecule A     \\
\begin{tabular}{|c c c|}
0.999965	 & 	-0.00764413	 & 	-0.00333648	 \\
-0.00765838	 & 	-0.999962	 & 	-0.00427708	 \\
0.00330366	 & 	-0.00430248	 & 	0.999985
\end{tabular}

\vtab
 EingenValues - Molecule A     \\
\begin{tabular}{|c c c|}
531.691	 & 	929.434	 & 	1355.4	 \\
\end{tabular}
\columnbreak

Molecule B \
4NFAACd

\includegraphics[width=6cm]{../Comparisons/ImagesFromVMD/4NFAACd.png}

Inertia Tensor - Molecule B \\
\begin{tabular}{|c c c|}
491.672	 & 	0.24486	 & 	-3.10016	 \\
0.24486	 & 	1231.15	 & 	2.19965	 \\
-3.10016	 & 	2.19965	 & 	1650.11
\end{tabular}

\vtab
 EingenVectors - Molecule B     \\
\begin{tabular}{|c c c|}
0.999996	 & 	-0.000339081	 & 	0.00267677	 \\
-0.000353124	 & 	-0.999986	 & 	0.00524747	 \\
-0.00267495	 & 	0.0052484	 & 	0.999983
\end{tabular}

\vtab
 EingenValues - Molecule B     \\
\begin{tabular}{|c c c|}
491.663	 & 	1231.14	 & 	1650.13	 \\
\end{tabular}

\end{center}
\end{multicols}

\vtab[-5mm]
\begin{tabular}{*{2}{m{0.38\textwidth}}}
\begin{center}
\textcolor{NavyBlue}{\Large Different}
\end{center}
&
\begin{center}
\includegraphics[height=6.5cm]{../Comparisons/Vectors/inertia_tensor_of_3NFAACl_and_4NFAACd.png}
\end{center}
\end{tabular}

 \newpage

\vtab[-3cm]
\begin{center}
{\large FireTest \tab Número 406}
\end{center}
\begin{multicols}{2}
\begin{center}

Molecule A \
3NFAACl

\includegraphics[width=6cm]{../Comparisons/ImagesFromVMD/3NFAACl.png}

Inertia Tensor - Molecule A \\
\begin{tabular}{|c c c|}
531.723	 & 	3.03424	 & 	2.73426	 \\
3.03424	 & 	929.418	 & 	-1.84284	 \\
2.73426	 & 	-1.84284	 & 	1355.39
\end{tabular}

\vtab
 EingenVectors - Molecule A     \\
\begin{tabular}{|c c c|}
0.999965	 & 	-0.00764413	 & 	-0.00333648	 \\
-0.00765838	 & 	-0.999962	 & 	-0.00427708	 \\
0.00330366	 & 	-0.00430248	 & 	0.999985
\end{tabular}

\vtab
 EingenValues - Molecule A     \\
\begin{tabular}{|c c c|}
531.691	 & 	929.434	 & 	1355.4	 \\
\end{tabular}
\columnbreak

Molecule B \
4NFAACe

\includegraphics[width=6cm]{../Comparisons/ImagesFromVMD/4NFAACe.png}

Inertia Tensor - Molecule B \\
\begin{tabular}{|c c c|}
489.025	 & 	-0.430035	 & 	3.98876	 \\
-0.430035	 & 	1233.71	 & 	-2.06505	 \\
3.98876	 & 	-2.06505	 & 	1641.79
\end{tabular}

\vtab
 EingenVectors - Molecule B     \\
\begin{tabular}{|c c c|}
0.999994	 & 	0.000567863	 & 	-0.00345908	 \\
0.000550336	 & 	-0.999987	 & 	-0.00506565	 \\
0.00346192	 & 	-0.00506372	 & 	0.999981
\end{tabular}

\vtab
 EingenValues - Molecule B     \\
\begin{tabular}{|c c c|}
489.011	 & 	1233.7	 & 	1641.81	 \\
\end{tabular}

\end{center}
\end{multicols}

\vtab[-5mm]
\begin{tabular}{*{2}{m{0.38\textwidth}}}
\begin{center}
\textcolor{NavyBlue}{\Large Different}
\end{center}
&
\begin{center}
\includegraphics[height=6.5cm]{../Comparisons/Vectors/inertia_tensor_of_3NFAACl_and_4NFAACe.png}
\end{center}
\end{tabular}

 \newpage

\vtab[-3cm]
\begin{center}
{\large FireTest \tab Número 407}
\end{center}
\begin{multicols}{2}
\begin{center}

Molecule A \
3NFAACl

\includegraphics[width=6cm]{../Comparisons/ImagesFromVMD/3NFAACl.png}

Inertia Tensor - Molecule A \\
\begin{tabular}{|c c c|}
531.723	 & 	3.03424	 & 	2.73426	 \\
3.03424	 & 	929.418	 & 	-1.84284	 \\
2.73426	 & 	-1.84284	 & 	1355.39
\end{tabular}

\vtab
 EingenVectors - Molecule A     \\
\begin{tabular}{|c c c|}
0.999965	 & 	-0.00764413	 & 	-0.00333648	 \\
-0.00765838	 & 	-0.999962	 & 	-0.00427708	 \\
0.00330366	 & 	-0.00430248	 & 	0.999985
\end{tabular}

\vtab
 EingenValues - Molecule A     \\
\begin{tabular}{|c c c|}
531.691	 & 	929.434	 & 	1355.4	 \\
\end{tabular}
\columnbreak

Molecule B \
4NFAACf

\includegraphics[width=6cm]{../Comparisons/ImagesFromVMD/4NFAACf.png}

Inertia Tensor - Molecule B \\
\begin{tabular}{|c c c|}
509.683	 & 	2.80651	 & 	-1.91422	 \\
2.80651	 & 	1219.11	 & 	2.66132	 \\
-1.91422	 & 	2.66132	 & 	1681.17
\end{tabular}

\vtab
 EingenVectors - Molecule B     \\
\begin{tabular}{|c c c|}
-0.999991	 & 	0.00396206	 & 	-0.00164298	 \\
-0.00397143	 & 	-0.999976	 & 	0.0057431	 \\
-0.00162019	 & 	0.00574957	 & 	0.999982
\end{tabular}

\vtab
 EingenValues - Molecule B     \\
\begin{tabular}{|c c c|}
509.668	 & 	1219.11	 & 	1681.18	 \\
\end{tabular}

\end{center}
\end{multicols}

\vtab[-5mm]
\begin{tabular}{*{2}{m{0.38\textwidth}}}
\begin{center}
\textcolor{NavyBlue}{\Large Different}
\end{center}
&
\begin{center}
\includegraphics[height=6.5cm]{../Comparisons/Vectors/inertia_tensor_of_3NFAACl_and_4NFAACf.png}
\end{center}
\end{tabular}

 \newpage

\vtab[-3cm]
\begin{center}
{\large FireTest \tab Número 408}
\end{center}
\begin{multicols}{2}
\begin{center}

Molecule A \
3NFAACl

\includegraphics[width=6cm]{../Comparisons/ImagesFromVMD/3NFAACl.png}

Inertia Tensor - Molecule A \\
\begin{tabular}{|c c c|}
531.723	 & 	3.03424	 & 	2.73426	 \\
3.03424	 & 	929.418	 & 	-1.84284	 \\
2.73426	 & 	-1.84284	 & 	1355.39
\end{tabular}

\vtab
 EingenVectors - Molecule A     \\
\begin{tabular}{|c c c|}
0.999965	 & 	-0.00764413	 & 	-0.00333648	 \\
-0.00765838	 & 	-0.999962	 & 	-0.00427708	 \\
0.00330366	 & 	-0.00430248	 & 	0.999985
\end{tabular}

\vtab
 EingenValues - Molecule A     \\
\begin{tabular}{|c c c|}
531.691	 & 	929.434	 & 	1355.4	 \\
\end{tabular}
\columnbreak

Molecule B \
4NFAACg

\includegraphics[width=6cm]{../Comparisons/ImagesFromVMD/4NFAACg.png}

Inertia Tensor - Molecule B \\
\begin{tabular}{|c c c|}
513.78	 & 	4.51917	 & 	0.266555	 \\
4.51917	 & 	1208.04	 & 	-1.18628	 \\
0.266555	 & 	-1.18628	 & 	1700.9
\end{tabular}

\vtab
 EingenVectors - Molecule B     \\
\begin{tabular}{|c c c|}
-0.999979	 & 	0.00650929	 & 	0.000231034	 \\
-0.00650983	 & 	-0.999976	 & 	-0.00240351	 \\
0.000215383	 & 	-0.00240496	 & 	0.999997
\end{tabular}

\vtab
 EingenValues - Molecule B     \\
\begin{tabular}{|c c c|}
513.751	 & 	1208.07	 & 	1700.9	 \\
\end{tabular}

\end{center}
\end{multicols}

\vtab[-5mm]
\begin{tabular}{*{2}{m{0.38\textwidth}}}
\begin{center}
\textcolor{NavyBlue}{\Large Different}
\end{center}
&
\begin{center}
\includegraphics[height=6.5cm]{../Comparisons/Vectors/inertia_tensor_of_3NFAACl_and_4NFAACg.png}
\end{center}
\end{tabular}

 \newpage

\vtab[-3cm]
\begin{center}
{\large FireTest \tab Número 409}
\end{center}
\begin{multicols}{2}
\begin{center}

Molecule A \
3NFAACl

\includegraphics[width=6cm]{../Comparisons/ImagesFromVMD/3NFAACl.png}

Inertia Tensor - Molecule A \\
\begin{tabular}{|c c c|}
531.723	 & 	3.03424	 & 	2.73426	 \\
3.03424	 & 	929.418	 & 	-1.84284	 \\
2.73426	 & 	-1.84284	 & 	1355.39
\end{tabular}

\vtab
 EingenVectors - Molecule A     \\
\begin{tabular}{|c c c|}
0.999965	 & 	-0.00764413	 & 	-0.00333648	 \\
-0.00765838	 & 	-0.999962	 & 	-0.00427708	 \\
0.00330366	 & 	-0.00430248	 & 	0.999985
\end{tabular}

\vtab
 EingenValues - Molecule A     \\
\begin{tabular}{|c c c|}
531.691	 & 	929.434	 & 	1355.4	 \\
\end{tabular}
\columnbreak

Molecule B \
4NFAACi

\includegraphics[width=6cm]{../Comparisons/ImagesFromVMD/4NFAACi.png}

Inertia Tensor - Molecule B \\
\begin{tabular}{|c c c|}
502.43	 & 	-0.602691	 & 	-4.86988	 \\
-0.602691	 & 	1232.26	 & 	0.407295	 \\
-4.86988	 & 	0.407295	 & 	1676
\end{tabular}

\vtab
 EingenVectors - Molecule B     \\
\begin{tabular}{|c c c|}
0.999991	 & 	0.000823447	 & 	0.00414923	 \\
0.000819608	 & 	-0.999999	 & 	0.00092687	 \\
-0.00414999	 & 	0.000923461	 & 	0.999991
\end{tabular}

\vtab
 EingenValues - Molecule B     \\
\begin{tabular}{|c c c|}
502.409	 & 	1232.26	 & 	1676.02	 \\
\end{tabular}

\end{center}
\end{multicols}

\vtab[-5mm]
\begin{tabular}{*{2}{m{0.38\textwidth}}}
\begin{center}
\textcolor{NavyBlue}{\Large Different}
\end{center}
&
\begin{center}
\includegraphics[height=6.5cm]{../Comparisons/Vectors/inertia_tensor_of_3NFAACl_and_4NFAACi.png}
\end{center}
\end{tabular}

 \newpage

\vtab[-3cm]
\begin{center}
{\large FireTest \tab Número 410}
\end{center}
\begin{multicols}{2}
\begin{center}

Molecule A \
3NFAACl

\includegraphics[width=6cm]{../Comparisons/ImagesFromVMD/3NFAACl.png}

Inertia Tensor - Molecule A \\
\begin{tabular}{|c c c|}
531.723	 & 	3.03424	 & 	2.73426	 \\
3.03424	 & 	929.418	 & 	-1.84284	 \\
2.73426	 & 	-1.84284	 & 	1355.39
\end{tabular}

\vtab
 EingenVectors - Molecule A     \\
\begin{tabular}{|c c c|}
0.999965	 & 	-0.00764413	 & 	-0.00333648	 \\
-0.00765838	 & 	-0.999962	 & 	-0.00427708	 \\
0.00330366	 & 	-0.00430248	 & 	0.999985
\end{tabular}

\vtab
 EingenValues - Molecule A     \\
\begin{tabular}{|c c c|}
531.691	 & 	929.434	 & 	1355.4	 \\
\end{tabular}
\columnbreak

Molecule B \
4NFAACj

\includegraphics[width=6cm]{../Comparisons/ImagesFromVMD/4NFAACj.png}

Inertia Tensor - Molecule B \\
\begin{tabular}{|c c c|}
510.047	 & 	9.97005	 & 	-3.6306	 \\
9.97005	 & 	1225.52	 & 	-0.981092	 \\
-3.6306	 & 	-0.981092	 & 	1680.82
\end{tabular}

\vtab
 EingenVectors - Molecule B     \\
\begin{tabular}{|c c c|}
-0.999898	 & 	0.0139264	 & 	-0.00308865	 \\
0.0139195	 & 	0.999901	 & 	0.00226627	 \\
-0.00311991	 & 	-0.00222305	 & 	0.999993
\end{tabular}

\vtab
 EingenValues - Molecule B     \\
\begin{tabular}{|c c c|}
509.897	 & 	1225.65	 & 	1680.83	 \\
\end{tabular}

\end{center}
\end{multicols}

\vtab[-5mm]
\begin{tabular}{*{2}{m{0.38\textwidth}}}
\begin{center}
\textcolor{NavyBlue}{\Large Different}
\end{center}
&
\begin{center}
\includegraphics[height=6.5cm]{../Comparisons/Vectors/inertia_tensor_of_3NFAACl_and_4NFAACj.png}
\end{center}
\end{tabular}

 \newpage

\vtab[-3cm]
\begin{center}
{\large FireTest \tab Número 411}
\end{center}
\begin{multicols}{2}
\begin{center}

Molecule A \
3NFAACl

\includegraphics[width=6cm]{../Comparisons/ImagesFromVMD/3NFAACl.png}

Inertia Tensor - Molecule A \\
\begin{tabular}{|c c c|}
531.723	 & 	3.03424	 & 	2.73426	 \\
3.03424	 & 	929.418	 & 	-1.84284	 \\
2.73426	 & 	-1.84284	 & 	1355.39
\end{tabular}

\vtab
 EingenVectors - Molecule A     \\
\begin{tabular}{|c c c|}
0.999965	 & 	-0.00764413	 & 	-0.00333648	 \\
-0.00765838	 & 	-0.999962	 & 	-0.00427708	 \\
0.00330366	 & 	-0.00430248	 & 	0.999985
\end{tabular}

\vtab
 EingenValues - Molecule A     \\
\begin{tabular}{|c c c|}
531.691	 & 	929.434	 & 	1355.4	 \\
\end{tabular}
\columnbreak

Molecule B \
4NFAACl-3

\includegraphics[width=6cm]{../Comparisons/ImagesFromVMD/4NFAACl-3.png}

Inertia Tensor - Molecule B \\
\begin{tabular}{|c c c|}
506.608	 & 	0.709539	 & 	-0.555426	 \\
0.709539	 & 	1222.37	 & 	-2.84005	 \\
-0.555426	 & 	-2.84005	 & 	1678.41
\end{tabular}

\vtab
 EingenVectors - Molecule B     \\
\begin{tabular}{|c c c|}
-0.999999	 & 	0.000989428	 & 	-0.000471595	 \\
-0.000986471	 & 	-0.99998	 & 	-0.00622856	 \\
-0.000477748	 & 	-0.00622809	 & 	0.99998
\end{tabular}

\vtab
 EingenValues - Molecule B     \\
\begin{tabular}{|c c c|}
506.607	 & 	1222.36	 & 	1678.43	 \\
\end{tabular}

\end{center}
\end{multicols}

\vtab[-5mm]
\begin{tabular}{*{2}{m{0.38\textwidth}}}
\begin{center}
\textcolor{NavyBlue}{\Large Different}
\end{center}
&
\begin{center}
\includegraphics[height=6.5cm]{../Comparisons/Vectors/inertia_tensor_of_3NFAACl_and_4NFAACl-3.png}
\end{center}
\end{tabular}

 \newpage

\vtab[-3cm]
\begin{center}
{\large FireTest \tab Número 412}
\end{center}
\begin{multicols}{2}
\begin{center}

Molecule A \
3NFAACm

\includegraphics[width=6cm]{../Comparisons/ImagesFromVMD/3NFAACm.png}

Inertia Tensor - Molecule A \\
\begin{tabular}{|c c c|}
532.546	 & 	13.7854	 & 	-15.4626	 \\
13.7854	 & 	1354.87	 & 	11.5786	 \\
-15.4626	 & 	11.5786	 & 	929.101
\end{tabular}

\vtab
 EingenVectors - Molecule A     \\
\begin{tabular}{|c c c|}
-0.999075	 & 	0.0172851	 & 	-0.0393769	 \\
0.0398168	 & 	0.0258942	 & 	-0.998871	 \\
-0.016246	 & 	-0.999515	 & 	-0.0265584
\end{tabular}

\vtab
 EingenValues - Molecule A     \\
\begin{tabular}{|c c c|}
531.698	 & 	929.417	 & 	1355.4	 \\
\end{tabular}
\columnbreak

Molecule B \
3NFAACn

\includegraphics[width=6cm]{../Comparisons/ImagesFromVMD/3NFAACn.png}

Inertia Tensor - Molecule B \\
\begin{tabular}{|c c c|}
531.896	 & 	3.78027	 & 	-13.1151	 \\
3.78027	 & 	1353.2	 & 	-7.47403	 \\
-13.1151	 & 	-7.47403	 & 	912.989
\end{tabular}

\vtab
 EingenVectors - Molecule B     \\
\begin{tabular}{|c c c|}
0.999403	 & 	-0.00428573	 & 	0.0342679	 \\
0.0341891	 & 	-0.0172718	 & 	-0.999266	 \\
0.00487445	 & 	0.999842	 & 	-0.017115
\end{tabular}

\vtab
 EingenValues - Molecule B     \\
\begin{tabular}{|c c c|}
531.43	 & 	913.309	 & 	1353.35	 \\
\end{tabular}

\end{center}
\end{multicols}

\vtab[-5mm]
\begin{tabular}{*{2}{m{0.38\textwidth}}}
\begin{center}
\textcolor{NavyBlue}{\Large Different}
\end{center}
&
\begin{center}
\includegraphics[height=6.5cm]{../Comparisons/Vectors/inertia_tensor_of_3NFAACm_and_3NFAACn.png}
\end{center}
\end{tabular}

 \newpage

\vtab[-3cm]
\begin{center}
{\large FireTest \tab Número 413}
\end{center}
\begin{multicols}{2}
\begin{center}

Molecule A \
3NFAACm

\includegraphics[width=6cm]{../Comparisons/ImagesFromVMD/3NFAACm.png}

Inertia Tensor - Molecule A \\
\begin{tabular}{|c c c|}
532.546	 & 	13.7854	 & 	-15.4626	 \\
13.7854	 & 	1354.87	 & 	11.5786	 \\
-15.4626	 & 	11.5786	 & 	929.101
\end{tabular}

\vtab
 EingenVectors - Molecule A     \\
\begin{tabular}{|c c c|}
-0.999075	 & 	0.0172851	 & 	-0.0393769	 \\
0.0398168	 & 	0.0258942	 & 	-0.998871	 \\
-0.016246	 & 	-0.999515	 & 	-0.0265584
\end{tabular}

\vtab
 EingenValues - Molecule A     \\
\begin{tabular}{|c c c|}
531.698	 & 	929.417	 & 	1355.4	 \\
\end{tabular}
\columnbreak

Molecule B \
4NFAACa

\includegraphics[width=6cm]{../Comparisons/ImagesFromVMD/4NFAACa.png}

Inertia Tensor - Molecule B \\
\begin{tabular}{|c c c|}
479.392	 & 	3.27131	 & 	4.22557	 \\
3.27131	 & 	1242.39	 & 	-0.852684	 \\
4.22557	 & 	-0.852684	 & 	1647.37
\end{tabular}

\vtab
 EingenVectors - Molecule B     \\
\begin{tabular}{|c c c|}
0.999984	 & 	-0.00429123	 & 	-0.00362083	 \\
-0.00429871	 & 	-0.999989	 & 	-0.0020607	 \\
0.00361195	 & 	-0.00207623	 & 	0.999991
\end{tabular}

\vtab
 EingenValues - Molecule B     \\
\begin{tabular}{|c c c|}
479.363	 & 	1242.41	 & 	1647.39	 \\
\end{tabular}

\end{center}
\end{multicols}

\vtab[-5mm]
\begin{tabular}{*{2}{m{0.38\textwidth}}}
\begin{center}
\textcolor{NavyBlue}{\Large Different}
\end{center}
&
\begin{center}
\includegraphics[height=6.5cm]{../Comparisons/Vectors/inertia_tensor_of_3NFAACm_and_4NFAACa.png}
\end{center}
\end{tabular}

 \newpage

\vtab[-3cm]
\begin{center}
{\large FireTest \tab Número 414}
\end{center}
\begin{multicols}{2}
\begin{center}

Molecule A \
3NFAACm

\includegraphics[width=6cm]{../Comparisons/ImagesFromVMD/3NFAACm.png}

Inertia Tensor - Molecule A \\
\begin{tabular}{|c c c|}
532.546	 & 	13.7854	 & 	-15.4626	 \\
13.7854	 & 	1354.87	 & 	11.5786	 \\
-15.4626	 & 	11.5786	 & 	929.101
\end{tabular}

\vtab
 EingenVectors - Molecule A     \\
\begin{tabular}{|c c c|}
-0.999075	 & 	0.0172851	 & 	-0.0393769	 \\
0.0398168	 & 	0.0258942	 & 	-0.998871	 \\
-0.016246	 & 	-0.999515	 & 	-0.0265584
\end{tabular}

\vtab
 EingenValues - Molecule A     \\
\begin{tabular}{|c c c|}
531.698	 & 	929.417	 & 	1355.4	 \\
\end{tabular}
\columnbreak

Molecule B \
4NFAACb

\includegraphics[width=6cm]{../Comparisons/ImagesFromVMD/4NFAACb.png}

Inertia Tensor - Molecule B \\
\begin{tabular}{|c c c|}
479.338	 & 	3.27331	 & 	-4.22553	 \\
3.27331	 & 	1242.4	 & 	0.852083	 \\
-4.22553	 & 	0.852083	 & 	1647.3
\end{tabular}

\vtab
 EingenVectors - Molecule B     \\
\begin{tabular}{|c c c|}
0.999984	 & 	-0.00429353	 & 	0.00362086	 \\
-0.004301	 & 	-0.999989	 & 	0.00205959	 \\
-0.00361198	 & 	0.00207513	 & 	0.999991
\end{tabular}

\vtab
 EingenValues - Molecule B     \\
\begin{tabular}{|c c c|}
479.308	 & 	1242.41	 & 	1647.31	 \\
\end{tabular}

\end{center}
\end{multicols}

\vtab[-5mm]
\begin{tabular}{*{2}{m{0.38\textwidth}}}
\begin{center}
\textcolor{NavyBlue}{\Large Different}
\end{center}
&
\begin{center}
\includegraphics[height=6.5cm]{../Comparisons/Vectors/inertia_tensor_of_3NFAACm_and_4NFAACb.png}
\end{center}
\end{tabular}

 \newpage

\vtab[-3cm]
\begin{center}
{\large FireTest \tab Número 415}
\end{center}
\begin{multicols}{2}
\begin{center}

Molecule A \
3NFAACm

\includegraphics[width=6cm]{../Comparisons/ImagesFromVMD/3NFAACm.png}

Inertia Tensor - Molecule A \\
\begin{tabular}{|c c c|}
532.546	 & 	13.7854	 & 	-15.4626	 \\
13.7854	 & 	1354.87	 & 	11.5786	 \\
-15.4626	 & 	11.5786	 & 	929.101
\end{tabular}

\vtab
 EingenVectors - Molecule A     \\
\begin{tabular}{|c c c|}
-0.999075	 & 	0.0172851	 & 	-0.0393769	 \\
0.0398168	 & 	0.0258942	 & 	-0.998871	 \\
-0.016246	 & 	-0.999515	 & 	-0.0265584
\end{tabular}

\vtab
 EingenValues - Molecule A     \\
\begin{tabular}{|c c c|}
531.698	 & 	929.417	 & 	1355.4	 \\
\end{tabular}
\columnbreak

Molecule B \
4NFAACc

\includegraphics[width=6cm]{../Comparisons/ImagesFromVMD/4NFAACc.png}

Inertia Tensor - Molecule B \\
\begin{tabular}{|c c c|}
482.067	 & 	-5.39474	 & 	-1.35857	 \\
-5.39474	 & 	1240.3	 & 	-2.54035	 \\
-1.35857	 & 	-2.54035	 & 	1647.06
\end{tabular}

\vtab
 EingenVectors - Molecule B     \\
\begin{tabular}{|c c c|}
-0.999974	 & 	-0.00711826	 & 	-0.0011816	 \\
0.00712547	 & 	-0.999955	 & 	-0.00622156	 \\
-0.00113726	 & 	-0.00622982	 & 	0.99998
\end{tabular}

\vtab
 EingenValues - Molecule B     \\
\begin{tabular}{|c c c|}
482.027	 & 	1240.32	 & 	1647.08	 \\
\end{tabular}

\end{center}
\end{multicols}

\vtab[-5mm]
\begin{tabular}{*{2}{m{0.38\textwidth}}}
\begin{center}
\textcolor{NavyBlue}{\Large Different}
\end{center}
&
\begin{center}
\includegraphics[height=6.5cm]{../Comparisons/Vectors/inertia_tensor_of_3NFAACm_and_4NFAACc.png}
\end{center}
\end{tabular}

 \newpage

\vtab[-3cm]
\begin{center}
{\large FireTest \tab Número 416}
\end{center}
\begin{multicols}{2}
\begin{center}

Molecule A \
3NFAACm

\includegraphics[width=6cm]{../Comparisons/ImagesFromVMD/3NFAACm.png}

Inertia Tensor - Molecule A \\
\begin{tabular}{|c c c|}
532.546	 & 	13.7854	 & 	-15.4626	 \\
13.7854	 & 	1354.87	 & 	11.5786	 \\
-15.4626	 & 	11.5786	 & 	929.101
\end{tabular}

\vtab
 EingenVectors - Molecule A     \\
\begin{tabular}{|c c c|}
-0.999075	 & 	0.0172851	 & 	-0.0393769	 \\
0.0398168	 & 	0.0258942	 & 	-0.998871	 \\
-0.016246	 & 	-0.999515	 & 	-0.0265584
\end{tabular}

\vtab
 EingenValues - Molecule A     \\
\begin{tabular}{|c c c|}
531.698	 & 	929.417	 & 	1355.4	 \\
\end{tabular}
\columnbreak

Molecule B \
4NFAACd

\includegraphics[width=6cm]{../Comparisons/ImagesFromVMD/4NFAACd.png}

Inertia Tensor - Molecule B \\
\begin{tabular}{|c c c|}
491.672	 & 	0.24486	 & 	-3.10016	 \\
0.24486	 & 	1231.15	 & 	2.19965	 \\
-3.10016	 & 	2.19965	 & 	1650.11
\end{tabular}

\vtab
 EingenVectors - Molecule B     \\
\begin{tabular}{|c c c|}
0.999996	 & 	-0.000339081	 & 	0.00267677	 \\
-0.000353124	 & 	-0.999986	 & 	0.00524747	 \\
-0.00267495	 & 	0.0052484	 & 	0.999983
\end{tabular}

\vtab
 EingenValues - Molecule B     \\
\begin{tabular}{|c c c|}
491.663	 & 	1231.14	 & 	1650.13	 \\
\end{tabular}

\end{center}
\end{multicols}

\vtab[-5mm]
\begin{tabular}{*{2}{m{0.38\textwidth}}}
\begin{center}
\textcolor{NavyBlue}{\Large Different}
\end{center}
&
\begin{center}
\includegraphics[height=6.5cm]{../Comparisons/Vectors/inertia_tensor_of_3NFAACm_and_4NFAACd.png}
\end{center}
\end{tabular}

 \newpage

\vtab[-3cm]
\begin{center}
{\large FireTest \tab Número 417}
\end{center}
\begin{multicols}{2}
\begin{center}

Molecule A \
3NFAACm

\includegraphics[width=6cm]{../Comparisons/ImagesFromVMD/3NFAACm.png}

Inertia Tensor - Molecule A \\
\begin{tabular}{|c c c|}
532.546	 & 	13.7854	 & 	-15.4626	 \\
13.7854	 & 	1354.87	 & 	11.5786	 \\
-15.4626	 & 	11.5786	 & 	929.101
\end{tabular}

\vtab
 EingenVectors - Molecule A     \\
\begin{tabular}{|c c c|}
-0.999075	 & 	0.0172851	 & 	-0.0393769	 \\
0.0398168	 & 	0.0258942	 & 	-0.998871	 \\
-0.016246	 & 	-0.999515	 & 	-0.0265584
\end{tabular}

\vtab
 EingenValues - Molecule A     \\
\begin{tabular}{|c c c|}
531.698	 & 	929.417	 & 	1355.4	 \\
\end{tabular}
\columnbreak

Molecule B \
4NFAACe

\includegraphics[width=6cm]{../Comparisons/ImagesFromVMD/4NFAACe.png}

Inertia Tensor - Molecule B \\
\begin{tabular}{|c c c|}
489.025	 & 	-0.430035	 & 	3.98876	 \\
-0.430035	 & 	1233.71	 & 	-2.06505	 \\
3.98876	 & 	-2.06505	 & 	1641.79
\end{tabular}

\vtab
 EingenVectors - Molecule B     \\
\begin{tabular}{|c c c|}
0.999994	 & 	0.000567863	 & 	-0.00345908	 \\
0.000550336	 & 	-0.999987	 & 	-0.00506565	 \\
0.00346192	 & 	-0.00506372	 & 	0.999981
\end{tabular}

\vtab
 EingenValues - Molecule B     \\
\begin{tabular}{|c c c|}
489.011	 & 	1233.7	 & 	1641.81	 \\
\end{tabular}

\end{center}
\end{multicols}

\vtab[-5mm]
\begin{tabular}{*{2}{m{0.38\textwidth}}}
\begin{center}
\textcolor{NavyBlue}{\Large Different}
\end{center}
&
\begin{center}
\includegraphics[height=6.5cm]{../Comparisons/Vectors/inertia_tensor_of_3NFAACm_and_4NFAACe.png}
\end{center}
\end{tabular}

 \newpage

\vtab[-3cm]
\begin{center}
{\large FireTest \tab Número 418}
\end{center}
\begin{multicols}{2}
\begin{center}

Molecule A \
3NFAACm

\includegraphics[width=6cm]{../Comparisons/ImagesFromVMD/3NFAACm.png}

Inertia Tensor - Molecule A \\
\begin{tabular}{|c c c|}
532.546	 & 	13.7854	 & 	-15.4626	 \\
13.7854	 & 	1354.87	 & 	11.5786	 \\
-15.4626	 & 	11.5786	 & 	929.101
\end{tabular}

\vtab
 EingenVectors - Molecule A     \\
\begin{tabular}{|c c c|}
-0.999075	 & 	0.0172851	 & 	-0.0393769	 \\
0.0398168	 & 	0.0258942	 & 	-0.998871	 \\
-0.016246	 & 	-0.999515	 & 	-0.0265584
\end{tabular}

\vtab
 EingenValues - Molecule A     \\
\begin{tabular}{|c c c|}
531.698	 & 	929.417	 & 	1355.4	 \\
\end{tabular}
\columnbreak

Molecule B \
4NFAACf

\includegraphics[width=6cm]{../Comparisons/ImagesFromVMD/4NFAACf.png}

Inertia Tensor - Molecule B \\
\begin{tabular}{|c c c|}
509.683	 & 	2.80651	 & 	-1.91422	 \\
2.80651	 & 	1219.11	 & 	2.66132	 \\
-1.91422	 & 	2.66132	 & 	1681.17
\end{tabular}

\vtab
 EingenVectors - Molecule B     \\
\begin{tabular}{|c c c|}
-0.999991	 & 	0.00396206	 & 	-0.00164298	 \\
-0.00397143	 & 	-0.999976	 & 	0.0057431	 \\
-0.00162019	 & 	0.00574957	 & 	0.999982
\end{tabular}

\vtab
 EingenValues - Molecule B     \\
\begin{tabular}{|c c c|}
509.668	 & 	1219.11	 & 	1681.18	 \\
\end{tabular}

\end{center}
\end{multicols}

\vtab[-5mm]
\begin{tabular}{*{2}{m{0.38\textwidth}}}
\begin{center}
\textcolor{NavyBlue}{\Large Different}
\end{center}
&
\begin{center}
\includegraphics[height=6.5cm]{../Comparisons/Vectors/inertia_tensor_of_3NFAACm_and_4NFAACf.png}
\end{center}
\end{tabular}

 \newpage

\vtab[-3cm]
\begin{center}
{\large FireTest \tab Número 419}
\end{center}
\begin{multicols}{2}
\begin{center}

Molecule A \
3NFAACm

\includegraphics[width=6cm]{../Comparisons/ImagesFromVMD/3NFAACm.png}

Inertia Tensor - Molecule A \\
\begin{tabular}{|c c c|}
532.546	 & 	13.7854	 & 	-15.4626	 \\
13.7854	 & 	1354.87	 & 	11.5786	 \\
-15.4626	 & 	11.5786	 & 	929.101
\end{tabular}

\vtab
 EingenVectors - Molecule A     \\
\begin{tabular}{|c c c|}
-0.999075	 & 	0.0172851	 & 	-0.0393769	 \\
0.0398168	 & 	0.0258942	 & 	-0.998871	 \\
-0.016246	 & 	-0.999515	 & 	-0.0265584
\end{tabular}

\vtab
 EingenValues - Molecule A     \\
\begin{tabular}{|c c c|}
531.698	 & 	929.417	 & 	1355.4	 \\
\end{tabular}
\columnbreak

Molecule B \
4NFAACg

\includegraphics[width=6cm]{../Comparisons/ImagesFromVMD/4NFAACg.png}

Inertia Tensor - Molecule B \\
\begin{tabular}{|c c c|}
513.78	 & 	4.51917	 & 	0.266555	 \\
4.51917	 & 	1208.04	 & 	-1.18628	 \\
0.266555	 & 	-1.18628	 & 	1700.9
\end{tabular}

\vtab
 EingenVectors - Molecule B     \\
\begin{tabular}{|c c c|}
-0.999979	 & 	0.00650929	 & 	0.000231034	 \\
-0.00650983	 & 	-0.999976	 & 	-0.00240351	 \\
0.000215383	 & 	-0.00240496	 & 	0.999997
\end{tabular}

\vtab
 EingenValues - Molecule B     \\
\begin{tabular}{|c c c|}
513.751	 & 	1208.07	 & 	1700.9	 \\
\end{tabular}

\end{center}
\end{multicols}

\vtab[-5mm]
\begin{tabular}{*{2}{m{0.38\textwidth}}}
\begin{center}
\textcolor{NavyBlue}{\Large Different}
\end{center}
&
\begin{center}
\includegraphics[height=6.5cm]{../Comparisons/Vectors/inertia_tensor_of_3NFAACm_and_4NFAACg.png}
\end{center}
\end{tabular}

 \newpage

\vtab[-3cm]
\begin{center}
{\large FireTest \tab Número 420}
\end{center}
\begin{multicols}{2}
\begin{center}

Molecule A \
3NFAACm

\includegraphics[width=6cm]{../Comparisons/ImagesFromVMD/3NFAACm.png}

Inertia Tensor - Molecule A \\
\begin{tabular}{|c c c|}
532.546	 & 	13.7854	 & 	-15.4626	 \\
13.7854	 & 	1354.87	 & 	11.5786	 \\
-15.4626	 & 	11.5786	 & 	929.101
\end{tabular}

\vtab
 EingenVectors - Molecule A     \\
\begin{tabular}{|c c c|}
-0.999075	 & 	0.0172851	 & 	-0.0393769	 \\
0.0398168	 & 	0.0258942	 & 	-0.998871	 \\
-0.016246	 & 	-0.999515	 & 	-0.0265584
\end{tabular}

\vtab
 EingenValues - Molecule A     \\
\begin{tabular}{|c c c|}
531.698	 & 	929.417	 & 	1355.4	 \\
\end{tabular}
\columnbreak

Molecule B \
4NFAACi

\includegraphics[width=6cm]{../Comparisons/ImagesFromVMD/4NFAACi.png}

Inertia Tensor - Molecule B \\
\begin{tabular}{|c c c|}
502.43	 & 	-0.602691	 & 	-4.86988	 \\
-0.602691	 & 	1232.26	 & 	0.407295	 \\
-4.86988	 & 	0.407295	 & 	1676
\end{tabular}

\vtab
 EingenVectors - Molecule B     \\
\begin{tabular}{|c c c|}
0.999991	 & 	0.000823447	 & 	0.00414923	 \\
0.000819608	 & 	-0.999999	 & 	0.00092687	 \\
-0.00414999	 & 	0.000923461	 & 	0.999991
\end{tabular}

\vtab
 EingenValues - Molecule B     \\
\begin{tabular}{|c c c|}
502.409	 & 	1232.26	 & 	1676.02	 \\
\end{tabular}

\end{center}
\end{multicols}

\vtab[-5mm]
\begin{tabular}{*{2}{m{0.38\textwidth}}}
\begin{center}
\textcolor{NavyBlue}{\Large Different}
\end{center}
&
\begin{center}
\includegraphics[height=6.5cm]{../Comparisons/Vectors/inertia_tensor_of_3NFAACm_and_4NFAACi.png}
\end{center}
\end{tabular}

 \newpage

\vtab[-3cm]
\begin{center}
{\large FireTest \tab Número 421}
\end{center}
\begin{multicols}{2}
\begin{center}

Molecule A \
3NFAACm

\includegraphics[width=6cm]{../Comparisons/ImagesFromVMD/3NFAACm.png}

Inertia Tensor - Molecule A \\
\begin{tabular}{|c c c|}
532.546	 & 	13.7854	 & 	-15.4626	 \\
13.7854	 & 	1354.87	 & 	11.5786	 \\
-15.4626	 & 	11.5786	 & 	929.101
\end{tabular}

\vtab
 EingenVectors - Molecule A     \\
\begin{tabular}{|c c c|}
-0.999075	 & 	0.0172851	 & 	-0.0393769	 \\
0.0398168	 & 	0.0258942	 & 	-0.998871	 \\
-0.016246	 & 	-0.999515	 & 	-0.0265584
\end{tabular}

\vtab
 EingenValues - Molecule A     \\
\begin{tabular}{|c c c|}
531.698	 & 	929.417	 & 	1355.4	 \\
\end{tabular}
\columnbreak

Molecule B \
4NFAACj

\includegraphics[width=6cm]{../Comparisons/ImagesFromVMD/4NFAACj.png}

Inertia Tensor - Molecule B \\
\begin{tabular}{|c c c|}
510.047	 & 	9.97005	 & 	-3.6306	 \\
9.97005	 & 	1225.52	 & 	-0.981092	 \\
-3.6306	 & 	-0.981092	 & 	1680.82
\end{tabular}

\vtab
 EingenVectors - Molecule B     \\
\begin{tabular}{|c c c|}
-0.999898	 & 	0.0139264	 & 	-0.00308865	 \\
0.0139195	 & 	0.999901	 & 	0.00226627	 \\
-0.00311991	 & 	-0.00222305	 & 	0.999993
\end{tabular}

\vtab
 EingenValues - Molecule B     \\
\begin{tabular}{|c c c|}
509.897	 & 	1225.65	 & 	1680.83	 \\
\end{tabular}

\end{center}
\end{multicols}

\vtab[-5mm]
\begin{tabular}{*{2}{m{0.38\textwidth}}}
\begin{center}
\textcolor{NavyBlue}{\Large Different}
\end{center}
&
\begin{center}
\includegraphics[height=6.5cm]{../Comparisons/Vectors/inertia_tensor_of_3NFAACm_and_4NFAACj.png}
\end{center}
\end{tabular}

 \newpage

\vtab[-3cm]
\begin{center}
{\large FireTest \tab Número 422}
\end{center}
\begin{multicols}{2}
\begin{center}

Molecule A \
3NFAACm

\includegraphics[width=6cm]{../Comparisons/ImagesFromVMD/3NFAACm.png}

Inertia Tensor - Molecule A \\
\begin{tabular}{|c c c|}
532.546	 & 	13.7854	 & 	-15.4626	 \\
13.7854	 & 	1354.87	 & 	11.5786	 \\
-15.4626	 & 	11.5786	 & 	929.101
\end{tabular}

\vtab
 EingenVectors - Molecule A     \\
\begin{tabular}{|c c c|}
-0.999075	 & 	0.0172851	 & 	-0.0393769	 \\
0.0398168	 & 	0.0258942	 & 	-0.998871	 \\
-0.016246	 & 	-0.999515	 & 	-0.0265584
\end{tabular}

\vtab
 EingenValues - Molecule A     \\
\begin{tabular}{|c c c|}
531.698	 & 	929.417	 & 	1355.4	 \\
\end{tabular}
\columnbreak

Molecule B \
4NFAACl-3

\includegraphics[width=6cm]{../Comparisons/ImagesFromVMD/4NFAACl-3.png}

Inertia Tensor - Molecule B \\
\begin{tabular}{|c c c|}
506.608	 & 	0.709539	 & 	-0.555426	 \\
0.709539	 & 	1222.37	 & 	-2.84005	 \\
-0.555426	 & 	-2.84005	 & 	1678.41
\end{tabular}

\vtab
 EingenVectors - Molecule B     \\
\begin{tabular}{|c c c|}
-0.999999	 & 	0.000989428	 & 	-0.000471595	 \\
-0.000986471	 & 	-0.99998	 & 	-0.00622856	 \\
-0.000477748	 & 	-0.00622809	 & 	0.99998
\end{tabular}

\vtab
 EingenValues - Molecule B     \\
\begin{tabular}{|c c c|}
506.607	 & 	1222.36	 & 	1678.43	 \\
\end{tabular}

\end{center}
\end{multicols}

\vtab[-5mm]
\begin{tabular}{*{2}{m{0.38\textwidth}}}
\begin{center}
\textcolor{NavyBlue}{\Large Different}
\end{center}
&
\begin{center}
\includegraphics[height=6.5cm]{../Comparisons/Vectors/inertia_tensor_of_3NFAACm_and_4NFAACl-3.png}
\end{center}
\end{tabular}

 \newpage

\vtab[-3cm]
\begin{center}
{\large FireTest \tab Número 423}
\end{center}
\begin{multicols}{2}
\begin{center}

Molecule A \
3NFAACn

\includegraphics[width=6cm]{../Comparisons/ImagesFromVMD/3NFAACn.png}

Inertia Tensor - Molecule A \\
\begin{tabular}{|c c c|}
531.896	 & 	3.78027	 & 	-13.1151	 \\
3.78027	 & 	1353.2	 & 	-7.47403	 \\
-13.1151	 & 	-7.47403	 & 	912.989
\end{tabular}

\vtab
 EingenVectors - Molecule A     \\
\begin{tabular}{|c c c|}
0.999403	 & 	-0.00428573	 & 	0.0342679	 \\
0.0341891	 & 	-0.0172718	 & 	-0.999266	 \\
0.00487445	 & 	0.999842	 & 	-0.017115
\end{tabular}

\vtab
 EingenValues - Molecule A     \\
\begin{tabular}{|c c c|}
531.43	 & 	913.309	 & 	1353.35	 \\
\end{tabular}
\columnbreak

Molecule B \
4NFAACa

\includegraphics[width=6cm]{../Comparisons/ImagesFromVMD/4NFAACa.png}

Inertia Tensor - Molecule B \\
\begin{tabular}{|c c c|}
479.392	 & 	3.27131	 & 	4.22557	 \\
3.27131	 & 	1242.39	 & 	-0.852684	 \\
4.22557	 & 	-0.852684	 & 	1647.37
\end{tabular}

\vtab
 EingenVectors - Molecule B     \\
\begin{tabular}{|c c c|}
0.999984	 & 	-0.00429123	 & 	-0.00362083	 \\
-0.00429871	 & 	-0.999989	 & 	-0.0020607	 \\
0.00361195	 & 	-0.00207623	 & 	0.999991
\end{tabular}

\vtab
 EingenValues - Molecule B     \\
\begin{tabular}{|c c c|}
479.363	 & 	1242.41	 & 	1647.39	 \\
\end{tabular}

\end{center}
\end{multicols}

\vtab[-5mm]
\begin{tabular}{*{2}{m{0.38\textwidth}}}
\begin{center}
\textcolor{NavyBlue}{\Large Different}
\end{center}
&
\begin{center}
\includegraphics[height=6.5cm]{../Comparisons/Vectors/inertia_tensor_of_3NFAACn_and_4NFAACa.png}
\end{center}
\end{tabular}

 \newpage

\vtab[-3cm]
\begin{center}
{\large FireTest \tab Número 424}
\end{center}
\begin{multicols}{2}
\begin{center}

Molecule A \
3NFAACn

\includegraphics[width=6cm]{../Comparisons/ImagesFromVMD/3NFAACn.png}

Inertia Tensor - Molecule A \\
\begin{tabular}{|c c c|}
531.896	 & 	3.78027	 & 	-13.1151	 \\
3.78027	 & 	1353.2	 & 	-7.47403	 \\
-13.1151	 & 	-7.47403	 & 	912.989
\end{tabular}

\vtab
 EingenVectors - Molecule A     \\
\begin{tabular}{|c c c|}
0.999403	 & 	-0.00428573	 & 	0.0342679	 \\
0.0341891	 & 	-0.0172718	 & 	-0.999266	 \\
0.00487445	 & 	0.999842	 & 	-0.017115
\end{tabular}

\vtab
 EingenValues - Molecule A     \\
\begin{tabular}{|c c c|}
531.43	 & 	913.309	 & 	1353.35	 \\
\end{tabular}
\columnbreak

Molecule B \
4NFAACb

\includegraphics[width=6cm]{../Comparisons/ImagesFromVMD/4NFAACb.png}

Inertia Tensor - Molecule B \\
\begin{tabular}{|c c c|}
479.338	 & 	3.27331	 & 	-4.22553	 \\
3.27331	 & 	1242.4	 & 	0.852083	 \\
-4.22553	 & 	0.852083	 & 	1647.3
\end{tabular}

\vtab
 EingenVectors - Molecule B     \\
\begin{tabular}{|c c c|}
0.999984	 & 	-0.00429353	 & 	0.00362086	 \\
-0.004301	 & 	-0.999989	 & 	0.00205959	 \\
-0.00361198	 & 	0.00207513	 & 	0.999991
\end{tabular}

\vtab
 EingenValues - Molecule B     \\
\begin{tabular}{|c c c|}
479.308	 & 	1242.41	 & 	1647.31	 \\
\end{tabular}

\end{center}
\end{multicols}

\vtab[-5mm]
\begin{tabular}{*{2}{m{0.38\textwidth}}}
\begin{center}
\textcolor{NavyBlue}{\Large Different}
\end{center}
&
\begin{center}
\includegraphics[height=6.5cm]{../Comparisons/Vectors/inertia_tensor_of_3NFAACn_and_4NFAACb.png}
\end{center}
\end{tabular}

 \newpage

\vtab[-3cm]
\begin{center}
{\large FireTest \tab Número 425}
\end{center}
\begin{multicols}{2}
\begin{center}

Molecule A \
3NFAACn

\includegraphics[width=6cm]{../Comparisons/ImagesFromVMD/3NFAACn.png}

Inertia Tensor - Molecule A \\
\begin{tabular}{|c c c|}
531.896	 & 	3.78027	 & 	-13.1151	 \\
3.78027	 & 	1353.2	 & 	-7.47403	 \\
-13.1151	 & 	-7.47403	 & 	912.989
\end{tabular}

\vtab
 EingenVectors - Molecule A     \\
\begin{tabular}{|c c c|}
0.999403	 & 	-0.00428573	 & 	0.0342679	 \\
0.0341891	 & 	-0.0172718	 & 	-0.999266	 \\
0.00487445	 & 	0.999842	 & 	-0.017115
\end{tabular}

\vtab
 EingenValues - Molecule A     \\
\begin{tabular}{|c c c|}
531.43	 & 	913.309	 & 	1353.35	 \\
\end{tabular}
\columnbreak

Molecule B \
4NFAACc

\includegraphics[width=6cm]{../Comparisons/ImagesFromVMD/4NFAACc.png}

Inertia Tensor - Molecule B \\
\begin{tabular}{|c c c|}
482.067	 & 	-5.39474	 & 	-1.35857	 \\
-5.39474	 & 	1240.3	 & 	-2.54035	 \\
-1.35857	 & 	-2.54035	 & 	1647.06
\end{tabular}

\vtab
 EingenVectors - Molecule B     \\
\begin{tabular}{|c c c|}
-0.999974	 & 	-0.00711826	 & 	-0.0011816	 \\
0.00712547	 & 	-0.999955	 & 	-0.00622156	 \\
-0.00113726	 & 	-0.00622982	 & 	0.99998
\end{tabular}

\vtab
 EingenValues - Molecule B     \\
\begin{tabular}{|c c c|}
482.027	 & 	1240.32	 & 	1647.08	 \\
\end{tabular}

\end{center}
\end{multicols}

\vtab[-5mm]
\begin{tabular}{*{2}{m{0.38\textwidth}}}
\begin{center}
\textcolor{NavyBlue}{\Large Different}
\end{center}
&
\begin{center}
\includegraphics[height=6.5cm]{../Comparisons/Vectors/inertia_tensor_of_3NFAACn_and_4NFAACc.png}
\end{center}
\end{tabular}

 \newpage

\vtab[-3cm]
\begin{center}
{\large FireTest \tab Número 426}
\end{center}
\begin{multicols}{2}
\begin{center}

Molecule A \
3NFAACn

\includegraphics[width=6cm]{../Comparisons/ImagesFromVMD/3NFAACn.png}

Inertia Tensor - Molecule A \\
\begin{tabular}{|c c c|}
531.896	 & 	3.78027	 & 	-13.1151	 \\
3.78027	 & 	1353.2	 & 	-7.47403	 \\
-13.1151	 & 	-7.47403	 & 	912.989
\end{tabular}

\vtab
 EingenVectors - Molecule A     \\
\begin{tabular}{|c c c|}
0.999403	 & 	-0.00428573	 & 	0.0342679	 \\
0.0341891	 & 	-0.0172718	 & 	-0.999266	 \\
0.00487445	 & 	0.999842	 & 	-0.017115
\end{tabular}

\vtab
 EingenValues - Molecule A     \\
\begin{tabular}{|c c c|}
531.43	 & 	913.309	 & 	1353.35	 \\
\end{tabular}
\columnbreak

Molecule B \
4NFAACd

\includegraphics[width=6cm]{../Comparisons/ImagesFromVMD/4NFAACd.png}

Inertia Tensor - Molecule B \\
\begin{tabular}{|c c c|}
491.672	 & 	0.24486	 & 	-3.10016	 \\
0.24486	 & 	1231.15	 & 	2.19965	 \\
-3.10016	 & 	2.19965	 & 	1650.11
\end{tabular}

\vtab
 EingenVectors - Molecule B     \\
\begin{tabular}{|c c c|}
0.999996	 & 	-0.000339081	 & 	0.00267677	 \\
-0.000353124	 & 	-0.999986	 & 	0.00524747	 \\
-0.00267495	 & 	0.0052484	 & 	0.999983
\end{tabular}

\vtab
 EingenValues - Molecule B     \\
\begin{tabular}{|c c c|}
491.663	 & 	1231.14	 & 	1650.13	 \\
\end{tabular}

\end{center}
\end{multicols}

\vtab[-5mm]
\begin{tabular}{*{2}{m{0.38\textwidth}}}
\begin{center}
\textcolor{NavyBlue}{\Large Different}
\end{center}
&
\begin{center}
\includegraphics[height=6.5cm]{../Comparisons/Vectors/inertia_tensor_of_3NFAACn_and_4NFAACd.png}
\end{center}
\end{tabular}

 \newpage

\vtab[-3cm]
\begin{center}
{\large FireTest \tab Número 427}
\end{center}
\begin{multicols}{2}
\begin{center}

Molecule A \
3NFAACn

\includegraphics[width=6cm]{../Comparisons/ImagesFromVMD/3NFAACn.png}

Inertia Tensor - Molecule A \\
\begin{tabular}{|c c c|}
531.896	 & 	3.78027	 & 	-13.1151	 \\
3.78027	 & 	1353.2	 & 	-7.47403	 \\
-13.1151	 & 	-7.47403	 & 	912.989
\end{tabular}

\vtab
 EingenVectors - Molecule A     \\
\begin{tabular}{|c c c|}
0.999403	 & 	-0.00428573	 & 	0.0342679	 \\
0.0341891	 & 	-0.0172718	 & 	-0.999266	 \\
0.00487445	 & 	0.999842	 & 	-0.017115
\end{tabular}

\vtab
 EingenValues - Molecule A     \\
\begin{tabular}{|c c c|}
531.43	 & 	913.309	 & 	1353.35	 \\
\end{tabular}
\columnbreak

Molecule B \
4NFAACe

\includegraphics[width=6cm]{../Comparisons/ImagesFromVMD/4NFAACe.png}

Inertia Tensor - Molecule B \\
\begin{tabular}{|c c c|}
489.025	 & 	-0.430035	 & 	3.98876	 \\
-0.430035	 & 	1233.71	 & 	-2.06505	 \\
3.98876	 & 	-2.06505	 & 	1641.79
\end{tabular}

\vtab
 EingenVectors - Molecule B     \\
\begin{tabular}{|c c c|}
0.999994	 & 	0.000567863	 & 	-0.00345908	 \\
0.000550336	 & 	-0.999987	 & 	-0.00506565	 \\
0.00346192	 & 	-0.00506372	 & 	0.999981
\end{tabular}

\vtab
 EingenValues - Molecule B     \\
\begin{tabular}{|c c c|}
489.011	 & 	1233.7	 & 	1641.81	 \\
\end{tabular}

\end{center}
\end{multicols}

\vtab[-5mm]
\begin{tabular}{*{2}{m{0.38\textwidth}}}
\begin{center}
\textcolor{NavyBlue}{\Large Different}
\end{center}
&
\begin{center}
\includegraphics[height=6.5cm]{../Comparisons/Vectors/inertia_tensor_of_3NFAACn_and_4NFAACe.png}
\end{center}
\end{tabular}

 \newpage

\vtab[-3cm]
\begin{center}
{\large FireTest \tab Número 428}
\end{center}
\begin{multicols}{2}
\begin{center}

Molecule A \
3NFAACn

\includegraphics[width=6cm]{../Comparisons/ImagesFromVMD/3NFAACn.png}

Inertia Tensor - Molecule A \\
\begin{tabular}{|c c c|}
531.896	 & 	3.78027	 & 	-13.1151	 \\
3.78027	 & 	1353.2	 & 	-7.47403	 \\
-13.1151	 & 	-7.47403	 & 	912.989
\end{tabular}

\vtab
 EingenVectors - Molecule A     \\
\begin{tabular}{|c c c|}
0.999403	 & 	-0.00428573	 & 	0.0342679	 \\
0.0341891	 & 	-0.0172718	 & 	-0.999266	 \\
0.00487445	 & 	0.999842	 & 	-0.017115
\end{tabular}

\vtab
 EingenValues - Molecule A     \\
\begin{tabular}{|c c c|}
531.43	 & 	913.309	 & 	1353.35	 \\
\end{tabular}
\columnbreak

Molecule B \
4NFAACf

\includegraphics[width=6cm]{../Comparisons/ImagesFromVMD/4NFAACf.png}

Inertia Tensor - Molecule B \\
\begin{tabular}{|c c c|}
509.683	 & 	2.80651	 & 	-1.91422	 \\
2.80651	 & 	1219.11	 & 	2.66132	 \\
-1.91422	 & 	2.66132	 & 	1681.17
\end{tabular}

\vtab
 EingenVectors - Molecule B     \\
\begin{tabular}{|c c c|}
-0.999991	 & 	0.00396206	 & 	-0.00164298	 \\
-0.00397143	 & 	-0.999976	 & 	0.0057431	 \\
-0.00162019	 & 	0.00574957	 & 	0.999982
\end{tabular}

\vtab
 EingenValues - Molecule B     \\
\begin{tabular}{|c c c|}
509.668	 & 	1219.11	 & 	1681.18	 \\
\end{tabular}

\end{center}
\end{multicols}

\vtab[-5mm]
\begin{tabular}{*{2}{m{0.38\textwidth}}}
\begin{center}
\textcolor{NavyBlue}{\Large Different}
\end{center}
&
\begin{center}
\includegraphics[height=6.5cm]{../Comparisons/Vectors/inertia_tensor_of_3NFAACn_and_4NFAACf.png}
\end{center}
\end{tabular}

 \newpage

\vtab[-3cm]
\begin{center}
{\large FireTest \tab Número 429}
\end{center}
\begin{multicols}{2}
\begin{center}

Molecule A \
3NFAACn

\includegraphics[width=6cm]{../Comparisons/ImagesFromVMD/3NFAACn.png}

Inertia Tensor - Molecule A \\
\begin{tabular}{|c c c|}
531.896	 & 	3.78027	 & 	-13.1151	 \\
3.78027	 & 	1353.2	 & 	-7.47403	 \\
-13.1151	 & 	-7.47403	 & 	912.989
\end{tabular}

\vtab
 EingenVectors - Molecule A     \\
\begin{tabular}{|c c c|}
0.999403	 & 	-0.00428573	 & 	0.0342679	 \\
0.0341891	 & 	-0.0172718	 & 	-0.999266	 \\
0.00487445	 & 	0.999842	 & 	-0.017115
\end{tabular}

\vtab
 EingenValues - Molecule A     \\
\begin{tabular}{|c c c|}
531.43	 & 	913.309	 & 	1353.35	 \\
\end{tabular}
\columnbreak

Molecule B \
4NFAACg

\includegraphics[width=6cm]{../Comparisons/ImagesFromVMD/4NFAACg.png}

Inertia Tensor - Molecule B \\
\begin{tabular}{|c c c|}
513.78	 & 	4.51917	 & 	0.266555	 \\
4.51917	 & 	1208.04	 & 	-1.18628	 \\
0.266555	 & 	-1.18628	 & 	1700.9
\end{tabular}

\vtab
 EingenVectors - Molecule B     \\
\begin{tabular}{|c c c|}
-0.999979	 & 	0.00650929	 & 	0.000231034	 \\
-0.00650983	 & 	-0.999976	 & 	-0.00240351	 \\
0.000215383	 & 	-0.00240496	 & 	0.999997
\end{tabular}

\vtab
 EingenValues - Molecule B     \\
\begin{tabular}{|c c c|}
513.751	 & 	1208.07	 & 	1700.9	 \\
\end{tabular}

\end{center}
\end{multicols}

\vtab[-5mm]
\begin{tabular}{*{2}{m{0.38\textwidth}}}
\begin{center}
\textcolor{NavyBlue}{\Large Different}
\end{center}
&
\begin{center}
\includegraphics[height=6.5cm]{../Comparisons/Vectors/inertia_tensor_of_3NFAACn_and_4NFAACg.png}
\end{center}
\end{tabular}

 \newpage

\vtab[-3cm]
\begin{center}
{\large FireTest \tab Número 430}
\end{center}
\begin{multicols}{2}
\begin{center}

Molecule A \
3NFAACn

\includegraphics[width=6cm]{../Comparisons/ImagesFromVMD/3NFAACn.png}

Inertia Tensor - Molecule A \\
\begin{tabular}{|c c c|}
531.896	 & 	3.78027	 & 	-13.1151	 \\
3.78027	 & 	1353.2	 & 	-7.47403	 \\
-13.1151	 & 	-7.47403	 & 	912.989
\end{tabular}

\vtab
 EingenVectors - Molecule A     \\
\begin{tabular}{|c c c|}
0.999403	 & 	-0.00428573	 & 	0.0342679	 \\
0.0341891	 & 	-0.0172718	 & 	-0.999266	 \\
0.00487445	 & 	0.999842	 & 	-0.017115
\end{tabular}

\vtab
 EingenValues - Molecule A     \\
\begin{tabular}{|c c c|}
531.43	 & 	913.309	 & 	1353.35	 \\
\end{tabular}
\columnbreak

Molecule B \
4NFAACi

\includegraphics[width=6cm]{../Comparisons/ImagesFromVMD/4NFAACi.png}

Inertia Tensor - Molecule B \\
\begin{tabular}{|c c c|}
502.43	 & 	-0.602691	 & 	-4.86988	 \\
-0.602691	 & 	1232.26	 & 	0.407295	 \\
-4.86988	 & 	0.407295	 & 	1676
\end{tabular}

\vtab
 EingenVectors - Molecule B     \\
\begin{tabular}{|c c c|}
0.999991	 & 	0.000823447	 & 	0.00414923	 \\
0.000819608	 & 	-0.999999	 & 	0.00092687	 \\
-0.00414999	 & 	0.000923461	 & 	0.999991
\end{tabular}

\vtab
 EingenValues - Molecule B     \\
\begin{tabular}{|c c c|}
502.409	 & 	1232.26	 & 	1676.02	 \\
\end{tabular}

\end{center}
\end{multicols}

\vtab[-5mm]
\begin{tabular}{*{2}{m{0.38\textwidth}}}
\begin{center}
\textcolor{NavyBlue}{\Large Different}
\end{center}
&
\begin{center}
\includegraphics[height=6.5cm]{../Comparisons/Vectors/inertia_tensor_of_3NFAACn_and_4NFAACi.png}
\end{center}
\end{tabular}

 \newpage

\vtab[-3cm]
\begin{center}
{\large FireTest \tab Número 431}
\end{center}
\begin{multicols}{2}
\begin{center}

Molecule A \
3NFAACn

\includegraphics[width=6cm]{../Comparisons/ImagesFromVMD/3NFAACn.png}

Inertia Tensor - Molecule A \\
\begin{tabular}{|c c c|}
531.896	 & 	3.78027	 & 	-13.1151	 \\
3.78027	 & 	1353.2	 & 	-7.47403	 \\
-13.1151	 & 	-7.47403	 & 	912.989
\end{tabular}

\vtab
 EingenVectors - Molecule A     \\
\begin{tabular}{|c c c|}
0.999403	 & 	-0.00428573	 & 	0.0342679	 \\
0.0341891	 & 	-0.0172718	 & 	-0.999266	 \\
0.00487445	 & 	0.999842	 & 	-0.017115
\end{tabular}

\vtab
 EingenValues - Molecule A     \\
\begin{tabular}{|c c c|}
531.43	 & 	913.309	 & 	1353.35	 \\
\end{tabular}
\columnbreak

Molecule B \
4NFAACj

\includegraphics[width=6cm]{../Comparisons/ImagesFromVMD/4NFAACj.png}

Inertia Tensor - Molecule B \\
\begin{tabular}{|c c c|}
510.047	 & 	9.97005	 & 	-3.6306	 \\
9.97005	 & 	1225.52	 & 	-0.981092	 \\
-3.6306	 & 	-0.981092	 & 	1680.82
\end{tabular}

\vtab
 EingenVectors - Molecule B     \\
\begin{tabular}{|c c c|}
-0.999898	 & 	0.0139264	 & 	-0.00308865	 \\
0.0139195	 & 	0.999901	 & 	0.00226627	 \\
-0.00311991	 & 	-0.00222305	 & 	0.999993
\end{tabular}

\vtab
 EingenValues - Molecule B     \\
\begin{tabular}{|c c c|}
509.897	 & 	1225.65	 & 	1680.83	 \\
\end{tabular}

\end{center}
\end{multicols}

\vtab[-5mm]
\begin{tabular}{*{2}{m{0.38\textwidth}}}
\begin{center}
\textcolor{NavyBlue}{\Large Different}
\end{center}
&
\begin{center}
\includegraphics[height=6.5cm]{../Comparisons/Vectors/inertia_tensor_of_3NFAACn_and_4NFAACj.png}
\end{center}
\end{tabular}

 \newpage

\vtab[-3cm]
\begin{center}
{\large FireTest \tab Número 432}
\end{center}
\begin{multicols}{2}
\begin{center}

Molecule A \
3NFAACn

\includegraphics[width=6cm]{../Comparisons/ImagesFromVMD/3NFAACn.png}

Inertia Tensor - Molecule A \\
\begin{tabular}{|c c c|}
531.896	 & 	3.78027	 & 	-13.1151	 \\
3.78027	 & 	1353.2	 & 	-7.47403	 \\
-13.1151	 & 	-7.47403	 & 	912.989
\end{tabular}

\vtab
 EingenVectors - Molecule A     \\
\begin{tabular}{|c c c|}
0.999403	 & 	-0.00428573	 & 	0.0342679	 \\
0.0341891	 & 	-0.0172718	 & 	-0.999266	 \\
0.00487445	 & 	0.999842	 & 	-0.017115
\end{tabular}

\vtab
 EingenValues - Molecule A     \\
\begin{tabular}{|c c c|}
531.43	 & 	913.309	 & 	1353.35	 \\
\end{tabular}
\columnbreak

Molecule B \
4NFAACl-3

\includegraphics[width=6cm]{../Comparisons/ImagesFromVMD/4NFAACl-3.png}

Inertia Tensor - Molecule B \\
\begin{tabular}{|c c c|}
506.608	 & 	0.709539	 & 	-0.555426	 \\
0.709539	 & 	1222.37	 & 	-2.84005	 \\
-0.555426	 & 	-2.84005	 & 	1678.41
\end{tabular}

\vtab
 EingenVectors - Molecule B     \\
\begin{tabular}{|c c c|}
-0.999999	 & 	0.000989428	 & 	-0.000471595	 \\
-0.000986471	 & 	-0.99998	 & 	-0.00622856	 \\
-0.000477748	 & 	-0.00622809	 & 	0.99998
\end{tabular}

\vtab
 EingenValues - Molecule B     \\
\begin{tabular}{|c c c|}
506.607	 & 	1222.36	 & 	1678.43	 \\
\end{tabular}

\end{center}
\end{multicols}

\vtab[-5mm]
\begin{tabular}{*{2}{m{0.38\textwidth}}}
\begin{center}
\textcolor{NavyBlue}{\Large Different}
\end{center}
&
\begin{center}
\includegraphics[height=6.5cm]{../Comparisons/Vectors/inertia_tensor_of_3NFAACn_and_4NFAACl-3.png}
\end{center}
\end{tabular}

 \newpage

\vtab[-3cm]
\begin{center}
{\large FireTest \tab Número 433}
\end{center}
\begin{multicols}{2}
\begin{center}

Molecule A \
4NFAACa

\includegraphics[width=6cm]{../Comparisons/ImagesFromVMD/4NFAACa.png}

Inertia Tensor - Molecule A \\
\begin{tabular}{|c c c|}
479.392	 & 	3.27131	 & 	4.22557	 \\
3.27131	 & 	1242.39	 & 	-0.852684	 \\
4.22557	 & 	-0.852684	 & 	1647.37
\end{tabular}

\vtab
 EingenVectors - Molecule A     \\
\begin{tabular}{|c c c|}
0.999984	 & 	-0.00429123	 & 	-0.00362083	 \\
-0.00429871	 & 	-0.999989	 & 	-0.0020607	 \\
0.00361195	 & 	-0.00207623	 & 	0.999991
\end{tabular}

\vtab
 EingenValues - Molecule A     \\
\begin{tabular}{|c c c|}
479.363	 & 	1242.41	 & 	1647.39	 \\
\end{tabular}
\columnbreak

Molecule B \
4NFAACb

\includegraphics[width=6cm]{../Comparisons/ImagesFromVMD/4NFAACb.png}

Inertia Tensor - Molecule B \\
\begin{tabular}{|c c c|}
479.338	 & 	3.27331	 & 	-4.22553	 \\
3.27331	 & 	1242.4	 & 	0.852083	 \\
-4.22553	 & 	0.852083	 & 	1647.3
\end{tabular}

\vtab
 EingenVectors - Molecule B     \\
\begin{tabular}{|c c c|}
0.999984	 & 	-0.00429353	 & 	0.00362086	 \\
-0.004301	 & 	-0.999989	 & 	0.00205959	 \\
-0.00361198	 & 	0.00207513	 & 	0.999991
\end{tabular}

\vtab
 EingenValues - Molecule B     \\
\begin{tabular}{|c c c|}
479.308	 & 	1242.41	 & 	1647.31	 \\
\end{tabular}

\end{center}
\end{multicols}

\vtab[-5mm]
\begin{tabular}{*{2}{m{0.38\textwidth}}}
\begin{center}
\textcolor{NavyBlue}{\Large Enantiomers}
\end{center}
&
\begin{center}
\includegraphics[height=6.5cm]{../Comparisons/Vectors/inertia_tensor_of_4NFAACa_and_4NFAACb.png}
\end{center}
\end{tabular}

 \newpage

\vtab[-3cm]
\begin{center}
{\large FireTest \tab Número 434}
\end{center}
\begin{multicols}{2}
\begin{center}

Molecule A \
4NFAACa

\includegraphics[width=6cm]{../Comparisons/ImagesFromVMD/4NFAACa.png}

Inertia Tensor - Molecule A \\
\begin{tabular}{|c c c|}
479.392	 & 	3.27131	 & 	4.22557	 \\
3.27131	 & 	1242.39	 & 	-0.852684	 \\
4.22557	 & 	-0.852684	 & 	1647.37
\end{tabular}

\vtab
 EingenVectors - Molecule A     \\
\begin{tabular}{|c c c|}
0.999984	 & 	-0.00429123	 & 	-0.00362083	 \\
-0.00429871	 & 	-0.999989	 & 	-0.0020607	 \\
0.00361195	 & 	-0.00207623	 & 	0.999991
\end{tabular}

\vtab
 EingenValues - Molecule A     \\
\begin{tabular}{|c c c|}
479.363	 & 	1242.41	 & 	1647.39	 \\
\end{tabular}
\columnbreak

Molecule B \
4NFAACc

\includegraphics[width=6cm]{../Comparisons/ImagesFromVMD/4NFAACc.png}

Inertia Tensor - Molecule B \\
\begin{tabular}{|c c c|}
482.067	 & 	-5.39474	 & 	-1.35857	 \\
-5.39474	 & 	1240.3	 & 	-2.54035	 \\
-1.35857	 & 	-2.54035	 & 	1647.06
\end{tabular}

\vtab
 EingenVectors - Molecule B     \\
\begin{tabular}{|c c c|}
-0.999974	 & 	-0.00711826	 & 	-0.0011816	 \\
0.00712547	 & 	-0.999955	 & 	-0.00622156	 \\
-0.00113726	 & 	-0.00622982	 & 	0.99998
\end{tabular}

\vtab
 EingenValues - Molecule B     \\
\begin{tabular}{|c c c|}
482.027	 & 	1240.32	 & 	1647.08	 \\
\end{tabular}

\end{center}
\end{multicols}

\vtab[-5mm]
\begin{tabular}{*{2}{m{0.38\textwidth}}}
\begin{center}
\textcolor{NavyBlue}{\Large Different}
\end{center}
&
\begin{center}
\includegraphics[height=6.5cm]{../Comparisons/Vectors/inertia_tensor_of_4NFAACa_and_4NFAACc.png}
\end{center}
\end{tabular}

 \newpage

\vtab[-3cm]
\begin{center}
{\large FireTest \tab Número 435}
\end{center}
\begin{multicols}{2}
\begin{center}

Molecule A \
4NFAACa

\includegraphics[width=6cm]{../Comparisons/ImagesFromVMD/4NFAACa.png}

Inertia Tensor - Molecule A \\
\begin{tabular}{|c c c|}
479.392	 & 	3.27131	 & 	4.22557	 \\
3.27131	 & 	1242.39	 & 	-0.852684	 \\
4.22557	 & 	-0.852684	 & 	1647.37
\end{tabular}

\vtab
 EingenVectors - Molecule A     \\
\begin{tabular}{|c c c|}
0.999984	 & 	-0.00429123	 & 	-0.00362083	 \\
-0.00429871	 & 	-0.999989	 & 	-0.0020607	 \\
0.00361195	 & 	-0.00207623	 & 	0.999991
\end{tabular}

\vtab
 EingenValues - Molecule A     \\
\begin{tabular}{|c c c|}
479.363	 & 	1242.41	 & 	1647.39	 \\
\end{tabular}
\columnbreak

Molecule B \
4NFAACd

\includegraphics[width=6cm]{../Comparisons/ImagesFromVMD/4NFAACd.png}

Inertia Tensor - Molecule B \\
\begin{tabular}{|c c c|}
491.672	 & 	0.24486	 & 	-3.10016	 \\
0.24486	 & 	1231.15	 & 	2.19965	 \\
-3.10016	 & 	2.19965	 & 	1650.11
\end{tabular}

\vtab
 EingenVectors - Molecule B     \\
\begin{tabular}{|c c c|}
0.999996	 & 	-0.000339081	 & 	0.00267677	 \\
-0.000353124	 & 	-0.999986	 & 	0.00524747	 \\
-0.00267495	 & 	0.0052484	 & 	0.999983
\end{tabular}

\vtab
 EingenValues - Molecule B     \\
\begin{tabular}{|c c c|}
491.663	 & 	1231.14	 & 	1650.13	 \\
\end{tabular}

\end{center}
\end{multicols}

\vtab[-5mm]
\begin{tabular}{*{2}{m{0.38\textwidth}}}
\begin{center}
\textcolor{NavyBlue}{\Large Different}
\end{center}
&
\begin{center}
\includegraphics[height=6.5cm]{../Comparisons/Vectors/inertia_tensor_of_4NFAACa_and_4NFAACd.png}
\end{center}
\end{tabular}

 \newpage

\vtab[-3cm]
\begin{center}
{\large FireTest \tab Número 436}
\end{center}
\begin{multicols}{2}
\begin{center}

Molecule A \
4NFAACa

\includegraphics[width=6cm]{../Comparisons/ImagesFromVMD/4NFAACa.png}

Inertia Tensor - Molecule A \\
\begin{tabular}{|c c c|}
479.392	 & 	3.27131	 & 	4.22557	 \\
3.27131	 & 	1242.39	 & 	-0.852684	 \\
4.22557	 & 	-0.852684	 & 	1647.37
\end{tabular}

\vtab
 EingenVectors - Molecule A     \\
\begin{tabular}{|c c c|}
0.999984	 & 	-0.00429123	 & 	-0.00362083	 \\
-0.00429871	 & 	-0.999989	 & 	-0.0020607	 \\
0.00361195	 & 	-0.00207623	 & 	0.999991
\end{tabular}

\vtab
 EingenValues - Molecule A     \\
\begin{tabular}{|c c c|}
479.363	 & 	1242.41	 & 	1647.39	 \\
\end{tabular}
\columnbreak

Molecule B \
4NFAACe

\includegraphics[width=6cm]{../Comparisons/ImagesFromVMD/4NFAACe.png}

Inertia Tensor - Molecule B \\
\begin{tabular}{|c c c|}
489.025	 & 	-0.430035	 & 	3.98876	 \\
-0.430035	 & 	1233.71	 & 	-2.06505	 \\
3.98876	 & 	-2.06505	 & 	1641.79
\end{tabular}

\vtab
 EingenVectors - Molecule B     \\
\begin{tabular}{|c c c|}
0.999994	 & 	0.000567863	 & 	-0.00345908	 \\
0.000550336	 & 	-0.999987	 & 	-0.00506565	 \\
0.00346192	 & 	-0.00506372	 & 	0.999981
\end{tabular}

\vtab
 EingenValues - Molecule B     \\
\begin{tabular}{|c c c|}
489.011	 & 	1233.7	 & 	1641.81	 \\
\end{tabular}

\end{center}
\end{multicols}

\vtab[-5mm]
\begin{tabular}{*{2}{m{0.38\textwidth}}}
\begin{center}
\textcolor{NavyBlue}{\Large Different}
\end{center}
&
\begin{center}
\includegraphics[height=6.5cm]{../Comparisons/Vectors/inertia_tensor_of_4NFAACa_and_4NFAACe.png}
\end{center}
\end{tabular}

 \newpage

\vtab[-3cm]
\begin{center}
{\large FireTest \tab Número 437}
\end{center}
\begin{multicols}{2}
\begin{center}

Molecule A \
4NFAACa

\includegraphics[width=6cm]{../Comparisons/ImagesFromVMD/4NFAACa.png}

Inertia Tensor - Molecule A \\
\begin{tabular}{|c c c|}
479.392	 & 	3.27131	 & 	4.22557	 \\
3.27131	 & 	1242.39	 & 	-0.852684	 \\
4.22557	 & 	-0.852684	 & 	1647.37
\end{tabular}

\vtab
 EingenVectors - Molecule A     \\
\begin{tabular}{|c c c|}
0.999984	 & 	-0.00429123	 & 	-0.00362083	 \\
-0.00429871	 & 	-0.999989	 & 	-0.0020607	 \\
0.00361195	 & 	-0.00207623	 & 	0.999991
\end{tabular}

\vtab
 EingenValues - Molecule A     \\
\begin{tabular}{|c c c|}
479.363	 & 	1242.41	 & 	1647.39	 \\
\end{tabular}
\columnbreak

Molecule B \
4NFAACf

\includegraphics[width=6cm]{../Comparisons/ImagesFromVMD/4NFAACf.png}

Inertia Tensor - Molecule B \\
\begin{tabular}{|c c c|}
509.683	 & 	2.80651	 & 	-1.91422	 \\
2.80651	 & 	1219.11	 & 	2.66132	 \\
-1.91422	 & 	2.66132	 & 	1681.17
\end{tabular}

\vtab
 EingenVectors - Molecule B     \\
\begin{tabular}{|c c c|}
-0.999991	 & 	0.00396206	 & 	-0.00164298	 \\
-0.00397143	 & 	-0.999976	 & 	0.0057431	 \\
-0.00162019	 & 	0.00574957	 & 	0.999982
\end{tabular}

\vtab
 EingenValues - Molecule B     \\
\begin{tabular}{|c c c|}
509.668	 & 	1219.11	 & 	1681.18	 \\
\end{tabular}

\end{center}
\end{multicols}

\vtab[-5mm]
\begin{tabular}{*{2}{m{0.38\textwidth}}}
\begin{center}
\textcolor{NavyBlue}{\Large Different}
\end{center}
&
\begin{center}
\includegraphics[height=6.5cm]{../Comparisons/Vectors/inertia_tensor_of_4NFAACa_and_4NFAACf.png}
\end{center}
\end{tabular}

 \newpage

\vtab[-3cm]
\begin{center}
{\large FireTest \tab Número 438}
\end{center}
\begin{multicols}{2}
\begin{center}

Molecule A \
4NFAACa

\includegraphics[width=6cm]{../Comparisons/ImagesFromVMD/4NFAACa.png}

Inertia Tensor - Molecule A \\
\begin{tabular}{|c c c|}
479.392	 & 	3.27131	 & 	4.22557	 \\
3.27131	 & 	1242.39	 & 	-0.852684	 \\
4.22557	 & 	-0.852684	 & 	1647.37
\end{tabular}

\vtab
 EingenVectors - Molecule A     \\
\begin{tabular}{|c c c|}
0.999984	 & 	-0.00429123	 & 	-0.00362083	 \\
-0.00429871	 & 	-0.999989	 & 	-0.0020607	 \\
0.00361195	 & 	-0.00207623	 & 	0.999991
\end{tabular}

\vtab
 EingenValues - Molecule A     \\
\begin{tabular}{|c c c|}
479.363	 & 	1242.41	 & 	1647.39	 \\
\end{tabular}
\columnbreak

Molecule B \
4NFAACg

\includegraphics[width=6cm]{../Comparisons/ImagesFromVMD/4NFAACg.png}

Inertia Tensor - Molecule B \\
\begin{tabular}{|c c c|}
513.78	 & 	4.51917	 & 	0.266555	 \\
4.51917	 & 	1208.04	 & 	-1.18628	 \\
0.266555	 & 	-1.18628	 & 	1700.9
\end{tabular}

\vtab
 EingenVectors - Molecule B     \\
\begin{tabular}{|c c c|}
-0.999979	 & 	0.00650929	 & 	0.000231034	 \\
-0.00650983	 & 	-0.999976	 & 	-0.00240351	 \\
0.000215383	 & 	-0.00240496	 & 	0.999997
\end{tabular}

\vtab
 EingenValues - Molecule B     \\
\begin{tabular}{|c c c|}
513.751	 & 	1208.07	 & 	1700.9	 \\
\end{tabular}

\end{center}
\end{multicols}

\vtab[-5mm]
\begin{tabular}{*{2}{m{0.38\textwidth}}}
\begin{center}
\textcolor{NavyBlue}{\Large Different}
\end{center}
&
\begin{center}
\includegraphics[height=6.5cm]{../Comparisons/Vectors/inertia_tensor_of_4NFAACa_and_4NFAACg.png}
\end{center}
\end{tabular}

 \newpage

\vtab[-3cm]
\begin{center}
{\large FireTest \tab Número 439}
\end{center}
\begin{multicols}{2}
\begin{center}

Molecule A \
4NFAACa

\includegraphics[width=6cm]{../Comparisons/ImagesFromVMD/4NFAACa.png}

Inertia Tensor - Molecule A \\
\begin{tabular}{|c c c|}
479.392	 & 	3.27131	 & 	4.22557	 \\
3.27131	 & 	1242.39	 & 	-0.852684	 \\
4.22557	 & 	-0.852684	 & 	1647.37
\end{tabular}

\vtab
 EingenVectors - Molecule A     \\
\begin{tabular}{|c c c|}
0.999984	 & 	-0.00429123	 & 	-0.00362083	 \\
-0.00429871	 & 	-0.999989	 & 	-0.0020607	 \\
0.00361195	 & 	-0.00207623	 & 	0.999991
\end{tabular}

\vtab
 EingenValues - Molecule A     \\
\begin{tabular}{|c c c|}
479.363	 & 	1242.41	 & 	1647.39	 \\
\end{tabular}
\columnbreak

Molecule B \
4NFAACi

\includegraphics[width=6cm]{../Comparisons/ImagesFromVMD/4NFAACi.png}

Inertia Tensor - Molecule B \\
\begin{tabular}{|c c c|}
502.43	 & 	-0.602691	 & 	-4.86988	 \\
-0.602691	 & 	1232.26	 & 	0.407295	 \\
-4.86988	 & 	0.407295	 & 	1676
\end{tabular}

\vtab
 EingenVectors - Molecule B     \\
\begin{tabular}{|c c c|}
0.999991	 & 	0.000823447	 & 	0.00414923	 \\
0.000819608	 & 	-0.999999	 & 	0.00092687	 \\
-0.00414999	 & 	0.000923461	 & 	0.999991
\end{tabular}

\vtab
 EingenValues - Molecule B     \\
\begin{tabular}{|c c c|}
502.409	 & 	1232.26	 & 	1676.02	 \\
\end{tabular}

\end{center}
\end{multicols}

\vtab[-5mm]
\begin{tabular}{*{2}{m{0.38\textwidth}}}
\begin{center}
\textcolor{NavyBlue}{\Large Different}
\end{center}
&
\begin{center}
\includegraphics[height=6.5cm]{../Comparisons/Vectors/inertia_tensor_of_4NFAACa_and_4NFAACi.png}
\end{center}
\end{tabular}

 \newpage

\vtab[-3cm]
\begin{center}
{\large FireTest \tab Número 440}
\end{center}
\begin{multicols}{2}
\begin{center}

Molecule A \
4NFAACa

\includegraphics[width=6cm]{../Comparisons/ImagesFromVMD/4NFAACa.png}

Inertia Tensor - Molecule A \\
\begin{tabular}{|c c c|}
479.392	 & 	3.27131	 & 	4.22557	 \\
3.27131	 & 	1242.39	 & 	-0.852684	 \\
4.22557	 & 	-0.852684	 & 	1647.37
\end{tabular}

\vtab
 EingenVectors - Molecule A     \\
\begin{tabular}{|c c c|}
0.999984	 & 	-0.00429123	 & 	-0.00362083	 \\
-0.00429871	 & 	-0.999989	 & 	-0.0020607	 \\
0.00361195	 & 	-0.00207623	 & 	0.999991
\end{tabular}

\vtab
 EingenValues - Molecule A     \\
\begin{tabular}{|c c c|}
479.363	 & 	1242.41	 & 	1647.39	 \\
\end{tabular}
\columnbreak

Molecule B \
4NFAACj

\includegraphics[width=6cm]{../Comparisons/ImagesFromVMD/4NFAACj.png}

Inertia Tensor - Molecule B \\
\begin{tabular}{|c c c|}
510.047	 & 	9.97005	 & 	-3.6306	 \\
9.97005	 & 	1225.52	 & 	-0.981092	 \\
-3.6306	 & 	-0.981092	 & 	1680.82
\end{tabular}

\vtab
 EingenVectors - Molecule B     \\
\begin{tabular}{|c c c|}
-0.999898	 & 	0.0139264	 & 	-0.00308865	 \\
0.0139195	 & 	0.999901	 & 	0.00226627	 \\
-0.00311991	 & 	-0.00222305	 & 	0.999993
\end{tabular}

\vtab
 EingenValues - Molecule B     \\
\begin{tabular}{|c c c|}
509.897	 & 	1225.65	 & 	1680.83	 \\
\end{tabular}

\end{center}
\end{multicols}

\vtab[-5mm]
\begin{tabular}{*{2}{m{0.38\textwidth}}}
\begin{center}
\textcolor{NavyBlue}{\Large Different}
\end{center}
&
\begin{center}
\includegraphics[height=6.5cm]{../Comparisons/Vectors/inertia_tensor_of_4NFAACa_and_4NFAACj.png}
\end{center}
\end{tabular}

 \newpage

\vtab[-3cm]
\begin{center}
{\large FireTest \tab Número 441}
\end{center}
\begin{multicols}{2}
\begin{center}

Molecule A \
4NFAACa

\includegraphics[width=6cm]{../Comparisons/ImagesFromVMD/4NFAACa.png}

Inertia Tensor - Molecule A \\
\begin{tabular}{|c c c|}
479.392	 & 	3.27131	 & 	4.22557	 \\
3.27131	 & 	1242.39	 & 	-0.852684	 \\
4.22557	 & 	-0.852684	 & 	1647.37
\end{tabular}

\vtab
 EingenVectors - Molecule A     \\
\begin{tabular}{|c c c|}
0.999984	 & 	-0.00429123	 & 	-0.00362083	 \\
-0.00429871	 & 	-0.999989	 & 	-0.0020607	 \\
0.00361195	 & 	-0.00207623	 & 	0.999991
\end{tabular}

\vtab
 EingenValues - Molecule A     \\
\begin{tabular}{|c c c|}
479.363	 & 	1242.41	 & 	1647.39	 \\
\end{tabular}
\columnbreak

Molecule B \
4NFAACl-3

\includegraphics[width=6cm]{../Comparisons/ImagesFromVMD/4NFAACl-3.png}

Inertia Tensor - Molecule B \\
\begin{tabular}{|c c c|}
506.608	 & 	0.709539	 & 	-0.555426	 \\
0.709539	 & 	1222.37	 & 	-2.84005	 \\
-0.555426	 & 	-2.84005	 & 	1678.41
\end{tabular}

\vtab
 EingenVectors - Molecule B     \\
\begin{tabular}{|c c c|}
-0.999999	 & 	0.000989428	 & 	-0.000471595	 \\
-0.000986471	 & 	-0.99998	 & 	-0.00622856	 \\
-0.000477748	 & 	-0.00622809	 & 	0.99998
\end{tabular}

\vtab
 EingenValues - Molecule B     \\
\begin{tabular}{|c c c|}
506.607	 & 	1222.36	 & 	1678.43	 \\
\end{tabular}

\end{center}
\end{multicols}

\vtab[-5mm]
\begin{tabular}{*{2}{m{0.38\textwidth}}}
\begin{center}
\textcolor{NavyBlue}{\Large Different}
\end{center}
&
\begin{center}
\includegraphics[height=6.5cm]{../Comparisons/Vectors/inertia_tensor_of_4NFAACa_and_4NFAACl-3.png}
\end{center}
\end{tabular}

 \newpage

\vtab[-3cm]
\begin{center}
{\large FireTest \tab Número 442}
\end{center}
\begin{multicols}{2}
\begin{center}

Molecule A \
4NFAACb

\includegraphics[width=6cm]{../Comparisons/ImagesFromVMD/4NFAACb.png}

Inertia Tensor - Molecule A \\
\begin{tabular}{|c c c|}
479.338	 & 	3.27331	 & 	-4.22553	 \\
3.27331	 & 	1242.4	 & 	0.852083	 \\
-4.22553	 & 	0.852083	 & 	1647.3
\end{tabular}

\vtab
 EingenVectors - Molecule A     \\
\begin{tabular}{|c c c|}
0.999984	 & 	-0.00429353	 & 	0.00362086	 \\
-0.004301	 & 	-0.999989	 & 	0.00205959	 \\
-0.00361198	 & 	0.00207513	 & 	0.999991
\end{tabular}

\vtab
 EingenValues - Molecule A     \\
\begin{tabular}{|c c c|}
479.308	 & 	1242.41	 & 	1647.31	 \\
\end{tabular}
\columnbreak

Molecule B \
4NFAACc

\includegraphics[width=6cm]{../Comparisons/ImagesFromVMD/4NFAACc.png}

Inertia Tensor - Molecule B \\
\begin{tabular}{|c c c|}
482.067	 & 	-5.39474	 & 	-1.35857	 \\
-5.39474	 & 	1240.3	 & 	-2.54035	 \\
-1.35857	 & 	-2.54035	 & 	1647.06
\end{tabular}

\vtab
 EingenVectors - Molecule B     \\
\begin{tabular}{|c c c|}
-0.999974	 & 	-0.00711826	 & 	-0.0011816	 \\
0.00712547	 & 	-0.999955	 & 	-0.00622156	 \\
-0.00113726	 & 	-0.00622982	 & 	0.99998
\end{tabular}

\vtab
 EingenValues - Molecule B     \\
\begin{tabular}{|c c c|}
482.027	 & 	1240.32	 & 	1647.08	 \\
\end{tabular}

\end{center}
\end{multicols}

\vtab[-5mm]
\begin{tabular}{*{2}{m{0.38\textwidth}}}
\begin{center}
\textcolor{NavyBlue}{\Large Different}
\end{center}
&
\begin{center}
\includegraphics[height=6.5cm]{../Comparisons/Vectors/inertia_tensor_of_4NFAACb_and_4NFAACc.png}
\end{center}
\end{tabular}

 \newpage

\vtab[-3cm]
\begin{center}
{\large FireTest \tab Número 443}
\end{center}
\begin{multicols}{2}
\begin{center}

Molecule A \
4NFAACb

\includegraphics[width=6cm]{../Comparisons/ImagesFromVMD/4NFAACb.png}

Inertia Tensor - Molecule A \\
\begin{tabular}{|c c c|}
479.338	 & 	3.27331	 & 	-4.22553	 \\
3.27331	 & 	1242.4	 & 	0.852083	 \\
-4.22553	 & 	0.852083	 & 	1647.3
\end{tabular}

\vtab
 EingenVectors - Molecule A     \\
\begin{tabular}{|c c c|}
0.999984	 & 	-0.00429353	 & 	0.00362086	 \\
-0.004301	 & 	-0.999989	 & 	0.00205959	 \\
-0.00361198	 & 	0.00207513	 & 	0.999991
\end{tabular}

\vtab
 EingenValues - Molecule A     \\
\begin{tabular}{|c c c|}
479.308	 & 	1242.41	 & 	1647.31	 \\
\end{tabular}
\columnbreak

Molecule B \
4NFAACd

\includegraphics[width=6cm]{../Comparisons/ImagesFromVMD/4NFAACd.png}

Inertia Tensor - Molecule B \\
\begin{tabular}{|c c c|}
491.672	 & 	0.24486	 & 	-3.10016	 \\
0.24486	 & 	1231.15	 & 	2.19965	 \\
-3.10016	 & 	2.19965	 & 	1650.11
\end{tabular}

\vtab
 EingenVectors - Molecule B     \\
\begin{tabular}{|c c c|}
0.999996	 & 	-0.000339081	 & 	0.00267677	 \\
-0.000353124	 & 	-0.999986	 & 	0.00524747	 \\
-0.00267495	 & 	0.0052484	 & 	0.999983
\end{tabular}

\vtab
 EingenValues - Molecule B     \\
\begin{tabular}{|c c c|}
491.663	 & 	1231.14	 & 	1650.13	 \\
\end{tabular}

\end{center}
\end{multicols}

\vtab[-5mm]
\begin{tabular}{*{2}{m{0.38\textwidth}}}
\begin{center}
\textcolor{NavyBlue}{\Large Different}
\end{center}
&
\begin{center}
\includegraphics[height=6.5cm]{../Comparisons/Vectors/inertia_tensor_of_4NFAACb_and_4NFAACd.png}
\end{center}
\end{tabular}

 \newpage

\vtab[-3cm]
\begin{center}
{\large FireTest \tab Número 444}
\end{center}
\begin{multicols}{2}
\begin{center}

Molecule A \
4NFAACb

\includegraphics[width=6cm]{../Comparisons/ImagesFromVMD/4NFAACb.png}

Inertia Tensor - Molecule A \\
\begin{tabular}{|c c c|}
479.338	 & 	3.27331	 & 	-4.22553	 \\
3.27331	 & 	1242.4	 & 	0.852083	 \\
-4.22553	 & 	0.852083	 & 	1647.3
\end{tabular}

\vtab
 EingenVectors - Molecule A     \\
\begin{tabular}{|c c c|}
0.999984	 & 	-0.00429353	 & 	0.00362086	 \\
-0.004301	 & 	-0.999989	 & 	0.00205959	 \\
-0.00361198	 & 	0.00207513	 & 	0.999991
\end{tabular}

\vtab
 EingenValues - Molecule A     \\
\begin{tabular}{|c c c|}
479.308	 & 	1242.41	 & 	1647.31	 \\
\end{tabular}
\columnbreak

Molecule B \
4NFAACe

\includegraphics[width=6cm]{../Comparisons/ImagesFromVMD/4NFAACe.png}

Inertia Tensor - Molecule B \\
\begin{tabular}{|c c c|}
489.025	 & 	-0.430035	 & 	3.98876	 \\
-0.430035	 & 	1233.71	 & 	-2.06505	 \\
3.98876	 & 	-2.06505	 & 	1641.79
\end{tabular}

\vtab
 EingenVectors - Molecule B     \\
\begin{tabular}{|c c c|}
0.999994	 & 	0.000567863	 & 	-0.00345908	 \\
0.000550336	 & 	-0.999987	 & 	-0.00506565	 \\
0.00346192	 & 	-0.00506372	 & 	0.999981
\end{tabular}

\vtab
 EingenValues - Molecule B     \\
\begin{tabular}{|c c c|}
489.011	 & 	1233.7	 & 	1641.81	 \\
\end{tabular}

\end{center}
\end{multicols}

\vtab[-5mm]
\begin{tabular}{*{2}{m{0.38\textwidth}}}
\begin{center}
\textcolor{NavyBlue}{\Large Different}
\end{center}
&
\begin{center}
\includegraphics[height=6.5cm]{../Comparisons/Vectors/inertia_tensor_of_4NFAACb_and_4NFAACe.png}
\end{center}
\end{tabular}

 \newpage

\vtab[-3cm]
\begin{center}
{\large FireTest \tab Número 445}
\end{center}
\begin{multicols}{2}
\begin{center}

Molecule A \
4NFAACb

\includegraphics[width=6cm]{../Comparisons/ImagesFromVMD/4NFAACb.png}

Inertia Tensor - Molecule A \\
\begin{tabular}{|c c c|}
479.338	 & 	3.27331	 & 	-4.22553	 \\
3.27331	 & 	1242.4	 & 	0.852083	 \\
-4.22553	 & 	0.852083	 & 	1647.3
\end{tabular}

\vtab
 EingenVectors - Molecule A     \\
\begin{tabular}{|c c c|}
0.999984	 & 	-0.00429353	 & 	0.00362086	 \\
-0.004301	 & 	-0.999989	 & 	0.00205959	 \\
-0.00361198	 & 	0.00207513	 & 	0.999991
\end{tabular}

\vtab
 EingenValues - Molecule A     \\
\begin{tabular}{|c c c|}
479.308	 & 	1242.41	 & 	1647.31	 \\
\end{tabular}
\columnbreak

Molecule B \
4NFAACf

\includegraphics[width=6cm]{../Comparisons/ImagesFromVMD/4NFAACf.png}

Inertia Tensor - Molecule B \\
\begin{tabular}{|c c c|}
509.683	 & 	2.80651	 & 	-1.91422	 \\
2.80651	 & 	1219.11	 & 	2.66132	 \\
-1.91422	 & 	2.66132	 & 	1681.17
\end{tabular}

\vtab
 EingenVectors - Molecule B     \\
\begin{tabular}{|c c c|}
-0.999991	 & 	0.00396206	 & 	-0.00164298	 \\
-0.00397143	 & 	-0.999976	 & 	0.0057431	 \\
-0.00162019	 & 	0.00574957	 & 	0.999982
\end{tabular}

\vtab
 EingenValues - Molecule B     \\
\begin{tabular}{|c c c|}
509.668	 & 	1219.11	 & 	1681.18	 \\
\end{tabular}

\end{center}
\end{multicols}

\vtab[-5mm]
\begin{tabular}{*{2}{m{0.38\textwidth}}}
\begin{center}
\textcolor{NavyBlue}{\Large Different}
\end{center}
&
\begin{center}
\includegraphics[height=6.5cm]{../Comparisons/Vectors/inertia_tensor_of_4NFAACb_and_4NFAACf.png}
\end{center}
\end{tabular}

 \newpage

\vtab[-3cm]
\begin{center}
{\large FireTest \tab Número 446}
\end{center}
\begin{multicols}{2}
\begin{center}

Molecule A \
4NFAACb

\includegraphics[width=6cm]{../Comparisons/ImagesFromVMD/4NFAACb.png}

Inertia Tensor - Molecule A \\
\begin{tabular}{|c c c|}
479.338	 & 	3.27331	 & 	-4.22553	 \\
3.27331	 & 	1242.4	 & 	0.852083	 \\
-4.22553	 & 	0.852083	 & 	1647.3
\end{tabular}

\vtab
 EingenVectors - Molecule A     \\
\begin{tabular}{|c c c|}
0.999984	 & 	-0.00429353	 & 	0.00362086	 \\
-0.004301	 & 	-0.999989	 & 	0.00205959	 \\
-0.00361198	 & 	0.00207513	 & 	0.999991
\end{tabular}

\vtab
 EingenValues - Molecule A     \\
\begin{tabular}{|c c c|}
479.308	 & 	1242.41	 & 	1647.31	 \\
\end{tabular}
\columnbreak

Molecule B \
4NFAACg

\includegraphics[width=6cm]{../Comparisons/ImagesFromVMD/4NFAACg.png}

Inertia Tensor - Molecule B \\
\begin{tabular}{|c c c|}
513.78	 & 	4.51917	 & 	0.266555	 \\
4.51917	 & 	1208.04	 & 	-1.18628	 \\
0.266555	 & 	-1.18628	 & 	1700.9
\end{tabular}

\vtab
 EingenVectors - Molecule B     \\
\begin{tabular}{|c c c|}
-0.999979	 & 	0.00650929	 & 	0.000231034	 \\
-0.00650983	 & 	-0.999976	 & 	-0.00240351	 \\
0.000215383	 & 	-0.00240496	 & 	0.999997
\end{tabular}

\vtab
 EingenValues - Molecule B     \\
\begin{tabular}{|c c c|}
513.751	 & 	1208.07	 & 	1700.9	 \\
\end{tabular}

\end{center}
\end{multicols}

\vtab[-5mm]
\begin{tabular}{*{2}{m{0.38\textwidth}}}
\begin{center}
\textcolor{NavyBlue}{\Large Different}
\end{center}
&
\begin{center}
\includegraphics[height=6.5cm]{../Comparisons/Vectors/inertia_tensor_of_4NFAACb_and_4NFAACg.png}
\end{center}
\end{tabular}

 \newpage

\vtab[-3cm]
\begin{center}
{\large FireTest \tab Número 447}
\end{center}
\begin{multicols}{2}
\begin{center}

Molecule A \
4NFAACb

\includegraphics[width=6cm]{../Comparisons/ImagesFromVMD/4NFAACb.png}

Inertia Tensor - Molecule A \\
\begin{tabular}{|c c c|}
479.338	 & 	3.27331	 & 	-4.22553	 \\
3.27331	 & 	1242.4	 & 	0.852083	 \\
-4.22553	 & 	0.852083	 & 	1647.3
\end{tabular}

\vtab
 EingenVectors - Molecule A     \\
\begin{tabular}{|c c c|}
0.999984	 & 	-0.00429353	 & 	0.00362086	 \\
-0.004301	 & 	-0.999989	 & 	0.00205959	 \\
-0.00361198	 & 	0.00207513	 & 	0.999991
\end{tabular}

\vtab
 EingenValues - Molecule A     \\
\begin{tabular}{|c c c|}
479.308	 & 	1242.41	 & 	1647.31	 \\
\end{tabular}
\columnbreak

Molecule B \
4NFAACi

\includegraphics[width=6cm]{../Comparisons/ImagesFromVMD/4NFAACi.png}

Inertia Tensor - Molecule B \\
\begin{tabular}{|c c c|}
502.43	 & 	-0.602691	 & 	-4.86988	 \\
-0.602691	 & 	1232.26	 & 	0.407295	 \\
-4.86988	 & 	0.407295	 & 	1676
\end{tabular}

\vtab
 EingenVectors - Molecule B     \\
\begin{tabular}{|c c c|}
0.999991	 & 	0.000823447	 & 	0.00414923	 \\
0.000819608	 & 	-0.999999	 & 	0.00092687	 \\
-0.00414999	 & 	0.000923461	 & 	0.999991
\end{tabular}

\vtab
 EingenValues - Molecule B     \\
\begin{tabular}{|c c c|}
502.409	 & 	1232.26	 & 	1676.02	 \\
\end{tabular}

\end{center}
\end{multicols}

\vtab[-5mm]
\begin{tabular}{*{2}{m{0.38\textwidth}}}
\begin{center}
\textcolor{NavyBlue}{\Large Different}
\end{center}
&
\begin{center}
\includegraphics[height=6.5cm]{../Comparisons/Vectors/inertia_tensor_of_4NFAACb_and_4NFAACi.png}
\end{center}
\end{tabular}

 \newpage

\vtab[-3cm]
\begin{center}
{\large FireTest \tab Número 448}
\end{center}
\begin{multicols}{2}
\begin{center}

Molecule A \
4NFAACb

\includegraphics[width=6cm]{../Comparisons/ImagesFromVMD/4NFAACb.png}

Inertia Tensor - Molecule A \\
\begin{tabular}{|c c c|}
479.338	 & 	3.27331	 & 	-4.22553	 \\
3.27331	 & 	1242.4	 & 	0.852083	 \\
-4.22553	 & 	0.852083	 & 	1647.3
\end{tabular}

\vtab
 EingenVectors - Molecule A     \\
\begin{tabular}{|c c c|}
0.999984	 & 	-0.00429353	 & 	0.00362086	 \\
-0.004301	 & 	-0.999989	 & 	0.00205959	 \\
-0.00361198	 & 	0.00207513	 & 	0.999991
\end{tabular}

\vtab
 EingenValues - Molecule A     \\
\begin{tabular}{|c c c|}
479.308	 & 	1242.41	 & 	1647.31	 \\
\end{tabular}
\columnbreak

Molecule B \
4NFAACj

\includegraphics[width=6cm]{../Comparisons/ImagesFromVMD/4NFAACj.png}

Inertia Tensor - Molecule B \\
\begin{tabular}{|c c c|}
510.047	 & 	9.97005	 & 	-3.6306	 \\
9.97005	 & 	1225.52	 & 	-0.981092	 \\
-3.6306	 & 	-0.981092	 & 	1680.82
\end{tabular}

\vtab
 EingenVectors - Molecule B     \\
\begin{tabular}{|c c c|}
-0.999898	 & 	0.0139264	 & 	-0.00308865	 \\
0.0139195	 & 	0.999901	 & 	0.00226627	 \\
-0.00311991	 & 	-0.00222305	 & 	0.999993
\end{tabular}

\vtab
 EingenValues - Molecule B     \\
\begin{tabular}{|c c c|}
509.897	 & 	1225.65	 & 	1680.83	 \\
\end{tabular}

\end{center}
\end{multicols}

\vtab[-5mm]
\begin{tabular}{*{2}{m{0.38\textwidth}}}
\begin{center}
\textcolor{NavyBlue}{\Large Different}
\end{center}
&
\begin{center}
\includegraphics[height=6.5cm]{../Comparisons/Vectors/inertia_tensor_of_4NFAACb_and_4NFAACj.png}
\end{center}
\end{tabular}

 \newpage

\vtab[-3cm]
\begin{center}
{\large FireTest \tab Número 449}
\end{center}
\begin{multicols}{2}
\begin{center}

Molecule A \
4NFAACb

\includegraphics[width=6cm]{../Comparisons/ImagesFromVMD/4NFAACb.png}

Inertia Tensor - Molecule A \\
\begin{tabular}{|c c c|}
479.338	 & 	3.27331	 & 	-4.22553	 \\
3.27331	 & 	1242.4	 & 	0.852083	 \\
-4.22553	 & 	0.852083	 & 	1647.3
\end{tabular}

\vtab
 EingenVectors - Molecule A     \\
\begin{tabular}{|c c c|}
0.999984	 & 	-0.00429353	 & 	0.00362086	 \\
-0.004301	 & 	-0.999989	 & 	0.00205959	 \\
-0.00361198	 & 	0.00207513	 & 	0.999991
\end{tabular}

\vtab
 EingenValues - Molecule A     \\
\begin{tabular}{|c c c|}
479.308	 & 	1242.41	 & 	1647.31	 \\
\end{tabular}
\columnbreak

Molecule B \
4NFAACl-3

\includegraphics[width=6cm]{../Comparisons/ImagesFromVMD/4NFAACl-3.png}

Inertia Tensor - Molecule B \\
\begin{tabular}{|c c c|}
506.608	 & 	0.709539	 & 	-0.555426	 \\
0.709539	 & 	1222.37	 & 	-2.84005	 \\
-0.555426	 & 	-2.84005	 & 	1678.41
\end{tabular}

\vtab
 EingenVectors - Molecule B     \\
\begin{tabular}{|c c c|}
-0.999999	 & 	0.000989428	 & 	-0.000471595	 \\
-0.000986471	 & 	-0.99998	 & 	-0.00622856	 \\
-0.000477748	 & 	-0.00622809	 & 	0.99998
\end{tabular}

\vtab
 EingenValues - Molecule B     \\
\begin{tabular}{|c c c|}
506.607	 & 	1222.36	 & 	1678.43	 \\
\end{tabular}

\end{center}
\end{multicols}

\vtab[-5mm]
\begin{tabular}{*{2}{m{0.38\textwidth}}}
\begin{center}
\textcolor{NavyBlue}{\Large Different}
\end{center}
&
\begin{center}
\includegraphics[height=6.5cm]{../Comparisons/Vectors/inertia_tensor_of_4NFAACb_and_4NFAACl-3.png}
\end{center}
\end{tabular}

 \newpage

\vtab[-3cm]
\begin{center}
{\large FireTest \tab Número 450}
\end{center}
\begin{multicols}{2}
\begin{center}

Molecule A \
4NFAACc

\includegraphics[width=6cm]{../Comparisons/ImagesFromVMD/4NFAACc.png}

Inertia Tensor - Molecule A \\
\begin{tabular}{|c c c|}
482.067	 & 	-5.39474	 & 	-1.35857	 \\
-5.39474	 & 	1240.3	 & 	-2.54035	 \\
-1.35857	 & 	-2.54035	 & 	1647.06
\end{tabular}

\vtab
 EingenVectors - Molecule A     \\
\begin{tabular}{|c c c|}
-0.999974	 & 	-0.00711826	 & 	-0.0011816	 \\
0.00712547	 & 	-0.999955	 & 	-0.00622156	 \\
-0.00113726	 & 	-0.00622982	 & 	0.99998
\end{tabular}

\vtab
 EingenValues - Molecule A     \\
\begin{tabular}{|c c c|}
482.027	 & 	1240.32	 & 	1647.08	 \\
\end{tabular}
\columnbreak

Molecule B \
4NFAACd

\includegraphics[width=6cm]{../Comparisons/ImagesFromVMD/4NFAACd.png}

Inertia Tensor - Molecule B \\
\begin{tabular}{|c c c|}
491.672	 & 	0.24486	 & 	-3.10016	 \\
0.24486	 & 	1231.15	 & 	2.19965	 \\
-3.10016	 & 	2.19965	 & 	1650.11
\end{tabular}

\vtab
 EingenVectors - Molecule B     \\
\begin{tabular}{|c c c|}
0.999996	 & 	-0.000339081	 & 	0.00267677	 \\
-0.000353124	 & 	-0.999986	 & 	0.00524747	 \\
-0.00267495	 & 	0.0052484	 & 	0.999983
\end{tabular}

\vtab
 EingenValues - Molecule B     \\
\begin{tabular}{|c c c|}
491.663	 & 	1231.14	 & 	1650.13	 \\
\end{tabular}

\end{center}
\end{multicols}

\vtab[-5mm]
\begin{tabular}{*{2}{m{0.38\textwidth}}}
\begin{center}
\textcolor{NavyBlue}{\Large Different}
\end{center}
&
\begin{center}
\includegraphics[height=6.5cm]{../Comparisons/Vectors/inertia_tensor_of_4NFAACc_and_4NFAACd.png}
\end{center}
\end{tabular}

 \newpage

\vtab[-3cm]
\begin{center}
{\large FireTest \tab Número 451}
\end{center}
\begin{multicols}{2}
\begin{center}

Molecule A \
4NFAACc

\includegraphics[width=6cm]{../Comparisons/ImagesFromVMD/4NFAACc.png}

Inertia Tensor - Molecule A \\
\begin{tabular}{|c c c|}
482.067	 & 	-5.39474	 & 	-1.35857	 \\
-5.39474	 & 	1240.3	 & 	-2.54035	 \\
-1.35857	 & 	-2.54035	 & 	1647.06
\end{tabular}

\vtab
 EingenVectors - Molecule A     \\
\begin{tabular}{|c c c|}
-0.999974	 & 	-0.00711826	 & 	-0.0011816	 \\
0.00712547	 & 	-0.999955	 & 	-0.00622156	 \\
-0.00113726	 & 	-0.00622982	 & 	0.99998
\end{tabular}

\vtab
 EingenValues - Molecule A     \\
\begin{tabular}{|c c c|}
482.027	 & 	1240.32	 & 	1647.08	 \\
\end{tabular}
\columnbreak

Molecule B \
4NFAACe

\includegraphics[width=6cm]{../Comparisons/ImagesFromVMD/4NFAACe.png}

Inertia Tensor - Molecule B \\
\begin{tabular}{|c c c|}
489.025	 & 	-0.430035	 & 	3.98876	 \\
-0.430035	 & 	1233.71	 & 	-2.06505	 \\
3.98876	 & 	-2.06505	 & 	1641.79
\end{tabular}

\vtab
 EingenVectors - Molecule B     \\
\begin{tabular}{|c c c|}
0.999994	 & 	0.000567863	 & 	-0.00345908	 \\
0.000550336	 & 	-0.999987	 & 	-0.00506565	 \\
0.00346192	 & 	-0.00506372	 & 	0.999981
\end{tabular}

\vtab
 EingenValues - Molecule B     \\
\begin{tabular}{|c c c|}
489.011	 & 	1233.7	 & 	1641.81	 \\
\end{tabular}

\end{center}
\end{multicols}

\vtab[-5mm]
\begin{tabular}{*{2}{m{0.38\textwidth}}}
\begin{center}
\textcolor{NavyBlue}{\Large Different}
\end{center}
&
\begin{center}
\includegraphics[height=6.5cm]{../Comparisons/Vectors/inertia_tensor_of_4NFAACc_and_4NFAACe.png}
\end{center}
\end{tabular}

 \newpage

\vtab[-3cm]
\begin{center}
{\large FireTest \tab Número 452}
\end{center}
\begin{multicols}{2}
\begin{center}

Molecule A \
4NFAACc

\includegraphics[width=6cm]{../Comparisons/ImagesFromVMD/4NFAACc.png}

Inertia Tensor - Molecule A \\
\begin{tabular}{|c c c|}
482.067	 & 	-5.39474	 & 	-1.35857	 \\
-5.39474	 & 	1240.3	 & 	-2.54035	 \\
-1.35857	 & 	-2.54035	 & 	1647.06
\end{tabular}

\vtab
 EingenVectors - Molecule A     \\
\begin{tabular}{|c c c|}
-0.999974	 & 	-0.00711826	 & 	-0.0011816	 \\
0.00712547	 & 	-0.999955	 & 	-0.00622156	 \\
-0.00113726	 & 	-0.00622982	 & 	0.99998
\end{tabular}

\vtab
 EingenValues - Molecule A     \\
\begin{tabular}{|c c c|}
482.027	 & 	1240.32	 & 	1647.08	 \\
\end{tabular}
\columnbreak

Molecule B \
4NFAACf

\includegraphics[width=6cm]{../Comparisons/ImagesFromVMD/4NFAACf.png}

Inertia Tensor - Molecule B \\
\begin{tabular}{|c c c|}
509.683	 & 	2.80651	 & 	-1.91422	 \\
2.80651	 & 	1219.11	 & 	2.66132	 \\
-1.91422	 & 	2.66132	 & 	1681.17
\end{tabular}

\vtab
 EingenVectors - Molecule B     \\
\begin{tabular}{|c c c|}
-0.999991	 & 	0.00396206	 & 	-0.00164298	 \\
-0.00397143	 & 	-0.999976	 & 	0.0057431	 \\
-0.00162019	 & 	0.00574957	 & 	0.999982
\end{tabular}

\vtab
 EingenValues - Molecule B     \\
\begin{tabular}{|c c c|}
509.668	 & 	1219.11	 & 	1681.18	 \\
\end{tabular}

\end{center}
\end{multicols}

\vtab[-5mm]
\begin{tabular}{*{2}{m{0.38\textwidth}}}
\begin{center}
\textcolor{NavyBlue}{\Large Different}
\end{center}
&
\begin{center}
\includegraphics[height=6.5cm]{../Comparisons/Vectors/inertia_tensor_of_4NFAACc_and_4NFAACf.png}
\end{center}
\end{tabular}

 \newpage

\vtab[-3cm]
\begin{center}
{\large FireTest \tab Número 453}
\end{center}
\begin{multicols}{2}
\begin{center}

Molecule A \
4NFAACc

\includegraphics[width=6cm]{../Comparisons/ImagesFromVMD/4NFAACc.png}

Inertia Tensor - Molecule A \\
\begin{tabular}{|c c c|}
482.067	 & 	-5.39474	 & 	-1.35857	 \\
-5.39474	 & 	1240.3	 & 	-2.54035	 \\
-1.35857	 & 	-2.54035	 & 	1647.06
\end{tabular}

\vtab
 EingenVectors - Molecule A     \\
\begin{tabular}{|c c c|}
-0.999974	 & 	-0.00711826	 & 	-0.0011816	 \\
0.00712547	 & 	-0.999955	 & 	-0.00622156	 \\
-0.00113726	 & 	-0.00622982	 & 	0.99998
\end{tabular}

\vtab
 EingenValues - Molecule A     \\
\begin{tabular}{|c c c|}
482.027	 & 	1240.32	 & 	1647.08	 \\
\end{tabular}
\columnbreak

Molecule B \
4NFAACg

\includegraphics[width=6cm]{../Comparisons/ImagesFromVMD/4NFAACg.png}

Inertia Tensor - Molecule B \\
\begin{tabular}{|c c c|}
513.78	 & 	4.51917	 & 	0.266555	 \\
4.51917	 & 	1208.04	 & 	-1.18628	 \\
0.266555	 & 	-1.18628	 & 	1700.9
\end{tabular}

\vtab
 EingenVectors - Molecule B     \\
\begin{tabular}{|c c c|}
-0.999979	 & 	0.00650929	 & 	0.000231034	 \\
-0.00650983	 & 	-0.999976	 & 	-0.00240351	 \\
0.000215383	 & 	-0.00240496	 & 	0.999997
\end{tabular}

\vtab
 EingenValues - Molecule B     \\
\begin{tabular}{|c c c|}
513.751	 & 	1208.07	 & 	1700.9	 \\
\end{tabular}

\end{center}
\end{multicols}

\vtab[-5mm]
\begin{tabular}{*{2}{m{0.38\textwidth}}}
\begin{center}
\textcolor{NavyBlue}{\Large Different}
\end{center}
&
\begin{center}
\includegraphics[height=6.5cm]{../Comparisons/Vectors/inertia_tensor_of_4NFAACc_and_4NFAACg.png}
\end{center}
\end{tabular}

 \newpage

\vtab[-3cm]
\begin{center}
{\large FireTest \tab Número 454}
\end{center}
\begin{multicols}{2}
\begin{center}

Molecule A \
4NFAACc

\includegraphics[width=6cm]{../Comparisons/ImagesFromVMD/4NFAACc.png}

Inertia Tensor - Molecule A \\
\begin{tabular}{|c c c|}
482.067	 & 	-5.39474	 & 	-1.35857	 \\
-5.39474	 & 	1240.3	 & 	-2.54035	 \\
-1.35857	 & 	-2.54035	 & 	1647.06
\end{tabular}

\vtab
 EingenVectors - Molecule A     \\
\begin{tabular}{|c c c|}
-0.999974	 & 	-0.00711826	 & 	-0.0011816	 \\
0.00712547	 & 	-0.999955	 & 	-0.00622156	 \\
-0.00113726	 & 	-0.00622982	 & 	0.99998
\end{tabular}

\vtab
 EingenValues - Molecule A     \\
\begin{tabular}{|c c c|}
482.027	 & 	1240.32	 & 	1647.08	 \\
\end{tabular}
\columnbreak

Molecule B \
4NFAACi

\includegraphics[width=6cm]{../Comparisons/ImagesFromVMD/4NFAACi.png}

Inertia Tensor - Molecule B \\
\begin{tabular}{|c c c|}
502.43	 & 	-0.602691	 & 	-4.86988	 \\
-0.602691	 & 	1232.26	 & 	0.407295	 \\
-4.86988	 & 	0.407295	 & 	1676
\end{tabular}

\vtab
 EingenVectors - Molecule B     \\
\begin{tabular}{|c c c|}
0.999991	 & 	0.000823447	 & 	0.00414923	 \\
0.000819608	 & 	-0.999999	 & 	0.00092687	 \\
-0.00414999	 & 	0.000923461	 & 	0.999991
\end{tabular}

\vtab
 EingenValues - Molecule B     \\
\begin{tabular}{|c c c|}
502.409	 & 	1232.26	 & 	1676.02	 \\
\end{tabular}

\end{center}
\end{multicols}

\vtab[-5mm]
\begin{tabular}{*{2}{m{0.38\textwidth}}}
\begin{center}
\textcolor{NavyBlue}{\Large Different}
\end{center}
&
\begin{center}
\includegraphics[height=6.5cm]{../Comparisons/Vectors/inertia_tensor_of_4NFAACc_and_4NFAACi.png}
\end{center}
\end{tabular}

 \newpage

\vtab[-3cm]
\begin{center}
{\large FireTest \tab Número 455}
\end{center}
\begin{multicols}{2}
\begin{center}

Molecule A \
4NFAACc

\includegraphics[width=6cm]{../Comparisons/ImagesFromVMD/4NFAACc.png}

Inertia Tensor - Molecule A \\
\begin{tabular}{|c c c|}
482.067	 & 	-5.39474	 & 	-1.35857	 \\
-5.39474	 & 	1240.3	 & 	-2.54035	 \\
-1.35857	 & 	-2.54035	 & 	1647.06
\end{tabular}

\vtab
 EingenVectors - Molecule A     \\
\begin{tabular}{|c c c|}
-0.999974	 & 	-0.00711826	 & 	-0.0011816	 \\
0.00712547	 & 	-0.999955	 & 	-0.00622156	 \\
-0.00113726	 & 	-0.00622982	 & 	0.99998
\end{tabular}

\vtab
 EingenValues - Molecule A     \\
\begin{tabular}{|c c c|}
482.027	 & 	1240.32	 & 	1647.08	 \\
\end{tabular}
\columnbreak

Molecule B \
4NFAACj

\includegraphics[width=6cm]{../Comparisons/ImagesFromVMD/4NFAACj.png}

Inertia Tensor - Molecule B \\
\begin{tabular}{|c c c|}
510.047	 & 	9.97005	 & 	-3.6306	 \\
9.97005	 & 	1225.52	 & 	-0.981092	 \\
-3.6306	 & 	-0.981092	 & 	1680.82
\end{tabular}

\vtab
 EingenVectors - Molecule B     \\
\begin{tabular}{|c c c|}
-0.999898	 & 	0.0139264	 & 	-0.00308865	 \\
0.0139195	 & 	0.999901	 & 	0.00226627	 \\
-0.00311991	 & 	-0.00222305	 & 	0.999993
\end{tabular}

\vtab
 EingenValues - Molecule B     \\
\begin{tabular}{|c c c|}
509.897	 & 	1225.65	 & 	1680.83	 \\
\end{tabular}

\end{center}
\end{multicols}

\vtab[-5mm]
\begin{tabular}{*{2}{m{0.38\textwidth}}}
\begin{center}
\textcolor{NavyBlue}{\Large Different}
\end{center}
&
\begin{center}
\includegraphics[height=6.5cm]{../Comparisons/Vectors/inertia_tensor_of_4NFAACc_and_4NFAACj.png}
\end{center}
\end{tabular}

 \newpage

\vtab[-3cm]
\begin{center}
{\large FireTest \tab Número 456}
\end{center}
\begin{multicols}{2}
\begin{center}

Molecule A \
4NFAACc

\includegraphics[width=6cm]{../Comparisons/ImagesFromVMD/4NFAACc.png}

Inertia Tensor - Molecule A \\
\begin{tabular}{|c c c|}
482.067	 & 	-5.39474	 & 	-1.35857	 \\
-5.39474	 & 	1240.3	 & 	-2.54035	 \\
-1.35857	 & 	-2.54035	 & 	1647.06
\end{tabular}

\vtab
 EingenVectors - Molecule A     \\
\begin{tabular}{|c c c|}
-0.999974	 & 	-0.00711826	 & 	-0.0011816	 \\
0.00712547	 & 	-0.999955	 & 	-0.00622156	 \\
-0.00113726	 & 	-0.00622982	 & 	0.99998
\end{tabular}

\vtab
 EingenValues - Molecule A     \\
\begin{tabular}{|c c c|}
482.027	 & 	1240.32	 & 	1647.08	 \\
\end{tabular}
\columnbreak

Molecule B \
4NFAACl-3

\includegraphics[width=6cm]{../Comparisons/ImagesFromVMD/4NFAACl-3.png}

Inertia Tensor - Molecule B \\
\begin{tabular}{|c c c|}
506.608	 & 	0.709539	 & 	-0.555426	 \\
0.709539	 & 	1222.37	 & 	-2.84005	 \\
-0.555426	 & 	-2.84005	 & 	1678.41
\end{tabular}

\vtab
 EingenVectors - Molecule B     \\
\begin{tabular}{|c c c|}
-0.999999	 & 	0.000989428	 & 	-0.000471595	 \\
-0.000986471	 & 	-0.99998	 & 	-0.00622856	 \\
-0.000477748	 & 	-0.00622809	 & 	0.99998
\end{tabular}

\vtab
 EingenValues - Molecule B     \\
\begin{tabular}{|c c c|}
506.607	 & 	1222.36	 & 	1678.43	 \\
\end{tabular}

\end{center}
\end{multicols}

\vtab[-5mm]
\begin{tabular}{*{2}{m{0.38\textwidth}}}
\begin{center}
\textcolor{NavyBlue}{\Large Different}
\end{center}
&
\begin{center}
\includegraphics[height=6.5cm]{../Comparisons/Vectors/inertia_tensor_of_4NFAACc_and_4NFAACl-3.png}
\end{center}
\end{tabular}

 \newpage

\vtab[-3cm]
\begin{center}
{\large FireTest \tab Número 457}
\end{center}
\begin{multicols}{2}
\begin{center}

Molecule A \
4NFAACd

\includegraphics[width=6cm]{../Comparisons/ImagesFromVMD/4NFAACd.png}

Inertia Tensor - Molecule A \\
\begin{tabular}{|c c c|}
491.672	 & 	0.24486	 & 	-3.10016	 \\
0.24486	 & 	1231.15	 & 	2.19965	 \\
-3.10016	 & 	2.19965	 & 	1650.11
\end{tabular}

\vtab
 EingenVectors - Molecule A     \\
\begin{tabular}{|c c c|}
0.999996	 & 	-0.000339081	 & 	0.00267677	 \\
-0.000353124	 & 	-0.999986	 & 	0.00524747	 \\
-0.00267495	 & 	0.0052484	 & 	0.999983
\end{tabular}

\vtab
 EingenValues - Molecule A     \\
\begin{tabular}{|c c c|}
491.663	 & 	1231.14	 & 	1650.13	 \\
\end{tabular}
\columnbreak

Molecule B \
4NFAACe

\includegraphics[width=6cm]{../Comparisons/ImagesFromVMD/4NFAACe.png}

Inertia Tensor - Molecule B \\
\begin{tabular}{|c c c|}
489.025	 & 	-0.430035	 & 	3.98876	 \\
-0.430035	 & 	1233.71	 & 	-2.06505	 \\
3.98876	 & 	-2.06505	 & 	1641.79
\end{tabular}

\vtab
 EingenVectors - Molecule B     \\
\begin{tabular}{|c c c|}
0.999994	 & 	0.000567863	 & 	-0.00345908	 \\
0.000550336	 & 	-0.999987	 & 	-0.00506565	 \\
0.00346192	 & 	-0.00506372	 & 	0.999981
\end{tabular}

\vtab
 EingenValues - Molecule B     \\
\begin{tabular}{|c c c|}
489.011	 & 	1233.7	 & 	1641.81	 \\
\end{tabular}

\end{center}
\end{multicols}

\vtab[-5mm]
\begin{tabular}{*{2}{m{0.38\textwidth}}}
\begin{center}
\textcolor{NavyBlue}{\Large Different}
\end{center}
&
\begin{center}
\includegraphics[height=6.5cm]{../Comparisons/Vectors/inertia_tensor_of_4NFAACd_and_4NFAACe.png}
\end{center}
\end{tabular}

 \newpage

\vtab[-3cm]
\begin{center}
{\large FireTest \tab Número 458}
\end{center}
\begin{multicols}{2}
\begin{center}

Molecule A \
4NFAACd

\includegraphics[width=6cm]{../Comparisons/ImagesFromVMD/4NFAACd.png}

Inertia Tensor - Molecule A \\
\begin{tabular}{|c c c|}
491.672	 & 	0.24486	 & 	-3.10016	 \\
0.24486	 & 	1231.15	 & 	2.19965	 \\
-3.10016	 & 	2.19965	 & 	1650.11
\end{tabular}

\vtab
 EingenVectors - Molecule A     \\
\begin{tabular}{|c c c|}
0.999996	 & 	-0.000339081	 & 	0.00267677	 \\
-0.000353124	 & 	-0.999986	 & 	0.00524747	 \\
-0.00267495	 & 	0.0052484	 & 	0.999983
\end{tabular}

\vtab
 EingenValues - Molecule A     \\
\begin{tabular}{|c c c|}
491.663	 & 	1231.14	 & 	1650.13	 \\
\end{tabular}
\columnbreak

Molecule B \
4NFAACf

\includegraphics[width=6cm]{../Comparisons/ImagesFromVMD/4NFAACf.png}

Inertia Tensor - Molecule B \\
\begin{tabular}{|c c c|}
509.683	 & 	2.80651	 & 	-1.91422	 \\
2.80651	 & 	1219.11	 & 	2.66132	 \\
-1.91422	 & 	2.66132	 & 	1681.17
\end{tabular}

\vtab
 EingenVectors - Molecule B     \\
\begin{tabular}{|c c c|}
-0.999991	 & 	0.00396206	 & 	-0.00164298	 \\
-0.00397143	 & 	-0.999976	 & 	0.0057431	 \\
-0.00162019	 & 	0.00574957	 & 	0.999982
\end{tabular}

\vtab
 EingenValues - Molecule B     \\
\begin{tabular}{|c c c|}
509.668	 & 	1219.11	 & 	1681.18	 \\
\end{tabular}

\end{center}
\end{multicols}

\vtab[-5mm]
\begin{tabular}{*{2}{m{0.38\textwidth}}}
\begin{center}
\textcolor{NavyBlue}{\Large Different}
\end{center}
&
\begin{center}
\includegraphics[height=6.5cm]{../Comparisons/Vectors/inertia_tensor_of_4NFAACd_and_4NFAACf.png}
\end{center}
\end{tabular}

 \newpage

\vtab[-3cm]
\begin{center}
{\large FireTest \tab Número 459}
\end{center}
\begin{multicols}{2}
\begin{center}

Molecule A \
4NFAACd

\includegraphics[width=6cm]{../Comparisons/ImagesFromVMD/4NFAACd.png}

Inertia Tensor - Molecule A \\
\begin{tabular}{|c c c|}
491.672	 & 	0.24486	 & 	-3.10016	 \\
0.24486	 & 	1231.15	 & 	2.19965	 \\
-3.10016	 & 	2.19965	 & 	1650.11
\end{tabular}

\vtab
 EingenVectors - Molecule A     \\
\begin{tabular}{|c c c|}
0.999996	 & 	-0.000339081	 & 	0.00267677	 \\
-0.000353124	 & 	-0.999986	 & 	0.00524747	 \\
-0.00267495	 & 	0.0052484	 & 	0.999983
\end{tabular}

\vtab
 EingenValues - Molecule A     \\
\begin{tabular}{|c c c|}
491.663	 & 	1231.14	 & 	1650.13	 \\
\end{tabular}
\columnbreak

Molecule B \
4NFAACg

\includegraphics[width=6cm]{../Comparisons/ImagesFromVMD/4NFAACg.png}

Inertia Tensor - Molecule B \\
\begin{tabular}{|c c c|}
513.78	 & 	4.51917	 & 	0.266555	 \\
4.51917	 & 	1208.04	 & 	-1.18628	 \\
0.266555	 & 	-1.18628	 & 	1700.9
\end{tabular}

\vtab
 EingenVectors - Molecule B     \\
\begin{tabular}{|c c c|}
-0.999979	 & 	0.00650929	 & 	0.000231034	 \\
-0.00650983	 & 	-0.999976	 & 	-0.00240351	 \\
0.000215383	 & 	-0.00240496	 & 	0.999997
\end{tabular}

\vtab
 EingenValues - Molecule B     \\
\begin{tabular}{|c c c|}
513.751	 & 	1208.07	 & 	1700.9	 \\
\end{tabular}

\end{center}
\end{multicols}

\vtab[-5mm]
\begin{tabular}{*{2}{m{0.38\textwidth}}}
\begin{center}
\textcolor{NavyBlue}{\Large Different}
\end{center}
&
\begin{center}
\includegraphics[height=6.5cm]{../Comparisons/Vectors/inertia_tensor_of_4NFAACd_and_4NFAACg.png}
\end{center}
\end{tabular}

 \newpage

\vtab[-3cm]
\begin{center}
{\large FireTest \tab Número 460}
\end{center}
\begin{multicols}{2}
\begin{center}

Molecule A \
4NFAACd

\includegraphics[width=6cm]{../Comparisons/ImagesFromVMD/4NFAACd.png}

Inertia Tensor - Molecule A \\
\begin{tabular}{|c c c|}
491.672	 & 	0.24486	 & 	-3.10016	 \\
0.24486	 & 	1231.15	 & 	2.19965	 \\
-3.10016	 & 	2.19965	 & 	1650.11
\end{tabular}

\vtab
 EingenVectors - Molecule A     \\
\begin{tabular}{|c c c|}
0.999996	 & 	-0.000339081	 & 	0.00267677	 \\
-0.000353124	 & 	-0.999986	 & 	0.00524747	 \\
-0.00267495	 & 	0.0052484	 & 	0.999983
\end{tabular}

\vtab
 EingenValues - Molecule A     \\
\begin{tabular}{|c c c|}
491.663	 & 	1231.14	 & 	1650.13	 \\
\end{tabular}
\columnbreak

Molecule B \
4NFAACi

\includegraphics[width=6cm]{../Comparisons/ImagesFromVMD/4NFAACi.png}

Inertia Tensor - Molecule B \\
\begin{tabular}{|c c c|}
502.43	 & 	-0.602691	 & 	-4.86988	 \\
-0.602691	 & 	1232.26	 & 	0.407295	 \\
-4.86988	 & 	0.407295	 & 	1676
\end{tabular}

\vtab
 EingenVectors - Molecule B     \\
\begin{tabular}{|c c c|}
0.999991	 & 	0.000823447	 & 	0.00414923	 \\
0.000819608	 & 	-0.999999	 & 	0.00092687	 \\
-0.00414999	 & 	0.000923461	 & 	0.999991
\end{tabular}

\vtab
 EingenValues - Molecule B     \\
\begin{tabular}{|c c c|}
502.409	 & 	1232.26	 & 	1676.02	 \\
\end{tabular}

\end{center}
\end{multicols}

\vtab[-5mm]
\begin{tabular}{*{2}{m{0.38\textwidth}}}
\begin{center}
\textcolor{NavyBlue}{\Large Different}
\end{center}
&
\begin{center}
\includegraphics[height=6.5cm]{../Comparisons/Vectors/inertia_tensor_of_4NFAACd_and_4NFAACi.png}
\end{center}
\end{tabular}

 \newpage

\vtab[-3cm]
\begin{center}
{\large FireTest \tab Número 461}
\end{center}
\begin{multicols}{2}
\begin{center}

Molecule A \
4NFAACd

\includegraphics[width=6cm]{../Comparisons/ImagesFromVMD/4NFAACd.png}

Inertia Tensor - Molecule A \\
\begin{tabular}{|c c c|}
491.672	 & 	0.24486	 & 	-3.10016	 \\
0.24486	 & 	1231.15	 & 	2.19965	 \\
-3.10016	 & 	2.19965	 & 	1650.11
\end{tabular}

\vtab
 EingenVectors - Molecule A     \\
\begin{tabular}{|c c c|}
0.999996	 & 	-0.000339081	 & 	0.00267677	 \\
-0.000353124	 & 	-0.999986	 & 	0.00524747	 \\
-0.00267495	 & 	0.0052484	 & 	0.999983
\end{tabular}

\vtab
 EingenValues - Molecule A     \\
\begin{tabular}{|c c c|}
491.663	 & 	1231.14	 & 	1650.13	 \\
\end{tabular}
\columnbreak

Molecule B \
4NFAACj

\includegraphics[width=6cm]{../Comparisons/ImagesFromVMD/4NFAACj.png}

Inertia Tensor - Molecule B \\
\begin{tabular}{|c c c|}
510.047	 & 	9.97005	 & 	-3.6306	 \\
9.97005	 & 	1225.52	 & 	-0.981092	 \\
-3.6306	 & 	-0.981092	 & 	1680.82
\end{tabular}

\vtab
 EingenVectors - Molecule B     \\
\begin{tabular}{|c c c|}
-0.999898	 & 	0.0139264	 & 	-0.00308865	 \\
0.0139195	 & 	0.999901	 & 	0.00226627	 \\
-0.00311991	 & 	-0.00222305	 & 	0.999993
\end{tabular}

\vtab
 EingenValues - Molecule B     \\
\begin{tabular}{|c c c|}
509.897	 & 	1225.65	 & 	1680.83	 \\
\end{tabular}

\end{center}
\end{multicols}

\vtab[-5mm]
\begin{tabular}{*{2}{m{0.38\textwidth}}}
\begin{center}
\textcolor{NavyBlue}{\Large Different}
\end{center}
&
\begin{center}
\includegraphics[height=6.5cm]{../Comparisons/Vectors/inertia_tensor_of_4NFAACd_and_4NFAACj.png}
\end{center}
\end{tabular}

 \newpage

\vtab[-3cm]
\begin{center}
{\large FireTest \tab Número 462}
\end{center}
\begin{multicols}{2}
\begin{center}

Molecule A \
4NFAACd

\includegraphics[width=6cm]{../Comparisons/ImagesFromVMD/4NFAACd.png}

Inertia Tensor - Molecule A \\
\begin{tabular}{|c c c|}
491.672	 & 	0.24486	 & 	-3.10016	 \\
0.24486	 & 	1231.15	 & 	2.19965	 \\
-3.10016	 & 	2.19965	 & 	1650.11
\end{tabular}

\vtab
 EingenVectors - Molecule A     \\
\begin{tabular}{|c c c|}
0.999996	 & 	-0.000339081	 & 	0.00267677	 \\
-0.000353124	 & 	-0.999986	 & 	0.00524747	 \\
-0.00267495	 & 	0.0052484	 & 	0.999983
\end{tabular}

\vtab
 EingenValues - Molecule A     \\
\begin{tabular}{|c c c|}
491.663	 & 	1231.14	 & 	1650.13	 \\
\end{tabular}
\columnbreak

Molecule B \
4NFAACl-3

\includegraphics[width=6cm]{../Comparisons/ImagesFromVMD/4NFAACl-3.png}

Inertia Tensor - Molecule B \\
\begin{tabular}{|c c c|}
506.608	 & 	0.709539	 & 	-0.555426	 \\
0.709539	 & 	1222.37	 & 	-2.84005	 \\
-0.555426	 & 	-2.84005	 & 	1678.41
\end{tabular}

\vtab
 EingenVectors - Molecule B     \\
\begin{tabular}{|c c c|}
-0.999999	 & 	0.000989428	 & 	-0.000471595	 \\
-0.000986471	 & 	-0.99998	 & 	-0.00622856	 \\
-0.000477748	 & 	-0.00622809	 & 	0.99998
\end{tabular}

\vtab
 EingenValues - Molecule B     \\
\begin{tabular}{|c c c|}
506.607	 & 	1222.36	 & 	1678.43	 \\
\end{tabular}

\end{center}
\end{multicols}

\vtab[-5mm]
\begin{tabular}{*{2}{m{0.38\textwidth}}}
\begin{center}
\textcolor{NavyBlue}{\Large Different}
\end{center}
&
\begin{center}
\includegraphics[height=6.5cm]{../Comparisons/Vectors/inertia_tensor_of_4NFAACd_and_4NFAACl-3.png}
\end{center}
\end{tabular}

 \newpage

\vtab[-3cm]
\begin{center}
{\large FireTest \tab Número 463}
\end{center}
\begin{multicols}{2}
\begin{center}

Molecule A \
4NFAACe

\includegraphics[width=6cm]{../Comparisons/ImagesFromVMD/4NFAACe.png}

Inertia Tensor - Molecule A \\
\begin{tabular}{|c c c|}
489.025	 & 	-0.430035	 & 	3.98876	 \\
-0.430035	 & 	1233.71	 & 	-2.06505	 \\
3.98876	 & 	-2.06505	 & 	1641.79
\end{tabular}

\vtab
 EingenVectors - Molecule A     \\
\begin{tabular}{|c c c|}
0.999994	 & 	0.000567863	 & 	-0.00345908	 \\
0.000550336	 & 	-0.999987	 & 	-0.00506565	 \\
0.00346192	 & 	-0.00506372	 & 	0.999981
\end{tabular}

\vtab
 EingenValues - Molecule A     \\
\begin{tabular}{|c c c|}
489.011	 & 	1233.7	 & 	1641.81	 \\
\end{tabular}
\columnbreak

Molecule B \
4NFAACf

\includegraphics[width=6cm]{../Comparisons/ImagesFromVMD/4NFAACf.png}

Inertia Tensor - Molecule B \\
\begin{tabular}{|c c c|}
509.683	 & 	2.80651	 & 	-1.91422	 \\
2.80651	 & 	1219.11	 & 	2.66132	 \\
-1.91422	 & 	2.66132	 & 	1681.17
\end{tabular}

\vtab
 EingenVectors - Molecule B     \\
\begin{tabular}{|c c c|}
-0.999991	 & 	0.00396206	 & 	-0.00164298	 \\
-0.00397143	 & 	-0.999976	 & 	0.0057431	 \\
-0.00162019	 & 	0.00574957	 & 	0.999982
\end{tabular}

\vtab
 EingenValues - Molecule B     \\
\begin{tabular}{|c c c|}
509.668	 & 	1219.11	 & 	1681.18	 \\
\end{tabular}

\end{center}
\end{multicols}

\vtab[-5mm]
\begin{tabular}{*{2}{m{0.38\textwidth}}}
\begin{center}
\textcolor{NavyBlue}{\Large Different}
\end{center}
&
\begin{center}
\includegraphics[height=6.5cm]{../Comparisons/Vectors/inertia_tensor_of_4NFAACe_and_4NFAACf.png}
\end{center}
\end{tabular}

 \newpage

\vtab[-3cm]
\begin{center}
{\large FireTest \tab Número 464}
\end{center}
\begin{multicols}{2}
\begin{center}

Molecule A \
4NFAACe

\includegraphics[width=6cm]{../Comparisons/ImagesFromVMD/4NFAACe.png}

Inertia Tensor - Molecule A \\
\begin{tabular}{|c c c|}
489.025	 & 	-0.430035	 & 	3.98876	 \\
-0.430035	 & 	1233.71	 & 	-2.06505	 \\
3.98876	 & 	-2.06505	 & 	1641.79
\end{tabular}

\vtab
 EingenVectors - Molecule A     \\
\begin{tabular}{|c c c|}
0.999994	 & 	0.000567863	 & 	-0.00345908	 \\
0.000550336	 & 	-0.999987	 & 	-0.00506565	 \\
0.00346192	 & 	-0.00506372	 & 	0.999981
\end{tabular}

\vtab
 EingenValues - Molecule A     \\
\begin{tabular}{|c c c|}
489.011	 & 	1233.7	 & 	1641.81	 \\
\end{tabular}
\columnbreak

Molecule B \
4NFAACg

\includegraphics[width=6cm]{../Comparisons/ImagesFromVMD/4NFAACg.png}

Inertia Tensor - Molecule B \\
\begin{tabular}{|c c c|}
513.78	 & 	4.51917	 & 	0.266555	 \\
4.51917	 & 	1208.04	 & 	-1.18628	 \\
0.266555	 & 	-1.18628	 & 	1700.9
\end{tabular}

\vtab
 EingenVectors - Molecule B     \\
\begin{tabular}{|c c c|}
-0.999979	 & 	0.00650929	 & 	0.000231034	 \\
-0.00650983	 & 	-0.999976	 & 	-0.00240351	 \\
0.000215383	 & 	-0.00240496	 & 	0.999997
\end{tabular}

\vtab
 EingenValues - Molecule B     \\
\begin{tabular}{|c c c|}
513.751	 & 	1208.07	 & 	1700.9	 \\
\end{tabular}

\end{center}
\end{multicols}

\vtab[-5mm]
\begin{tabular}{*{2}{m{0.38\textwidth}}}
\begin{center}
\textcolor{NavyBlue}{\Large Different}
\end{center}
&
\begin{center}
\includegraphics[height=6.5cm]{../Comparisons/Vectors/inertia_tensor_of_4NFAACe_and_4NFAACg.png}
\end{center}
\end{tabular}

 \newpage

\vtab[-3cm]
\begin{center}
{\large FireTest \tab Número 465}
\end{center}
\begin{multicols}{2}
\begin{center}

Molecule A \
4NFAACe

\includegraphics[width=6cm]{../Comparisons/ImagesFromVMD/4NFAACe.png}

Inertia Tensor - Molecule A \\
\begin{tabular}{|c c c|}
489.025	 & 	-0.430035	 & 	3.98876	 \\
-0.430035	 & 	1233.71	 & 	-2.06505	 \\
3.98876	 & 	-2.06505	 & 	1641.79
\end{tabular}

\vtab
 EingenVectors - Molecule A     \\
\begin{tabular}{|c c c|}
0.999994	 & 	0.000567863	 & 	-0.00345908	 \\
0.000550336	 & 	-0.999987	 & 	-0.00506565	 \\
0.00346192	 & 	-0.00506372	 & 	0.999981
\end{tabular}

\vtab
 EingenValues - Molecule A     \\
\begin{tabular}{|c c c|}
489.011	 & 	1233.7	 & 	1641.81	 \\
\end{tabular}
\columnbreak

Molecule B \
4NFAACi

\includegraphics[width=6cm]{../Comparisons/ImagesFromVMD/4NFAACi.png}

Inertia Tensor - Molecule B \\
\begin{tabular}{|c c c|}
502.43	 & 	-0.602691	 & 	-4.86988	 \\
-0.602691	 & 	1232.26	 & 	0.407295	 \\
-4.86988	 & 	0.407295	 & 	1676
\end{tabular}

\vtab
 EingenVectors - Molecule B     \\
\begin{tabular}{|c c c|}
0.999991	 & 	0.000823447	 & 	0.00414923	 \\
0.000819608	 & 	-0.999999	 & 	0.00092687	 \\
-0.00414999	 & 	0.000923461	 & 	0.999991
\end{tabular}

\vtab
 EingenValues - Molecule B     \\
\begin{tabular}{|c c c|}
502.409	 & 	1232.26	 & 	1676.02	 \\
\end{tabular}

\end{center}
\end{multicols}

\vtab[-5mm]
\begin{tabular}{*{2}{m{0.38\textwidth}}}
\begin{center}
\textcolor{NavyBlue}{\Large Different}
\end{center}
&
\begin{center}
\includegraphics[height=6.5cm]{../Comparisons/Vectors/inertia_tensor_of_4NFAACe_and_4NFAACi.png}
\end{center}
\end{tabular}

 \newpage

\vtab[-3cm]
\begin{center}
{\large FireTest \tab Número 466}
\end{center}
\begin{multicols}{2}
\begin{center}

Molecule A \
4NFAACe

\includegraphics[width=6cm]{../Comparisons/ImagesFromVMD/4NFAACe.png}

Inertia Tensor - Molecule A \\
\begin{tabular}{|c c c|}
489.025	 & 	-0.430035	 & 	3.98876	 \\
-0.430035	 & 	1233.71	 & 	-2.06505	 \\
3.98876	 & 	-2.06505	 & 	1641.79
\end{tabular}

\vtab
 EingenVectors - Molecule A     \\
\begin{tabular}{|c c c|}
0.999994	 & 	0.000567863	 & 	-0.00345908	 \\
0.000550336	 & 	-0.999987	 & 	-0.00506565	 \\
0.00346192	 & 	-0.00506372	 & 	0.999981
\end{tabular}

\vtab
 EingenValues - Molecule A     \\
\begin{tabular}{|c c c|}
489.011	 & 	1233.7	 & 	1641.81	 \\
\end{tabular}
\columnbreak

Molecule B \
4NFAACj

\includegraphics[width=6cm]{../Comparisons/ImagesFromVMD/4NFAACj.png}

Inertia Tensor - Molecule B \\
\begin{tabular}{|c c c|}
510.047	 & 	9.97005	 & 	-3.6306	 \\
9.97005	 & 	1225.52	 & 	-0.981092	 \\
-3.6306	 & 	-0.981092	 & 	1680.82
\end{tabular}

\vtab
 EingenVectors - Molecule B     \\
\begin{tabular}{|c c c|}
-0.999898	 & 	0.0139264	 & 	-0.00308865	 \\
0.0139195	 & 	0.999901	 & 	0.00226627	 \\
-0.00311991	 & 	-0.00222305	 & 	0.999993
\end{tabular}

\vtab
 EingenValues - Molecule B     \\
\begin{tabular}{|c c c|}
509.897	 & 	1225.65	 & 	1680.83	 \\
\end{tabular}

\end{center}
\end{multicols}

\vtab[-5mm]
\begin{tabular}{*{2}{m{0.38\textwidth}}}
\begin{center}
\textcolor{NavyBlue}{\Large Different}
\end{center}
&
\begin{center}
\includegraphics[height=6.5cm]{../Comparisons/Vectors/inertia_tensor_of_4NFAACe_and_4NFAACj.png}
\end{center}
\end{tabular}

 \newpage

\vtab[-3cm]
\begin{center}
{\large FireTest \tab Número 467}
\end{center}
\begin{multicols}{2}
\begin{center}

Molecule A \
4NFAACe

\includegraphics[width=6cm]{../Comparisons/ImagesFromVMD/4NFAACe.png}

Inertia Tensor - Molecule A \\
\begin{tabular}{|c c c|}
489.025	 & 	-0.430035	 & 	3.98876	 \\
-0.430035	 & 	1233.71	 & 	-2.06505	 \\
3.98876	 & 	-2.06505	 & 	1641.79
\end{tabular}

\vtab
 EingenVectors - Molecule A     \\
\begin{tabular}{|c c c|}
0.999994	 & 	0.000567863	 & 	-0.00345908	 \\
0.000550336	 & 	-0.999987	 & 	-0.00506565	 \\
0.00346192	 & 	-0.00506372	 & 	0.999981
\end{tabular}

\vtab
 EingenValues - Molecule A     \\
\begin{tabular}{|c c c|}
489.011	 & 	1233.7	 & 	1641.81	 \\
\end{tabular}
\columnbreak

Molecule B \
4NFAACl-3

\includegraphics[width=6cm]{../Comparisons/ImagesFromVMD/4NFAACl-3.png}

Inertia Tensor - Molecule B \\
\begin{tabular}{|c c c|}
506.608	 & 	0.709539	 & 	-0.555426	 \\
0.709539	 & 	1222.37	 & 	-2.84005	 \\
-0.555426	 & 	-2.84005	 & 	1678.41
\end{tabular}

\vtab
 EingenVectors - Molecule B     \\
\begin{tabular}{|c c c|}
-0.999999	 & 	0.000989428	 & 	-0.000471595	 \\
-0.000986471	 & 	-0.99998	 & 	-0.00622856	 \\
-0.000477748	 & 	-0.00622809	 & 	0.99998
\end{tabular}

\vtab
 EingenValues - Molecule B     \\
\begin{tabular}{|c c c|}
506.607	 & 	1222.36	 & 	1678.43	 \\
\end{tabular}

\end{center}
\end{multicols}

\vtab[-5mm]
\begin{tabular}{*{2}{m{0.38\textwidth}}}
\begin{center}
\textcolor{NavyBlue}{\Large Different}
\end{center}
&
\begin{center}
\includegraphics[height=6.5cm]{../Comparisons/Vectors/inertia_tensor_of_4NFAACe_and_4NFAACl-3.png}
\end{center}
\end{tabular}

 \newpage

\vtab[-3cm]
\begin{center}
{\large FireTest \tab Número 468}
\end{center}
\begin{multicols}{2}
\begin{center}

Molecule A \
4NFAACf

\includegraphics[width=6cm]{../Comparisons/ImagesFromVMD/4NFAACf.png}

Inertia Tensor - Molecule A \\
\begin{tabular}{|c c c|}
509.683	 & 	2.80651	 & 	-1.91422	 \\
2.80651	 & 	1219.11	 & 	2.66132	 \\
-1.91422	 & 	2.66132	 & 	1681.17
\end{tabular}

\vtab
 EingenVectors - Molecule A     \\
\begin{tabular}{|c c c|}
-0.999991	 & 	0.00396206	 & 	-0.00164298	 \\
-0.00397143	 & 	-0.999976	 & 	0.0057431	 \\
-0.00162019	 & 	0.00574957	 & 	0.999982
\end{tabular}

\vtab
 EingenValues - Molecule A     \\
\begin{tabular}{|c c c|}
509.668	 & 	1219.11	 & 	1681.18	 \\
\end{tabular}
\columnbreak

Molecule B \
4NFAACg

\includegraphics[width=6cm]{../Comparisons/ImagesFromVMD/4NFAACg.png}

Inertia Tensor - Molecule B \\
\begin{tabular}{|c c c|}
513.78	 & 	4.51917	 & 	0.266555	 \\
4.51917	 & 	1208.04	 & 	-1.18628	 \\
0.266555	 & 	-1.18628	 & 	1700.9
\end{tabular}

\vtab
 EingenVectors - Molecule B     \\
\begin{tabular}{|c c c|}
-0.999979	 & 	0.00650929	 & 	0.000231034	 \\
-0.00650983	 & 	-0.999976	 & 	-0.00240351	 \\
0.000215383	 & 	-0.00240496	 & 	0.999997
\end{tabular}

\vtab
 EingenValues - Molecule B     \\
\begin{tabular}{|c c c|}
513.751	 & 	1208.07	 & 	1700.9	 \\
\end{tabular}

\end{center}
\end{multicols}

\vtab[-5mm]
\begin{tabular}{*{2}{m{0.38\textwidth}}}
\begin{center}
\textcolor{NavyBlue}{\Large Different}
\end{center}
&
\begin{center}
\includegraphics[height=6.5cm]{../Comparisons/Vectors/inertia_tensor_of_4NFAACf_and_4NFAACg.png}
\end{center}
\end{tabular}

 \newpage

\vtab[-3cm]
\begin{center}
{\large FireTest \tab Número 469}
\end{center}
\begin{multicols}{2}
\begin{center}

Molecule A \
4NFAACf

\includegraphics[width=6cm]{../Comparisons/ImagesFromVMD/4NFAACf.png}

Inertia Tensor - Molecule A \\
\begin{tabular}{|c c c|}
509.683	 & 	2.80651	 & 	-1.91422	 \\
2.80651	 & 	1219.11	 & 	2.66132	 \\
-1.91422	 & 	2.66132	 & 	1681.17
\end{tabular}

\vtab
 EingenVectors - Molecule A     \\
\begin{tabular}{|c c c|}
-0.999991	 & 	0.00396206	 & 	-0.00164298	 \\
-0.00397143	 & 	-0.999976	 & 	0.0057431	 \\
-0.00162019	 & 	0.00574957	 & 	0.999982
\end{tabular}

\vtab
 EingenValues - Molecule A     \\
\begin{tabular}{|c c c|}
509.668	 & 	1219.11	 & 	1681.18	 \\
\end{tabular}
\columnbreak

Molecule B \
4NFAACi

\includegraphics[width=6cm]{../Comparisons/ImagesFromVMD/4NFAACi.png}

Inertia Tensor - Molecule B \\
\begin{tabular}{|c c c|}
502.43	 & 	-0.602691	 & 	-4.86988	 \\
-0.602691	 & 	1232.26	 & 	0.407295	 \\
-4.86988	 & 	0.407295	 & 	1676
\end{tabular}

\vtab
 EingenVectors - Molecule B     \\
\begin{tabular}{|c c c|}
0.999991	 & 	0.000823447	 & 	0.00414923	 \\
0.000819608	 & 	-0.999999	 & 	0.00092687	 \\
-0.00414999	 & 	0.000923461	 & 	0.999991
\end{tabular}

\vtab
 EingenValues - Molecule B     \\
\begin{tabular}{|c c c|}
502.409	 & 	1232.26	 & 	1676.02	 \\
\end{tabular}

\end{center}
\end{multicols}

\vtab[-5mm]
\begin{tabular}{*{2}{m{0.38\textwidth}}}
\begin{center}
\textcolor{NavyBlue}{\Large Different}
\end{center}
&
\begin{center}
\includegraphics[height=6.5cm]{../Comparisons/Vectors/inertia_tensor_of_4NFAACf_and_4NFAACi.png}
\end{center}
\end{tabular}

 \newpage

\vtab[-3cm]
\begin{center}
{\large FireTest \tab Número 470}
\end{center}
\begin{multicols}{2}
\begin{center}

Molecule A \
4NFAACf

\includegraphics[width=6cm]{../Comparisons/ImagesFromVMD/4NFAACf.png}

Inertia Tensor - Molecule A \\
\begin{tabular}{|c c c|}
509.683	 & 	2.80651	 & 	-1.91422	 \\
2.80651	 & 	1219.11	 & 	2.66132	 \\
-1.91422	 & 	2.66132	 & 	1681.17
\end{tabular}

\vtab
 EingenVectors - Molecule A     \\
\begin{tabular}{|c c c|}
-0.999991	 & 	0.00396206	 & 	-0.00164298	 \\
-0.00397143	 & 	-0.999976	 & 	0.0057431	 \\
-0.00162019	 & 	0.00574957	 & 	0.999982
\end{tabular}

\vtab
 EingenValues - Molecule A     \\
\begin{tabular}{|c c c|}
509.668	 & 	1219.11	 & 	1681.18	 \\
\end{tabular}
\columnbreak

Molecule B \
4NFAACj

\includegraphics[width=6cm]{../Comparisons/ImagesFromVMD/4NFAACj.png}

Inertia Tensor - Molecule B \\
\begin{tabular}{|c c c|}
510.047	 & 	9.97005	 & 	-3.6306	 \\
9.97005	 & 	1225.52	 & 	-0.981092	 \\
-3.6306	 & 	-0.981092	 & 	1680.82
\end{tabular}

\vtab
 EingenVectors - Molecule B     \\
\begin{tabular}{|c c c|}
-0.999898	 & 	0.0139264	 & 	-0.00308865	 \\
0.0139195	 & 	0.999901	 & 	0.00226627	 \\
-0.00311991	 & 	-0.00222305	 & 	0.999993
\end{tabular}

\vtab
 EingenValues - Molecule B     \\
\begin{tabular}{|c c c|}
509.897	 & 	1225.65	 & 	1680.83	 \\
\end{tabular}

\end{center}
\end{multicols}

\vtab[-5mm]
\begin{tabular}{*{2}{m{0.38\textwidth}}}
\begin{center}
\textcolor{NavyBlue}{\Large Different}
\end{center}
&
\begin{center}
\includegraphics[height=6.5cm]{../Comparisons/Vectors/inertia_tensor_of_4NFAACf_and_4NFAACj.png}
\end{center}
\end{tabular}

 \newpage

\vtab[-3cm]
\begin{center}
{\large FireTest \tab Número 471}
\end{center}
\begin{multicols}{2}
\begin{center}

Molecule A \
4NFAACf

\includegraphics[width=6cm]{../Comparisons/ImagesFromVMD/4NFAACf.png}

Inertia Tensor - Molecule A \\
\begin{tabular}{|c c c|}
509.683	 & 	2.80651	 & 	-1.91422	 \\
2.80651	 & 	1219.11	 & 	2.66132	 \\
-1.91422	 & 	2.66132	 & 	1681.17
\end{tabular}

\vtab
 EingenVectors - Molecule A     \\
\begin{tabular}{|c c c|}
-0.999991	 & 	0.00396206	 & 	-0.00164298	 \\
-0.00397143	 & 	-0.999976	 & 	0.0057431	 \\
-0.00162019	 & 	0.00574957	 & 	0.999982
\end{tabular}

\vtab
 EingenValues - Molecule A     \\
\begin{tabular}{|c c c|}
509.668	 & 	1219.11	 & 	1681.18	 \\
\end{tabular}
\columnbreak

Molecule B \
4NFAACl-3

\includegraphics[width=6cm]{../Comparisons/ImagesFromVMD/4NFAACl-3.png}

Inertia Tensor - Molecule B \\
\begin{tabular}{|c c c|}
506.608	 & 	0.709539	 & 	-0.555426	 \\
0.709539	 & 	1222.37	 & 	-2.84005	 \\
-0.555426	 & 	-2.84005	 & 	1678.41
\end{tabular}

\vtab
 EingenVectors - Molecule B     \\
\begin{tabular}{|c c c|}
-0.999999	 & 	0.000989428	 & 	-0.000471595	 \\
-0.000986471	 & 	-0.99998	 & 	-0.00622856	 \\
-0.000477748	 & 	-0.00622809	 & 	0.99998
\end{tabular}

\vtab
 EingenValues - Molecule B     \\
\begin{tabular}{|c c c|}
506.607	 & 	1222.36	 & 	1678.43	 \\
\end{tabular}

\end{center}
\end{multicols}

\vtab[-5mm]
\begin{tabular}{*{2}{m{0.38\textwidth}}}
\begin{center}
\textcolor{NavyBlue}{\Large Different}
\end{center}
&
\begin{center}
\includegraphics[height=6.5cm]{../Comparisons/Vectors/inertia_tensor_of_4NFAACf_and_4NFAACl-3.png}
\end{center}
\end{tabular}

 \newpage

\vtab[-3cm]
\begin{center}
{\large FireTest \tab Número 472}
\end{center}
\begin{multicols}{2}
\begin{center}

Molecule A \
4NFAACg

\includegraphics[width=6cm]{../Comparisons/ImagesFromVMD/4NFAACg.png}

Inertia Tensor - Molecule A \\
\begin{tabular}{|c c c|}
513.78	 & 	4.51917	 & 	0.266555	 \\
4.51917	 & 	1208.04	 & 	-1.18628	 \\
0.266555	 & 	-1.18628	 & 	1700.9
\end{tabular}

\vtab
 EingenVectors - Molecule A     \\
\begin{tabular}{|c c c|}
-0.999979	 & 	0.00650929	 & 	0.000231034	 \\
-0.00650983	 & 	-0.999976	 & 	-0.00240351	 \\
0.000215383	 & 	-0.00240496	 & 	0.999997
\end{tabular}

\vtab
 EingenValues - Molecule A     \\
\begin{tabular}{|c c c|}
513.751	 & 	1208.07	 & 	1700.9	 \\
\end{tabular}
\columnbreak

Molecule B \
4NFAACi

\includegraphics[width=6cm]{../Comparisons/ImagesFromVMD/4NFAACi.png}

Inertia Tensor - Molecule B \\
\begin{tabular}{|c c c|}
502.43	 & 	-0.602691	 & 	-4.86988	 \\
-0.602691	 & 	1232.26	 & 	0.407295	 \\
-4.86988	 & 	0.407295	 & 	1676
\end{tabular}

\vtab
 EingenVectors - Molecule B     \\
\begin{tabular}{|c c c|}
0.999991	 & 	0.000823447	 & 	0.00414923	 \\
0.000819608	 & 	-0.999999	 & 	0.00092687	 \\
-0.00414999	 & 	0.000923461	 & 	0.999991
\end{tabular}

\vtab
 EingenValues - Molecule B     \\
\begin{tabular}{|c c c|}
502.409	 & 	1232.26	 & 	1676.02	 \\
\end{tabular}

\end{center}
\end{multicols}

\vtab[-5mm]
\begin{tabular}{*{2}{m{0.38\textwidth}}}
\begin{center}
\textcolor{NavyBlue}{\Large Different}
\end{center}
&
\begin{center}
\includegraphics[height=6.5cm]{../Comparisons/Vectors/inertia_tensor_of_4NFAACg_and_4NFAACi.png}
\end{center}
\end{tabular}

 \newpage

\vtab[-3cm]
\begin{center}
{\large FireTest \tab Número 473}
\end{center}
\begin{multicols}{2}
\begin{center}

Molecule A \
4NFAACg

\includegraphics[width=6cm]{../Comparisons/ImagesFromVMD/4NFAACg.png}

Inertia Tensor - Molecule A \\
\begin{tabular}{|c c c|}
513.78	 & 	4.51917	 & 	0.266555	 \\
4.51917	 & 	1208.04	 & 	-1.18628	 \\
0.266555	 & 	-1.18628	 & 	1700.9
\end{tabular}

\vtab
 EingenVectors - Molecule A     \\
\begin{tabular}{|c c c|}
-0.999979	 & 	0.00650929	 & 	0.000231034	 \\
-0.00650983	 & 	-0.999976	 & 	-0.00240351	 \\
0.000215383	 & 	-0.00240496	 & 	0.999997
\end{tabular}

\vtab
 EingenValues - Molecule A     \\
\begin{tabular}{|c c c|}
513.751	 & 	1208.07	 & 	1700.9	 \\
\end{tabular}
\columnbreak

Molecule B \
4NFAACj

\includegraphics[width=6cm]{../Comparisons/ImagesFromVMD/4NFAACj.png}

Inertia Tensor - Molecule B \\
\begin{tabular}{|c c c|}
510.047	 & 	9.97005	 & 	-3.6306	 \\
9.97005	 & 	1225.52	 & 	-0.981092	 \\
-3.6306	 & 	-0.981092	 & 	1680.82
\end{tabular}

\vtab
 EingenVectors - Molecule B     \\
\begin{tabular}{|c c c|}
-0.999898	 & 	0.0139264	 & 	-0.00308865	 \\
0.0139195	 & 	0.999901	 & 	0.00226627	 \\
-0.00311991	 & 	-0.00222305	 & 	0.999993
\end{tabular}

\vtab
 EingenValues - Molecule B     \\
\begin{tabular}{|c c c|}
509.897	 & 	1225.65	 & 	1680.83	 \\
\end{tabular}

\end{center}
\end{multicols}

\vtab[-5mm]
\begin{tabular}{*{2}{m{0.38\textwidth}}}
\begin{center}
\textcolor{NavyBlue}{\Large Different}
\end{center}
&
\begin{center}
\includegraphics[height=6.5cm]{../Comparisons/Vectors/inertia_tensor_of_4NFAACg_and_4NFAACj.png}
\end{center}
\end{tabular}

 \newpage

\vtab[-3cm]
\begin{center}
{\large FireTest \tab Número 474}
\end{center}
\begin{multicols}{2}
\begin{center}

Molecule A \
4NFAACg

\includegraphics[width=6cm]{../Comparisons/ImagesFromVMD/4NFAACg.png}

Inertia Tensor - Molecule A \\
\begin{tabular}{|c c c|}
513.78	 & 	4.51917	 & 	0.266555	 \\
4.51917	 & 	1208.04	 & 	-1.18628	 \\
0.266555	 & 	-1.18628	 & 	1700.9
\end{tabular}

\vtab
 EingenVectors - Molecule A     \\
\begin{tabular}{|c c c|}
-0.999979	 & 	0.00650929	 & 	0.000231034	 \\
-0.00650983	 & 	-0.999976	 & 	-0.00240351	 \\
0.000215383	 & 	-0.00240496	 & 	0.999997
\end{tabular}

\vtab
 EingenValues - Molecule A     \\
\begin{tabular}{|c c c|}
513.751	 & 	1208.07	 & 	1700.9	 \\
\end{tabular}
\columnbreak

Molecule B \
4NFAACl-3

\includegraphics[width=6cm]{../Comparisons/ImagesFromVMD/4NFAACl-3.png}

Inertia Tensor - Molecule B \\
\begin{tabular}{|c c c|}
506.608	 & 	0.709539	 & 	-0.555426	 \\
0.709539	 & 	1222.37	 & 	-2.84005	 \\
-0.555426	 & 	-2.84005	 & 	1678.41
\end{tabular}

\vtab
 EingenVectors - Molecule B     \\
\begin{tabular}{|c c c|}
-0.999999	 & 	0.000989428	 & 	-0.000471595	 \\
-0.000986471	 & 	-0.99998	 & 	-0.00622856	 \\
-0.000477748	 & 	-0.00622809	 & 	0.99998
\end{tabular}

\vtab
 EingenValues - Molecule B     \\
\begin{tabular}{|c c c|}
506.607	 & 	1222.36	 & 	1678.43	 \\
\end{tabular}

\end{center}
\end{multicols}

\vtab[-5mm]
\begin{tabular}{*{2}{m{0.38\textwidth}}}
\begin{center}
\textcolor{NavyBlue}{\Large Different}
\end{center}
&
\begin{center}
\includegraphics[height=6.5cm]{../Comparisons/Vectors/inertia_tensor_of_4NFAACg_and_4NFAACl-3.png}
\end{center}
\end{tabular}

 \newpage

\vtab[-3cm]
\begin{center}
{\large FireTest \tab Número 475}
\end{center}
\begin{multicols}{2}
\begin{center}

Molecule A \
4NFAACi

\includegraphics[width=6cm]{../Comparisons/ImagesFromVMD/4NFAACi.png}

Inertia Tensor - Molecule A \\
\begin{tabular}{|c c c|}
502.43	 & 	-0.602691	 & 	-4.86988	 \\
-0.602691	 & 	1232.26	 & 	0.407295	 \\
-4.86988	 & 	0.407295	 & 	1676
\end{tabular}

\vtab
 EingenVectors - Molecule A     \\
\begin{tabular}{|c c c|}
0.999991	 & 	0.000823447	 & 	0.00414923	 \\
0.000819608	 & 	-0.999999	 & 	0.00092687	 \\
-0.00414999	 & 	0.000923461	 & 	0.999991
\end{tabular}

\vtab
 EingenValues - Molecule A     \\
\begin{tabular}{|c c c|}
502.409	 & 	1232.26	 & 	1676.02	 \\
\end{tabular}
\columnbreak

Molecule B \
4NFAACj

\includegraphics[width=6cm]{../Comparisons/ImagesFromVMD/4NFAACj.png}

Inertia Tensor - Molecule B \\
\begin{tabular}{|c c c|}
510.047	 & 	9.97005	 & 	-3.6306	 \\
9.97005	 & 	1225.52	 & 	-0.981092	 \\
-3.6306	 & 	-0.981092	 & 	1680.82
\end{tabular}

\vtab
 EingenVectors - Molecule B     \\
\begin{tabular}{|c c c|}
-0.999898	 & 	0.0139264	 & 	-0.00308865	 \\
0.0139195	 & 	0.999901	 & 	0.00226627	 \\
-0.00311991	 & 	-0.00222305	 & 	0.999993
\end{tabular}

\vtab
 EingenValues - Molecule B     \\
\begin{tabular}{|c c c|}
509.897	 & 	1225.65	 & 	1680.83	 \\
\end{tabular}

\end{center}
\end{multicols}

\vtab[-5mm]
\begin{tabular}{*{2}{m{0.38\textwidth}}}
\begin{center}
\textcolor{NavyBlue}{\Large Different}
\end{center}
&
\begin{center}
\includegraphics[height=6.5cm]{../Comparisons/Vectors/inertia_tensor_of_4NFAACi_and_4NFAACj.png}
\end{center}
\end{tabular}

 \newpage

\vtab[-3cm]
\begin{center}
{\large FireTest \tab Número 476}
\end{center}
\begin{multicols}{2}
\begin{center}

Molecule A \
4NFAACi

\includegraphics[width=6cm]{../Comparisons/ImagesFromVMD/4NFAACi.png}

Inertia Tensor - Molecule A \\
\begin{tabular}{|c c c|}
502.43	 & 	-0.602691	 & 	-4.86988	 \\
-0.602691	 & 	1232.26	 & 	0.407295	 \\
-4.86988	 & 	0.407295	 & 	1676
\end{tabular}

\vtab
 EingenVectors - Molecule A     \\
\begin{tabular}{|c c c|}
0.999991	 & 	0.000823447	 & 	0.00414923	 \\
0.000819608	 & 	-0.999999	 & 	0.00092687	 \\
-0.00414999	 & 	0.000923461	 & 	0.999991
\end{tabular}

\vtab
 EingenValues - Molecule A     \\
\begin{tabular}{|c c c|}
502.409	 & 	1232.26	 & 	1676.02	 \\
\end{tabular}
\columnbreak

Molecule B \
4NFAACl-3

\includegraphics[width=6cm]{../Comparisons/ImagesFromVMD/4NFAACl-3.png}

Inertia Tensor - Molecule B \\
\begin{tabular}{|c c c|}
506.608	 & 	0.709539	 & 	-0.555426	 \\
0.709539	 & 	1222.37	 & 	-2.84005	 \\
-0.555426	 & 	-2.84005	 & 	1678.41
\end{tabular}

\vtab
 EingenVectors - Molecule B     \\
\begin{tabular}{|c c c|}
-0.999999	 & 	0.000989428	 & 	-0.000471595	 \\
-0.000986471	 & 	-0.99998	 & 	-0.00622856	 \\
-0.000477748	 & 	-0.00622809	 & 	0.99998
\end{tabular}

\vtab
 EingenValues - Molecule B     \\
\begin{tabular}{|c c c|}
506.607	 & 	1222.36	 & 	1678.43	 \\
\end{tabular}

\end{center}
\end{multicols}

\vtab[-5mm]
\begin{tabular}{*{2}{m{0.38\textwidth}}}
\begin{center}
\textcolor{NavyBlue}{\Large Different}
\end{center}
&
\begin{center}
\includegraphics[height=6.5cm]{../Comparisons/Vectors/inertia_tensor_of_4NFAACi_and_4NFAACl-3.png}
\end{center}
\end{tabular}

 \newpage

\vtab[-3cm]
\begin{center}
{\large FireTest \tab Número 477}
\end{center}
\begin{multicols}{2}
\begin{center}

Molecule A \
4NFAACj

\includegraphics[width=6cm]{../Comparisons/ImagesFromVMD/4NFAACj.png}

Inertia Tensor - Molecule A \\
\begin{tabular}{|c c c|}
510.047	 & 	9.97005	 & 	-3.6306	 \\
9.97005	 & 	1225.52	 & 	-0.981092	 \\
-3.6306	 & 	-0.981092	 & 	1680.82
\end{tabular}

\vtab
 EingenVectors - Molecule A     \\
\begin{tabular}{|c c c|}
-0.999898	 & 	0.0139264	 & 	-0.00308865	 \\
0.0139195	 & 	0.999901	 & 	0.00226627	 \\
-0.00311991	 & 	-0.00222305	 & 	0.999993
\end{tabular}

\vtab
 EingenValues - Molecule A     \\
\begin{tabular}{|c c c|}
509.897	 & 	1225.65	 & 	1680.83	 \\
\end{tabular}
\columnbreak

Molecule B \
4NFAACl-3

\includegraphics[width=6cm]{../Comparisons/ImagesFromVMD/4NFAACl-3.png}

Inertia Tensor - Molecule B \\
\begin{tabular}{|c c c|}
506.608	 & 	0.709539	 & 	-0.555426	 \\
0.709539	 & 	1222.37	 & 	-2.84005	 \\
-0.555426	 & 	-2.84005	 & 	1678.41
\end{tabular}

\vtab
 EingenVectors - Molecule B     \\
\begin{tabular}{|c c c|}
-0.999999	 & 	0.000989428	 & 	-0.000471595	 \\
-0.000986471	 & 	-0.99998	 & 	-0.00622856	 \\
-0.000477748	 & 	-0.00622809	 & 	0.99998
\end{tabular}

\vtab
 EingenValues - Molecule B     \\
\begin{tabular}{|c c c|}
506.607	 & 	1222.36	 & 	1678.43	 \\
\end{tabular}

\end{center}
\end{multicols}

\vtab[-5mm]
\begin{tabular}{*{2}{m{0.38\textwidth}}}
\begin{center}
\textcolor{NavyBlue}{\Large Different}
\end{center}
&
\begin{center}
\includegraphics[height=6.5cm]{../Comparisons/Vectors/inertia_tensor_of_4NFAACj_and_4NFAACl-3.png}
\end{center}
\end{tabular}

 \newpage

\vtab[-3cm]
\begin{center}
{\large IsomerC6H14 \tab Número 478}
\end{center}
\begin{multicols}{2}
\begin{center}

Molecule A \
2,2-dimethyl-butane\_out\_G09

\includegraphics[width=6cm]{../Comparisons/ImagesFromVMD/2,2-dimethyl-butane_out_G09.png}

Inertia Tensor - Molecule A \\
\begin{tabular}{|c c c|}
139.709	 & 	36.9725	 & 	0	 \\
36.9725	 & 	183.699	 & 	0	 \\
0	 & 	0	 & 	205.419
\end{tabular}

\vtab
 EingenVectors - Molecule A     \\
\begin{tabular}{|c c c|}
-0.869274	 & 	0.494331	 & 	0	 \\
0.494331	 & 	0.869274	 & 	0	 \\
0	 & 	0	 & 	1
\end{tabular}

\vtab
 EingenValues - Molecule A     \\
\begin{tabular}{|c c c|}
118.683	 & 	204.724	 & 	205.419	 \\
\end{tabular}
\columnbreak

Molecule B \
2,2-dimethyl-butane\_out\_G09\_invertion

\includegraphics[width=6cm]{../Comparisons/ImagesFromVMD/2,2-dimethyl-butane_out_G09_invertion.png}

Inertia Tensor - Molecule B \\
\begin{tabular}{|c c c|}
139.709	 & 	36.9725	 & 	0	 \\
36.9725	 & 	183.699	 & 	0	 \\
0	 & 	0	 & 	205.419
\end{tabular}

\vtab
 EingenVectors - Molecule B     \\
\begin{tabular}{|c c c|}
-0.869273	 & 	0.494332	 & 	0	 \\
0.494332	 & 	0.869273	 & 	0	 \\
0	 & 	0	 & 	1
\end{tabular}

\vtab
 EingenValues - Molecule B     \\
\begin{tabular}{|c c c|}
118.684	 & 	204.724	 & 	205.419	 \\
\end{tabular}

\end{center}
\end{multicols}

\vtab[-5mm]
\begin{tabular}{*{2}{m{0.38\textwidth}}}
\begin{center}
\textcolor{NavyBlue}{\Large Equal}
\end{center}
&
\begin{center}
\includegraphics[height=6.5cm]{../Comparisons/Vectors/inertia_tensor_of_2,2-dimethyl-butane_out_G09_and_2,2-dimethyl-butane_out_G09_invertion.png}
\end{center}
\end{tabular}

 \newpage

\vtab[-3cm]
\begin{center}
{\large IsomerC6H14 \tab Número 479}
\end{center}
\begin{multicols}{2}
\begin{center}

Molecule A \
2,2-dimethyl-butane\_out\_G09

\includegraphics[width=6cm]{../Comparisons/ImagesFromVMD/2,2-dimethyl-butane_out_G09.png}

Inertia Tensor - Molecule A \\
\begin{tabular}{|c c c|}
139.709	 & 	36.9725	 & 	0	 \\
36.9725	 & 	183.699	 & 	0	 \\
0	 & 	0	 & 	205.419
\end{tabular}

\vtab
 EingenVectors - Molecule A     \\
\begin{tabular}{|c c c|}
-0.869274	 & 	0.494331	 & 	0	 \\
0.494331	 & 	0.869274	 & 	0	 \\
0	 & 	0	 & 	1
\end{tabular}

\vtab
 EingenValues - Molecule A     \\
\begin{tabular}{|c c c|}
118.683	 & 	204.724	 & 	205.419	 \\
\end{tabular}
\columnbreak

Molecule B \
2,3-dimethyl-butane\_out\_G09

\includegraphics[width=6cm]{../Comparisons/ImagesFromVMD/2,3-dimethyl-butane_out_G09.png}

Inertia Tensor - Molecule B \\
\begin{tabular}{|c c c|}
270.146	 & 	-16.9873	 & 	0	 \\
-16.9873	 & 	122.541	 & 	0	 \\
0	 & 	0	 & 	177.08
\end{tabular}

\vtab
 EingenVectors - Molecule B     \\
\begin{tabular}{|c c c|}
-0.112876	 & 	-0.993609	 & 	0	 \\
0	 & 	0	 & 	1	 \\
0.993609	 & 	-0.112876	 & 	-0
\end{tabular}

\vtab
 EingenValues - Molecule B     \\
\begin{tabular}{|c c c|}
120.611	 & 	177.08	 & 	272.075	 \\
\end{tabular}

\end{center}
\end{multicols}

\vtab[-5mm]
\begin{tabular}{*{2}{m{0.38\textwidth}}}
\begin{center}
\textcolor{NavyBlue}{\Large Different}
\end{center}
&
\begin{center}
\includegraphics[height=6.5cm]{../Comparisons/Vectors/inertia_tensor_of_2,2-dimethyl-butane_out_G09_and_2,3-dimethyl-butane_out_G09.png}
\end{center}
\end{tabular}

 \newpage

\vtab[-3cm]
\begin{center}
{\large IsomerC6H14 \tab Número 480}
\end{center}
\begin{multicols}{2}
\begin{center}

Molecule A \
2,2-dimethyl-butane\_out\_G09

\includegraphics[width=6cm]{../Comparisons/ImagesFromVMD/2,2-dimethyl-butane_out_G09.png}

Inertia Tensor - Molecule A \\
\begin{tabular}{|c c c|}
139.709	 & 	36.9725	 & 	0	 \\
36.9725	 & 	183.699	 & 	0	 \\
0	 & 	0	 & 	205.419
\end{tabular}

\vtab
 EingenVectors - Molecule A     \\
\begin{tabular}{|c c c|}
-0.869274	 & 	0.494331	 & 	0	 \\
0.494331	 & 	0.869274	 & 	0	 \\
0	 & 	0	 & 	1
\end{tabular}

\vtab
 EingenValues - Molecule A     \\
\begin{tabular}{|c c c|}
118.683	 & 	204.724	 & 	205.419	 \\
\end{tabular}
\columnbreak

Molecule B \
2,3-dimethyl-butane\_out\_G09\_invertion

\includegraphics[width=6cm]{../Comparisons/ImagesFromVMD/2,3-dimethyl-butane_out_G09_invertion.png}

Inertia Tensor - Molecule B \\
\begin{tabular}{|c c c|}
270.146	 & 	-16.9873	 & 	0	 \\
-16.9873	 & 	122.541	 & 	0	 \\
0	 & 	0	 & 	177.08
\end{tabular}

\vtab
 EingenVectors - Molecule B     \\
\begin{tabular}{|c c c|}
-0.112875	 & 	-0.993609	 & 	0	 \\
0	 & 	0	 & 	1	 \\
0.993609	 & 	-0.112875	 & 	-0
\end{tabular}

\vtab
 EingenValues - Molecule B     \\
\begin{tabular}{|c c c|}
120.611	 & 	177.08	 & 	272.076	 \\
\end{tabular}

\end{center}
\end{multicols}

\vtab[-5mm]
\begin{tabular}{*{2}{m{0.38\textwidth}}}
\begin{center}
\textcolor{NavyBlue}{\Large Different}
\end{center}
&
\begin{center}
\includegraphics[height=6.5cm]{../Comparisons/Vectors/inertia_tensor_of_2,2-dimethyl-butane_out_G09_and_2,3-dimethyl-butane_out_G09_invertion.png}
\end{center}
\end{tabular}

 \newpage

\vtab[-3cm]
\begin{center}
{\large IsomerC6H14 \tab Número 481}
\end{center}
\begin{multicols}{2}
\begin{center}

Molecule A \
2,2-dimethyl-butane\_out\_G09

\includegraphics[width=6cm]{../Comparisons/ImagesFromVMD/2,2-dimethyl-butane_out_G09.png}

Inertia Tensor - Molecule A \\
\begin{tabular}{|c c c|}
139.709	 & 	36.9725	 & 	0	 \\
36.9725	 & 	183.699	 & 	0	 \\
0	 & 	0	 & 	205.419
\end{tabular}

\vtab
 EingenVectors - Molecule A     \\
\begin{tabular}{|c c c|}
-0.869274	 & 	0.494331	 & 	0	 \\
0.494331	 & 	0.869274	 & 	0	 \\
0	 & 	0	 & 	1
\end{tabular}

\vtab
 EingenValues - Molecule A     \\
\begin{tabular}{|c c c|}
118.683	 & 	204.724	 & 	205.419	 \\
\end{tabular}
\columnbreak

Molecule B \
2-methyl-pentane\_out\_G09

\includegraphics[width=6cm]{../Comparisons/ImagesFromVMD/2-methyl-pentane_out_G09.png}

Inertia Tensor - Molecule B \\
\begin{tabular}{|c c c|}
76.7701	 & 	-0.234564	 & 	-0.159008	 \\
-0.234564	 & 	299.061	 & 	-0.102167	 \\
-0.159008	 & 	-0.102167	 & 	347.557
\end{tabular}

\vtab
 EingenVectors - Molecule B     \\
\begin{tabular}{|c c c|}
-0.999999	 & 	-0.00105548	 & 	-0.000587604	 \\
-0.00105671	 & 	0.999997	 & 	0.00210321	 \\
-0.000585382	 & 	-0.00210383	 & 	0.999998
\end{tabular}

\vtab
 EingenValues - Molecule B     \\
\begin{tabular}{|c c c|}
76.7698	 & 	299.061	 & 	347.558	 \\
\end{tabular}

\end{center}
\end{multicols}

\vtab[-5mm]
\begin{tabular}{*{2}{m{0.38\textwidth}}}
\begin{center}
\textcolor{NavyBlue}{\Large Different}
\end{center}
&
\begin{center}
\includegraphics[height=6.5cm]{../Comparisons/Vectors/inertia_tensor_of_2,2-dimethyl-butane_out_G09_and_2-methyl-pentane_out_G09.png}
\end{center}
\end{tabular}

 \newpage

\vtab[-3cm]
\begin{center}
{\large IsomerC6H14 \tab Número 482}
\end{center}
\begin{multicols}{2}
\begin{center}

Molecule A \
2,2-dimethyl-butane\_out\_G09

\includegraphics[width=6cm]{../Comparisons/ImagesFromVMD/2,2-dimethyl-butane_out_G09.png}

Inertia Tensor - Molecule A \\
\begin{tabular}{|c c c|}
139.709	 & 	36.9725	 & 	0	 \\
36.9725	 & 	183.699	 & 	0	 \\
0	 & 	0	 & 	205.419
\end{tabular}

\vtab
 EingenVectors - Molecule A     \\
\begin{tabular}{|c c c|}
-0.869274	 & 	0.494331	 & 	0	 \\
0.494331	 & 	0.869274	 & 	0	 \\
0	 & 	0	 & 	1
\end{tabular}

\vtab
 EingenValues - Molecule A     \\
\begin{tabular}{|c c c|}
118.683	 & 	204.724	 & 	205.419	 \\
\end{tabular}
\columnbreak

Molecule B \
2-methyl-pentane\_out\_G09\_invertion

\includegraphics[width=6cm]{../Comparisons/ImagesFromVMD/2-methyl-pentane_out_G09_invertion.png}

Inertia Tensor - Molecule B \\
\begin{tabular}{|c c c|}
76.7702	 & 	-0.234595	 & 	-0.158982	 \\
-0.234595	 & 	299.061	 & 	-0.102162	 \\
-0.158982	 & 	-0.102162	 & 	347.558
\end{tabular}

\vtab
 EingenVectors - Molecule B     \\
\begin{tabular}{|c c c|}
-0.999999	 & 	-0.00105562	 & 	-0.000587505	 \\
-0.00105685	 & 	0.999997	 & 	0.00210312	 \\
-0.000585284	 & 	-0.00210374	 & 	0.999998
\end{tabular}

\vtab
 EingenValues - Molecule B     \\
\begin{tabular}{|c c c|}
76.7699	 & 	299.061	 & 	347.558	 \\
\end{tabular}

\end{center}
\end{multicols}

\vtab[-5mm]
\begin{tabular}{*{2}{m{0.38\textwidth}}}
\begin{center}
\textcolor{NavyBlue}{\Large Different}
\end{center}
&
\begin{center}
\includegraphics[height=6.5cm]{../Comparisons/Vectors/inertia_tensor_of_2,2-dimethyl-butane_out_G09_and_2-methyl-pentane_out_G09_invertion.png}
\end{center}
\end{tabular}

 \newpage

\vtab[-3cm]
\begin{center}
{\large IsomerC6H14 \tab Número 483}
\end{center}
\begin{multicols}{2}
\begin{center}

Molecule A \
2,2-dimethyl-butane\_out\_G09

\includegraphics[width=6cm]{../Comparisons/ImagesFromVMD/2,2-dimethyl-butane_out_G09.png}

Inertia Tensor - Molecule A \\
\begin{tabular}{|c c c|}
139.709	 & 	36.9725	 & 	0	 \\
36.9725	 & 	183.699	 & 	0	 \\
0	 & 	0	 & 	205.419
\end{tabular}

\vtab
 EingenVectors - Molecule A     \\
\begin{tabular}{|c c c|}
-0.869274	 & 	0.494331	 & 	0	 \\
0.494331	 & 	0.869274	 & 	0	 \\
0	 & 	0	 & 	1
\end{tabular}

\vtab
 EingenValues - Molecule A     \\
\begin{tabular}{|c c c|}
118.683	 & 	204.724	 & 	205.419	 \\
\end{tabular}
\columnbreak

Molecule B \
3-methyl-pentane\_out\_G09

\includegraphics[width=6cm]{../Comparisons/ImagesFromVMD/3-methyl-pentane_out_G09.png}

Inertia Tensor - Molecule B \\
\begin{tabular}{|c c c|}
294.318	 & 	21.9352	 & 	0	 \\
21.9352	 & 	302.288	 & 	0	 \\
0	 & 	0	 & 	77.8079
\end{tabular}

\vtab
 EingenVectors - Molecule B     \\
\begin{tabular}{|c c c|}
0	 & 	0	 & 	1	 \\
-0.767709	 & 	0.640799	 & 	0	 \\
0.640799	 & 	0.767709	 & 	0
\end{tabular}

\vtab
 EingenValues - Molecule B     \\
\begin{tabular}{|c c c|}
77.8079	 & 	276.009	 & 	320.597	 \\
\end{tabular}

\end{center}
\end{multicols}

\vtab[-5mm]
\begin{tabular}{*{2}{m{0.38\textwidth}}}
\begin{center}
\textcolor{NavyBlue}{\Large Different}
\end{center}
&
\begin{center}
\includegraphics[height=6.5cm]{../Comparisons/Vectors/inertia_tensor_of_2,2-dimethyl-butane_out_G09_and_3-methyl-pentane_out_G09.png}
\end{center}
\end{tabular}

 \newpage

\vtab[-3cm]
\begin{center}
{\large IsomerC6H14 \tab Número 484}
\end{center}
\begin{multicols}{2}
\begin{center}

Molecule A \
2,2-dimethyl-butane\_out\_G09

\includegraphics[width=6cm]{../Comparisons/ImagesFromVMD/2,2-dimethyl-butane_out_G09.png}

Inertia Tensor - Molecule A \\
\begin{tabular}{|c c c|}
139.709	 & 	36.9725	 & 	0	 \\
36.9725	 & 	183.699	 & 	0	 \\
0	 & 	0	 & 	205.419
\end{tabular}

\vtab
 EingenVectors - Molecule A     \\
\begin{tabular}{|c c c|}
-0.869274	 & 	0.494331	 & 	0	 \\
0.494331	 & 	0.869274	 & 	0	 \\
0	 & 	0	 & 	1
\end{tabular}

\vtab
 EingenValues - Molecule A     \\
\begin{tabular}{|c c c|}
118.683	 & 	204.724	 & 	205.419	 \\
\end{tabular}
\columnbreak

Molecule B \
3-methyl-pentane\_out\_G09\_invertion

\includegraphics[width=6cm]{../Comparisons/ImagesFromVMD/3-methyl-pentane_out_G09_invertion.png}

Inertia Tensor - Molecule B \\
\begin{tabular}{|c c c|}
294.318	 & 	21.9353	 & 	0	 \\
21.9353	 & 	302.289	 & 	0	 \\
0	 & 	0	 & 	77.8081
\end{tabular}

\vtab
 EingenVectors - Molecule B     \\
\begin{tabular}{|c c c|}
0	 & 	0	 & 	1	 \\
-0.76771	 & 	0.640798	 & 	0	 \\
0.640798	 & 	0.76771	 & 	0
\end{tabular}

\vtab
 EingenValues - Molecule B     \\
\begin{tabular}{|c c c|}
77.8081	 & 	276.009	 & 	320.598	 \\
\end{tabular}

\end{center}
\end{multicols}

\vtab[-5mm]
\begin{tabular}{*{2}{m{0.38\textwidth}}}
\begin{center}
\textcolor{NavyBlue}{\Large Different}
\end{center}
&
\begin{center}
\includegraphics[height=6.5cm]{../Comparisons/Vectors/inertia_tensor_of_2,2-dimethyl-butane_out_G09_and_3-methyl-pentane_out_G09_invertion.png}
\end{center}
\end{tabular}

 \newpage

\vtab[-3cm]
\begin{center}
{\large IsomerC6H14 \tab Número 485}
\end{center}
\begin{multicols}{2}
\begin{center}

Molecule A \
2,2-dimethyl-butane\_out\_G09

\includegraphics[width=6cm]{../Comparisons/ImagesFromVMD/2,2-dimethyl-butane_out_G09.png}

Inertia Tensor - Molecule A \\
\begin{tabular}{|c c c|}
139.709	 & 	36.9725	 & 	0	 \\
36.9725	 & 	183.699	 & 	0	 \\
0	 & 	0	 & 	205.419
\end{tabular}

\vtab
 EingenVectors - Molecule A     \\
\begin{tabular}{|c c c|}
-0.869274	 & 	0.494331	 & 	0	 \\
0.494331	 & 	0.869274	 & 	0	 \\
0	 & 	0	 & 	1
\end{tabular}

\vtab
 EingenValues - Molecule A     \\
\begin{tabular}{|c c c|}
118.683	 & 	204.724	 & 	205.419	 \\
\end{tabular}
\columnbreak

Molecule B \
hexane\_out\_G09

\includegraphics[width=6cm]{../Comparisons/ImagesFromVMD/hexane_out_G09.png}

Inertia Tensor - Molecule B \\
\begin{tabular}{|c c c|}
346.244	 & 	-179.129	 & 	0	 \\
-179.129	 & 	137.676	 & 	0	 \\
0	 & 	0	 & 	465.177
\end{tabular}

\vtab
 EingenVectors - Molecule B     \\
\begin{tabular}{|c c c|}
-0.498436	 & 	-0.866927	 & 	0	 \\
0.866927	 & 	-0.498436	 & 	-0	 \\
0	 & 	0	 & 	1
\end{tabular}

\vtab
 EingenValues - Molecule B     \\
\begin{tabular}{|c c c|}
34.6869	 & 	449.234	 & 	465.177	 \\
\end{tabular}

\end{center}
\end{multicols}

\vtab[-5mm]
\begin{tabular}{*{2}{m{0.38\textwidth}}}
\begin{center}
\textcolor{NavyBlue}{\Large Different}
\end{center}
&
\begin{center}
\includegraphics[height=6.5cm]{../Comparisons/Vectors/inertia_tensor_of_2,2-dimethyl-butane_out_G09_and_hexane_out_G09.png}
\end{center}
\end{tabular}

 \newpage

\vtab[-3cm]
\begin{center}
{\large IsomerC6H14 \tab Número 486}
\end{center}
\begin{multicols}{2}
\begin{center}

Molecule A \
2,2-dimethyl-butane\_out\_G09

\includegraphics[width=6cm]{../Comparisons/ImagesFromVMD/2,2-dimethyl-butane_out_G09.png}

Inertia Tensor - Molecule A \\
\begin{tabular}{|c c c|}
139.709	 & 	36.9725	 & 	0	 \\
36.9725	 & 	183.699	 & 	0	 \\
0	 & 	0	 & 	205.419
\end{tabular}

\vtab
 EingenVectors - Molecule A     \\
\begin{tabular}{|c c c|}
-0.869274	 & 	0.494331	 & 	0	 \\
0.494331	 & 	0.869274	 & 	0	 \\
0	 & 	0	 & 	1
\end{tabular}

\vtab
 EingenValues - Molecule A     \\
\begin{tabular}{|c c c|}
118.683	 & 	204.724	 & 	205.419	 \\
\end{tabular}
\columnbreak

Molecule B \
hexane\_out\_G09\_invertion

\includegraphics[width=6cm]{../Comparisons/ImagesFromVMD/hexane_out_G09_invertion.png}

Inertia Tensor - Molecule B \\
\begin{tabular}{|c c c|}
346.245	 & 	-179.129	 & 	0	 \\
-179.129	 & 	137.676	 & 	0	 \\
0	 & 	0	 & 	465.177
\end{tabular}

\vtab
 EingenVectors - Molecule B     \\
\begin{tabular}{|c c c|}
-0.498435	 & 	-0.866927	 & 	0	 \\
0.866927	 & 	-0.498435	 & 	-0	 \\
0	 & 	0	 & 	1
\end{tabular}

\vtab
 EingenValues - Molecule B     \\
\begin{tabular}{|c c c|}
34.6868	 & 	449.234	 & 	465.177	 \\
\end{tabular}

\end{center}
\end{multicols}

\vtab[-5mm]
\begin{tabular}{*{2}{m{0.38\textwidth}}}
\begin{center}
\textcolor{NavyBlue}{\Large Different}
\end{center}
&
\begin{center}
\includegraphics[height=6.5cm]{../Comparisons/Vectors/inertia_tensor_of_2,2-dimethyl-butane_out_G09_and_hexane_out_G09_invertion.png}
\end{center}
\end{tabular}

 \newpage

\vtab[-3cm]
\begin{center}
{\large IsomerC6H14 \tab Número 487}
\end{center}
\begin{multicols}{2}
\begin{center}

Molecule A \
2,2-dimethyl-butane\_out\_G09\_invertion

\includegraphics[width=6cm]{../Comparisons/ImagesFromVMD/2,2-dimethyl-butane_out_G09_invertion.png}

Inertia Tensor - Molecule A \\
\begin{tabular}{|c c c|}
139.709	 & 	36.9725	 & 	0	 \\
36.9725	 & 	183.699	 & 	0	 \\
0	 & 	0	 & 	205.419
\end{tabular}

\vtab
 EingenVectors - Molecule A     \\
\begin{tabular}{|c c c|}
-0.869273	 & 	0.494332	 & 	0	 \\
0.494332	 & 	0.869273	 & 	0	 \\
0	 & 	0	 & 	1
\end{tabular}

\vtab
 EingenValues - Molecule A     \\
\begin{tabular}{|c c c|}
118.684	 & 	204.724	 & 	205.419	 \\
\end{tabular}
\columnbreak

Molecule B \
2,3-dimethyl-butane\_out\_G09

\includegraphics[width=6cm]{../Comparisons/ImagesFromVMD/2,3-dimethyl-butane_out_G09.png}

Inertia Tensor - Molecule B \\
\begin{tabular}{|c c c|}
270.146	 & 	-16.9873	 & 	0	 \\
-16.9873	 & 	122.541	 & 	0	 \\
0	 & 	0	 & 	177.08
\end{tabular}

\vtab
 EingenVectors - Molecule B     \\
\begin{tabular}{|c c c|}
-0.112876	 & 	-0.993609	 & 	0	 \\
0	 & 	0	 & 	1	 \\
0.993609	 & 	-0.112876	 & 	-0
\end{tabular}

\vtab
 EingenValues - Molecule B     \\
\begin{tabular}{|c c c|}
120.611	 & 	177.08	 & 	272.075	 \\
\end{tabular}

\end{center}
\end{multicols}

\vtab[-5mm]
\begin{tabular}{*{2}{m{0.38\textwidth}}}
\begin{center}
\textcolor{NavyBlue}{\Large Different}
\end{center}
&
\begin{center}
\includegraphics[height=6.5cm]{../Comparisons/Vectors/inertia_tensor_of_2,2-dimethyl-butane_out_G09_invertion_and_2,3-dimethyl-butane_out_G09.png}
\end{center}
\end{tabular}

 \newpage

\vtab[-3cm]
\begin{center}
{\large IsomerC6H14 \tab Número 488}
\end{center}
\begin{multicols}{2}
\begin{center}

Molecule A \
2,2-dimethyl-butane\_out\_G09\_invertion

\includegraphics[width=6cm]{../Comparisons/ImagesFromVMD/2,2-dimethyl-butane_out_G09_invertion.png}

Inertia Tensor - Molecule A \\
\begin{tabular}{|c c c|}
139.709	 & 	36.9725	 & 	0	 \\
36.9725	 & 	183.699	 & 	0	 \\
0	 & 	0	 & 	205.419
\end{tabular}

\vtab
 EingenVectors - Molecule A     \\
\begin{tabular}{|c c c|}
-0.869273	 & 	0.494332	 & 	0	 \\
0.494332	 & 	0.869273	 & 	0	 \\
0	 & 	0	 & 	1
\end{tabular}

\vtab
 EingenValues - Molecule A     \\
\begin{tabular}{|c c c|}
118.684	 & 	204.724	 & 	205.419	 \\
\end{tabular}
\columnbreak

Molecule B \
2,3-dimethyl-butane\_out\_G09\_invertion

\includegraphics[width=6cm]{../Comparisons/ImagesFromVMD/2,3-dimethyl-butane_out_G09_invertion.png}

Inertia Tensor - Molecule B \\
\begin{tabular}{|c c c|}
270.146	 & 	-16.9873	 & 	0	 \\
-16.9873	 & 	122.541	 & 	0	 \\
0	 & 	0	 & 	177.08
\end{tabular}

\vtab
 EingenVectors - Molecule B     \\
\begin{tabular}{|c c c|}
-0.112875	 & 	-0.993609	 & 	0	 \\
0	 & 	0	 & 	1	 \\
0.993609	 & 	-0.112875	 & 	-0
\end{tabular}

\vtab
 EingenValues - Molecule B     \\
\begin{tabular}{|c c c|}
120.611	 & 	177.08	 & 	272.076	 \\
\end{tabular}

\end{center}
\end{multicols}

\vtab[-5mm]
\begin{tabular}{*{2}{m{0.38\textwidth}}}
\begin{center}
\textcolor{NavyBlue}{\Large Different}
\end{center}
&
\begin{center}
\includegraphics[height=6.5cm]{../Comparisons/Vectors/inertia_tensor_of_2,2-dimethyl-butane_out_G09_invertion_and_2,3-dimethyl-butane_out_G09_invertion.png}
\end{center}
\end{tabular}

 \newpage

\vtab[-3cm]
\begin{center}
{\large IsomerC6H14 \tab Número 489}
\end{center}
\begin{multicols}{2}
\begin{center}

Molecule A \
2,2-dimethyl-butane\_out\_G09\_invertion

\includegraphics[width=6cm]{../Comparisons/ImagesFromVMD/2,2-dimethyl-butane_out_G09_invertion.png}

Inertia Tensor - Molecule A \\
\begin{tabular}{|c c c|}
139.709	 & 	36.9725	 & 	0	 \\
36.9725	 & 	183.699	 & 	0	 \\
0	 & 	0	 & 	205.419
\end{tabular}

\vtab
 EingenVectors - Molecule A     \\
\begin{tabular}{|c c c|}
-0.869273	 & 	0.494332	 & 	0	 \\
0.494332	 & 	0.869273	 & 	0	 \\
0	 & 	0	 & 	1
\end{tabular}

\vtab
 EingenValues - Molecule A     \\
\begin{tabular}{|c c c|}
118.684	 & 	204.724	 & 	205.419	 \\
\end{tabular}
\columnbreak

Molecule B \
2-methyl-pentane\_out\_G09

\includegraphics[width=6cm]{../Comparisons/ImagesFromVMD/2-methyl-pentane_out_G09.png}

Inertia Tensor - Molecule B \\
\begin{tabular}{|c c c|}
76.7701	 & 	-0.234564	 & 	-0.159008	 \\
-0.234564	 & 	299.061	 & 	-0.102167	 \\
-0.159008	 & 	-0.102167	 & 	347.557
\end{tabular}

\vtab
 EingenVectors - Molecule B     \\
\begin{tabular}{|c c c|}
-0.999999	 & 	-0.00105548	 & 	-0.000587604	 \\
-0.00105671	 & 	0.999997	 & 	0.00210321	 \\
-0.000585382	 & 	-0.00210383	 & 	0.999998
\end{tabular}

\vtab
 EingenValues - Molecule B     \\
\begin{tabular}{|c c c|}
76.7698	 & 	299.061	 & 	347.558	 \\
\end{tabular}

\end{center}
\end{multicols}

\vtab[-5mm]
\begin{tabular}{*{2}{m{0.38\textwidth}}}
\begin{center}
\textcolor{NavyBlue}{\Large Different}
\end{center}
&
\begin{center}
\includegraphics[height=6.5cm]{../Comparisons/Vectors/inertia_tensor_of_2,2-dimethyl-butane_out_G09_invertion_and_2-methyl-pentane_out_G09.png}
\end{center}
\end{tabular}

 \newpage

\vtab[-3cm]
\begin{center}
{\large IsomerC6H14 \tab Número 490}
\end{center}
\begin{multicols}{2}
\begin{center}

Molecule A \
2,2-dimethyl-butane\_out\_G09\_invertion

\includegraphics[width=6cm]{../Comparisons/ImagesFromVMD/2,2-dimethyl-butane_out_G09_invertion.png}

Inertia Tensor - Molecule A \\
\begin{tabular}{|c c c|}
139.709	 & 	36.9725	 & 	0	 \\
36.9725	 & 	183.699	 & 	0	 \\
0	 & 	0	 & 	205.419
\end{tabular}

\vtab
 EingenVectors - Molecule A     \\
\begin{tabular}{|c c c|}
-0.869273	 & 	0.494332	 & 	0	 \\
0.494332	 & 	0.869273	 & 	0	 \\
0	 & 	0	 & 	1
\end{tabular}

\vtab
 EingenValues - Molecule A     \\
\begin{tabular}{|c c c|}
118.684	 & 	204.724	 & 	205.419	 \\
\end{tabular}
\columnbreak

Molecule B \
2-methyl-pentane\_out\_G09\_invertion

\includegraphics[width=6cm]{../Comparisons/ImagesFromVMD/2-methyl-pentane_out_G09_invertion.png}

Inertia Tensor - Molecule B \\
\begin{tabular}{|c c c|}
76.7702	 & 	-0.234595	 & 	-0.158982	 \\
-0.234595	 & 	299.061	 & 	-0.102162	 \\
-0.158982	 & 	-0.102162	 & 	347.558
\end{tabular}

\vtab
 EingenVectors - Molecule B     \\
\begin{tabular}{|c c c|}
-0.999999	 & 	-0.00105562	 & 	-0.000587505	 \\
-0.00105685	 & 	0.999997	 & 	0.00210312	 \\
-0.000585284	 & 	-0.00210374	 & 	0.999998
\end{tabular}

\vtab
 EingenValues - Molecule B     \\
\begin{tabular}{|c c c|}
76.7699	 & 	299.061	 & 	347.558	 \\
\end{tabular}

\end{center}
\end{multicols}

\vtab[-5mm]
\begin{tabular}{*{2}{m{0.38\textwidth}}}
\begin{center}
\textcolor{NavyBlue}{\Large Different}
\end{center}
&
\begin{center}
\includegraphics[height=6.5cm]{../Comparisons/Vectors/inertia_tensor_of_2,2-dimethyl-butane_out_G09_invertion_and_2-methyl-pentane_out_G09_invertion.png}
\end{center}
\end{tabular}

 \newpage

\vtab[-3cm]
\begin{center}
{\large IsomerC6H14 \tab Número 491}
\end{center}
\begin{multicols}{2}
\begin{center}

Molecule A \
2,2-dimethyl-butane\_out\_G09\_invertion

\includegraphics[width=6cm]{../Comparisons/ImagesFromVMD/2,2-dimethyl-butane_out_G09_invertion.png}

Inertia Tensor - Molecule A \\
\begin{tabular}{|c c c|}
139.709	 & 	36.9725	 & 	0	 \\
36.9725	 & 	183.699	 & 	0	 \\
0	 & 	0	 & 	205.419
\end{tabular}

\vtab
 EingenVectors - Molecule A     \\
\begin{tabular}{|c c c|}
-0.869273	 & 	0.494332	 & 	0	 \\
0.494332	 & 	0.869273	 & 	0	 \\
0	 & 	0	 & 	1
\end{tabular}

\vtab
 EingenValues - Molecule A     \\
\begin{tabular}{|c c c|}
118.684	 & 	204.724	 & 	205.419	 \\
\end{tabular}
\columnbreak

Molecule B \
3-methyl-pentane\_out\_G09

\includegraphics[width=6cm]{../Comparisons/ImagesFromVMD/3-methyl-pentane_out_G09.png}

Inertia Tensor - Molecule B \\
\begin{tabular}{|c c c|}
294.318	 & 	21.9352	 & 	0	 \\
21.9352	 & 	302.288	 & 	0	 \\
0	 & 	0	 & 	77.8079
\end{tabular}

\vtab
 EingenVectors - Molecule B     \\
\begin{tabular}{|c c c|}
0	 & 	0	 & 	1	 \\
-0.767709	 & 	0.640799	 & 	0	 \\
0.640799	 & 	0.767709	 & 	0
\end{tabular}

\vtab
 EingenValues - Molecule B     \\
\begin{tabular}{|c c c|}
77.8079	 & 	276.009	 & 	320.597	 \\
\end{tabular}

\end{center}
\end{multicols}

\vtab[-5mm]
\begin{tabular}{*{2}{m{0.38\textwidth}}}
\begin{center}
\textcolor{NavyBlue}{\Large Different}
\end{center}
&
\begin{center}
\includegraphics[height=6.5cm]{../Comparisons/Vectors/inertia_tensor_of_2,2-dimethyl-butane_out_G09_invertion_and_3-methyl-pentane_out_G09.png}
\end{center}
\end{tabular}

 \newpage

\vtab[-3cm]
\begin{center}
{\large IsomerC6H14 \tab Número 492}
\end{center}
\begin{multicols}{2}
\begin{center}

Molecule A \
2,2-dimethyl-butane\_out\_G09\_invertion

\includegraphics[width=6cm]{../Comparisons/ImagesFromVMD/2,2-dimethyl-butane_out_G09_invertion.png}

Inertia Tensor - Molecule A \\
\begin{tabular}{|c c c|}
139.709	 & 	36.9725	 & 	0	 \\
36.9725	 & 	183.699	 & 	0	 \\
0	 & 	0	 & 	205.419
\end{tabular}

\vtab
 EingenVectors - Molecule A     \\
\begin{tabular}{|c c c|}
-0.869273	 & 	0.494332	 & 	0	 \\
0.494332	 & 	0.869273	 & 	0	 \\
0	 & 	0	 & 	1
\end{tabular}

\vtab
 EingenValues - Molecule A     \\
\begin{tabular}{|c c c|}
118.684	 & 	204.724	 & 	205.419	 \\
\end{tabular}
\columnbreak

Molecule B \
3-methyl-pentane\_out\_G09\_invertion

\includegraphics[width=6cm]{../Comparisons/ImagesFromVMD/3-methyl-pentane_out_G09_invertion.png}

Inertia Tensor - Molecule B \\
\begin{tabular}{|c c c|}
294.318	 & 	21.9353	 & 	0	 \\
21.9353	 & 	302.289	 & 	0	 \\
0	 & 	0	 & 	77.8081
\end{tabular}

\vtab
 EingenVectors - Molecule B     \\
\begin{tabular}{|c c c|}
0	 & 	0	 & 	1	 \\
-0.76771	 & 	0.640798	 & 	0	 \\
0.640798	 & 	0.76771	 & 	0
\end{tabular}

\vtab
 EingenValues - Molecule B     \\
\begin{tabular}{|c c c|}
77.8081	 & 	276.009	 & 	320.598	 \\
\end{tabular}

\end{center}
\end{multicols}

\vtab[-5mm]
\begin{tabular}{*{2}{m{0.38\textwidth}}}
\begin{center}
\textcolor{NavyBlue}{\Large Different}
\end{center}
&
\begin{center}
\includegraphics[height=6.5cm]{../Comparisons/Vectors/inertia_tensor_of_2,2-dimethyl-butane_out_G09_invertion_and_3-methyl-pentane_out_G09_invertion.png}
\end{center}
\end{tabular}

 \newpage

\vtab[-3cm]
\begin{center}
{\large IsomerC6H14 \tab Número 493}
\end{center}
\begin{multicols}{2}
\begin{center}

Molecule A \
2,2-dimethyl-butane\_out\_G09\_invertion

\includegraphics[width=6cm]{../Comparisons/ImagesFromVMD/2,2-dimethyl-butane_out_G09_invertion.png}

Inertia Tensor - Molecule A \\
\begin{tabular}{|c c c|}
139.709	 & 	36.9725	 & 	0	 \\
36.9725	 & 	183.699	 & 	0	 \\
0	 & 	0	 & 	205.419
\end{tabular}

\vtab
 EingenVectors - Molecule A     \\
\begin{tabular}{|c c c|}
-0.869273	 & 	0.494332	 & 	0	 \\
0.494332	 & 	0.869273	 & 	0	 \\
0	 & 	0	 & 	1
\end{tabular}

\vtab
 EingenValues - Molecule A     \\
\begin{tabular}{|c c c|}
118.684	 & 	204.724	 & 	205.419	 \\
\end{tabular}
\columnbreak

Molecule B \
hexane\_out\_G09

\includegraphics[width=6cm]{../Comparisons/ImagesFromVMD/hexane_out_G09.png}

Inertia Tensor - Molecule B \\
\begin{tabular}{|c c c|}
346.244	 & 	-179.129	 & 	0	 \\
-179.129	 & 	137.676	 & 	0	 \\
0	 & 	0	 & 	465.177
\end{tabular}

\vtab
 EingenVectors - Molecule B     \\
\begin{tabular}{|c c c|}
-0.498436	 & 	-0.866927	 & 	0	 \\
0.866927	 & 	-0.498436	 & 	-0	 \\
0	 & 	0	 & 	1
\end{tabular}

\vtab
 EingenValues - Molecule B     \\
\begin{tabular}{|c c c|}
34.6869	 & 	449.234	 & 	465.177	 \\
\end{tabular}

\end{center}
\end{multicols}

\vtab[-5mm]
\begin{tabular}{*{2}{m{0.38\textwidth}}}
\begin{center}
\textcolor{NavyBlue}{\Large Different}
\end{center}
&
\begin{center}
\includegraphics[height=6.5cm]{../Comparisons/Vectors/inertia_tensor_of_2,2-dimethyl-butane_out_G09_invertion_and_hexane_out_G09.png}
\end{center}
\end{tabular}

 \newpage

\vtab[-3cm]
\begin{center}
{\large IsomerC6H14 \tab Número 494}
\end{center}
\begin{multicols}{2}
\begin{center}

Molecule A \
2,2-dimethyl-butane\_out\_G09\_invertion

\includegraphics[width=6cm]{../Comparisons/ImagesFromVMD/2,2-dimethyl-butane_out_G09_invertion.png}

Inertia Tensor - Molecule A \\
\begin{tabular}{|c c c|}
139.709	 & 	36.9725	 & 	0	 \\
36.9725	 & 	183.699	 & 	0	 \\
0	 & 	0	 & 	205.419
\end{tabular}

\vtab
 EingenVectors - Molecule A     \\
\begin{tabular}{|c c c|}
-0.869273	 & 	0.494332	 & 	0	 \\
0.494332	 & 	0.869273	 & 	0	 \\
0	 & 	0	 & 	1
\end{tabular}

\vtab
 EingenValues - Molecule A     \\
\begin{tabular}{|c c c|}
118.684	 & 	204.724	 & 	205.419	 \\
\end{tabular}
\columnbreak

Molecule B \
hexane\_out\_G09\_invertion

\includegraphics[width=6cm]{../Comparisons/ImagesFromVMD/hexane_out_G09_invertion.png}

Inertia Tensor - Molecule B \\
\begin{tabular}{|c c c|}
346.245	 & 	-179.129	 & 	0	 \\
-179.129	 & 	137.676	 & 	0	 \\
0	 & 	0	 & 	465.177
\end{tabular}

\vtab
 EingenVectors - Molecule B     \\
\begin{tabular}{|c c c|}
-0.498435	 & 	-0.866927	 & 	0	 \\
0.866927	 & 	-0.498435	 & 	-0	 \\
0	 & 	0	 & 	1
\end{tabular}

\vtab
 EingenValues - Molecule B     \\
\begin{tabular}{|c c c|}
34.6868	 & 	449.234	 & 	465.177	 \\
\end{tabular}

\end{center}
\end{multicols}

\vtab[-5mm]
\begin{tabular}{*{2}{m{0.38\textwidth}}}
\begin{center}
\textcolor{NavyBlue}{\Large Different}
\end{center}
&
\begin{center}
\includegraphics[height=6.5cm]{../Comparisons/Vectors/inertia_tensor_of_2,2-dimethyl-butane_out_G09_invertion_and_hexane_out_G09_invertion.png}
\end{center}
\end{tabular}

 \newpage

\vtab[-3cm]
\begin{center}
{\large IsomerC6H14 \tab Número 495}
\end{center}
\begin{multicols}{2}
\begin{center}

Molecule A \
2,3-dimethyl-butane\_out\_G09

\includegraphics[width=6cm]{../Comparisons/ImagesFromVMD/2,3-dimethyl-butane_out_G09.png}

Inertia Tensor - Molecule A \\
\begin{tabular}{|c c c|}
270.146	 & 	-16.9873	 & 	0	 \\
-16.9873	 & 	122.541	 & 	0	 \\
0	 & 	0	 & 	177.08
\end{tabular}

\vtab
 EingenVectors - Molecule A     \\
\begin{tabular}{|c c c|}
-0.112876	 & 	-0.993609	 & 	0	 \\
0	 & 	0	 & 	1	 \\
0.993609	 & 	-0.112876	 & 	-0
\end{tabular}

\vtab
 EingenValues - Molecule A     \\
\begin{tabular}{|c c c|}
120.611	 & 	177.08	 & 	272.075	 \\
\end{tabular}
\columnbreak

Molecule B \
2,3-dimethyl-butane\_out\_G09\_invertion

\includegraphics[width=6cm]{../Comparisons/ImagesFromVMD/2,3-dimethyl-butane_out_G09_invertion.png}

Inertia Tensor - Molecule B \\
\begin{tabular}{|c c c|}
270.146	 & 	-16.9873	 & 	0	 \\
-16.9873	 & 	122.541	 & 	0	 \\
0	 & 	0	 & 	177.08
\end{tabular}

\vtab
 EingenVectors - Molecule B     \\
\begin{tabular}{|c c c|}
-0.112875	 & 	-0.993609	 & 	0	 \\
0	 & 	0	 & 	1	 \\
0.993609	 & 	-0.112875	 & 	-0
\end{tabular}

\vtab
 EingenValues - Molecule B     \\
\begin{tabular}{|c c c|}
120.611	 & 	177.08	 & 	272.076	 \\
\end{tabular}

\end{center}
\end{multicols}

\vtab[-5mm]
\begin{tabular}{*{2}{m{0.38\textwidth}}}
\begin{center}
\textcolor{NavyBlue}{\Large Equal}
\end{center}
&
\begin{center}
\includegraphics[height=6.5cm]{../Comparisons/Vectors/inertia_tensor_of_2,3-dimethyl-butane_out_G09_and_2,3-dimethyl-butane_out_G09_invertion.png}
\end{center}
\end{tabular}

 \newpage

\vtab[-3cm]
\begin{center}
{\large IsomerC6H14 \tab Número 496}
\end{center}
\begin{multicols}{2}
\begin{center}

Molecule A \
2,3-dimethyl-butane\_out\_G09

\includegraphics[width=6cm]{../Comparisons/ImagesFromVMD/2,3-dimethyl-butane_out_G09.png}

Inertia Tensor - Molecule A \\
\begin{tabular}{|c c c|}
270.146	 & 	-16.9873	 & 	0	 \\
-16.9873	 & 	122.541	 & 	0	 \\
0	 & 	0	 & 	177.08
\end{tabular}

\vtab
 EingenVectors - Molecule A     \\
\begin{tabular}{|c c c|}
-0.112876	 & 	-0.993609	 & 	0	 \\
0	 & 	0	 & 	1	 \\
0.993609	 & 	-0.112876	 & 	-0
\end{tabular}

\vtab
 EingenValues - Molecule A     \\
\begin{tabular}{|c c c|}
120.611	 & 	177.08	 & 	272.075	 \\
\end{tabular}
\columnbreak

Molecule B \
2-methyl-pentane\_out\_G09

\includegraphics[width=6cm]{../Comparisons/ImagesFromVMD/2-methyl-pentane_out_G09.png}

Inertia Tensor - Molecule B \\
\begin{tabular}{|c c c|}
76.7701	 & 	-0.234564	 & 	-0.159008	 \\
-0.234564	 & 	299.061	 & 	-0.102167	 \\
-0.159008	 & 	-0.102167	 & 	347.557
\end{tabular}

\vtab
 EingenVectors - Molecule B     \\
\begin{tabular}{|c c c|}
-0.999999	 & 	-0.00105548	 & 	-0.000587604	 \\
-0.00105671	 & 	0.999997	 & 	0.00210321	 \\
-0.000585382	 & 	-0.00210383	 & 	0.999998
\end{tabular}

\vtab
 EingenValues - Molecule B     \\
\begin{tabular}{|c c c|}
76.7698	 & 	299.061	 & 	347.558	 \\
\end{tabular}

\end{center}
\end{multicols}

\vtab[-5mm]
\begin{tabular}{*{2}{m{0.38\textwidth}}}
\begin{center}
\textcolor{NavyBlue}{\Large Different}
\end{center}
&
\begin{center}
\includegraphics[height=6.5cm]{../Comparisons/Vectors/inertia_tensor_of_2,3-dimethyl-butane_out_G09_and_2-methyl-pentane_out_G09.png}
\end{center}
\end{tabular}

 \newpage

\vtab[-3cm]
\begin{center}
{\large IsomerC6H14 \tab Número 497}
\end{center}
\begin{multicols}{2}
\begin{center}

Molecule A \
2,3-dimethyl-butane\_out\_G09

\includegraphics[width=6cm]{../Comparisons/ImagesFromVMD/2,3-dimethyl-butane_out_G09.png}

Inertia Tensor - Molecule A \\
\begin{tabular}{|c c c|}
270.146	 & 	-16.9873	 & 	0	 \\
-16.9873	 & 	122.541	 & 	0	 \\
0	 & 	0	 & 	177.08
\end{tabular}

\vtab
 EingenVectors - Molecule A     \\
\begin{tabular}{|c c c|}
-0.112876	 & 	-0.993609	 & 	0	 \\
0	 & 	0	 & 	1	 \\
0.993609	 & 	-0.112876	 & 	-0
\end{tabular}

\vtab
 EingenValues - Molecule A     \\
\begin{tabular}{|c c c|}
120.611	 & 	177.08	 & 	272.075	 \\
\end{tabular}
\columnbreak

Molecule B \
2-methyl-pentane\_out\_G09\_invertion

\includegraphics[width=6cm]{../Comparisons/ImagesFromVMD/2-methyl-pentane_out_G09_invertion.png}

Inertia Tensor - Molecule B \\
\begin{tabular}{|c c c|}
76.7702	 & 	-0.234595	 & 	-0.158982	 \\
-0.234595	 & 	299.061	 & 	-0.102162	 \\
-0.158982	 & 	-0.102162	 & 	347.558
\end{tabular}

\vtab
 EingenVectors - Molecule B     \\
\begin{tabular}{|c c c|}
-0.999999	 & 	-0.00105562	 & 	-0.000587505	 \\
-0.00105685	 & 	0.999997	 & 	0.00210312	 \\
-0.000585284	 & 	-0.00210374	 & 	0.999998
\end{tabular}

\vtab
 EingenValues - Molecule B     \\
\begin{tabular}{|c c c|}
76.7699	 & 	299.061	 & 	347.558	 \\
\end{tabular}

\end{center}
\end{multicols}

\vtab[-5mm]
\begin{tabular}{*{2}{m{0.38\textwidth}}}
\begin{center}
\textcolor{NavyBlue}{\Large Different}
\end{center}
&
\begin{center}
\includegraphics[height=6.5cm]{../Comparisons/Vectors/inertia_tensor_of_2,3-dimethyl-butane_out_G09_and_2-methyl-pentane_out_G09_invertion.png}
\end{center}
\end{tabular}

 \newpage

\vtab[-3cm]
\begin{center}
{\large IsomerC6H14 \tab Número 498}
\end{center}
\begin{multicols}{2}
\begin{center}

Molecule A \
2,3-dimethyl-butane\_out\_G09

\includegraphics[width=6cm]{../Comparisons/ImagesFromVMD/2,3-dimethyl-butane_out_G09.png}

Inertia Tensor - Molecule A \\
\begin{tabular}{|c c c|}
270.146	 & 	-16.9873	 & 	0	 \\
-16.9873	 & 	122.541	 & 	0	 \\
0	 & 	0	 & 	177.08
\end{tabular}

\vtab
 EingenVectors - Molecule A     \\
\begin{tabular}{|c c c|}
-0.112876	 & 	-0.993609	 & 	0	 \\
0	 & 	0	 & 	1	 \\
0.993609	 & 	-0.112876	 & 	-0
\end{tabular}

\vtab
 EingenValues - Molecule A     \\
\begin{tabular}{|c c c|}
120.611	 & 	177.08	 & 	272.075	 \\
\end{tabular}
\columnbreak

Molecule B \
3-methyl-pentane\_out\_G09

\includegraphics[width=6cm]{../Comparisons/ImagesFromVMD/3-methyl-pentane_out_G09.png}

Inertia Tensor - Molecule B \\
\begin{tabular}{|c c c|}
294.318	 & 	21.9352	 & 	0	 \\
21.9352	 & 	302.288	 & 	0	 \\
0	 & 	0	 & 	77.8079
\end{tabular}

\vtab
 EingenVectors - Molecule B     \\
\begin{tabular}{|c c c|}
0	 & 	0	 & 	1	 \\
-0.767709	 & 	0.640799	 & 	0	 \\
0.640799	 & 	0.767709	 & 	0
\end{tabular}

\vtab
 EingenValues - Molecule B     \\
\begin{tabular}{|c c c|}
77.8079	 & 	276.009	 & 	320.597	 \\
\end{tabular}

\end{center}
\end{multicols}

\vtab[-5mm]
\begin{tabular}{*{2}{m{0.38\textwidth}}}
\begin{center}
\textcolor{NavyBlue}{\Large Different}
\end{center}
&
\begin{center}
\includegraphics[height=6.5cm]{../Comparisons/Vectors/inertia_tensor_of_2,3-dimethyl-butane_out_G09_and_3-methyl-pentane_out_G09.png}
\end{center}
\end{tabular}

 \newpage

\vtab[-3cm]
\begin{center}
{\large IsomerC6H14 \tab Número 499}
\end{center}
\begin{multicols}{2}
\begin{center}

Molecule A \
2,3-dimethyl-butane\_out\_G09

\includegraphics[width=6cm]{../Comparisons/ImagesFromVMD/2,3-dimethyl-butane_out_G09.png}

Inertia Tensor - Molecule A \\
\begin{tabular}{|c c c|}
270.146	 & 	-16.9873	 & 	0	 \\
-16.9873	 & 	122.541	 & 	0	 \\
0	 & 	0	 & 	177.08
\end{tabular}

\vtab
 EingenVectors - Molecule A     \\
\begin{tabular}{|c c c|}
-0.112876	 & 	-0.993609	 & 	0	 \\
0	 & 	0	 & 	1	 \\
0.993609	 & 	-0.112876	 & 	-0
\end{tabular}

\vtab
 EingenValues - Molecule A     \\
\begin{tabular}{|c c c|}
120.611	 & 	177.08	 & 	272.075	 \\
\end{tabular}
\columnbreak

Molecule B \
3-methyl-pentane\_out\_G09\_invertion

\includegraphics[width=6cm]{../Comparisons/ImagesFromVMD/3-methyl-pentane_out_G09_invertion.png}

Inertia Tensor - Molecule B \\
\begin{tabular}{|c c c|}
294.318	 & 	21.9353	 & 	0	 \\
21.9353	 & 	302.289	 & 	0	 \\
0	 & 	0	 & 	77.8081
\end{tabular}

\vtab
 EingenVectors - Molecule B     \\
\begin{tabular}{|c c c|}
0	 & 	0	 & 	1	 \\
-0.76771	 & 	0.640798	 & 	0	 \\
0.640798	 & 	0.76771	 & 	0
\end{tabular}

\vtab
 EingenValues - Molecule B     \\
\begin{tabular}{|c c c|}
77.8081	 & 	276.009	 & 	320.598	 \\
\end{tabular}

\end{center}
\end{multicols}

\vtab[-5mm]
\begin{tabular}{*{2}{m{0.38\textwidth}}}
\begin{center}
\textcolor{NavyBlue}{\Large Different}
\end{center}
&
\begin{center}
\includegraphics[height=6.5cm]{../Comparisons/Vectors/inertia_tensor_of_2,3-dimethyl-butane_out_G09_and_3-methyl-pentane_out_G09_invertion.png}
\end{center}
\end{tabular}

 \newpage

\vtab[-3cm]
\begin{center}
{\large IsomerC6H14 \tab Número 500}
\end{center}
\begin{multicols}{2}
\begin{center}

Molecule A \
2,3-dimethyl-butane\_out\_G09

\includegraphics[width=6cm]{../Comparisons/ImagesFromVMD/2,3-dimethyl-butane_out_G09.png}

Inertia Tensor - Molecule A \\
\begin{tabular}{|c c c|}
270.146	 & 	-16.9873	 & 	0	 \\
-16.9873	 & 	122.541	 & 	0	 \\
0	 & 	0	 & 	177.08
\end{tabular}

\vtab
 EingenVectors - Molecule A     \\
\begin{tabular}{|c c c|}
-0.112876	 & 	-0.993609	 & 	0	 \\
0	 & 	0	 & 	1	 \\
0.993609	 & 	-0.112876	 & 	-0
\end{tabular}

\vtab
 EingenValues - Molecule A     \\
\begin{tabular}{|c c c|}
120.611	 & 	177.08	 & 	272.075	 \\
\end{tabular}
\columnbreak

Molecule B \
hexane\_out\_G09

\includegraphics[width=6cm]{../Comparisons/ImagesFromVMD/hexane_out_G09.png}

Inertia Tensor - Molecule B \\
\begin{tabular}{|c c c|}
346.244	 & 	-179.129	 & 	0	 \\
-179.129	 & 	137.676	 & 	0	 \\
0	 & 	0	 & 	465.177
\end{tabular}

\vtab
 EingenVectors - Molecule B     \\
\begin{tabular}{|c c c|}
-0.498436	 & 	-0.866927	 & 	0	 \\
0.866927	 & 	-0.498436	 & 	-0	 \\
0	 & 	0	 & 	1
\end{tabular}

\vtab
 EingenValues - Molecule B     \\
\begin{tabular}{|c c c|}
34.6869	 & 	449.234	 & 	465.177	 \\
\end{tabular}

\end{center}
\end{multicols}

\vtab[-5mm]
\begin{tabular}{*{2}{m{0.38\textwidth}}}
\begin{center}
\textcolor{NavyBlue}{\Large Different}
\end{center}
&
\begin{center}
\includegraphics[height=6.5cm]{../Comparisons/Vectors/inertia_tensor_of_2,3-dimethyl-butane_out_G09_and_hexane_out_G09.png}
\end{center}
\end{tabular}

 \newpage

\vtab[-3cm]
\begin{center}
{\large IsomerC6H14 \tab Número 501}
\end{center}
\begin{multicols}{2}
\begin{center}

Molecule A \
2,3-dimethyl-butane\_out\_G09

\includegraphics[width=6cm]{../Comparisons/ImagesFromVMD/2,3-dimethyl-butane_out_G09.png}

Inertia Tensor - Molecule A \\
\begin{tabular}{|c c c|}
270.146	 & 	-16.9873	 & 	0	 \\
-16.9873	 & 	122.541	 & 	0	 \\
0	 & 	0	 & 	177.08
\end{tabular}

\vtab
 EingenVectors - Molecule A     \\
\begin{tabular}{|c c c|}
-0.112876	 & 	-0.993609	 & 	0	 \\
0	 & 	0	 & 	1	 \\
0.993609	 & 	-0.112876	 & 	-0
\end{tabular}

\vtab
 EingenValues - Molecule A     \\
\begin{tabular}{|c c c|}
120.611	 & 	177.08	 & 	272.075	 \\
\end{tabular}
\columnbreak

Molecule B \
hexane\_out\_G09\_invertion

\includegraphics[width=6cm]{../Comparisons/ImagesFromVMD/hexane_out_G09_invertion.png}

Inertia Tensor - Molecule B \\
\begin{tabular}{|c c c|}
346.245	 & 	-179.129	 & 	0	 \\
-179.129	 & 	137.676	 & 	0	 \\
0	 & 	0	 & 	465.177
\end{tabular}

\vtab
 EingenVectors - Molecule B     \\
\begin{tabular}{|c c c|}
-0.498435	 & 	-0.866927	 & 	0	 \\
0.866927	 & 	-0.498435	 & 	-0	 \\
0	 & 	0	 & 	1
\end{tabular}

\vtab
 EingenValues - Molecule B     \\
\begin{tabular}{|c c c|}
34.6868	 & 	449.234	 & 	465.177	 \\
\end{tabular}

\end{center}
\end{multicols}

\vtab[-5mm]
\begin{tabular}{*{2}{m{0.38\textwidth}}}
\begin{center}
\textcolor{NavyBlue}{\Large Different}
\end{center}
&
\begin{center}
\includegraphics[height=6.5cm]{../Comparisons/Vectors/inertia_tensor_of_2,3-dimethyl-butane_out_G09_and_hexane_out_G09_invertion.png}
\end{center}
\end{tabular}

 \newpage

\vtab[-3cm]
\begin{center}
{\large IsomerC6H14 \tab Número 502}
\end{center}
\begin{multicols}{2}
\begin{center}

Molecule A \
2,3-dimethyl-butane\_out\_G09\_invertion

\includegraphics[width=6cm]{../Comparisons/ImagesFromVMD/2,3-dimethyl-butane_out_G09_invertion.png}

Inertia Tensor - Molecule A \\
\begin{tabular}{|c c c|}
270.146	 & 	-16.9873	 & 	0	 \\
-16.9873	 & 	122.541	 & 	0	 \\
0	 & 	0	 & 	177.08
\end{tabular}

\vtab
 EingenVectors - Molecule A     \\
\begin{tabular}{|c c c|}
-0.112875	 & 	-0.993609	 & 	0	 \\
0	 & 	0	 & 	1	 \\
0.993609	 & 	-0.112875	 & 	-0
\end{tabular}

\vtab
 EingenValues - Molecule A     \\
\begin{tabular}{|c c c|}
120.611	 & 	177.08	 & 	272.076	 \\
\end{tabular}
\columnbreak

Molecule B \
2-methyl-pentane\_out\_G09

\includegraphics[width=6cm]{../Comparisons/ImagesFromVMD/2-methyl-pentane_out_G09.png}

Inertia Tensor - Molecule B \\
\begin{tabular}{|c c c|}
76.7701	 & 	-0.234564	 & 	-0.159008	 \\
-0.234564	 & 	299.061	 & 	-0.102167	 \\
-0.159008	 & 	-0.102167	 & 	347.557
\end{tabular}

\vtab
 EingenVectors - Molecule B     \\
\begin{tabular}{|c c c|}
-0.999999	 & 	-0.00105548	 & 	-0.000587604	 \\
-0.00105671	 & 	0.999997	 & 	0.00210321	 \\
-0.000585382	 & 	-0.00210383	 & 	0.999998
\end{tabular}

\vtab
 EingenValues - Molecule B     \\
\begin{tabular}{|c c c|}
76.7698	 & 	299.061	 & 	347.558	 \\
\end{tabular}

\end{center}
\end{multicols}

\vtab[-5mm]
\begin{tabular}{*{2}{m{0.38\textwidth}}}
\begin{center}
\textcolor{NavyBlue}{\Large Different}
\end{center}
&
\begin{center}
\includegraphics[height=6.5cm]{../Comparisons/Vectors/inertia_tensor_of_2,3-dimethyl-butane_out_G09_invertion_and_2-methyl-pentane_out_G09.png}
\end{center}
\end{tabular}

 \newpage

\vtab[-3cm]
\begin{center}
{\large IsomerC6H14 \tab Número 503}
\end{center}
\begin{multicols}{2}
\begin{center}

Molecule A \
2,3-dimethyl-butane\_out\_G09\_invertion

\includegraphics[width=6cm]{../Comparisons/ImagesFromVMD/2,3-dimethyl-butane_out_G09_invertion.png}

Inertia Tensor - Molecule A \\
\begin{tabular}{|c c c|}
270.146	 & 	-16.9873	 & 	0	 \\
-16.9873	 & 	122.541	 & 	0	 \\
0	 & 	0	 & 	177.08
\end{tabular}

\vtab
 EingenVectors - Molecule A     \\
\begin{tabular}{|c c c|}
-0.112875	 & 	-0.993609	 & 	0	 \\
0	 & 	0	 & 	1	 \\
0.993609	 & 	-0.112875	 & 	-0
\end{tabular}

\vtab
 EingenValues - Molecule A     \\
\begin{tabular}{|c c c|}
120.611	 & 	177.08	 & 	272.076	 \\
\end{tabular}
\columnbreak

Molecule B \
2-methyl-pentane\_out\_G09\_invertion

\includegraphics[width=6cm]{../Comparisons/ImagesFromVMD/2-methyl-pentane_out_G09_invertion.png}

Inertia Tensor - Molecule B \\
\begin{tabular}{|c c c|}
76.7702	 & 	-0.234595	 & 	-0.158982	 \\
-0.234595	 & 	299.061	 & 	-0.102162	 \\
-0.158982	 & 	-0.102162	 & 	347.558
\end{tabular}

\vtab
 EingenVectors - Molecule B     \\
\begin{tabular}{|c c c|}
-0.999999	 & 	-0.00105562	 & 	-0.000587505	 \\
-0.00105685	 & 	0.999997	 & 	0.00210312	 \\
-0.000585284	 & 	-0.00210374	 & 	0.999998
\end{tabular}

\vtab
 EingenValues - Molecule B     \\
\begin{tabular}{|c c c|}
76.7699	 & 	299.061	 & 	347.558	 \\
\end{tabular}

\end{center}
\end{multicols}

\vtab[-5mm]
\begin{tabular}{*{2}{m{0.38\textwidth}}}
\begin{center}
\textcolor{NavyBlue}{\Large Different}
\end{center}
&
\begin{center}
\includegraphics[height=6.5cm]{../Comparisons/Vectors/inertia_tensor_of_2,3-dimethyl-butane_out_G09_invertion_and_2-methyl-pentane_out_G09_invertion.png}
\end{center}
\end{tabular}

 \newpage

\vtab[-3cm]
\begin{center}
{\large IsomerC6H14 \tab Número 504}
\end{center}
\begin{multicols}{2}
\begin{center}

Molecule A \
2,3-dimethyl-butane\_out\_G09\_invertion

\includegraphics[width=6cm]{../Comparisons/ImagesFromVMD/2,3-dimethyl-butane_out_G09_invertion.png}

Inertia Tensor - Molecule A \\
\begin{tabular}{|c c c|}
270.146	 & 	-16.9873	 & 	0	 \\
-16.9873	 & 	122.541	 & 	0	 \\
0	 & 	0	 & 	177.08
\end{tabular}

\vtab
 EingenVectors - Molecule A     \\
\begin{tabular}{|c c c|}
-0.112875	 & 	-0.993609	 & 	0	 \\
0	 & 	0	 & 	1	 \\
0.993609	 & 	-0.112875	 & 	-0
\end{tabular}

\vtab
 EingenValues - Molecule A     \\
\begin{tabular}{|c c c|}
120.611	 & 	177.08	 & 	272.076	 \\
\end{tabular}
\columnbreak

Molecule B \
3-methyl-pentane\_out\_G09

\includegraphics[width=6cm]{../Comparisons/ImagesFromVMD/3-methyl-pentane_out_G09.png}

Inertia Tensor - Molecule B \\
\begin{tabular}{|c c c|}
294.318	 & 	21.9352	 & 	0	 \\
21.9352	 & 	302.288	 & 	0	 \\
0	 & 	0	 & 	77.8079
\end{tabular}

\vtab
 EingenVectors - Molecule B     \\
\begin{tabular}{|c c c|}
0	 & 	0	 & 	1	 \\
-0.767709	 & 	0.640799	 & 	0	 \\
0.640799	 & 	0.767709	 & 	0
\end{tabular}

\vtab
 EingenValues - Molecule B     \\
\begin{tabular}{|c c c|}
77.8079	 & 	276.009	 & 	320.597	 \\
\end{tabular}

\end{center}
\end{multicols}

\vtab[-5mm]
\begin{tabular}{*{2}{m{0.38\textwidth}}}
\begin{center}
\textcolor{NavyBlue}{\Large Different}
\end{center}
&
\begin{center}
\includegraphics[height=6.5cm]{../Comparisons/Vectors/inertia_tensor_of_2,3-dimethyl-butane_out_G09_invertion_and_3-methyl-pentane_out_G09.png}
\end{center}
\end{tabular}

 \newpage

\vtab[-3cm]
\begin{center}
{\large IsomerC6H14 \tab Número 505}
\end{center}
\begin{multicols}{2}
\begin{center}

Molecule A \
2,3-dimethyl-butane\_out\_G09\_invertion

\includegraphics[width=6cm]{../Comparisons/ImagesFromVMD/2,3-dimethyl-butane_out_G09_invertion.png}

Inertia Tensor - Molecule A \\
\begin{tabular}{|c c c|}
270.146	 & 	-16.9873	 & 	0	 \\
-16.9873	 & 	122.541	 & 	0	 \\
0	 & 	0	 & 	177.08
\end{tabular}

\vtab
 EingenVectors - Molecule A     \\
\begin{tabular}{|c c c|}
-0.112875	 & 	-0.993609	 & 	0	 \\
0	 & 	0	 & 	1	 \\
0.993609	 & 	-0.112875	 & 	-0
\end{tabular}

\vtab
 EingenValues - Molecule A     \\
\begin{tabular}{|c c c|}
120.611	 & 	177.08	 & 	272.076	 \\
\end{tabular}
\columnbreak

Molecule B \
3-methyl-pentane\_out\_G09\_invertion

\includegraphics[width=6cm]{../Comparisons/ImagesFromVMD/3-methyl-pentane_out_G09_invertion.png}

Inertia Tensor - Molecule B \\
\begin{tabular}{|c c c|}
294.318	 & 	21.9353	 & 	0	 \\
21.9353	 & 	302.289	 & 	0	 \\
0	 & 	0	 & 	77.8081
\end{tabular}

\vtab
 EingenVectors - Molecule B     \\
\begin{tabular}{|c c c|}
0	 & 	0	 & 	1	 \\
-0.76771	 & 	0.640798	 & 	0	 \\
0.640798	 & 	0.76771	 & 	0
\end{tabular}

\vtab
 EingenValues - Molecule B     \\
\begin{tabular}{|c c c|}
77.8081	 & 	276.009	 & 	320.598	 \\
\end{tabular}

\end{center}
\end{multicols}

\vtab[-5mm]
\begin{tabular}{*{2}{m{0.38\textwidth}}}
\begin{center}
\textcolor{NavyBlue}{\Large Different}
\end{center}
&
\begin{center}
\includegraphics[height=6.5cm]{../Comparisons/Vectors/inertia_tensor_of_2,3-dimethyl-butane_out_G09_invertion_and_3-methyl-pentane_out_G09_invertion.png}
\end{center}
\end{tabular}

 \newpage

\vtab[-3cm]
\begin{center}
{\large IsomerC6H14 \tab Número 506}
\end{center}
\begin{multicols}{2}
\begin{center}

Molecule A \
2,3-dimethyl-butane\_out\_G09\_invertion

\includegraphics[width=6cm]{../Comparisons/ImagesFromVMD/2,3-dimethyl-butane_out_G09_invertion.png}

Inertia Tensor - Molecule A \\
\begin{tabular}{|c c c|}
270.146	 & 	-16.9873	 & 	0	 \\
-16.9873	 & 	122.541	 & 	0	 \\
0	 & 	0	 & 	177.08
\end{tabular}

\vtab
 EingenVectors - Molecule A     \\
\begin{tabular}{|c c c|}
-0.112875	 & 	-0.993609	 & 	0	 \\
0	 & 	0	 & 	1	 \\
0.993609	 & 	-0.112875	 & 	-0
\end{tabular}

\vtab
 EingenValues - Molecule A     \\
\begin{tabular}{|c c c|}
120.611	 & 	177.08	 & 	272.076	 \\
\end{tabular}
\columnbreak

Molecule B \
hexane\_out\_G09

\includegraphics[width=6cm]{../Comparisons/ImagesFromVMD/hexane_out_G09.png}

Inertia Tensor - Molecule B \\
\begin{tabular}{|c c c|}
346.244	 & 	-179.129	 & 	0	 \\
-179.129	 & 	137.676	 & 	0	 \\
0	 & 	0	 & 	465.177
\end{tabular}

\vtab
 EingenVectors - Molecule B     \\
\begin{tabular}{|c c c|}
-0.498436	 & 	-0.866927	 & 	0	 \\
0.866927	 & 	-0.498436	 & 	-0	 \\
0	 & 	0	 & 	1
\end{tabular}

\vtab
 EingenValues - Molecule B     \\
\begin{tabular}{|c c c|}
34.6869	 & 	449.234	 & 	465.177	 \\
\end{tabular}

\end{center}
\end{multicols}

\vtab[-5mm]
\begin{tabular}{*{2}{m{0.38\textwidth}}}
\begin{center}
\textcolor{NavyBlue}{\Large Different}
\end{center}
&
\begin{center}
\includegraphics[height=6.5cm]{../Comparisons/Vectors/inertia_tensor_of_2,3-dimethyl-butane_out_G09_invertion_and_hexane_out_G09.png}
\end{center}
\end{tabular}

 \newpage

\vtab[-3cm]
\begin{center}
{\large IsomerC6H14 \tab Número 507}
\end{center}
\begin{multicols}{2}
\begin{center}

Molecule A \
2,3-dimethyl-butane\_out\_G09\_invertion

\includegraphics[width=6cm]{../Comparisons/ImagesFromVMD/2,3-dimethyl-butane_out_G09_invertion.png}

Inertia Tensor - Molecule A \\
\begin{tabular}{|c c c|}
270.146	 & 	-16.9873	 & 	0	 \\
-16.9873	 & 	122.541	 & 	0	 \\
0	 & 	0	 & 	177.08
\end{tabular}

\vtab
 EingenVectors - Molecule A     \\
\begin{tabular}{|c c c|}
-0.112875	 & 	-0.993609	 & 	0	 \\
0	 & 	0	 & 	1	 \\
0.993609	 & 	-0.112875	 & 	-0
\end{tabular}

\vtab
 EingenValues - Molecule A     \\
\begin{tabular}{|c c c|}
120.611	 & 	177.08	 & 	272.076	 \\
\end{tabular}
\columnbreak

Molecule B \
hexane\_out\_G09\_invertion

\includegraphics[width=6cm]{../Comparisons/ImagesFromVMD/hexane_out_G09_invertion.png}

Inertia Tensor - Molecule B \\
\begin{tabular}{|c c c|}
346.245	 & 	-179.129	 & 	0	 \\
-179.129	 & 	137.676	 & 	0	 \\
0	 & 	0	 & 	465.177
\end{tabular}

\vtab
 EingenVectors - Molecule B     \\
\begin{tabular}{|c c c|}
-0.498435	 & 	-0.866927	 & 	0	 \\
0.866927	 & 	-0.498435	 & 	-0	 \\
0	 & 	0	 & 	1
\end{tabular}

\vtab
 EingenValues - Molecule B     \\
\begin{tabular}{|c c c|}
34.6868	 & 	449.234	 & 	465.177	 \\
\end{tabular}

\end{center}
\end{multicols}

\vtab[-5mm]
\begin{tabular}{*{2}{m{0.38\textwidth}}}
\begin{center}
\textcolor{NavyBlue}{\Large Different}
\end{center}
&
\begin{center}
\includegraphics[height=6.5cm]{../Comparisons/Vectors/inertia_tensor_of_2,3-dimethyl-butane_out_G09_invertion_and_hexane_out_G09_invertion.png}
\end{center}
\end{tabular}

 \newpage

\vtab[-3cm]
\begin{center}
{\large IsomerC6H14 \tab Número 508}
\end{center}
\begin{multicols}{2}
\begin{center}

Molecule A \
2-methyl-pentane\_out\_G09

\includegraphics[width=6cm]{../Comparisons/ImagesFromVMD/2-methyl-pentane_out_G09.png}

Inertia Tensor - Molecule A \\
\begin{tabular}{|c c c|}
76.7701	 & 	-0.234564	 & 	-0.159008	 \\
-0.234564	 & 	299.061	 & 	-0.102167	 \\
-0.159008	 & 	-0.102167	 & 	347.557
\end{tabular}

\vtab
 EingenVectors - Molecule A     \\
\begin{tabular}{|c c c|}
-0.999999	 & 	-0.00105548	 & 	-0.000587604	 \\
-0.00105671	 & 	0.999997	 & 	0.00210321	 \\
-0.000585382	 & 	-0.00210383	 & 	0.999998
\end{tabular}

\vtab
 EingenValues - Molecule A     \\
\begin{tabular}{|c c c|}
76.7698	 & 	299.061	 & 	347.558	 \\
\end{tabular}
\columnbreak

Molecule B \
2-methyl-pentane\_out\_G09\_invertion

\includegraphics[width=6cm]{../Comparisons/ImagesFromVMD/2-methyl-pentane_out_G09_invertion.png}

Inertia Tensor - Molecule B \\
\begin{tabular}{|c c c|}
76.7702	 & 	-0.234595	 & 	-0.158982	 \\
-0.234595	 & 	299.061	 & 	-0.102162	 \\
-0.158982	 & 	-0.102162	 & 	347.558
\end{tabular}

\vtab
 EingenVectors - Molecule B     \\
\begin{tabular}{|c c c|}
-0.999999	 & 	-0.00105562	 & 	-0.000587505	 \\
-0.00105685	 & 	0.999997	 & 	0.00210312	 \\
-0.000585284	 & 	-0.00210374	 & 	0.999998
\end{tabular}

\vtab
 EingenValues - Molecule B     \\
\begin{tabular}{|c c c|}
76.7699	 & 	299.061	 & 	347.558	 \\
\end{tabular}

\end{center}
\end{multicols}

\vtab[-5mm]
\begin{tabular}{*{2}{m{0.38\textwidth}}}
\begin{center}
\textcolor{NavyBlue}{\Large Enantiomers}
\end{center}
&
\begin{center}
\includegraphics[height=6.5cm]{../Comparisons/Vectors/inertia_tensor_of_2-methyl-pentane_out_G09_and_2-methyl-pentane_out_G09_invertion.png}
\end{center}
\end{tabular}

 \newpage

\vtab[-3cm]
\begin{center}
{\large IsomerC6H14 \tab Número 509}
\end{center}
\begin{multicols}{2}
\begin{center}

Molecule A \
2-methyl-pentane\_out\_G09

\includegraphics[width=6cm]{../Comparisons/ImagesFromVMD/2-methyl-pentane_out_G09.png}

Inertia Tensor - Molecule A \\
\begin{tabular}{|c c c|}
76.7701	 & 	-0.234564	 & 	-0.159008	 \\
-0.234564	 & 	299.061	 & 	-0.102167	 \\
-0.159008	 & 	-0.102167	 & 	347.557
\end{tabular}

\vtab
 EingenVectors - Molecule A     \\
\begin{tabular}{|c c c|}
-0.999999	 & 	-0.00105548	 & 	-0.000587604	 \\
-0.00105671	 & 	0.999997	 & 	0.00210321	 \\
-0.000585382	 & 	-0.00210383	 & 	0.999998
\end{tabular}

\vtab
 EingenValues - Molecule A     \\
\begin{tabular}{|c c c|}
76.7698	 & 	299.061	 & 	347.558	 \\
\end{tabular}
\columnbreak

Molecule B \
3-methyl-pentane\_out\_G09

\includegraphics[width=6cm]{../Comparisons/ImagesFromVMD/3-methyl-pentane_out_G09.png}

Inertia Tensor - Molecule B \\
\begin{tabular}{|c c c|}
294.318	 & 	21.9352	 & 	0	 \\
21.9352	 & 	302.288	 & 	0	 \\
0	 & 	0	 & 	77.8079
\end{tabular}

\vtab
 EingenVectors - Molecule B     \\
\begin{tabular}{|c c c|}
0	 & 	0	 & 	1	 \\
-0.767709	 & 	0.640799	 & 	0	 \\
0.640799	 & 	0.767709	 & 	0
\end{tabular}

\vtab
 EingenValues - Molecule B     \\
\begin{tabular}{|c c c|}
77.8079	 & 	276.009	 & 	320.597	 \\
\end{tabular}

\end{center}
\end{multicols}

\vtab[-5mm]
\begin{tabular}{*{2}{m{0.38\textwidth}}}
\begin{center}
\textcolor{NavyBlue}{\Large Different}
\end{center}
&
\begin{center}
\includegraphics[height=6.5cm]{../Comparisons/Vectors/inertia_tensor_of_2-methyl-pentane_out_G09_and_3-methyl-pentane_out_G09.png}
\end{center}
\end{tabular}

 \newpage

\vtab[-3cm]
\begin{center}
{\large IsomerC6H14 \tab Número 510}
\end{center}
\begin{multicols}{2}
\begin{center}

Molecule A \
2-methyl-pentane\_out\_G09

\includegraphics[width=6cm]{../Comparisons/ImagesFromVMD/2-methyl-pentane_out_G09.png}

Inertia Tensor - Molecule A \\
\begin{tabular}{|c c c|}
76.7701	 & 	-0.234564	 & 	-0.159008	 \\
-0.234564	 & 	299.061	 & 	-0.102167	 \\
-0.159008	 & 	-0.102167	 & 	347.557
\end{tabular}

\vtab
 EingenVectors - Molecule A     \\
\begin{tabular}{|c c c|}
-0.999999	 & 	-0.00105548	 & 	-0.000587604	 \\
-0.00105671	 & 	0.999997	 & 	0.00210321	 \\
-0.000585382	 & 	-0.00210383	 & 	0.999998
\end{tabular}

\vtab
 EingenValues - Molecule A     \\
\begin{tabular}{|c c c|}
76.7698	 & 	299.061	 & 	347.558	 \\
\end{tabular}
\columnbreak

Molecule B \
3-methyl-pentane\_out\_G09\_invertion

\includegraphics[width=6cm]{../Comparisons/ImagesFromVMD/3-methyl-pentane_out_G09_invertion.png}

Inertia Tensor - Molecule B \\
\begin{tabular}{|c c c|}
294.318	 & 	21.9353	 & 	0	 \\
21.9353	 & 	302.289	 & 	0	 \\
0	 & 	0	 & 	77.8081
\end{tabular}

\vtab
 EingenVectors - Molecule B     \\
\begin{tabular}{|c c c|}
0	 & 	0	 & 	1	 \\
-0.76771	 & 	0.640798	 & 	0	 \\
0.640798	 & 	0.76771	 & 	0
\end{tabular}

\vtab
 EingenValues - Molecule B     \\
\begin{tabular}{|c c c|}
77.8081	 & 	276.009	 & 	320.598	 \\
\end{tabular}

\end{center}
\end{multicols}

\vtab[-5mm]
\begin{tabular}{*{2}{m{0.38\textwidth}}}
\begin{center}
\textcolor{NavyBlue}{\Large Different}
\end{center}
&
\begin{center}
\includegraphics[height=6.5cm]{../Comparisons/Vectors/inertia_tensor_of_2-methyl-pentane_out_G09_and_3-methyl-pentane_out_G09_invertion.png}
\end{center}
\end{tabular}

 \newpage

\vtab[-3cm]
\begin{center}
{\large IsomerC6H14 \tab Número 511}
\end{center}
\begin{multicols}{2}
\begin{center}

Molecule A \
2-methyl-pentane\_out\_G09

\includegraphics[width=6cm]{../Comparisons/ImagesFromVMD/2-methyl-pentane_out_G09.png}

Inertia Tensor - Molecule A \\
\begin{tabular}{|c c c|}
76.7701	 & 	-0.234564	 & 	-0.159008	 \\
-0.234564	 & 	299.061	 & 	-0.102167	 \\
-0.159008	 & 	-0.102167	 & 	347.557
\end{tabular}

\vtab
 EingenVectors - Molecule A     \\
\begin{tabular}{|c c c|}
-0.999999	 & 	-0.00105548	 & 	-0.000587604	 \\
-0.00105671	 & 	0.999997	 & 	0.00210321	 \\
-0.000585382	 & 	-0.00210383	 & 	0.999998
\end{tabular}

\vtab
 EingenValues - Molecule A     \\
\begin{tabular}{|c c c|}
76.7698	 & 	299.061	 & 	347.558	 \\
\end{tabular}
\columnbreak

Molecule B \
hexane\_out\_G09

\includegraphics[width=6cm]{../Comparisons/ImagesFromVMD/hexane_out_G09.png}

Inertia Tensor - Molecule B \\
\begin{tabular}{|c c c|}
346.244	 & 	-179.129	 & 	0	 \\
-179.129	 & 	137.676	 & 	0	 \\
0	 & 	0	 & 	465.177
\end{tabular}

\vtab
 EingenVectors - Molecule B     \\
\begin{tabular}{|c c c|}
-0.498436	 & 	-0.866927	 & 	0	 \\
0.866927	 & 	-0.498436	 & 	-0	 \\
0	 & 	0	 & 	1
\end{tabular}

\vtab
 EingenValues - Molecule B     \\
\begin{tabular}{|c c c|}
34.6869	 & 	449.234	 & 	465.177	 \\
\end{tabular}

\end{center}
\end{multicols}

\vtab[-5mm]
\begin{tabular}{*{2}{m{0.38\textwidth}}}
\begin{center}
\textcolor{NavyBlue}{\Large Different}
\end{center}
&
\begin{center}
\includegraphics[height=6.5cm]{../Comparisons/Vectors/inertia_tensor_of_2-methyl-pentane_out_G09_and_hexane_out_G09.png}
\end{center}
\end{tabular}

 \newpage

\vtab[-3cm]
\begin{center}
{\large IsomerC6H14 \tab Número 512}
\end{center}
\begin{multicols}{2}
\begin{center}

Molecule A \
2-methyl-pentane\_out\_G09

\includegraphics[width=6cm]{../Comparisons/ImagesFromVMD/2-methyl-pentane_out_G09.png}

Inertia Tensor - Molecule A \\
\begin{tabular}{|c c c|}
76.7701	 & 	-0.234564	 & 	-0.159008	 \\
-0.234564	 & 	299.061	 & 	-0.102167	 \\
-0.159008	 & 	-0.102167	 & 	347.557
\end{tabular}

\vtab
 EingenVectors - Molecule A     \\
\begin{tabular}{|c c c|}
-0.999999	 & 	-0.00105548	 & 	-0.000587604	 \\
-0.00105671	 & 	0.999997	 & 	0.00210321	 \\
-0.000585382	 & 	-0.00210383	 & 	0.999998
\end{tabular}

\vtab
 EingenValues - Molecule A     \\
\begin{tabular}{|c c c|}
76.7698	 & 	299.061	 & 	347.558	 \\
\end{tabular}
\columnbreak

Molecule B \
hexane\_out\_G09\_invertion

\includegraphics[width=6cm]{../Comparisons/ImagesFromVMD/hexane_out_G09_invertion.png}

Inertia Tensor - Molecule B \\
\begin{tabular}{|c c c|}
346.245	 & 	-179.129	 & 	0	 \\
-179.129	 & 	137.676	 & 	0	 \\
0	 & 	0	 & 	465.177
\end{tabular}

\vtab
 EingenVectors - Molecule B     \\
\begin{tabular}{|c c c|}
-0.498435	 & 	-0.866927	 & 	0	 \\
0.866927	 & 	-0.498435	 & 	-0	 \\
0	 & 	0	 & 	1
\end{tabular}

\vtab
 EingenValues - Molecule B     \\
\begin{tabular}{|c c c|}
34.6868	 & 	449.234	 & 	465.177	 \\
\end{tabular}

\end{center}
\end{multicols}

\vtab[-5mm]
\begin{tabular}{*{2}{m{0.38\textwidth}}}
\begin{center}
\textcolor{NavyBlue}{\Large Different}
\end{center}
&
\begin{center}
\includegraphics[height=6.5cm]{../Comparisons/Vectors/inertia_tensor_of_2-methyl-pentane_out_G09_and_hexane_out_G09_invertion.png}
\end{center}
\end{tabular}

 \newpage

\vtab[-3cm]
\begin{center}
{\large IsomerC6H14 \tab Número 513}
\end{center}
\begin{multicols}{2}
\begin{center}

Molecule A \
2-methyl-pentane\_out\_G09\_invertion

\includegraphics[width=6cm]{../Comparisons/ImagesFromVMD/2-methyl-pentane_out_G09_invertion.png}

Inertia Tensor - Molecule A \\
\begin{tabular}{|c c c|}
76.7702	 & 	-0.234595	 & 	-0.158982	 \\
-0.234595	 & 	299.061	 & 	-0.102162	 \\
-0.158982	 & 	-0.102162	 & 	347.558
\end{tabular}

\vtab
 EingenVectors - Molecule A     \\
\begin{tabular}{|c c c|}
-0.999999	 & 	-0.00105562	 & 	-0.000587505	 \\
-0.00105685	 & 	0.999997	 & 	0.00210312	 \\
-0.000585284	 & 	-0.00210374	 & 	0.999998
\end{tabular}

\vtab
 EingenValues - Molecule A     \\
\begin{tabular}{|c c c|}
76.7699	 & 	299.061	 & 	347.558	 \\
\end{tabular}
\columnbreak

Molecule B \
3-methyl-pentane\_out\_G09

\includegraphics[width=6cm]{../Comparisons/ImagesFromVMD/3-methyl-pentane_out_G09.png}

Inertia Tensor - Molecule B \\
\begin{tabular}{|c c c|}
294.318	 & 	21.9352	 & 	0	 \\
21.9352	 & 	302.288	 & 	0	 \\
0	 & 	0	 & 	77.8079
\end{tabular}

\vtab
 EingenVectors - Molecule B     \\
\begin{tabular}{|c c c|}
0	 & 	0	 & 	1	 \\
-0.767709	 & 	0.640799	 & 	0	 \\
0.640799	 & 	0.767709	 & 	0
\end{tabular}

\vtab
 EingenValues - Molecule B     \\
\begin{tabular}{|c c c|}
77.8079	 & 	276.009	 & 	320.597	 \\
\end{tabular}

\end{center}
\end{multicols}

\vtab[-5mm]
\begin{tabular}{*{2}{m{0.38\textwidth}}}
\begin{center}
\textcolor{NavyBlue}{\Large Different}
\end{center}
&
\begin{center}
\includegraphics[height=6.5cm]{../Comparisons/Vectors/inertia_tensor_of_2-methyl-pentane_out_G09_invertion_and_3-methyl-pentane_out_G09.png}
\end{center}
\end{tabular}

 \newpage

\vtab[-3cm]
\begin{center}
{\large IsomerC6H14 \tab Número 514}
\end{center}
\begin{multicols}{2}
\begin{center}

Molecule A \
2-methyl-pentane\_out\_G09\_invertion

\includegraphics[width=6cm]{../Comparisons/ImagesFromVMD/2-methyl-pentane_out_G09_invertion.png}

Inertia Tensor - Molecule A \\
\begin{tabular}{|c c c|}
76.7702	 & 	-0.234595	 & 	-0.158982	 \\
-0.234595	 & 	299.061	 & 	-0.102162	 \\
-0.158982	 & 	-0.102162	 & 	347.558
\end{tabular}

\vtab
 EingenVectors - Molecule A     \\
\begin{tabular}{|c c c|}
-0.999999	 & 	-0.00105562	 & 	-0.000587505	 \\
-0.00105685	 & 	0.999997	 & 	0.00210312	 \\
-0.000585284	 & 	-0.00210374	 & 	0.999998
\end{tabular}

\vtab
 EingenValues - Molecule A     \\
\begin{tabular}{|c c c|}
76.7699	 & 	299.061	 & 	347.558	 \\
\end{tabular}
\columnbreak

Molecule B \
3-methyl-pentane\_out\_G09\_invertion

\includegraphics[width=6cm]{../Comparisons/ImagesFromVMD/3-methyl-pentane_out_G09_invertion.png}

Inertia Tensor - Molecule B \\
\begin{tabular}{|c c c|}
294.318	 & 	21.9353	 & 	0	 \\
21.9353	 & 	302.289	 & 	0	 \\
0	 & 	0	 & 	77.8081
\end{tabular}

\vtab
 EingenVectors - Molecule B     \\
\begin{tabular}{|c c c|}
0	 & 	0	 & 	1	 \\
-0.76771	 & 	0.640798	 & 	0	 \\
0.640798	 & 	0.76771	 & 	0
\end{tabular}

\vtab
 EingenValues - Molecule B     \\
\begin{tabular}{|c c c|}
77.8081	 & 	276.009	 & 	320.598	 \\
\end{tabular}

\end{center}
\end{multicols}

\vtab[-5mm]
\begin{tabular}{*{2}{m{0.38\textwidth}}}
\begin{center}
\textcolor{NavyBlue}{\Large Different}
\end{center}
&
\begin{center}
\includegraphics[height=6.5cm]{../Comparisons/Vectors/inertia_tensor_of_2-methyl-pentane_out_G09_invertion_and_3-methyl-pentane_out_G09_invertion.png}
\end{center}
\end{tabular}

 \newpage

\vtab[-3cm]
\begin{center}
{\large IsomerC6H14 \tab Número 515}
\end{center}
\begin{multicols}{2}
\begin{center}

Molecule A \
2-methyl-pentane\_out\_G09\_invertion

\includegraphics[width=6cm]{../Comparisons/ImagesFromVMD/2-methyl-pentane_out_G09_invertion.png}

Inertia Tensor - Molecule A \\
\begin{tabular}{|c c c|}
76.7702	 & 	-0.234595	 & 	-0.158982	 \\
-0.234595	 & 	299.061	 & 	-0.102162	 \\
-0.158982	 & 	-0.102162	 & 	347.558
\end{tabular}

\vtab
 EingenVectors - Molecule A     \\
\begin{tabular}{|c c c|}
-0.999999	 & 	-0.00105562	 & 	-0.000587505	 \\
-0.00105685	 & 	0.999997	 & 	0.00210312	 \\
-0.000585284	 & 	-0.00210374	 & 	0.999998
\end{tabular}

\vtab
 EingenValues - Molecule A     \\
\begin{tabular}{|c c c|}
76.7699	 & 	299.061	 & 	347.558	 \\
\end{tabular}
\columnbreak

Molecule B \
hexane\_out\_G09

\includegraphics[width=6cm]{../Comparisons/ImagesFromVMD/hexane_out_G09.png}

Inertia Tensor - Molecule B \\
\begin{tabular}{|c c c|}
346.244	 & 	-179.129	 & 	0	 \\
-179.129	 & 	137.676	 & 	0	 \\
0	 & 	0	 & 	465.177
\end{tabular}

\vtab
 EingenVectors - Molecule B     \\
\begin{tabular}{|c c c|}
-0.498436	 & 	-0.866927	 & 	0	 \\
0.866927	 & 	-0.498436	 & 	-0	 \\
0	 & 	0	 & 	1
\end{tabular}

\vtab
 EingenValues - Molecule B     \\
\begin{tabular}{|c c c|}
34.6869	 & 	449.234	 & 	465.177	 \\
\end{tabular}

\end{center}
\end{multicols}

\vtab[-5mm]
\begin{tabular}{*{2}{m{0.38\textwidth}}}
\begin{center}
\textcolor{NavyBlue}{\Large Different}
\end{center}
&
\begin{center}
\includegraphics[height=6.5cm]{../Comparisons/Vectors/inertia_tensor_of_2-methyl-pentane_out_G09_invertion_and_hexane_out_G09.png}
\end{center}
\end{tabular}

 \newpage

\vtab[-3cm]
\begin{center}
{\large IsomerC6H14 \tab Número 516}
\end{center}
\begin{multicols}{2}
\begin{center}

Molecule A \
2-methyl-pentane\_out\_G09\_invertion

\includegraphics[width=6cm]{../Comparisons/ImagesFromVMD/2-methyl-pentane_out_G09_invertion.png}

Inertia Tensor - Molecule A \\
\begin{tabular}{|c c c|}
76.7702	 & 	-0.234595	 & 	-0.158982	 \\
-0.234595	 & 	299.061	 & 	-0.102162	 \\
-0.158982	 & 	-0.102162	 & 	347.558
\end{tabular}

\vtab
 EingenVectors - Molecule A     \\
\begin{tabular}{|c c c|}
-0.999999	 & 	-0.00105562	 & 	-0.000587505	 \\
-0.00105685	 & 	0.999997	 & 	0.00210312	 \\
-0.000585284	 & 	-0.00210374	 & 	0.999998
\end{tabular}

\vtab
 EingenValues - Molecule A     \\
\begin{tabular}{|c c c|}
76.7699	 & 	299.061	 & 	347.558	 \\
\end{tabular}
\columnbreak

Molecule B \
hexane\_out\_G09\_invertion

\includegraphics[width=6cm]{../Comparisons/ImagesFromVMD/hexane_out_G09_invertion.png}

Inertia Tensor - Molecule B \\
\begin{tabular}{|c c c|}
346.245	 & 	-179.129	 & 	0	 \\
-179.129	 & 	137.676	 & 	0	 \\
0	 & 	0	 & 	465.177
\end{tabular}

\vtab
 EingenVectors - Molecule B     \\
\begin{tabular}{|c c c|}
-0.498435	 & 	-0.866927	 & 	0	 \\
0.866927	 & 	-0.498435	 & 	-0	 \\
0	 & 	0	 & 	1
\end{tabular}

\vtab
 EingenValues - Molecule B     \\
\begin{tabular}{|c c c|}
34.6868	 & 	449.234	 & 	465.177	 \\
\end{tabular}

\end{center}
\end{multicols}

\vtab[-5mm]
\begin{tabular}{*{2}{m{0.38\textwidth}}}
\begin{center}
\textcolor{NavyBlue}{\Large Different}
\end{center}
&
\begin{center}
\includegraphics[height=6.5cm]{../Comparisons/Vectors/inertia_tensor_of_2-methyl-pentane_out_G09_invertion_and_hexane_out_G09_invertion.png}
\end{center}
\end{tabular}

 \newpage

\vtab[-3cm]
\begin{center}
{\large IsomerC6H14 \tab Número 517}
\end{center}
\begin{multicols}{2}
\begin{center}

Molecule A \
3-methyl-pentane\_out\_G09

\includegraphics[width=6cm]{../Comparisons/ImagesFromVMD/3-methyl-pentane_out_G09.png}

Inertia Tensor - Molecule A \\
\begin{tabular}{|c c c|}
294.318	 & 	21.9352	 & 	0	 \\
21.9352	 & 	302.288	 & 	0	 \\
0	 & 	0	 & 	77.8079
\end{tabular}

\vtab
 EingenVectors - Molecule A     \\
\begin{tabular}{|c c c|}
0	 & 	0	 & 	1	 \\
-0.767709	 & 	0.640799	 & 	0	 \\
0.640799	 & 	0.767709	 & 	0
\end{tabular}

\vtab
 EingenValues - Molecule A     \\
\begin{tabular}{|c c c|}
77.8079	 & 	276.009	 & 	320.597	 \\
\end{tabular}
\columnbreak

Molecule B \
3-methyl-pentane\_out\_G09\_invertion

\includegraphics[width=6cm]{../Comparisons/ImagesFromVMD/3-methyl-pentane_out_G09_invertion.png}

Inertia Tensor - Molecule B \\
\begin{tabular}{|c c c|}
294.318	 & 	21.9353	 & 	0	 \\
21.9353	 & 	302.289	 & 	0	 \\
0	 & 	0	 & 	77.8081
\end{tabular}

\vtab
 EingenVectors - Molecule B     \\
\begin{tabular}{|c c c|}
0	 & 	0	 & 	1	 \\
-0.76771	 & 	0.640798	 & 	0	 \\
0.640798	 & 	0.76771	 & 	0
\end{tabular}

\vtab
 EingenValues - Molecule B     \\
\begin{tabular}{|c c c|}
77.8081	 & 	276.009	 & 	320.598	 \\
\end{tabular}

\end{center}
\end{multicols}

\vtab[-5mm]
\begin{tabular}{*{2}{m{0.38\textwidth}}}
\begin{center}
\textcolor{NavyBlue}{\Large Equal}
\end{center}
&
\begin{center}
\includegraphics[height=6.5cm]{../Comparisons/Vectors/inertia_tensor_of_3-methyl-pentane_out_G09_and_3-methyl-pentane_out_G09_invertion.png}
\end{center}
\end{tabular}

 \newpage

\vtab[-3cm]
\begin{center}
{\large IsomerC6H14 \tab Número 518}
\end{center}
\begin{multicols}{2}
\begin{center}

Molecule A \
3-methyl-pentane\_out\_G09

\includegraphics[width=6cm]{../Comparisons/ImagesFromVMD/3-methyl-pentane_out_G09.png}

Inertia Tensor - Molecule A \\
\begin{tabular}{|c c c|}
294.318	 & 	21.9352	 & 	0	 \\
21.9352	 & 	302.288	 & 	0	 \\
0	 & 	0	 & 	77.8079
\end{tabular}

\vtab
 EingenVectors - Molecule A     \\
\begin{tabular}{|c c c|}
0	 & 	0	 & 	1	 \\
-0.767709	 & 	0.640799	 & 	0	 \\
0.640799	 & 	0.767709	 & 	0
\end{tabular}

\vtab
 EingenValues - Molecule A     \\
\begin{tabular}{|c c c|}
77.8079	 & 	276.009	 & 	320.597	 \\
\end{tabular}
\columnbreak

Molecule B \
hexane\_out\_G09

\includegraphics[width=6cm]{../Comparisons/ImagesFromVMD/hexane_out_G09.png}

Inertia Tensor - Molecule B \\
\begin{tabular}{|c c c|}
346.244	 & 	-179.129	 & 	0	 \\
-179.129	 & 	137.676	 & 	0	 \\
0	 & 	0	 & 	465.177
\end{tabular}

\vtab
 EingenVectors - Molecule B     \\
\begin{tabular}{|c c c|}
-0.498436	 & 	-0.866927	 & 	0	 \\
0.866927	 & 	-0.498436	 & 	-0	 \\
0	 & 	0	 & 	1
\end{tabular}

\vtab
 EingenValues - Molecule B     \\
\begin{tabular}{|c c c|}
34.6869	 & 	449.234	 & 	465.177	 \\
\end{tabular}

\end{center}
\end{multicols}

\vtab[-5mm]
\begin{tabular}{*{2}{m{0.38\textwidth}}}
\begin{center}
\textcolor{NavyBlue}{\Large Different}
\end{center}
&
\begin{center}
\includegraphics[height=6.5cm]{../Comparisons/Vectors/inertia_tensor_of_3-methyl-pentane_out_G09_and_hexane_out_G09.png}
\end{center}
\end{tabular}

 \newpage

\vtab[-3cm]
\begin{center}
{\large IsomerC6H14 \tab Número 519}
\end{center}
\begin{multicols}{2}
\begin{center}

Molecule A \
3-methyl-pentane\_out\_G09

\includegraphics[width=6cm]{../Comparisons/ImagesFromVMD/3-methyl-pentane_out_G09.png}

Inertia Tensor - Molecule A \\
\begin{tabular}{|c c c|}
294.318	 & 	21.9352	 & 	0	 \\
21.9352	 & 	302.288	 & 	0	 \\
0	 & 	0	 & 	77.8079
\end{tabular}

\vtab
 EingenVectors - Molecule A     \\
\begin{tabular}{|c c c|}
0	 & 	0	 & 	1	 \\
-0.767709	 & 	0.640799	 & 	0	 \\
0.640799	 & 	0.767709	 & 	0
\end{tabular}

\vtab
 EingenValues - Molecule A     \\
\begin{tabular}{|c c c|}
77.8079	 & 	276.009	 & 	320.597	 \\
\end{tabular}
\columnbreak

Molecule B \
hexane\_out\_G09\_invertion

\includegraphics[width=6cm]{../Comparisons/ImagesFromVMD/hexane_out_G09_invertion.png}

Inertia Tensor - Molecule B \\
\begin{tabular}{|c c c|}
346.245	 & 	-179.129	 & 	0	 \\
-179.129	 & 	137.676	 & 	0	 \\
0	 & 	0	 & 	465.177
\end{tabular}

\vtab
 EingenVectors - Molecule B     \\
\begin{tabular}{|c c c|}
-0.498435	 & 	-0.866927	 & 	0	 \\
0.866927	 & 	-0.498435	 & 	-0	 \\
0	 & 	0	 & 	1
\end{tabular}

\vtab
 EingenValues - Molecule B     \\
\begin{tabular}{|c c c|}
34.6868	 & 	449.234	 & 	465.177	 \\
\end{tabular}

\end{center}
\end{multicols}

\vtab[-5mm]
\begin{tabular}{*{2}{m{0.38\textwidth}}}
\begin{center}
\textcolor{NavyBlue}{\Large Different}
\end{center}
&
\begin{center}
\includegraphics[height=6.5cm]{../Comparisons/Vectors/inertia_tensor_of_3-methyl-pentane_out_G09_and_hexane_out_G09_invertion.png}
\end{center}
\end{tabular}

 \newpage

\vtab[-3cm]
\begin{center}
{\large IsomerC6H14 \tab Número 520}
\end{center}
\begin{multicols}{2}
\begin{center}

Molecule A \
3-methyl-pentane\_out\_G09\_invertion

\includegraphics[width=6cm]{../Comparisons/ImagesFromVMD/3-methyl-pentane_out_G09_invertion.png}

Inertia Tensor - Molecule A \\
\begin{tabular}{|c c c|}
294.318	 & 	21.9353	 & 	0	 \\
21.9353	 & 	302.289	 & 	0	 \\
0	 & 	0	 & 	77.8081
\end{tabular}

\vtab
 EingenVectors - Molecule A     \\
\begin{tabular}{|c c c|}
0	 & 	0	 & 	1	 \\
-0.76771	 & 	0.640798	 & 	0	 \\
0.640798	 & 	0.76771	 & 	0
\end{tabular}

\vtab
 EingenValues - Molecule A     \\
\begin{tabular}{|c c c|}
77.8081	 & 	276.009	 & 	320.598	 \\
\end{tabular}
\columnbreak

Molecule B \
hexane\_out\_G09

\includegraphics[width=6cm]{../Comparisons/ImagesFromVMD/hexane_out_G09.png}

Inertia Tensor - Molecule B \\
\begin{tabular}{|c c c|}
346.244	 & 	-179.129	 & 	0	 \\
-179.129	 & 	137.676	 & 	0	 \\
0	 & 	0	 & 	465.177
\end{tabular}

\vtab
 EingenVectors - Molecule B     \\
\begin{tabular}{|c c c|}
-0.498436	 & 	-0.866927	 & 	0	 \\
0.866927	 & 	-0.498436	 & 	-0	 \\
0	 & 	0	 & 	1
\end{tabular}

\vtab
 EingenValues - Molecule B     \\
\begin{tabular}{|c c c|}
34.6869	 & 	449.234	 & 	465.177	 \\
\end{tabular}

\end{center}
\end{multicols}

\vtab[-5mm]
\begin{tabular}{*{2}{m{0.38\textwidth}}}
\begin{center}
\textcolor{NavyBlue}{\Large Different}
\end{center}
&
\begin{center}
\includegraphics[height=6.5cm]{../Comparisons/Vectors/inertia_tensor_of_3-methyl-pentane_out_G09_invertion_and_hexane_out_G09.png}
\end{center}
\end{tabular}

 \newpage

\vtab[-3cm]
\begin{center}
{\large IsomerC6H14 \tab Número 521}
\end{center}
\begin{multicols}{2}
\begin{center}

Molecule A \
3-methyl-pentane\_out\_G09\_invertion

\includegraphics[width=6cm]{../Comparisons/ImagesFromVMD/3-methyl-pentane_out_G09_invertion.png}

Inertia Tensor - Molecule A \\
\begin{tabular}{|c c c|}
294.318	 & 	21.9353	 & 	0	 \\
21.9353	 & 	302.289	 & 	0	 \\
0	 & 	0	 & 	77.8081
\end{tabular}

\vtab
 EingenVectors - Molecule A     \\
\begin{tabular}{|c c c|}
0	 & 	0	 & 	1	 \\
-0.76771	 & 	0.640798	 & 	0	 \\
0.640798	 & 	0.76771	 & 	0
\end{tabular}

\vtab
 EingenValues - Molecule A     \\
\begin{tabular}{|c c c|}
77.8081	 & 	276.009	 & 	320.598	 \\
\end{tabular}
\columnbreak

Molecule B \
hexane\_out\_G09\_invertion

\includegraphics[width=6cm]{../Comparisons/ImagesFromVMD/hexane_out_G09_invertion.png}

Inertia Tensor - Molecule B \\
\begin{tabular}{|c c c|}
346.245	 & 	-179.129	 & 	0	 \\
-179.129	 & 	137.676	 & 	0	 \\
0	 & 	0	 & 	465.177
\end{tabular}

\vtab
 EingenVectors - Molecule B     \\
\begin{tabular}{|c c c|}
-0.498435	 & 	-0.866927	 & 	0	 \\
0.866927	 & 	-0.498435	 & 	-0	 \\
0	 & 	0	 & 	1
\end{tabular}

\vtab
 EingenValues - Molecule B     \\
\begin{tabular}{|c c c|}
34.6868	 & 	449.234	 & 	465.177	 \\
\end{tabular}

\end{center}
\end{multicols}

\vtab[-5mm]
\begin{tabular}{*{2}{m{0.38\textwidth}}}
\begin{center}
\textcolor{NavyBlue}{\Large Different}
\end{center}
&
\begin{center}
\includegraphics[height=6.5cm]{../Comparisons/Vectors/inertia_tensor_of_3-methyl-pentane_out_G09_invertion_and_hexane_out_G09_invertion.png}
\end{center}
\end{tabular}

 \newpage

\vtab[-3cm]
\begin{center}
{\large IsomerC6H14 \tab Número 522}
\end{center}
\begin{multicols}{2}
\begin{center}

Molecule A \
hexane\_out\_G09

\includegraphics[width=6cm]{../Comparisons/ImagesFromVMD/hexane_out_G09.png}

Inertia Tensor - Molecule A \\
\begin{tabular}{|c c c|}
346.244	 & 	-179.129	 & 	0	 \\
-179.129	 & 	137.676	 & 	0	 \\
0	 & 	0	 & 	465.177
\end{tabular}

\vtab
 EingenVectors - Molecule A     \\
\begin{tabular}{|c c c|}
-0.498436	 & 	-0.866927	 & 	0	 \\
0.866927	 & 	-0.498436	 & 	-0	 \\
0	 & 	0	 & 	1
\end{tabular}

\vtab
 EingenValues - Molecule A     \\
\begin{tabular}{|c c c|}
34.6869	 & 	449.234	 & 	465.177	 \\
\end{tabular}
\columnbreak

Molecule B \
hexane\_out\_G09\_invertion

\includegraphics[width=6cm]{../Comparisons/ImagesFromVMD/hexane_out_G09_invertion.png}

Inertia Tensor - Molecule B \\
\begin{tabular}{|c c c|}
346.245	 & 	-179.129	 & 	0	 \\
-179.129	 & 	137.676	 & 	0	 \\
0	 & 	0	 & 	465.177
\end{tabular}

\vtab
 EingenVectors - Molecule B     \\
\begin{tabular}{|c c c|}
-0.498435	 & 	-0.866927	 & 	0	 \\
0.866927	 & 	-0.498435	 & 	-0	 \\
0	 & 	0	 & 	1
\end{tabular}

\vtab
 EingenValues - Molecule B     \\
\begin{tabular}{|c c c|}
34.6868	 & 	449.234	 & 	465.177	 \\
\end{tabular}

\end{center}
\end{multicols}

\vtab[-5mm]
\begin{tabular}{*{2}{m{0.38\textwidth}}}
\begin{center}
\textcolor{NavyBlue}{\Large Equal}
\end{center}
&
\begin{center}
\includegraphics[height=6.5cm]{../Comparisons/Vectors/inertia_tensor_of_hexane_out_G09_and_hexane_out_G09_invertion.png}
\end{center}
\end{tabular}

 \newpage

\vtab[-3cm]
\begin{center}
{\large NeoConformation \tab Número 523}
\end{center}
\begin{multicols}{2}
\begin{center}

Molecule A \
neopentane\_Symmetry\_out\_G09

\includegraphics[width=6cm]{../Comparisons/ImagesFromVMD/neopentane_Symmetry_out_G09.png}

Inertia Tensor - Molecule A \\
\begin{tabular}{|c c c|}
114.662	 & 	0.000161889	 & 	0.000173717	 \\
0.000161889	 & 	114.667	 & 	-0.000279052	 \\
0.000173717	 & 	-0.000279052	 & 	114.672
\end{tabular}

\vtab
 EingenVectors - Molecule A     \\
\begin{tabular}{|c c c|}
-0.999243	 & 	0.0340426	 & 	0.0188303	 \\
-0.0350564	 & 	-0.997791	 & 	-0.0564246	 \\
0.0168679	 & 	-0.057042	 & 	0.998229
\end{tabular}

\vtab
 EingenValues - Molecule A     \\
\begin{tabular}{|c c c|}
114.662	 & 	114.667	 & 	114.672	 \\
\end{tabular}
\columnbreak

Molecule B \
neopentane\_Symmetry\_out\_G09\_invertion

\includegraphics[width=6cm]{../Comparisons/ImagesFromVMD/neopentane_Symmetry_out_G09_invertion.png}

Inertia Tensor - Molecule B \\
\begin{tabular}{|c c c|}
114.662	 & 	0.000161889	 & 	0.000173717	 \\
0.000161889	 & 	114.667	 & 	-0.000279052	 \\
0.000173717	 & 	-0.000279052	 & 	114.672
\end{tabular}

\vtab
 EingenVectors - Molecule B     \\
\begin{tabular}{|c c c|}
-0.999243	 & 	0.0340426	 & 	0.0188303	 \\
-0.0350564	 & 	-0.997791	 & 	-0.0564246	 \\
0.0168679	 & 	-0.057042	 & 	0.998229
\end{tabular}

\vtab
 EingenValues - Molecule B     \\
\begin{tabular}{|c c c|}
114.662	 & 	114.667	 & 	114.672	 \\
\end{tabular}

\end{center}
\end{multicols}

\vtab[-5mm]
\begin{tabular}{*{2}{m{0.38\textwidth}}}
\begin{center}
\textcolor{NavyBlue}{\Large Equal}
\end{center}
&
\begin{center}
\includegraphics[height=6.5cm]{../Comparisons/Vectors/inertia_tensor_of_neopentane_Symmetry_out_G09_and_neopentane_Symmetry_out_G09_invertion.png}
\end{center}
\end{tabular}

 \newpage

\vtab[-3cm]
\begin{center}
{\large NeoConformation \tab Número 524}
\end{center}
\begin{multicols}{2}
\begin{center}

Molecule A \
neopentane\_Symmetry\_out\_G09

\includegraphics[width=6cm]{../Comparisons/ImagesFromVMD/neopentane_Symmetry_out_G09.png}

Inertia Tensor - Molecule A \\
\begin{tabular}{|c c c|}
114.662	 & 	0.000161889	 & 	0.000173717	 \\
0.000161889	 & 	114.667	 & 	-0.000279052	 \\
0.000173717	 & 	-0.000279052	 & 	114.672
\end{tabular}

\vtab
 EingenVectors - Molecule A     \\
\begin{tabular}{|c c c|}
-0.999243	 & 	0.0340426	 & 	0.0188303	 \\
-0.0350564	 & 	-0.997791	 & 	-0.0564246	 \\
0.0168679	 & 	-0.057042	 & 	0.998229
\end{tabular}

\vtab
 EingenValues - Molecule A     \\
\begin{tabular}{|c c c|}
114.662	 & 	114.667	 & 	114.672	 \\
\end{tabular}
\columnbreak

Molecule B \
neopentane\_Symmetry\_out\_G09\_rot\_x45\_y45\_z60

\includegraphics[width=6cm]{../Comparisons/ImagesFromVMD/neopentane_Symmetry_out_G09_rot_x45_y45_z60.png}

Inertia Tensor - Molecule B \\
\begin{tabular}{|c c c|}
114.67	 & 	0.000461506	 & 	0.00322567	 \\
0.000461506	 & 	114.665	 & 	-0.00251676	 \\
0.00322567	 & 	-0.00251676	 & 	114.666
\end{tabular}

\vtab
 EingenVectors - Molecule B     \\
\begin{tabular}{|c c c|}
0.324711	 & 	-0.637189	 & 	-0.698965	 \\
-0.422625	 & 	-0.758878	 & 	0.495472	 \\
0.846139	 & 	-0.134515	 & 	0.515708
\end{tabular}

\vtab
 EingenValues - Molecule B     \\
\begin{tabular}{|c c c|}
114.662	 & 	114.667	 & 	114.672	 \\
\end{tabular}

\end{center}
\end{multicols}

\vtab[-5mm]
\begin{tabular}{*{2}{m{0.38\textwidth}}}
\begin{center}
\textcolor{NavyBlue}{\Large Equal}
\end{center}
&
\begin{center}
\includegraphics[height=6.5cm]{../Comparisons/Vectors/inertia_tensor_of_neopentane_Symmetry_out_G09_and_neopentane_Symmetry_out_G09_rot_x45_y45_z60.png}
\end{center}
\end{tabular}

 \newpage

\vtab[-3cm]
\begin{center}
{\large NeoConformation \tab Número 525}
\end{center}
\begin{multicols}{2}
\begin{center}

Molecule A \
neopentane\_Symmetry\_out\_G09

\includegraphics[width=6cm]{../Comparisons/ImagesFromVMD/neopentane_Symmetry_out_G09.png}

Inertia Tensor - Molecule A \\
\begin{tabular}{|c c c|}
114.662	 & 	0.000161889	 & 	0.000173717	 \\
0.000161889	 & 	114.667	 & 	-0.000279052	 \\
0.000173717	 & 	-0.000279052	 & 	114.672
\end{tabular}

\vtab
 EingenVectors - Molecule A     \\
\begin{tabular}{|c c c|}
-0.999243	 & 	0.0340426	 & 	0.0188303	 \\
-0.0350564	 & 	-0.997791	 & 	-0.0564246	 \\
0.0168679	 & 	-0.057042	 & 	0.998229
\end{tabular}

\vtab
 EingenValues - Molecule A     \\
\begin{tabular}{|c c c|}
114.662	 & 	114.667	 & 	114.672	 \\
\end{tabular}
\columnbreak

Molecule B \
neopentane\_out\_G09

\includegraphics[width=6cm]{../Comparisons/ImagesFromVMD/neopentane_out_G09.png}

Inertia Tensor - Molecule B \\
\begin{tabular}{|c c c|}
114.662	 & 	0.000172734	 & 	0.000184265	 \\
0.000172734	 & 	114.667	 & 	-0.000265353	 \\
0.000184265	 & 	-0.000265353	 & 	114.672
\end{tabular}

\vtab
 EingenVectors - Molecule B     \\
\begin{tabular}{|c c c|}
-0.999133	 & 	0.0364989	 & 	0.020046	 \\
-0.0375297	 & 	-0.997851	 & 	-0.053713	 \\
0.0180425	 & 	-0.0544187	 & 	0.998355
\end{tabular}

\vtab
 EingenValues - Molecule B     \\
\begin{tabular}{|c c c|}
114.662	 & 	114.667	 & 	114.672	 \\
\end{tabular}

\end{center}
\end{multicols}

\vtab[-5mm]
\begin{tabular}{*{2}{m{0.38\textwidth}}}
\begin{center}
\textcolor{NavyBlue}{\Large Equal}
\end{center}
&
\begin{center}
\includegraphics[height=6.5cm]{../Comparisons/Vectors/inertia_tensor_of_neopentane_Symmetry_out_G09_and_neopentane_out_G09.png}
\end{center}
\end{tabular}

 \newpage

\vtab[-3cm]
\begin{center}
{\large NeoConformation \tab Número 526}
\end{center}
\begin{multicols}{2}
\begin{center}

Molecule A \
neopentane\_Symmetry\_out\_G09

\includegraphics[width=6cm]{../Comparisons/ImagesFromVMD/neopentane_Symmetry_out_G09.png}

Inertia Tensor - Molecule A \\
\begin{tabular}{|c c c|}
114.662	 & 	0.000161889	 & 	0.000173717	 \\
0.000161889	 & 	114.667	 & 	-0.000279052	 \\
0.000173717	 & 	-0.000279052	 & 	114.672
\end{tabular}

\vtab
 EingenVectors - Molecule A     \\
\begin{tabular}{|c c c|}
-0.999243	 & 	0.0340426	 & 	0.0188303	 \\
-0.0350564	 & 	-0.997791	 & 	-0.0564246	 \\
0.0168679	 & 	-0.057042	 & 	0.998229
\end{tabular}

\vtab
 EingenValues - Molecule A     \\
\begin{tabular}{|c c c|}
114.662	 & 	114.667	 & 	114.672	 \\
\end{tabular}
\columnbreak

Molecule B \
neopentane\_out\_G09\_invertion

\includegraphics[width=6cm]{../Comparisons/ImagesFromVMD/neopentane_out_G09_invertion.png}

Inertia Tensor - Molecule B \\
\begin{tabular}{|c c c|}
114.662	 & 	0.000171373	 & 	0.000184362	 \\
0.000171373	 & 	114.667	 & 	-0.000257766	 \\
0.000184362	 & 	-0.000257766	 & 	114.672
\end{tabular}

\vtab
 EingenVectors - Molecule B     \\
\begin{tabular}{|c c c|}
-0.99917	 & 	0.035564	 & 	0.0198547	 \\
-0.0365606	 & 	-0.997961	 & 	-0.0523166	 \\
0.0179537	 & 	-0.0529991	 & 	0.998433
\end{tabular}

\vtab
 EingenValues - Molecule B     \\
\begin{tabular}{|c c c|}
114.662	 & 	114.667	 & 	114.672	 \\
\end{tabular}

\end{center}
\end{multicols}

\vtab[-5mm]
\begin{tabular}{*{2}{m{0.38\textwidth}}}
\begin{center}
\textcolor{NavyBlue}{\Large Equal}
\end{center}
&
\begin{center}
\includegraphics[height=6.5cm]{../Comparisons/Vectors/inertia_tensor_of_neopentane_Symmetry_out_G09_and_neopentane_out_G09_invertion.png}
\end{center}
\end{tabular}

 \newpage

\vtab[-3cm]
\begin{center}
{\large NeoConformation \tab Número 527}
\end{center}
\begin{multicols}{2}
\begin{center}

Molecule A \
neopentane\_Symmetry\_out\_G09

\includegraphics[width=6cm]{../Comparisons/ImagesFromVMD/neopentane_Symmetry_out_G09.png}

Inertia Tensor - Molecule A \\
\begin{tabular}{|c c c|}
114.662	 & 	0.000161889	 & 	0.000173717	 \\
0.000161889	 & 	114.667	 & 	-0.000279052	 \\
0.000173717	 & 	-0.000279052	 & 	114.672
\end{tabular}

\vtab
 EingenVectors - Molecule A     \\
\begin{tabular}{|c c c|}
-0.999243	 & 	0.0340426	 & 	0.0188303	 \\
-0.0350564	 & 	-0.997791	 & 	-0.0564246	 \\
0.0168679	 & 	-0.057042	 & 	0.998229
\end{tabular}

\vtab
 EingenValues - Molecule A     \\
\begin{tabular}{|c c c|}
114.662	 & 	114.667	 & 	114.672	 \\
\end{tabular}
\columnbreak

Molecule B \
neopentane\_out\_G09\_rot\_x15-y15-z15

\includegraphics[width=6cm]{../Comparisons/ImagesFromVMD/neopentane_out_G09_rot_x15-y15-z15.png}

Inertia Tensor - Molecule B \\
\begin{tabular}{|c c c|}
114.663	 & 	0.00148446	 & 	0.00262579	 \\
0.00148446	 & 	114.666	 & 	0.000207396	 \\
0.00262579	 & 	0.000207396	 & 	114.671
\end{tabular}

\vtab
 EingenVectors - Molecule B     \\
\begin{tabular}{|c c c|}
0.919257	 & 	-0.288184	 & 	-0.268173	 \\
-0.236481	 & 	-0.948878	 & 	0.209061	 \\
0.314712	 & 	0.128763	 & 	0.940413
\end{tabular}

\vtab
 EingenValues - Molecule B     \\
\begin{tabular}{|c c c|}
114.662	 & 	114.667	 & 	114.672	 \\
\end{tabular}

\end{center}
\end{multicols}

\vtab[-5mm]
\begin{tabular}{*{2}{m{0.38\textwidth}}}
\begin{center}
\textcolor{NavyBlue}{\Large Equal}
\end{center}
&
\begin{center}
\includegraphics[height=6.5cm]{../Comparisons/Vectors/inertia_tensor_of_neopentane_Symmetry_out_G09_and_neopentane_out_G09_rot_x15-y15-z15.png}
\end{center}
\end{tabular}

 \newpage

\vtab[-3cm]
\begin{center}
{\large NeoConformation \tab Número 528}
\end{center}
\begin{multicols}{2}
\begin{center}
Molecule A \\ 
neopentane\_Symmetry\_out\_G09
\includegraphics[width=8cm]{../Comparisons/ImagesFromVMD/neopentane_Symmetry_out_G09.png}
\\
\vtab

\columnbreak
Molecule B \\ 
tert-butylamine\_out\_G09
\includegraphics[width=8cm]{../Comparisons/ImagesFromVMD/tert-butylamine_out_G09.png}
\\
\vtab


\end{center}
\end{multicols}
\begin{center}
\textcolor{NavyBlue}{\Large Different}
\end{center}

 \newpage

\vtab[-3cm]
\begin{center}
{\large NeoConformation \tab Número 529}
\end{center}
\begin{multicols}{2}
\begin{center}
Molecule A \\ 
neopentane\_Symmetry\_out\_G09
\includegraphics[width=8cm]{../Comparisons/ImagesFromVMD/neopentane_Symmetry_out_G09.png}
\\
\vtab

\columnbreak
Molecule B \\ 
tert-butylamine\_out\_G09\_invertion
\includegraphics[width=8cm]{../Comparisons/ImagesFromVMD/tert-butylamine_out_G09_invertion.png}
\\
\vtab


\end{center}
\end{multicols}
\begin{center}
\textcolor{NavyBlue}{\Large Different}
\end{center}

 \newpage

\vtab[-3cm]
\begin{center}
{\large NeoConformation \tab Número 530}
\end{center}
\begin{multicols}{2}
\begin{center}
Molecule A \\ 
neopentane\_Symmetry\_out\_G09
\includegraphics[width=8cm]{../Comparisons/ImagesFromVMD/neopentane_Symmetry_out_G09.png}
\\
\vtab

\columnbreak
Molecule B \\ 
tert-butylamine\_out\_G09\_rot\_x15\_y15\_z15
\includegraphics[width=8cm]{../Comparisons/ImagesFromVMD/tert-butylamine_out_G09_rot_x15_y15_z15.png}
\\
\vtab


\end{center}
\end{multicols}
\begin{center}
\textcolor{NavyBlue}{\Large Different}
\end{center}

 \newpage

\vtab[-3cm]
\begin{center}
{\large NeoConformation \tab Número 531}
\end{center}
\begin{multicols}{2}
\begin{center}
Molecule A \\ 
neopentane\_Symmetry\_out\_G09
\includegraphics[width=8cm]{../Comparisons/ImagesFromVMD/neopentane_Symmetry_out_G09.png}
\\
\vtab

\columnbreak
Molecule B \\ 
tetramethylsilane\_out\_G09
\includegraphics[width=8cm]{../Comparisons/ImagesFromVMD/tetramethylsilane_out_G09.png}
\\
\vtab


\end{center}
\end{multicols}
\begin{center}
\textcolor{NavyBlue}{\Large Different}
\end{center}

 \newpage

\vtab[-3cm]
\begin{center}
{\large NeoConformation \tab Número 532}
\end{center}
\begin{multicols}{2}
\begin{center}
Molecule A \\ 
neopentane\_Symmetry\_out\_G09
\includegraphics[width=8cm]{../Comparisons/ImagesFromVMD/neopentane_Symmetry_out_G09.png}
\\
\vtab

\columnbreak
Molecule B \\ 
tetramethylsilane\_out\_G09\_invertion
\includegraphics[width=8cm]{../Comparisons/ImagesFromVMD/tetramethylsilane_out_G09_invertion.png}
\\
\vtab


\end{center}
\end{multicols}
\begin{center}
\textcolor{NavyBlue}{\Large Different}
\end{center}

 \newpage

\vtab[-3cm]
\begin{center}
{\large NeoConformation \tab Número 533}
\end{center}
\begin{multicols}{2}
\begin{center}

Molecule A \
neopentane\_Symmetry\_out\_G09\_invertion

\includegraphics[width=6cm]{../Comparisons/ImagesFromVMD/neopentane_Symmetry_out_G09_invertion.png}

Inertia Tensor - Molecule A \\
\begin{tabular}{|c c c|}
114.662	 & 	0.000161889	 & 	0.000173717	 \\
0.000161889	 & 	114.667	 & 	-0.000279052	 \\
0.000173717	 & 	-0.000279052	 & 	114.672
\end{tabular}

\vtab
 EingenVectors - Molecule A     \\
\begin{tabular}{|c c c|}
-0.999243	 & 	0.0340426	 & 	0.0188303	 \\
-0.0350564	 & 	-0.997791	 & 	-0.0564246	 \\
0.0168679	 & 	-0.057042	 & 	0.998229
\end{tabular}

\vtab
 EingenValues - Molecule A     \\
\begin{tabular}{|c c c|}
114.662	 & 	114.667	 & 	114.672	 \\
\end{tabular}
\columnbreak

Molecule B \
neopentane\_Symmetry\_out\_G09\_rot\_x45\_y45\_z60

\includegraphics[width=6cm]{../Comparisons/ImagesFromVMD/neopentane_Symmetry_out_G09_rot_x45_y45_z60.png}

Inertia Tensor - Molecule B \\
\begin{tabular}{|c c c|}
114.67	 & 	0.000461506	 & 	0.00322567	 \\
0.000461506	 & 	114.665	 & 	-0.00251676	 \\
0.00322567	 & 	-0.00251676	 & 	114.666
\end{tabular}

\vtab
 EingenVectors - Molecule B     \\
\begin{tabular}{|c c c|}
0.324711	 & 	-0.637189	 & 	-0.698965	 \\
-0.422625	 & 	-0.758878	 & 	0.495472	 \\
0.846139	 & 	-0.134515	 & 	0.515708
\end{tabular}

\vtab
 EingenValues - Molecule B     \\
\begin{tabular}{|c c c|}
114.662	 & 	114.667	 & 	114.672	 \\
\end{tabular}

\end{center}
\end{multicols}

\vtab[-5mm]
\begin{tabular}{*{2}{m{0.38\textwidth}}}
\begin{center}
\textcolor{NavyBlue}{\Large Equal}
\end{center}
&
\begin{center}
\includegraphics[height=6.5cm]{../Comparisons/Vectors/inertia_tensor_of_neopentane_Symmetry_out_G09_invertion_and_neopentane_Symmetry_out_G09_rot_x45_y45_z60.png}
\end{center}
\end{tabular}

 \newpage

\vtab[-3cm]
\begin{center}
{\large NeoConformation \tab Número 534}
\end{center}
\begin{multicols}{2}
\begin{center}

Molecule A \
neopentane\_Symmetry\_out\_G09\_invertion

\includegraphics[width=6cm]{../Comparisons/ImagesFromVMD/neopentane_Symmetry_out_G09_invertion.png}

Inertia Tensor - Molecule A \\
\begin{tabular}{|c c c|}
114.662	 & 	0.000161889	 & 	0.000173717	 \\
0.000161889	 & 	114.667	 & 	-0.000279052	 \\
0.000173717	 & 	-0.000279052	 & 	114.672
\end{tabular}

\vtab
 EingenVectors - Molecule A     \\
\begin{tabular}{|c c c|}
-0.999243	 & 	0.0340426	 & 	0.0188303	 \\
-0.0350564	 & 	-0.997791	 & 	-0.0564246	 \\
0.0168679	 & 	-0.057042	 & 	0.998229
\end{tabular}

\vtab
 EingenValues - Molecule A     \\
\begin{tabular}{|c c c|}
114.662	 & 	114.667	 & 	114.672	 \\
\end{tabular}
\columnbreak

Molecule B \
neopentane\_out\_G09

\includegraphics[width=6cm]{../Comparisons/ImagesFromVMD/neopentane_out_G09.png}

Inertia Tensor - Molecule B \\
\begin{tabular}{|c c c|}
114.662	 & 	0.000172734	 & 	0.000184265	 \\
0.000172734	 & 	114.667	 & 	-0.000265353	 \\
0.000184265	 & 	-0.000265353	 & 	114.672
\end{tabular}

\vtab
 EingenVectors - Molecule B     \\
\begin{tabular}{|c c c|}
-0.999133	 & 	0.0364989	 & 	0.020046	 \\
-0.0375297	 & 	-0.997851	 & 	-0.053713	 \\
0.0180425	 & 	-0.0544187	 & 	0.998355
\end{tabular}

\vtab
 EingenValues - Molecule B     \\
\begin{tabular}{|c c c|}
114.662	 & 	114.667	 & 	114.672	 \\
\end{tabular}

\end{center}
\end{multicols}

\vtab[-5mm]
\begin{tabular}{*{2}{m{0.38\textwidth}}}
\begin{center}
\textcolor{NavyBlue}{\Large Equal}
\end{center}
&
\begin{center}
\includegraphics[height=6.5cm]{../Comparisons/Vectors/inertia_tensor_of_neopentane_Symmetry_out_G09_invertion_and_neopentane_out_G09.png}
\end{center}
\end{tabular}

 \newpage

\vtab[-3cm]
\begin{center}
{\large NeoConformation \tab Número 535}
\end{center}
\begin{multicols}{2}
\begin{center}

Molecule A \
neopentane\_Symmetry\_out\_G09\_invertion

\includegraphics[width=6cm]{../Comparisons/ImagesFromVMD/neopentane_Symmetry_out_G09_invertion.png}

Inertia Tensor - Molecule A \\
\begin{tabular}{|c c c|}
114.662	 & 	0.000161889	 & 	0.000173717	 \\
0.000161889	 & 	114.667	 & 	-0.000279052	 \\
0.000173717	 & 	-0.000279052	 & 	114.672
\end{tabular}

\vtab
 EingenVectors - Molecule A     \\
\begin{tabular}{|c c c|}
-0.999243	 & 	0.0340426	 & 	0.0188303	 \\
-0.0350564	 & 	-0.997791	 & 	-0.0564246	 \\
0.0168679	 & 	-0.057042	 & 	0.998229
\end{tabular}

\vtab
 EingenValues - Molecule A     \\
\begin{tabular}{|c c c|}
114.662	 & 	114.667	 & 	114.672	 \\
\end{tabular}
\columnbreak

Molecule B \
neopentane\_out\_G09\_invertion

\includegraphics[width=6cm]{../Comparisons/ImagesFromVMD/neopentane_out_G09_invertion.png}

Inertia Tensor - Molecule B \\
\begin{tabular}{|c c c|}
114.662	 & 	0.000171373	 & 	0.000184362	 \\
0.000171373	 & 	114.667	 & 	-0.000257766	 \\
0.000184362	 & 	-0.000257766	 & 	114.672
\end{tabular}

\vtab
 EingenVectors - Molecule B     \\
\begin{tabular}{|c c c|}
-0.99917	 & 	0.035564	 & 	0.0198547	 \\
-0.0365606	 & 	-0.997961	 & 	-0.0523166	 \\
0.0179537	 & 	-0.0529991	 & 	0.998433
\end{tabular}

\vtab
 EingenValues - Molecule B     \\
\begin{tabular}{|c c c|}
114.662	 & 	114.667	 & 	114.672	 \\
\end{tabular}

\end{center}
\end{multicols}

\vtab[-5mm]
\begin{tabular}{*{2}{m{0.38\textwidth}}}
\begin{center}
\textcolor{NavyBlue}{\Large Equal}
\end{center}
&
\begin{center}
\includegraphics[height=6.5cm]{../Comparisons/Vectors/inertia_tensor_of_neopentane_Symmetry_out_G09_invertion_and_neopentane_out_G09_invertion.png}
\end{center}
\end{tabular}

 \newpage

\vtab[-3cm]
\begin{center}
{\large NeoConformation \tab Número 536}
\end{center}
\begin{multicols}{2}
\begin{center}

Molecule A \
neopentane\_Symmetry\_out\_G09\_invertion

\includegraphics[width=6cm]{../Comparisons/ImagesFromVMD/neopentane_Symmetry_out_G09_invertion.png}

Inertia Tensor - Molecule A \\
\begin{tabular}{|c c c|}
114.662	 & 	0.000161889	 & 	0.000173717	 \\
0.000161889	 & 	114.667	 & 	-0.000279052	 \\
0.000173717	 & 	-0.000279052	 & 	114.672
\end{tabular}

\vtab
 EingenVectors - Molecule A     \\
\begin{tabular}{|c c c|}
-0.999243	 & 	0.0340426	 & 	0.0188303	 \\
-0.0350564	 & 	-0.997791	 & 	-0.0564246	 \\
0.0168679	 & 	-0.057042	 & 	0.998229
\end{tabular}

\vtab
 EingenValues - Molecule A     \\
\begin{tabular}{|c c c|}
114.662	 & 	114.667	 & 	114.672	 \\
\end{tabular}
\columnbreak

Molecule B \
neopentane\_out\_G09\_rot\_x15-y15-z15

\includegraphics[width=6cm]{../Comparisons/ImagesFromVMD/neopentane_out_G09_rot_x15-y15-z15.png}

Inertia Tensor - Molecule B \\
\begin{tabular}{|c c c|}
114.663	 & 	0.00148446	 & 	0.00262579	 \\
0.00148446	 & 	114.666	 & 	0.000207396	 \\
0.00262579	 & 	0.000207396	 & 	114.671
\end{tabular}

\vtab
 EingenVectors - Molecule B     \\
\begin{tabular}{|c c c|}
0.919257	 & 	-0.288184	 & 	-0.268173	 \\
-0.236481	 & 	-0.948878	 & 	0.209061	 \\
0.314712	 & 	0.128763	 & 	0.940413
\end{tabular}

\vtab
 EingenValues - Molecule B     \\
\begin{tabular}{|c c c|}
114.662	 & 	114.667	 & 	114.672	 \\
\end{tabular}

\end{center}
\end{multicols}

\vtab[-5mm]
\begin{tabular}{*{2}{m{0.38\textwidth}}}
\begin{center}
\textcolor{NavyBlue}{\Large Equal}
\end{center}
&
\begin{center}
\includegraphics[height=6.5cm]{../Comparisons/Vectors/inertia_tensor_of_neopentane_Symmetry_out_G09_invertion_and_neopentane_out_G09_rot_x15-y15-z15.png}
\end{center}
\end{tabular}

 \newpage

\vtab[-3cm]
\begin{center}
{\large NeoConformation \tab Número 537}
\end{center}
\begin{multicols}{2}
\begin{center}
Molecule A \\ 
neopentane\_Symmetry\_out\_G09\_invertion
\includegraphics[width=8cm]{../Comparisons/ImagesFromVMD/neopentane_Symmetry_out_G09_invertion.png}
\\
\vtab

\columnbreak
Molecule B \\ 
tert-butylamine\_out\_G09
\includegraphics[width=8cm]{../Comparisons/ImagesFromVMD/tert-butylamine_out_G09.png}
\\
\vtab


\end{center}
\end{multicols}
\begin{center}
\textcolor{NavyBlue}{\Large Different}
\end{center}

 \newpage

\vtab[-3cm]
\begin{center}
{\large NeoConformation \tab Número 538}
\end{center}
\begin{multicols}{2}
\begin{center}
Molecule A \\ 
neopentane\_Symmetry\_out\_G09\_invertion
\includegraphics[width=8cm]{../Comparisons/ImagesFromVMD/neopentane_Symmetry_out_G09_invertion.png}
\\
\vtab

\columnbreak
Molecule B \\ 
tert-butylamine\_out\_G09\_invertion
\includegraphics[width=8cm]{../Comparisons/ImagesFromVMD/tert-butylamine_out_G09_invertion.png}
\\
\vtab


\end{center}
\end{multicols}
\begin{center}
\textcolor{NavyBlue}{\Large Different}
\end{center}

 \newpage

\vtab[-3cm]
\begin{center}
{\large NeoConformation \tab Número 539}
\end{center}
\begin{multicols}{2}
\begin{center}
Molecule A \\ 
neopentane\_Symmetry\_out\_G09\_invertion
\includegraphics[width=8cm]{../Comparisons/ImagesFromVMD/neopentane_Symmetry_out_G09_invertion.png}
\\
\vtab

\columnbreak
Molecule B \\ 
tert-butylamine\_out\_G09\_rot\_x15\_y15\_z15
\includegraphics[width=8cm]{../Comparisons/ImagesFromVMD/tert-butylamine_out_G09_rot_x15_y15_z15.png}
\\
\vtab


\end{center}
\end{multicols}
\begin{center}
\textcolor{NavyBlue}{\Large Different}
\end{center}

 \newpage

\vtab[-3cm]
\begin{center}
{\large NeoConformation \tab Número 540}
\end{center}
\begin{multicols}{2}
\begin{center}
Molecule A \\ 
neopentane\_Symmetry\_out\_G09\_invertion
\includegraphics[width=8cm]{../Comparisons/ImagesFromVMD/neopentane_Symmetry_out_G09_invertion.png}
\\
\vtab

\columnbreak
Molecule B \\ 
tetramethylsilane\_out\_G09
\includegraphics[width=8cm]{../Comparisons/ImagesFromVMD/tetramethylsilane_out_G09.png}
\\
\vtab


\end{center}
\end{multicols}
\begin{center}
\textcolor{NavyBlue}{\Large Different}
\end{center}

 \newpage

\vtab[-3cm]
\begin{center}
{\large NeoConformation \tab Número 541}
\end{center}
\begin{multicols}{2}
\begin{center}
Molecule A \\ 
neopentane\_Symmetry\_out\_G09\_invertion
\includegraphics[width=8cm]{../Comparisons/ImagesFromVMD/neopentane_Symmetry_out_G09_invertion.png}
\\
\vtab

\columnbreak
Molecule B \\ 
tetramethylsilane\_out\_G09\_invertion
\includegraphics[width=8cm]{../Comparisons/ImagesFromVMD/tetramethylsilane_out_G09_invertion.png}
\\
\vtab


\end{center}
\end{multicols}
\begin{center}
\textcolor{NavyBlue}{\Large Different}
\end{center}

 \newpage

\vtab[-3cm]
\begin{center}
{\large NeoConformation \tab Número 542}
\end{center}
\begin{multicols}{2}
\begin{center}

Molecule A \
neopentane\_Symmetry\_out\_G09\_rot\_x45\_y45\_z60

\includegraphics[width=6cm]{../Comparisons/ImagesFromVMD/neopentane_Symmetry_out_G09_rot_x45_y45_z60.png}

Inertia Tensor - Molecule A \\
\begin{tabular}{|c c c|}
114.67	 & 	0.000461506	 & 	0.00322567	 \\
0.000461506	 & 	114.665	 & 	-0.00251676	 \\
0.00322567	 & 	-0.00251676	 & 	114.666
\end{tabular}

\vtab
 EingenVectors - Molecule A     \\
\begin{tabular}{|c c c|}
0.324711	 & 	-0.637189	 & 	-0.698965	 \\
-0.422625	 & 	-0.758878	 & 	0.495472	 \\
0.846139	 & 	-0.134515	 & 	0.515708
\end{tabular}

\vtab
 EingenValues - Molecule A     \\
\begin{tabular}{|c c c|}
114.662	 & 	114.667	 & 	114.672	 \\
\end{tabular}
\columnbreak

Molecule B \
neopentane\_out\_G09

\includegraphics[width=6cm]{../Comparisons/ImagesFromVMD/neopentane_out_G09.png}

Inertia Tensor - Molecule B \\
\begin{tabular}{|c c c|}
114.662	 & 	0.000172734	 & 	0.000184265	 \\
0.000172734	 & 	114.667	 & 	-0.000265353	 \\
0.000184265	 & 	-0.000265353	 & 	114.672
\end{tabular}

\vtab
 EingenVectors - Molecule B     \\
\begin{tabular}{|c c c|}
-0.999133	 & 	0.0364989	 & 	0.020046	 \\
-0.0375297	 & 	-0.997851	 & 	-0.053713	 \\
0.0180425	 & 	-0.0544187	 & 	0.998355
\end{tabular}

\vtab
 EingenValues - Molecule B     \\
\begin{tabular}{|c c c|}
114.662	 & 	114.667	 & 	114.672	 \\
\end{tabular}

\end{center}
\end{multicols}

\vtab[-5mm]
\begin{tabular}{*{2}{m{0.38\textwidth}}}
\begin{center}
\textcolor{NavyBlue}{\Large Equal}
\end{center}
&
\begin{center}
\includegraphics[height=6.5cm]{../Comparisons/Vectors/inertia_tensor_of_neopentane_Symmetry_out_G09_rot_x45_y45_z60_and_neopentane_out_G09.png}
\end{center}
\end{tabular}

 \newpage

\vtab[-3cm]
\begin{center}
{\large NeoConformation \tab Número 543}
\end{center}
\begin{multicols}{2}
\begin{center}

Molecule A \
neopentane\_Symmetry\_out\_G09\_rot\_x45\_y45\_z60

\includegraphics[width=6cm]{../Comparisons/ImagesFromVMD/neopentane_Symmetry_out_G09_rot_x45_y45_z60.png}

Inertia Tensor - Molecule A \\
\begin{tabular}{|c c c|}
114.67	 & 	0.000461506	 & 	0.00322567	 \\
0.000461506	 & 	114.665	 & 	-0.00251676	 \\
0.00322567	 & 	-0.00251676	 & 	114.666
\end{tabular}

\vtab
 EingenVectors - Molecule A     \\
\begin{tabular}{|c c c|}
0.324711	 & 	-0.637189	 & 	-0.698965	 \\
-0.422625	 & 	-0.758878	 & 	0.495472	 \\
0.846139	 & 	-0.134515	 & 	0.515708
\end{tabular}

\vtab
 EingenValues - Molecule A     \\
\begin{tabular}{|c c c|}
114.662	 & 	114.667	 & 	114.672	 \\
\end{tabular}
\columnbreak

Molecule B \
neopentane\_out\_G09\_invertion

\includegraphics[width=6cm]{../Comparisons/ImagesFromVMD/neopentane_out_G09_invertion.png}

Inertia Tensor - Molecule B \\
\begin{tabular}{|c c c|}
114.662	 & 	0.000171373	 & 	0.000184362	 \\
0.000171373	 & 	114.667	 & 	-0.000257766	 \\
0.000184362	 & 	-0.000257766	 & 	114.672
\end{tabular}

\vtab
 EingenVectors - Molecule B     \\
\begin{tabular}{|c c c|}
-0.99917	 & 	0.035564	 & 	0.0198547	 \\
-0.0365606	 & 	-0.997961	 & 	-0.0523166	 \\
0.0179537	 & 	-0.0529991	 & 	0.998433
\end{tabular}

\vtab
 EingenValues - Molecule B     \\
\begin{tabular}{|c c c|}
114.662	 & 	114.667	 & 	114.672	 \\
\end{tabular}

\end{center}
\end{multicols}

\vtab[-5mm]
\begin{tabular}{*{2}{m{0.38\textwidth}}}
\begin{center}
\textcolor{NavyBlue}{\Large Equal}
\end{center}
&
\begin{center}
\includegraphics[height=6.5cm]{../Comparisons/Vectors/inertia_tensor_of_neopentane_Symmetry_out_G09_rot_x45_y45_z60_and_neopentane_out_G09_invertion.png}
\end{center}
\end{tabular}

 \newpage

\vtab[-3cm]
\begin{center}
{\large NeoConformation \tab Número 544}
\end{center}
\begin{multicols}{2}
\begin{center}

Molecule A \
neopentane\_Symmetry\_out\_G09\_rot\_x45\_y45\_z60

\includegraphics[width=6cm]{../Comparisons/ImagesFromVMD/neopentane_Symmetry_out_G09_rot_x45_y45_z60.png}

Inertia Tensor - Molecule A \\
\begin{tabular}{|c c c|}
114.67	 & 	0.000461506	 & 	0.00322567	 \\
0.000461506	 & 	114.665	 & 	-0.00251676	 \\
0.00322567	 & 	-0.00251676	 & 	114.666
\end{tabular}

\vtab
 EingenVectors - Molecule A     \\
\begin{tabular}{|c c c|}
0.324711	 & 	-0.637189	 & 	-0.698965	 \\
-0.422625	 & 	-0.758878	 & 	0.495472	 \\
0.846139	 & 	-0.134515	 & 	0.515708
\end{tabular}

\vtab
 EingenValues - Molecule A     \\
\begin{tabular}{|c c c|}
114.662	 & 	114.667	 & 	114.672	 \\
\end{tabular}
\columnbreak

Molecule B \
neopentane\_out\_G09\_rot\_x15-y15-z15

\includegraphics[width=6cm]{../Comparisons/ImagesFromVMD/neopentane_out_G09_rot_x15-y15-z15.png}

Inertia Tensor - Molecule B \\
\begin{tabular}{|c c c|}
114.663	 & 	0.00148446	 & 	0.00262579	 \\
0.00148446	 & 	114.666	 & 	0.000207396	 \\
0.00262579	 & 	0.000207396	 & 	114.671
\end{tabular}

\vtab
 EingenVectors - Molecule B     \\
\begin{tabular}{|c c c|}
0.919257	 & 	-0.288184	 & 	-0.268173	 \\
-0.236481	 & 	-0.948878	 & 	0.209061	 \\
0.314712	 & 	0.128763	 & 	0.940413
\end{tabular}

\vtab
 EingenValues - Molecule B     \\
\begin{tabular}{|c c c|}
114.662	 & 	114.667	 & 	114.672	 \\
\end{tabular}

\end{center}
\end{multicols}

\vtab[-5mm]
\begin{tabular}{*{2}{m{0.38\textwidth}}}
\begin{center}
\textcolor{NavyBlue}{\Large Equal}
\end{center}
&
\begin{center}
\includegraphics[height=6.5cm]{../Comparisons/Vectors/inertia_tensor_of_neopentane_Symmetry_out_G09_rot_x45_y45_z60_and_neopentane_out_G09_rot_x15-y15-z15.png}
\end{center}
\end{tabular}

 \newpage

\vtab[-3cm]
\begin{center}
{\large NeoConformation \tab Número 545}
\end{center}
\begin{multicols}{2}
\begin{center}
Molecule A \\ 
neopentane\_Symmetry\_out\_G09\_rot\_x45\_y45\_z60
\includegraphics[width=8cm]{../Comparisons/ImagesFromVMD/neopentane_Symmetry_out_G09_rot_x45_y45_z60.png}
\\
\vtab

\columnbreak
Molecule B \\ 
tert-butylamine\_out\_G09
\includegraphics[width=8cm]{../Comparisons/ImagesFromVMD/tert-butylamine_out_G09.png}
\\
\vtab


\end{center}
\end{multicols}
\begin{center}
\textcolor{NavyBlue}{\Large Different}
\end{center}

 \newpage

\vtab[-3cm]
\begin{center}
{\large NeoConformation \tab Número 546}
\end{center}
\begin{multicols}{2}
\begin{center}
Molecule A \\ 
neopentane\_Symmetry\_out\_G09\_rot\_x45\_y45\_z60
\includegraphics[width=8cm]{../Comparisons/ImagesFromVMD/neopentane_Symmetry_out_G09_rot_x45_y45_z60.png}
\\
\vtab

\columnbreak
Molecule B \\ 
tert-butylamine\_out\_G09\_invertion
\includegraphics[width=8cm]{../Comparisons/ImagesFromVMD/tert-butylamine_out_G09_invertion.png}
\\
\vtab


\end{center}
\end{multicols}
\begin{center}
\textcolor{NavyBlue}{\Large Different}
\end{center}

 \newpage

\vtab[-3cm]
\begin{center}
{\large NeoConformation \tab Número 547}
\end{center}
\begin{multicols}{2}
\begin{center}
Molecule A \\ 
neopentane\_Symmetry\_out\_G09\_rot\_x45\_y45\_z60
\includegraphics[width=8cm]{../Comparisons/ImagesFromVMD/neopentane_Symmetry_out_G09_rot_x45_y45_z60.png}
\\
\vtab

\columnbreak
Molecule B \\ 
tert-butylamine\_out\_G09\_rot\_x15\_y15\_z15
\includegraphics[width=8cm]{../Comparisons/ImagesFromVMD/tert-butylamine_out_G09_rot_x15_y15_z15.png}
\\
\vtab


\end{center}
\end{multicols}
\begin{center}
\textcolor{NavyBlue}{\Large Different}
\end{center}

 \newpage

\vtab[-3cm]
\begin{center}
{\large NeoConformation \tab Número 548}
\end{center}
\begin{multicols}{2}
\begin{center}
Molecule A \\ 
neopentane\_Symmetry\_out\_G09\_rot\_x45\_y45\_z60
\includegraphics[width=8cm]{../Comparisons/ImagesFromVMD/neopentane_Symmetry_out_G09_rot_x45_y45_z60.png}
\\
\vtab

\columnbreak
Molecule B \\ 
tetramethylsilane\_out\_G09
\includegraphics[width=8cm]{../Comparisons/ImagesFromVMD/tetramethylsilane_out_G09.png}
\\
\vtab


\end{center}
\end{multicols}
\begin{center}
\textcolor{NavyBlue}{\Large Different}
\end{center}

 \newpage

\vtab[-3cm]
\begin{center}
{\large NeoConformation \tab Número 549}
\end{center}
\begin{multicols}{2}
\begin{center}
Molecule A \\ 
neopentane\_Symmetry\_out\_G09\_rot\_x45\_y45\_z60
\includegraphics[width=8cm]{../Comparisons/ImagesFromVMD/neopentane_Symmetry_out_G09_rot_x45_y45_z60.png}
\\
\vtab

\columnbreak
Molecule B \\ 
tetramethylsilane\_out\_G09\_invertion
\includegraphics[width=8cm]{../Comparisons/ImagesFromVMD/tetramethylsilane_out_G09_invertion.png}
\\
\vtab


\end{center}
\end{multicols}
\begin{center}
\textcolor{NavyBlue}{\Large Different}
\end{center}

 \newpage

\vtab[-3cm]
\begin{center}
{\large NeoConformation \tab Número 550}
\end{center}
\begin{multicols}{2}
\begin{center}

Molecule A \
neopentane\_out\_G09

\includegraphics[width=6cm]{../Comparisons/ImagesFromVMD/neopentane_out_G09.png}

Inertia Tensor - Molecule A \\
\begin{tabular}{|c c c|}
114.662	 & 	0.000172734	 & 	0.000184265	 \\
0.000172734	 & 	114.667	 & 	-0.000265353	 \\
0.000184265	 & 	-0.000265353	 & 	114.672
\end{tabular}

\vtab
 EingenVectors - Molecule A     \\
\begin{tabular}{|c c c|}
-0.999133	 & 	0.0364989	 & 	0.020046	 \\
-0.0375297	 & 	-0.997851	 & 	-0.053713	 \\
0.0180425	 & 	-0.0544187	 & 	0.998355
\end{tabular}

\vtab
 EingenValues - Molecule A     \\
\begin{tabular}{|c c c|}
114.662	 & 	114.667	 & 	114.672	 \\
\end{tabular}
\columnbreak

Molecule B \
neopentane\_out\_G09\_invertion

\includegraphics[width=6cm]{../Comparisons/ImagesFromVMD/neopentane_out_G09_invertion.png}

Inertia Tensor - Molecule B \\
\begin{tabular}{|c c c|}
114.662	 & 	0.000171373	 & 	0.000184362	 \\
0.000171373	 & 	114.667	 & 	-0.000257766	 \\
0.000184362	 & 	-0.000257766	 & 	114.672
\end{tabular}

\vtab
 EingenVectors - Molecule B     \\
\begin{tabular}{|c c c|}
-0.99917	 & 	0.035564	 & 	0.0198547	 \\
-0.0365606	 & 	-0.997961	 & 	-0.0523166	 \\
0.0179537	 & 	-0.0529991	 & 	0.998433
\end{tabular}

\vtab
 EingenValues - Molecule B     \\
\begin{tabular}{|c c c|}
114.662	 & 	114.667	 & 	114.672	 \\
\end{tabular}

\end{center}
\end{multicols}

\vtab[-5mm]
\begin{tabular}{*{2}{m{0.38\textwidth}}}
\begin{center}
\textcolor{NavyBlue}{\Large Equal}
\end{center}
&
\begin{center}
\includegraphics[height=6.5cm]{../Comparisons/Vectors/inertia_tensor_of_neopentane_out_G09_and_neopentane_out_G09_invertion.png}
\end{center}
\end{tabular}

 \newpage

\vtab[-3cm]
\begin{center}
{\large NeoConformation \tab Número 551}
\end{center}
\begin{multicols}{2}
\begin{center}

Molecule A \
neopentane\_out\_G09

\includegraphics[width=6cm]{../Comparisons/ImagesFromVMD/neopentane_out_G09.png}

Inertia Tensor - Molecule A \\
\begin{tabular}{|c c c|}
114.662	 & 	0.000172734	 & 	0.000184265	 \\
0.000172734	 & 	114.667	 & 	-0.000265353	 \\
0.000184265	 & 	-0.000265353	 & 	114.672
\end{tabular}

\vtab
 EingenVectors - Molecule A     \\
\begin{tabular}{|c c c|}
-0.999133	 & 	0.0364989	 & 	0.020046	 \\
-0.0375297	 & 	-0.997851	 & 	-0.053713	 \\
0.0180425	 & 	-0.0544187	 & 	0.998355
\end{tabular}

\vtab
 EingenValues - Molecule A     \\
\begin{tabular}{|c c c|}
114.662	 & 	114.667	 & 	114.672	 \\
\end{tabular}
\columnbreak

Molecule B \
neopentane\_out\_G09\_rot\_x15-y15-z15

\includegraphics[width=6cm]{../Comparisons/ImagesFromVMD/neopentane_out_G09_rot_x15-y15-z15.png}

Inertia Tensor - Molecule B \\
\begin{tabular}{|c c c|}
114.663	 & 	0.00148446	 & 	0.00262579	 \\
0.00148446	 & 	114.666	 & 	0.000207396	 \\
0.00262579	 & 	0.000207396	 & 	114.671
\end{tabular}

\vtab
 EingenVectors - Molecule B     \\
\begin{tabular}{|c c c|}
0.919257	 & 	-0.288184	 & 	-0.268173	 \\
-0.236481	 & 	-0.948878	 & 	0.209061	 \\
0.314712	 & 	0.128763	 & 	0.940413
\end{tabular}

\vtab
 EingenValues - Molecule B     \\
\begin{tabular}{|c c c|}
114.662	 & 	114.667	 & 	114.672	 \\
\end{tabular}

\end{center}
\end{multicols}

\vtab[-5mm]
\begin{tabular}{*{2}{m{0.38\textwidth}}}
\begin{center}
\textcolor{NavyBlue}{\Large Equal}
\end{center}
&
\begin{center}
\includegraphics[height=6.5cm]{../Comparisons/Vectors/inertia_tensor_of_neopentane_out_G09_and_neopentane_out_G09_rot_x15-y15-z15.png}
\end{center}
\end{tabular}

 \newpage

\vtab[-3cm]
\begin{center}
{\large NeoConformation \tab Número 552}
\end{center}
\begin{multicols}{2}
\begin{center}
Molecule A \\ 
neopentane\_out\_G09
\includegraphics[width=8cm]{../Comparisons/ImagesFromVMD/neopentane_out_G09.png}
\\
\vtab

\columnbreak
Molecule B \\ 
tert-butylamine\_out\_G09
\includegraphics[width=8cm]{../Comparisons/ImagesFromVMD/tert-butylamine_out_G09.png}
\\
\vtab


\end{center}
\end{multicols}
\begin{center}
\textcolor{NavyBlue}{\Large Different}
\end{center}

 \newpage

\vtab[-3cm]
\begin{center}
{\large NeoConformation \tab Número 553}
\end{center}
\begin{multicols}{2}
\begin{center}
Molecule A \\ 
neopentane\_out\_G09
\includegraphics[width=8cm]{../Comparisons/ImagesFromVMD/neopentane_out_G09.png}
\\
\vtab

\columnbreak
Molecule B \\ 
tert-butylamine\_out\_G09\_invertion
\includegraphics[width=8cm]{../Comparisons/ImagesFromVMD/tert-butylamine_out_G09_invertion.png}
\\
\vtab


\end{center}
\end{multicols}
\begin{center}
\textcolor{NavyBlue}{\Large Different}
\end{center}

 \newpage

\vtab[-3cm]
\begin{center}
{\large NeoConformation \tab Número 554}
\end{center}
\begin{multicols}{2}
\begin{center}
Molecule A \\ 
neopentane\_out\_G09
\includegraphics[width=8cm]{../Comparisons/ImagesFromVMD/neopentane_out_G09.png}
\\
\vtab

\columnbreak
Molecule B \\ 
tert-butylamine\_out\_G09\_rot\_x15\_y15\_z15
\includegraphics[width=8cm]{../Comparisons/ImagesFromVMD/tert-butylamine_out_G09_rot_x15_y15_z15.png}
\\
\vtab


\end{center}
\end{multicols}
\begin{center}
\textcolor{NavyBlue}{\Large Different}
\end{center}

 \newpage

\vtab[-3cm]
\begin{center}
{\large NeoConformation \tab Número 555}
\end{center}
\begin{multicols}{2}
\begin{center}
Molecule A \\ 
neopentane\_out\_G09
\includegraphics[width=8cm]{../Comparisons/ImagesFromVMD/neopentane_out_G09.png}
\\
\vtab

\columnbreak
Molecule B \\ 
tetramethylsilane\_out\_G09
\includegraphics[width=8cm]{../Comparisons/ImagesFromVMD/tetramethylsilane_out_G09.png}
\\
\vtab


\end{center}
\end{multicols}
\begin{center}
\textcolor{NavyBlue}{\Large Different}
\end{center}

 \newpage

\vtab[-3cm]
\begin{center}
{\large NeoConformation \tab Número 556}
\end{center}
\begin{multicols}{2}
\begin{center}
Molecule A \\ 
neopentane\_out\_G09
\includegraphics[width=8cm]{../Comparisons/ImagesFromVMD/neopentane_out_G09.png}
\\
\vtab

\columnbreak
Molecule B \\ 
tetramethylsilane\_out\_G09\_invertion
\includegraphics[width=8cm]{../Comparisons/ImagesFromVMD/tetramethylsilane_out_G09_invertion.png}
\\
\vtab


\end{center}
\end{multicols}
\begin{center}
\textcolor{NavyBlue}{\Large Different}
\end{center}

 \newpage

\vtab[-3cm]
\begin{center}
{\large NeoConformation \tab Número 557}
\end{center}
\begin{multicols}{2}
\begin{center}

Molecule A \
neopentane\_out\_G09\_invertion

\includegraphics[width=6cm]{../Comparisons/ImagesFromVMD/neopentane_out_G09_invertion.png}

Inertia Tensor - Molecule A \\
\begin{tabular}{|c c c|}
114.662	 & 	0.000171373	 & 	0.000184362	 \\
0.000171373	 & 	114.667	 & 	-0.000257766	 \\
0.000184362	 & 	-0.000257766	 & 	114.672
\end{tabular}

\vtab
 EingenVectors - Molecule A     \\
\begin{tabular}{|c c c|}
-0.99917	 & 	0.035564	 & 	0.0198547	 \\
-0.0365606	 & 	-0.997961	 & 	-0.0523166	 \\
0.0179537	 & 	-0.0529991	 & 	0.998433
\end{tabular}

\vtab
 EingenValues - Molecule A     \\
\begin{tabular}{|c c c|}
114.662	 & 	114.667	 & 	114.672	 \\
\end{tabular}
\columnbreak

Molecule B \
neopentane\_out\_G09\_rot\_x15-y15-z15

\includegraphics[width=6cm]{../Comparisons/ImagesFromVMD/neopentane_out_G09_rot_x15-y15-z15.png}

Inertia Tensor - Molecule B \\
\begin{tabular}{|c c c|}
114.663	 & 	0.00148446	 & 	0.00262579	 \\
0.00148446	 & 	114.666	 & 	0.000207396	 \\
0.00262579	 & 	0.000207396	 & 	114.671
\end{tabular}

\vtab
 EingenVectors - Molecule B     \\
\begin{tabular}{|c c c|}
0.919257	 & 	-0.288184	 & 	-0.268173	 \\
-0.236481	 & 	-0.948878	 & 	0.209061	 \\
0.314712	 & 	0.128763	 & 	0.940413
\end{tabular}

\vtab
 EingenValues - Molecule B     \\
\begin{tabular}{|c c c|}
114.662	 & 	114.667	 & 	114.672	 \\
\end{tabular}

\end{center}
\end{multicols}

\vtab[-5mm]
\begin{tabular}{*{2}{m{0.38\textwidth}}}
\begin{center}
\textcolor{NavyBlue}{\Large Equal}
\end{center}
&
\begin{center}
\includegraphics[height=6.5cm]{../Comparisons/Vectors/inertia_tensor_of_neopentane_out_G09_invertion_and_neopentane_out_G09_rot_x15-y15-z15.png}
\end{center}
\end{tabular}

 \newpage

\vtab[-3cm]
\begin{center}
{\large NeoConformation \tab Número 558}
\end{center}
\begin{multicols}{2}
\begin{center}
Molecule A \\ 
neopentane\_out\_G09\_invertion
\includegraphics[width=8cm]{../Comparisons/ImagesFromVMD/neopentane_out_G09_invertion.png}
\\
\vtab

\columnbreak
Molecule B \\ 
tert-butylamine\_out\_G09
\includegraphics[width=8cm]{../Comparisons/ImagesFromVMD/tert-butylamine_out_G09.png}
\\
\vtab


\end{center}
\end{multicols}
\begin{center}
\textcolor{NavyBlue}{\Large Different}
\end{center}

 \newpage

\vtab[-3cm]
\begin{center}
{\large NeoConformation \tab Número 559}
\end{center}
\begin{multicols}{2}
\begin{center}
Molecule A \\ 
neopentane\_out\_G09\_invertion
\includegraphics[width=8cm]{../Comparisons/ImagesFromVMD/neopentane_out_G09_invertion.png}
\\
\vtab

\columnbreak
Molecule B \\ 
tert-butylamine\_out\_G09\_invertion
\includegraphics[width=8cm]{../Comparisons/ImagesFromVMD/tert-butylamine_out_G09_invertion.png}
\\
\vtab


\end{center}
\end{multicols}
\begin{center}
\textcolor{NavyBlue}{\Large Different}
\end{center}

 \newpage

\vtab[-3cm]
\begin{center}
{\large NeoConformation \tab Número 560}
\end{center}
\begin{multicols}{2}
\begin{center}
Molecule A \\ 
neopentane\_out\_G09\_invertion
\includegraphics[width=8cm]{../Comparisons/ImagesFromVMD/neopentane_out_G09_invertion.png}
\\
\vtab

\columnbreak
Molecule B \\ 
tert-butylamine\_out\_G09\_rot\_x15\_y15\_z15
\includegraphics[width=8cm]{../Comparisons/ImagesFromVMD/tert-butylamine_out_G09_rot_x15_y15_z15.png}
\\
\vtab


\end{center}
\end{multicols}
\begin{center}
\textcolor{NavyBlue}{\Large Different}
\end{center}

 \newpage

\vtab[-3cm]
\begin{center}
{\large NeoConformation \tab Número 561}
\end{center}
\begin{multicols}{2}
\begin{center}
Molecule A \\ 
neopentane\_out\_G09\_invertion
\includegraphics[width=8cm]{../Comparisons/ImagesFromVMD/neopentane_out_G09_invertion.png}
\\
\vtab

\columnbreak
Molecule B \\ 
tetramethylsilane\_out\_G09
\includegraphics[width=8cm]{../Comparisons/ImagesFromVMD/tetramethylsilane_out_G09.png}
\\
\vtab


\end{center}
\end{multicols}
\begin{center}
\textcolor{NavyBlue}{\Large Different}
\end{center}

 \newpage

\vtab[-3cm]
\begin{center}
{\large NeoConformation \tab Número 562}
\end{center}
\begin{multicols}{2}
\begin{center}
Molecule A \\ 
neopentane\_out\_G09\_invertion
\includegraphics[width=8cm]{../Comparisons/ImagesFromVMD/neopentane_out_G09_invertion.png}
\\
\vtab

\columnbreak
Molecule B \\ 
tetramethylsilane\_out\_G09\_invertion
\includegraphics[width=8cm]{../Comparisons/ImagesFromVMD/tetramethylsilane_out_G09_invertion.png}
\\
\vtab


\end{center}
\end{multicols}
\begin{center}
\textcolor{NavyBlue}{\Large Different}
\end{center}

 \newpage

\vtab[-3cm]
\begin{center}
{\large NeoConformation \tab Número 563}
\end{center}
\begin{multicols}{2}
\begin{center}
Molecule A \\ 
neopentane\_out\_G09\_rot\_x15-y15-z15
\includegraphics[width=8cm]{../Comparisons/ImagesFromVMD/neopentane_out_G09_rot_x15-y15-z15.png}
\\
\vtab

\columnbreak
Molecule B \\ 
tert-butylamine\_out\_G09
\includegraphics[width=8cm]{../Comparisons/ImagesFromVMD/tert-butylamine_out_G09.png}
\\
\vtab


\end{center}
\end{multicols}
\begin{center}
\textcolor{NavyBlue}{\Large Different}
\end{center}

 \newpage

\vtab[-3cm]
\begin{center}
{\large NeoConformation \tab Número 564}
\end{center}
\begin{multicols}{2}
\begin{center}
Molecule A \\ 
neopentane\_out\_G09\_rot\_x15-y15-z15
\includegraphics[width=8cm]{../Comparisons/ImagesFromVMD/neopentane_out_G09_rot_x15-y15-z15.png}
\\
\vtab

\columnbreak
Molecule B \\ 
tert-butylamine\_out\_G09\_invertion
\includegraphics[width=8cm]{../Comparisons/ImagesFromVMD/tert-butylamine_out_G09_invertion.png}
\\
\vtab


\end{center}
\end{multicols}
\begin{center}
\textcolor{NavyBlue}{\Large Different}
\end{center}

 \newpage

\vtab[-3cm]
\begin{center}
{\large NeoConformation \tab Número 565}
\end{center}
\begin{multicols}{2}
\begin{center}
Molecule A \\ 
neopentane\_out\_G09\_rot\_x15-y15-z15
\includegraphics[width=8cm]{../Comparisons/ImagesFromVMD/neopentane_out_G09_rot_x15-y15-z15.png}
\\
\vtab

\columnbreak
Molecule B \\ 
tert-butylamine\_out\_G09\_rot\_x15\_y15\_z15
\includegraphics[width=8cm]{../Comparisons/ImagesFromVMD/tert-butylamine_out_G09_rot_x15_y15_z15.png}
\\
\vtab


\end{center}
\end{multicols}
\begin{center}
\textcolor{NavyBlue}{\Large Different}
\end{center}

 \newpage

\vtab[-3cm]
\begin{center}
{\large NeoConformation \tab Número 566}
\end{center}
\begin{multicols}{2}
\begin{center}
Molecule A \\ 
neopentane\_out\_G09\_rot\_x15-y15-z15
\includegraphics[width=8cm]{../Comparisons/ImagesFromVMD/neopentane_out_G09_rot_x15-y15-z15.png}
\\
\vtab

\columnbreak
Molecule B \\ 
tetramethylsilane\_out\_G09
\includegraphics[width=8cm]{../Comparisons/ImagesFromVMD/tetramethylsilane_out_G09.png}
\\
\vtab


\end{center}
\end{multicols}
\begin{center}
\textcolor{NavyBlue}{\Large Different}
\end{center}

 \newpage

\vtab[-3cm]
\begin{center}
{\large NeoConformation \tab Número 567}
\end{center}
\begin{multicols}{2}
\begin{center}
Molecule A \\ 
neopentane\_out\_G09\_rot\_x15-y15-z15
\includegraphics[width=8cm]{../Comparisons/ImagesFromVMD/neopentane_out_G09_rot_x15-y15-z15.png}
\\
\vtab

\columnbreak
Molecule B \\ 
tetramethylsilane\_out\_G09\_invertion
\includegraphics[width=8cm]{../Comparisons/ImagesFromVMD/tetramethylsilane_out_G09_invertion.png}
\\
\vtab


\end{center}
\end{multicols}
\begin{center}
\textcolor{NavyBlue}{\Large Different}
\end{center}

 \newpage

\vtab[-3cm]
\begin{center}
{\large NeoConformation \tab Número 568}
\end{center}
\begin{multicols}{2}
\begin{center}

Molecule A \
tert-butylamine\_out\_G09

\includegraphics[width=6cm]{../Comparisons/ImagesFromVMD/tert-butylamine_out_G09.png}

Inertia Tensor - Molecule A \\
\begin{tabular}{|c c c|}
110.655	 & 	-1.46159e-05	 & 	4.47474e-05	 \\
-1.46159e-05	 & 	111.529	 & 	0.82631	 \\
4.47474e-05	 & 	0.82631	 & 	113.378
\end{tabular}

\vtab
 EingenVectors - Molecule A     \\
\begin{tabular}{|c c c|}
-1	 & 	-4.53031e-05	 & 	3.01901e-05	 \\
5.30915e-05	 & 	-0.934229	 & 	0.356674	 \\
1.2046e-05	 & 	0.356674	 & 	0.934229
\end{tabular}

\vtab
 EingenValues - Molecule A     \\
\begin{tabular}{|c c c|}
110.655	 & 	111.213	 & 	113.693	 \\
\end{tabular}
\columnbreak

Molecule B \
tert-butylamine\_out\_G09\_invertion

\includegraphics[width=6cm]{../Comparisons/ImagesFromVMD/tert-butylamine_out_G09_invertion.png}

Inertia Tensor - Molecule B \\
\begin{tabular}{|c c c|}
110.655	 & 	-4.00493e-05	 & 	5.05193e-05	 \\
-4.00493e-05	 & 	111.529	 & 	0.826271	 \\
5.05193e-05	 & 	0.826271	 & 	113.378
\end{tabular}

\vtab
 EingenVectors - Molecule B     \\
\begin{tabular}{|c c c|}
-1	 & 	-8.8937e-05	 & 	4.55534e-05	 \\
9.93354e-05	 & 	-0.934224	 & 	0.356687	 \\
1.08344e-05	 & 	0.356687	 & 	0.934224
\end{tabular}

\vtab
 EingenValues - Molecule B     \\
\begin{tabular}{|c c c|}
110.655	 & 	111.213	 & 	113.693	 \\
\end{tabular}

\end{center}
\end{multicols}

\vtab[-5mm]
\begin{tabular}{*{2}{m{0.38\textwidth}}}
\begin{center}
\textcolor{NavyBlue}{\Large Enantiomers}
\end{center}
&
\begin{center}
\includegraphics[height=6.5cm]{../Comparisons/Vectors/inertia_tensor_of_tert-butylamine_out_G09_and_tert-butylamine_out_G09_invertion.png}
\end{center}
\end{tabular}

 \newpage

\vtab[-3cm]
\begin{center}
{\large NeoConformation \tab Número 569}
\end{center}
\begin{multicols}{2}
\begin{center}

Molecule A \
tert-butylamine\_out\_G09

\includegraphics[width=6cm]{../Comparisons/ImagesFromVMD/tert-butylamine_out_G09.png}

Inertia Tensor - Molecule A \\
\begin{tabular}{|c c c|}
110.655	 & 	-1.46159e-05	 & 	4.47474e-05	 \\
-1.46159e-05	 & 	111.529	 & 	0.82631	 \\
4.47474e-05	 & 	0.82631	 & 	113.378
\end{tabular}

\vtab
 EingenVectors - Molecule A     \\
\begin{tabular}{|c c c|}
-1	 & 	-4.53031e-05	 & 	3.01901e-05	 \\
5.30915e-05	 & 	-0.934229	 & 	0.356674	 \\
1.2046e-05	 & 	0.356674	 & 	0.934229
\end{tabular}

\vtab
 EingenValues - Molecule A     \\
\begin{tabular}{|c c c|}
110.655	 & 	111.213	 & 	113.693	 \\
\end{tabular}
\columnbreak

Molecule B \
tert-butylamine\_out\_G09\_rot\_x15\_y15\_z15

\includegraphics[width=6cm]{../Comparisons/ImagesFromVMD/tert-butylamine_out_G09_rot_x15_y15_z15.png}

Inertia Tensor - Molecule B \\
\begin{tabular}{|c c c|}
111.039	 & 	0.579971	 & 	0.822165	 \\
0.579971	 & 	111.829	 & 	0.957523	 \\
0.822165	 & 	0.957523	 & 	112.694
\end{tabular}

\vtab
 EingenVectors - Molecule B     \\
\begin{tabular}{|c c c|}
-0.933012	 & 	0.249963	 & 	0.258859	 \\
-0.0630357	 & 	-0.821768	 & 	0.566325	 \\
0.354282	 & 	0.512071	 & 	0.782476
\end{tabular}

\vtab
 EingenValues - Molecule B     \\
\begin{tabular}{|c c c|}
110.655	 & 	111.213	 & 	113.693	 \\
\end{tabular}

\end{center}
\end{multicols}

\vtab[-5mm]
\begin{tabular}{*{2}{m{0.38\textwidth}}}
\begin{center}
\textcolor{NavyBlue}{\Large Equal}
\end{center}
&
\begin{center}
\includegraphics[height=6.5cm]{../Comparisons/Vectors/inertia_tensor_of_tert-butylamine_out_G09_and_tert-butylamine_out_G09_rot_x15_y15_z15.png}
\end{center}
\end{tabular}

 \newpage

\vtab[-3cm]
\begin{center}
{\large NeoConformation \tab Número 570}
\end{center}
\begin{multicols}{2}
\begin{center}
Molecule A \\ 
tert-butylamine\_out\_G09
\includegraphics[width=8cm]{../Comparisons/ImagesFromVMD/tert-butylamine_out_G09.png}
\\
\vtab

\columnbreak
Molecule B \\ 
tetramethylsilane\_out\_G09
\includegraphics[width=8cm]{../Comparisons/ImagesFromVMD/tetramethylsilane_out_G09.png}
\\
\vtab


\end{center}
\end{multicols}
\begin{center}
\textcolor{NavyBlue}{\Large Different}
\end{center}

 \newpage

\vtab[-3cm]
\begin{center}
{\large NeoConformation \tab Número 571}
\end{center}
\begin{multicols}{2}
\begin{center}
Molecule A \\ 
tert-butylamine\_out\_G09
\includegraphics[width=8cm]{../Comparisons/ImagesFromVMD/tert-butylamine_out_G09.png}
\\
\vtab

\columnbreak
Molecule B \\ 
tetramethylsilane\_out\_G09\_invertion
\includegraphics[width=8cm]{../Comparisons/ImagesFromVMD/tetramethylsilane_out_G09_invertion.png}
\\
\vtab


\end{center}
\end{multicols}
\begin{center}
\textcolor{NavyBlue}{\Large Different}
\end{center}

 \newpage

\vtab[-3cm]
\begin{center}
{\large NeoConformation \tab Número 572}
\end{center}
\begin{multicols}{2}
\begin{center}

Molecule A \
tert-butylamine\_out\_G09\_invertion

\includegraphics[width=6cm]{../Comparisons/ImagesFromVMD/tert-butylamine_out_G09_invertion.png}

Inertia Tensor - Molecule A \\
\begin{tabular}{|c c c|}
110.655	 & 	-4.00493e-05	 & 	5.05193e-05	 \\
-4.00493e-05	 & 	111.529	 & 	0.826271	 \\
5.05193e-05	 & 	0.826271	 & 	113.378
\end{tabular}

\vtab
 EingenVectors - Molecule A     \\
\begin{tabular}{|c c c|}
-1	 & 	-8.8937e-05	 & 	4.55534e-05	 \\
9.93354e-05	 & 	-0.934224	 & 	0.356687	 \\
1.08344e-05	 & 	0.356687	 & 	0.934224
\end{tabular}

\vtab
 EingenValues - Molecule A     \\
\begin{tabular}{|c c c|}
110.655	 & 	111.213	 & 	113.693	 \\
\end{tabular}
\columnbreak

Molecule B \
tert-butylamine\_out\_G09\_rot\_x15\_y15\_z15

\includegraphics[width=6cm]{../Comparisons/ImagesFromVMD/tert-butylamine_out_G09_rot_x15_y15_z15.png}

Inertia Tensor - Molecule B \\
\begin{tabular}{|c c c|}
111.039	 & 	0.579971	 & 	0.822165	 \\
0.579971	 & 	111.829	 & 	0.957523	 \\
0.822165	 & 	0.957523	 & 	112.694
\end{tabular}

\vtab
 EingenVectors - Molecule B     \\
\begin{tabular}{|c c c|}
-0.933012	 & 	0.249963	 & 	0.258859	 \\
-0.0630357	 & 	-0.821768	 & 	0.566325	 \\
0.354282	 & 	0.512071	 & 	0.782476
\end{tabular}

\vtab
 EingenValues - Molecule B     \\
\begin{tabular}{|c c c|}
110.655	 & 	111.213	 & 	113.693	 \\
\end{tabular}

\end{center}
\end{multicols}

\vtab[-5mm]
\begin{tabular}{*{2}{m{0.38\textwidth}}}
\begin{center}
\textcolor{NavyBlue}{\Large Enantiomers}
\end{center}
&
\begin{center}
\includegraphics[height=6.5cm]{../Comparisons/Vectors/inertia_tensor_of_tert-butylamine_out_G09_invertion_and_tert-butylamine_out_G09_rot_x15_y15_z15.png}
\end{center}
\end{tabular}

 \newpage

\vtab[-3cm]
\begin{center}
{\large NeoConformation \tab Número 573}
\end{center}
\begin{multicols}{2}
\begin{center}
Molecule A \\ 
tert-butylamine\_out\_G09\_invertion
\includegraphics[width=8cm]{../Comparisons/ImagesFromVMD/tert-butylamine_out_G09_invertion.png}
\\
\vtab

\columnbreak
Molecule B \\ 
tetramethylsilane\_out\_G09
\includegraphics[width=8cm]{../Comparisons/ImagesFromVMD/tetramethylsilane_out_G09.png}
\\
\vtab


\end{center}
\end{multicols}
\begin{center}
\textcolor{NavyBlue}{\Large Different}
\end{center}

 \newpage

\vtab[-3cm]
\begin{center}
{\large NeoConformation \tab Número 574}
\end{center}
\begin{multicols}{2}
\begin{center}
Molecule A \\ 
tert-butylamine\_out\_G09\_invertion
\includegraphics[width=8cm]{../Comparisons/ImagesFromVMD/tert-butylamine_out_G09_invertion.png}
\\
\vtab

\columnbreak
Molecule B \\ 
tetramethylsilane\_out\_G09\_invertion
\includegraphics[width=8cm]{../Comparisons/ImagesFromVMD/tetramethylsilane_out_G09_invertion.png}
\\
\vtab


\end{center}
\end{multicols}
\begin{center}
\textcolor{NavyBlue}{\Large Different}
\end{center}

 \newpage

\vtab[-3cm]
\begin{center}
{\large NeoConformation \tab Número 575}
\end{center}
\begin{multicols}{2}
\begin{center}
Molecule A \\ 
tert-butylamine\_out\_G09\_rot\_x15\_y15\_z15
\includegraphics[width=8cm]{../Comparisons/ImagesFromVMD/tert-butylamine_out_G09_rot_x15_y15_z15.png}
\\
\vtab

\columnbreak
Molecule B \\ 
tetramethylsilane\_out\_G09
\includegraphics[width=8cm]{../Comparisons/ImagesFromVMD/tetramethylsilane_out_G09.png}
\\
\vtab


\end{center}
\end{multicols}
\begin{center}
\textcolor{NavyBlue}{\Large Different}
\end{center}

 \newpage

\vtab[-3cm]
\begin{center}
{\large NeoConformation \tab Número 576}
\end{center}
\begin{multicols}{2}
\begin{center}
Molecule A \\ 
tert-butylamine\_out\_G09\_rot\_x15\_y15\_z15
\includegraphics[width=8cm]{../Comparisons/ImagesFromVMD/tert-butylamine_out_G09_rot_x15_y15_z15.png}
\\
\vtab

\columnbreak
Molecule B \\ 
tetramethylsilane\_out\_G09\_invertion
\includegraphics[width=8cm]{../Comparisons/ImagesFromVMD/tetramethylsilane_out_G09_invertion.png}
\\
\vtab


\end{center}
\end{multicols}
\begin{center}
\textcolor{NavyBlue}{\Large Different}
\end{center}

 \newpage

\vtab[-3cm]
\begin{center}
{\large NeoConformation \tab Número 577}
\end{center}
\begin{multicols}{2}
\begin{center}

Molecule A \
tetramethylsilane\_out\_G09

\includegraphics[width=6cm]{../Comparisons/ImagesFromVMD/tetramethylsilane_out_G09.png}

Inertia Tensor - Molecule A \\
\begin{tabular}{|c c c|}
166.054	 & 	0	 & 	0	 \\
0	 & 	166.054	 & 	0	 \\
0	 & 	0	 & 	166.054
\end{tabular}

\vtab
 EingenVectors - Molecule A     \\
\begin{tabular}{|c c c|}
1	 & 	0	 & 	0	 \\
0	 & 	1	 & 	0	 \\
0	 & 	0	 & 	1
\end{tabular}

\vtab
 EingenValues - Molecule A     \\
\begin{tabular}{|c c c|}
166.054	 & 	166.054	 & 	166.054	 \\
\end{tabular}
\columnbreak

Molecule B \
tetramethylsilane\_out\_G09\_invertion

\includegraphics[width=6cm]{../Comparisons/ImagesFromVMD/tetramethylsilane_out_G09_invertion.png}

Inertia Tensor - Molecule B \\
\begin{tabular}{|c c c|}
166.053	 & 	0	 & 	0	 \\
0	 & 	166.053	 & 	0	 \\
0	 & 	0	 & 	166.053
\end{tabular}

\vtab
 EingenVectors - Molecule B     \\
\begin{tabular}{|c c c|}
1	 & 	0	 & 	0	 \\
0	 & 	1	 & 	0	 \\
0	 & 	0	 & 	1
\end{tabular}

\vtab
 EingenValues - Molecule B     \\
\begin{tabular}{|c c c|}
166.053	 & 	166.053	 & 	166.053	 \\
\end{tabular}

\end{center}
\end{multicols}

\vtab[-5mm]
\begin{tabular}{*{2}{m{0.38\textwidth}}}
\begin{center}
\textcolor{NavyBlue}{\Large Equal}
\end{center}
&
\begin{center}
\includegraphics[height=6.5cm]{../Comparisons/Vectors/inertia_tensor_of_tetramethylsilane_out_G09_and_tetramethylsilane_out_G09_invertion.png}
\end{center}
\end{tabular}

 \newpage

\vtab[-3cm]
\begin{center}
{\large RotationSameMolecule \tab Número 578}
\end{center}
\begin{multicols}{2}
\begin{center}

Molecule A \
Porphin\_out\_G09

\includegraphics[width=6cm]{../Comparisons/ImagesFromVMD/Porphin_out_G09.png}

Inertia Tensor - Molecule A \\
\begin{tabular}{|c c c|}
1914.06	 & 	0	 & 	0	 \\
0	 & 	1914.06	 & 	0	 \\
0	 & 	0	 & 	3828.12
\end{tabular}

\vtab
 EingenVectors - Molecule A     \\
\begin{tabular}{|c c c|}
1	 & 	0	 & 	0	 \\
0	 & 	1	 & 	0	 \\
0	 & 	0	 & 	1
\end{tabular}

\vtab
 EingenValues - Molecule A     \\
\begin{tabular}{|c c c|}
1914.06	 & 	1914.06	 & 	3828.12	 \\
\end{tabular}
\columnbreak

Molecule B \
Porphin\_out\_G09\_invertion

\includegraphics[width=6cm]{../Comparisons/ImagesFromVMD/Porphin_out_G09_invertion.png}

Inertia Tensor - Molecule B \\
\begin{tabular}{|c c c|}
1914.06	 & 	0	 & 	0	 \\
0	 & 	1914.06	 & 	0	 \\
0	 & 	0	 & 	3828.12
\end{tabular}

\vtab
 EingenVectors - Molecule B     \\
\begin{tabular}{|c c c|}
1	 & 	0	 & 	0	 \\
0	 & 	1	 & 	0	 \\
0	 & 	0	 & 	1
\end{tabular}

\vtab
 EingenValues - Molecule B     \\
\begin{tabular}{|c c c|}
1914.06	 & 	1914.06	 & 	3828.12	 \\
\end{tabular}

\end{center}
\end{multicols}

\vtab[-5mm]
\begin{tabular}{*{2}{m{0.38\textwidth}}}
\begin{center}
\textcolor{NavyBlue}{\Large Equal}
\end{center}
&
\begin{center}
\includegraphics[height=6.5cm]{../Comparisons/Vectors/inertia_tensor_of_Porphin_out_G09_and_Porphin_out_G09_invertion.png}
\end{center}
\end{tabular}

 \newpage

\vtab[-3cm]
\begin{center}
{\large RotationSameMolecule \tab Número 579}
\end{center}
\begin{multicols}{2}
\begin{center}
Molecule A \\ 
Porphin\_out\_G09
\includegraphics[width=8cm]{../Comparisons/ImagesFromVMD/Porphin_out_G09.png}
\\
\vtab

\columnbreak
Molecule B \\ 
Tetrabenzoporphyrin\_out\_G09
\includegraphics[width=8cm]{../Comparisons/ImagesFromVMD/Tetrabenzoporphyrin_out_G09.png}
\\
\vtab


\end{center}
\end{multicols}
\begin{center}
\textcolor{NavyBlue}{\Large Different}
\end{center}

 \newpage

\vtab[-3cm]
\begin{center}
{\large RotationSameMolecule \tab Número 580}
\end{center}
\begin{multicols}{2}
\begin{center}
Molecule A \\ 
Porphin\_out\_G09
\includegraphics[width=8cm]{../Comparisons/ImagesFromVMD/Porphin_out_G09.png}
\\
\vtab

\columnbreak
Molecule B \\ 
Tetrabenzoporphyrin\_out\_G09\_invertion
\includegraphics[width=8cm]{../Comparisons/ImagesFromVMD/Tetrabenzoporphyrin_out_G09_invertion.png}
\\
\vtab


\end{center}
\end{multicols}
\begin{center}
\textcolor{NavyBlue}{\Large Different}
\end{center}

 \newpage

\vtab[-3cm]
\begin{center}
{\large RotationSameMolecule \tab Número 581}
\end{center}
\begin{multicols}{2}
\begin{center}
Molecule A \\ 
Porphin\_out\_G09
\includegraphics[width=8cm]{../Comparisons/ImagesFromVMD/Porphin_out_G09.png}
\\
\vtab

\columnbreak
Molecule B \\ 
Tetrabenzoporphyrin\_rotated02\_out\_G09
\includegraphics[width=8cm]{../Comparisons/ImagesFromVMD/Tetrabenzoporphyrin_rotated02_out_G09.png}
\\
\vtab


\end{center}
\end{multicols}
\begin{center}
\textcolor{NavyBlue}{\Large Different}
\end{center}

 \newpage

\vtab[-3cm]
\begin{center}
{\large RotationSameMolecule \tab Número 582}
\end{center}
\begin{multicols}{2}
\begin{center}
Molecule A \\ 
Porphin\_out\_G09
\includegraphics[width=8cm]{../Comparisons/ImagesFromVMD/Porphin_out_G09.png}
\\
\vtab

\columnbreak
Molecule B \\ 
Tetrabenzoporphyrin\_rotated02\_out\_G09\_invertion
\includegraphics[width=8cm]{../Comparisons/ImagesFromVMD/Tetrabenzoporphyrin_rotated02_out_G09_invertion.png}
\\
\vtab


\end{center}
\end{multicols}
\begin{center}
\textcolor{NavyBlue}{\Large Different}
\end{center}

 \newpage

\vtab[-3cm]
\begin{center}
{\large RotationSameMolecule \tab Número 583}
\end{center}
\begin{multicols}{2}
\begin{center}
Molecule A \\ 
Porphin\_out\_G09
\includegraphics[width=8cm]{../Comparisons/ImagesFromVMD/Porphin_out_G09.png}
\\
\vtab

\columnbreak
Molecule B \\ 
Tetrabenzoporphyrin\_rotated\_out\_G09
\includegraphics[width=8cm]{../Comparisons/ImagesFromVMD/Tetrabenzoporphyrin_rotated_out_G09.png}
\\
\vtab


\end{center}
\end{multicols}
\begin{center}
\textcolor{NavyBlue}{\Large Different}
\end{center}

 \newpage

\vtab[-3cm]
\begin{center}
{\large RotationSameMolecule \tab Número 584}
\end{center}
\begin{multicols}{2}
\begin{center}
Molecule A \\ 
Porphin\_out\_G09
\includegraphics[width=8cm]{../Comparisons/ImagesFromVMD/Porphin_out_G09.png}
\\
\vtab

\columnbreak
Molecule B \\ 
Tetrabenzoporphyrin\_rotated\_out\_G09\_invertion
\includegraphics[width=8cm]{../Comparisons/ImagesFromVMD/Tetrabenzoporphyrin_rotated_out_G09_invertion.png}
\\
\vtab


\end{center}
\end{multicols}
\begin{center}
\textcolor{NavyBlue}{\Large Different}
\end{center}

 \newpage

\vtab[-3cm]
\begin{center}
{\large RotationSameMolecule \tab Número 585}
\end{center}
\begin{multicols}{2}
\begin{center}
Molecule A \\ 
Porphin\_out\_G09\_invertion
\includegraphics[width=8cm]{../Comparisons/ImagesFromVMD/Porphin_out_G09_invertion.png}
\\
\vtab

\columnbreak
Molecule B \\ 
Tetrabenzoporphyrin\_out\_G09
\includegraphics[width=8cm]{../Comparisons/ImagesFromVMD/Tetrabenzoporphyrin_out_G09.png}
\\
\vtab


\end{center}
\end{multicols}
\begin{center}
\textcolor{NavyBlue}{\Large Different}
\end{center}

 \newpage

\vtab[-3cm]
\begin{center}
{\large RotationSameMolecule \tab Número 586}
\end{center}
\begin{multicols}{2}
\begin{center}
Molecule A \\ 
Porphin\_out\_G09\_invertion
\includegraphics[width=8cm]{../Comparisons/ImagesFromVMD/Porphin_out_G09_invertion.png}
\\
\vtab

\columnbreak
Molecule B \\ 
Tetrabenzoporphyrin\_out\_G09\_invertion
\includegraphics[width=8cm]{../Comparisons/ImagesFromVMD/Tetrabenzoporphyrin_out_G09_invertion.png}
\\
\vtab


\end{center}
\end{multicols}
\begin{center}
\textcolor{NavyBlue}{\Large Different}
\end{center}

 \newpage

\vtab[-3cm]
\begin{center}
{\large RotationSameMolecule \tab Número 587}
\end{center}
\begin{multicols}{2}
\begin{center}
Molecule A \\ 
Porphin\_out\_G09\_invertion
\includegraphics[width=8cm]{../Comparisons/ImagesFromVMD/Porphin_out_G09_invertion.png}
\\
\vtab

\columnbreak
Molecule B \\ 
Tetrabenzoporphyrin\_rotated02\_out\_G09
\includegraphics[width=8cm]{../Comparisons/ImagesFromVMD/Tetrabenzoporphyrin_rotated02_out_G09.png}
\\
\vtab


\end{center}
\end{multicols}
\begin{center}
\textcolor{NavyBlue}{\Large Different}
\end{center}

 \newpage

\vtab[-3cm]
\begin{center}
{\large RotationSameMolecule \tab Número 588}
\end{center}
\begin{multicols}{2}
\begin{center}
Molecule A \\ 
Porphin\_out\_G09\_invertion
\includegraphics[width=8cm]{../Comparisons/ImagesFromVMD/Porphin_out_G09_invertion.png}
\\
\vtab

\columnbreak
Molecule B \\ 
Tetrabenzoporphyrin\_rotated02\_out\_G09\_invertion
\includegraphics[width=8cm]{../Comparisons/ImagesFromVMD/Tetrabenzoporphyrin_rotated02_out_G09_invertion.png}
\\
\vtab


\end{center}
\end{multicols}
\begin{center}
\textcolor{NavyBlue}{\Large Different}
\end{center}

 \newpage

\vtab[-3cm]
\begin{center}
{\large RotationSameMolecule \tab Número 589}
\end{center}
\begin{multicols}{2}
\begin{center}
Molecule A \\ 
Porphin\_out\_G09\_invertion
\includegraphics[width=8cm]{../Comparisons/ImagesFromVMD/Porphin_out_G09_invertion.png}
\\
\vtab

\columnbreak
Molecule B \\ 
Tetrabenzoporphyrin\_rotated\_out\_G09
\includegraphics[width=8cm]{../Comparisons/ImagesFromVMD/Tetrabenzoporphyrin_rotated_out_G09.png}
\\
\vtab


\end{center}
\end{multicols}
\begin{center}
\textcolor{NavyBlue}{\Large Different}
\end{center}

 \newpage

\vtab[-3cm]
\begin{center}
{\large RotationSameMolecule \tab Número 590}
\end{center}
\begin{multicols}{2}
\begin{center}
Molecule A \\ 
Porphin\_out\_G09\_invertion
\includegraphics[width=8cm]{../Comparisons/ImagesFromVMD/Porphin_out_G09_invertion.png}
\\
\vtab

\columnbreak
Molecule B \\ 
Tetrabenzoporphyrin\_rotated\_out\_G09\_invertion
\includegraphics[width=8cm]{../Comparisons/ImagesFromVMD/Tetrabenzoporphyrin_rotated_out_G09_invertion.png}
\\
\vtab


\end{center}
\end{multicols}
\begin{center}
\textcolor{NavyBlue}{\Large Different}
\end{center}

 \newpage

\vtab[-3cm]
\begin{center}
{\large RotationSameMolecule \tab Número 591}
\end{center}
\begin{multicols}{2}
\begin{center}

Molecule A \
Tetrabenzoporphyrin\_out\_G09

\includegraphics[width=6cm]{../Comparisons/ImagesFromVMD/Tetrabenzoporphyrin_out_G09.png}

Inertia Tensor - Molecule A \\
\begin{tabular}{|c c c|}
11864.8	 & 	0	 & 	0	 \\
0	 & 	5941.95	 & 	0	 \\
0	 & 	0	 & 	5922.88
\end{tabular}

\vtab
 EingenVectors - Molecule A     \\
\begin{tabular}{|c c c|}
0	 & 	0	 & 	1	 \\
0	 & 	1	 & 	0	 \\
1	 & 	0	 & 	0
\end{tabular}

\vtab
 EingenValues - Molecule A     \\
\begin{tabular}{|c c c|}
5922.88	 & 	5941.95	 & 	11864.8	 \\
\end{tabular}
\columnbreak

Molecule B \
Tetrabenzoporphyrin\_out\_G09\_invertion

\includegraphics[width=6cm]{../Comparisons/ImagesFromVMD/Tetrabenzoporphyrin_out_G09_invertion.png}

Inertia Tensor - Molecule B \\
\begin{tabular}{|c c c|}
11864.8	 & 	0	 & 	0	 \\
0	 & 	5941.95	 & 	0	 \\
0	 & 	0	 & 	5922.88
\end{tabular}

\vtab
 EingenVectors - Molecule B     \\
\begin{tabular}{|c c c|}
0	 & 	0	 & 	1	 \\
0	 & 	1	 & 	0	 \\
1	 & 	0	 & 	0
\end{tabular}

\vtab
 EingenValues - Molecule B     \\
\begin{tabular}{|c c c|}
5922.88	 & 	5941.95	 & 	11864.8	 \\
\end{tabular}

\end{center}
\end{multicols}

\vtab[-5mm]
\begin{tabular}{*{2}{m{0.38\textwidth}}}
\begin{center}
\textcolor{NavyBlue}{\Large Equal}
\end{center}
&
\begin{center}
\includegraphics[height=6.5cm]{../Comparisons/Vectors/inertia_tensor_of_Tetrabenzoporphyrin_out_G09_and_Tetrabenzoporphyrin_out_G09_invertion.png}
\end{center}
\end{tabular}

 \newpage

\vtab[-3cm]
\begin{center}
{\large RotationSameMolecule \tab Número 592}
\end{center}
\begin{multicols}{2}
\begin{center}

Molecule A \
Tetrabenzoporphyrin\_out\_G09

\includegraphics[width=6cm]{../Comparisons/ImagesFromVMD/Tetrabenzoporphyrin_out_G09.png}

Inertia Tensor - Molecule A \\
\begin{tabular}{|c c c|}
11864.8	 & 	0	 & 	0	 \\
0	 & 	5941.95	 & 	0	 \\
0	 & 	0	 & 	5922.88
\end{tabular}

\vtab
 EingenVectors - Molecule A     \\
\begin{tabular}{|c c c|}
0	 & 	0	 & 	1	 \\
0	 & 	1	 & 	0	 \\
1	 & 	0	 & 	0
\end{tabular}

\vtab
 EingenValues - Molecule A     \\
\begin{tabular}{|c c c|}
5922.88	 & 	5941.95	 & 	11864.8	 \\
\end{tabular}
\columnbreak

Molecule B \
Tetrabenzoporphyrin\_rotated02\_out\_G09

\includegraphics[width=6cm]{../Comparisons/ImagesFromVMD/Tetrabenzoporphyrin_rotated02_out_G09.png}

Inertia Tensor - Molecule B \\
\begin{tabular}{|c c c|}
7791.57	 & 	2641.52	 & 	-788.168	 \\
2641.52	 & 	9673.67	 & 	-1103.82	 \\
-788.168	 & 	-1103.82	 & 	6265.27
\end{tabular}

\vtab
 EingenVectors - Molecule B     \\
\begin{tabular}{|c c c|}
-0.771476	 & 	0.397136	 & 	-0.497099	 \\
0.301236	 & 	-0.460189	 & 	-0.835154	 \\
-0.56043	 & 	-0.794046	 & 	0.235394
\end{tabular}

\vtab
 EingenValues - Molecule B     \\
\begin{tabular}{|c c c|}
5923.93	 & 	5941.33	 & 	11865.3	 \\
\end{tabular}

\end{center}
\end{multicols}

\vtab[-5mm]
\begin{tabular}{*{2}{m{0.38\textwidth}}}
\begin{center}
\textcolor{NavyBlue}{\Large Enantiomers}
\end{center}
&
\begin{center}
\includegraphics[height=6.5cm]{../Comparisons/Vectors/inertia_tensor_of_Tetrabenzoporphyrin_out_G09_and_Tetrabenzoporphyrin_rotated02_out_G09.png}
\end{center}
\end{tabular}

 \newpage

\vtab[-3cm]
\begin{center}
{\large RotationSameMolecule \tab Número 593}
\end{center}
\begin{multicols}{2}
\begin{center}

Molecule A \
Tetrabenzoporphyrin\_out\_G09

\includegraphics[width=6cm]{../Comparisons/ImagesFromVMD/Tetrabenzoporphyrin_out_G09.png}

Inertia Tensor - Molecule A \\
\begin{tabular}{|c c c|}
11864.8	 & 	0	 & 	0	 \\
0	 & 	5941.95	 & 	0	 \\
0	 & 	0	 & 	5922.88
\end{tabular}

\vtab
 EingenVectors - Molecule A     \\
\begin{tabular}{|c c c|}
0	 & 	0	 & 	1	 \\
0	 & 	1	 & 	0	 \\
1	 & 	0	 & 	0
\end{tabular}

\vtab
 EingenValues - Molecule A     \\
\begin{tabular}{|c c c|}
5922.88	 & 	5941.95	 & 	11864.8	 \\
\end{tabular}
\columnbreak

Molecule B \
Tetrabenzoporphyrin\_rotated02\_out\_G09\_invertion

\includegraphics[width=6cm]{../Comparisons/ImagesFromVMD/Tetrabenzoporphyrin_rotated02_out_G09_invertion.png}

Inertia Tensor - Molecule B \\
\begin{tabular}{|c c c|}
7791.56	 & 	2641.52	 & 	-788.166	 \\
2641.52	 & 	9673.67	 & 	-1103.82	 \\
-788.166	 & 	-1103.82	 & 	6265.27
\end{tabular}

\vtab
 EingenVectors - Molecule B     \\
\begin{tabular}{|c c c|}
-0.771478	 & 	0.39714	 & 	-0.497093	 \\
0.30123	 & 	-0.460186	 & 	-0.835158	 \\
-0.56043	 & 	-0.794046	 & 	0.235393
\end{tabular}

\vtab
 EingenValues - Molecule B     \\
\begin{tabular}{|c c c|}
5923.93	 & 	5941.33	 & 	11865.3	 \\
\end{tabular}

\end{center}
\end{multicols}

\vtab[-5mm]
\begin{tabular}{*{2}{m{0.38\textwidth}}}
\begin{center}
\textcolor{NavyBlue}{\Large Enantiomers}
\end{center}
&
\begin{center}
\includegraphics[height=6.5cm]{../Comparisons/Vectors/inertia_tensor_of_Tetrabenzoporphyrin_out_G09_and_Tetrabenzoporphyrin_rotated02_out_G09_invertion.png}
\end{center}
\end{tabular}

 \newpage

\vtab[-3cm]
\begin{center}
{\large RotationSameMolecule \tab Número 594}
\end{center}
\begin{multicols}{2}
\begin{center}

Molecule A \
Tetrabenzoporphyrin\_out\_G09

\includegraphics[width=6cm]{../Comparisons/ImagesFromVMD/Tetrabenzoporphyrin_out_G09.png}

Inertia Tensor - Molecule A \\
\begin{tabular}{|c c c|}
11864.8	 & 	0	 & 	0	 \\
0	 & 	5941.95	 & 	0	 \\
0	 & 	0	 & 	5922.88
\end{tabular}

\vtab
 EingenVectors - Molecule A     \\
\begin{tabular}{|c c c|}
0	 & 	0	 & 	1	 \\
0	 & 	1	 & 	0	 \\
1	 & 	0	 & 	0
\end{tabular}

\vtab
 EingenValues - Molecule A     \\
\begin{tabular}{|c c c|}
5922.88	 & 	5941.95	 & 	11864.8	 \\
\end{tabular}
\columnbreak

Molecule B \
Tetrabenzoporphyrin\_rotated\_out\_G09

\includegraphics[width=6cm]{../Comparisons/ImagesFromVMD/Tetrabenzoporphyrin_rotated_out_G09.png}

Inertia Tensor - Molecule B \\
\begin{tabular}{|c c c|}
11864.8	 & 	0	 & 	0	 \\
0	 & 	5941.95	 & 	0	 \\
0	 & 	0	 & 	5922.88
\end{tabular}

\vtab
 EingenVectors - Molecule B     \\
\begin{tabular}{|c c c|}
0	 & 	0	 & 	1	 \\
0	 & 	1	 & 	0	 \\
1	 & 	0	 & 	0
\end{tabular}

\vtab
 EingenValues - Molecule B     \\
\begin{tabular}{|c c c|}
5922.88	 & 	5941.95	 & 	11864.8	 \\
\end{tabular}

\end{center}
\end{multicols}

\vtab[-5mm]
\begin{tabular}{*{2}{m{0.38\textwidth}}}
\begin{center}
\textcolor{NavyBlue}{\Large Equal}
\end{center}
&
\begin{center}
\includegraphics[height=6.5cm]{../Comparisons/Vectors/inertia_tensor_of_Tetrabenzoporphyrin_out_G09_and_Tetrabenzoporphyrin_rotated_out_G09.png}
\end{center}
\end{tabular}

 \newpage

\vtab[-3cm]
\begin{center}
{\large RotationSameMolecule \tab Número 595}
\end{center}
\begin{multicols}{2}
\begin{center}

Molecule A \
Tetrabenzoporphyrin\_out\_G09

\includegraphics[width=6cm]{../Comparisons/ImagesFromVMD/Tetrabenzoporphyrin_out_G09.png}

Inertia Tensor - Molecule A \\
\begin{tabular}{|c c c|}
11864.8	 & 	0	 & 	0	 \\
0	 & 	5941.95	 & 	0	 \\
0	 & 	0	 & 	5922.88
\end{tabular}

\vtab
 EingenVectors - Molecule A     \\
\begin{tabular}{|c c c|}
0	 & 	0	 & 	1	 \\
0	 & 	1	 & 	0	 \\
1	 & 	0	 & 	0
\end{tabular}

\vtab
 EingenValues - Molecule A     \\
\begin{tabular}{|c c c|}
5922.88	 & 	5941.95	 & 	11864.8	 \\
\end{tabular}
\columnbreak

Molecule B \
Tetrabenzoporphyrin\_rotated\_out\_G09\_invertion

\includegraphics[width=6cm]{../Comparisons/ImagesFromVMD/Tetrabenzoporphyrin_rotated_out_G09_invertion.png}

Inertia Tensor - Molecule B \\
\begin{tabular}{|c c c|}
11864.8	 & 	0	 & 	0	 \\
0	 & 	5941.95	 & 	0	 \\
0	 & 	0	 & 	5922.88
\end{tabular}

\vtab
 EingenVectors - Molecule B     \\
\begin{tabular}{|c c c|}
0	 & 	0	 & 	1	 \\
0	 & 	1	 & 	0	 \\
1	 & 	0	 & 	0
\end{tabular}

\vtab
 EingenValues - Molecule B     \\
\begin{tabular}{|c c c|}
5922.88	 & 	5941.95	 & 	11864.8	 \\
\end{tabular}

\end{center}
\end{multicols}

\vtab[-5mm]
\begin{tabular}{*{2}{m{0.38\textwidth}}}
\begin{center}
\textcolor{NavyBlue}{\Large Equal}
\end{center}
&
\begin{center}
\includegraphics[height=6.5cm]{../Comparisons/Vectors/inertia_tensor_of_Tetrabenzoporphyrin_out_G09_and_Tetrabenzoporphyrin_rotated_out_G09_invertion.png}
\end{center}
\end{tabular}

 \newpage

\vtab[-3cm]
\begin{center}
{\large RotationSameMolecule \tab Número 596}
\end{center}
\begin{multicols}{2}
\begin{center}

Molecule A \
Tetrabenzoporphyrin\_out\_G09\_invertion

\includegraphics[width=6cm]{../Comparisons/ImagesFromVMD/Tetrabenzoporphyrin_out_G09_invertion.png}

Inertia Tensor - Molecule A \\
\begin{tabular}{|c c c|}
11864.8	 & 	0	 & 	0	 \\
0	 & 	5941.95	 & 	0	 \\
0	 & 	0	 & 	5922.88
\end{tabular}

\vtab
 EingenVectors - Molecule A     \\
\begin{tabular}{|c c c|}
0	 & 	0	 & 	1	 \\
0	 & 	1	 & 	0	 \\
1	 & 	0	 & 	0
\end{tabular}

\vtab
 EingenValues - Molecule A     \\
\begin{tabular}{|c c c|}
5922.88	 & 	5941.95	 & 	11864.8	 \\
\end{tabular}
\columnbreak

Molecule B \
Tetrabenzoporphyrin\_rotated02\_out\_G09

\includegraphics[width=6cm]{../Comparisons/ImagesFromVMD/Tetrabenzoporphyrin_rotated02_out_G09.png}

Inertia Tensor - Molecule B \\
\begin{tabular}{|c c c|}
7791.57	 & 	2641.52	 & 	-788.168	 \\
2641.52	 & 	9673.67	 & 	-1103.82	 \\
-788.168	 & 	-1103.82	 & 	6265.27
\end{tabular}

\vtab
 EingenVectors - Molecule B     \\
\begin{tabular}{|c c c|}
-0.771476	 & 	0.397136	 & 	-0.497099	 \\
0.301236	 & 	-0.460189	 & 	-0.835154	 \\
-0.56043	 & 	-0.794046	 & 	0.235394
\end{tabular}

\vtab
 EingenValues - Molecule B     \\
\begin{tabular}{|c c c|}
5923.93	 & 	5941.33	 & 	11865.3	 \\
\end{tabular}

\end{center}
\end{multicols}

\vtab[-5mm]
\begin{tabular}{*{2}{m{0.38\textwidth}}}
\begin{center}
\textcolor{NavyBlue}{\Large Enantiomers}
\end{center}
&
\begin{center}
\includegraphics[height=6.5cm]{../Comparisons/Vectors/inertia_tensor_of_Tetrabenzoporphyrin_out_G09_invertion_and_Tetrabenzoporphyrin_rotated02_out_G09.png}
\end{center}
\end{tabular}

 \newpage

\vtab[-3cm]
\begin{center}
{\large RotationSameMolecule \tab Número 597}
\end{center}
\begin{multicols}{2}
\begin{center}

Molecule A \
Tetrabenzoporphyrin\_out\_G09\_invertion

\includegraphics[width=6cm]{../Comparisons/ImagesFromVMD/Tetrabenzoporphyrin_out_G09_invertion.png}

Inertia Tensor - Molecule A \\
\begin{tabular}{|c c c|}
11864.8	 & 	0	 & 	0	 \\
0	 & 	5941.95	 & 	0	 \\
0	 & 	0	 & 	5922.88
\end{tabular}

\vtab
 EingenVectors - Molecule A     \\
\begin{tabular}{|c c c|}
0	 & 	0	 & 	1	 \\
0	 & 	1	 & 	0	 \\
1	 & 	0	 & 	0
\end{tabular}

\vtab
 EingenValues - Molecule A     \\
\begin{tabular}{|c c c|}
5922.88	 & 	5941.95	 & 	11864.8	 \\
\end{tabular}
\columnbreak

Molecule B \
Tetrabenzoporphyrin\_rotated02\_out\_G09\_invertion

\includegraphics[width=6cm]{../Comparisons/ImagesFromVMD/Tetrabenzoporphyrin_rotated02_out_G09_invertion.png}

Inertia Tensor - Molecule B \\
\begin{tabular}{|c c c|}
7791.56	 & 	2641.52	 & 	-788.166	 \\
2641.52	 & 	9673.67	 & 	-1103.82	 \\
-788.166	 & 	-1103.82	 & 	6265.27
\end{tabular}

\vtab
 EingenVectors - Molecule B     \\
\begin{tabular}{|c c c|}
-0.771478	 & 	0.39714	 & 	-0.497093	 \\
0.30123	 & 	-0.460186	 & 	-0.835158	 \\
-0.56043	 & 	-0.794046	 & 	0.235393
\end{tabular}

\vtab
 EingenValues - Molecule B     \\
\begin{tabular}{|c c c|}
5923.93	 & 	5941.33	 & 	11865.3	 \\
\end{tabular}

\end{center}
\end{multicols}

\vtab[-5mm]
\begin{tabular}{*{2}{m{0.38\textwidth}}}
\begin{center}
\textcolor{NavyBlue}{\Large Enantiomers}
\end{center}
&
\begin{center}
\includegraphics[height=6.5cm]{../Comparisons/Vectors/inertia_tensor_of_Tetrabenzoporphyrin_out_G09_invertion_and_Tetrabenzoporphyrin_rotated02_out_G09_invertion.png}
\end{center}
\end{tabular}

 \newpage

\vtab[-3cm]
\begin{center}
{\large RotationSameMolecule \tab Número 598}
\end{center}
\begin{multicols}{2}
\begin{center}

Molecule A \
Tetrabenzoporphyrin\_out\_G09\_invertion

\includegraphics[width=6cm]{../Comparisons/ImagesFromVMD/Tetrabenzoporphyrin_out_G09_invertion.png}

Inertia Tensor - Molecule A \\
\begin{tabular}{|c c c|}
11864.8	 & 	0	 & 	0	 \\
0	 & 	5941.95	 & 	0	 \\
0	 & 	0	 & 	5922.88
\end{tabular}

\vtab
 EingenVectors - Molecule A     \\
\begin{tabular}{|c c c|}
0	 & 	0	 & 	1	 \\
0	 & 	1	 & 	0	 \\
1	 & 	0	 & 	0
\end{tabular}

\vtab
 EingenValues - Molecule A     \\
\begin{tabular}{|c c c|}
5922.88	 & 	5941.95	 & 	11864.8	 \\
\end{tabular}
\columnbreak

Molecule B \
Tetrabenzoporphyrin\_rotated\_out\_G09

\includegraphics[width=6cm]{../Comparisons/ImagesFromVMD/Tetrabenzoporphyrin_rotated_out_G09.png}

Inertia Tensor - Molecule B \\
\begin{tabular}{|c c c|}
11864.8	 & 	0	 & 	0	 \\
0	 & 	5941.95	 & 	0	 \\
0	 & 	0	 & 	5922.88
\end{tabular}

\vtab
 EingenVectors - Molecule B     \\
\begin{tabular}{|c c c|}
0	 & 	0	 & 	1	 \\
0	 & 	1	 & 	0	 \\
1	 & 	0	 & 	0
\end{tabular}

\vtab
 EingenValues - Molecule B     \\
\begin{tabular}{|c c c|}
5922.88	 & 	5941.95	 & 	11864.8	 \\
\end{tabular}

\end{center}
\end{multicols}

\vtab[-5mm]
\begin{tabular}{*{2}{m{0.38\textwidth}}}
\begin{center}
\textcolor{NavyBlue}{\Large Equal}
\end{center}
&
\begin{center}
\includegraphics[height=6.5cm]{../Comparisons/Vectors/inertia_tensor_of_Tetrabenzoporphyrin_out_G09_invertion_and_Tetrabenzoporphyrin_rotated_out_G09.png}
\end{center}
\end{tabular}

 \newpage

\vtab[-3cm]
\begin{center}
{\large RotationSameMolecule \tab Número 599}
\end{center}
\begin{multicols}{2}
\begin{center}

Molecule A \
Tetrabenzoporphyrin\_out\_G09\_invertion

\includegraphics[width=6cm]{../Comparisons/ImagesFromVMD/Tetrabenzoporphyrin_out_G09_invertion.png}

Inertia Tensor - Molecule A \\
\begin{tabular}{|c c c|}
11864.8	 & 	0	 & 	0	 \\
0	 & 	5941.95	 & 	0	 \\
0	 & 	0	 & 	5922.88
\end{tabular}

\vtab
 EingenVectors - Molecule A     \\
\begin{tabular}{|c c c|}
0	 & 	0	 & 	1	 \\
0	 & 	1	 & 	0	 \\
1	 & 	0	 & 	0
\end{tabular}

\vtab
 EingenValues - Molecule A     \\
\begin{tabular}{|c c c|}
5922.88	 & 	5941.95	 & 	11864.8	 \\
\end{tabular}
\columnbreak

Molecule B \
Tetrabenzoporphyrin\_rotated\_out\_G09\_invertion

\includegraphics[width=6cm]{../Comparisons/ImagesFromVMD/Tetrabenzoporphyrin_rotated_out_G09_invertion.png}

Inertia Tensor - Molecule B \\
\begin{tabular}{|c c c|}
11864.8	 & 	0	 & 	0	 \\
0	 & 	5941.95	 & 	0	 \\
0	 & 	0	 & 	5922.88
\end{tabular}

\vtab
 EingenVectors - Molecule B     \\
\begin{tabular}{|c c c|}
0	 & 	0	 & 	1	 \\
0	 & 	1	 & 	0	 \\
1	 & 	0	 & 	0
\end{tabular}

\vtab
 EingenValues - Molecule B     \\
\begin{tabular}{|c c c|}
5922.88	 & 	5941.95	 & 	11864.8	 \\
\end{tabular}

\end{center}
\end{multicols}

\vtab[-5mm]
\begin{tabular}{*{2}{m{0.38\textwidth}}}
\begin{center}
\textcolor{NavyBlue}{\Large Equal}
\end{center}
&
\begin{center}
\includegraphics[height=6.5cm]{../Comparisons/Vectors/inertia_tensor_of_Tetrabenzoporphyrin_out_G09_invertion_and_Tetrabenzoporphyrin_rotated_out_G09_invertion.png}
\end{center}
\end{tabular}

 \newpage

\vtab[-3cm]
\begin{center}
{\large RotationSameMolecule \tab Número 600}
\end{center}
\begin{multicols}{2}
\begin{center}

Molecule A \
Tetrabenzoporphyrin\_rotated02\_out\_G09

\includegraphics[width=6cm]{../Comparisons/ImagesFromVMD/Tetrabenzoporphyrin_rotated02_out_G09.png}

Inertia Tensor - Molecule A \\
\begin{tabular}{|c c c|}
7791.57	 & 	2641.52	 & 	-788.168	 \\
2641.52	 & 	9673.67	 & 	-1103.82	 \\
-788.168	 & 	-1103.82	 & 	6265.27
\end{tabular}

\vtab
 EingenVectors - Molecule A     \\
\begin{tabular}{|c c c|}
-0.771476	 & 	0.397136	 & 	-0.497099	 \\
0.301236	 & 	-0.460189	 & 	-0.835154	 \\
-0.56043	 & 	-0.794046	 & 	0.235394
\end{tabular}

\vtab
 EingenValues - Molecule A     \\
\begin{tabular}{|c c c|}
5923.93	 & 	5941.33	 & 	11865.3	 \\
\end{tabular}
\columnbreak

Molecule B \
Tetrabenzoporphyrin\_rotated02\_out\_G09\_invertion

\includegraphics[width=6cm]{../Comparisons/ImagesFromVMD/Tetrabenzoporphyrin_rotated02_out_G09_invertion.png}

Inertia Tensor - Molecule B \\
\begin{tabular}{|c c c|}
7791.56	 & 	2641.52	 & 	-788.166	 \\
2641.52	 & 	9673.67	 & 	-1103.82	 \\
-788.166	 & 	-1103.82	 & 	6265.27
\end{tabular}

\vtab
 EingenVectors - Molecule B     \\
\begin{tabular}{|c c c|}
-0.771478	 & 	0.39714	 & 	-0.497093	 \\
0.30123	 & 	-0.460186	 & 	-0.835158	 \\
-0.56043	 & 	-0.794046	 & 	0.235393
\end{tabular}

\vtab
 EingenValues - Molecule B     \\
\begin{tabular}{|c c c|}
5923.93	 & 	5941.33	 & 	11865.3	 \\
\end{tabular}

\end{center}
\end{multicols}

\vtab[-5mm]
\begin{tabular}{*{2}{m{0.38\textwidth}}}
\begin{center}
\textcolor{NavyBlue}{\Large Equal}
\end{center}
&
\begin{center}
\includegraphics[height=6.5cm]{../Comparisons/Vectors/inertia_tensor_of_Tetrabenzoporphyrin_rotated02_out_G09_and_Tetrabenzoporphyrin_rotated02_out_G09_invertion.png}
\end{center}
\end{tabular}

 \newpage

\vtab[-3cm]
\begin{center}
{\large RotationSameMolecule \tab Número 601}
\end{center}
\begin{multicols}{2}
\begin{center}

Molecule A \
Tetrabenzoporphyrin\_rotated02\_out\_G09

\includegraphics[width=6cm]{../Comparisons/ImagesFromVMD/Tetrabenzoporphyrin_rotated02_out_G09.png}

Inertia Tensor - Molecule A \\
\begin{tabular}{|c c c|}
7791.57	 & 	2641.52	 & 	-788.168	 \\
2641.52	 & 	9673.67	 & 	-1103.82	 \\
-788.168	 & 	-1103.82	 & 	6265.27
\end{tabular}

\vtab
 EingenVectors - Molecule A     \\
\begin{tabular}{|c c c|}
-0.771476	 & 	0.397136	 & 	-0.497099	 \\
0.301236	 & 	-0.460189	 & 	-0.835154	 \\
-0.56043	 & 	-0.794046	 & 	0.235394
\end{tabular}

\vtab
 EingenValues - Molecule A     \\
\begin{tabular}{|c c c|}
5923.93	 & 	5941.33	 & 	11865.3	 \\
\end{tabular}
\columnbreak

Molecule B \
Tetrabenzoporphyrin\_rotated\_out\_G09

\includegraphics[width=6cm]{../Comparisons/ImagesFromVMD/Tetrabenzoporphyrin_rotated_out_G09.png}

Inertia Tensor - Molecule B \\
\begin{tabular}{|c c c|}
11864.8	 & 	0	 & 	0	 \\
0	 & 	5941.95	 & 	0	 \\
0	 & 	0	 & 	5922.88
\end{tabular}

\vtab
 EingenVectors - Molecule B     \\
\begin{tabular}{|c c c|}
0	 & 	0	 & 	1	 \\
0	 & 	1	 & 	0	 \\
1	 & 	0	 & 	0
\end{tabular}

\vtab
 EingenValues - Molecule B     \\
\begin{tabular}{|c c c|}
5922.88	 & 	5941.95	 & 	11864.8	 \\
\end{tabular}

\end{center}
\end{multicols}

\vtab[-5mm]
\begin{tabular}{*{2}{m{0.38\textwidth}}}
\begin{center}
\textcolor{NavyBlue}{\Large Enantiomers}
\end{center}
&
\begin{center}
\includegraphics[height=6.5cm]{../Comparisons/Vectors/inertia_tensor_of_Tetrabenzoporphyrin_rotated02_out_G09_and_Tetrabenzoporphyrin_rotated_out_G09.png}
\end{center}
\end{tabular}

 \newpage

\vtab[-3cm]
\begin{center}
{\large RotationSameMolecule \tab Número 602}
\end{center}
\begin{multicols}{2}
\begin{center}

Molecule A \
Tetrabenzoporphyrin\_rotated02\_out\_G09

\includegraphics[width=6cm]{../Comparisons/ImagesFromVMD/Tetrabenzoporphyrin_rotated02_out_G09.png}

Inertia Tensor - Molecule A \\
\begin{tabular}{|c c c|}
7791.57	 & 	2641.52	 & 	-788.168	 \\
2641.52	 & 	9673.67	 & 	-1103.82	 \\
-788.168	 & 	-1103.82	 & 	6265.27
\end{tabular}

\vtab
 EingenVectors - Molecule A     \\
\begin{tabular}{|c c c|}
-0.771476	 & 	0.397136	 & 	-0.497099	 \\
0.301236	 & 	-0.460189	 & 	-0.835154	 \\
-0.56043	 & 	-0.794046	 & 	0.235394
\end{tabular}

\vtab
 EingenValues - Molecule A     \\
\begin{tabular}{|c c c|}
5923.93	 & 	5941.33	 & 	11865.3	 \\
\end{tabular}
\columnbreak

Molecule B \
Tetrabenzoporphyrin\_rotated\_out\_G09\_invertion

\includegraphics[width=6cm]{../Comparisons/ImagesFromVMD/Tetrabenzoporphyrin_rotated_out_G09_invertion.png}

Inertia Tensor - Molecule B \\
\begin{tabular}{|c c c|}
11864.8	 & 	0	 & 	0	 \\
0	 & 	5941.95	 & 	0	 \\
0	 & 	0	 & 	5922.88
\end{tabular}

\vtab
 EingenVectors - Molecule B     \\
\begin{tabular}{|c c c|}
0	 & 	0	 & 	1	 \\
0	 & 	1	 & 	0	 \\
1	 & 	0	 & 	0
\end{tabular}

\vtab
 EingenValues - Molecule B     \\
\begin{tabular}{|c c c|}
5922.88	 & 	5941.95	 & 	11864.8	 \\
\end{tabular}

\end{center}
\end{multicols}

\vtab[-5mm]
\begin{tabular}{*{2}{m{0.38\textwidth}}}
\begin{center}
\textcolor{NavyBlue}{\Large Enantiomers}
\end{center}
&
\begin{center}
\includegraphics[height=6.5cm]{../Comparisons/Vectors/inertia_tensor_of_Tetrabenzoporphyrin_rotated02_out_G09_and_Tetrabenzoporphyrin_rotated_out_G09_invertion.png}
\end{center}
\end{tabular}

 \newpage

\vtab[-3cm]
\begin{center}
{\large RotationSameMolecule \tab Número 603}
\end{center}
\begin{multicols}{2}
\begin{center}

Molecule A \
Tetrabenzoporphyrin\_rotated02\_out\_G09\_invertion

\includegraphics[width=6cm]{../Comparisons/ImagesFromVMD/Tetrabenzoporphyrin_rotated02_out_G09_invertion.png}

Inertia Tensor - Molecule A \\
\begin{tabular}{|c c c|}
7791.56	 & 	2641.52	 & 	-788.166	 \\
2641.52	 & 	9673.67	 & 	-1103.82	 \\
-788.166	 & 	-1103.82	 & 	6265.27
\end{tabular}

\vtab
 EingenVectors - Molecule A     \\
\begin{tabular}{|c c c|}
-0.771478	 & 	0.39714	 & 	-0.497093	 \\
0.30123	 & 	-0.460186	 & 	-0.835158	 \\
-0.56043	 & 	-0.794046	 & 	0.235393
\end{tabular}

\vtab
 EingenValues - Molecule A     \\
\begin{tabular}{|c c c|}
5923.93	 & 	5941.33	 & 	11865.3	 \\
\end{tabular}
\columnbreak

Molecule B \
Tetrabenzoporphyrin\_rotated\_out\_G09

\includegraphics[width=6cm]{../Comparisons/ImagesFromVMD/Tetrabenzoporphyrin_rotated_out_G09.png}

Inertia Tensor - Molecule B \\
\begin{tabular}{|c c c|}
11864.8	 & 	0	 & 	0	 \\
0	 & 	5941.95	 & 	0	 \\
0	 & 	0	 & 	5922.88
\end{tabular}

\vtab
 EingenVectors - Molecule B     \\
\begin{tabular}{|c c c|}
0	 & 	0	 & 	1	 \\
0	 & 	1	 & 	0	 \\
1	 & 	0	 & 	0
\end{tabular}

\vtab
 EingenValues - Molecule B     \\
\begin{tabular}{|c c c|}
5922.88	 & 	5941.95	 & 	11864.8	 \\
\end{tabular}

\end{center}
\end{multicols}

\vtab[-5mm]
\begin{tabular}{*{2}{m{0.38\textwidth}}}
\begin{center}
\textcolor{NavyBlue}{\Large Enantiomers}
\end{center}
&
\begin{center}
\includegraphics[height=6.5cm]{../Comparisons/Vectors/inertia_tensor_of_Tetrabenzoporphyrin_rotated02_out_G09_invertion_and_Tetrabenzoporphyrin_rotated_out_G09.png}
\end{center}
\end{tabular}

 \newpage

\vtab[-3cm]
\begin{center}
{\large RotationSameMolecule \tab Número 604}
\end{center}
\begin{multicols}{2}
\begin{center}

Molecule A \
Tetrabenzoporphyrin\_rotated02\_out\_G09\_invertion

\includegraphics[width=6cm]{../Comparisons/ImagesFromVMD/Tetrabenzoporphyrin_rotated02_out_G09_invertion.png}

Inertia Tensor - Molecule A \\
\begin{tabular}{|c c c|}
7791.56	 & 	2641.52	 & 	-788.166	 \\
2641.52	 & 	9673.67	 & 	-1103.82	 \\
-788.166	 & 	-1103.82	 & 	6265.27
\end{tabular}

\vtab
 EingenVectors - Molecule A     \\
\begin{tabular}{|c c c|}
-0.771478	 & 	0.39714	 & 	-0.497093	 \\
0.30123	 & 	-0.460186	 & 	-0.835158	 \\
-0.56043	 & 	-0.794046	 & 	0.235393
\end{tabular}

\vtab
 EingenValues - Molecule A     \\
\begin{tabular}{|c c c|}
5923.93	 & 	5941.33	 & 	11865.3	 \\
\end{tabular}
\columnbreak

Molecule B \
Tetrabenzoporphyrin\_rotated\_out\_G09\_invertion

\includegraphics[width=6cm]{../Comparisons/ImagesFromVMD/Tetrabenzoporphyrin_rotated_out_G09_invertion.png}

Inertia Tensor - Molecule B \\
\begin{tabular}{|c c c|}
11864.8	 & 	0	 & 	0	 \\
0	 & 	5941.95	 & 	0	 \\
0	 & 	0	 & 	5922.88
\end{tabular}

\vtab
 EingenVectors - Molecule B     \\
\begin{tabular}{|c c c|}
0	 & 	0	 & 	1	 \\
0	 & 	1	 & 	0	 \\
1	 & 	0	 & 	0
\end{tabular}

\vtab
 EingenValues - Molecule B     \\
\begin{tabular}{|c c c|}
5922.88	 & 	5941.95	 & 	11864.8	 \\
\end{tabular}

\end{center}
\end{multicols}

\vtab[-5mm]
\begin{tabular}{*{2}{m{0.38\textwidth}}}
\begin{center}
\textcolor{NavyBlue}{\Large Enantiomers}
\end{center}
&
\begin{center}
\includegraphics[height=6.5cm]{../Comparisons/Vectors/inertia_tensor_of_Tetrabenzoporphyrin_rotated02_out_G09_invertion_and_Tetrabenzoporphyrin_rotated_out_G09_invertion.png}
\end{center}
\end{tabular}

 \newpage

\vtab[-3cm]
\begin{center}
{\large RotationSameMolecule \tab Número 605}
\end{center}
\begin{multicols}{2}
\begin{center}

Molecule A \
Tetrabenzoporphyrin\_rotated\_out\_G09

\includegraphics[width=6cm]{../Comparisons/ImagesFromVMD/Tetrabenzoporphyrin_rotated_out_G09.png}

Inertia Tensor - Molecule A \\
\begin{tabular}{|c c c|}
11864.8	 & 	0	 & 	0	 \\
0	 & 	5941.95	 & 	0	 \\
0	 & 	0	 & 	5922.88
\end{tabular}

\vtab
 EingenVectors - Molecule A     \\
\begin{tabular}{|c c c|}
0	 & 	0	 & 	1	 \\
0	 & 	1	 & 	0	 \\
1	 & 	0	 & 	0
\end{tabular}

\vtab
 EingenValues - Molecule A     \\
\begin{tabular}{|c c c|}
5922.88	 & 	5941.95	 & 	11864.8	 \\
\end{tabular}
\columnbreak

Molecule B \
Tetrabenzoporphyrin\_rotated\_out\_G09\_invertion

\includegraphics[width=6cm]{../Comparisons/ImagesFromVMD/Tetrabenzoporphyrin_rotated_out_G09_invertion.png}

Inertia Tensor - Molecule B \\
\begin{tabular}{|c c c|}
11864.8	 & 	0	 & 	0	 \\
0	 & 	5941.95	 & 	0	 \\
0	 & 	0	 & 	5922.88
\end{tabular}

\vtab
 EingenVectors - Molecule B     \\
\begin{tabular}{|c c c|}
0	 & 	0	 & 	1	 \\
0	 & 	1	 & 	0	 \\
1	 & 	0	 & 	0
\end{tabular}

\vtab
 EingenValues - Molecule B     \\
\begin{tabular}{|c c c|}
5922.88	 & 	5941.95	 & 	11864.8	 \\
\end{tabular}

\end{center}
\end{multicols}

\vtab[-5mm]
\begin{tabular}{*{2}{m{0.38\textwidth}}}
\begin{center}
\textcolor{NavyBlue}{\Large Equal}
\end{center}
&
\begin{center}
\includegraphics[height=6.5cm]{../Comparisons/Vectors/inertia_tensor_of_Tetrabenzoporphyrin_rotated_out_G09_and_Tetrabenzoporphyrin_rotated_out_G09_invertion.png}
\end{center}
\end{tabular}

 \newpage

\vtab[-3cm]
\begin{center}
{\large Sugars \tab Número 606}
\end{center}
\begin{multicols}{2}
\begin{center}

Molecule A \
R-R-R-Fructuose\_out\_G09

\includegraphics[width=6cm]{../Comparisons/ImagesFromVMD/R-R-R-Fructuose_out_G09.png}

Inertia Tensor - Molecule A \\
\begin{tabular}{|c c c|}
270.201	 & 	9.83383	 & 	1.05361	 \\
9.83383	 & 	1104.24	 & 	-0.466663	 \\
1.05361	 & 	-0.466663	 & 	1244.5
\end{tabular}

\vtab
 EingenVectors - Molecule A     \\
\begin{tabular}{|c c c|}
0.99993	 & 	-0.0117888	 & 	-0.00108684	 \\
-0.0117923	 & 	-0.999925	 & 	-0.00324079	 \\
0.00104856	 & 	-0.00325338	 & 	0.999994
\end{tabular}

\vtab
 EingenValues - Molecule A     \\
\begin{tabular}{|c c c|}
270.084	 & 	1104.35	 & 	1244.51	 \\
\end{tabular}
\columnbreak

Molecule B \
R-R-S-Fructuose\_out\_G09

\includegraphics[width=6cm]{../Comparisons/ImagesFromVMD/R-R-S-Fructuose_out_G09.png}

Inertia Tensor - Molecule B \\
\begin{tabular}{|c c c|}
258.009	 & 	6.71977	 & 	-10.1101	 \\
6.71977	 & 	1154.74	 & 	-1.24568	 \\
-10.1101	 & 	-1.24568	 & 	1276.63
\end{tabular}

\vtab
 EingenVectors - Molecule B     \\
\begin{tabular}{|c c c|}
-0.999923	 & 	0.00747806	 & 	-0.00991386	 \\
0.00737058	 & 	0.999914	 & 	0.010833	 \\
-0.00999402	 & 	-0.0107591	 & 	0.999892
\end{tabular}

\vtab
 EingenValues - Molecule B     \\
\begin{tabular}{|c c c|}
257.858	 & 	1154.77	 & 	1276.75	 \\
\end{tabular}

\end{center}
\end{multicols}

\vtab[-5mm]
\begin{tabular}{*{2}{m{0.38\textwidth}}}
\begin{center}
\textcolor{NavyBlue}{\Large Different}
\end{center}
&
\begin{center}
\includegraphics[height=6.5cm]{../Comparisons/Vectors/inertia_tensor_of_R-R-R-Fructuose_out_G09_and_R-R-S-Fructuose_out_G09.png}
\end{center}
\end{tabular}

 \newpage

\vtab[-3cm]
\begin{center}
{\large Sugars \tab Número 607}
\end{center}
\begin{multicols}{2}
\begin{center}

Molecule A \
R-R-R-Fructuose\_out\_G09

\includegraphics[width=6cm]{../Comparisons/ImagesFromVMD/R-R-R-Fructuose_out_G09.png}

Inertia Tensor - Molecule A \\
\begin{tabular}{|c c c|}
270.201	 & 	9.83383	 & 	1.05361	 \\
9.83383	 & 	1104.24	 & 	-0.466663	 \\
1.05361	 & 	-0.466663	 & 	1244.5
\end{tabular}

\vtab
 EingenVectors - Molecule A     \\
\begin{tabular}{|c c c|}
0.99993	 & 	-0.0117888	 & 	-0.00108684	 \\
-0.0117923	 & 	-0.999925	 & 	-0.00324079	 \\
0.00104856	 & 	-0.00325338	 & 	0.999994
\end{tabular}

\vtab
 EingenValues - Molecule A     \\
\begin{tabular}{|c c c|}
270.084	 & 	1104.35	 & 	1244.51	 \\
\end{tabular}
\columnbreak

Molecule B \
R-S-R-Fructuose\_out\_G09

\includegraphics[width=6cm]{../Comparisons/ImagesFromVMD/R-S-R-Fructuose_out_G09.png}

Inertia Tensor - Molecule B \\
\begin{tabular}{|c c c|}
203.246	 & 	14.6965	 & 	0.489268	 \\
14.6965	 & 	1215.65	 & 	-0.989769	 \\
0.489268	 & 	-0.989769	 & 	1332.21
\end{tabular}

\vtab
 EingenVectors - Molecule B     \\
\begin{tabular}{|c c c|}
-0.999895	 & 	0.0145123	 & 	0.000445973	 \\
-0.0145155	 & 	-0.999859	 & 	-0.00844446	 \\
0.000323361	 & 	-0.00845004	 & 	0.999964
\end{tabular}

\vtab
 EingenValues - Molecule B     \\
\begin{tabular}{|c c c|}
203.032	 & 	1215.85	 & 	1332.21	 \\
\end{tabular}

\end{center}
\end{multicols}

\vtab[-5mm]
\begin{tabular}{*{2}{m{0.38\textwidth}}}
\begin{center}
\textcolor{NavyBlue}{\Large Different}
\end{center}
&
\begin{center}
\includegraphics[height=6.5cm]{../Comparisons/Vectors/inertia_tensor_of_R-R-R-Fructuose_out_G09_and_R-S-R-Fructuose_out_G09.png}
\end{center}
\end{tabular}

 \newpage

\vtab[-3cm]
\begin{center}
{\large Sugars \tab Número 608}
\end{center}
\begin{multicols}{2}
\begin{center}

Molecule A \
R-R-R-Fructuose\_out\_G09

\includegraphics[width=6cm]{../Comparisons/ImagesFromVMD/R-R-R-Fructuose_out_G09.png}

Inertia Tensor - Molecule A \\
\begin{tabular}{|c c c|}
270.201	 & 	9.83383	 & 	1.05361	 \\
9.83383	 & 	1104.24	 & 	-0.466663	 \\
1.05361	 & 	-0.466663	 & 	1244.5
\end{tabular}

\vtab
 EingenVectors - Molecule A     \\
\begin{tabular}{|c c c|}
0.99993	 & 	-0.0117888	 & 	-0.00108684	 \\
-0.0117923	 & 	-0.999925	 & 	-0.00324079	 \\
0.00104856	 & 	-0.00325338	 & 	0.999994
\end{tabular}

\vtab
 EingenValues - Molecule A     \\
\begin{tabular}{|c c c|}
270.084	 & 	1104.35	 & 	1244.51	 \\
\end{tabular}
\columnbreak

Molecule B \
R-S-S-Fructuose\_out\_G09

\includegraphics[width=6cm]{../Comparisons/ImagesFromVMD/R-S-S-Fructuose_out_G09.png}

Inertia Tensor - Molecule B \\
\begin{tabular}{|c c c|}
214.933	 & 	-13.9132	 & 	-6.26614	 \\
-13.9132	 & 	1186.84	 & 	0.567805	 \\
-6.26614	 & 	0.567805	 & 	1322.37
\end{tabular}

\vtab
 EingenVectors - Molecule B     \\
\begin{tabular}{|c c c|}
0.999882	 & 	0.0143069	 & 	0.00564903	 \\
0.0142795	 & 	-0.999886	 & 	0.00485623	 \\
-0.00571786	 & 	0.00477499	 & 	0.999972
\end{tabular}

\vtab
 EingenValues - Molecule B     \\
\begin{tabular}{|c c c|}
214.698	 & 	1187.04	 & 	1322.41	 \\
\end{tabular}

\end{center}
\end{multicols}

\vtab[-5mm]
\begin{tabular}{*{2}{m{0.38\textwidth}}}
\begin{center}
\textcolor{NavyBlue}{\Large Different}
\end{center}
&
\begin{center}
\includegraphics[height=6.5cm]{../Comparisons/Vectors/inertia_tensor_of_R-R-R-Fructuose_out_G09_and_R-S-S-Fructuose_out_G09.png}
\end{center}
\end{tabular}

 \newpage

\vtab[-3cm]
\begin{center}
{\large Sugars \tab Número 609}
\end{center}
\begin{multicols}{2}
\begin{center}

Molecule A \
R-R-R-Fructuose\_out\_G09

\includegraphics[width=6cm]{../Comparisons/ImagesFromVMD/R-R-R-Fructuose_out_G09.png}

Inertia Tensor - Molecule A \\
\begin{tabular}{|c c c|}
270.201	 & 	9.83383	 & 	1.05361	 \\
9.83383	 & 	1104.24	 & 	-0.466663	 \\
1.05361	 & 	-0.466663	 & 	1244.5
\end{tabular}

\vtab
 EingenVectors - Molecule A     \\
\begin{tabular}{|c c c|}
0.99993	 & 	-0.0117888	 & 	-0.00108684	 \\
-0.0117923	 & 	-0.999925	 & 	-0.00324079	 \\
0.00104856	 & 	-0.00325338	 & 	0.999994
\end{tabular}

\vtab
 EingenValues - Molecule A     \\
\begin{tabular}{|c c c|}
270.084	 & 	1104.35	 & 	1244.51	 \\
\end{tabular}
\columnbreak

Molecule B \
S-R-R-Fructuose\_out\_G09

\includegraphics[width=6cm]{../Comparisons/ImagesFromVMD/S-R-R-Fructuose_out_G09.png}

Inertia Tensor - Molecule B \\
\begin{tabular}{|c c c|}
214.974	 & 	13.9189	 & 	6.25845	 \\
13.9189	 & 	1186.65	 & 	0.565787	 \\
6.25845	 & 	0.565787	 & 	1322.09
\end{tabular}

\vtab
 EingenVectors - Molecule B     \\
\begin{tabular}{|c c c|}
-0.999882	 & 	0.0143163	 & 	0.00564373	 \\
0.014289	 & 	0.999886	 & 	-0.00484396	 \\
0.00571244	 & 	0.00476274	 & 	0.999972
\end{tabular}

\vtab
 EingenValues - Molecule B     \\
\begin{tabular}{|c c c|}
214.739	 & 	1186.84	 & 	1322.13	 \\
\end{tabular}

\end{center}
\end{multicols}

\vtab[-5mm]
\begin{tabular}{*{2}{m{0.38\textwidth}}}
\begin{center}
\textcolor{NavyBlue}{\Large Different}
\end{center}
&
\begin{center}
\includegraphics[height=6.5cm]{../Comparisons/Vectors/inertia_tensor_of_R-R-R-Fructuose_out_G09_and_S-R-R-Fructuose_out_G09.png}
\end{center}
\end{tabular}

 \newpage

\vtab[-3cm]
\begin{center}
{\large Sugars \tab Número 610}
\end{center}
\begin{multicols}{2}
\begin{center}

Molecule A \
R-R-R-Fructuose\_out\_G09

\includegraphics[width=6cm]{../Comparisons/ImagesFromVMD/R-R-R-Fructuose_out_G09.png}

Inertia Tensor - Molecule A \\
\begin{tabular}{|c c c|}
270.201	 & 	9.83383	 & 	1.05361	 \\
9.83383	 & 	1104.24	 & 	-0.466663	 \\
1.05361	 & 	-0.466663	 & 	1244.5
\end{tabular}

\vtab
 EingenVectors - Molecule A     \\
\begin{tabular}{|c c c|}
0.99993	 & 	-0.0117888	 & 	-0.00108684	 \\
-0.0117923	 & 	-0.999925	 & 	-0.00324079	 \\
0.00104856	 & 	-0.00325338	 & 	0.999994
\end{tabular}

\vtab
 EingenValues - Molecule A     \\
\begin{tabular}{|c c c|}
270.084	 & 	1104.35	 & 	1244.51	 \\
\end{tabular}
\columnbreak

Molecule B \
S-R-S-Fructuose\_out\_G09

\includegraphics[width=6cm]{../Comparisons/ImagesFromVMD/S-R-S-Fructuose_out_G09.png}

Inertia Tensor - Molecule B \\
\begin{tabular}{|c c c|}
203.262	 & 	-14.7021	 & 	-0.492665	 \\
-14.7021	 & 	1215.64	 & 	-0.989546	 \\
-0.492665	 & 	-0.989546	 & 	1332.23
\end{tabular}

\vtab
 EingenVectors - Molecule B     \\
\begin{tabular}{|c c c|}
-0.999895	 & 	-0.0145182	 & 	-0.000448979	 \\
0.0145214	 & 	-0.999859	 & 	-0.00843991	 \\
-0.000326384	 & 	-0.00844554	 & 	0.999964
\end{tabular}

\vtab
 EingenValues - Molecule B     \\
\begin{tabular}{|c c c|}
203.048	 & 	1215.85	 & 	1332.24	 \\
\end{tabular}

\end{center}
\end{multicols}

\vtab[-5mm]
\begin{tabular}{*{2}{m{0.38\textwidth}}}
\begin{center}
\textcolor{NavyBlue}{\Large Different}
\end{center}
&
\begin{center}
\includegraphics[height=6.5cm]{../Comparisons/Vectors/inertia_tensor_of_R-R-R-Fructuose_out_G09_and_S-R-S-Fructuose_out_G09.png}
\end{center}
\end{tabular}

 \newpage

\vtab[-3cm]
\begin{center}
{\large Sugars \tab Número 611}
\end{center}
\begin{multicols}{2}
\begin{center}

Molecule A \
R-R-R-Fructuose\_out\_G09

\includegraphics[width=6cm]{../Comparisons/ImagesFromVMD/R-R-R-Fructuose_out_G09.png}

Inertia Tensor - Molecule A \\
\begin{tabular}{|c c c|}
270.201	 & 	9.83383	 & 	1.05361	 \\
9.83383	 & 	1104.24	 & 	-0.466663	 \\
1.05361	 & 	-0.466663	 & 	1244.5
\end{tabular}

\vtab
 EingenVectors - Molecule A     \\
\begin{tabular}{|c c c|}
0.99993	 & 	-0.0117888	 & 	-0.00108684	 \\
-0.0117923	 & 	-0.999925	 & 	-0.00324079	 \\
0.00104856	 & 	-0.00325338	 & 	0.999994
\end{tabular}

\vtab
 EingenValues - Molecule A     \\
\begin{tabular}{|c c c|}
270.084	 & 	1104.35	 & 	1244.51	 \\
\end{tabular}
\columnbreak

Molecule B \
S-S-R-Fructuose\_out\_G09

\includegraphics[width=6cm]{../Comparisons/ImagesFromVMD/S-S-R-Fructuose_out_G09.png}

Inertia Tensor - Molecule B \\
\begin{tabular}{|c c c|}
257.995	 & 	-6.7346	 & 	10.0985	 \\
-6.7346	 & 	1154.74	 & 	-1.24098	 \\
10.0985	 & 	-1.24098	 & 	1276.66
\end{tabular}

\vtab
 EingenVectors - Molecule B     \\
\begin{tabular}{|c c c|}
0.999923	 & 	0.00749452	 & 	-0.00990206	 \\
0.00738758	 & 	-0.999914	 & 	-0.0107925	 \\
0.0099821	 & 	-0.0107185	 & 	0.999893
\end{tabular}

\vtab
 EingenValues - Molecule B     \\
\begin{tabular}{|c c c|}
257.845	 & 	1154.78	 & 	1276.78	 \\
\end{tabular}

\end{center}
\end{multicols}

\vtab[-5mm]
\begin{tabular}{*{2}{m{0.38\textwidth}}}
\begin{center}
\textcolor{NavyBlue}{\Large Different}
\end{center}
&
\begin{center}
\includegraphics[height=6.5cm]{../Comparisons/Vectors/inertia_tensor_of_R-R-R-Fructuose_out_G09_and_S-S-R-Fructuose_out_G09.png}
\end{center}
\end{tabular}

 \newpage

\vtab[-3cm]
\begin{center}
{\large Sugars \tab Número 612}
\end{center}
\begin{multicols}{2}
\begin{center}

Molecule A \
R-R-R-Fructuose\_out\_G09

\includegraphics[width=6cm]{../Comparisons/ImagesFromVMD/R-R-R-Fructuose_out_G09.png}

Inertia Tensor - Molecule A \\
\begin{tabular}{|c c c|}
270.201	 & 	9.83383	 & 	1.05361	 \\
9.83383	 & 	1104.24	 & 	-0.466663	 \\
1.05361	 & 	-0.466663	 & 	1244.5
\end{tabular}

\vtab
 EingenVectors - Molecule A     \\
\begin{tabular}{|c c c|}
0.99993	 & 	-0.0117888	 & 	-0.00108684	 \\
-0.0117923	 & 	-0.999925	 & 	-0.00324079	 \\
0.00104856	 & 	-0.00325338	 & 	0.999994
\end{tabular}

\vtab
 EingenValues - Molecule A     \\
\begin{tabular}{|c c c|}
270.084	 & 	1104.35	 & 	1244.51	 \\
\end{tabular}
\columnbreak

Molecule B \
S-S-S-Fructuose\_out\_G09

\includegraphics[width=6cm]{../Comparisons/ImagesFromVMD/S-S-S-Fructuose_out_G09.png}

Inertia Tensor - Molecule B \\
\begin{tabular}{|c c c|}
224.224	 & 	13.0406	 & 	-2.54457	 \\
13.0406	 & 	1180.14	 & 	-1.05607	 \\
-2.54457	 & 	-1.05607	 & 	1363.89
\end{tabular}

\vtab
 EingenVectors - Molecule B     \\
\begin{tabular}{|c c c|}
-0.999905	 & 	0.0136356	 & 	-0.00221952	 \\
0.0136222	 & 	0.99989	 & 	0.00594071	 \\
-0.00230028	 & 	-0.00590991	 & 	0.99998
\end{tabular}

\vtab
 EingenValues - Molecule B     \\
\begin{tabular}{|c c c|}
224.041	 & 	1180.31	 & 	1363.9	 \\
\end{tabular}

\end{center}
\end{multicols}

\vtab[-5mm]
\begin{tabular}{*{2}{m{0.38\textwidth}}}
\begin{center}
\textcolor{NavyBlue}{\Large Different}
\end{center}
&
\begin{center}
\includegraphics[height=6.5cm]{../Comparisons/Vectors/inertia_tensor_of_R-R-R-Fructuose_out_G09_and_S-S-S-Fructuose_out_G09.png}
\end{center}
\end{tabular}

 \newpage

\vtab[-3cm]
\begin{center}
{\large Sugars \tab Número 613}
\end{center}
\begin{multicols}{2}
\begin{center}

Molecule A \
R-R-S-Fructuose\_out\_G09

\includegraphics[width=6cm]{../Comparisons/ImagesFromVMD/R-R-S-Fructuose_out_G09.png}

Inertia Tensor - Molecule A \\
\begin{tabular}{|c c c|}
258.009	 & 	6.71977	 & 	-10.1101	 \\
6.71977	 & 	1154.74	 & 	-1.24568	 \\
-10.1101	 & 	-1.24568	 & 	1276.63
\end{tabular}

\vtab
 EingenVectors - Molecule A     \\
\begin{tabular}{|c c c|}
-0.999923	 & 	0.00747806	 & 	-0.00991386	 \\
0.00737058	 & 	0.999914	 & 	0.010833	 \\
-0.00999402	 & 	-0.0107591	 & 	0.999892
\end{tabular}

\vtab
 EingenValues - Molecule A     \\
\begin{tabular}{|c c c|}
257.858	 & 	1154.77	 & 	1276.75	 \\
\end{tabular}
\columnbreak

Molecule B \
R-S-R-Fructuose\_out\_G09

\includegraphics[width=6cm]{../Comparisons/ImagesFromVMD/R-S-R-Fructuose_out_G09.png}

Inertia Tensor - Molecule B \\
\begin{tabular}{|c c c|}
203.246	 & 	14.6965	 & 	0.489268	 \\
14.6965	 & 	1215.65	 & 	-0.989769	 \\
0.489268	 & 	-0.989769	 & 	1332.21
\end{tabular}

\vtab
 EingenVectors - Molecule B     \\
\begin{tabular}{|c c c|}
-0.999895	 & 	0.0145123	 & 	0.000445973	 \\
-0.0145155	 & 	-0.999859	 & 	-0.00844446	 \\
0.000323361	 & 	-0.00845004	 & 	0.999964
\end{tabular}

\vtab
 EingenValues - Molecule B     \\
\begin{tabular}{|c c c|}
203.032	 & 	1215.85	 & 	1332.21	 \\
\end{tabular}

\end{center}
\end{multicols}

\vtab[-5mm]
\begin{tabular}{*{2}{m{0.38\textwidth}}}
\begin{center}
\textcolor{NavyBlue}{\Large Different}
\end{center}
&
\begin{center}
\includegraphics[height=6.5cm]{../Comparisons/Vectors/inertia_tensor_of_R-R-S-Fructuose_out_G09_and_R-S-R-Fructuose_out_G09.png}
\end{center}
\end{tabular}

 \newpage

\vtab[-3cm]
\begin{center}
{\large Sugars \tab Número 614}
\end{center}
\begin{multicols}{2}
\begin{center}

Molecule A \
R-R-S-Fructuose\_out\_G09

\includegraphics[width=6cm]{../Comparisons/ImagesFromVMD/R-R-S-Fructuose_out_G09.png}

Inertia Tensor - Molecule A \\
\begin{tabular}{|c c c|}
258.009	 & 	6.71977	 & 	-10.1101	 \\
6.71977	 & 	1154.74	 & 	-1.24568	 \\
-10.1101	 & 	-1.24568	 & 	1276.63
\end{tabular}

\vtab
 EingenVectors - Molecule A     \\
\begin{tabular}{|c c c|}
-0.999923	 & 	0.00747806	 & 	-0.00991386	 \\
0.00737058	 & 	0.999914	 & 	0.010833	 \\
-0.00999402	 & 	-0.0107591	 & 	0.999892
\end{tabular}

\vtab
 EingenValues - Molecule A     \\
\begin{tabular}{|c c c|}
257.858	 & 	1154.77	 & 	1276.75	 \\
\end{tabular}
\columnbreak

Molecule B \
R-S-S-Fructuose\_out\_G09

\includegraphics[width=6cm]{../Comparisons/ImagesFromVMD/R-S-S-Fructuose_out_G09.png}

Inertia Tensor - Molecule B \\
\begin{tabular}{|c c c|}
214.933	 & 	-13.9132	 & 	-6.26614	 \\
-13.9132	 & 	1186.84	 & 	0.567805	 \\
-6.26614	 & 	0.567805	 & 	1322.37
\end{tabular}

\vtab
 EingenVectors - Molecule B     \\
\begin{tabular}{|c c c|}
0.999882	 & 	0.0143069	 & 	0.00564903	 \\
0.0142795	 & 	-0.999886	 & 	0.00485623	 \\
-0.00571786	 & 	0.00477499	 & 	0.999972
\end{tabular}

\vtab
 EingenValues - Molecule B     \\
\begin{tabular}{|c c c|}
214.698	 & 	1187.04	 & 	1322.41	 \\
\end{tabular}

\end{center}
\end{multicols}

\vtab[-5mm]
\begin{tabular}{*{2}{m{0.38\textwidth}}}
\begin{center}
\textcolor{NavyBlue}{\Large Different}
\end{center}
&
\begin{center}
\includegraphics[height=6.5cm]{../Comparisons/Vectors/inertia_tensor_of_R-R-S-Fructuose_out_G09_and_R-S-S-Fructuose_out_G09.png}
\end{center}
\end{tabular}

 \newpage

\vtab[-3cm]
\begin{center}
{\large Sugars \tab Número 615}
\end{center}
\begin{multicols}{2}
\begin{center}

Molecule A \
R-R-S-Fructuose\_out\_G09

\includegraphics[width=6cm]{../Comparisons/ImagesFromVMD/R-R-S-Fructuose_out_G09.png}

Inertia Tensor - Molecule A \\
\begin{tabular}{|c c c|}
258.009	 & 	6.71977	 & 	-10.1101	 \\
6.71977	 & 	1154.74	 & 	-1.24568	 \\
-10.1101	 & 	-1.24568	 & 	1276.63
\end{tabular}

\vtab
 EingenVectors - Molecule A     \\
\begin{tabular}{|c c c|}
-0.999923	 & 	0.00747806	 & 	-0.00991386	 \\
0.00737058	 & 	0.999914	 & 	0.010833	 \\
-0.00999402	 & 	-0.0107591	 & 	0.999892
\end{tabular}

\vtab
 EingenValues - Molecule A     \\
\begin{tabular}{|c c c|}
257.858	 & 	1154.77	 & 	1276.75	 \\
\end{tabular}
\columnbreak

Molecule B \
S-R-R-Fructuose\_out\_G09

\includegraphics[width=6cm]{../Comparisons/ImagesFromVMD/S-R-R-Fructuose_out_G09.png}

Inertia Tensor - Molecule B \\
\begin{tabular}{|c c c|}
214.974	 & 	13.9189	 & 	6.25845	 \\
13.9189	 & 	1186.65	 & 	0.565787	 \\
6.25845	 & 	0.565787	 & 	1322.09
\end{tabular}

\vtab
 EingenVectors - Molecule B     \\
\begin{tabular}{|c c c|}
-0.999882	 & 	0.0143163	 & 	0.00564373	 \\
0.014289	 & 	0.999886	 & 	-0.00484396	 \\
0.00571244	 & 	0.00476274	 & 	0.999972
\end{tabular}

\vtab
 EingenValues - Molecule B     \\
\begin{tabular}{|c c c|}
214.739	 & 	1186.84	 & 	1322.13	 \\
\end{tabular}

\end{center}
\end{multicols}

\vtab[-5mm]
\begin{tabular}{*{2}{m{0.38\textwidth}}}
\begin{center}
\textcolor{NavyBlue}{\Large Different}
\end{center}
&
\begin{center}
\includegraphics[height=6.5cm]{../Comparisons/Vectors/inertia_tensor_of_R-R-S-Fructuose_out_G09_and_S-R-R-Fructuose_out_G09.png}
\end{center}
\end{tabular}

 \newpage

\vtab[-3cm]
\begin{center}
{\large Sugars \tab Número 616}
\end{center}
\begin{multicols}{2}
\begin{center}

Molecule A \
R-R-S-Fructuose\_out\_G09

\includegraphics[width=6cm]{../Comparisons/ImagesFromVMD/R-R-S-Fructuose_out_G09.png}

Inertia Tensor - Molecule A \\
\begin{tabular}{|c c c|}
258.009	 & 	6.71977	 & 	-10.1101	 \\
6.71977	 & 	1154.74	 & 	-1.24568	 \\
-10.1101	 & 	-1.24568	 & 	1276.63
\end{tabular}

\vtab
 EingenVectors - Molecule A     \\
\begin{tabular}{|c c c|}
-0.999923	 & 	0.00747806	 & 	-0.00991386	 \\
0.00737058	 & 	0.999914	 & 	0.010833	 \\
-0.00999402	 & 	-0.0107591	 & 	0.999892
\end{tabular}

\vtab
 EingenValues - Molecule A     \\
\begin{tabular}{|c c c|}
257.858	 & 	1154.77	 & 	1276.75	 \\
\end{tabular}
\columnbreak

Molecule B \
S-R-S-Fructuose\_out\_G09

\includegraphics[width=6cm]{../Comparisons/ImagesFromVMD/S-R-S-Fructuose_out_G09.png}

Inertia Tensor - Molecule B \\
\begin{tabular}{|c c c|}
203.262	 & 	-14.7021	 & 	-0.492665	 \\
-14.7021	 & 	1215.64	 & 	-0.989546	 \\
-0.492665	 & 	-0.989546	 & 	1332.23
\end{tabular}

\vtab
 EingenVectors - Molecule B     \\
\begin{tabular}{|c c c|}
-0.999895	 & 	-0.0145182	 & 	-0.000448979	 \\
0.0145214	 & 	-0.999859	 & 	-0.00843991	 \\
-0.000326384	 & 	-0.00844554	 & 	0.999964
\end{tabular}

\vtab
 EingenValues - Molecule B     \\
\begin{tabular}{|c c c|}
203.048	 & 	1215.85	 & 	1332.24	 \\
\end{tabular}

\end{center}
\end{multicols}

\vtab[-5mm]
\begin{tabular}{*{2}{m{0.38\textwidth}}}
\begin{center}
\textcolor{NavyBlue}{\Large Different}
\end{center}
&
\begin{center}
\includegraphics[height=6.5cm]{../Comparisons/Vectors/inertia_tensor_of_R-R-S-Fructuose_out_G09_and_S-R-S-Fructuose_out_G09.png}
\end{center}
\end{tabular}

 \newpage

\vtab[-3cm]
\begin{center}
{\large Sugars \tab Número 617}
\end{center}
\begin{multicols}{2}
\begin{center}

Molecule A \
R-R-S-Fructuose\_out\_G09

\includegraphics[width=6cm]{../Comparisons/ImagesFromVMD/R-R-S-Fructuose_out_G09.png}

Inertia Tensor - Molecule A \\
\begin{tabular}{|c c c|}
258.009	 & 	6.71977	 & 	-10.1101	 \\
6.71977	 & 	1154.74	 & 	-1.24568	 \\
-10.1101	 & 	-1.24568	 & 	1276.63
\end{tabular}

\vtab
 EingenVectors - Molecule A     \\
\begin{tabular}{|c c c|}
-0.999923	 & 	0.00747806	 & 	-0.00991386	 \\
0.00737058	 & 	0.999914	 & 	0.010833	 \\
-0.00999402	 & 	-0.0107591	 & 	0.999892
\end{tabular}

\vtab
 EingenValues - Molecule A     \\
\begin{tabular}{|c c c|}
257.858	 & 	1154.77	 & 	1276.75	 \\
\end{tabular}
\columnbreak

Molecule B \
S-S-R-Fructuose\_out\_G09

\includegraphics[width=6cm]{../Comparisons/ImagesFromVMD/S-S-R-Fructuose_out_G09.png}

Inertia Tensor - Molecule B \\
\begin{tabular}{|c c c|}
257.995	 & 	-6.7346	 & 	10.0985	 \\
-6.7346	 & 	1154.74	 & 	-1.24098	 \\
10.0985	 & 	-1.24098	 & 	1276.66
\end{tabular}

\vtab
 EingenVectors - Molecule B     \\
\begin{tabular}{|c c c|}
0.999923	 & 	0.00749452	 & 	-0.00990206	 \\
0.00738758	 & 	-0.999914	 & 	-0.0107925	 \\
0.0099821	 & 	-0.0107185	 & 	0.999893
\end{tabular}

\vtab
 EingenValues - Molecule B     \\
\begin{tabular}{|c c c|}
257.845	 & 	1154.78	 & 	1276.78	 \\
\end{tabular}

\end{center}
\end{multicols}

\vtab[-5mm]
\begin{tabular}{*{2}{m{0.38\textwidth}}}
\begin{center}
\textcolor{NavyBlue}{\Large Enantiomers}
\end{center}
&
\begin{center}
\includegraphics[height=6.5cm]{../Comparisons/Vectors/inertia_tensor_of_R-R-S-Fructuose_out_G09_and_S-S-R-Fructuose_out_G09.png}
\end{center}
\end{tabular}

 \newpage

\vtab[-3cm]
\begin{center}
{\large Sugars \tab Número 618}
\end{center}
\begin{multicols}{2}
\begin{center}

Molecule A \
R-R-S-Fructuose\_out\_G09

\includegraphics[width=6cm]{../Comparisons/ImagesFromVMD/R-R-S-Fructuose_out_G09.png}

Inertia Tensor - Molecule A \\
\begin{tabular}{|c c c|}
258.009	 & 	6.71977	 & 	-10.1101	 \\
6.71977	 & 	1154.74	 & 	-1.24568	 \\
-10.1101	 & 	-1.24568	 & 	1276.63
\end{tabular}

\vtab
 EingenVectors - Molecule A     \\
\begin{tabular}{|c c c|}
-0.999923	 & 	0.00747806	 & 	-0.00991386	 \\
0.00737058	 & 	0.999914	 & 	0.010833	 \\
-0.00999402	 & 	-0.0107591	 & 	0.999892
\end{tabular}

\vtab
 EingenValues - Molecule A     \\
\begin{tabular}{|c c c|}
257.858	 & 	1154.77	 & 	1276.75	 \\
\end{tabular}
\columnbreak

Molecule B \
S-S-S-Fructuose\_out\_G09

\includegraphics[width=6cm]{../Comparisons/ImagesFromVMD/S-S-S-Fructuose_out_G09.png}

Inertia Tensor - Molecule B \\
\begin{tabular}{|c c c|}
224.224	 & 	13.0406	 & 	-2.54457	 \\
13.0406	 & 	1180.14	 & 	-1.05607	 \\
-2.54457	 & 	-1.05607	 & 	1363.89
\end{tabular}

\vtab
 EingenVectors - Molecule B     \\
\begin{tabular}{|c c c|}
-0.999905	 & 	0.0136356	 & 	-0.00221952	 \\
0.0136222	 & 	0.99989	 & 	0.00594071	 \\
-0.00230028	 & 	-0.00590991	 & 	0.99998
\end{tabular}

\vtab
 EingenValues - Molecule B     \\
\begin{tabular}{|c c c|}
224.041	 & 	1180.31	 & 	1363.9	 \\
\end{tabular}

\end{center}
\end{multicols}

\vtab[-5mm]
\begin{tabular}{*{2}{m{0.38\textwidth}}}
\begin{center}
\textcolor{NavyBlue}{\Large Different}
\end{center}
&
\begin{center}
\includegraphics[height=6.5cm]{../Comparisons/Vectors/inertia_tensor_of_R-R-S-Fructuose_out_G09_and_S-S-S-Fructuose_out_G09.png}
\end{center}
\end{tabular}

 \newpage

\vtab[-3cm]
\begin{center}
{\large Sugars \tab Número 619}
\end{center}
\begin{multicols}{2}
\begin{center}

Molecule A \
R-S-R-Fructuose\_out\_G09

\includegraphics[width=6cm]{../Comparisons/ImagesFromVMD/R-S-R-Fructuose_out_G09.png}

Inertia Tensor - Molecule A \\
\begin{tabular}{|c c c|}
203.246	 & 	14.6965	 & 	0.489268	 \\
14.6965	 & 	1215.65	 & 	-0.989769	 \\
0.489268	 & 	-0.989769	 & 	1332.21
\end{tabular}

\vtab
 EingenVectors - Molecule A     \\
\begin{tabular}{|c c c|}
-0.999895	 & 	0.0145123	 & 	0.000445973	 \\
-0.0145155	 & 	-0.999859	 & 	-0.00844446	 \\
0.000323361	 & 	-0.00845004	 & 	0.999964
\end{tabular}

\vtab
 EingenValues - Molecule A     \\
\begin{tabular}{|c c c|}
203.032	 & 	1215.85	 & 	1332.21	 \\
\end{tabular}
\columnbreak

Molecule B \
R-S-S-Fructuose\_out\_G09

\includegraphics[width=6cm]{../Comparisons/ImagesFromVMD/R-S-S-Fructuose_out_G09.png}

Inertia Tensor - Molecule B \\
\begin{tabular}{|c c c|}
214.933	 & 	-13.9132	 & 	-6.26614	 \\
-13.9132	 & 	1186.84	 & 	0.567805	 \\
-6.26614	 & 	0.567805	 & 	1322.37
\end{tabular}

\vtab
 EingenVectors - Molecule B     \\
\begin{tabular}{|c c c|}
0.999882	 & 	0.0143069	 & 	0.00564903	 \\
0.0142795	 & 	-0.999886	 & 	0.00485623	 \\
-0.00571786	 & 	0.00477499	 & 	0.999972
\end{tabular}

\vtab
 EingenValues - Molecule B     \\
\begin{tabular}{|c c c|}
214.698	 & 	1187.04	 & 	1322.41	 \\
\end{tabular}

\end{center}
\end{multicols}

\vtab[-5mm]
\begin{tabular}{*{2}{m{0.38\textwidth}}}
\begin{center}
\textcolor{NavyBlue}{\Large Different}
\end{center}
&
\begin{center}
\includegraphics[height=6.5cm]{../Comparisons/Vectors/inertia_tensor_of_R-S-R-Fructuose_out_G09_and_R-S-S-Fructuose_out_G09.png}
\end{center}
\end{tabular}

 \newpage

\vtab[-3cm]
\begin{center}
{\large Sugars \tab Número 620}
\end{center}
\begin{multicols}{2}
\begin{center}

Molecule A \
R-S-R-Fructuose\_out\_G09

\includegraphics[width=6cm]{../Comparisons/ImagesFromVMD/R-S-R-Fructuose_out_G09.png}

Inertia Tensor - Molecule A \\
\begin{tabular}{|c c c|}
203.246	 & 	14.6965	 & 	0.489268	 \\
14.6965	 & 	1215.65	 & 	-0.989769	 \\
0.489268	 & 	-0.989769	 & 	1332.21
\end{tabular}

\vtab
 EingenVectors - Molecule A     \\
\begin{tabular}{|c c c|}
-0.999895	 & 	0.0145123	 & 	0.000445973	 \\
-0.0145155	 & 	-0.999859	 & 	-0.00844446	 \\
0.000323361	 & 	-0.00845004	 & 	0.999964
\end{tabular}

\vtab
 EingenValues - Molecule A     \\
\begin{tabular}{|c c c|}
203.032	 & 	1215.85	 & 	1332.21	 \\
\end{tabular}
\columnbreak

Molecule B \
S-R-R-Fructuose\_out\_G09

\includegraphics[width=6cm]{../Comparisons/ImagesFromVMD/S-R-R-Fructuose_out_G09.png}

Inertia Tensor - Molecule B \\
\begin{tabular}{|c c c|}
214.974	 & 	13.9189	 & 	6.25845	 \\
13.9189	 & 	1186.65	 & 	0.565787	 \\
6.25845	 & 	0.565787	 & 	1322.09
\end{tabular}

\vtab
 EingenVectors - Molecule B     \\
\begin{tabular}{|c c c|}
-0.999882	 & 	0.0143163	 & 	0.00564373	 \\
0.014289	 & 	0.999886	 & 	-0.00484396	 \\
0.00571244	 & 	0.00476274	 & 	0.999972
\end{tabular}

\vtab
 EingenValues - Molecule B     \\
\begin{tabular}{|c c c|}
214.739	 & 	1186.84	 & 	1322.13	 \\
\end{tabular}

\end{center}
\end{multicols}

\vtab[-5mm]
\begin{tabular}{*{2}{m{0.38\textwidth}}}
\begin{center}
\textcolor{NavyBlue}{\Large Different}
\end{center}
&
\begin{center}
\includegraphics[height=6.5cm]{../Comparisons/Vectors/inertia_tensor_of_R-S-R-Fructuose_out_G09_and_S-R-R-Fructuose_out_G09.png}
\end{center}
\end{tabular}

 \newpage

\vtab[-3cm]
\begin{center}
{\large Sugars \tab Número 621}
\end{center}
\begin{multicols}{2}
\begin{center}

Molecule A \
R-S-R-Fructuose\_out\_G09

\includegraphics[width=6cm]{../Comparisons/ImagesFromVMD/R-S-R-Fructuose_out_G09.png}

Inertia Tensor - Molecule A \\
\begin{tabular}{|c c c|}
203.246	 & 	14.6965	 & 	0.489268	 \\
14.6965	 & 	1215.65	 & 	-0.989769	 \\
0.489268	 & 	-0.989769	 & 	1332.21
\end{tabular}

\vtab
 EingenVectors - Molecule A     \\
\begin{tabular}{|c c c|}
-0.999895	 & 	0.0145123	 & 	0.000445973	 \\
-0.0145155	 & 	-0.999859	 & 	-0.00844446	 \\
0.000323361	 & 	-0.00845004	 & 	0.999964
\end{tabular}

\vtab
 EingenValues - Molecule A     \\
\begin{tabular}{|c c c|}
203.032	 & 	1215.85	 & 	1332.21	 \\
\end{tabular}
\columnbreak

Molecule B \
S-R-S-Fructuose\_out\_G09

\includegraphics[width=6cm]{../Comparisons/ImagesFromVMD/S-R-S-Fructuose_out_G09.png}

Inertia Tensor - Molecule B \\
\begin{tabular}{|c c c|}
203.262	 & 	-14.7021	 & 	-0.492665	 \\
-14.7021	 & 	1215.64	 & 	-0.989546	 \\
-0.492665	 & 	-0.989546	 & 	1332.23
\end{tabular}

\vtab
 EingenVectors - Molecule B     \\
\begin{tabular}{|c c c|}
-0.999895	 & 	-0.0145182	 & 	-0.000448979	 \\
0.0145214	 & 	-0.999859	 & 	-0.00843991	 \\
-0.000326384	 & 	-0.00844554	 & 	0.999964
\end{tabular}

\vtab
 EingenValues - Molecule B     \\
\begin{tabular}{|c c c|}
203.048	 & 	1215.85	 & 	1332.24	 \\
\end{tabular}

\end{center}
\end{multicols}

\vtab[-5mm]
\begin{tabular}{*{2}{m{0.38\textwidth}}}
\begin{center}
\textcolor{NavyBlue}{\Large Enantiomers}
\end{center}
&
\begin{center}
\includegraphics[height=6.5cm]{../Comparisons/Vectors/inertia_tensor_of_R-S-R-Fructuose_out_G09_and_S-R-S-Fructuose_out_G09.png}
\end{center}
\end{tabular}

 \newpage

\vtab[-3cm]
\begin{center}
{\large Sugars \tab Número 622}
\end{center}
\begin{multicols}{2}
\begin{center}

Molecule A \
R-S-R-Fructuose\_out\_G09

\includegraphics[width=6cm]{../Comparisons/ImagesFromVMD/R-S-R-Fructuose_out_G09.png}

Inertia Tensor - Molecule A \\
\begin{tabular}{|c c c|}
203.246	 & 	14.6965	 & 	0.489268	 \\
14.6965	 & 	1215.65	 & 	-0.989769	 \\
0.489268	 & 	-0.989769	 & 	1332.21
\end{tabular}

\vtab
 EingenVectors - Molecule A     \\
\begin{tabular}{|c c c|}
-0.999895	 & 	0.0145123	 & 	0.000445973	 \\
-0.0145155	 & 	-0.999859	 & 	-0.00844446	 \\
0.000323361	 & 	-0.00845004	 & 	0.999964
\end{tabular}

\vtab
 EingenValues - Molecule A     \\
\begin{tabular}{|c c c|}
203.032	 & 	1215.85	 & 	1332.21	 \\
\end{tabular}
\columnbreak

Molecule B \
S-S-R-Fructuose\_out\_G09

\includegraphics[width=6cm]{../Comparisons/ImagesFromVMD/S-S-R-Fructuose_out_G09.png}

Inertia Tensor - Molecule B \\
\begin{tabular}{|c c c|}
257.995	 & 	-6.7346	 & 	10.0985	 \\
-6.7346	 & 	1154.74	 & 	-1.24098	 \\
10.0985	 & 	-1.24098	 & 	1276.66
\end{tabular}

\vtab
 EingenVectors - Molecule B     \\
\begin{tabular}{|c c c|}
0.999923	 & 	0.00749452	 & 	-0.00990206	 \\
0.00738758	 & 	-0.999914	 & 	-0.0107925	 \\
0.0099821	 & 	-0.0107185	 & 	0.999893
\end{tabular}

\vtab
 EingenValues - Molecule B     \\
\begin{tabular}{|c c c|}
257.845	 & 	1154.78	 & 	1276.78	 \\
\end{tabular}

\end{center}
\end{multicols}

\vtab[-5mm]
\begin{tabular}{*{2}{m{0.38\textwidth}}}
\begin{center}
\textcolor{NavyBlue}{\Large Different}
\end{center}
&
\begin{center}
\includegraphics[height=6.5cm]{../Comparisons/Vectors/inertia_tensor_of_R-S-R-Fructuose_out_G09_and_S-S-R-Fructuose_out_G09.png}
\end{center}
\end{tabular}

 \newpage

\vtab[-3cm]
\begin{center}
{\large Sugars \tab Número 623}
\end{center}
\begin{multicols}{2}
\begin{center}

Molecule A \
R-S-R-Fructuose\_out\_G09

\includegraphics[width=6cm]{../Comparisons/ImagesFromVMD/R-S-R-Fructuose_out_G09.png}

Inertia Tensor - Molecule A \\
\begin{tabular}{|c c c|}
203.246	 & 	14.6965	 & 	0.489268	 \\
14.6965	 & 	1215.65	 & 	-0.989769	 \\
0.489268	 & 	-0.989769	 & 	1332.21
\end{tabular}

\vtab
 EingenVectors - Molecule A     \\
\begin{tabular}{|c c c|}
-0.999895	 & 	0.0145123	 & 	0.000445973	 \\
-0.0145155	 & 	-0.999859	 & 	-0.00844446	 \\
0.000323361	 & 	-0.00845004	 & 	0.999964
\end{tabular}

\vtab
 EingenValues - Molecule A     \\
\begin{tabular}{|c c c|}
203.032	 & 	1215.85	 & 	1332.21	 \\
\end{tabular}
\columnbreak

Molecule B \
S-S-S-Fructuose\_out\_G09

\includegraphics[width=6cm]{../Comparisons/ImagesFromVMD/S-S-S-Fructuose_out_G09.png}

Inertia Tensor - Molecule B \\
\begin{tabular}{|c c c|}
224.224	 & 	13.0406	 & 	-2.54457	 \\
13.0406	 & 	1180.14	 & 	-1.05607	 \\
-2.54457	 & 	-1.05607	 & 	1363.89
\end{tabular}

\vtab
 EingenVectors - Molecule B     \\
\begin{tabular}{|c c c|}
-0.999905	 & 	0.0136356	 & 	-0.00221952	 \\
0.0136222	 & 	0.99989	 & 	0.00594071	 \\
-0.00230028	 & 	-0.00590991	 & 	0.99998
\end{tabular}

\vtab
 EingenValues - Molecule B     \\
\begin{tabular}{|c c c|}
224.041	 & 	1180.31	 & 	1363.9	 \\
\end{tabular}

\end{center}
\end{multicols}

\vtab[-5mm]
\begin{tabular}{*{2}{m{0.38\textwidth}}}
\begin{center}
\textcolor{NavyBlue}{\Large Different}
\end{center}
&
\begin{center}
\includegraphics[height=6.5cm]{../Comparisons/Vectors/inertia_tensor_of_R-S-R-Fructuose_out_G09_and_S-S-S-Fructuose_out_G09.png}
\end{center}
\end{tabular}

 \newpage

\vtab[-3cm]
\begin{center}
{\large Sugars \tab Número 624}
\end{center}
\begin{multicols}{2}
\begin{center}

Molecule A \
R-S-S-Fructuose\_out\_G09

\includegraphics[width=6cm]{../Comparisons/ImagesFromVMD/R-S-S-Fructuose_out_G09.png}

Inertia Tensor - Molecule A \\
\begin{tabular}{|c c c|}
214.933	 & 	-13.9132	 & 	-6.26614	 \\
-13.9132	 & 	1186.84	 & 	0.567805	 \\
-6.26614	 & 	0.567805	 & 	1322.37
\end{tabular}

\vtab
 EingenVectors - Molecule A     \\
\begin{tabular}{|c c c|}
0.999882	 & 	0.0143069	 & 	0.00564903	 \\
0.0142795	 & 	-0.999886	 & 	0.00485623	 \\
-0.00571786	 & 	0.00477499	 & 	0.999972
\end{tabular}

\vtab
 EingenValues - Molecule A     \\
\begin{tabular}{|c c c|}
214.698	 & 	1187.04	 & 	1322.41	 \\
\end{tabular}
\columnbreak

Molecule B \
S-R-R-Fructuose\_out\_G09

\includegraphics[width=6cm]{../Comparisons/ImagesFromVMD/S-R-R-Fructuose_out_G09.png}

Inertia Tensor - Molecule B \\
\begin{tabular}{|c c c|}
214.974	 & 	13.9189	 & 	6.25845	 \\
13.9189	 & 	1186.65	 & 	0.565787	 \\
6.25845	 & 	0.565787	 & 	1322.09
\end{tabular}

\vtab
 EingenVectors - Molecule B     \\
\begin{tabular}{|c c c|}
-0.999882	 & 	0.0143163	 & 	0.00564373	 \\
0.014289	 & 	0.999886	 & 	-0.00484396	 \\
0.00571244	 & 	0.00476274	 & 	0.999972
\end{tabular}

\vtab
 EingenValues - Molecule B     \\
\begin{tabular}{|c c c|}
214.739	 & 	1186.84	 & 	1322.13	 \\
\end{tabular}

\end{center}
\end{multicols}

\vtab[-5mm]
\begin{tabular}{*{2}{m{0.38\textwidth}}}
\begin{center}
\textcolor{NavyBlue}{\Large Enantiomers}
\end{center}
&
\begin{center}
\includegraphics[height=6.5cm]{../Comparisons/Vectors/inertia_tensor_of_R-S-S-Fructuose_out_G09_and_S-R-R-Fructuose_out_G09.png}
\end{center}
\end{tabular}

 \newpage

\vtab[-3cm]
\begin{center}
{\large Sugars \tab Número 625}
\end{center}
\begin{multicols}{2}
\begin{center}

Molecule A \
R-S-S-Fructuose\_out\_G09

\includegraphics[width=6cm]{../Comparisons/ImagesFromVMD/R-S-S-Fructuose_out_G09.png}

Inertia Tensor - Molecule A \\
\begin{tabular}{|c c c|}
214.933	 & 	-13.9132	 & 	-6.26614	 \\
-13.9132	 & 	1186.84	 & 	0.567805	 \\
-6.26614	 & 	0.567805	 & 	1322.37
\end{tabular}

\vtab
 EingenVectors - Molecule A     \\
\begin{tabular}{|c c c|}
0.999882	 & 	0.0143069	 & 	0.00564903	 \\
0.0142795	 & 	-0.999886	 & 	0.00485623	 \\
-0.00571786	 & 	0.00477499	 & 	0.999972
\end{tabular}

\vtab
 EingenValues - Molecule A     \\
\begin{tabular}{|c c c|}
214.698	 & 	1187.04	 & 	1322.41	 \\
\end{tabular}
\columnbreak

Molecule B \
S-R-S-Fructuose\_out\_G09

\includegraphics[width=6cm]{../Comparisons/ImagesFromVMD/S-R-S-Fructuose_out_G09.png}

Inertia Tensor - Molecule B \\
\begin{tabular}{|c c c|}
203.262	 & 	-14.7021	 & 	-0.492665	 \\
-14.7021	 & 	1215.64	 & 	-0.989546	 \\
-0.492665	 & 	-0.989546	 & 	1332.23
\end{tabular}

\vtab
 EingenVectors - Molecule B     \\
\begin{tabular}{|c c c|}
-0.999895	 & 	-0.0145182	 & 	-0.000448979	 \\
0.0145214	 & 	-0.999859	 & 	-0.00843991	 \\
-0.000326384	 & 	-0.00844554	 & 	0.999964
\end{tabular}

\vtab
 EingenValues - Molecule B     \\
\begin{tabular}{|c c c|}
203.048	 & 	1215.85	 & 	1332.24	 \\
\end{tabular}

\end{center}
\end{multicols}

\vtab[-5mm]
\begin{tabular}{*{2}{m{0.38\textwidth}}}
\begin{center}
\textcolor{NavyBlue}{\Large Different}
\end{center}
&
\begin{center}
\includegraphics[height=6.5cm]{../Comparisons/Vectors/inertia_tensor_of_R-S-S-Fructuose_out_G09_and_S-R-S-Fructuose_out_G09.png}
\end{center}
\end{tabular}

 \newpage

\vtab[-3cm]
\begin{center}
{\large Sugars \tab Número 626}
\end{center}
\begin{multicols}{2}
\begin{center}

Molecule A \
R-S-S-Fructuose\_out\_G09

\includegraphics[width=6cm]{../Comparisons/ImagesFromVMD/R-S-S-Fructuose_out_G09.png}

Inertia Tensor - Molecule A \\
\begin{tabular}{|c c c|}
214.933	 & 	-13.9132	 & 	-6.26614	 \\
-13.9132	 & 	1186.84	 & 	0.567805	 \\
-6.26614	 & 	0.567805	 & 	1322.37
\end{tabular}

\vtab
 EingenVectors - Molecule A     \\
\begin{tabular}{|c c c|}
0.999882	 & 	0.0143069	 & 	0.00564903	 \\
0.0142795	 & 	-0.999886	 & 	0.00485623	 \\
-0.00571786	 & 	0.00477499	 & 	0.999972
\end{tabular}

\vtab
 EingenValues - Molecule A     \\
\begin{tabular}{|c c c|}
214.698	 & 	1187.04	 & 	1322.41	 \\
\end{tabular}
\columnbreak

Molecule B \
S-S-R-Fructuose\_out\_G09

\includegraphics[width=6cm]{../Comparisons/ImagesFromVMD/S-S-R-Fructuose_out_G09.png}

Inertia Tensor - Molecule B \\
\begin{tabular}{|c c c|}
257.995	 & 	-6.7346	 & 	10.0985	 \\
-6.7346	 & 	1154.74	 & 	-1.24098	 \\
10.0985	 & 	-1.24098	 & 	1276.66
\end{tabular}

\vtab
 EingenVectors - Molecule B     \\
\begin{tabular}{|c c c|}
0.999923	 & 	0.00749452	 & 	-0.00990206	 \\
0.00738758	 & 	-0.999914	 & 	-0.0107925	 \\
0.0099821	 & 	-0.0107185	 & 	0.999893
\end{tabular}

\vtab
 EingenValues - Molecule B     \\
\begin{tabular}{|c c c|}
257.845	 & 	1154.78	 & 	1276.78	 \\
\end{tabular}

\end{center}
\end{multicols}

\vtab[-5mm]
\begin{tabular}{*{2}{m{0.38\textwidth}}}
\begin{center}
\textcolor{NavyBlue}{\Large Different}
\end{center}
&
\begin{center}
\includegraphics[height=6.5cm]{../Comparisons/Vectors/inertia_tensor_of_R-S-S-Fructuose_out_G09_and_S-S-R-Fructuose_out_G09.png}
\end{center}
\end{tabular}

 \newpage

\vtab[-3cm]
\begin{center}
{\large Sugars \tab Número 627}
\end{center}
\begin{multicols}{2}
\begin{center}

Molecule A \
R-S-S-Fructuose\_out\_G09

\includegraphics[width=6cm]{../Comparisons/ImagesFromVMD/R-S-S-Fructuose_out_G09.png}

Inertia Tensor - Molecule A \\
\begin{tabular}{|c c c|}
214.933	 & 	-13.9132	 & 	-6.26614	 \\
-13.9132	 & 	1186.84	 & 	0.567805	 \\
-6.26614	 & 	0.567805	 & 	1322.37
\end{tabular}

\vtab
 EingenVectors - Molecule A     \\
\begin{tabular}{|c c c|}
0.999882	 & 	0.0143069	 & 	0.00564903	 \\
0.0142795	 & 	-0.999886	 & 	0.00485623	 \\
-0.00571786	 & 	0.00477499	 & 	0.999972
\end{tabular}

\vtab
 EingenValues - Molecule A     \\
\begin{tabular}{|c c c|}
214.698	 & 	1187.04	 & 	1322.41	 \\
\end{tabular}
\columnbreak

Molecule B \
S-S-S-Fructuose\_out\_G09

\includegraphics[width=6cm]{../Comparisons/ImagesFromVMD/S-S-S-Fructuose_out_G09.png}

Inertia Tensor - Molecule B \\
\begin{tabular}{|c c c|}
224.224	 & 	13.0406	 & 	-2.54457	 \\
13.0406	 & 	1180.14	 & 	-1.05607	 \\
-2.54457	 & 	-1.05607	 & 	1363.89
\end{tabular}

\vtab
 EingenVectors - Molecule B     \\
\begin{tabular}{|c c c|}
-0.999905	 & 	0.0136356	 & 	-0.00221952	 \\
0.0136222	 & 	0.99989	 & 	0.00594071	 \\
-0.00230028	 & 	-0.00590991	 & 	0.99998
\end{tabular}

\vtab
 EingenValues - Molecule B     \\
\begin{tabular}{|c c c|}
224.041	 & 	1180.31	 & 	1363.9	 \\
\end{tabular}

\end{center}
\end{multicols}

\vtab[-5mm]
\begin{tabular}{*{2}{m{0.38\textwidth}}}
\begin{center}
\textcolor{NavyBlue}{\Large Different}
\end{center}
&
\begin{center}
\includegraphics[height=6.5cm]{../Comparisons/Vectors/inertia_tensor_of_R-S-S-Fructuose_out_G09_and_S-S-S-Fructuose_out_G09.png}
\end{center}
\end{tabular}

 \newpage

\vtab[-3cm]
\begin{center}
{\large Sugars \tab Número 628}
\end{center}
\begin{multicols}{2}
\begin{center}

Molecule A \
S-R-R-Fructuose\_out\_G09

\includegraphics[width=6cm]{../Comparisons/ImagesFromVMD/S-R-R-Fructuose_out_G09.png}

Inertia Tensor - Molecule A \\
\begin{tabular}{|c c c|}
214.974	 & 	13.9189	 & 	6.25845	 \\
13.9189	 & 	1186.65	 & 	0.565787	 \\
6.25845	 & 	0.565787	 & 	1322.09
\end{tabular}

\vtab
 EingenVectors - Molecule A     \\
\begin{tabular}{|c c c|}
-0.999882	 & 	0.0143163	 & 	0.00564373	 \\
0.014289	 & 	0.999886	 & 	-0.00484396	 \\
0.00571244	 & 	0.00476274	 & 	0.999972
\end{tabular}

\vtab
 EingenValues - Molecule A     \\
\begin{tabular}{|c c c|}
214.739	 & 	1186.84	 & 	1322.13	 \\
\end{tabular}
\columnbreak

Molecule B \
S-R-S-Fructuose\_out\_G09

\includegraphics[width=6cm]{../Comparisons/ImagesFromVMD/S-R-S-Fructuose_out_G09.png}

Inertia Tensor - Molecule B \\
\begin{tabular}{|c c c|}
203.262	 & 	-14.7021	 & 	-0.492665	 \\
-14.7021	 & 	1215.64	 & 	-0.989546	 \\
-0.492665	 & 	-0.989546	 & 	1332.23
\end{tabular}

\vtab
 EingenVectors - Molecule B     \\
\begin{tabular}{|c c c|}
-0.999895	 & 	-0.0145182	 & 	-0.000448979	 \\
0.0145214	 & 	-0.999859	 & 	-0.00843991	 \\
-0.000326384	 & 	-0.00844554	 & 	0.999964
\end{tabular}

\vtab
 EingenValues - Molecule B     \\
\begin{tabular}{|c c c|}
203.048	 & 	1215.85	 & 	1332.24	 \\
\end{tabular}

\end{center}
\end{multicols}

\vtab[-5mm]
\begin{tabular}{*{2}{m{0.38\textwidth}}}
\begin{center}
\textcolor{NavyBlue}{\Large Different}
\end{center}
&
\begin{center}
\includegraphics[height=6.5cm]{../Comparisons/Vectors/inertia_tensor_of_S-R-R-Fructuose_out_G09_and_S-R-S-Fructuose_out_G09.png}
\end{center}
\end{tabular}

 \newpage

\vtab[-3cm]
\begin{center}
{\large Sugars \tab Número 629}
\end{center}
\begin{multicols}{2}
\begin{center}

Molecule A \
S-R-R-Fructuose\_out\_G09

\includegraphics[width=6cm]{../Comparisons/ImagesFromVMD/S-R-R-Fructuose_out_G09.png}

Inertia Tensor - Molecule A \\
\begin{tabular}{|c c c|}
214.974	 & 	13.9189	 & 	6.25845	 \\
13.9189	 & 	1186.65	 & 	0.565787	 \\
6.25845	 & 	0.565787	 & 	1322.09
\end{tabular}

\vtab
 EingenVectors - Molecule A     \\
\begin{tabular}{|c c c|}
-0.999882	 & 	0.0143163	 & 	0.00564373	 \\
0.014289	 & 	0.999886	 & 	-0.00484396	 \\
0.00571244	 & 	0.00476274	 & 	0.999972
\end{tabular}

\vtab
 EingenValues - Molecule A     \\
\begin{tabular}{|c c c|}
214.739	 & 	1186.84	 & 	1322.13	 \\
\end{tabular}
\columnbreak

Molecule B \
S-S-R-Fructuose\_out\_G09

\includegraphics[width=6cm]{../Comparisons/ImagesFromVMD/S-S-R-Fructuose_out_G09.png}

Inertia Tensor - Molecule B \\
\begin{tabular}{|c c c|}
257.995	 & 	-6.7346	 & 	10.0985	 \\
-6.7346	 & 	1154.74	 & 	-1.24098	 \\
10.0985	 & 	-1.24098	 & 	1276.66
\end{tabular}

\vtab
 EingenVectors - Molecule B     \\
\begin{tabular}{|c c c|}
0.999923	 & 	0.00749452	 & 	-0.00990206	 \\
0.00738758	 & 	-0.999914	 & 	-0.0107925	 \\
0.0099821	 & 	-0.0107185	 & 	0.999893
\end{tabular}

\vtab
 EingenValues - Molecule B     \\
\begin{tabular}{|c c c|}
257.845	 & 	1154.78	 & 	1276.78	 \\
\end{tabular}

\end{center}
\end{multicols}

\vtab[-5mm]
\begin{tabular}{*{2}{m{0.38\textwidth}}}
\begin{center}
\textcolor{NavyBlue}{\Large Different}
\end{center}
&
\begin{center}
\includegraphics[height=6.5cm]{../Comparisons/Vectors/inertia_tensor_of_S-R-R-Fructuose_out_G09_and_S-S-R-Fructuose_out_G09.png}
\end{center}
\end{tabular}

 \newpage

\vtab[-3cm]
\begin{center}
{\large Sugars \tab Número 630}
\end{center}
\begin{multicols}{2}
\begin{center}

Molecule A \
S-R-R-Fructuose\_out\_G09

\includegraphics[width=6cm]{../Comparisons/ImagesFromVMD/S-R-R-Fructuose_out_G09.png}

Inertia Tensor - Molecule A \\
\begin{tabular}{|c c c|}
214.974	 & 	13.9189	 & 	6.25845	 \\
13.9189	 & 	1186.65	 & 	0.565787	 \\
6.25845	 & 	0.565787	 & 	1322.09
\end{tabular}

\vtab
 EingenVectors - Molecule A     \\
\begin{tabular}{|c c c|}
-0.999882	 & 	0.0143163	 & 	0.00564373	 \\
0.014289	 & 	0.999886	 & 	-0.00484396	 \\
0.00571244	 & 	0.00476274	 & 	0.999972
\end{tabular}

\vtab
 EingenValues - Molecule A     \\
\begin{tabular}{|c c c|}
214.739	 & 	1186.84	 & 	1322.13	 \\
\end{tabular}
\columnbreak

Molecule B \
S-S-S-Fructuose\_out\_G09

\includegraphics[width=6cm]{../Comparisons/ImagesFromVMD/S-S-S-Fructuose_out_G09.png}

Inertia Tensor - Molecule B \\
\begin{tabular}{|c c c|}
224.224	 & 	13.0406	 & 	-2.54457	 \\
13.0406	 & 	1180.14	 & 	-1.05607	 \\
-2.54457	 & 	-1.05607	 & 	1363.89
\end{tabular}

\vtab
 EingenVectors - Molecule B     \\
\begin{tabular}{|c c c|}
-0.999905	 & 	0.0136356	 & 	-0.00221952	 \\
0.0136222	 & 	0.99989	 & 	0.00594071	 \\
-0.00230028	 & 	-0.00590991	 & 	0.99998
\end{tabular}

\vtab
 EingenValues - Molecule B     \\
\begin{tabular}{|c c c|}
224.041	 & 	1180.31	 & 	1363.9	 \\
\end{tabular}

\end{center}
\end{multicols}

\vtab[-5mm]
\begin{tabular}{*{2}{m{0.38\textwidth}}}
\begin{center}
\textcolor{NavyBlue}{\Large Different}
\end{center}
&
\begin{center}
\includegraphics[height=6.5cm]{../Comparisons/Vectors/inertia_tensor_of_S-R-R-Fructuose_out_G09_and_S-S-S-Fructuose_out_G09.png}
\end{center}
\end{tabular}

 \newpage

\vtab[-3cm]
\begin{center}
{\large Sugars \tab Número 631}
\end{center}
\begin{multicols}{2}
\begin{center}

Molecule A \
S-R-S-Fructuose\_out\_G09

\includegraphics[width=6cm]{../Comparisons/ImagesFromVMD/S-R-S-Fructuose_out_G09.png}

Inertia Tensor - Molecule A \\
\begin{tabular}{|c c c|}
203.262	 & 	-14.7021	 & 	-0.492665	 \\
-14.7021	 & 	1215.64	 & 	-0.989546	 \\
-0.492665	 & 	-0.989546	 & 	1332.23
\end{tabular}

\vtab
 EingenVectors - Molecule A     \\
\begin{tabular}{|c c c|}
-0.999895	 & 	-0.0145182	 & 	-0.000448979	 \\
0.0145214	 & 	-0.999859	 & 	-0.00843991	 \\
-0.000326384	 & 	-0.00844554	 & 	0.999964
\end{tabular}

\vtab
 EingenValues - Molecule A     \\
\begin{tabular}{|c c c|}
203.048	 & 	1215.85	 & 	1332.24	 \\
\end{tabular}
\columnbreak

Molecule B \
S-S-R-Fructuose\_out\_G09

\includegraphics[width=6cm]{../Comparisons/ImagesFromVMD/S-S-R-Fructuose_out_G09.png}

Inertia Tensor - Molecule B \\
\begin{tabular}{|c c c|}
257.995	 & 	-6.7346	 & 	10.0985	 \\
-6.7346	 & 	1154.74	 & 	-1.24098	 \\
10.0985	 & 	-1.24098	 & 	1276.66
\end{tabular}

\vtab
 EingenVectors - Molecule B     \\
\begin{tabular}{|c c c|}
0.999923	 & 	0.00749452	 & 	-0.00990206	 \\
0.00738758	 & 	-0.999914	 & 	-0.0107925	 \\
0.0099821	 & 	-0.0107185	 & 	0.999893
\end{tabular}

\vtab
 EingenValues - Molecule B     \\
\begin{tabular}{|c c c|}
257.845	 & 	1154.78	 & 	1276.78	 \\
\end{tabular}

\end{center}
\end{multicols}

\vtab[-5mm]
\begin{tabular}{*{2}{m{0.38\textwidth}}}
\begin{center}
\textcolor{NavyBlue}{\Large Different}
\end{center}
&
\begin{center}
\includegraphics[height=6.5cm]{../Comparisons/Vectors/inertia_tensor_of_S-R-S-Fructuose_out_G09_and_S-S-R-Fructuose_out_G09.png}
\end{center}
\end{tabular}

 \newpage

\vtab[-3cm]
\begin{center}
{\large Sugars \tab Número 632}
\end{center}
\begin{multicols}{2}
\begin{center}

Molecule A \
S-R-S-Fructuose\_out\_G09

\includegraphics[width=6cm]{../Comparisons/ImagesFromVMD/S-R-S-Fructuose_out_G09.png}

Inertia Tensor - Molecule A \\
\begin{tabular}{|c c c|}
203.262	 & 	-14.7021	 & 	-0.492665	 \\
-14.7021	 & 	1215.64	 & 	-0.989546	 \\
-0.492665	 & 	-0.989546	 & 	1332.23
\end{tabular}

\vtab
 EingenVectors - Molecule A     \\
\begin{tabular}{|c c c|}
-0.999895	 & 	-0.0145182	 & 	-0.000448979	 \\
0.0145214	 & 	-0.999859	 & 	-0.00843991	 \\
-0.000326384	 & 	-0.00844554	 & 	0.999964
\end{tabular}

\vtab
 EingenValues - Molecule A     \\
\begin{tabular}{|c c c|}
203.048	 & 	1215.85	 & 	1332.24	 \\
\end{tabular}
\columnbreak

Molecule B \
S-S-S-Fructuose\_out\_G09

\includegraphics[width=6cm]{../Comparisons/ImagesFromVMD/S-S-S-Fructuose_out_G09.png}

Inertia Tensor - Molecule B \\
\begin{tabular}{|c c c|}
224.224	 & 	13.0406	 & 	-2.54457	 \\
13.0406	 & 	1180.14	 & 	-1.05607	 \\
-2.54457	 & 	-1.05607	 & 	1363.89
\end{tabular}

\vtab
 EingenVectors - Molecule B     \\
\begin{tabular}{|c c c|}
-0.999905	 & 	0.0136356	 & 	-0.00221952	 \\
0.0136222	 & 	0.99989	 & 	0.00594071	 \\
-0.00230028	 & 	-0.00590991	 & 	0.99998
\end{tabular}

\vtab
 EingenValues - Molecule B     \\
\begin{tabular}{|c c c|}
224.041	 & 	1180.31	 & 	1363.9	 \\
\end{tabular}

\end{center}
\end{multicols}

\vtab[-5mm]
\begin{tabular}{*{2}{m{0.38\textwidth}}}
\begin{center}
\textcolor{NavyBlue}{\Large Different}
\end{center}
&
\begin{center}
\includegraphics[height=6.5cm]{../Comparisons/Vectors/inertia_tensor_of_S-R-S-Fructuose_out_G09_and_S-S-S-Fructuose_out_G09.png}
\end{center}
\end{tabular}

 \newpage

\vtab[-3cm]
\begin{center}
{\large Sugars \tab Número 633}
\end{center}
\begin{multicols}{2}
\begin{center}

Molecule A \
S-S-R-Fructuose\_out\_G09

\includegraphics[width=6cm]{../Comparisons/ImagesFromVMD/S-S-R-Fructuose_out_G09.png}

Inertia Tensor - Molecule A \\
\begin{tabular}{|c c c|}
257.995	 & 	-6.7346	 & 	10.0985	 \\
-6.7346	 & 	1154.74	 & 	-1.24098	 \\
10.0985	 & 	-1.24098	 & 	1276.66
\end{tabular}

\vtab
 EingenVectors - Molecule A     \\
\begin{tabular}{|c c c|}
0.999923	 & 	0.00749452	 & 	-0.00990206	 \\
0.00738758	 & 	-0.999914	 & 	-0.0107925	 \\
0.0099821	 & 	-0.0107185	 & 	0.999893
\end{tabular}

\vtab
 EingenValues - Molecule A     \\
\begin{tabular}{|c c c|}
257.845	 & 	1154.78	 & 	1276.78	 \\
\end{tabular}
\columnbreak

Molecule B \
S-S-S-Fructuose\_out\_G09

\includegraphics[width=6cm]{../Comparisons/ImagesFromVMD/S-S-S-Fructuose_out_G09.png}

Inertia Tensor - Molecule B \\
\begin{tabular}{|c c c|}
224.224	 & 	13.0406	 & 	-2.54457	 \\
13.0406	 & 	1180.14	 & 	-1.05607	 \\
-2.54457	 & 	-1.05607	 & 	1363.89
\end{tabular}

\vtab
 EingenVectors - Molecule B     \\
\begin{tabular}{|c c c|}
-0.999905	 & 	0.0136356	 & 	-0.00221952	 \\
0.0136222	 & 	0.99989	 & 	0.00594071	 \\
-0.00230028	 & 	-0.00590991	 & 	0.99998
\end{tabular}

\vtab
 EingenValues - Molecule B     \\
\begin{tabular}{|c c c|}
224.041	 & 	1180.31	 & 	1363.9	 \\
\end{tabular}

\end{center}
\end{multicols}

\vtab[-5mm]
\begin{tabular}{*{2}{m{0.38\textwidth}}}
\begin{center}
\textcolor{NavyBlue}{\Large Different}
\end{center}
&
\begin{center}
\includegraphics[height=6.5cm]{../Comparisons/Vectors/inertia_tensor_of_S-S-R-Fructuose_out_G09_and_S-S-S-Fructuose_out_G09.png}
\end{center}
\end{tabular}

 \newpage

\vtab[-3cm]
\begin{center}
{\large SymmetricEnantiomers \tab Número 634}
\end{center}
\begin{multicols}{2}
\begin{center}

Molecule A \
r-enan\_out\_G09

\includegraphics[width=6cm]{../Comparisons/ImagesFromVMD/r-enan_out_G09.png}

Inertia Tensor - Molecule A \\
\begin{tabular}{|c c c|}
103.745	 & 	1.51033	 & 	1.05456	 \\
1.51033	 & 	629.842	 & 	1.67376	 \\
1.05456	 & 	1.67376	 & 	722.189
\end{tabular}

\vtab
 EingenVectors - Molecule A     \\
\begin{tabular}{|c c c|}
-0.999994	 & 	0.00286537	 & 	0.0016974	 \\
-0.00283409	 & 	-0.999831	 & 	0.0181489	 \\
0.00174912	 & 	0.018144	 & 	0.999834
\end{tabular}

\vtab
 EingenValues - Molecule A     \\
\begin{tabular}{|c c c|}
103.739	 & 	629.816	 & 	722.221	 \\
\end{tabular}
\columnbreak

Molecule B \
r-enan\_out\_G09\_invertion\_output

\includegraphics[width=6cm]{../Comparisons/ImagesFromVMD/r-enan_out_G09_invertion_output.png}

Inertia Tensor - Molecule B \\
\begin{tabular}{|c c c|}
103.745	 & 	1.51041	 & 	1.05457	 \\
1.51041	 & 	629.841	 & 	1.67376	 \\
1.05457	 & 	1.67376	 & 	722.187
\end{tabular}

\vtab
 EingenVectors - Molecule B     \\
\begin{tabular}{|c c c|}
-0.999994	 & 	0.00286553	 & 	0.00169742	 \\
-0.00283425	 & 	-0.999831	 & 	0.018149	 \\
0.00174914	 & 	0.0181441	 & 	0.999834
\end{tabular}

\vtab
 EingenValues - Molecule B     \\
\begin{tabular}{|c c c|}
103.739	 & 	629.815	 & 	722.22	 \\
\end{tabular}

\end{center}
\end{multicols}

\vtab[-5mm]
\begin{tabular}{*{2}{m{0.38\textwidth}}}
\begin{center}
\textcolor{NavyBlue}{\Large Enantiomers}
\end{center}
&
\begin{center}
\includegraphics[height=6.5cm]{../Comparisons/Vectors/inertia_tensor_of_r-enan_out_G09_and_r-enan_out_G09_invertion_output.png}
\end{center}
\end{tabular}

 \newpage

\vtab[-3cm]
\begin{center}
{\large SymmetricEnantiomers \tab Número 635}
\end{center}
\begin{multicols}{2}
\begin{center}

Molecule A \
r-enan\_out\_G09

\includegraphics[width=6cm]{../Comparisons/ImagesFromVMD/r-enan_out_G09.png}

Inertia Tensor - Molecule A \\
\begin{tabular}{|c c c|}
103.745	 & 	1.51033	 & 	1.05456	 \\
1.51033	 & 	629.842	 & 	1.67376	 \\
1.05456	 & 	1.67376	 & 	722.189
\end{tabular}

\vtab
 EingenVectors - Molecule A     \\
\begin{tabular}{|c c c|}
-0.999994	 & 	0.00286537	 & 	0.0016974	 \\
-0.00283409	 & 	-0.999831	 & 	0.0181489	 \\
0.00174912	 & 	0.018144	 & 	0.999834
\end{tabular}

\vtab
 EingenValues - Molecule A     \\
\begin{tabular}{|c c c|}
103.739	 & 	629.816	 & 	722.221	 \\
\end{tabular}
\columnbreak

Molecule B \
r-enan\_rotated\_out\_G09

\includegraphics[width=6cm]{../Comparisons/ImagesFromVMD/r-enan_rotated_out_G09.png}

Inertia Tensor - Molecule B \\
\begin{tabular}{|c c c|}
103.745	 & 	1.51033	 & 	1.05456	 \\
1.51033	 & 	629.842	 & 	1.67376	 \\
1.05456	 & 	1.67376	 & 	722.189
\end{tabular}

\vtab
 EingenVectors - Molecule B     \\
\begin{tabular}{|c c c|}
-0.999994	 & 	0.00286537	 & 	0.0016974	 \\
-0.00283409	 & 	-0.999831	 & 	0.0181489	 \\
0.00174912	 & 	0.018144	 & 	0.999834
\end{tabular}

\vtab
 EingenValues - Molecule B     \\
\begin{tabular}{|c c c|}
103.739	 & 	629.816	 & 	722.221	 \\
\end{tabular}

\end{center}
\end{multicols}

\vtab[-5mm]
\begin{tabular}{*{2}{m{0.38\textwidth}}}
\begin{center}
\textcolor{NavyBlue}{\Large Equal}
\end{center}
&
\begin{center}
\includegraphics[height=6.5cm]{../Comparisons/Vectors/inertia_tensor_of_r-enan_out_G09_and_r-enan_rotated_out_G09.png}
\end{center}
\end{tabular}

 \newpage

\vtab[-3cm]
\begin{center}
{\large SymmetricEnantiomers \tab Número 636}
\end{center}
\begin{multicols}{2}
\begin{center}

Molecule A \
r-enan\_out\_G09

\includegraphics[width=6cm]{../Comparisons/ImagesFromVMD/r-enan_out_G09.png}

Inertia Tensor - Molecule A \\
\begin{tabular}{|c c c|}
103.745	 & 	1.51033	 & 	1.05456	 \\
1.51033	 & 	629.842	 & 	1.67376	 \\
1.05456	 & 	1.67376	 & 	722.189
\end{tabular}

\vtab
 EingenVectors - Molecule A     \\
\begin{tabular}{|c c c|}
-0.999994	 & 	0.00286537	 & 	0.0016974	 \\
-0.00283409	 & 	-0.999831	 & 	0.0181489	 \\
0.00174912	 & 	0.018144	 & 	0.999834
\end{tabular}

\vtab
 EingenValues - Molecule A     \\
\begin{tabular}{|c c c|}
103.739	 & 	629.816	 & 	722.221	 \\
\end{tabular}
\columnbreak

Molecule B \
s-enan\_out\_G09

\includegraphics[width=6cm]{../Comparisons/ImagesFromVMD/s-enan_out_G09.png}

Inertia Tensor - Molecule B \\
\begin{tabular}{|c c c|}
103.707	 & 	-1.51142	 & 	-1.05403	 \\
-1.51142	 & 	630.103	 & 	1.67204	 \\
-1.05403	 & 	1.67204	 & 	722.42
\end{tabular}

\vtab
 EingenVectors - Molecule B     \\
\begin{tabular}{|c c c|}
-0.999994	 & 	-0.00286582	 & 	-0.00169581	 \\
0.0028346	 & 	-0.999832	 & 	0.018136	 \\
-0.0017475	 & 	0.0181311	 & 	0.999834
\end{tabular}

\vtab
 EingenValues - Molecule B     \\
\begin{tabular}{|c c c|}
103.701	 & 	630.077	 & 	722.452	 \\
\end{tabular}

\end{center}
\end{multicols}

\vtab[-5mm]
\begin{tabular}{*{2}{m{0.38\textwidth}}}
\begin{center}
\textcolor{NavyBlue}{\Large Enantiomers}
\end{center}
&
\begin{center}
\includegraphics[height=6.5cm]{../Comparisons/Vectors/inertia_tensor_of_r-enan_out_G09_and_s-enan_out_G09.png}
\end{center}
\end{tabular}

 \newpage

\vtab[-3cm]
\begin{center}
{\large SymmetricEnantiomers \tab Número 637}
\end{center}
\begin{multicols}{2}
\begin{center}

Molecule A \
r-enan\_out\_G09

\includegraphics[width=6cm]{../Comparisons/ImagesFromVMD/r-enan_out_G09.png}

Inertia Tensor - Molecule A \\
\begin{tabular}{|c c c|}
103.745	 & 	1.51033	 & 	1.05456	 \\
1.51033	 & 	629.842	 & 	1.67376	 \\
1.05456	 & 	1.67376	 & 	722.189
\end{tabular}

\vtab
 EingenVectors - Molecule A     \\
\begin{tabular}{|c c c|}
-0.999994	 & 	0.00286537	 & 	0.0016974	 \\
-0.00283409	 & 	-0.999831	 & 	0.0181489	 \\
0.00174912	 & 	0.018144	 & 	0.999834
\end{tabular}

\vtab
 EingenValues - Molecule A     \\
\begin{tabular}{|c c c|}
103.739	 & 	629.816	 & 	722.221	 \\
\end{tabular}
\columnbreak

Molecule B \
s-enan\_out\_G09\_invertion\_output

\includegraphics[width=6cm]{../Comparisons/ImagesFromVMD/s-enan_out_G09_invertion_output.png}

Inertia Tensor - Molecule B \\
\begin{tabular}{|c c c|}
103.707	 & 	-1.51147	 & 	-1.05403	 \\
-1.51147	 & 	630.105	 & 	1.67201	 \\
-1.05403	 & 	1.67201	 & 	722.422
\end{tabular}

\vtab
 EingenVectors - Molecule B     \\
\begin{tabular}{|c c c|}
-0.999994	 & 	-0.0028659	 & 	-0.0016958	 \\
0.00283468	 & 	-0.999832	 & 	0.0181359	 \\
-0.00174749	 & 	0.018131	 & 	0.999834
\end{tabular}

\vtab
 EingenValues - Molecule B     \\
\begin{tabular}{|c c c|}
103.7	 & 	630.079	 & 	722.455	 \\
\end{tabular}

\end{center}
\end{multicols}

\vtab[-5mm]
\begin{tabular}{*{2}{m{0.38\textwidth}}}
\begin{center}
\textcolor{NavyBlue}{\Large Equal}
\end{center}
&
\begin{center}
\includegraphics[height=6.5cm]{../Comparisons/Vectors/inertia_tensor_of_r-enan_out_G09_and_s-enan_out_G09_invertion_output.png}
\end{center}
\end{tabular}

 \newpage

\vtab[-3cm]
\begin{center}
{\large SymmetricEnantiomers \tab Número 638}
\end{center}
\begin{multicols}{2}
\begin{center}

Molecule A \
r-enan\_out\_G09

\includegraphics[width=6cm]{../Comparisons/ImagesFromVMD/r-enan_out_G09.png}

Inertia Tensor - Molecule A \\
\begin{tabular}{|c c c|}
103.745	 & 	1.51033	 & 	1.05456	 \\
1.51033	 & 	629.842	 & 	1.67376	 \\
1.05456	 & 	1.67376	 & 	722.189
\end{tabular}

\vtab
 EingenVectors - Molecule A     \\
\begin{tabular}{|c c c|}
-0.999994	 & 	0.00286537	 & 	0.0016974	 \\
-0.00283409	 & 	-0.999831	 & 	0.0181489	 \\
0.00174912	 & 	0.018144	 & 	0.999834
\end{tabular}

\vtab
 EingenValues - Molecule A     \\
\begin{tabular}{|c c c|}
103.739	 & 	629.816	 & 	722.221	 \\
\end{tabular}
\columnbreak

Molecule B \
s-enan\_rotated\_out\_G09

\includegraphics[width=6cm]{../Comparisons/ImagesFromVMD/s-enan_rotated_out_G09.png}

Inertia Tensor - Molecule B \\
\begin{tabular}{|c c c|}
103.707	 & 	-1.51138	 & 	-1.05403	 \\
-1.51138	 & 	630.103	 & 	1.67204	 \\
-1.05403	 & 	1.67204	 & 	722.42
\end{tabular}

\vtab
 EingenVectors - Molecule B     \\
\begin{tabular}{|c c c|}
-0.999994	 & 	-0.00286575	 & 	-0.00169582	 \\
0.00283453	 & 	-0.999832	 & 	0.018136	 \\
-0.0017475	 & 	0.0181311	 & 	0.999834
\end{tabular}

\vtab
 EingenValues - Molecule B     \\
\begin{tabular}{|c c c|}
103.701	 & 	630.077	 & 	722.452	 \\
\end{tabular}

\end{center}
\end{multicols}

\vtab[-5mm]
\begin{tabular}{*{2}{m{0.38\textwidth}}}
\begin{center}
\textcolor{NavyBlue}{\Large Enantiomers}
\end{center}
&
\begin{center}
\includegraphics[height=6.5cm]{../Comparisons/Vectors/inertia_tensor_of_r-enan_out_G09_and_s-enan_rotated_out_G09.png}
\end{center}
\end{tabular}

 \newpage

\vtab[-3cm]
\begin{center}
{\large SymmetricEnantiomers \tab Número 639}
\end{center}
\begin{multicols}{2}
\begin{center}

Molecule A \
r-enan\_out\_G09\_invertion\_output

\includegraphics[width=6cm]{../Comparisons/ImagesFromVMD/r-enan_out_G09_invertion_output.png}

Inertia Tensor - Molecule A \\
\begin{tabular}{|c c c|}
103.745	 & 	1.51041	 & 	1.05457	 \\
1.51041	 & 	629.841	 & 	1.67376	 \\
1.05457	 & 	1.67376	 & 	722.187
\end{tabular}

\vtab
 EingenVectors - Molecule A     \\
\begin{tabular}{|c c c|}
-0.999994	 & 	0.00286553	 & 	0.00169742	 \\
-0.00283425	 & 	-0.999831	 & 	0.018149	 \\
0.00174914	 & 	0.0181441	 & 	0.999834
\end{tabular}

\vtab
 EingenValues - Molecule A     \\
\begin{tabular}{|c c c|}
103.739	 & 	629.815	 & 	722.22	 \\
\end{tabular}
\columnbreak

Molecule B \
r-enan\_rotated\_out\_G09

\includegraphics[width=6cm]{../Comparisons/ImagesFromVMD/r-enan_rotated_out_G09.png}

Inertia Tensor - Molecule B \\
\begin{tabular}{|c c c|}
103.745	 & 	1.51033	 & 	1.05456	 \\
1.51033	 & 	629.842	 & 	1.67376	 \\
1.05456	 & 	1.67376	 & 	722.189
\end{tabular}

\vtab
 EingenVectors - Molecule B     \\
\begin{tabular}{|c c c|}
-0.999994	 & 	0.00286537	 & 	0.0016974	 \\
-0.00283409	 & 	-0.999831	 & 	0.0181489	 \\
0.00174912	 & 	0.018144	 & 	0.999834
\end{tabular}

\vtab
 EingenValues - Molecule B     \\
\begin{tabular}{|c c c|}
103.739	 & 	629.816	 & 	722.221	 \\
\end{tabular}

\end{center}
\end{multicols}

\vtab[-5mm]
\begin{tabular}{*{2}{m{0.38\textwidth}}}
\begin{center}
\textcolor{NavyBlue}{\Large Enantiomers}
\end{center}
&
\begin{center}
\includegraphics[height=6.5cm]{../Comparisons/Vectors/inertia_tensor_of_r-enan_out_G09_invertion_output_and_r-enan_rotated_out_G09.png}
\end{center}
\end{tabular}

 \newpage

\vtab[-3cm]
\begin{center}
{\large SymmetricEnantiomers \tab Número 640}
\end{center}
\begin{multicols}{2}
\begin{center}

Molecule A \
r-enan\_out\_G09\_invertion\_output

\includegraphics[width=6cm]{../Comparisons/ImagesFromVMD/r-enan_out_G09_invertion_output.png}

Inertia Tensor - Molecule A \\
\begin{tabular}{|c c c|}
103.745	 & 	1.51041	 & 	1.05457	 \\
1.51041	 & 	629.841	 & 	1.67376	 \\
1.05457	 & 	1.67376	 & 	722.187
\end{tabular}

\vtab
 EingenVectors - Molecule A     \\
\begin{tabular}{|c c c|}
-0.999994	 & 	0.00286553	 & 	0.00169742	 \\
-0.00283425	 & 	-0.999831	 & 	0.018149	 \\
0.00174914	 & 	0.0181441	 & 	0.999834
\end{tabular}

\vtab
 EingenValues - Molecule A     \\
\begin{tabular}{|c c c|}
103.739	 & 	629.815	 & 	722.22	 \\
\end{tabular}
\columnbreak

Molecule B \
s-enan\_out\_G09

\includegraphics[width=6cm]{../Comparisons/ImagesFromVMD/s-enan_out_G09.png}

Inertia Tensor - Molecule B \\
\begin{tabular}{|c c c|}
103.707	 & 	-1.51142	 & 	-1.05403	 \\
-1.51142	 & 	630.103	 & 	1.67204	 \\
-1.05403	 & 	1.67204	 & 	722.42
\end{tabular}

\vtab
 EingenVectors - Molecule B     \\
\begin{tabular}{|c c c|}
-0.999994	 & 	-0.00286582	 & 	-0.00169581	 \\
0.0028346	 & 	-0.999832	 & 	0.018136	 \\
-0.0017475	 & 	0.0181311	 & 	0.999834
\end{tabular}

\vtab
 EingenValues - Molecule B     \\
\begin{tabular}{|c c c|}
103.701	 & 	630.077	 & 	722.452	 \\
\end{tabular}

\end{center}
\end{multicols}

\vtab[-5mm]
\begin{tabular}{*{2}{m{0.38\textwidth}}}
\begin{center}
\textcolor{NavyBlue}{\Large Equal}
\end{center}
&
\begin{center}
\includegraphics[height=6.5cm]{../Comparisons/Vectors/inertia_tensor_of_r-enan_out_G09_invertion_output_and_s-enan_out_G09.png}
\end{center}
\end{tabular}

 \newpage

\vtab[-3cm]
\begin{center}
{\large SymmetricEnantiomers \tab Número 641}
\end{center}
\begin{multicols}{2}
\begin{center}

Molecule A \
r-enan\_out\_G09\_invertion\_output

\includegraphics[width=6cm]{../Comparisons/ImagesFromVMD/r-enan_out_G09_invertion_output.png}

Inertia Tensor - Molecule A \\
\begin{tabular}{|c c c|}
103.745	 & 	1.51041	 & 	1.05457	 \\
1.51041	 & 	629.841	 & 	1.67376	 \\
1.05457	 & 	1.67376	 & 	722.187
\end{tabular}

\vtab
 EingenVectors - Molecule A     \\
\begin{tabular}{|c c c|}
-0.999994	 & 	0.00286553	 & 	0.00169742	 \\
-0.00283425	 & 	-0.999831	 & 	0.018149	 \\
0.00174914	 & 	0.0181441	 & 	0.999834
\end{tabular}

\vtab
 EingenValues - Molecule A     \\
\begin{tabular}{|c c c|}
103.739	 & 	629.815	 & 	722.22	 \\
\end{tabular}
\columnbreak

Molecule B \
s-enan\_out\_G09\_invertion\_output

\includegraphics[width=6cm]{../Comparisons/ImagesFromVMD/s-enan_out_G09_invertion_output.png}

Inertia Tensor - Molecule B \\
\begin{tabular}{|c c c|}
103.707	 & 	-1.51147	 & 	-1.05403	 \\
-1.51147	 & 	630.105	 & 	1.67201	 \\
-1.05403	 & 	1.67201	 & 	722.422
\end{tabular}

\vtab
 EingenVectors - Molecule B     \\
\begin{tabular}{|c c c|}
-0.999994	 & 	-0.0028659	 & 	-0.0016958	 \\
0.00283468	 & 	-0.999832	 & 	0.0181359	 \\
-0.00174749	 & 	0.018131	 & 	0.999834
\end{tabular}

\vtab
 EingenValues - Molecule B     \\
\begin{tabular}{|c c c|}
103.7	 & 	630.079	 & 	722.455	 \\
\end{tabular}

\end{center}
\end{multicols}

\vtab[-5mm]
\begin{tabular}{*{2}{m{0.38\textwidth}}}
\begin{center}
\textcolor{NavyBlue}{\Large Enantiomers}
\end{center}
&
\begin{center}
\includegraphics[height=6.5cm]{../Comparisons/Vectors/inertia_tensor_of_r-enan_out_G09_invertion_output_and_s-enan_out_G09_invertion_output.png}
\end{center}
\end{tabular}

 \newpage

\vtab[-3cm]
\begin{center}
{\large SymmetricEnantiomers \tab Número 642}
\end{center}
\begin{multicols}{2}
\begin{center}

Molecule A \
r-enan\_out\_G09\_invertion\_output

\includegraphics[width=6cm]{../Comparisons/ImagesFromVMD/r-enan_out_G09_invertion_output.png}

Inertia Tensor - Molecule A \\
\begin{tabular}{|c c c|}
103.745	 & 	1.51041	 & 	1.05457	 \\
1.51041	 & 	629.841	 & 	1.67376	 \\
1.05457	 & 	1.67376	 & 	722.187
\end{tabular}

\vtab
 EingenVectors - Molecule A     \\
\begin{tabular}{|c c c|}
-0.999994	 & 	0.00286553	 & 	0.00169742	 \\
-0.00283425	 & 	-0.999831	 & 	0.018149	 \\
0.00174914	 & 	0.0181441	 & 	0.999834
\end{tabular}

\vtab
 EingenValues - Molecule A     \\
\begin{tabular}{|c c c|}
103.739	 & 	629.815	 & 	722.22	 \\
\end{tabular}
\columnbreak

Molecule B \
s-enan\_rotated\_out\_G09

\includegraphics[width=6cm]{../Comparisons/ImagesFromVMD/s-enan_rotated_out_G09.png}

Inertia Tensor - Molecule B \\
\begin{tabular}{|c c c|}
103.707	 & 	-1.51138	 & 	-1.05403	 \\
-1.51138	 & 	630.103	 & 	1.67204	 \\
-1.05403	 & 	1.67204	 & 	722.42
\end{tabular}

\vtab
 EingenVectors - Molecule B     \\
\begin{tabular}{|c c c|}
-0.999994	 & 	-0.00286575	 & 	-0.00169582	 \\
0.00283453	 & 	-0.999832	 & 	0.018136	 \\
-0.0017475	 & 	0.0181311	 & 	0.999834
\end{tabular}

\vtab
 EingenValues - Molecule B     \\
\begin{tabular}{|c c c|}
103.701	 & 	630.077	 & 	722.452	 \\
\end{tabular}

\end{center}
\end{multicols}

\vtab[-5mm]
\begin{tabular}{*{2}{m{0.38\textwidth}}}
\begin{center}
\textcolor{NavyBlue}{\Large Equal}
\end{center}
&
\begin{center}
\includegraphics[height=6.5cm]{../Comparisons/Vectors/inertia_tensor_of_r-enan_out_G09_invertion_output_and_s-enan_rotated_out_G09.png}
\end{center}
\end{tabular}

 \newpage

\vtab[-3cm]
\begin{center}
{\large SymmetricEnantiomers \tab Número 643}
\end{center}
\begin{multicols}{2}
\begin{center}

Molecule A \
r-enan\_rotated\_out\_G09

\includegraphics[width=6cm]{../Comparisons/ImagesFromVMD/r-enan_rotated_out_G09.png}

Inertia Tensor - Molecule A \\
\begin{tabular}{|c c c|}
103.745	 & 	1.51033	 & 	1.05456	 \\
1.51033	 & 	629.842	 & 	1.67376	 \\
1.05456	 & 	1.67376	 & 	722.189
\end{tabular}

\vtab
 EingenVectors - Molecule A     \\
\begin{tabular}{|c c c|}
-0.999994	 & 	0.00286537	 & 	0.0016974	 \\
-0.00283409	 & 	-0.999831	 & 	0.0181489	 \\
0.00174912	 & 	0.018144	 & 	0.999834
\end{tabular}

\vtab
 EingenValues - Molecule A     \\
\begin{tabular}{|c c c|}
103.739	 & 	629.816	 & 	722.221	 \\
\end{tabular}
\columnbreak

Molecule B \
s-enan\_out\_G09

\includegraphics[width=6cm]{../Comparisons/ImagesFromVMD/s-enan_out_G09.png}

Inertia Tensor - Molecule B \\
\begin{tabular}{|c c c|}
103.707	 & 	-1.51142	 & 	-1.05403	 \\
-1.51142	 & 	630.103	 & 	1.67204	 \\
-1.05403	 & 	1.67204	 & 	722.42
\end{tabular}

\vtab
 EingenVectors - Molecule B     \\
\begin{tabular}{|c c c|}
-0.999994	 & 	-0.00286582	 & 	-0.00169581	 \\
0.0028346	 & 	-0.999832	 & 	0.018136	 \\
-0.0017475	 & 	0.0181311	 & 	0.999834
\end{tabular}

\vtab
 EingenValues - Molecule B     \\
\begin{tabular}{|c c c|}
103.701	 & 	630.077	 & 	722.452	 \\
\end{tabular}

\end{center}
\end{multicols}

\vtab[-5mm]
\begin{tabular}{*{2}{m{0.38\textwidth}}}
\begin{center}
\textcolor{NavyBlue}{\Large Enantiomers}
\end{center}
&
\begin{center}
\includegraphics[height=6.5cm]{../Comparisons/Vectors/inertia_tensor_of_r-enan_rotated_out_G09_and_s-enan_out_G09.png}
\end{center}
\end{tabular}

 \newpage

\vtab[-3cm]
\begin{center}
{\large SymmetricEnantiomers \tab Número 644}
\end{center}
\begin{multicols}{2}
\begin{center}

Molecule A \
r-enan\_rotated\_out\_G09

\includegraphics[width=6cm]{../Comparisons/ImagesFromVMD/r-enan_rotated_out_G09.png}

Inertia Tensor - Molecule A \\
\begin{tabular}{|c c c|}
103.745	 & 	1.51033	 & 	1.05456	 \\
1.51033	 & 	629.842	 & 	1.67376	 \\
1.05456	 & 	1.67376	 & 	722.189
\end{tabular}

\vtab
 EingenVectors - Molecule A     \\
\begin{tabular}{|c c c|}
-0.999994	 & 	0.00286537	 & 	0.0016974	 \\
-0.00283409	 & 	-0.999831	 & 	0.0181489	 \\
0.00174912	 & 	0.018144	 & 	0.999834
\end{tabular}

\vtab
 EingenValues - Molecule A     \\
\begin{tabular}{|c c c|}
103.739	 & 	629.816	 & 	722.221	 \\
\end{tabular}
\columnbreak

Molecule B \
s-enan\_out\_G09\_invertion\_output

\includegraphics[width=6cm]{../Comparisons/ImagesFromVMD/s-enan_out_G09_invertion_output.png}

Inertia Tensor - Molecule B \\
\begin{tabular}{|c c c|}
103.707	 & 	-1.51147	 & 	-1.05403	 \\
-1.51147	 & 	630.105	 & 	1.67201	 \\
-1.05403	 & 	1.67201	 & 	722.422
\end{tabular}

\vtab
 EingenVectors - Molecule B     \\
\begin{tabular}{|c c c|}
-0.999994	 & 	-0.0028659	 & 	-0.0016958	 \\
0.00283468	 & 	-0.999832	 & 	0.0181359	 \\
-0.00174749	 & 	0.018131	 & 	0.999834
\end{tabular}

\vtab
 EingenValues - Molecule B     \\
\begin{tabular}{|c c c|}
103.7	 & 	630.079	 & 	722.455	 \\
\end{tabular}

\end{center}
\end{multicols}

\vtab[-5mm]
\begin{tabular}{*{2}{m{0.38\textwidth}}}
\begin{center}
\textcolor{NavyBlue}{\Large Equal}
\end{center}
&
\begin{center}
\includegraphics[height=6.5cm]{../Comparisons/Vectors/inertia_tensor_of_r-enan_rotated_out_G09_and_s-enan_out_G09_invertion_output.png}
\end{center}
\end{tabular}

 \newpage

\vtab[-3cm]
\begin{center}
{\large SymmetricEnantiomers \tab Número 645}
\end{center}
\begin{multicols}{2}
\begin{center}

Molecule A \
r-enan\_rotated\_out\_G09

\includegraphics[width=6cm]{../Comparisons/ImagesFromVMD/r-enan_rotated_out_G09.png}

Inertia Tensor - Molecule A \\
\begin{tabular}{|c c c|}
103.745	 & 	1.51033	 & 	1.05456	 \\
1.51033	 & 	629.842	 & 	1.67376	 \\
1.05456	 & 	1.67376	 & 	722.189
\end{tabular}

\vtab
 EingenVectors - Molecule A     \\
\begin{tabular}{|c c c|}
-0.999994	 & 	0.00286537	 & 	0.0016974	 \\
-0.00283409	 & 	-0.999831	 & 	0.0181489	 \\
0.00174912	 & 	0.018144	 & 	0.999834
\end{tabular}

\vtab
 EingenValues - Molecule A     \\
\begin{tabular}{|c c c|}
103.739	 & 	629.816	 & 	722.221	 \\
\end{tabular}
\columnbreak

Molecule B \
s-enan\_rotated\_out\_G09

\includegraphics[width=6cm]{../Comparisons/ImagesFromVMD/s-enan_rotated_out_G09.png}

Inertia Tensor - Molecule B \\
\begin{tabular}{|c c c|}
103.707	 & 	-1.51138	 & 	-1.05403	 \\
-1.51138	 & 	630.103	 & 	1.67204	 \\
-1.05403	 & 	1.67204	 & 	722.42
\end{tabular}

\vtab
 EingenVectors - Molecule B     \\
\begin{tabular}{|c c c|}
-0.999994	 & 	-0.00286575	 & 	-0.00169582	 \\
0.00283453	 & 	-0.999832	 & 	0.018136	 \\
-0.0017475	 & 	0.0181311	 & 	0.999834
\end{tabular}

\vtab
 EingenValues - Molecule B     \\
\begin{tabular}{|c c c|}
103.701	 & 	630.077	 & 	722.452	 \\
\end{tabular}

\end{center}
\end{multicols}

\vtab[-5mm]
\begin{tabular}{*{2}{m{0.38\textwidth}}}
\begin{center}
\textcolor{NavyBlue}{\Large Enantiomers}
\end{center}
&
\begin{center}
\includegraphics[height=6.5cm]{../Comparisons/Vectors/inertia_tensor_of_r-enan_rotated_out_G09_and_s-enan_rotated_out_G09.png}
\end{center}
\end{tabular}

 \newpage

\vtab[-3cm]
\begin{center}
{\large SymmetricEnantiomers \tab Número 646}
\end{center}
\begin{multicols}{2}
\begin{center}

Molecule A \
s-enan\_out\_G09

\includegraphics[width=6cm]{../Comparisons/ImagesFromVMD/s-enan_out_G09.png}

Inertia Tensor - Molecule A \\
\begin{tabular}{|c c c|}
103.707	 & 	-1.51142	 & 	-1.05403	 \\
-1.51142	 & 	630.103	 & 	1.67204	 \\
-1.05403	 & 	1.67204	 & 	722.42
\end{tabular}

\vtab
 EingenVectors - Molecule A     \\
\begin{tabular}{|c c c|}
-0.999994	 & 	-0.00286582	 & 	-0.00169581	 \\
0.0028346	 & 	-0.999832	 & 	0.018136	 \\
-0.0017475	 & 	0.0181311	 & 	0.999834
\end{tabular}

\vtab
 EingenValues - Molecule A     \\
\begin{tabular}{|c c c|}
103.701	 & 	630.077	 & 	722.452	 \\
\end{tabular}
\columnbreak

Molecule B \
s-enan\_out\_G09\_invertion\_output

\includegraphics[width=6cm]{../Comparisons/ImagesFromVMD/s-enan_out_G09_invertion_output.png}

Inertia Tensor - Molecule B \\
\begin{tabular}{|c c c|}
103.707	 & 	-1.51147	 & 	-1.05403	 \\
-1.51147	 & 	630.105	 & 	1.67201	 \\
-1.05403	 & 	1.67201	 & 	722.422
\end{tabular}

\vtab
 EingenVectors - Molecule B     \\
\begin{tabular}{|c c c|}
-0.999994	 & 	-0.0028659	 & 	-0.0016958	 \\
0.00283468	 & 	-0.999832	 & 	0.0181359	 \\
-0.00174749	 & 	0.018131	 & 	0.999834
\end{tabular}

\vtab
 EingenValues - Molecule B     \\
\begin{tabular}{|c c c|}
103.7	 & 	630.079	 & 	722.455	 \\
\end{tabular}

\end{center}
\end{multicols}

\vtab[-5mm]
\begin{tabular}{*{2}{m{0.38\textwidth}}}
\begin{center}
\textcolor{NavyBlue}{\Large Enantiomers}
\end{center}
&
\begin{center}
\includegraphics[height=6.5cm]{../Comparisons/Vectors/inertia_tensor_of_s-enan_out_G09_and_s-enan_out_G09_invertion_output.png}
\end{center}
\end{tabular}

 \newpage

\vtab[-3cm]
\begin{center}
{\large SymmetricEnantiomers \tab Número 647}
\end{center}
\begin{multicols}{2}
\begin{center}

Molecule A \
s-enan\_out\_G09

\includegraphics[width=6cm]{../Comparisons/ImagesFromVMD/s-enan_out_G09.png}

Inertia Tensor - Molecule A \\
\begin{tabular}{|c c c|}
103.707	 & 	-1.51142	 & 	-1.05403	 \\
-1.51142	 & 	630.103	 & 	1.67204	 \\
-1.05403	 & 	1.67204	 & 	722.42
\end{tabular}

\vtab
 EingenVectors - Molecule A     \\
\begin{tabular}{|c c c|}
-0.999994	 & 	-0.00286582	 & 	-0.00169581	 \\
0.0028346	 & 	-0.999832	 & 	0.018136	 \\
-0.0017475	 & 	0.0181311	 & 	0.999834
\end{tabular}

\vtab
 EingenValues - Molecule A     \\
\begin{tabular}{|c c c|}
103.701	 & 	630.077	 & 	722.452	 \\
\end{tabular}
\columnbreak

Molecule B \
s-enan\_rotated\_out\_G09

\includegraphics[width=6cm]{../Comparisons/ImagesFromVMD/s-enan_rotated_out_G09.png}

Inertia Tensor - Molecule B \\
\begin{tabular}{|c c c|}
103.707	 & 	-1.51138	 & 	-1.05403	 \\
-1.51138	 & 	630.103	 & 	1.67204	 \\
-1.05403	 & 	1.67204	 & 	722.42
\end{tabular}

\vtab
 EingenVectors - Molecule B     \\
\begin{tabular}{|c c c|}
-0.999994	 & 	-0.00286575	 & 	-0.00169582	 \\
0.00283453	 & 	-0.999832	 & 	0.018136	 \\
-0.0017475	 & 	0.0181311	 & 	0.999834
\end{tabular}

\vtab
 EingenValues - Molecule B     \\
\begin{tabular}{|c c c|}
103.701	 & 	630.077	 & 	722.452	 \\
\end{tabular}

\end{center}
\end{multicols}

\vtab[-5mm]
\begin{tabular}{*{2}{m{0.38\textwidth}}}
\begin{center}
\textcolor{NavyBlue}{\Large Equal}
\end{center}
&
\begin{center}
\includegraphics[height=6.5cm]{../Comparisons/Vectors/inertia_tensor_of_s-enan_out_G09_and_s-enan_rotated_out_G09.png}
\end{center}
\end{tabular}

 \newpage

\vtab[-3cm]
\begin{center}
{\large SymmetricEnantiomers \tab Número 648}
\end{center}
\begin{multicols}{2}
\begin{center}

Molecule A \
s-enan\_out\_G09\_invertion\_output

\includegraphics[width=6cm]{../Comparisons/ImagesFromVMD/s-enan_out_G09_invertion_output.png}

Inertia Tensor - Molecule A \\
\begin{tabular}{|c c c|}
103.707	 & 	-1.51147	 & 	-1.05403	 \\
-1.51147	 & 	630.105	 & 	1.67201	 \\
-1.05403	 & 	1.67201	 & 	722.422
\end{tabular}

\vtab
 EingenVectors - Molecule A     \\
\begin{tabular}{|c c c|}
-0.999994	 & 	-0.0028659	 & 	-0.0016958	 \\
0.00283468	 & 	-0.999832	 & 	0.0181359	 \\
-0.00174749	 & 	0.018131	 & 	0.999834
\end{tabular}

\vtab
 EingenValues - Molecule A     \\
\begin{tabular}{|c c c|}
103.7	 & 	630.079	 & 	722.455	 \\
\end{tabular}
\columnbreak

Molecule B \
s-enan\_rotated\_out\_G09

\includegraphics[width=6cm]{../Comparisons/ImagesFromVMD/s-enan_rotated_out_G09.png}

Inertia Tensor - Molecule B \\
\begin{tabular}{|c c c|}
103.707	 & 	-1.51138	 & 	-1.05403	 \\
-1.51138	 & 	630.103	 & 	1.67204	 \\
-1.05403	 & 	1.67204	 & 	722.42
\end{tabular}

\vtab
 EingenVectors - Molecule B     \\
\begin{tabular}{|c c c|}
-0.999994	 & 	-0.00286575	 & 	-0.00169582	 \\
0.00283453	 & 	-0.999832	 & 	0.018136	 \\
-0.0017475	 & 	0.0181311	 & 	0.999834
\end{tabular}

\vtab
 EingenValues - Molecule B     \\
\begin{tabular}{|c c c|}
103.701	 & 	630.077	 & 	722.452	 \\
\end{tabular}

\end{center}
\end{multicols}

\vtab[-5mm]
\begin{tabular}{*{2}{m{0.38\textwidth}}}
\begin{center}
\textcolor{NavyBlue}{\Large Enantiomers}
\end{center}
&
\begin{center}
\includegraphics[height=6.5cm]{../Comparisons/Vectors/inertia_tensor_of_s-enan_out_G09_invertion_output_and_s-enan_rotated_out_G09.png}
\end{center}
\end{tabular}

 \newpage

\vtab[-3cm]
\begin{center}
{\large TesisExample \tab Número 649}
\end{center}
\begin{multicols}{2}
\begin{center}

Molecule A \
test-R\_out\_G09

\includegraphics[width=6cm]{../Comparisons/ImagesFromVMD/test-R_out_G09.png}

Inertia Tensor - Molecule A \\
\begin{tabular}{|c c c|}
146.29	 & 	5.10986	 & 	-3.31731	 \\
5.10986	 & 	442.839	 & 	1.31633	 \\
-3.31731	 & 	1.31633	 & 	530.688
\end{tabular}

\vtab
 EingenVectors - Molecule A     \\
\begin{tabular}{|c c c|}
0.999813	 & 	-0.0172596	 & 	0.00868475	 \\
-0.017383	 & 	-0.999746	 & 	0.0143352	 \\
-0.00843513	 & 	0.0144835	 & 	0.99986
\end{tabular}

\vtab
 EingenValues - Molecule A     \\
\begin{tabular}{|c c c|}
146.173	 & 	442.909	 & 	530.735	 \\
\end{tabular}
\columnbreak

Molecule B \
test-R\_out\_G09\_invertion

\includegraphics[width=6cm]{../Comparisons/ImagesFromVMD/test-R_out_G09_invertion.png}

Inertia Tensor - Molecule B \\
\begin{tabular}{|c c c|}
146.29	 & 	5.10969	 & 	-3.31735	 \\
5.10969	 & 	442.839	 & 	1.31631	 \\
-3.31735	 & 	1.31631	 & 	530.688
\end{tabular}

\vtab
 EingenVectors - Molecule B     \\
\begin{tabular}{|c c c|}
0.999813	 & 	-0.017259	 & 	0.00868483	 \\
-0.0173824	 & 	-0.999746	 & 	0.014335	 \\
-0.00843522	 & 	0.0144833	 & 	0.99986
\end{tabular}

\vtab
 EingenValues - Molecule B     \\
\begin{tabular}{|c c c|}
146.173	 & 	442.909	 & 	530.735	 \\
\end{tabular}

\end{center}
\end{multicols}

\vtab[-5mm]
\begin{tabular}{*{2}{m{0.38\textwidth}}}
\begin{center}
\textcolor{NavyBlue}{\Large Enantiomers}
\end{center}
&
\begin{center}
\includegraphics[height=6.5cm]{../Comparisons/Vectors/inertia_tensor_of_test-R_out_G09_and_test-R_out_G09_invertion.png}
\end{center}
\end{tabular}

 \newpage

\vtab[-3cm]
\begin{center}
{\large TesisExample \tab Número 650}
\end{center}
\begin{multicols}{2}
\begin{center}

Molecule A \
test-R\_out\_G09

\includegraphics[width=6cm]{../Comparisons/ImagesFromVMD/test-R_out_G09.png}

Inertia Tensor - Molecule A \\
\begin{tabular}{|c c c|}
146.29	 & 	5.10986	 & 	-3.31731	 \\
5.10986	 & 	442.839	 & 	1.31633	 \\
-3.31731	 & 	1.31633	 & 	530.688
\end{tabular}

\vtab
 EingenVectors - Molecule A     \\
\begin{tabular}{|c c c|}
0.999813	 & 	-0.0172596	 & 	0.00868475	 \\
-0.017383	 & 	-0.999746	 & 	0.0143352	 \\
-0.00843513	 & 	0.0144835	 & 	0.99986
\end{tabular}

\vtab
 EingenValues - Molecule A     \\
\begin{tabular}{|c c c|}
146.173	 & 	442.909	 & 	530.735	 \\
\end{tabular}
\columnbreak

Molecule B \
test-R\_out\_G09\_rot-45-45-45

\includegraphics[width=6cm]{../Comparisons/ImagesFromVMD/test-R_out_G09_rot-45-45-45.png}

Inertia Tensor - Molecule B \\
\begin{tabular}{|c c c|}
430.951	 & 	89.0849	 & 	141.239	 \\
89.0849	 & 	367.04	 & 	-98.5816	 \\
141.239	 & 	-98.5816	 & 	321.828
\end{tabular}

\vtab
 EingenVectors - Molecule B     \\
\begin{tabular}{|c c c|}
0.504792	 & 	-0.513366	 & 	-0.694003	 \\
0.142865	 & 	0.842546	 & 	-0.519332	 \\
-0.851337	 & 	-0.163006	 & 	-0.498652
\end{tabular}

\vtab
 EingenValues - Molecule B     \\
\begin{tabular}{|c c c|}
146.173	 & 	442.909	 & 	530.735	 \\
\end{tabular}

\end{center}
\end{multicols}

\vtab[-5mm]
\begin{tabular}{*{2}{m{0.38\textwidth}}}
\begin{center}
\textcolor{NavyBlue}{\Large Equal}
\end{center}
&
\begin{center}
\includegraphics[height=6.5cm]{../Comparisons/Vectors/inertia_tensor_of_test-R_out_G09_and_test-R_out_G09_rot-45-45-45.png}
\end{center}
\end{tabular}

 \newpage

\vtab[-3cm]
\begin{center}
{\large TesisExample \tab Número 651}
\end{center}
\begin{multicols}{2}
\begin{center}

Molecule A \
test-R\_out\_G09

\includegraphics[width=6cm]{../Comparisons/ImagesFromVMD/test-R_out_G09.png}

Inertia Tensor - Molecule A \\
\begin{tabular}{|c c c|}
146.29	 & 	5.10986	 & 	-3.31731	 \\
5.10986	 & 	442.839	 & 	1.31633	 \\
-3.31731	 & 	1.31633	 & 	530.688
\end{tabular}

\vtab
 EingenVectors - Molecule A     \\
\begin{tabular}{|c c c|}
0.999813	 & 	-0.0172596	 & 	0.00868475	 \\
-0.017383	 & 	-0.999746	 & 	0.0143352	 \\
-0.00843513	 & 	0.0144835	 & 	0.99986
\end{tabular}

\vtab
 EingenValues - Molecule A     \\
\begin{tabular}{|c c c|}
146.173	 & 	442.909	 & 	530.735	 \\
\end{tabular}
\columnbreak

Molecule B \
test-S\_out\_G09

\includegraphics[width=6cm]{../Comparisons/ImagesFromVMD/test-S_out_G09.png}

Inertia Tensor - Molecule B \\
\begin{tabular}{|c c c|}
146.29	 & 	-5.10986	 & 	3.31731	 \\
-5.10986	 & 	442.839	 & 	1.31633	 \\
3.31731	 & 	1.31633	 & 	530.688
\end{tabular}

\vtab
 EingenVectors - Molecule B     \\
\begin{tabular}{|c c c|}
-0.999813	 & 	-0.0172596	 & 	0.00868475	 \\
-0.017383	 & 	0.999746	 & 	-0.0143352	 \\
0.00843513	 & 	0.0144835	 & 	0.99986
\end{tabular}

\vtab
 EingenValues - Molecule B     \\
\begin{tabular}{|c c c|}
146.173	 & 	442.909	 & 	530.735	 \\
\end{tabular}

\end{center}
\end{multicols}

\vtab[-5mm]
\begin{tabular}{*{2}{m{0.38\textwidth}}}
\begin{center}
\textcolor{NavyBlue}{\Large Enantiomers}
\end{center}
&
\begin{center}
\includegraphics[height=6.5cm]{../Comparisons/Vectors/inertia_tensor_of_test-R_out_G09_and_test-S_out_G09.png}
\end{center}
\end{tabular}

 \newpage

\vtab[-3cm]
\begin{center}
{\large TesisExample \tab Número 652}
\end{center}
\begin{multicols}{2}
\begin{center}

Molecule A \
test-R\_out\_G09

\includegraphics[width=6cm]{../Comparisons/ImagesFromVMD/test-R_out_G09.png}

Inertia Tensor - Molecule A \\
\begin{tabular}{|c c c|}
146.29	 & 	5.10986	 & 	-3.31731	 \\
5.10986	 & 	442.839	 & 	1.31633	 \\
-3.31731	 & 	1.31633	 & 	530.688
\end{tabular}

\vtab
 EingenVectors - Molecule A     \\
\begin{tabular}{|c c c|}
0.999813	 & 	-0.0172596	 & 	0.00868475	 \\
-0.017383	 & 	-0.999746	 & 	0.0143352	 \\
-0.00843513	 & 	0.0144835	 & 	0.99986
\end{tabular}

\vtab
 EingenValues - Molecule A     \\
\begin{tabular}{|c c c|}
146.173	 & 	442.909	 & 	530.735	 \\
\end{tabular}
\columnbreak

Molecule B \
test-S\_out\_G09\_invertion

\includegraphics[width=6cm]{../Comparisons/ImagesFromVMD/test-S_out_G09_invertion.png}

Inertia Tensor - Molecule B \\
\begin{tabular}{|c c c|}
146.29	 & 	-5.10969	 & 	3.31735	 \\
-5.10969	 & 	442.839	 & 	1.31631	 \\
3.31735	 & 	1.31631	 & 	530.688
\end{tabular}

\vtab
 EingenVectors - Molecule B     \\
\begin{tabular}{|c c c|}
-0.999813	 & 	-0.017259	 & 	0.00868483	 \\
-0.0173824	 & 	0.999746	 & 	-0.014335	 \\
0.00843522	 & 	0.0144833	 & 	0.99986
\end{tabular}

\vtab
 EingenValues - Molecule B     \\
\begin{tabular}{|c c c|}
146.173	 & 	442.909	 & 	530.735	 \\
\end{tabular}

\end{center}
\end{multicols}

\vtab[-5mm]
\begin{tabular}{*{2}{m{0.38\textwidth}}}
\begin{center}
\textcolor{NavyBlue}{\Large Equal}
\end{center}
&
\begin{center}
\includegraphics[height=6.5cm]{../Comparisons/Vectors/inertia_tensor_of_test-R_out_G09_and_test-S_out_G09_invertion.png}
\end{center}
\end{tabular}

 \newpage

\vtab[-3cm]
\begin{center}
{\large TesisExample \tab Número 653}
\end{center}
\begin{multicols}{2}
\begin{center}

Molecule A \
test-R\_out\_G09

\includegraphics[width=6cm]{../Comparisons/ImagesFromVMD/test-R_out_G09.png}

Inertia Tensor - Molecule A \\
\begin{tabular}{|c c c|}
146.29	 & 	5.10986	 & 	-3.31731	 \\
5.10986	 & 	442.839	 & 	1.31633	 \\
-3.31731	 & 	1.31633	 & 	530.688
\end{tabular}

\vtab
 EingenVectors - Molecule A     \\
\begin{tabular}{|c c c|}
0.999813	 & 	-0.0172596	 & 	0.00868475	 \\
-0.017383	 & 	-0.999746	 & 	0.0143352	 \\
-0.00843513	 & 	0.0144835	 & 	0.99986
\end{tabular}

\vtab
 EingenValues - Molecule A     \\
\begin{tabular}{|c c c|}
146.173	 & 	442.909	 & 	530.735	 \\
\end{tabular}
\columnbreak

Molecule B \
test-S\_out\_G09\_rot-45-45-45

\includegraphics[width=6cm]{../Comparisons/ImagesFromVMD/test-S_out_G09_rot-45-45-45.png}

Inertia Tensor - Molecule B \\
\begin{tabular}{|c c c|}
435.116	 & 	83.1261	 & 	142.506	 \\
83.1261	 & 	374.791	 & 	-97.314	 \\
142.506	 & 	-97.314	 & 	309.91
\end{tabular}

\vtab
 EingenVectors - Molecule B     \\
\begin{tabular}{|c c c|}
0.495022	 & 	-0.486446	 & 	-0.719947	 \\
0.125483	 & 	0.859929	 & 	-0.494748	 \\
-0.859771	 & 	-0.15457	 & 	-0.486725
\end{tabular}

\vtab
 EingenValues - Molecule B     \\
\begin{tabular}{|c c c|}
146.173	 & 	442.909	 & 	530.735	 \\
\end{tabular}

\end{center}
\end{multicols}

\vtab[-5mm]
\begin{tabular}{*{2}{m{0.38\textwidth}}}
\begin{center}
\textcolor{NavyBlue}{\Large Enantiomers}
\end{center}
&
\begin{center}
\includegraphics[height=6.5cm]{../Comparisons/Vectors/inertia_tensor_of_test-R_out_G09_and_test-S_out_G09_rot-45-45-45.png}
\end{center}
\end{tabular}

 \newpage

\vtab[-3cm]
\begin{center}
{\large TesisExample \tab Número 654}
\end{center}
\begin{multicols}{2}
\begin{center}

Molecule A \
test-R\_out\_G09\_invertion

\includegraphics[width=6cm]{../Comparisons/ImagesFromVMD/test-R_out_G09_invertion.png}

Inertia Tensor - Molecule A \\
\begin{tabular}{|c c c|}
146.29	 & 	5.10969	 & 	-3.31735	 \\
5.10969	 & 	442.839	 & 	1.31631	 \\
-3.31735	 & 	1.31631	 & 	530.688
\end{tabular}

\vtab
 EingenVectors - Molecule A     \\
\begin{tabular}{|c c c|}
0.999813	 & 	-0.017259	 & 	0.00868483	 \\
-0.0173824	 & 	-0.999746	 & 	0.014335	 \\
-0.00843522	 & 	0.0144833	 & 	0.99986
\end{tabular}

\vtab
 EingenValues - Molecule A     \\
\begin{tabular}{|c c c|}
146.173	 & 	442.909	 & 	530.735	 \\
\end{tabular}
\columnbreak

Molecule B \
test-R\_out\_G09\_rot-45-45-45

\includegraphics[width=6cm]{../Comparisons/ImagesFromVMD/test-R_out_G09_rot-45-45-45.png}

Inertia Tensor - Molecule B \\
\begin{tabular}{|c c c|}
430.951	 & 	89.0849	 & 	141.239	 \\
89.0849	 & 	367.04	 & 	-98.5816	 \\
141.239	 & 	-98.5816	 & 	321.828
\end{tabular}

\vtab
 EingenVectors - Molecule B     \\
\begin{tabular}{|c c c|}
0.504792	 & 	-0.513366	 & 	-0.694003	 \\
0.142865	 & 	0.842546	 & 	-0.519332	 \\
-0.851337	 & 	-0.163006	 & 	-0.498652
\end{tabular}

\vtab
 EingenValues - Molecule B     \\
\begin{tabular}{|c c c|}
146.173	 & 	442.909	 & 	530.735	 \\
\end{tabular}

\end{center}
\end{multicols}

\vtab[-5mm]
\begin{tabular}{*{2}{m{0.38\textwidth}}}
\begin{center}
\textcolor{NavyBlue}{\Large Enantiomers}
\end{center}
&
\begin{center}
\includegraphics[height=6.5cm]{../Comparisons/Vectors/inertia_tensor_of_test-R_out_G09_invertion_and_test-R_out_G09_rot-45-45-45.png}
\end{center}
\end{tabular}

 \newpage

\vtab[-3cm]
\begin{center}
{\large TesisExample \tab Número 655}
\end{center}
\begin{multicols}{2}
\begin{center}

Molecule A \
test-R\_out\_G09\_invertion

\includegraphics[width=6cm]{../Comparisons/ImagesFromVMD/test-R_out_G09_invertion.png}

Inertia Tensor - Molecule A \\
\begin{tabular}{|c c c|}
146.29	 & 	5.10969	 & 	-3.31735	 \\
5.10969	 & 	442.839	 & 	1.31631	 \\
-3.31735	 & 	1.31631	 & 	530.688
\end{tabular}

\vtab
 EingenVectors - Molecule A     \\
\begin{tabular}{|c c c|}
0.999813	 & 	-0.017259	 & 	0.00868483	 \\
-0.0173824	 & 	-0.999746	 & 	0.014335	 \\
-0.00843522	 & 	0.0144833	 & 	0.99986
\end{tabular}

\vtab
 EingenValues - Molecule A     \\
\begin{tabular}{|c c c|}
146.173	 & 	442.909	 & 	530.735	 \\
\end{tabular}
\columnbreak

Molecule B \
test-S\_out\_G09

\includegraphics[width=6cm]{../Comparisons/ImagesFromVMD/test-S_out_G09.png}

Inertia Tensor - Molecule B \\
\begin{tabular}{|c c c|}
146.29	 & 	-5.10986	 & 	3.31731	 \\
-5.10986	 & 	442.839	 & 	1.31633	 \\
3.31731	 & 	1.31633	 & 	530.688
\end{tabular}

\vtab
 EingenVectors - Molecule B     \\
\begin{tabular}{|c c c|}
-0.999813	 & 	-0.0172596	 & 	0.00868475	 \\
-0.017383	 & 	0.999746	 & 	-0.0143352	 \\
0.00843513	 & 	0.0144835	 & 	0.99986
\end{tabular}

\vtab
 EingenValues - Molecule B     \\
\begin{tabular}{|c c c|}
146.173	 & 	442.909	 & 	530.735	 \\
\end{tabular}

\end{center}
\end{multicols}

\vtab[-5mm]
\begin{tabular}{*{2}{m{0.38\textwidth}}}
\begin{center}
\textcolor{NavyBlue}{\Large Equal}
\end{center}
&
\begin{center}
\includegraphics[height=6.5cm]{../Comparisons/Vectors/inertia_tensor_of_test-R_out_G09_invertion_and_test-S_out_G09.png}
\end{center}
\end{tabular}

 \newpage

\vtab[-3cm]
\begin{center}
{\large TesisExample \tab Número 656}
\end{center}
\begin{multicols}{2}
\begin{center}

Molecule A \
test-R\_out\_G09\_invertion

\includegraphics[width=6cm]{../Comparisons/ImagesFromVMD/test-R_out_G09_invertion.png}

Inertia Tensor - Molecule A \\
\begin{tabular}{|c c c|}
146.29	 & 	5.10969	 & 	-3.31735	 \\
5.10969	 & 	442.839	 & 	1.31631	 \\
-3.31735	 & 	1.31631	 & 	530.688
\end{tabular}

\vtab
 EingenVectors - Molecule A     \\
\begin{tabular}{|c c c|}
0.999813	 & 	-0.017259	 & 	0.00868483	 \\
-0.0173824	 & 	-0.999746	 & 	0.014335	 \\
-0.00843522	 & 	0.0144833	 & 	0.99986
\end{tabular}

\vtab
 EingenValues - Molecule A     \\
\begin{tabular}{|c c c|}
146.173	 & 	442.909	 & 	530.735	 \\
\end{tabular}
\columnbreak

Molecule B \
test-S\_out\_G09\_invertion

\includegraphics[width=6cm]{../Comparisons/ImagesFromVMD/test-S_out_G09_invertion.png}

Inertia Tensor - Molecule B \\
\begin{tabular}{|c c c|}
146.29	 & 	-5.10969	 & 	3.31735	 \\
-5.10969	 & 	442.839	 & 	1.31631	 \\
3.31735	 & 	1.31631	 & 	530.688
\end{tabular}

\vtab
 EingenVectors - Molecule B     \\
\begin{tabular}{|c c c|}
-0.999813	 & 	-0.017259	 & 	0.00868483	 \\
-0.0173824	 & 	0.999746	 & 	-0.014335	 \\
0.00843522	 & 	0.0144833	 & 	0.99986
\end{tabular}

\vtab
 EingenValues - Molecule B     \\
\begin{tabular}{|c c c|}
146.173	 & 	442.909	 & 	530.735	 \\
\end{tabular}

\end{center}
\end{multicols}

\vtab[-5mm]
\begin{tabular}{*{2}{m{0.38\textwidth}}}
\begin{center}
\textcolor{NavyBlue}{\Large Enantiomers}
\end{center}
&
\begin{center}
\includegraphics[height=6.5cm]{../Comparisons/Vectors/inertia_tensor_of_test-R_out_G09_invertion_and_test-S_out_G09_invertion.png}
\end{center}
\end{tabular}

 \newpage

\vtab[-3cm]
\begin{center}
{\large TesisExample \tab Número 657}
\end{center}
\begin{multicols}{2}
\begin{center}

Molecule A \
test-R\_out\_G09\_invertion

\includegraphics[width=6cm]{../Comparisons/ImagesFromVMD/test-R_out_G09_invertion.png}

Inertia Tensor - Molecule A \\
\begin{tabular}{|c c c|}
146.29	 & 	5.10969	 & 	-3.31735	 \\
5.10969	 & 	442.839	 & 	1.31631	 \\
-3.31735	 & 	1.31631	 & 	530.688
\end{tabular}

\vtab
 EingenVectors - Molecule A     \\
\begin{tabular}{|c c c|}
0.999813	 & 	-0.017259	 & 	0.00868483	 \\
-0.0173824	 & 	-0.999746	 & 	0.014335	 \\
-0.00843522	 & 	0.0144833	 & 	0.99986
\end{tabular}

\vtab
 EingenValues - Molecule A     \\
\begin{tabular}{|c c c|}
146.173	 & 	442.909	 & 	530.735	 \\
\end{tabular}
\columnbreak

Molecule B \
test-S\_out\_G09\_rot-45-45-45

\includegraphics[width=6cm]{../Comparisons/ImagesFromVMD/test-S_out_G09_rot-45-45-45.png}

Inertia Tensor - Molecule B \\
\begin{tabular}{|c c c|}
435.116	 & 	83.1261	 & 	142.506	 \\
83.1261	 & 	374.791	 & 	-97.314	 \\
142.506	 & 	-97.314	 & 	309.91
\end{tabular}

\vtab
 EingenVectors - Molecule B     \\
\begin{tabular}{|c c c|}
0.495022	 & 	-0.486446	 & 	-0.719947	 \\
0.125483	 & 	0.859929	 & 	-0.494748	 \\
-0.859771	 & 	-0.15457	 & 	-0.486725
\end{tabular}

\vtab
 EingenValues - Molecule B     \\
\begin{tabular}{|c c c|}
146.173	 & 	442.909	 & 	530.735	 \\
\end{tabular}

\end{center}
\end{multicols}

\vtab[-5mm]
\begin{tabular}{*{2}{m{0.38\textwidth}}}
\begin{center}
\textcolor{NavyBlue}{\Large Equal}
\end{center}
&
\begin{center}
\includegraphics[height=6.5cm]{../Comparisons/Vectors/inertia_tensor_of_test-R_out_G09_invertion_and_test-S_out_G09_rot-45-45-45.png}
\end{center}
\end{tabular}

 \newpage

\vtab[-3cm]
\begin{center}
{\large TesisExample \tab Número 658}
\end{center}
\begin{multicols}{2}
\begin{center}

Molecule A \
test-R\_out\_G09\_rot-45-45-45

\includegraphics[width=6cm]{../Comparisons/ImagesFromVMD/test-R_out_G09_rot-45-45-45.png}

Inertia Tensor - Molecule A \\
\begin{tabular}{|c c c|}
430.951	 & 	89.0849	 & 	141.239	 \\
89.0849	 & 	367.04	 & 	-98.5816	 \\
141.239	 & 	-98.5816	 & 	321.828
\end{tabular}

\vtab
 EingenVectors - Molecule A     \\
\begin{tabular}{|c c c|}
0.504792	 & 	-0.513366	 & 	-0.694003	 \\
0.142865	 & 	0.842546	 & 	-0.519332	 \\
-0.851337	 & 	-0.163006	 & 	-0.498652
\end{tabular}

\vtab
 EingenValues - Molecule A     \\
\begin{tabular}{|c c c|}
146.173	 & 	442.909	 & 	530.735	 \\
\end{tabular}
\columnbreak

Molecule B \
test-S\_out\_G09

\includegraphics[width=6cm]{../Comparisons/ImagesFromVMD/test-S_out_G09.png}

Inertia Tensor - Molecule B \\
\begin{tabular}{|c c c|}
146.29	 & 	-5.10986	 & 	3.31731	 \\
-5.10986	 & 	442.839	 & 	1.31633	 \\
3.31731	 & 	1.31633	 & 	530.688
\end{tabular}

\vtab
 EingenVectors - Molecule B     \\
\begin{tabular}{|c c c|}
-0.999813	 & 	-0.0172596	 & 	0.00868475	 \\
-0.017383	 & 	0.999746	 & 	-0.0143352	 \\
0.00843513	 & 	0.0144835	 & 	0.99986
\end{tabular}

\vtab
 EingenValues - Molecule B     \\
\begin{tabular}{|c c c|}
146.173	 & 	442.909	 & 	530.735	 \\
\end{tabular}

\end{center}
\end{multicols}

\vtab[-5mm]
\begin{tabular}{*{2}{m{0.38\textwidth}}}
\begin{center}
\textcolor{NavyBlue}{\Large Enantiomers}
\end{center}
&
\begin{center}
\includegraphics[height=6.5cm]{../Comparisons/Vectors/inertia_tensor_of_test-R_out_G09_rot-45-45-45_and_test-S_out_G09.png}
\end{center}
\end{tabular}

 \newpage

\vtab[-3cm]
\begin{center}
{\large TesisExample \tab Número 659}
\end{center}
\begin{multicols}{2}
\begin{center}

Molecule A \
test-R\_out\_G09\_rot-45-45-45

\includegraphics[width=6cm]{../Comparisons/ImagesFromVMD/test-R_out_G09_rot-45-45-45.png}

Inertia Tensor - Molecule A \\
\begin{tabular}{|c c c|}
430.951	 & 	89.0849	 & 	141.239	 \\
89.0849	 & 	367.04	 & 	-98.5816	 \\
141.239	 & 	-98.5816	 & 	321.828
\end{tabular}

\vtab
 EingenVectors - Molecule A     \\
\begin{tabular}{|c c c|}
0.504792	 & 	-0.513366	 & 	-0.694003	 \\
0.142865	 & 	0.842546	 & 	-0.519332	 \\
-0.851337	 & 	-0.163006	 & 	-0.498652
\end{tabular}

\vtab
 EingenValues - Molecule A     \\
\begin{tabular}{|c c c|}
146.173	 & 	442.909	 & 	530.735	 \\
\end{tabular}
\columnbreak

Molecule B \
test-S\_out\_G09\_invertion

\includegraphics[width=6cm]{../Comparisons/ImagesFromVMD/test-S_out_G09_invertion.png}

Inertia Tensor - Molecule B \\
\begin{tabular}{|c c c|}
146.29	 & 	-5.10969	 & 	3.31735	 \\
-5.10969	 & 	442.839	 & 	1.31631	 \\
3.31735	 & 	1.31631	 & 	530.688
\end{tabular}

\vtab
 EingenVectors - Molecule B     \\
\begin{tabular}{|c c c|}
-0.999813	 & 	-0.017259	 & 	0.00868483	 \\
-0.0173824	 & 	0.999746	 & 	-0.014335	 \\
0.00843522	 & 	0.0144833	 & 	0.99986
\end{tabular}

\vtab
 EingenValues - Molecule B     \\
\begin{tabular}{|c c c|}
146.173	 & 	442.909	 & 	530.735	 \\
\end{tabular}

\end{center}
\end{multicols}

\vtab[-5mm]
\begin{tabular}{*{2}{m{0.38\textwidth}}}
\begin{center}
\textcolor{NavyBlue}{\Large Equal}
\end{center}
&
\begin{center}
\includegraphics[height=6.5cm]{../Comparisons/Vectors/inertia_tensor_of_test-R_out_G09_rot-45-45-45_and_test-S_out_G09_invertion.png}
\end{center}
\end{tabular}

 \newpage

\vtab[-3cm]
\begin{center}
{\large TesisExample \tab Número 660}
\end{center}
\begin{multicols}{2}
\begin{center}

Molecule A \
test-R\_out\_G09\_rot-45-45-45

\includegraphics[width=6cm]{../Comparisons/ImagesFromVMD/test-R_out_G09_rot-45-45-45.png}

Inertia Tensor - Molecule A \\
\begin{tabular}{|c c c|}
430.951	 & 	89.0849	 & 	141.239	 \\
89.0849	 & 	367.04	 & 	-98.5816	 \\
141.239	 & 	-98.5816	 & 	321.828
\end{tabular}

\vtab
 EingenVectors - Molecule A     \\
\begin{tabular}{|c c c|}
0.504792	 & 	-0.513366	 & 	-0.694003	 \\
0.142865	 & 	0.842546	 & 	-0.519332	 \\
-0.851337	 & 	-0.163006	 & 	-0.498652
\end{tabular}

\vtab
 EingenValues - Molecule A     \\
\begin{tabular}{|c c c|}
146.173	 & 	442.909	 & 	530.735	 \\
\end{tabular}
\columnbreak

Molecule B \
test-S\_out\_G09\_rot-45-45-45

\includegraphics[width=6cm]{../Comparisons/ImagesFromVMD/test-S_out_G09_rot-45-45-45.png}

Inertia Tensor - Molecule B \\
\begin{tabular}{|c c c|}
435.116	 & 	83.1261	 & 	142.506	 \\
83.1261	 & 	374.791	 & 	-97.314	 \\
142.506	 & 	-97.314	 & 	309.91
\end{tabular}

\vtab
 EingenVectors - Molecule B     \\
\begin{tabular}{|c c c|}
0.495022	 & 	-0.486446	 & 	-0.719947	 \\
0.125483	 & 	0.859929	 & 	-0.494748	 \\
-0.859771	 & 	-0.15457	 & 	-0.486725
\end{tabular}

\vtab
 EingenValues - Molecule B     \\
\begin{tabular}{|c c c|}
146.173	 & 	442.909	 & 	530.735	 \\
\end{tabular}

\end{center}
\end{multicols}

\vtab[-5mm]
\begin{tabular}{*{2}{m{0.38\textwidth}}}
\begin{center}
\textcolor{NavyBlue}{\Large Enantiomers}
\end{center}
&
\begin{center}
\includegraphics[height=6.5cm]{../Comparisons/Vectors/inertia_tensor_of_test-R_out_G09_rot-45-45-45_and_test-S_out_G09_rot-45-45-45.png}
\end{center}
\end{tabular}

 \newpage

\vtab[-3cm]
\begin{center}
{\large TesisExample \tab Número 661}
\end{center}
\begin{multicols}{2}
\begin{center}

Molecule A \
test-S\_out\_G09

\includegraphics[width=6cm]{../Comparisons/ImagesFromVMD/test-S_out_G09.png}

Inertia Tensor - Molecule A \\
\begin{tabular}{|c c c|}
146.29	 & 	-5.10986	 & 	3.31731	 \\
-5.10986	 & 	442.839	 & 	1.31633	 \\
3.31731	 & 	1.31633	 & 	530.688
\end{tabular}

\vtab
 EingenVectors - Molecule A     \\
\begin{tabular}{|c c c|}
-0.999813	 & 	-0.0172596	 & 	0.00868475	 \\
-0.017383	 & 	0.999746	 & 	-0.0143352	 \\
0.00843513	 & 	0.0144835	 & 	0.99986
\end{tabular}

\vtab
 EingenValues - Molecule A     \\
\begin{tabular}{|c c c|}
146.173	 & 	442.909	 & 	530.735	 \\
\end{tabular}
\columnbreak

Molecule B \
test-S\_out\_G09\_invertion

\includegraphics[width=6cm]{../Comparisons/ImagesFromVMD/test-S_out_G09_invertion.png}

Inertia Tensor - Molecule B \\
\begin{tabular}{|c c c|}
146.29	 & 	-5.10969	 & 	3.31735	 \\
-5.10969	 & 	442.839	 & 	1.31631	 \\
3.31735	 & 	1.31631	 & 	530.688
\end{tabular}

\vtab
 EingenVectors - Molecule B     \\
\begin{tabular}{|c c c|}
-0.999813	 & 	-0.017259	 & 	0.00868483	 \\
-0.0173824	 & 	0.999746	 & 	-0.014335	 \\
0.00843522	 & 	0.0144833	 & 	0.99986
\end{tabular}

\vtab
 EingenValues - Molecule B     \\
\begin{tabular}{|c c c|}
146.173	 & 	442.909	 & 	530.735	 \\
\end{tabular}

\end{center}
\end{multicols}

\vtab[-5mm]
\begin{tabular}{*{2}{m{0.38\textwidth}}}
\begin{center}
\textcolor{NavyBlue}{\Large Enantiomers}
\end{center}
&
\begin{center}
\includegraphics[height=6.5cm]{../Comparisons/Vectors/inertia_tensor_of_test-S_out_G09_and_test-S_out_G09_invertion.png}
\end{center}
\end{tabular}

 \newpage

\vtab[-3cm]
\begin{center}
{\large TesisExample \tab Número 662}
\end{center}
\begin{multicols}{2}
\begin{center}

Molecule A \
test-S\_out\_G09

\includegraphics[width=6cm]{../Comparisons/ImagesFromVMD/test-S_out_G09.png}

Inertia Tensor - Molecule A \\
\begin{tabular}{|c c c|}
146.29	 & 	-5.10986	 & 	3.31731	 \\
-5.10986	 & 	442.839	 & 	1.31633	 \\
3.31731	 & 	1.31633	 & 	530.688
\end{tabular}

\vtab
 EingenVectors - Molecule A     \\
\begin{tabular}{|c c c|}
-0.999813	 & 	-0.0172596	 & 	0.00868475	 \\
-0.017383	 & 	0.999746	 & 	-0.0143352	 \\
0.00843513	 & 	0.0144835	 & 	0.99986
\end{tabular}

\vtab
 EingenValues - Molecule A     \\
\begin{tabular}{|c c c|}
146.173	 & 	442.909	 & 	530.735	 \\
\end{tabular}
\columnbreak

Molecule B \
test-S\_out\_G09\_rot-45-45-45

\includegraphics[width=6cm]{../Comparisons/ImagesFromVMD/test-S_out_G09_rot-45-45-45.png}

Inertia Tensor - Molecule B \\
\begin{tabular}{|c c c|}
435.116	 & 	83.1261	 & 	142.506	 \\
83.1261	 & 	374.791	 & 	-97.314	 \\
142.506	 & 	-97.314	 & 	309.91
\end{tabular}

\vtab
 EingenVectors - Molecule B     \\
\begin{tabular}{|c c c|}
0.495022	 & 	-0.486446	 & 	-0.719947	 \\
0.125483	 & 	0.859929	 & 	-0.494748	 \\
-0.859771	 & 	-0.15457	 & 	-0.486725
\end{tabular}

\vtab
 EingenValues - Molecule B     \\
\begin{tabular}{|c c c|}
146.173	 & 	442.909	 & 	530.735	 \\
\end{tabular}

\end{center}
\end{multicols}

\vtab[-5mm]
\begin{tabular}{*{2}{m{0.38\textwidth}}}
\begin{center}
\textcolor{NavyBlue}{\Large Equal}
\end{center}
&
\begin{center}
\includegraphics[height=6.5cm]{../Comparisons/Vectors/inertia_tensor_of_test-S_out_G09_and_test-S_out_G09_rot-45-45-45.png}
\end{center}
\end{tabular}

 \newpage

\vtab[-3cm]
\begin{center}
{\large TesisExample \tab Número 663}
\end{center}
\begin{multicols}{2}
\begin{center}

Molecule A \
test-S\_out\_G09\_invertion

\includegraphics[width=6cm]{../Comparisons/ImagesFromVMD/test-S_out_G09_invertion.png}

Inertia Tensor - Molecule A \\
\begin{tabular}{|c c c|}
146.29	 & 	-5.10969	 & 	3.31735	 \\
-5.10969	 & 	442.839	 & 	1.31631	 \\
3.31735	 & 	1.31631	 & 	530.688
\end{tabular}

\vtab
 EingenVectors - Molecule A     \\
\begin{tabular}{|c c c|}
-0.999813	 & 	-0.017259	 & 	0.00868483	 \\
-0.0173824	 & 	0.999746	 & 	-0.014335	 \\
0.00843522	 & 	0.0144833	 & 	0.99986
\end{tabular}

\vtab
 EingenValues - Molecule A     \\
\begin{tabular}{|c c c|}
146.173	 & 	442.909	 & 	530.735	 \\
\end{tabular}
\columnbreak

Molecule B \
test-S\_out\_G09\_rot-45-45-45

\includegraphics[width=6cm]{../Comparisons/ImagesFromVMD/test-S_out_G09_rot-45-45-45.png}

Inertia Tensor - Molecule B \\
\begin{tabular}{|c c c|}
435.116	 & 	83.1261	 & 	142.506	 \\
83.1261	 & 	374.791	 & 	-97.314	 \\
142.506	 & 	-97.314	 & 	309.91
\end{tabular}

\vtab
 EingenVectors - Molecule B     \\
\begin{tabular}{|c c c|}
0.495022	 & 	-0.486446	 & 	-0.719947	 \\
0.125483	 & 	0.859929	 & 	-0.494748	 \\
-0.859771	 & 	-0.15457	 & 	-0.486725
\end{tabular}

\vtab
 EingenValues - Molecule B     \\
\begin{tabular}{|c c c|}
146.173	 & 	442.909	 & 	530.735	 \\
\end{tabular}

\end{center}
\end{multicols}

\vtab[-5mm]
\begin{tabular}{*{2}{m{0.38\textwidth}}}
\begin{center}
\textcolor{NavyBlue}{\Large Enantiomers}
\end{center}
&
\begin{center}
\includegraphics[height=6.5cm]{../Comparisons/Vectors/inertia_tensor_of_test-S_out_G09_invertion_and_test-S_out_G09_rot-45-45-45.png}
\end{center}
\end{tabular}

 \newpage

 %%%%%%%%%%%%%%%%%%%%

\vtab[-2cm]
\tab[-2cm]
\begin{tabular}{c|m{8cm}|c|c}
\# & Moléculas & Restultado esperado & Resultado programa \\\\ \hline\hline
\multirow{4}{*}{\tab[2mm] 1 \tab[2mm]} & r-2-chlorobutane\_out\_G09 &
\multirow{3}{*}{\textcolor{Red}{\bf Without comparation}} & \multirow{3}{*}{\textcolor{Red}{\bf Enantiomers}}
\\
& E = No Data \tab Freq =No Data   &    &  \\ \cline{2-2}
& r-2-chlorobutane\_out\_G09\_inversion   & \multicolumn{2}{c}{\multirow{3}{*}
 {RMS = 0.02719433}}
\\
& E = No Data \tab Freq =No Data   &    \multicolumn{2}{c}{}  \\ \hline
\multirow{4}{*}{\tab[2mm] 2 \tab[2mm]} & r-2-chlorobutane\_out\_G09 &
\multirow{3}{*}{Equal} & \multirow{3}{*}{Equal}
\\
& E = No Data \tab Freq =No Data   &    &  \\ \cline{2-2}
& r-2-chlorobutane\_rotated\_out\_G09   & \multicolumn{2}{c}{\multirow{3}{*}
{ RMS = 1.236488E-09}}
\\
& E = No Data \tab Freq =No Data   &    \multicolumn{2}{c}{}  \\ \hline
\multirow{4}{*}{\tab[2mm] 3 \tab[2mm]} & r-2-chlorobutane\_out\_G09 &
\multirow{3}{*}{\textcolor{Red}{\bf Stereoisomer}} & \multirow{3}{*}{\textcolor{Red}{\bf Enantiomers}}
\\
& E = No Data \tab Freq =No Data   &    &  \\ \cline{2-2}
& s-2-chlorobutane\_out\_G09   & \multicolumn{2}{c}{\multirow{3}{*}
 {RMS = 0.2294273}}
\\
& E = No Data \tab Freq =No Data   &    \multicolumn{2}{c}{}  \\ \hline
\multirow{4}{*}{\tab[2mm] 4 \tab[2mm]} & r-2-chlorobutane\_out\_G09 &
\multirow{3}{*}{Equal} & \multirow{3}{*}{Equal}
\\
& E = No Data \tab Freq =No Data   &    &  \\ \cline{2-2}
& s-2-chlorobutane\_out\_G09\_inversion   & \multicolumn{2}{c}{\multirow{3}{*}
{\textcolor{Red}{ RMS = 0.2566261}}}
\\
& E = No Data \tab Freq =No Data   &    \multicolumn{2}{c}{}  \\ \hline
\multirow{4}{*}{\tab[2mm] 5 \tab[2mm]} & r-2-chlorobutane\_out\_G09 &
\multirow{3}{*}{\textcolor{Red}{\bf Stereoisomer}} & \multirow{3}{*}{\textcolor{Red}{\bf Enantiomers}}
\\
& E = No Data \tab Freq =No Data   &    &  \\ \cline{2-2}
& s-2-chlorobutane\_rotated\_out\_G09   & \multicolumn{2}{c}{\multirow{3}{*}
 {RMS = 0.2294272}}
\\
& E = No Data \tab Freq =No Data   &    \multicolumn{2}{c}{}  \\ \hline
\multirow{4}{*}{\tab[2mm] 6 \tab[2mm]} & r-2-chlorobutane\_out\_G09\_inversion &
\multirow{3}{*}{\textcolor{Red}{\bf Stereoisomer}} & \multirow{3}{*}{\textcolor{Red}{\bf Enantiomers}}
\\
& E = No Data \tab Freq =No Data   &    &  \\ \cline{2-2}
& r-2-chlorobutane\_rotated\_out\_G09   & \multicolumn{2}{c}{\multirow{3}{*}
 {RMS = 0.02719433}}
\\
& E = No Data \tab Freq =No Data   &    \multicolumn{2}{c}{}  \\ \hline
\multirow{4}{*}{\tab[2mm] 7 \tab[2mm]} & r-2-chlorobutane\_out\_G09\_inversion &
\multirow{3}{*}{Equal} & \multirow{3}{*}{Equal}
\\
& E = No Data \tab Freq =No Data   &    &  \\ \cline{2-2}
& s-2-chlorobutane\_out\_G09   & \multicolumn{2}{c}{\multirow{3}{*}
{\textcolor{Red}{ RMS = 0.202233}}}
\\
& E = No Data \tab Freq =No Data   &    \multicolumn{2}{c}{}  \\ \hline
\end{tabular}
\newpage

\vtab[-2cm]
\tab[-2cm]
\begin{tabular}{c|m{8cm}|c|c}
\# & Moléculas & Restultado esperado & Resultado programa \\\\ \hline\hline
\multirow{4}{*}{\tab[2mm] 8 \tab[2mm]} & r-2-chlorobutane\_out\_G09\_inversion &
\multirow{3}{*}{\textcolor{Red}{\bf Stereoisomer}} & \multirow{3}{*}{\textcolor{Red}{\bf Enantiomers}}
\\
& E = No Data \tab Freq =No Data   &    &  \\ \cline{2-2}
& s-2-chlorobutane\_out\_G09\_inversion   & \multicolumn{2}{c}{\multirow{3}{*}
 {RMS = 0.2294318}}
\\
& E = No Data \tab Freq =No Data   &    \multicolumn{2}{c}{}  \\ \hline
\multirow{4}{*}{\tab[2mm] 9 \tab[2mm]} & r-2-chlorobutane\_out\_G09\_inversion &
\multirow{3}{*}{Equal} & \multirow{3}{*}{Equal}
\\
& E = No Data \tab Freq =No Data   &    &  \\ \cline{2-2}
& s-2-chlorobutane\_rotated\_out\_G09   & \multicolumn{2}{c}{\multirow{3}{*}
{\textcolor{Red}{ RMS = 0.2022328}}}
\\
& E = No Data \tab Freq =No Data   &    \multicolumn{2}{c}{}  \\ \hline
\multirow{4}{*}{\tab[2mm] 10 \tab[2mm]} & r-2-chlorobutane\_rotated\_out\_G09 &
\multirow{3}{*}{\textcolor{Red}{\bf Stereoisomer}} & \multirow{3}{*}{\textcolor{Red}{\bf Enantiomers}}
\\
& E = No Data \tab Freq =No Data   &    &  \\ \cline{2-2}
& s-2-chlorobutane\_out\_G09   & \multicolumn{2}{c}{\multirow{3}{*}
 {RMS = 0.2294273}}
\\
& E = No Data \tab Freq =No Data   &    \multicolumn{2}{c}{}  \\ \hline
\multirow{4}{*}{\tab[2mm] 11 \tab[2mm]} & r-2-chlorobutane\_rotated\_out\_G09 &
\multirow{3}{*}{Equal} & \multirow{3}{*}{Equal}
\\
& E = No Data \tab Freq =No Data   &    &  \\ \cline{2-2}
& s-2-chlorobutane\_out\_G09\_inversion   & \multicolumn{2}{c}{\multirow{3}{*}
{\textcolor{Red}{ RMS = 0.2566261}}}
\\
& E = No Data \tab Freq =No Data   &    \multicolumn{2}{c}{}  \\ \hline
\multirow{4}{*}{\tab[2mm] 12 \tab[2mm]} & r-2-chlorobutane\_rotated\_out\_G09 &
\multirow{3}{*}{\textcolor{Red}{\bf Stereoisomer}} & \multirow{3}{*}{\textcolor{Red}{\bf Enantiomers}}
\\
& E = No Data \tab Freq =No Data   &    &  \\ \cline{2-2}
& s-2-chlorobutane\_rotated\_out\_G09   & \multicolumn{2}{c}{\multirow{3}{*}
 {RMS = 0.2294272}}
\\
& E = No Data \tab Freq =No Data   &    \multicolumn{2}{c}{}  \\ \hline
\multirow{4}{*}{\tab[2mm] 13 \tab[2mm]} & s-2-chlorobutane\_out\_G09 &
\multirow{3}{*}{\textcolor{Red}{\bf Without comparation}} & \multirow{3}{*}{\textcolor{Red}{\bf Enantiomers}}
\\
& E = No Data \tab Freq =No Data   &    &  \\ \cline{2-2}
& s-2-chlorobutane\_out\_G09\_inversion   & \multicolumn{2}{c}{\multirow{3}{*}
 {RMS = 0.02719881}}
\\
& E = No Data \tab Freq =No Data   &    \multicolumn{2}{c}{}  \\ \hline
\multirow{4}{*}{\tab[2mm] 14 \tab[2mm]} & s-2-chlorobutane\_out\_G09 &
\multirow{3}{*}{Equal} & \multirow{3}{*}{Equal}
\\
& E = No Data \tab Freq =No Data   &    &  \\ \cline{2-2}
& s-2-chlorobutane\_rotated\_out\_G09   & \multicolumn{2}{c}{\multirow{3}{*}
{ RMS = 1.306092E-07}}
\\
& E = No Data \tab Freq =No Data   &    \multicolumn{2}{c}{}  \\ \hline
\end{tabular}
\newpage

\vtab[-2cm]
\tab[-2cm]
\begin{tabular}{c|m{8cm}|c|c}
\# & Moléculas & Restultado esperado & Resultado programa \\\\ \hline\hline
\multirow{4}{*}{\tab[2mm] 15 \tab[2mm]} & s-2-chlorobutane\_out\_G09\_inversion &
\multirow{3}{*}{\textcolor{Red}{\bf Stereoisomer}} & \multirow{3}{*}{\textcolor{Red}{\bf Enantiomers}}
\\
& E = No Data \tab Freq =No Data   &    &  \\ \cline{2-2}
& s-2-chlorobutane\_rotated\_out\_G09   & \multicolumn{2}{c}{\multirow{3}{*}
 {RMS = 0.02719894}}
\\
& E = No Data \tab Freq =No Data   &    \multicolumn{2}{c}{}  \\ \hline
\multirow{4}{*}{\tab[2mm] 16 \tab[2mm]} & 3NFAACa &
\multirow{3}{*}{\textcolor{Red}{\bf Equal}} & \multirow{3}{*}{\textcolor{Red}{\bf Enantiomers}}
\\
& E = No Data \tab Freq =No Data   &    &  \\ \cline{2-2}
& 3NFAACb   & \multicolumn{2}{c}{\multirow{3}{*}
 {RMS = 0.02603601}}
\\
& E = No Data \tab Freq =No Data   &    \multicolumn{2}{c}{}  \\ \hline
\multirow{4}{*}{\tab[2mm] 17 \tab[2mm]} & 3NFAACa &
\multirow{3}{*}{Different} & \multirow{3}{*}{Different}
\\
& E = No Data \tab Freq =No Data   &    &  \\ \cline{2-2}
& 3NFAACc   & \multicolumn{2}{c}{\multirow{3}{*}
 {RMS = 0.0334016}}
\\
& E = No Data \tab Freq =No Data   &    \multicolumn{2}{c}{}  \\ \hline
\multirow{4}{*}{\tab[2mm] 18 \tab[2mm]} & 3NFAACa &
\multirow{3}{*}{Different} & \multirow{3}{*}{Different}
\\
& E = No Data \tab Freq =No Data   &    &  \\ \cline{2-2}
& 3NFAACd   & \multicolumn{2}{c}{\multirow{3}{*}
 {RMS = 0.0888511}}
\\
& E = No Data \tab Freq =No Data   &    \multicolumn{2}{c}{}  \\ \hline
\multirow{4}{*}{\tab[2mm] 19 \tab[2mm]} & 3NFAACa &
\multirow{3}{*}{Different} & \multirow{3}{*}{Different}
\\
& E = No Data \tab Freq =No Data   &    &  \\ \cline{2-2}
& 3NFAACe   & \multicolumn{2}{c}{\multirow{3}{*}
 {RMS = 0.162132}}
\\
& E = No Data \tab Freq =No Data   &    \multicolumn{2}{c}{}  \\ \hline
\multirow{4}{*}{\tab[2mm] 20 \tab[2mm]} & 3NFAACa &
\multirow{3}{*}{\textcolor{Red}{\bf Equal}} & \multirow{3}{*}{\textcolor{Red}{\bf Enantiomers}}
\\
& E = No Data \tab Freq =No Data   &    &  \\ \cline{2-2}
& 3NFAACf   & \multicolumn{2}{c}{\multirow{3}{*}
 {RMS = 0.02649945}}
\\
& E = No Data \tab Freq =No Data   &    \multicolumn{2}{c}{}  \\ \hline
\multirow{4}{*}{\tab[2mm] 21 \tab[2mm]} & 3NFAACa &
\multirow{3}{*}{Different} & \multirow{3}{*}{Different}
\\
& E = No Data \tab Freq =No Data   &    &  \\ \cline{2-2}
& 3NFAACg   & \multicolumn{2}{c}{\multirow{3}{*}
 {RMS = 0.0330848}}
\\
& E = No Data \tab Freq =No Data   &    \multicolumn{2}{c}{}  \\ \hline
\end{tabular}
\newpage

\vtab[-2cm]
\tab[-2cm]
\begin{tabular}{c|m{8cm}|c|c}
\# & Moléculas & Restultado esperado & Resultado programa \\\\ \hline\hline
\multirow{4}{*}{\tab[2mm] 22 \tab[2mm]} & 3NFAACa &
\multirow{3}{*}{Different} & \multirow{3}{*}{Different}
\\
& E = No Data \tab Freq =No Data   &    &  \\ \cline{2-2}
& 3NFAACh   & \multicolumn{2}{c}{\multirow{3}{*}
 {RMS = 0.064417}}
\\
& E = No Data \tab Freq =No Data   &    \multicolumn{2}{c}{}  \\ \hline
\multirow{4}{*}{\tab[2mm] 23 \tab[2mm]} & 3NFAACa &
\multirow{3}{*}{Different} & \multirow{3}{*}{Different}
\\
& E = No Data \tab Freq =No Data   &    &  \\ \cline{2-2}
& 3NFAACi   & \multicolumn{2}{c}{\multirow{3}{*}
 {RMS = 0.0837021}}
\\
& E = No Data \tab Freq =No Data   &    \multicolumn{2}{c}{}  \\ \hline
\multirow{4}{*}{\tab[2mm] 24 \tab[2mm]} & 3NFAACa &
\multirow{3}{*}{Different} & \multirow{3}{*}{Different}
\\
& E = No Data \tab Freq =No Data   &    &  \\ \cline{2-2}
& 3NFAACj   & \multicolumn{2}{c}{\multirow{3}{*}
 {RMS = 0.0903201}}
\\
& E = No Data \tab Freq =No Data   &    \multicolumn{2}{c}{}  \\ \hline
\multirow{4}{*}{\tab[2mm] 25 \tab[2mm]} & 3NFAACa &
\multirow{3}{*}{Different} & \multirow{3}{*}{Different}
\\
& E = No Data \tab Freq =No Data   &    &  \\ \cline{2-2}
& 3NFAACk   & \multicolumn{2}{c}{\multirow{3}{*}
 {RMS = 0.0114116}}
\\
& E = No Data \tab Freq =No Data   &    \multicolumn{2}{c}{}  \\ \hline
\multirow{4}{*}{\tab[2mm] 26 \tab[2mm]} & 3NFAACa &
\multirow{3}{*}{Different} & \multirow{3}{*}{Different}
\\
& E = No Data \tab Freq =No Data   &    &  \\ \cline{2-2}
& 3NFAACl   & \multicolumn{2}{c}{\multirow{3}{*}
 {RMS = 0.0362294}}
\\
& E = No Data \tab Freq =No Data   &    \multicolumn{2}{c}{}  \\ \hline
\multirow{4}{*}{\tab[2mm] 27 \tab[2mm]} & 3NFAACa &
\multirow{3}{*}{Different} & \multirow{3}{*}{Different}
\\
& E = No Data \tab Freq =No Data   &    &  \\ \cline{2-2}
& 3NFAACm   & \multicolumn{2}{c}{\multirow{3}{*}
 {RMS = 0.152833}}
\\
& E = No Data \tab Freq =No Data   &    \multicolumn{2}{c}{}  \\ \hline
\multirow{4}{*}{\tab[2mm] 28 \tab[2mm]} & 3NFAACa &
\multirow{3}{*}{Different} & \multirow{3}{*}{Different}
\\
& E = No Data \tab Freq =No Data   &    &  \\ \cline{2-2}
& 3NFAACn   & \multicolumn{2}{c}{\multirow{3}{*}
 {RMS = 0.157182}}
\\
& E = No Data \tab Freq =No Data   &    \multicolumn{2}{c}{}  \\ \hline
\end{tabular}
\newpage

\vtab[-2cm]
\tab[-2cm]
\begin{tabular}{c|m{8cm}|c|c}
\# & Moléculas & Restultado esperado & Resultado programa \\\\ \hline\hline
\multirow{4}{*}{\tab[2mm] 29 \tab[2mm]} & 3NFAACa &
\multirow{3}{*}{Different} & \multirow{3}{*}{Different}
\\
& E = No Data \tab Freq =No Data   &    &  \\ \cline{2-2}
& 4NFAACa   & \multicolumn{2}{c}{\multirow{3}{*}
 {RMS = 0.219778}}
\\
& E = No Data \tab Freq =No Data   &    \multicolumn{2}{c}{}  \\ \hline
\multirow{4}{*}{\tab[2mm] 30 \tab[2mm]} & 3NFAACa &
\multirow{3}{*}{Different} & \multirow{3}{*}{Different}
\\
& E = No Data \tab Freq =No Data   &    &  \\ \cline{2-2}
& 4NFAACb   & \multicolumn{2}{c}{\multirow{3}{*}
 {RMS = 0.258442}}
\\
& E = No Data \tab Freq =No Data   &    \multicolumn{2}{c}{}  \\ \hline
\multirow{4}{*}{\tab[2mm] 31 \tab[2mm]} & 3NFAACa &
\multirow{3}{*}{Different} & \multirow{3}{*}{Different}
\\
& E = No Data \tab Freq =No Data   &    &  \\ \cline{2-2}
& 4NFAACc   & \multicolumn{2}{c}{\multirow{3}{*}
 {RMS = 0.0186255}}
\\
& E = No Data \tab Freq =No Data   &    \multicolumn{2}{c}{}  \\ \hline
\multirow{4}{*}{\tab[2mm] 32 \tab[2mm]} & 3NFAACa &
\multirow{3}{*}{Different} & \multirow{3}{*}{Different}
\\
& E = No Data \tab Freq =No Data   &    &  \\ \cline{2-2}
& 4NFAACd   & \multicolumn{2}{c}{\multirow{3}{*}
 {RMS = 0.0149078}}
\\
& E = No Data \tab Freq =No Data   &    \multicolumn{2}{c}{}  \\ \hline
\multirow{4}{*}{\tab[2mm] 33 \tab[2mm]} & 3NFAACa &
\multirow{3}{*}{Different} & \multirow{3}{*}{Different}
\\
& E = No Data \tab Freq =No Data   &    &  \\ \cline{2-2}
& 4NFAACe   & \multicolumn{2}{c}{\multirow{3}{*}
 {RMS = 0.0228438}}
\\
& E = No Data \tab Freq =No Data   &    \multicolumn{2}{c}{}  \\ \hline
\multirow{4}{*}{\tab[2mm] 34 \tab[2mm]} & 3NFAACa &
\multirow{3}{*}{Different} & \multirow{3}{*}{Different}
\\
& E = No Data \tab Freq =No Data   &    &  \\ \cline{2-2}
& 4NFAACf   & \multicolumn{2}{c}{\multirow{3}{*}
 {RMS = 0.118302}}
\\
& E = No Data \tab Freq =No Data   &    \multicolumn{2}{c}{}  \\ \hline
\multirow{4}{*}{\tab[2mm] 35 \tab[2mm]} & 3NFAACa &
\multirow{3}{*}{Different} & \multirow{3}{*}{Different}
\\
& E = No Data \tab Freq =No Data   &    &  \\ \cline{2-2}
& 4NFAACg   & \multicolumn{2}{c}{\multirow{3}{*}
 {RMS = 0.0338585}}
\\
& E = No Data \tab Freq =No Data   &    \multicolumn{2}{c}{}  \\ \hline
\end{tabular}
\newpage

\vtab[-2cm]
\tab[-2cm]
\begin{tabular}{c|m{8cm}|c|c}
\# & Moléculas & Restultado esperado & Resultado programa \\\\ \hline\hline
\multirow{4}{*}{\tab[2mm] 36 \tab[2mm]} & 3NFAACa &
\multirow{3}{*}{Different} & \multirow{3}{*}{Different}
\\
& E = No Data \tab Freq =No Data   &    &  \\ \cline{2-2}
& 4NFAACi   & \multicolumn{2}{c}{\multirow{3}{*}
 {RMS = 0.00146242}}
\\
& E = No Data \tab Freq =No Data   &    \multicolumn{2}{c}{}  \\ \hline
\multirow{4}{*}{\tab[2mm] 37 \tab[2mm]} & 3NFAACa &
\multirow{3}{*}{Different} & \multirow{3}{*}{Different}
\\
& E = No Data \tab Freq =No Data   &    &  \\ \cline{2-2}
& 4NFAACj   & \multicolumn{2}{c}{\multirow{3}{*}
 {RMS = 0.0868356}}
\\
& E = No Data \tab Freq =No Data   &    \multicolumn{2}{c}{}  \\ \hline
\multirow{4}{*}{\tab[2mm] 38 \tab[2mm]} & 3NFAACa &
\multirow{3}{*}{Different} & \multirow{3}{*}{Different}
\\
& E = No Data \tab Freq =No Data   &    &  \\ \cline{2-2}
& 4NFAACl-3   & \multicolumn{2}{c}{\multirow{3}{*}
 {RMS = 0.0414826}}
\\
& E = No Data \tab Freq =No Data   &    \multicolumn{2}{c}{}  \\ \hline
\multirow{4}{*}{\tab[2mm] 39 \tab[2mm]} & 3NFAACb &
\multirow{3}{*}{Different} & \multirow{3}{*}{Different}
\\
& E = No Data \tab Freq =No Data   &    &  \\ \cline{2-2}
& 3NFAACc   & \multicolumn{2}{c}{\multirow{3}{*}
 {RMS = 0.0881291}}
\\
& E = No Data \tab Freq =No Data   &    \multicolumn{2}{c}{}  \\ \hline
\multirow{4}{*}{\tab[2mm] 40 \tab[2mm]} & 3NFAACb &
\multirow{3}{*}{Different} & \multirow{3}{*}{Different}
\\
& E = No Data \tab Freq =No Data   &    &  \\ \cline{2-2}
& 3NFAACd   & \multicolumn{2}{c}{\multirow{3}{*}
 {RMS = 0.143579}}
\\
& E = No Data \tab Freq =No Data   &    \multicolumn{2}{c}{}  \\ \hline
\multirow{4}{*}{\tab[2mm] 41 \tab[2mm]} & 3NFAACb &
\multirow{3}{*}{Different} & \multirow{3}{*}{Different}
\\
& E = No Data \tab Freq =No Data   &    &  \\ \cline{2-2}
& 3NFAACe   & \multicolumn{2}{c}{\multirow{3}{*}
 {RMS = 0.107404}}
\\
& E = No Data \tab Freq =No Data   &    \multicolumn{2}{c}{}  \\ \hline
\multirow{4}{*}{\tab[2mm] 42 \tab[2mm]} & 3NFAACb &
\multirow{3}{*}{\textcolor{Red}{\bf Equal}} & \multirow{3}{*}{\textcolor{Red}{\bf Enantiomers}}
\\
& E = No Data \tab Freq =No Data   &    &  \\ \cline{2-2}
& 3NFAACf   & \multicolumn{2}{c}{\multirow{3}{*}
 {RMS = 0.0004634431}}
\\
& E = No Data \tab Freq =No Data   &    \multicolumn{2}{c}{}  \\ \hline
\end{tabular}
\newpage

\vtab[-2cm]
\tab[-2cm]
\begin{tabular}{c|m{8cm}|c|c}
\# & Moléculas & Restultado esperado & Resultado programa \\\\ \hline\hline
\multirow{4}{*}{\tab[2mm] 43 \tab[2mm]} & 3NFAACb &
\multirow{3}{*}{Different} & \multirow{3}{*}{Different}
\\
& E = No Data \tab Freq =No Data   &    &  \\ \cline{2-2}
& 3NFAACg   & \multicolumn{2}{c}{\multirow{3}{*}
 {RMS = 0.0878122}}
\\
& E = No Data \tab Freq =No Data   &    \multicolumn{2}{c}{}  \\ \hline
\multirow{4}{*}{\tab[2mm] 44 \tab[2mm]} & 3NFAACb &
\multirow{3}{*}{Different} & \multirow{3}{*}{Different}
\\
& E = No Data \tab Freq =No Data   &    &  \\ \cline{2-2}
& 3NFAACh   & \multicolumn{2}{c}{\multirow{3}{*}
 {RMS = 0.00968954}}
\\
& E = No Data \tab Freq =No Data   &    \multicolumn{2}{c}{}  \\ \hline
\multirow{4}{*}{\tab[2mm] 45 \tab[2mm]} & 3NFAACb &
\multirow{3}{*}{Different} & \multirow{3}{*}{Different}
\\
& E = No Data \tab Freq =No Data   &    &  \\ \cline{2-2}
& 3NFAACi   & \multicolumn{2}{c}{\multirow{3}{*}
 {RMS = 0.13843}}
\\
& E = No Data \tab Freq =No Data   &    \multicolumn{2}{c}{}  \\ \hline
\multirow{4}{*}{\tab[2mm] 46 \tab[2mm]} & 3NFAACb &
\multirow{3}{*}{Different} & \multirow{3}{*}{Different}
\\
& E = No Data \tab Freq =No Data   &    &  \\ \cline{2-2}
& 3NFAACj   & \multicolumn{2}{c}{\multirow{3}{*}
 {RMS = 0.145048}}
\\
& E = No Data \tab Freq =No Data   &    \multicolumn{2}{c}{}  \\ \hline
\multirow{4}{*}{\tab[2mm] 47 \tab[2mm]} & 3NFAACb &
\multirow{3}{*}{Different} & \multirow{3}{*}{Different}
\\
& E = No Data \tab Freq =No Data   &    &  \\ \cline{2-2}
& 3NFAACk   & \multicolumn{2}{c}{\multirow{3}{*}
 {RMS = 0.0661391}}
\\
& E = No Data \tab Freq =No Data   &    \multicolumn{2}{c}{}  \\ \hline
\multirow{4}{*}{\tab[2mm] 48 \tab[2mm]} & 3NFAACb &
\multirow{3}{*}{Different} & \multirow{3}{*}{Different}
\\
& E = No Data \tab Freq =No Data   &    &  \\ \cline{2-2}
& 3NFAACl   & \multicolumn{2}{c}{\multirow{3}{*}
 {RMS = 0.018498}}
\\
& E = No Data \tab Freq =No Data   &    \multicolumn{2}{c}{}  \\ \hline
\multirow{4}{*}{\tab[2mm] 49 \tab[2mm]} & 3NFAACb &
\multirow{3}{*}{Different} & \multirow{3}{*}{Different}
\\
& E = No Data \tab Freq =No Data   &    &  \\ \cline{2-2}
& 3NFAACm   & \multicolumn{2}{c}{\multirow{3}{*}
 {RMS = 0.0981054}}
\\
& E = No Data \tab Freq =No Data   &    \multicolumn{2}{c}{}  \\ \hline
\end{tabular}
\newpage

\vtab[-2cm]
\tab[-2cm]
\begin{tabular}{c|m{8cm}|c|c}
\# & Moléculas & Restultado esperado & Resultado programa \\\\ \hline\hline
\multirow{4}{*}{\tab[2mm] 50 \tab[2mm]} & 3NFAACb &
\multirow{3}{*}{Different} & \multirow{3}{*}{Different}
\\
& E = No Data \tab Freq =No Data   &    &  \\ \cline{2-2}
& 3NFAACn   & \multicolumn{2}{c}{\multirow{3}{*}
 {RMS = 0.102455}}
\\
& E = No Data \tab Freq =No Data   &    \multicolumn{2}{c}{}  \\ \hline
\multirow{4}{*}{\tab[2mm] 51 \tab[2mm]} & 3NFAACb &
\multirow{3}{*}{Different} & \multirow{3}{*}{Different}
\\
& E = No Data \tab Freq =No Data   &    &  \\ \cline{2-2}
& 4NFAACa   & \multicolumn{2}{c}{\multirow{3}{*}
 {RMS = 0.274505}}
\\
& E = No Data \tab Freq =No Data   &    \multicolumn{2}{c}{}  \\ \hline
\multirow{4}{*}{\tab[2mm] 52 \tab[2mm]} & 3NFAACb &
\multirow{3}{*}{Different} & \multirow{3}{*}{Different}
\\
& E = No Data \tab Freq =No Data   &    &  \\ \cline{2-2}
& 4NFAACb   & \multicolumn{2}{c}{\multirow{3}{*}
 {RMS = 0.203714}}
\\
& E = No Data \tab Freq =No Data   &    \multicolumn{2}{c}{}  \\ \hline
\multirow{4}{*}{\tab[2mm] 53 \tab[2mm]} & 3NFAACb &
\multirow{3}{*}{Different} & \multirow{3}{*}{Different}
\\
& E = No Data \tab Freq =No Data   &    &  \\ \cline{2-2}
& 4NFAACc   & \multicolumn{2}{c}{\multirow{3}{*}
 {RMS = 0.036102}}
\\
& E = No Data \tab Freq =No Data   &    \multicolumn{2}{c}{}  \\ \hline
\multirow{4}{*}{\tab[2mm] 54 \tab[2mm]} & 3NFAACb &
\multirow{3}{*}{Different} & \multirow{3}{*}{Different}
\\
& E = No Data \tab Freq =No Data   &    &  \\ \cline{2-2}
& 4NFAACd   & \multicolumn{2}{c}{\multirow{3}{*}
 {RMS = 0.0398197}}
\\
& E = No Data \tab Freq =No Data   &    \multicolumn{2}{c}{}  \\ \hline
\multirow{4}{*}{\tab[2mm] 55 \tab[2mm]} & 3NFAACb &
\multirow{3}{*}{Different} & \multirow{3}{*}{Different}
\\
& E = No Data \tab Freq =No Data   &    &  \\ \cline{2-2}
& 4NFAACe   & \multicolumn{2}{c}{\multirow{3}{*}
 {RMS = 0.0318836}}
\\
& E = No Data \tab Freq =No Data   &    \multicolumn{2}{c}{}  \\ \hline
\multirow{4}{*}{\tab[2mm] 56 \tab[2mm]} & 3NFAACb &
\multirow{3}{*}{Different} & \multirow{3}{*}{Different}
\\
& E = No Data \tab Freq =No Data   &    &  \\ \cline{2-2}
& 4NFAACf   & \multicolumn{2}{c}{\multirow{3}{*}
 {RMS = 0.0635749}}
\\
& E = No Data \tab Freq =No Data   &    \multicolumn{2}{c}{}  \\ \hline
\end{tabular}
\newpage

\vtab[-2cm]
\tab[-2cm]
\begin{tabular}{c|m{8cm}|c|c}
\# & Moléculas & Restultado esperado & Resultado programa \\\\ \hline\hline
\multirow{4}{*}{\tab[2mm] 57 \tab[2mm]} & 3NFAACb &
\multirow{3}{*}{Different} & \multirow{3}{*}{Different}
\\
& E = No Data \tab Freq =No Data   &    &  \\ \cline{2-2}
& 4NFAACg   & \multicolumn{2}{c}{\multirow{3}{*}
 {RMS = 0.020869}}
\\
& E = No Data \tab Freq =No Data   &    \multicolumn{2}{c}{}  \\ \hline
\multirow{4}{*}{\tab[2mm] 58 \tab[2mm]} & 3NFAACb &
\multirow{3}{*}{Different} & \multirow{3}{*}{Different}
\\
& E = No Data \tab Freq =No Data   &    &  \\ \cline{2-2}
& 4NFAACi   & \multicolumn{2}{c}{\multirow{3}{*}
 {RMS = 0.0561899}}
\\
& E = No Data \tab Freq =No Data   &    \multicolumn{2}{c}{}  \\ \hline
\multirow{4}{*}{\tab[2mm] 59 \tab[2mm]} & 3NFAACb &
\multirow{3}{*}{Different} & \multirow{3}{*}{Different}
\\
& E = No Data \tab Freq =No Data   &    &  \\ \cline{2-2}
& 4NFAACj   & \multicolumn{2}{c}{\multirow{3}{*}
 {RMS = 0.0321081}}
\\
& E = No Data \tab Freq =No Data   &    \multicolumn{2}{c}{}  \\ \hline
\multirow{4}{*}{\tab[2mm] 60 \tab[2mm]} & 3NFAACb &
\multirow{3}{*}{Different} & \multirow{3}{*}{Different}
\\
& E = No Data \tab Freq =No Data   &    &  \\ \cline{2-2}
& 4NFAACl-3   & \multicolumn{2}{c}{\multirow{3}{*}
 {RMS = 0.0132449}}
\\
& E = No Data \tab Freq =No Data   &    \multicolumn{2}{c}{}  \\ \hline
\multirow{4}{*}{\tab[2mm] 61 \tab[2mm]} & 3NFAACc &
\multirow{3}{*}{Different} & \multirow{3}{*}{Different}
\\
& E = No Data \tab Freq =No Data   &    &  \\ \cline{2-2}
& 3NFAACd   & \multicolumn{2}{c}{\multirow{3}{*}
 {RMS = 0.0554495}}
\\
& E = No Data \tab Freq =No Data   &    \multicolumn{2}{c}{}  \\ \hline
\multirow{4}{*}{\tab[2mm] 62 \tab[2mm]} & 3NFAACc &
\multirow{3}{*}{Different} & \multirow{3}{*}{Different}
\\
& E = No Data \tab Freq =No Data   &    &  \\ \cline{2-2}
& 3NFAACe   & \multicolumn{2}{c}{\multirow{3}{*}
 {RMS = 0.195533}}
\\
& E = No Data \tab Freq =No Data   &    \multicolumn{2}{c}{}  \\ \hline
\multirow{4}{*}{\tab[2mm] 63 \tab[2mm]} & 3NFAACc &
\multirow{3}{*}{Different} & \multirow{3}{*}{Different}
\\
& E = No Data \tab Freq =No Data   &    &  \\ \cline{2-2}
& 3NFAACf   & \multicolumn{2}{c}{\multirow{3}{*}
 {RMS = 0.0879682}}
\\
& E = No Data \tab Freq =No Data   &    \multicolumn{2}{c}{}  \\ \hline
\end{tabular}
\newpage

\vtab[-2cm]
\tab[-2cm]
\begin{tabular}{c|m{8cm}|c|c}
\# & Moléculas & Restultado esperado & Resultado programa \\\\ \hline\hline
\multirow{4}{*}{\tab[2mm] 64 \tab[2mm]} & 3NFAACc &
\multirow{3}{*}{Different} & \multirow{3}{*}{Different}
\\
& E = No Data \tab Freq =No Data   &    &  \\ \cline{2-2}
& 3NFAACg   & \multicolumn{2}{c}{\multirow{3}{*}
 {RMS = 0.000316817}}
\\
& E = No Data \tab Freq =No Data   &    \multicolumn{2}{c}{}  \\ \hline
\multirow{4}{*}{\tab[2mm] 65 \tab[2mm]} & 3NFAACc &
\multirow{3}{*}{Different} & \multirow{3}{*}{Different}
\\
& E = No Data \tab Freq =No Data   &    &  \\ \cline{2-2}
& 3NFAACh   & \multicolumn{2}{c}{\multirow{3}{*}
 {RMS = 0.0978186}}
\\
& E = No Data \tab Freq =No Data   &    \multicolumn{2}{c}{}  \\ \hline
\multirow{4}{*}{\tab[2mm] 66 \tab[2mm]} & 3NFAACc &
\multirow{3}{*}{Different} & \multirow{3}{*}{Different}
\\
& E = No Data \tab Freq =No Data   &    &  \\ \cline{2-2}
& 3NFAACi   & \multicolumn{2}{c}{\multirow{3}{*}
 {RMS = 0.0503005}}
\\
& E = No Data \tab Freq =No Data   &    \multicolumn{2}{c}{}  \\ \hline
\multirow{4}{*}{\tab[2mm] 67 \tab[2mm]} & 3NFAACc &
\multirow{3}{*}{Different} & \multirow{3}{*}{Different}
\\
& E = No Data \tab Freq =No Data   &    &  \\ \cline{2-2}
& 3NFAACj   & \multicolumn{2}{c}{\multirow{3}{*}
 {RMS = 0.0569185}}
\\
& E = No Data \tab Freq =No Data   &    \multicolumn{2}{c}{}  \\ \hline
\multirow{4}{*}{\tab[2mm] 68 \tab[2mm]} & 3NFAACc &
\multirow{3}{*}{Different} & \multirow{3}{*}{Different}
\\
& E = No Data \tab Freq =No Data   &    &  \\ \cline{2-2}
& 3NFAACk   & \multicolumn{2}{c}{\multirow{3}{*}
 {RMS = 0.02199}}
\\
& E = No Data \tab Freq =No Data   &    \multicolumn{2}{c}{}  \\ \hline
\multirow{4}{*}{\tab[2mm] 69 \tab[2mm]} & 3NFAACc &
\multirow{3}{*}{Different} & \multirow{3}{*}{Different}
\\
& E = No Data \tab Freq =No Data   &    &  \\ \cline{2-2}
& 3NFAACl   & \multicolumn{2}{c}{\multirow{3}{*}
 {RMS = 0.069631}}
\\
& E = No Data \tab Freq =No Data   &    \multicolumn{2}{c}{}  \\ \hline
\multirow{4}{*}{\tab[2mm] 70 \tab[2mm]} & 3NFAACc &
\multirow{3}{*}{Different} & \multirow{3}{*}{Different}
\\
& E = No Data \tab Freq =No Data   &    &  \\ \cline{2-2}
& 3NFAACm   & \multicolumn{2}{c}{\multirow{3}{*}
 {RMS = 0.186234}}
\\
& E = No Data \tab Freq =No Data   &    \multicolumn{2}{c}{}  \\ \hline
\end{tabular}
\newpage

\vtab[-2cm]
\tab[-2cm]
\begin{tabular}{c|m{8cm}|c|c}
\# & Moléculas & Restultado esperado & Resultado programa \\\\ \hline\hline
\multirow{4}{*}{\tab[2mm] 71 \tab[2mm]} & 3NFAACc &
\multirow{3}{*}{\textcolor{Red}{\bf Stereoisomer}} & \multirow{3}{*}{\textcolor{Red}{\bf Enantiomers}}
\\
& E = No Data \tab Freq =No Data   &    &  \\ \cline{2-2}
& 3NFAACn   & \multicolumn{2}{c}{\multirow{3}{*}
 {RMS = 0.01522269}}
\\
& E = No Data \tab Freq =No Data   &    \multicolumn{2}{c}{}  \\ \hline
\multirow{4}{*}{\tab[2mm] 72 \tab[2mm]} & 3NFAACc &
\multirow{3}{*}{Different} & \multirow{3}{*}{Different}
\\
& E = No Data \tab Freq =No Data   &    &  \\ \cline{2-2}
& 4NFAACa   & \multicolumn{2}{c}{\multirow{3}{*}
 {RMS = 0.186376}}
\\
& E = No Data \tab Freq =No Data   &    \multicolumn{2}{c}{}  \\ \hline
\multirow{4}{*}{\tab[2mm] 73 \tab[2mm]} & 3NFAACc &
\multirow{3}{*}{Different} & \multirow{3}{*}{Different}
\\
& E = No Data \tab Freq =No Data   &    &  \\ \cline{2-2}
& 4NFAACb   & \multicolumn{2}{c}{\multirow{3}{*}
 {RMS = 0.291844}}
\\
& E = No Data \tab Freq =No Data   &    \multicolumn{2}{c}{}  \\ \hline
\multirow{4}{*}{\tab[2mm] 74 \tab[2mm]} & 3NFAACc &
\multirow{3}{*}{Different} & \multirow{3}{*}{Different}
\\
& E = No Data \tab Freq =No Data   &    &  \\ \cline{2-2}
& 4NFAACc   & \multicolumn{2}{c}{\multirow{3}{*}
 {RMS = 0.0520271}}
\\
& E = No Data \tab Freq =No Data   &    \multicolumn{2}{c}{}  \\ \hline
\multirow{4}{*}{\tab[2mm] 75 \tab[2mm]} & 3NFAACc &
\multirow{3}{*}{Different} & \multirow{3}{*}{Different}
\\
& E = No Data \tab Freq =No Data   &    &  \\ \cline{2-2}
& 4NFAACd   & \multicolumn{2}{c}{\multirow{3}{*}
 {RMS = 0.0483094}}
\\
& E = No Data \tab Freq =No Data   &    \multicolumn{2}{c}{}  \\ \hline
\multirow{4}{*}{\tab[2mm] 76 \tab[2mm]} & 3NFAACc &
\multirow{3}{*}{Different} & \multirow{3}{*}{Different}
\\
& E = No Data \tab Freq =No Data   &    &  \\ \cline{2-2}
& 4NFAACe   & \multicolumn{2}{c}{\multirow{3}{*}
 {RMS = 0.0562454}}
\\
& E = No Data \tab Freq =No Data   &    \multicolumn{2}{c}{}  \\ \hline
\multirow{4}{*}{\tab[2mm] 77 \tab[2mm]} & 3NFAACc &
\multirow{3}{*}{Different} & \multirow{3}{*}{Different}
\\
& E = No Data \tab Freq =No Data   &    &  \\ \cline{2-2}
& 4NFAACf   & \multicolumn{2}{c}{\multirow{3}{*}
 {RMS = 0.151704}}
\\
& E = No Data \tab Freq =No Data   &    \multicolumn{2}{c}{}  \\ \hline
\end{tabular}
\newpage

\vtab[-2cm]
\tab[-2cm]
\begin{tabular}{c|m{8cm}|c|c}
\# & Moléculas & Restultado esperado & Resultado programa \\\\ \hline\hline
\multirow{4}{*}{\tab[2mm] 78 \tab[2mm]} & 3NFAACc &
\multirow{3}{*}{Different} & \multirow{3}{*}{Different}
\\
& E = No Data \tab Freq =No Data   &    &  \\ \cline{2-2}
& 4NFAACg   & \multicolumn{2}{c}{\multirow{3}{*}
 {RMS = 0.0672601}}
\\
& E = No Data \tab Freq =No Data   &    \multicolumn{2}{c}{}  \\ \hline
\multirow{4}{*}{\tab[2mm] 79 \tab[2mm]} & 3NFAACc &
\multirow{3}{*}{Different} & \multirow{3}{*}{Different}
\\
& E = No Data \tab Freq =No Data   &    &  \\ \cline{2-2}
& 4NFAACi   & \multicolumn{2}{c}{\multirow{3}{*}
 {RMS = 0.0319392}}
\\
& E = No Data \tab Freq =No Data   &    \multicolumn{2}{c}{}  \\ \hline
\multirow{4}{*}{\tab[2mm] 80 \tab[2mm]} & 3NFAACc &
\multirow{3}{*}{Different} & \multirow{3}{*}{Different}
\\
& E = No Data \tab Freq =No Data   &    &  \\ \cline{2-2}
& 4NFAACj   & \multicolumn{2}{c}{\multirow{3}{*}
 {RMS = 0.120237}}
\\
& E = No Data \tab Freq =No Data   &    \multicolumn{2}{c}{}  \\ \hline
\multirow{4}{*}{\tab[2mm] 81 \tab[2mm]} & 3NFAACc &
\multirow{3}{*}{Different} & \multirow{3}{*}{Different}
\\
& E = No Data \tab Freq =No Data   &    &  \\ \cline{2-2}
& 4NFAACl-3   & \multicolumn{2}{c}{\multirow{3}{*}
 {RMS = 0.0748842}}
\\
& E = No Data \tab Freq =No Data   &    \multicolumn{2}{c}{}  \\ \hline
\multirow{4}{*}{\tab[2mm] 82 \tab[2mm]} & 3NFAACd &
\multirow{3}{*}{\textcolor{Red}{\bf Equal}} & \multirow{3}{*}{\textcolor{Red}{\bf Enantiomers}}
\\
& E = No Data \tab Freq =No Data   &    &  \\ \cline{2-2}
& 3NFAACe   & \multicolumn{2}{c}{\multirow{3}{*}
 {RMS = 0.0004675477}}
\\
& E = No Data \tab Freq =No Data   &    \multicolumn{2}{c}{}  \\ \hline
\multirow{4}{*}{\tab[2mm] 83 \tab[2mm]} & 3NFAACd &
\multirow{3}{*}{Different} & \multirow{3}{*}{Different}
\\
& E = No Data \tab Freq =No Data   &    &  \\ \cline{2-2}
& 3NFAACf   & \multicolumn{2}{c}{\multirow{3}{*}
 {RMS = 0.143418}}
\\
& E = No Data \tab Freq =No Data   &    \multicolumn{2}{c}{}  \\ \hline
\multirow{4}{*}{\tab[2mm] 84 \tab[2mm]} & 3NFAACd &
\multirow{3}{*}{Different} & \multirow{3}{*}{Different}
\\
& E = No Data \tab Freq =No Data   &    &  \\ \cline{2-2}
& 3NFAACg   & \multicolumn{2}{c}{\multirow{3}{*}
 {RMS = 0.0557663}}
\\
& E = No Data \tab Freq =No Data   &    \multicolumn{2}{c}{}  \\ \hline
\end{tabular}
\newpage

\vtab[-2cm]
\tab[-2cm]
\begin{tabular}{c|m{8cm}|c|c}
\# & Moléculas & Restultado esperado & Resultado programa \\\\ \hline\hline
\multirow{4}{*}{\tab[2mm] 85 \tab[2mm]} & 3NFAACd &
\multirow{3}{*}{Different} & \multirow{3}{*}{Different}
\\
& E = No Data \tab Freq =No Data   &    &  \\ \cline{2-2}
& 3NFAACh   & \multicolumn{2}{c}{\multirow{3}{*}
 {RMS = 0.153268}}
\\
& E = No Data \tab Freq =No Data   &    \multicolumn{2}{c}{}  \\ \hline
\multirow{4}{*}{\tab[2mm] 86 \tab[2mm]} & 3NFAACd &
\multirow{3}{*}{Different} & \multirow{3}{*}{Different}
\\
& E = No Data \tab Freq =No Data   &    &  \\ \cline{2-2}
& 3NFAACi   & \multicolumn{2}{c}{\multirow{3}{*}
 {RMS = 0.00514902}}
\\
& E = No Data \tab Freq =No Data   &    \multicolumn{2}{c}{}  \\ \hline
\multirow{4}{*}{\tab[2mm] 87 \tab[2mm]} & 3NFAACd &
\multirow{3}{*}{Different} & \multirow{3}{*}{Different}
\\
& E = No Data \tab Freq =No Data   &    &  \\ \cline{2-2}
& 3NFAACj   & \multicolumn{2}{c}{\multirow{3}{*}
 {RMS = 0.00146894}}
\\
& E = No Data \tab Freq =No Data   &    \multicolumn{2}{c}{}  \\ \hline
\multirow{4}{*}{\tab[2mm] 88 \tab[2mm]} & 3NFAACd &
\multirow{3}{*}{Different} & \multirow{3}{*}{Different}
\\
& E = No Data \tab Freq =No Data   &    &  \\ \cline{2-2}
& 3NFAACk   & \multicolumn{2}{c}{\multirow{3}{*}
 {RMS = 0.0774395}}
\\
& E = No Data \tab Freq =No Data   &    \multicolumn{2}{c}{}  \\ \hline
\multirow{4}{*}{\tab[2mm] 89 \tab[2mm]} & 3NFAACd &
\multirow{3}{*}{Different} & \multirow{3}{*}{Different}
\\
& E = No Data \tab Freq =No Data   &    &  \\ \cline{2-2}
& 3NFAACl   & \multicolumn{2}{c}{\multirow{3}{*}
 {RMS = 0.125081}}
\\
& E = No Data \tab Freq =No Data   &    \multicolumn{2}{c}{}  \\ \hline
\multirow{4}{*}{\tab[2mm] 90 \tab[2mm]} & 3NFAACd &
\multirow{3}{*}{Different} & \multirow{3}{*}{Different}
\\
& E = No Data \tab Freq =No Data   &    &  \\ \cline{2-2}
& 3NFAACm   & \multicolumn{2}{c}{\multirow{3}{*}
 {RMS = 0.241684}}
\\
& E = No Data \tab Freq =No Data   &    \multicolumn{2}{c}{}  \\ \hline
\multirow{4}{*}{\tab[2mm] 91 \tab[2mm]} & 3NFAACd &
\multirow{3}{*}{Different} & \multirow{3}{*}{Different}
\\
& E = No Data \tab Freq =No Data   &    &  \\ \cline{2-2}
& 3NFAACn   & \multicolumn{2}{c}{\multirow{3}{*}
 {RMS = 0.246033}}
\\
& E = No Data \tab Freq =No Data   &    \multicolumn{2}{c}{}  \\ \hline
\end{tabular}
\newpage

\vtab[-2cm]
\tab[-2cm]
\begin{tabular}{c|m{8cm}|c|c}
\# & Moléculas & Restultado esperado & Resultado programa \\\\ \hline\hline
\multirow{4}{*}{\tab[2mm] 92 \tab[2mm]} & 3NFAACd &
\multirow{3}{*}{Different} & \multirow{3}{*}{Different}
\\
& E = No Data \tab Freq =No Data   &    &  \\ \cline{2-2}
& 4NFAACa   & \multicolumn{2}{c}{\multirow{3}{*}
 {RMS = 0.130927}}
\\
& E = No Data \tab Freq =No Data   &    \multicolumn{2}{c}{}  \\ \hline
\multirow{4}{*}{\tab[2mm] 93 \tab[2mm]} & 3NFAACd &
\multirow{3}{*}{Different} & \multirow{3}{*}{Different}
\\
& E = No Data \tab Freq =No Data   &    &  \\ \cline{2-2}
& 4NFAACb   & \multicolumn{2}{c}{\multirow{3}{*}
 {RMS = 0.347293}}
\\
& E = No Data \tab Freq =No Data   &    \multicolumn{2}{c}{}  \\ \hline
\multirow{4}{*}{\tab[2mm] 94 \tab[2mm]} & 3NFAACd &
\multirow{3}{*}{Different} & \multirow{3}{*}{Different}
\\
& E = No Data \tab Freq =No Data   &    &  \\ \cline{2-2}
& 4NFAACc   & \multicolumn{2}{c}{\multirow{3}{*}
 {RMS = 0.107477}}
\\
& E = No Data \tab Freq =No Data   &    \multicolumn{2}{c}{}  \\ \hline
\multirow{4}{*}{\tab[2mm] 95 \tab[2mm]} & 3NFAACd &
\multirow{3}{*}{Different} & \multirow{3}{*}{Different}
\\
& E = No Data \tab Freq =No Data   &    &  \\ \cline{2-2}
& 4NFAACd   & \multicolumn{2}{c}{\multirow{3}{*}
 {RMS = 0.103759}}
\\
& E = No Data \tab Freq =No Data   &    \multicolumn{2}{c}{}  \\ \hline
\multirow{4}{*}{\tab[2mm] 96 \tab[2mm]} & 3NFAACd &
\multirow{3}{*}{Different} & \multirow{3}{*}{Different}
\\
& E = No Data \tab Freq =No Data   &    &  \\ \cline{2-2}
& 4NFAACe   & \multicolumn{2}{c}{\multirow{3}{*}
 {RMS = 0.111695}}
\\
& E = No Data \tab Freq =No Data   &    \multicolumn{2}{c}{}  \\ \hline
\multirow{4}{*}{\tab[2mm] 97 \tab[2mm]} & 3NFAACd &
\multirow{3}{*}{Different} & \multirow{3}{*}{Different}
\\
& E = No Data \tab Freq =No Data   &    &  \\ \cline{2-2}
& 4NFAACf   & \multicolumn{2}{c}{\multirow{3}{*}
 {RMS = 0.207153}}
\\
& E = No Data \tab Freq =No Data   &    \multicolumn{2}{c}{}  \\ \hline
\multirow{4}{*}{\tab[2mm] 98 \tab[2mm]} & 3NFAACd &
\multirow{3}{*}{Different} & \multirow{3}{*}{Different}
\\
& E = No Data \tab Freq =No Data   &    &  \\ \cline{2-2}
& 4NFAACg   & \multicolumn{2}{c}{\multirow{3}{*}
 {RMS = 0.12271}}
\\
& E = No Data \tab Freq =No Data   &    \multicolumn{2}{c}{}  \\ \hline
\end{tabular}
\newpage

\vtab[-2cm]
\tab[-2cm]
\begin{tabular}{c|m{8cm}|c|c}
\# & Moléculas & Restultado esperado & Resultado programa \\\\ \hline\hline
\multirow{4}{*}{\tab[2mm] 99 \tab[2mm]} & 3NFAACd &
\multirow{3}{*}{Different} & \multirow{3}{*}{Different}
\\
& E = No Data \tab Freq =No Data   &    &  \\ \cline{2-2}
& 4NFAACi   & \multicolumn{2}{c}{\multirow{3}{*}
 {RMS = 0.0873887}}
\\
& E = No Data \tab Freq =No Data   &    \multicolumn{2}{c}{}  \\ \hline
\multirow{4}{*}{\tab[2mm] 100 \tab[2mm]} & 3NFAACd &
\multirow{3}{*}{Different} & \multirow{3}{*}{Different}
\\
& E = No Data \tab Freq =No Data   &    &  \\ \cline{2-2}
& 4NFAACj   & \multicolumn{2}{c}{\multirow{3}{*}
 {RMS = 0.175687}}
\\
& E = No Data \tab Freq =No Data   &    \multicolumn{2}{c}{}  \\ \hline
\multirow{4}{*}{\tab[2mm] 101 \tab[2mm]} & 3NFAACd &
\multirow{3}{*}{Different} & \multirow{3}{*}{Different}
\\
& E = No Data \tab Freq =No Data   &    &  \\ \cline{2-2}
& 4NFAACl-3   & \multicolumn{2}{c}{\multirow{3}{*}
 {RMS = 0.130334}}
\\
& E = No Data \tab Freq =No Data   &    \multicolumn{2}{c}{}  \\ \hline
\multirow{4}{*}{\tab[2mm] 102 \tab[2mm]} & 3NFAACe &
\multirow{3}{*}{Different} & \multirow{3}{*}{Different}
\\
& E = No Data \tab Freq =No Data   &    &  \\ \cline{2-2}
& 3NFAACf   & \multicolumn{2}{c}{\multirow{3}{*}
 {RMS = 0.107565}}
\\
& E = No Data \tab Freq =No Data   &    \multicolumn{2}{c}{}  \\ \hline
\multirow{4}{*}{\tab[2mm] 103 \tab[2mm]} & 3NFAACe &
\multirow{3}{*}{Different} & \multirow{3}{*}{Different}
\\
& E = No Data \tab Freq =No Data   &    &  \\ \cline{2-2}
& 3NFAACg   & \multicolumn{2}{c}{\multirow{3}{*}
 {RMS = 0.195216}}
\\
& E = No Data \tab Freq =No Data   &    \multicolumn{2}{c}{}  \\ \hline
\multirow{4}{*}{\tab[2mm] 104 \tab[2mm]} & 3NFAACe &
\multirow{3}{*}{Different} & \multirow{3}{*}{Different}
\\
& E = No Data \tab Freq =No Data   &    &  \\ \cline{2-2}
& 3NFAACh   & \multicolumn{2}{c}{\multirow{3}{*}
 {RMS = 0.0977146}}
\\
& E = No Data \tab Freq =No Data   &    \multicolumn{2}{c}{}  \\ \hline
\multirow{4}{*}{\tab[2mm] 105 \tab[2mm]} & 3NFAACe &
\multirow{3}{*}{Different} & \multirow{3}{*}{Different}
\\
& E = No Data \tab Freq =No Data   &    &  \\ \cline{2-2}
& 3NFAACi   & \multicolumn{2}{c}{\multirow{3}{*}
 {RMS = 0.245834}}
\\
& E = No Data \tab Freq =No Data   &    \multicolumn{2}{c}{}  \\ \hline
\end{tabular}
\newpage

\vtab[-2cm]
\tab[-2cm]
\begin{tabular}{c|m{8cm}|c|c}
\# & Moléculas & Restultado esperado & Resultado programa \\\\ \hline\hline
\multirow{4}{*}{\tab[2mm] 106 \tab[2mm]} & 3NFAACe &
\multirow{3}{*}{Different} & \multirow{3}{*}{Different}
\\
& E = No Data \tab Freq =No Data   &    &  \\ \cline{2-2}
& 3NFAACj   & \multicolumn{2}{c}{\multirow{3}{*}
 {RMS = 0.252452}}
\\
& E = No Data \tab Freq =No Data   &    \multicolumn{2}{c}{}  \\ \hline
\multirow{4}{*}{\tab[2mm] 107 \tab[2mm]} & 3NFAACe &
\multirow{3}{*}{Different} & \multirow{3}{*}{Different}
\\
& E = No Data \tab Freq =No Data   &    &  \\ \cline{2-2}
& 3NFAACk   & \multicolumn{2}{c}{\multirow{3}{*}
 {RMS = 0.173543}}
\\
& E = No Data \tab Freq =No Data   &    \multicolumn{2}{c}{}  \\ \hline
\multirow{4}{*}{\tab[2mm] 108 \tab[2mm]} & 3NFAACe &
\multirow{3}{*}{Different} & \multirow{3}{*}{Different}
\\
& E = No Data \tab Freq =No Data   &    &  \\ \cline{2-2}
& 3NFAACl   & \multicolumn{2}{c}{\multirow{3}{*}
 {RMS = 0.125902}}
\\
& E = No Data \tab Freq =No Data   &    \multicolumn{2}{c}{}  \\ \hline
\multirow{4}{*}{\tab[2mm] 109 \tab[2mm]} & 3NFAACe &
\multirow{3}{*}{Different} & \multirow{3}{*}{Different}
\\
& E = No Data \tab Freq =No Data   &    &  \\ \cline{2-2}
& 3NFAACm   & \multicolumn{2}{c}{\multirow{3}{*}
 {RMS = 0.00929873}}
\\
& E = No Data \tab Freq =No Data   &    \multicolumn{2}{c}{}  \\ \hline
\multirow{4}{*}{\tab[2mm] 110 \tab[2mm]} & 3NFAACe &
\multirow{3}{*}{Different} & \multirow{3}{*}{Different}
\\
& E = No Data \tab Freq =No Data   &    &  \\ \cline{2-2}
& 3NFAACn   & \multicolumn{2}{c}{\multirow{3}{*}
 {RMS = 0.00494937}}
\\
& E = No Data \tab Freq =No Data   &    \multicolumn{2}{c}{}  \\ \hline
\multirow{4}{*}{\tab[2mm] 111 \tab[2mm]} & 3NFAACe &
\multirow{3}{*}{Different} & \multirow{3}{*}{Different}
\\
& E = No Data \tab Freq =No Data   &    &  \\ \cline{2-2}
& 4NFAACa   & \multicolumn{2}{c}{\multirow{3}{*}
 {RMS = 0.381909}}
\\
& E = No Data \tab Freq =No Data   &    \multicolumn{2}{c}{}  \\ \hline
\multirow{4}{*}{\tab[2mm] 112 \tab[2mm]} & 3NFAACe &
\multirow{3}{*}{Different} & \multirow{3}{*}{Different}
\\
& E = No Data \tab Freq =No Data   &    &  \\ \cline{2-2}
& 4NFAACb   & \multicolumn{2}{c}{\multirow{3}{*}
 {RMS = 0.0963103}}
\\
& E = No Data \tab Freq =No Data   &    \multicolumn{2}{c}{}  \\ \hline
\end{tabular}
\newpage

\vtab[-2cm]
\tab[-2cm]
\begin{tabular}{c|m{8cm}|c|c}
\# & Moléculas & Restultado esperado & Resultado programa \\\\ \hline\hline
\multirow{4}{*}{\tab[2mm] 113 \tab[2mm]} & 3NFAACe &
\multirow{3}{*}{Different} & \multirow{3}{*}{Different}
\\
& E = No Data \tab Freq =No Data   &    &  \\ \cline{2-2}
& 4NFAACc   & \multicolumn{2}{c}{\multirow{3}{*}
 {RMS = 0.143506}}
\\
& E = No Data \tab Freq =No Data   &    \multicolumn{2}{c}{}  \\ \hline
\multirow{4}{*}{\tab[2mm] 114 \tab[2mm]} & 3NFAACe &
\multirow{3}{*}{Different} & \multirow{3}{*}{Different}
\\
& E = No Data \tab Freq =No Data   &    &  \\ \cline{2-2}
& 4NFAACd   & \multicolumn{2}{c}{\multirow{3}{*}
 {RMS = 0.147224}}
\\
& E = No Data \tab Freq =No Data   &    \multicolumn{2}{c}{}  \\ \hline
\multirow{4}{*}{\tab[2mm] 115 \tab[2mm]} & 3NFAACe &
\multirow{3}{*}{Different} & \multirow{3}{*}{Different}
\\
& E = No Data \tab Freq =No Data   &    &  \\ \cline{2-2}
& 4NFAACe   & \multicolumn{2}{c}{\multirow{3}{*}
 {RMS = 0.139288}}
\\
& E = No Data \tab Freq =No Data   &    \multicolumn{2}{c}{}  \\ \hline
\multirow{4}{*}{\tab[2mm] 116 \tab[2mm]} & 3NFAACe &
\multirow{3}{*}{Different} & \multirow{3}{*}{Different}
\\
& E = No Data \tab Freq =No Data   &    &  \\ \cline{2-2}
& 4NFAACf   & \multicolumn{2}{c}{\multirow{3}{*}
 {RMS = 0.0438293}}
\\
& E = No Data \tab Freq =No Data   &    \multicolumn{2}{c}{}  \\ \hline
\multirow{4}{*}{\tab[2mm] 117 \tab[2mm]} & 3NFAACe &
\multirow{3}{*}{Different} & \multirow{3}{*}{Different}
\\
& E = No Data \tab Freq =No Data   &    &  \\ \cline{2-2}
& 4NFAACg   & \multicolumn{2}{c}{\multirow{3}{*}
 {RMS = 0.128273}}
\\
& E = No Data \tab Freq =No Data   &    \multicolumn{2}{c}{}  \\ \hline
\multirow{4}{*}{\tab[2mm] 118 \tab[2mm]} & 3NFAACe &
\multirow{3}{*}{Different} & \multirow{3}{*}{Different}
\\
& E = No Data \tab Freq =No Data   &    &  \\ \cline{2-2}
& 4NFAACi   & \multicolumn{2}{c}{\multirow{3}{*}
 {RMS = 0.163594}}
\\
& E = No Data \tab Freq =No Data   &    \multicolumn{2}{c}{}  \\ \hline
\multirow{4}{*}{\tab[2mm] 119 \tab[2mm]} & 3NFAACe &
\multirow{3}{*}{Different} & \multirow{3}{*}{Different}
\\
& E = No Data \tab Freq =No Data   &    &  \\ \cline{2-2}
& 4NFAACj   & \multicolumn{2}{c}{\multirow{3}{*}
 {RMS = 0.075296}}
\\
& E = No Data \tab Freq =No Data   &    \multicolumn{2}{c}{}  \\ \hline
\end{tabular}
\newpage

\vtab[-2cm]
\tab[-2cm]
\begin{tabular}{c|m{8cm}|c|c}
\# & Moléculas & Restultado esperado & Resultado programa \\\\ \hline\hline
\multirow{4}{*}{\tab[2mm] 120 \tab[2mm]} & 3NFAACe &
\multirow{3}{*}{Different} & \multirow{3}{*}{Different}
\\
& E = No Data \tab Freq =No Data   &    &  \\ \cline{2-2}
& 4NFAACl-3   & \multicolumn{2}{c}{\multirow{3}{*}
 {RMS = 0.120649}}
\\
& E = No Data \tab Freq =No Data   &    \multicolumn{2}{c}{}  \\ \hline
\multirow{4}{*}{\tab[2mm] 121 \tab[2mm]} & 3NFAACf &
\multirow{3}{*}{Different} & \multirow{3}{*}{Different}
\\
& E = No Data \tab Freq =No Data   &    &  \\ \cline{2-2}
& 3NFAACg   & \multicolumn{2}{c}{\multirow{3}{*}
 {RMS = 0.0876514}}
\\
& E = No Data \tab Freq =No Data   &    \multicolumn{2}{c}{}  \\ \hline
\multirow{4}{*}{\tab[2mm] 122 \tab[2mm]} & 3NFAACf &
\multirow{3}{*}{Different} & \multirow{3}{*}{Different}
\\
& E = No Data \tab Freq =No Data   &    &  \\ \cline{2-2}
& 3NFAACh   & \multicolumn{2}{c}{\multirow{3}{*}
 {RMS = 0.00985042}}
\\
& E = No Data \tab Freq =No Data   &    \multicolumn{2}{c}{}  \\ \hline
\multirow{4}{*}{\tab[2mm] 123 \tab[2mm]} & 3NFAACf &
\multirow{3}{*}{Different} & \multirow{3}{*}{Different}
\\
& E = No Data \tab Freq =No Data   &    &  \\ \cline{2-2}
& 3NFAACi   & \multicolumn{2}{c}{\multirow{3}{*}
 {RMS = 0.138269}}
\\
& E = No Data \tab Freq =No Data   &    \multicolumn{2}{c}{}  \\ \hline
\multirow{4}{*}{\tab[2mm] 124 \tab[2mm]} & 3NFAACf &
\multirow{3}{*}{Different} & \multirow{3}{*}{Different}
\\
& E = No Data \tab Freq =No Data   &    &  \\ \cline{2-2}
& 3NFAACj   & \multicolumn{2}{c}{\multirow{3}{*}
 {RMS = 0.144887}}
\\
& E = No Data \tab Freq =No Data   &    \multicolumn{2}{c}{}  \\ \hline
\multirow{4}{*}{\tab[2mm] 125 \tab[2mm]} & 3NFAACf &
\multirow{3}{*}{Different} & \multirow{3}{*}{Different}
\\
& E = No Data \tab Freq =No Data   &    &  \\ \cline{2-2}
& 3NFAACk   & \multicolumn{2}{c}{\multirow{3}{*}
 {RMS = 0.0659782}}
\\
& E = No Data \tab Freq =No Data   &    \multicolumn{2}{c}{}  \\ \hline
\multirow{4}{*}{\tab[2mm] 126 \tab[2mm]} & 3NFAACf &
\multirow{3}{*}{Different} & \multirow{3}{*}{Different}
\\
& E = No Data \tab Freq =No Data   &    &  \\ \cline{2-2}
& 3NFAACl   & \multicolumn{2}{c}{\multirow{3}{*}
 {RMS = 0.0183372}}
\\
& E = No Data \tab Freq =No Data   &    \multicolumn{2}{c}{}  \\ \hline
\end{tabular}
\newpage

\vtab[-2cm]
\tab[-2cm]
\begin{tabular}{c|m{8cm}|c|c}
\# & Moléculas & Restultado esperado & Resultado programa \\\\ \hline\hline
\multirow{4}{*}{\tab[2mm] 127 \tab[2mm]} & 3NFAACf &
\multirow{3}{*}{Different} & \multirow{3}{*}{Different}
\\
& E = No Data \tab Freq =No Data   &    &  \\ \cline{2-2}
& 3NFAACm   & \multicolumn{2}{c}{\multirow{3}{*}
 {RMS = 0.0982663}}
\\
& E = No Data \tab Freq =No Data   &    \multicolumn{2}{c}{}  \\ \hline
\multirow{4}{*}{\tab[2mm] 128 \tab[2mm]} & 3NFAACf &
\multirow{3}{*}{Different} & \multirow{3}{*}{Different}
\\
& E = No Data \tab Freq =No Data   &    &  \\ \cline{2-2}
& 3NFAACn   & \multicolumn{2}{c}{\multirow{3}{*}
 {RMS = 0.102616}}
\\
& E = No Data \tab Freq =No Data   &    \multicolumn{2}{c}{}  \\ \hline
\multirow{4}{*}{\tab[2mm] 129 \tab[2mm]} & 3NFAACf &
\multirow{3}{*}{Different} & \multirow{3}{*}{Different}
\\
& E = No Data \tab Freq =No Data   &    &  \\ \cline{2-2}
& 4NFAACa   & \multicolumn{2}{c}{\multirow{3}{*}
 {RMS = 0.274344}}
\\
& E = No Data \tab Freq =No Data   &    \multicolumn{2}{c}{}  \\ \hline
\multirow{4}{*}{\tab[2mm] 130 \tab[2mm]} & 3NFAACf &
\multirow{3}{*}{Different} & \multirow{3}{*}{Different}
\\
& E = No Data \tab Freq =No Data   &    &  \\ \cline{2-2}
& 4NFAACb   & \multicolumn{2}{c}{\multirow{3}{*}
 {RMS = 0.203875}}
\\
& E = No Data \tab Freq =No Data   &    \multicolumn{2}{c}{}  \\ \hline
\multirow{4}{*}{\tab[2mm] 131 \tab[2mm]} & 3NFAACf &
\multirow{3}{*}{Different} & \multirow{3}{*}{Different}
\\
& E = No Data \tab Freq =No Data   &    &  \\ \cline{2-2}
& 4NFAACc   & \multicolumn{2}{c}{\multirow{3}{*}
 {RMS = 0.0359411}}
\\
& E = No Data \tab Freq =No Data   &    \multicolumn{2}{c}{}  \\ \hline
\multirow{4}{*}{\tab[2mm] 132 \tab[2mm]} & 3NFAACf &
\multirow{3}{*}{Different} & \multirow{3}{*}{Different}
\\
& E = No Data \tab Freq =No Data   &    &  \\ \cline{2-2}
& 4NFAACd   & \multicolumn{2}{c}{\multirow{3}{*}
 {RMS = 0.0396588}}
\\
& E = No Data \tab Freq =No Data   &    \multicolumn{2}{c}{}  \\ \hline
\multirow{4}{*}{\tab[2mm] 133 \tab[2mm]} & 3NFAACf &
\multirow{3}{*}{Different} & \multirow{3}{*}{Different}
\\
& E = No Data \tab Freq =No Data   &    &  \\ \cline{2-2}
& 4NFAACe   & \multicolumn{2}{c}{\multirow{3}{*}
 {RMS = 0.0317228}}
\\
& E = No Data \tab Freq =No Data   &    \multicolumn{2}{c}{}  \\ \hline
\end{tabular}
\newpage

\vtab[-2cm]
\tab[-2cm]
\begin{tabular}{c|m{8cm}|c|c}
\# & Moléculas & Restultado esperado & Resultado programa \\\\ \hline\hline
\multirow{4}{*}{\tab[2mm] 134 \tab[2mm]} & 3NFAACf &
\multirow{3}{*}{Different} & \multirow{3}{*}{Different}
\\
& E = No Data \tab Freq =No Data   &    &  \\ \cline{2-2}
& 4NFAACf   & \multicolumn{2}{c}{\multirow{3}{*}
 {RMS = 0.0637358}}
\\
& E = No Data \tab Freq =No Data   &    \multicolumn{2}{c}{}  \\ \hline
\multirow{4}{*}{\tab[2mm] 135 \tab[2mm]} & 3NFAACf &
\multirow{3}{*}{Different} & \multirow{3}{*}{Different}
\\
& E = No Data \tab Freq =No Data   &    &  \\ \cline{2-2}
& 4NFAACg   & \multicolumn{2}{c}{\multirow{3}{*}
 {RMS = 0.0207081}}
\\
& E = No Data \tab Freq =No Data   &    \multicolumn{2}{c}{}  \\ \hline
\multirow{4}{*}{\tab[2mm] 136 \tab[2mm]} & 3NFAACf &
\multirow{3}{*}{Different} & \multirow{3}{*}{Different}
\\
& E = No Data \tab Freq =No Data   &    &  \\ \cline{2-2}
& 4NFAACi   & \multicolumn{2}{c}{\multirow{3}{*}
 {RMS = 0.056029}}
\\
& E = No Data \tab Freq =No Data   &    \multicolumn{2}{c}{}  \\ \hline
\multirow{4}{*}{\tab[2mm] 137 \tab[2mm]} & 3NFAACf &
\multirow{3}{*}{Different} & \multirow{3}{*}{Different}
\\
& E = No Data \tab Freq =No Data   &    &  \\ \cline{2-2}
& 4NFAACj   & \multicolumn{2}{c}{\multirow{3}{*}
 {RMS = 0.032269}}
\\
& E = No Data \tab Freq =No Data   &    \multicolumn{2}{c}{}  \\ \hline
\multirow{4}{*}{\tab[2mm] 138 \tab[2mm]} & 3NFAACf &
\multirow{3}{*}{Different} & \multirow{3}{*}{Different}
\\
& E = No Data \tab Freq =No Data   &    &  \\ \cline{2-2}
& 4NFAACl-3   & \multicolumn{2}{c}{\multirow{3}{*}
 {RMS = 0.013084}}
\\
& E = No Data \tab Freq =No Data   &    \multicolumn{2}{c}{}  \\ \hline
\multirow{4}{*}{\tab[2mm] 139 \tab[2mm]} & 3NFAACg &
\multirow{3}{*}{Different} & \multirow{3}{*}{Different}
\\
& E = No Data \tab Freq =No Data   &    &  \\ \cline{2-2}
& 3NFAACh   & \multicolumn{2}{c}{\multirow{3}{*}
 {RMS = 0.0975018}}
\\
& E = No Data \tab Freq =No Data   &    \multicolumn{2}{c}{}  \\ \hline
\multirow{4}{*}{\tab[2mm] 140 \tab[2mm]} & 3NFAACg &
\multirow{3}{*}{Different} & \multirow{3}{*}{Different}
\\
& E = No Data \tab Freq =No Data   &    &  \\ \cline{2-2}
& 3NFAACi   & \multicolumn{2}{c}{\multirow{3}{*}
 {RMS = 0.0506173}}
\\
& E = No Data \tab Freq =No Data   &    \multicolumn{2}{c}{}  \\ \hline
\end{tabular}
\newpage

\vtab[-2cm]
\tab[-2cm]
\begin{tabular}{c|m{8cm}|c|c}
\# & Moléculas & Restultado esperado & Resultado programa \\\\ \hline\hline
\multirow{4}{*}{\tab[2mm] 141 \tab[2mm]} & 3NFAACg &
\multirow{3}{*}{Different} & \multirow{3}{*}{Different}
\\
& E = No Data \tab Freq =No Data   &    &  \\ \cline{2-2}
& 3NFAACj   & \multicolumn{2}{c}{\multirow{3}{*}
 {RMS = 0.0572353}}
\\
& E = No Data \tab Freq =No Data   &    \multicolumn{2}{c}{}  \\ \hline
\multirow{4}{*}{\tab[2mm] 142 \tab[2mm]} & 3NFAACg &
\multirow{3}{*}{Different} & \multirow{3}{*}{Different}
\\
& E = No Data \tab Freq =No Data   &    &  \\ \cline{2-2}
& 3NFAACk   & \multicolumn{2}{c}{\multirow{3}{*}
 {RMS = 0.0216732}}
\\
& E = No Data \tab Freq =No Data   &    \multicolumn{2}{c}{}  \\ \hline
\multirow{4}{*}{\tab[2mm] 143 \tab[2mm]} & 3NFAACg &
\multirow{3}{*}{Different} & \multirow{3}{*}{Different}
\\
& E = No Data \tab Freq =No Data   &    &  \\ \cline{2-2}
& 3NFAACl   & \multicolumn{2}{c}{\multirow{3}{*}
 {RMS = 0.0693142}}
\\
& E = No Data \tab Freq =No Data   &    \multicolumn{2}{c}{}  \\ \hline
\multirow{4}{*}{\tab[2mm] 144 \tab[2mm]} & 3NFAACg &
\multirow{3}{*}{Different} & \multirow{3}{*}{Different}
\\
& E = No Data \tab Freq =No Data   &    &  \\ \cline{2-2}
& 3NFAACm   & \multicolumn{2}{c}{\multirow{3}{*}
 {RMS = 0.185918}}
\\
& E = No Data \tab Freq =No Data   &    \multicolumn{2}{c}{}  \\ \hline
\multirow{4}{*}{\tab[2mm] 145 \tab[2mm]} & 3NFAACg &
\multirow{3}{*}{Different} & \multirow{3}{*}{Different}
\\
& E = No Data \tab Freq =No Data   &    &  \\ \cline{2-2}
& 3NFAACn   & \multicolumn{2}{c}{\multirow{3}{*}
 {RMS = 0.190267}}
\\
& E = No Data \tab Freq =No Data   &    \multicolumn{2}{c}{}  \\ \hline
\multirow{4}{*}{\tab[2mm] 146 \tab[2mm]} & 3NFAACg &
\multirow{3}{*}{Different} & \multirow{3}{*}{Different}
\\
& E = No Data \tab Freq =No Data   &    &  \\ \cline{2-2}
& 4NFAACa   & \multicolumn{2}{c}{\multirow{3}{*}
 {RMS = 0.186693}}
\\
& E = No Data \tab Freq =No Data   &    \multicolumn{2}{c}{}  \\ \hline
\multirow{4}{*}{\tab[2mm] 147 \tab[2mm]} & 3NFAACg &
\multirow{3}{*}{Different} & \multirow{3}{*}{Different}
\\
& E = No Data \tab Freq =No Data   &    &  \\ \cline{2-2}
& 4NFAACb   & \multicolumn{2}{c}{\multirow{3}{*}
 {RMS = 0.291527}}
\\
& E = No Data \tab Freq =No Data   &    \multicolumn{2}{c}{}  \\ \hline
\end{tabular}
\newpage

\vtab[-2cm]
\tab[-2cm]
\begin{tabular}{c|m{8cm}|c|c}
\# & Moléculas & Restultado esperado & Resultado programa \\\\ \hline\hline
\multirow{4}{*}{\tab[2mm] 148 \tab[2mm]} & 3NFAACg &
\multirow{3}{*}{Different} & \multirow{3}{*}{Different}
\\
& E = No Data \tab Freq =No Data   &    &  \\ \cline{2-2}
& 4NFAACc   & \multicolumn{2}{c}{\multirow{3}{*}
 {RMS = 0.0517103}}
\\
& E = No Data \tab Freq =No Data   &    \multicolumn{2}{c}{}  \\ \hline
\multirow{4}{*}{\tab[2mm] 149 \tab[2mm]} & 3NFAACg &
\multirow{3}{*}{Different} & \multirow{3}{*}{Different}
\\
& E = No Data \tab Freq =No Data   &    &  \\ \cline{2-2}
& 4NFAACd   & \multicolumn{2}{c}{\multirow{3}{*}
 {RMS = 0.0479926}}
\\
& E = No Data \tab Freq =No Data   &    \multicolumn{2}{c}{}  \\ \hline
\multirow{4}{*}{\tab[2mm] 150 \tab[2mm]} & 3NFAACg &
\multirow{3}{*}{Different} & \multirow{3}{*}{Different}
\\
& E = No Data \tab Freq =No Data   &    &  \\ \cline{2-2}
& 4NFAACe   & \multicolumn{2}{c}{\multirow{3}{*}
 {RMS = 0.0559286}}
\\
& E = No Data \tab Freq =No Data   &    \multicolumn{2}{c}{}  \\ \hline
\multirow{4}{*}{\tab[2mm] 151 \tab[2mm]} & 3NFAACg &
\multirow{3}{*}{Different} & \multirow{3}{*}{Different}
\\
& E = No Data \tab Freq =No Data   &    &  \\ \cline{2-2}
& 4NFAACf   & \multicolumn{2}{c}{\multirow{3}{*}
 {RMS = 0.151387}}
\\
& E = No Data \tab Freq =No Data   &    \multicolumn{2}{c}{}  \\ \hline
\multirow{4}{*}{\tab[2mm] 152 \tab[2mm]} & 3NFAACg &
\multirow{3}{*}{Different} & \multirow{3}{*}{Different}
\\
& E = No Data \tab Freq =No Data   &    &  \\ \cline{2-2}
& 4NFAACg   & \multicolumn{2}{c}{\multirow{3}{*}
 {RMS = 0.0669433}}
\\
& E = No Data \tab Freq =No Data   &    \multicolumn{2}{c}{}  \\ \hline
\multirow{4}{*}{\tab[2mm] 153 \tab[2mm]} & 3NFAACg &
\multirow{3}{*}{Different} & \multirow{3}{*}{Different}
\\
& E = No Data \tab Freq =No Data   &    &  \\ \cline{2-2}
& 4NFAACi   & \multicolumn{2}{c}{\multirow{3}{*}
 {RMS = 0.0316224}}
\\
& E = No Data \tab Freq =No Data   &    \multicolumn{2}{c}{}  \\ \hline
\multirow{4}{*}{\tab[2mm] 154 \tab[2mm]} & 3NFAACg &
\multirow{3}{*}{Different} & \multirow{3}{*}{Different}
\\
& E = No Data \tab Freq =No Data   &    &  \\ \cline{2-2}
& 4NFAACj   & \multicolumn{2}{c}{\multirow{3}{*}
 {RMS = 0.11992}}
\\
& E = No Data \tab Freq =No Data   &    \multicolumn{2}{c}{}  \\ \hline
\end{tabular}
\newpage

\vtab[-2cm]
\tab[-2cm]
\begin{tabular}{c|m{8cm}|c|c}
\# & Moléculas & Restultado esperado & Resultado programa \\\\ \hline\hline
\multirow{4}{*}{\tab[2mm] 155 \tab[2mm]} & 3NFAACg &
\multirow{3}{*}{Different} & \multirow{3}{*}{Different}
\\
& E = No Data \tab Freq =No Data   &    &  \\ \cline{2-2}
& 4NFAACl-3   & \multicolumn{2}{c}{\multirow{3}{*}
 {RMS = 0.0745674}}
\\
& E = No Data \tab Freq =No Data   &    \multicolumn{2}{c}{}  \\ \hline
\multirow{4}{*}{\tab[2mm] 156 \tab[2mm]} & 3NFAACh &
\multirow{3}{*}{Different} & \multirow{3}{*}{Different}
\\
& E = No Data \tab Freq =No Data   &    &  \\ \cline{2-2}
& 3NFAACi   & \multicolumn{2}{c}{\multirow{3}{*}
 {RMS = 0.148119}}
\\
& E = No Data \tab Freq =No Data   &    \multicolumn{2}{c}{}  \\ \hline
\multirow{4}{*}{\tab[2mm] 157 \tab[2mm]} & 3NFAACh &
\multirow{3}{*}{Different} & \multirow{3}{*}{Different}
\\
& E = No Data \tab Freq =No Data   &    &  \\ \cline{2-2}
& 3NFAACj   & \multicolumn{2}{c}{\multirow{3}{*}
 {RMS = 0.154737}}
\\
& E = No Data \tab Freq =No Data   &    \multicolumn{2}{c}{}  \\ \hline
\multirow{4}{*}{\tab[2mm] 158 \tab[2mm]} & 3NFAACh &
\multirow{3}{*}{Different} & \multirow{3}{*}{Different}
\\
& E = No Data \tab Freq =No Data   &    &  \\ \cline{2-2}
& 3NFAACk   & \multicolumn{2}{c}{\multirow{3}{*}
 {RMS = 0.0758286}}
\\
& E = No Data \tab Freq =No Data   &    \multicolumn{2}{c}{}  \\ \hline
\multirow{4}{*}{\tab[2mm] 159 \tab[2mm]} & 3NFAACh &
\multirow{3}{*}{Different} & \multirow{3}{*}{Different}
\\
& E = No Data \tab Freq =No Data   &    &  \\ \cline{2-2}
& 3NFAACl   & \multicolumn{2}{c}{\multirow{3}{*}
 {RMS = 0.0281876}}
\\
& E = No Data \tab Freq =No Data   &    \multicolumn{2}{c}{}  \\ \hline
\multirow{4}{*}{\tab[2mm] 160 \tab[2mm]} & 3NFAACh &
\multirow{3}{*}{Different} & \multirow{3}{*}{Different}
\\
& E = No Data \tab Freq =No Data   &    &  \\ \cline{2-2}
& 3NFAACm   & \multicolumn{2}{c}{\multirow{3}{*}
 {RMS = 0.0884159}}
\\
& E = No Data \tab Freq =No Data   &    \multicolumn{2}{c}{}  \\ \hline
\multirow{4}{*}{\tab[2mm] 161 \tab[2mm]} & 3NFAACh &
\multirow{3}{*}{Different} & \multirow{3}{*}{Different}
\\
& E = No Data \tab Freq =No Data   &    &  \\ \cline{2-2}
& 3NFAACn   & \multicolumn{2}{c}{\multirow{3}{*}
 {RMS = 0.0927652}}
\\
& E = No Data \tab Freq =No Data   &    \multicolumn{2}{c}{}  \\ \hline
\end{tabular}
\newpage

\vtab[-2cm]
\tab[-2cm]
\begin{tabular}{c|m{8cm}|c|c}
\# & Moléculas & Restultado esperado & Resultado programa \\\\ \hline\hline
\multirow{4}{*}{\tab[2mm] 162 \tab[2mm]} & 3NFAACh &
\multirow{3}{*}{Different} & \multirow{3}{*}{Different}
\\
& E = No Data \tab Freq =No Data   &    &  \\ \cline{2-2}
& 4NFAACa   & \multicolumn{2}{c}{\multirow{3}{*}
 {RMS = 0.284195}}
\\
& E = No Data \tab Freq =No Data   &    \multicolumn{2}{c}{}  \\ \hline
\multirow{4}{*}{\tab[2mm] 163 \tab[2mm]} & 3NFAACh &
\multirow{3}{*}{Different} & \multirow{3}{*}{Different}
\\
& E = No Data \tab Freq =No Data   &    &  \\ \cline{2-2}
& 4NFAACb   & \multicolumn{2}{c}{\multirow{3}{*}
 {RMS = 0.194025}}
\\
& E = No Data \tab Freq =No Data   &    \multicolumn{2}{c}{}  \\ \hline
\multirow{4}{*}{\tab[2mm] 164 \tab[2mm]} & 3NFAACh &
\multirow{3}{*}{Different} & \multirow{3}{*}{Different}
\\
& E = No Data \tab Freq =No Data   &    &  \\ \cline{2-2}
& 4NFAACc   & \multicolumn{2}{c}{\multirow{3}{*}
 {RMS = 0.0457915}}
\\
& E = No Data \tab Freq =No Data   &    \multicolumn{2}{c}{}  \\ \hline
\multirow{4}{*}{\tab[2mm] 165 \tab[2mm]} & 3NFAACh &
\multirow{3}{*}{Different} & \multirow{3}{*}{Different}
\\
& E = No Data \tab Freq =No Data   &    &  \\ \cline{2-2}
& 4NFAACd   & \multicolumn{2}{c}{\multirow{3}{*}
 {RMS = 0.0495092}}
\\
& E = No Data \tab Freq =No Data   &    \multicolumn{2}{c}{}  \\ \hline
\multirow{4}{*}{\tab[2mm] 166 \tab[2mm]} & 3NFAACh &
\multirow{3}{*}{Different} & \multirow{3}{*}{Different}
\\
& E = No Data \tab Freq =No Data   &    &  \\ \cline{2-2}
& 4NFAACe   & \multicolumn{2}{c}{\multirow{3}{*}
 {RMS = 0.0415732}}
\\
& E = No Data \tab Freq =No Data   &    \multicolumn{2}{c}{}  \\ \hline
\multirow{4}{*}{\tab[2mm] 167 \tab[2mm]} & 3NFAACh &
\multirow{3}{*}{Different} & \multirow{3}{*}{Different}
\\
& E = No Data \tab Freq =No Data   &    &  \\ \cline{2-2}
& 4NFAACf   & \multicolumn{2}{c}{\multirow{3}{*}
 {RMS = 0.0538853}}
\\
& E = No Data \tab Freq =No Data   &    \multicolumn{2}{c}{}  \\ \hline
\multirow{4}{*}{\tab[2mm] 168 \tab[2mm]} & 3NFAACh &
\multirow{3}{*}{Different} & \multirow{3}{*}{Different}
\\
& E = No Data \tab Freq =No Data   &    &  \\ \cline{2-2}
& 4NFAACg   & \multicolumn{2}{c}{\multirow{3}{*}
 {RMS = 0.0305585}}
\\
& E = No Data \tab Freq =No Data   &    \multicolumn{2}{c}{}  \\ \hline
\end{tabular}
\newpage

\vtab[-2cm]
\tab[-2cm]
\begin{tabular}{c|m{8cm}|c|c}
\# & Moléculas & Restultado esperado & Resultado programa \\\\ \hline\hline
\multirow{4}{*}{\tab[2mm] 169 \tab[2mm]} & 3NFAACh &
\multirow{3}{*}{Different} & \multirow{3}{*}{Different}
\\
& E = No Data \tab Freq =No Data   &    &  \\ \cline{2-2}
& 4NFAACi   & \multicolumn{2}{c}{\multirow{3}{*}
 {RMS = 0.0658794}}
\\
& E = No Data \tab Freq =No Data   &    \multicolumn{2}{c}{}  \\ \hline
\multirow{4}{*}{\tab[2mm] 170 \tab[2mm]} & 3NFAACh &
\multirow{3}{*}{Different} & \multirow{3}{*}{Different}
\\
& E = No Data \tab Freq =No Data   &    &  \\ \cline{2-2}
& 4NFAACj   & \multicolumn{2}{c}{\multirow{3}{*}
 {RMS = 0.0224186}}
\\
& E = No Data \tab Freq =No Data   &    \multicolumn{2}{c}{}  \\ \hline
\multirow{4}{*}{\tab[2mm] 171 \tab[2mm]} & 3NFAACh &
\multirow{3}{*}{Different} & \multirow{3}{*}{Different}
\\
& E = No Data \tab Freq =No Data   &    &  \\ \cline{2-2}
& 4NFAACl-3   & \multicolumn{2}{c}{\multirow{3}{*}
 {RMS = 0.0229344}}
\\
& E = No Data \tab Freq =No Data   &    \multicolumn{2}{c}{}  \\ \hline
\multirow{4}{*}{\tab[2mm] 172 \tab[2mm]} & 3NFAACi &
\multirow{3}{*}{Different} & \multirow{3}{*}{Different}
\\
& E = No Data \tab Freq =No Data   &    &  \\ \cline{2-2}
& 3NFAACj   & \multicolumn{2}{c}{\multirow{3}{*}
 {RMS = 0.00661796}}
\\
& E = No Data \tab Freq =No Data   &    \multicolumn{2}{c}{}  \\ \hline
\multirow{4}{*}{\tab[2mm] 173 \tab[2mm]} & 3NFAACi &
\multirow{3}{*}{Different} & \multirow{3}{*}{Different}
\\
& E = No Data \tab Freq =No Data   &    &  \\ \cline{2-2}
& 3NFAACk   & \multicolumn{2}{c}{\multirow{3}{*}
 {RMS = 0.0722905}}
\\
& E = No Data \tab Freq =No Data   &    \multicolumn{2}{c}{}  \\ \hline
\multirow{4}{*}{\tab[2mm] 174 \tab[2mm]} & 3NFAACi &
\multirow{3}{*}{Different} & \multirow{3}{*}{Different}
\\
& E = No Data \tab Freq =No Data   &    &  \\ \cline{2-2}
& 3NFAACl   & \multicolumn{2}{c}{\multirow{3}{*}
 {RMS = 0.119932}}
\\
& E = No Data \tab Freq =No Data   &    \multicolumn{2}{c}{}  \\ \hline
\multirow{4}{*}{\tab[2mm] 175 \tab[2mm]} & 3NFAACi &
\multirow{3}{*}{Different} & \multirow{3}{*}{Different}
\\
& E = No Data \tab Freq =No Data   &    &  \\ \cline{2-2}
& 3NFAACm   & \multicolumn{2}{c}{\multirow{3}{*}
 {RMS = 0.236535}}
\\
& E = No Data \tab Freq =No Data   &    \multicolumn{2}{c}{}  \\ \hline
\end{tabular}
\newpage

\vtab[-2cm]
\tab[-2cm]
\begin{tabular}{c|m{8cm}|c|c}
\# & Moléculas & Restultado esperado & Resultado programa \\\\ \hline\hline
\multirow{4}{*}{\tab[2mm] 176 \tab[2mm]} & 3NFAACi &
\multirow{3}{*}{Different} & \multirow{3}{*}{Different}
\\
& E = No Data \tab Freq =No Data   &    &  \\ \cline{2-2}
& 3NFAACn   & \multicolumn{2}{c}{\multirow{3}{*}
 {RMS = 0.240884}}
\\
& E = No Data \tab Freq =No Data   &    \multicolumn{2}{c}{}  \\ \hline
\multirow{4}{*}{\tab[2mm] 177 \tab[2mm]} & 3NFAACi &
\multirow{3}{*}{Different} & \multirow{3}{*}{Different}
\\
& E = No Data \tab Freq =No Data   &    &  \\ \cline{2-2}
& 4NFAACa   & \multicolumn{2}{c}{\multirow{3}{*}
 {RMS = 0.136076}}
\\
& E = No Data \tab Freq =No Data   &    \multicolumn{2}{c}{}  \\ \hline
\multirow{4}{*}{\tab[2mm] 178 \tab[2mm]} & 3NFAACi &
\multirow{3}{*}{Different} & \multirow{3}{*}{Different}
\\
& E = No Data \tab Freq =No Data   &    &  \\ \cline{2-2}
& 4NFAACb   & \multicolumn{2}{c}{\multirow{3}{*}
 {RMS = 0.342144}}
\\
& E = No Data \tab Freq =No Data   &    \multicolumn{2}{c}{}  \\ \hline
\multirow{4}{*}{\tab[2mm] 179 \tab[2mm]} & 3NFAACi &
\multirow{3}{*}{Different} & \multirow{3}{*}{Different}
\\
& E = No Data \tab Freq =No Data   &    &  \\ \cline{2-2}
& 4NFAACc   & \multicolumn{2}{c}{\multirow{3}{*}
 {RMS = 0.102328}}
\\
& E = No Data \tab Freq =No Data   &    \multicolumn{2}{c}{}  \\ \hline
\multirow{4}{*}{\tab[2mm] 180 \tab[2mm]} & 3NFAACi &
\multirow{3}{*}{Different} & \multirow{3}{*}{Different}
\\
& E = No Data \tab Freq =No Data   &    &  \\ \cline{2-2}
& 4NFAACd   & \multicolumn{2}{c}{\multirow{3}{*}
 {RMS = 0.0986099}}
\\
& E = No Data \tab Freq =No Data   &    \multicolumn{2}{c}{}  \\ \hline
\multirow{4}{*}{\tab[2mm] 181 \tab[2mm]} & 3NFAACi &
\multirow{3}{*}{Different} & \multirow{3}{*}{Different}
\\
& E = No Data \tab Freq =No Data   &    &  \\ \cline{2-2}
& 4NFAACe   & \multicolumn{2}{c}{\multirow{3}{*}
 {RMS = 0.106546}}
\\
& E = No Data \tab Freq =No Data   &    \multicolumn{2}{c}{}  \\ \hline
\multirow{4}{*}{\tab[2mm] 182 \tab[2mm]} & 3NFAACi &
\multirow{3}{*}{Different} & \multirow{3}{*}{Different}
\\
& E = No Data \tab Freq =No Data   &    &  \\ \cline{2-2}
& 4NFAACf   & \multicolumn{2}{c}{\multirow{3}{*}
 {RMS = 0.202004}}
\\
& E = No Data \tab Freq =No Data   &    \multicolumn{2}{c}{}  \\ \hline
\end{tabular}
\newpage

\vtab[-2cm]
\tab[-2cm]
\begin{tabular}{c|m{8cm}|c|c}
\# & Moléculas & Restultado esperado & Resultado programa \\\\ \hline\hline
\multirow{4}{*}{\tab[2mm] 183 \tab[2mm]} & 3NFAACi &
\multirow{3}{*}{Different} & \multirow{3}{*}{Different}
\\
& E = No Data \tab Freq =No Data   &    &  \\ \cline{2-2}
& 4NFAACg   & \multicolumn{2}{c}{\multirow{3}{*}
 {RMS = 0.117561}}
\\
& E = No Data \tab Freq =No Data   &    \multicolumn{2}{c}{}  \\ \hline
\multirow{4}{*}{\tab[2mm] 184 \tab[2mm]} & 3NFAACi &
\multirow{3}{*}{Different} & \multirow{3}{*}{Different}
\\
& E = No Data \tab Freq =No Data   &    &  \\ \cline{2-2}
& 4NFAACi   & \multicolumn{2}{c}{\multirow{3}{*}
 {RMS = 0.0822397}}
\\
& E = No Data \tab Freq =No Data   &    \multicolumn{2}{c}{}  \\ \hline
\multirow{4}{*}{\tab[2mm] 185 \tab[2mm]} & 3NFAACi &
\multirow{3}{*}{Different} & \multirow{3}{*}{Different}
\\
& E = No Data \tab Freq =No Data   &    &  \\ \cline{2-2}
& 4NFAACj   & \multicolumn{2}{c}{\multirow{3}{*}
 {RMS = 0.170538}}
\\
& E = No Data \tab Freq =No Data   &    \multicolumn{2}{c}{}  \\ \hline
\multirow{4}{*}{\tab[2mm] 186 \tab[2mm]} & 3NFAACi &
\multirow{3}{*}{Different} & \multirow{3}{*}{Different}
\\
& E = No Data \tab Freq =No Data   &    &  \\ \cline{2-2}
& 4NFAACl-3   & \multicolumn{2}{c}{\multirow{3}{*}
 {RMS = 0.125185}}
\\
& E = No Data \tab Freq =No Data   &    \multicolumn{2}{c}{}  \\ \hline
\multirow{4}{*}{\tab[2mm] 187 \tab[2mm]} & 3NFAACj &
\multirow{3}{*}{Different} & \multirow{3}{*}{Different}
\\
& E = No Data \tab Freq =No Data   &    &  \\ \cline{2-2}
& 3NFAACk   & \multicolumn{2}{c}{\multirow{3}{*}
 {RMS = 0.0789084}}
\\
& E = No Data \tab Freq =No Data   &    \multicolumn{2}{c}{}  \\ \hline
\multirow{4}{*}{\tab[2mm] 188 \tab[2mm]} & 3NFAACj &
\multirow{3}{*}{Different} & \multirow{3}{*}{Different}
\\
& E = No Data \tab Freq =No Data   &    &  \\ \cline{2-2}
& 3NFAACl   & \multicolumn{2}{c}{\multirow{3}{*}
 {RMS = 0.126549}}
\\
& E = No Data \tab Freq =No Data   &    \multicolumn{2}{c}{}  \\ \hline
\multirow{4}{*}{\tab[2mm] 189 \tab[2mm]} & 3NFAACj &
\multirow{3}{*}{Different} & \multirow{3}{*}{Different}
\\
& E = No Data \tab Freq =No Data   &    &  \\ \cline{2-2}
& 3NFAACm   & \multicolumn{2}{c}{\multirow{3}{*}
 {RMS = 0.243153}}
\\
& E = No Data \tab Freq =No Data   &    \multicolumn{2}{c}{}  \\ \hline
\end{tabular}
\newpage

\vtab[-2cm]
\tab[-2cm]
\begin{tabular}{c|m{8cm}|c|c}
\# & Moléculas & Restultado esperado & Resultado programa \\\\ \hline\hline
\multirow{4}{*}{\tab[2mm] 190 \tab[2mm]} & 3NFAACj &
\multirow{3}{*}{Different} & \multirow{3}{*}{Different}
\\
& E = No Data \tab Freq =No Data   &    &  \\ \cline{2-2}
& 3NFAACn   & \multicolumn{2}{c}{\multirow{3}{*}
 {RMS = 0.247502}}
\\
& E = No Data \tab Freq =No Data   &    \multicolumn{2}{c}{}  \\ \hline
\multirow{4}{*}{\tab[2mm] 191 \tab[2mm]} & 3NFAACj &
\multirow{3}{*}{Different} & \multirow{3}{*}{Different}
\\
& E = No Data \tab Freq =No Data   &    &  \\ \cline{2-2}
& 4NFAACa   & \multicolumn{2}{c}{\multirow{3}{*}
 {RMS = 0.129458}}
\\
& E = No Data \tab Freq =No Data   &    \multicolumn{2}{c}{}  \\ \hline
\multirow{4}{*}{\tab[2mm] 192 \tab[2mm]} & 3NFAACj &
\multirow{3}{*}{Different} & \multirow{3}{*}{Different}
\\
& E = No Data \tab Freq =No Data   &    &  \\ \cline{2-2}
& 4NFAACb   & \multicolumn{2}{c}{\multirow{3}{*}
 {RMS = 0.348762}}
\\
& E = No Data \tab Freq =No Data   &    \multicolumn{2}{c}{}  \\ \hline
\multirow{4}{*}{\tab[2mm] 193 \tab[2mm]} & 3NFAACj &
\multirow{3}{*}{Different} & \multirow{3}{*}{Different}
\\
& E = No Data \tab Freq =No Data   &    &  \\ \cline{2-2}
& 4NFAACc   & \multicolumn{2}{c}{\multirow{3}{*}
 {RMS = 0.108946}}
\\
& E = No Data \tab Freq =No Data   &    \multicolumn{2}{c}{}  \\ \hline
\multirow{4}{*}{\tab[2mm] 194 \tab[2mm]} & 3NFAACj &
\multirow{3}{*}{Different} & \multirow{3}{*}{Different}
\\
& E = No Data \tab Freq =No Data   &    &  \\ \cline{2-2}
& 4NFAACd   & \multicolumn{2}{c}{\multirow{3}{*}
 {RMS = 0.105228}}
\\
& E = No Data \tab Freq =No Data   &    \multicolumn{2}{c}{}  \\ \hline
\multirow{4}{*}{\tab[2mm] 195 \tab[2mm]} & 3NFAACj &
\multirow{3}{*}{Different} & \multirow{3}{*}{Different}
\\
& E = No Data \tab Freq =No Data   &    &  \\ \cline{2-2}
& 4NFAACe   & \multicolumn{2}{c}{\multirow{3}{*}
 {RMS = 0.113164}}
\\
& E = No Data \tab Freq =No Data   &    \multicolumn{2}{c}{}  \\ \hline
\multirow{4}{*}{\tab[2mm] 196 \tab[2mm]} & 3NFAACj &
\multirow{3}{*}{Different} & \multirow{3}{*}{Different}
\\
& E = No Data \tab Freq =No Data   &    &  \\ \cline{2-2}
& 4NFAACf   & \multicolumn{2}{c}{\multirow{3}{*}
 {RMS = 0.208622}}
\\
& E = No Data \tab Freq =No Data   &    \multicolumn{2}{c}{}  \\ \hline
\end{tabular}
\newpage

\vtab[-2cm]
\tab[-2cm]
\begin{tabular}{c|m{8cm}|c|c}
\# & Moléculas & Restultado esperado & Resultado programa \\\\ \hline\hline
\multirow{4}{*}{\tab[2mm] 197 \tab[2mm]} & 3NFAACj &
\multirow{3}{*}{Different} & \multirow{3}{*}{Different}
\\
& E = No Data \tab Freq =No Data   &    &  \\ \cline{2-2}
& 4NFAACg   & \multicolumn{2}{c}{\multirow{3}{*}
 {RMS = 0.124179}}
\\
& E = No Data \tab Freq =No Data   &    \multicolumn{2}{c}{}  \\ \hline
\multirow{4}{*}{\tab[2mm] 198 \tab[2mm]} & 3NFAACj &
\multirow{3}{*}{Different} & \multirow{3}{*}{Different}
\\
& E = No Data \tab Freq =No Data   &    &  \\ \cline{2-2}
& 4NFAACi   & \multicolumn{2}{c}{\multirow{3}{*}
 {RMS = 0.0888577}}
\\
& E = No Data \tab Freq =No Data   &    \multicolumn{2}{c}{}  \\ \hline
\multirow{4}{*}{\tab[2mm] 199 \tab[2mm]} & 3NFAACj &
\multirow{3}{*}{Different} & \multirow{3}{*}{Different}
\\
& E = No Data \tab Freq =No Data   &    &  \\ \cline{2-2}
& 4NFAACj   & \multicolumn{2}{c}{\multirow{3}{*}
 {RMS = 0.177156}}
\\
& E = No Data \tab Freq =No Data   &    \multicolumn{2}{c}{}  \\ \hline
\multirow{4}{*}{\tab[2mm] 200 \tab[2mm]} & 3NFAACj &
\multirow{3}{*}{Different} & \multirow{3}{*}{Different}
\\
& E = No Data \tab Freq =No Data   &    &  \\ \cline{2-2}
& 4NFAACl-3   & \multicolumn{2}{c}{\multirow{3}{*}
 {RMS = 0.131803}}
\\
& E = No Data \tab Freq =No Data   &    \multicolumn{2}{c}{}  \\ \hline
\multirow{4}{*}{\tab[2mm] 201 \tab[2mm]} & 3NFAACk &
\multirow{3}{*}{Different} & \multirow{3}{*}{Different}
\\
& E = No Data \tab Freq =No Data   &    &  \\ \cline{2-2}
& 3NFAACl   & \multicolumn{2}{c}{\multirow{3}{*}
 {RMS = 0.047641}}
\\
& E = No Data \tab Freq =No Data   &    \multicolumn{2}{c}{}  \\ \hline
\multirow{4}{*}{\tab[2mm] 202 \tab[2mm]} & 3NFAACk &
\multirow{3}{*}{Different} & \multirow{3}{*}{Different}
\\
& E = No Data \tab Freq =No Data   &    &  \\ \cline{2-2}
& 3NFAACm   & \multicolumn{2}{c}{\multirow{3}{*}
 {RMS = 0.164244}}
\\
& E = No Data \tab Freq =No Data   &    \multicolumn{2}{c}{}  \\ \hline
\multirow{4}{*}{\tab[2mm] 203 \tab[2mm]} & 3NFAACk &
\multirow{3}{*}{Different} & \multirow{3}{*}{Different}
\\
& E = No Data \tab Freq =No Data   &    &  \\ \cline{2-2}
& 3NFAACn   & \multicolumn{2}{c}{\multirow{3}{*}
 {RMS = 0.168594}}
\\
& E = No Data \tab Freq =No Data   &    \multicolumn{2}{c}{}  \\ \hline
\end{tabular}
\newpage

\vtab[-2cm]
\tab[-2cm]
\begin{tabular}{c|m{8cm}|c|c}
\# & Moléculas & Restultado esperado & Resultado programa \\\\ \hline\hline
\multirow{4}{*}{\tab[2mm] 204 \tab[2mm]} & 3NFAACk &
\multirow{3}{*}{Different} & \multirow{3}{*}{Different}
\\
& E = No Data \tab Freq =No Data   &    &  \\ \cline{2-2}
& 4NFAACa   & \multicolumn{2}{c}{\multirow{3}{*}
 {RMS = 0.208366}}
\\
& E = No Data \tab Freq =No Data   &    \multicolumn{2}{c}{}  \\ \hline
\multirow{4}{*}{\tab[2mm] 205 \tab[2mm]} & 3NFAACk &
\multirow{3}{*}{Different} & \multirow{3}{*}{Different}
\\
& E = No Data \tab Freq =No Data   &    &  \\ \cline{2-2}
& 4NFAACb   & \multicolumn{2}{c}{\multirow{3}{*}
 {RMS = 0.269854}}
\\
& E = No Data \tab Freq =No Data   &    \multicolumn{2}{c}{}  \\ \hline
\multirow{4}{*}{\tab[2mm] 206 \tab[2mm]} & 3NFAACk &
\multirow{3}{*}{Different} & \multirow{3}{*}{Different}
\\
& E = No Data \tab Freq =No Data   &    &  \\ \cline{2-2}
& 4NFAACc   & \multicolumn{2}{c}{\multirow{3}{*}
 {RMS = 0.0300371}}
\\
& E = No Data \tab Freq =No Data   &    \multicolumn{2}{c}{}  \\ \hline
\multirow{4}{*}{\tab[2mm] 207 \tab[2mm]} & 3NFAACk &
\multirow{3}{*}{Different} & \multirow{3}{*}{Different}
\\
& E = No Data \tab Freq =No Data   &    &  \\ \cline{2-2}
& 4NFAACd   & \multicolumn{2}{c}{\multirow{3}{*}
 {RMS = 0.0263194}}
\\
& E = No Data \tab Freq =No Data   &    \multicolumn{2}{c}{}  \\ \hline
\multirow{4}{*}{\tab[2mm] 208 \tab[2mm]} & 3NFAACk &
\multirow{3}{*}{Different} & \multirow{3}{*}{Different}
\\
& E = No Data \tab Freq =No Data   &    &  \\ \cline{2-2}
& 4NFAACe   & \multicolumn{2}{c}{\multirow{3}{*}
 {RMS = 0.0342554}}
\\
& E = No Data \tab Freq =No Data   &    \multicolumn{2}{c}{}  \\ \hline
\multirow{4}{*}{\tab[2mm] 209 \tab[2mm]} & 3NFAACk &
\multirow{3}{*}{Different} & \multirow{3}{*}{Different}
\\
& E = No Data \tab Freq =No Data   &    &  \\ \cline{2-2}
& 4NFAACf   & \multicolumn{2}{c}{\multirow{3}{*}
 {RMS = 0.129714}}
\\
& E = No Data \tab Freq =No Data   &    \multicolumn{2}{c}{}  \\ \hline
\multirow{4}{*}{\tab[2mm] 210 \tab[2mm]} & 3NFAACk &
\multirow{3}{*}{Different} & \multirow{3}{*}{Different}
\\
& E = No Data \tab Freq =No Data   &    &  \\ \cline{2-2}
& 4NFAACg   & \multicolumn{2}{c}{\multirow{3}{*}
 {RMS = 0.0452701}}
\\
& E = No Data \tab Freq =No Data   &    \multicolumn{2}{c}{}  \\ \hline
\end{tabular}
\newpage

\vtab[-2cm]
\tab[-2cm]
\begin{tabular}{c|m{8cm}|c|c}
\# & Moléculas & Restultado esperado & Resultado programa \\\\ \hline\hline
\multirow{4}{*}{\tab[2mm] 211 \tab[2mm]} & 3NFAACk &
\multirow{3}{*}{Different} & \multirow{3}{*}{Different}
\\
& E = No Data \tab Freq =No Data   &    &  \\ \cline{2-2}
& 4NFAACi   & \multicolumn{2}{c}{\multirow{3}{*}
 {RMS = 0.00994921}}
\\
& E = No Data \tab Freq =No Data   &    \multicolumn{2}{c}{}  \\ \hline
\multirow{4}{*}{\tab[2mm] 212 \tab[2mm]} & 3NFAACk &
\multirow{3}{*}{Different} & \multirow{3}{*}{Different}
\\
& E = No Data \tab Freq =No Data   &    &  \\ \cline{2-2}
& 4NFAACj   & \multicolumn{2}{c}{\multirow{3}{*}
 {RMS = 0.0982472}}
\\
& E = No Data \tab Freq =No Data   &    \multicolumn{2}{c}{}  \\ \hline
\multirow{4}{*}{\tab[2mm] 213 \tab[2mm]} & 3NFAACk &
\multirow{3}{*}{Different} & \multirow{3}{*}{Different}
\\
& E = No Data \tab Freq =No Data   &    &  \\ \cline{2-2}
& 4NFAACl-3   & \multicolumn{2}{c}{\multirow{3}{*}
 {RMS = 0.0528942}}
\\
& E = No Data \tab Freq =No Data   &    \multicolumn{2}{c}{}  \\ \hline
\multirow{4}{*}{\tab[2mm] 214 \tab[2mm]} & 3NFAACl &
\multirow{3}{*}{\textcolor{Red}{\bf Stereoisomer}} & \multirow{3}{*}{\textcolor{Red}{\bf Enantiomers}}
\\
& E = No Data \tab Freq =No Data   &    &  \\ \cline{2-2}
& 3NFAACm   & \multicolumn{2}{c}{\multirow{3}{*}
 {RMS = 0.02978757}}
\\
& E = No Data \tab Freq =No Data   &    \multicolumn{2}{c}{}  \\ \hline
\multirow{4}{*}{\tab[2mm] 215 \tab[2mm]} & 3NFAACl &
\multirow{3}{*}{Different} & \multirow{3}{*}{Different}
\\
& E = No Data \tab Freq =No Data   &    &  \\ \cline{2-2}
& 3NFAACn   & \multicolumn{2}{c}{\multirow{3}{*}
 {RMS = 0.120953}}
\\
& E = No Data \tab Freq =No Data   &    \multicolumn{2}{c}{}  \\ \hline
\multirow{4}{*}{\tab[2mm] 216 \tab[2mm]} & 3NFAACl &
\multirow{3}{*}{Different} & \multirow{3}{*}{Different}
\\
& E = No Data \tab Freq =No Data   &    &  \\ \cline{2-2}
& 4NFAACa   & \multicolumn{2}{c}{\multirow{3}{*}
 {RMS = 0.256007}}
\\
& E = No Data \tab Freq =No Data   &    \multicolumn{2}{c}{}  \\ \hline
\multirow{4}{*}{\tab[2mm] 217 \tab[2mm]} & 3NFAACl &
\multirow{3}{*}{Different} & \multirow{3}{*}{Different}
\\
& E = No Data \tab Freq =No Data   &    &  \\ \cline{2-2}
& 4NFAACb   & \multicolumn{2}{c}{\multirow{3}{*}
 {RMS = 0.222213}}
\\
& E = No Data \tab Freq =No Data   &    \multicolumn{2}{c}{}  \\ \hline
\end{tabular}
\newpage

\vtab[-2cm]
\tab[-2cm]
\begin{tabular}{c|m{8cm}|c|c}
\# & Moléculas & Restultado esperado & Resultado programa \\\\ \hline\hline
\multirow{4}{*}{\tab[2mm] 218 \tab[2mm]} & 3NFAACl &
\multirow{3}{*}{Different} & \multirow{3}{*}{Different}
\\
& E = No Data \tab Freq =No Data   &    &  \\ \cline{2-2}
& 4NFAACc   & \multicolumn{2}{c}{\multirow{3}{*}
 {RMS = 0.0176039}}
\\
& E = No Data \tab Freq =No Data   &    \multicolumn{2}{c}{}  \\ \hline
\multirow{4}{*}{\tab[2mm] 219 \tab[2mm]} & 3NFAACl &
\multirow{3}{*}{Different} & \multirow{3}{*}{Different}
\\
& E = No Data \tab Freq =No Data   &    &  \\ \cline{2-2}
& 4NFAACd   & \multicolumn{2}{c}{\multirow{3}{*}
 {RMS = 0.0213216}}
\\
& E = No Data \tab Freq =No Data   &    \multicolumn{2}{c}{}  \\ \hline
\multirow{4}{*}{\tab[2mm] 220 \tab[2mm]} & 3NFAACl &
\multirow{3}{*}{Different} & \multirow{3}{*}{Different}
\\
& E = No Data \tab Freq =No Data   &    &  \\ \cline{2-2}
& 4NFAACe   & \multicolumn{2}{c}{\multirow{3}{*}
 {RMS = 0.0133856}}
\\
& E = No Data \tab Freq =No Data   &    \multicolumn{2}{c}{}  \\ \hline
\multirow{4}{*}{\tab[2mm] 221 \tab[2mm]} & 3NFAACl &
\multirow{3}{*}{Different} & \multirow{3}{*}{Different}
\\
& E = No Data \tab Freq =No Data   &    &  \\ \cline{2-2}
& 4NFAACf   & \multicolumn{2}{c}{\multirow{3}{*}
 {RMS = 0.0820729}}
\\
& E = No Data \tab Freq =No Data   &    \multicolumn{2}{c}{}  \\ \hline
\multirow{4}{*}{\tab[2mm] 222 \tab[2mm]} & 3NFAACl &
\multirow{3}{*}{Different} & \multirow{3}{*}{Different}
\\
& E = No Data \tab Freq =No Data   &    &  \\ \cline{2-2}
& 4NFAACg   & \multicolumn{2}{c}{\multirow{3}{*}
 {RMS = 0.00237094}}
\\
& E = No Data \tab Freq =No Data   &    \multicolumn{2}{c}{}  \\ \hline
\multirow{4}{*}{\tab[2mm] 223 \tab[2mm]} & 3NFAACl &
\multirow{3}{*}{Different} & \multirow{3}{*}{Different}
\\
& E = No Data \tab Freq =No Data   &    &  \\ \cline{2-2}
& 4NFAACi   & \multicolumn{2}{c}{\multirow{3}{*}
 {RMS = 0.0376918}}
\\
& E = No Data \tab Freq =No Data   &    \multicolumn{2}{c}{}  \\ \hline
\multirow{4}{*}{\tab[2mm] 224 \tab[2mm]} & 3NFAACl &
\multirow{3}{*}{Different} & \multirow{3}{*}{Different}
\\
& E = No Data \tab Freq =No Data   &    &  \\ \cline{2-2}
& 4NFAACj   & \multicolumn{2}{c}{\multirow{3}{*}
 {RMS = 0.0506062}}
\\
& E = No Data \tab Freq =No Data   &    \multicolumn{2}{c}{}  \\ \hline
\end{tabular}
\newpage

\vtab[-2cm]
\tab[-2cm]
\begin{tabular}{c|m{8cm}|c|c}
\# & Moléculas & Restultado esperado & Resultado programa \\\\ \hline\hline
\multirow{4}{*}{\tab[2mm] 225 \tab[2mm]} & 3NFAACl &
\multirow{3}{*}{Different} & \multirow{3}{*}{Different}
\\
& E = No Data \tab Freq =No Data   &    &  \\ \cline{2-2}
& 4NFAACl-3   & \multicolumn{2}{c}{\multirow{3}{*}
 {RMS = 0.0052532}}
\\
& E = No Data \tab Freq =No Data   &    \multicolumn{2}{c}{}  \\ \hline
\multirow{4}{*}{\tab[2mm] 226 \tab[2mm]} & 3NFAACm &
\multirow{3}{*}{Different} & \multirow{3}{*}{Different}
\\
& E = No Data \tab Freq =No Data   &    &  \\ \cline{2-2}
& 3NFAACn   & \multicolumn{2}{c}{\multirow{3}{*}
 {RMS = 0.00434937}}
\\
& E = No Data \tab Freq =No Data   &    \multicolumn{2}{c}{}  \\ \hline
\multirow{4}{*}{\tab[2mm] 227 \tab[2mm]} & 3NFAACm &
\multirow{3}{*}{Different} & \multirow{3}{*}{Different}
\\
& E = No Data \tab Freq =No Data   &    &  \\ \cline{2-2}
& 4NFAACa   & \multicolumn{2}{c}{\multirow{3}{*}
 {RMS = 0.372611}}
\\
& E = No Data \tab Freq =No Data   &    \multicolumn{2}{c}{}  \\ \hline
\multirow{4}{*}{\tab[2mm] 228 \tab[2mm]} & 3NFAACm &
\multirow{3}{*}{Different} & \multirow{3}{*}{Different}
\\
& E = No Data \tab Freq =No Data   &    &  \\ \cline{2-2}
& 4NFAACb   & \multicolumn{2}{c}{\multirow{3}{*}
 {RMS = 0.105609}}
\\
& E = No Data \tab Freq =No Data   &    \multicolumn{2}{c}{}  \\ \hline
\multirow{4}{*}{\tab[2mm] 229 \tab[2mm]} & 3NFAACm &
\multirow{3}{*}{Different} & \multirow{3}{*}{Different}
\\
& E = No Data \tab Freq =No Data   &    &  \\ \cline{2-2}
& 4NFAACc   & \multicolumn{2}{c}{\multirow{3}{*}
 {RMS = 0.134207}}
\\
& E = No Data \tab Freq =No Data   &    \multicolumn{2}{c}{}  \\ \hline
\multirow{4}{*}{\tab[2mm] 230 \tab[2mm]} & 3NFAACm &
\multirow{3}{*}{Different} & \multirow{3}{*}{Different}
\\
& E = No Data \tab Freq =No Data   &    &  \\ \cline{2-2}
& 4NFAACd   & \multicolumn{2}{c}{\multirow{3}{*}
 {RMS = 0.137925}}
\\
& E = No Data \tab Freq =No Data   &    \multicolumn{2}{c}{}  \\ \hline
\multirow{4}{*}{\tab[2mm] 231 \tab[2mm]} & 3NFAACm &
\multirow{3}{*}{Different} & \multirow{3}{*}{Different}
\\
& E = No Data \tab Freq =No Data   &    &  \\ \cline{2-2}
& 4NFAACe   & \multicolumn{2}{c}{\multirow{3}{*}
 {RMS = 0.129989}}
\\
& E = No Data \tab Freq =No Data   &    \multicolumn{2}{c}{}  \\ \hline
\end{tabular}
\newpage

\vtab[-2cm]
\tab[-2cm]
\begin{tabular}{c|m{8cm}|c|c}
\# & Moléculas & Restultado esperado & Resultado programa \\\\ \hline\hline
\multirow{4}{*}{\tab[2mm] 232 \tab[2mm]} & 3NFAACm &
\multirow{3}{*}{Different} & \multirow{3}{*}{Different}
\\
& E = No Data \tab Freq =No Data   &    &  \\ \cline{2-2}
& 4NFAACf   & \multicolumn{2}{c}{\multirow{3}{*}
 {RMS = 0.0345305}}
\\
& E = No Data \tab Freq =No Data   &    \multicolumn{2}{c}{}  \\ \hline
\multirow{4}{*}{\tab[2mm] 233 \tab[2mm]} & 3NFAACm &
\multirow{3}{*}{Different} & \multirow{3}{*}{Different}
\\
& E = No Data \tab Freq =No Data   &    &  \\ \cline{2-2}
& 4NFAACg   & \multicolumn{2}{c}{\multirow{3}{*}
 {RMS = 0.118974}}
\\
& E = No Data \tab Freq =No Data   &    \multicolumn{2}{c}{}  \\ \hline
\multirow{4}{*}{\tab[2mm] 234 \tab[2mm]} & 3NFAACm &
\multirow{3}{*}{Different} & \multirow{3}{*}{Different}
\\
& E = No Data \tab Freq =No Data   &    &  \\ \cline{2-2}
& 4NFAACi   & \multicolumn{2}{c}{\multirow{3}{*}
 {RMS = 0.154295}}
\\
& E = No Data \tab Freq =No Data   &    \multicolumn{2}{c}{}  \\ \hline
\multirow{4}{*}{\tab[2mm] 235 \tab[2mm]} & 3NFAACm &
\multirow{3}{*}{Different} & \multirow{3}{*}{Different}
\\
& E = No Data \tab Freq =No Data   &    &  \\ \cline{2-2}
& 4NFAACj   & \multicolumn{2}{c}{\multirow{3}{*}
 {RMS = 0.0659973}}
\\
& E = No Data \tab Freq =No Data   &    \multicolumn{2}{c}{}  \\ \hline
\multirow{4}{*}{\tab[2mm] 236 \tab[2mm]} & 3NFAACm &
\multirow{3}{*}{Different} & \multirow{3}{*}{Different}
\\
& E = No Data \tab Freq =No Data   &    &  \\ \cline{2-2}
& 4NFAACl-3   & \multicolumn{2}{c}{\multirow{3}{*}
 {RMS = 0.11135}}
\\
& E = No Data \tab Freq =No Data   &    \multicolumn{2}{c}{}  \\ \hline
\multirow{4}{*}{\tab[2mm] 237 \tab[2mm]} & 3NFAACn &
\multirow{3}{*}{Different} & \multirow{3}{*}{Different}
\\
& E = No Data \tab Freq =No Data   &    &  \\ \cline{2-2}
& 4NFAACa   & \multicolumn{2}{c}{\multirow{3}{*}
 {RMS = 0.37696}}
\\
& E = No Data \tab Freq =No Data   &    \multicolumn{2}{c}{}  \\ \hline
\multirow{4}{*}{\tab[2mm] 238 \tab[2mm]} & 3NFAACn &
\multirow{3}{*}{Different} & \multirow{3}{*}{Different}
\\
& E = No Data \tab Freq =No Data   &    &  \\ \cline{2-2}
& 4NFAACb   & \multicolumn{2}{c}{\multirow{3}{*}
 {RMS = 0.10126}}
\\
& E = No Data \tab Freq =No Data   &    \multicolumn{2}{c}{}  \\ \hline
\end{tabular}
\newpage

\vtab[-2cm]
\tab[-2cm]
\begin{tabular}{c|m{8cm}|c|c}
\# & Moléculas & Restultado esperado & Resultado programa \\\\ \hline\hline
\multirow{4}{*}{\tab[2mm] 239 \tab[2mm]} & 3NFAACn &
\multirow{3}{*}{Different} & \multirow{3}{*}{Different}
\\
& E = No Data \tab Freq =No Data   &    &  \\ \cline{2-2}
& 4NFAACc   & \multicolumn{2}{c}{\multirow{3}{*}
 {RMS = 0.138557}}
\\
& E = No Data \tab Freq =No Data   &    \multicolumn{2}{c}{}  \\ \hline
\multirow{4}{*}{\tab[2mm] 240 \tab[2mm]} & 3NFAACn &
\multirow{3}{*}{Different} & \multirow{3}{*}{Different}
\\
& E = No Data \tab Freq =No Data   &    &  \\ \cline{2-2}
& 4NFAACd   & \multicolumn{2}{c}{\multirow{3}{*}
 {RMS = 0.142274}}
\\
& E = No Data \tab Freq =No Data   &    \multicolumn{2}{c}{}  \\ \hline
\multirow{4}{*}{\tab[2mm] 241 \tab[2mm]} & 3NFAACn &
\multirow{3}{*}{Different} & \multirow{3}{*}{Different}
\\
& E = No Data \tab Freq =No Data   &    &  \\ \cline{2-2}
& 4NFAACe   & \multicolumn{2}{c}{\multirow{3}{*}
 {RMS = 0.134338}}
\\
& E = No Data \tab Freq =No Data   &    \multicolumn{2}{c}{}  \\ \hline
\multirow{4}{*}{\tab[2mm] 242 \tab[2mm]} & 3NFAACn &
\multirow{3}{*}{Different} & \multirow{3}{*}{Different}
\\
& E = No Data \tab Freq =No Data   &    &  \\ \cline{2-2}
& 4NFAACf   & \multicolumn{2}{c}{\multirow{3}{*}
 {RMS = 0.0388799}}
\\
& E = No Data \tab Freq =No Data   &    \multicolumn{2}{c}{}  \\ \hline
\multirow{4}{*}{\tab[2mm] 243 \tab[2mm]} & 3NFAACn &
\multirow{3}{*}{Different} & \multirow{3}{*}{Different}
\\
& E = No Data \tab Freq =No Data   &    &  \\ \cline{2-2}
& 4NFAACg   & \multicolumn{2}{c}{\multirow{3}{*}
 {RMS = 0.123324}}
\\
& E = No Data \tab Freq =No Data   &    \multicolumn{2}{c}{}  \\ \hline
\multirow{4}{*}{\tab[2mm] 244 \tab[2mm]} & 3NFAACn &
\multirow{3}{*}{Different} & \multirow{3}{*}{Different}
\\
& E = No Data \tab Freq =No Data   &    &  \\ \cline{2-2}
& 4NFAACi   & \multicolumn{2}{c}{\multirow{3}{*}
 {RMS = 0.158645}}
\\
& E = No Data \tab Freq =No Data   &    \multicolumn{2}{c}{}  \\ \hline
\multirow{4}{*}{\tab[2mm] 245 \tab[2mm]} & 3NFAACn &
\multirow{3}{*}{Different} & \multirow{3}{*}{Different}
\\
& E = No Data \tab Freq =No Data   &    &  \\ \cline{2-2}
& 4NFAACj   & \multicolumn{2}{c}{\multirow{3}{*}
 {RMS = 0.0703467}}
\\
& E = No Data \tab Freq =No Data   &    \multicolumn{2}{c}{}  \\ \hline
\end{tabular}
\newpage

\vtab[-2cm]
\tab[-2cm]
\begin{tabular}{c|m{8cm}|c|c}
\# & Moléculas & Restultado esperado & Resultado programa \\\\ \hline\hline
\multirow{4}{*}{\tab[2mm] 246 \tab[2mm]} & 3NFAACn &
\multirow{3}{*}{Different} & \multirow{3}{*}{Different}
\\
& E = No Data \tab Freq =No Data   &    &  \\ \cline{2-2}
& 4NFAACl-3   & \multicolumn{2}{c}{\multirow{3}{*}
 {RMS = 0.1157}}
\\
& E = No Data \tab Freq =No Data   &    \multicolumn{2}{c}{}  \\ \hline
\multirow{4}{*}{\tab[2mm] 247 \tab[2mm]} & 4NFAACa &
\multirow{3}{*}{\textcolor{Red}{\bf Stereoisomer}} & \multirow{3}{*}{\textcolor{Red}{\bf Enantiomers}}
\\
& E = No Data \tab Freq =No Data   &    &  \\ \cline{2-2}
& 4NFAACb   & \multicolumn{2}{c}{\multirow{3}{*}
 {RMS = 0.485443}}
\\
& E = No Data \tab Freq =No Data   &    \multicolumn{2}{c}{}  \\ \hline
\multirow{4}{*}{\tab[2mm] 248 \tab[2mm]} & 4NFAACa &
\multirow{3}{*}{Different} & \multirow{3}{*}{Different}
\\
& E = No Data \tab Freq =No Data   &    &  \\ \cline{2-2}
& 4NFAACc   & \multicolumn{2}{c}{\multirow{3}{*}
 {RMS = 0.238403}}
\\
& E = No Data \tab Freq =No Data   &    \multicolumn{2}{c}{}  \\ \hline
\multirow{4}{*}{\tab[2mm] 249 \tab[2mm]} & 4NFAACa &
\multirow{3}{*}{Different} & \multirow{3}{*}{Different}
\\
& E = No Data \tab Freq =No Data   &    &  \\ \cline{2-2}
& 4NFAACd   & \multicolumn{2}{c}{\multirow{3}{*}
 {RMS = 0.234686}}
\\
& E = No Data \tab Freq =No Data   &    \multicolumn{2}{c}{}  \\ \hline
\multirow{4}{*}{\tab[2mm] 250 \tab[2mm]} & 4NFAACa &
\multirow{3}{*}{Different} & \multirow{3}{*}{Different}
\\
& E = No Data \tab Freq =No Data   &    &  \\ \cline{2-2}
& 4NFAACe   & \multicolumn{2}{c}{\multirow{3}{*}
 {RMS = 0.242622}}
\\
& E = No Data \tab Freq =No Data   &    \multicolumn{2}{c}{}  \\ \hline
\multirow{4}{*}{\tab[2mm] 251 \tab[2mm]} & 4NFAACa &
\multirow{3}{*}{Different} & \multirow{3}{*}{Different}
\\
& E = No Data \tab Freq =No Data   &    &  \\ \cline{2-2}
& 4NFAACf   & \multicolumn{2}{c}{\multirow{3}{*}
 {RMS = 0.33808}}
\\
& E = No Data \tab Freq =No Data   &    \multicolumn{2}{c}{}  \\ \hline
\multirow{4}{*}{\tab[2mm] 252 \tab[2mm]} & 4NFAACa &
\multirow{3}{*}{Different} & \multirow{3}{*}{Different}
\\
& E = No Data \tab Freq =No Data   &    &  \\ \cline{2-2}
& 4NFAACg   & \multicolumn{2}{c}{\multirow{3}{*}
 {RMS = 0.253636}}
\\
& E = No Data \tab Freq =No Data   &    \multicolumn{2}{c}{}  \\ \hline
\end{tabular}
\newpage

\vtab[-2cm]
\tab[-2cm]
\begin{tabular}{c|m{8cm}|c|c}
\# & Moléculas & Restultado esperado & Resultado programa \\\\ \hline\hline
\multirow{4}{*}{\tab[2mm] 253 \tab[2mm]} & 4NFAACa &
\multirow{3}{*}{Different} & \multirow{3}{*}{Different}
\\
& E = No Data \tab Freq =No Data   &    &  \\ \cline{2-2}
& 4NFAACi   & \multicolumn{2}{c}{\multirow{3}{*}
 {RMS = 0.218315}}
\\
& E = No Data \tab Freq =No Data   &    \multicolumn{2}{c}{}  \\ \hline
\multirow{4}{*}{\tab[2mm] 254 \tab[2mm]} & 4NFAACa &
\multirow{3}{*}{Different} & \multirow{3}{*}{Different}
\\
& E = No Data \tab Freq =No Data   &    &  \\ \cline{2-2}
& 4NFAACj   & \multicolumn{2}{c}{\multirow{3}{*}
 {RMS = 0.306613}}
\\
& E = No Data \tab Freq =No Data   &    \multicolumn{2}{c}{}  \\ \hline
\multirow{4}{*}{\tab[2mm] 255 \tab[2mm]} & 4NFAACa &
\multirow{3}{*}{Different} & \multirow{3}{*}{Different}
\\
& E = No Data \tab Freq =No Data   &    &  \\ \cline{2-2}
& 4NFAACl-3   & \multicolumn{2}{c}{\multirow{3}{*}
 {RMS = 0.26126}}
\\
& E = No Data \tab Freq =No Data   &    \multicolumn{2}{c}{}  \\ \hline
\multirow{4}{*}{\tab[2mm] 256 \tab[2mm]} & 4NFAACb &
\multirow{3}{*}{Different} & \multirow{3}{*}{Different}
\\
& E = No Data \tab Freq =No Data   &    &  \\ \cline{2-2}
& 4NFAACc   & \multicolumn{2}{c}{\multirow{3}{*}
 {RMS = 0.239816}}
\\
& E = No Data \tab Freq =No Data   &    \multicolumn{2}{c}{}  \\ \hline
\multirow{4}{*}{\tab[2mm] 257 \tab[2mm]} & 4NFAACb &
\multirow{3}{*}{Different} & \multirow{3}{*}{Different}
\\
& E = No Data \tab Freq =No Data   &    &  \\ \cline{2-2}
& 4NFAACd   & \multicolumn{2}{c}{\multirow{3}{*}
 {RMS = 0.243534}}
\\
& E = No Data \tab Freq =No Data   &    \multicolumn{2}{c}{}  \\ \hline
\multirow{4}{*}{\tab[2mm] 258 \tab[2mm]} & 4NFAACb &
\multirow{3}{*}{Different} & \multirow{3}{*}{Different}
\\
& E = No Data \tab Freq =No Data   &    &  \\ \cline{2-2}
& 4NFAACe   & \multicolumn{2}{c}{\multirow{3}{*}
 {RMS = 0.235598}}
\\
& E = No Data \tab Freq =No Data   &    \multicolumn{2}{c}{}  \\ \hline
\multirow{4}{*}{\tab[2mm] 259 \tab[2mm]} & 4NFAACb &
\multirow{3}{*}{Different} & \multirow{3}{*}{Different}
\\
& E = No Data \tab Freq =No Data   &    &  \\ \cline{2-2}
& 4NFAACf   & \multicolumn{2}{c}{\multirow{3}{*}
 {RMS = 0.14014}}
\\
& E = No Data \tab Freq =No Data   &    \multicolumn{2}{c}{}  \\ \hline
\end{tabular}
\newpage

\vtab[-2cm]
\tab[-2cm]
\begin{tabular}{c|m{8cm}|c|c}
\# & Moléculas & Restultado esperado & Resultado programa \\\\ \hline\hline
\multirow{4}{*}{\tab[2mm] 260 \tab[2mm]} & 4NFAACb &
\multirow{3}{*}{Different} & \multirow{3}{*}{Different}
\\
& E = No Data \tab Freq =No Data   &    &  \\ \cline{2-2}
& 4NFAACg   & \multicolumn{2}{c}{\multirow{3}{*}
 {RMS = 0.224583}}
\\
& E = No Data \tab Freq =No Data   &    \multicolumn{2}{c}{}  \\ \hline
\multirow{4}{*}{\tab[2mm] 261 \tab[2mm]} & 4NFAACb &
\multirow{3}{*}{Different} & \multirow{3}{*}{Different}
\\
& E = No Data \tab Freq =No Data   &    &  \\ \cline{2-2}
& 4NFAACi   & \multicolumn{2}{c}{\multirow{3}{*}
 {RMS = 0.259904}}
\\
& E = No Data \tab Freq =No Data   &    \multicolumn{2}{c}{}  \\ \hline
\multirow{4}{*}{\tab[2mm] 262 \tab[2mm]} & 4NFAACb &
\multirow{3}{*}{Different} & \multirow{3}{*}{Different}
\\
& E = No Data \tab Freq =No Data   &    &  \\ \cline{2-2}
& 4NFAACj   & \multicolumn{2}{c}{\multirow{3}{*}
 {RMS = 0.171606}}
\\
& E = No Data \tab Freq =No Data   &    \multicolumn{2}{c}{}  \\ \hline
\multirow{4}{*}{\tab[2mm] 263 \tab[2mm]} & 4NFAACb &
\multirow{3}{*}{Different} & \multirow{3}{*}{Different}
\\
& E = No Data \tab Freq =No Data   &    &  \\ \cline{2-2}
& 4NFAACl-3   & \multicolumn{2}{c}{\multirow{3}{*}
 {RMS = 0.216959}}
\\
& E = No Data \tab Freq =No Data   &    \multicolumn{2}{c}{}  \\ \hline
\multirow{4}{*}{\tab[2mm] 264 \tab[2mm]} & 4NFAACc &
\multirow{3}{*}{Different} & \multirow{3}{*}{Different}
\\
& E = No Data \tab Freq =No Data   &    &  \\ \cline{2-2}
& 4NFAACd   & \multicolumn{2}{c}{\multirow{3}{*}
 {RMS = 0.00371771}}
\\
& E = No Data \tab Freq =No Data   &    \multicolumn{2}{c}{}  \\ \hline
\multirow{4}{*}{\tab[2mm] 265 \tab[2mm]} & 4NFAACc &
\multirow{3}{*}{Different} & \multirow{3}{*}{Different}
\\
& E = No Data \tab Freq =No Data   &    &  \\ \cline{2-2}
& 4NFAACe   & \multicolumn{2}{c}{\multirow{3}{*}
 {RMS = 0.00421832}}
\\
& E = No Data \tab Freq =No Data   &    \multicolumn{2}{c}{}  \\ \hline
\multirow{4}{*}{\tab[2mm] 266 \tab[2mm]} & 4NFAACc &
\multirow{3}{*}{Different} & \multirow{3}{*}{Different}
\\
& E = No Data \tab Freq =No Data   &    &  \\ \cline{2-2}
& 4NFAACf   & \multicolumn{2}{c}{\multirow{3}{*}
 {RMS = 0.0996768}}
\\
& E = No Data \tab Freq =No Data   &    \multicolumn{2}{c}{}  \\ \hline
\end{tabular}
\newpage

\vtab[-2cm]
\tab[-2cm]
\begin{tabular}{c|m{8cm}|c|c}
\# & Moléculas & Restultado esperado & Resultado programa \\\\ \hline\hline
\multirow{4}{*}{\tab[2mm] 267 \tab[2mm]} & 4NFAACc &
\multirow{3}{*}{Different} & \multirow{3}{*}{Different}
\\
& E = No Data \tab Freq =No Data   &    &  \\ \cline{2-2}
& 4NFAACg   & \multicolumn{2}{c}{\multirow{3}{*}
 {RMS = 0.015233}}
\\
& E = No Data \tab Freq =No Data   &    \multicolumn{2}{c}{}  \\ \hline
\multirow{4}{*}{\tab[2mm] 268 \tab[2mm]} & 4NFAACc &
\multirow{3}{*}{Different} & \multirow{3}{*}{Different}
\\
& E = No Data \tab Freq =No Data   &    &  \\ \cline{2-2}
& 4NFAACi   & \multicolumn{2}{c}{\multirow{3}{*}
 {RMS = 0.0200879}}
\\
& E = No Data \tab Freq =No Data   &    \multicolumn{2}{c}{}  \\ \hline
\multirow{4}{*}{\tab[2mm] 269 \tab[2mm]} & 4NFAACc &
\multirow{3}{*}{Different} & \multirow{3}{*}{Different}
\\
& E = No Data \tab Freq =No Data   &    &  \\ \cline{2-2}
& 4NFAACj   & \multicolumn{2}{c}{\multirow{3}{*}
 {RMS = 0.0682101}}
\\
& E = No Data \tab Freq =No Data   &    \multicolumn{2}{c}{}  \\ \hline
\multirow{4}{*}{\tab[2mm] 270 \tab[2mm]} & 4NFAACc &
\multirow{3}{*}{Different} & \multirow{3}{*}{Different}
\\
& E = No Data \tab Freq =No Data   &    &  \\ \cline{2-2}
& 4NFAACl-3   & \multicolumn{2}{c}{\multirow{3}{*}
 {RMS = 0.0228571}}
\\
& E = No Data \tab Freq =No Data   &    \multicolumn{2}{c}{}  \\ \hline
\multirow{4}{*}{\tab[2mm] 271 \tab[2mm]} & 4NFAACd &
\multirow{3}{*}{Different} & \multirow{3}{*}{Different}
\\
& E = No Data \tab Freq =No Data   &    &  \\ \cline{2-2}
& 4NFAACe   & \multicolumn{2}{c}{\multirow{3}{*}
 {RMS = 0.00793602}}
\\
& E = No Data \tab Freq =No Data   &    \multicolumn{2}{c}{}  \\ \hline
\multirow{4}{*}{\tab[2mm] 272 \tab[2mm]} & 4NFAACd &
\multirow{3}{*}{Different} & \multirow{3}{*}{Different}
\\
& E = No Data \tab Freq =No Data   &    &  \\ \cline{2-2}
& 4NFAACf   & \multicolumn{2}{c}{\multirow{3}{*}
 {RMS = 0.103395}}
\\
& E = No Data \tab Freq =No Data   &    \multicolumn{2}{c}{}  \\ \hline
\multirow{4}{*}{\tab[2mm] 273 \tab[2mm]} & 4NFAACd &
\multirow{3}{*}{Different} & \multirow{3}{*}{Different}
\\
& E = No Data \tab Freq =No Data   &    &  \\ \cline{2-2}
& 4NFAACg   & \multicolumn{2}{c}{\multirow{3}{*}
 {RMS = 0.0189507}}
\\
& E = No Data \tab Freq =No Data   &    \multicolumn{2}{c}{}  \\ \hline
\end{tabular}
\newpage

\vtab[-2cm]
\tab[-2cm]
\begin{tabular}{c|m{8cm}|c|c}
\# & Moléculas & Restultado esperado & Resultado programa \\\\ \hline\hline
\multirow{4}{*}{\tab[2mm] 274 \tab[2mm]} & 4NFAACd &
\multirow{3}{*}{Different} & \multirow{3}{*}{Different}
\\
& E = No Data \tab Freq =No Data   &    &  \\ \cline{2-2}
& 4NFAACi   & \multicolumn{2}{c}{\multirow{3}{*}
 {RMS = 0.0163702}}
\\
& E = No Data \tab Freq =No Data   &    \multicolumn{2}{c}{}  \\ \hline
\multirow{4}{*}{\tab[2mm] 275 \tab[2mm]} & 4NFAACd &
\multirow{3}{*}{Different} & \multirow{3}{*}{Different}
\\
& E = No Data \tab Freq =No Data   &    &  \\ \cline{2-2}
& 4NFAACj   & \multicolumn{2}{c}{\multirow{3}{*}
 {RMS = 0.0719278}}
\\
& E = No Data \tab Freq =No Data   &    \multicolumn{2}{c}{}  \\ \hline
\multirow{4}{*}{\tab[2mm] 276 \tab[2mm]} & 4NFAACd &
\multirow{3}{*}{Different} & \multirow{3}{*}{Different}
\\
& E = No Data \tab Freq =No Data   &    &  \\ \cline{2-2}
& 4NFAACl-3   & \multicolumn{2}{c}{\multirow{3}{*}
 {RMS = 0.0265748}}
\\
& E = No Data \tab Freq =No Data   &    \multicolumn{2}{c}{}  \\ \hline
\multirow{4}{*}{\tab[2mm] 277 \tab[2mm]} & 4NFAACe &
\multirow{3}{*}{Different} & \multirow{3}{*}{Different}
\\
& E = No Data \tab Freq =No Data   &    &  \\ \cline{2-2}
& 4NFAACf   & \multicolumn{2}{c}{\multirow{3}{*}
 {RMS = 0.0954585}}
\\
& E = No Data \tab Freq =No Data   &    \multicolumn{2}{c}{}  \\ \hline
\multirow{4}{*}{\tab[2mm] 278 \tab[2mm]} & 4NFAACe &
\multirow{3}{*}{Different} & \multirow{3}{*}{Different}
\\
& E = No Data \tab Freq =No Data   &    &  \\ \cline{2-2}
& 4NFAACg   & \multicolumn{2}{c}{\multirow{3}{*}
 {RMS = 0.0110146}}
\\
& E = No Data \tab Freq =No Data   &    \multicolumn{2}{c}{}  \\ \hline
\multirow{4}{*}{\tab[2mm] 279 \tab[2mm]} & 4NFAACe &
\multirow{3}{*}{Different} & \multirow{3}{*}{Different}
\\
& E = No Data \tab Freq =No Data   &    &  \\ \cline{2-2}
& 4NFAACi   & \multicolumn{2}{c}{\multirow{3}{*}
 {RMS = 0.0243062}}
\\
& E = No Data \tab Freq =No Data   &    \multicolumn{2}{c}{}  \\ \hline
\multirow{4}{*}{\tab[2mm] 280 \tab[2mm]} & 4NFAACe &
\multirow{3}{*}{Different} & \multirow{3}{*}{Different}
\\
& E = No Data \tab Freq =No Data   &    &  \\ \cline{2-2}
& 4NFAACj   & \multicolumn{2}{c}{\multirow{3}{*}
 {RMS = 0.0639918}}
\\
& E = No Data \tab Freq =No Data   &    \multicolumn{2}{c}{}  \\ \hline
\end{tabular}
\newpage

\vtab[-2cm]
\tab[-2cm]
\begin{tabular}{c|m{8cm}|c|c}
\# & Moléculas & Restultado esperado & Resultado programa \\\\ \hline\hline
\multirow{4}{*}{\tab[2mm] 281 \tab[2mm]} & 4NFAACe &
\multirow{3}{*}{Different} & \multirow{3}{*}{Different}
\\
& E = No Data \tab Freq =No Data   &    &  \\ \cline{2-2}
& 4NFAACl-3   & \multicolumn{2}{c}{\multirow{3}{*}
 {RMS = 0.0186388}}
\\
& E = No Data \tab Freq =No Data   &    \multicolumn{2}{c}{}  \\ \hline
\multirow{4}{*}{\tab[2mm] 282 \tab[2mm]} & 4NFAACf &
\multirow{3}{*}{Different} & \multirow{3}{*}{Different}
\\
& E = No Data \tab Freq =No Data   &    &  \\ \cline{2-2}
& 4NFAACg   & \multicolumn{2}{c}{\multirow{3}{*}
 {RMS = 0.0844439}}
\\
& E = No Data \tab Freq =No Data   &    \multicolumn{2}{c}{}  \\ \hline
\multirow{4}{*}{\tab[2mm] 283 \tab[2mm]} & 4NFAACf &
\multirow{3}{*}{Different} & \multirow{3}{*}{Different}
\\
& E = No Data \tab Freq =No Data   &    &  \\ \cline{2-2}
& 4NFAACi   & \multicolumn{2}{c}{\multirow{3}{*}
 {RMS = 0.119765}}
\\
& E = No Data \tab Freq =No Data   &    \multicolumn{2}{c}{}  \\ \hline
\multirow{4}{*}{\tab[2mm] 284 \tab[2mm]} & 4NFAACf &
\multirow{3}{*}{Different} & \multirow{3}{*}{Different}
\\
& E = No Data \tab Freq =No Data   &    &  \\ \cline{2-2}
& 4NFAACj   & \multicolumn{2}{c}{\multirow{3}{*}
 {RMS = 0.0314668}}
\\
& E = No Data \tab Freq =No Data   &    \multicolumn{2}{c}{}  \\ \hline
\multirow{4}{*}{\tab[2mm] 285 \tab[2mm]} & 4NFAACf &
\multirow{3}{*}{Different} & \multirow{3}{*}{Different}
\\
& E = No Data \tab Freq =No Data   &    &  \\ \cline{2-2}
& 4NFAACl-3   & \multicolumn{2}{c}{\multirow{3}{*}
 {RMS = 0.0768197}}
\\
& E = No Data \tab Freq =No Data   &    \multicolumn{2}{c}{}  \\ \hline
\multirow{4}{*}{\tab[2mm] 286 \tab[2mm]} & 4NFAACg &
\multirow{3}{*}{Different} & \multirow{3}{*}{Different}
\\
& E = No Data \tab Freq =No Data   &    &  \\ \cline{2-2}
& 4NFAACi   & \multicolumn{2}{c}{\multirow{3}{*}
 {RMS = 0.0353209}}
\\
& E = No Data \tab Freq =No Data   &    \multicolumn{2}{c}{}  \\ \hline
\multirow{4}{*}{\tab[2mm] 287 \tab[2mm]} & 4NFAACg &
\multirow{3}{*}{Different} & \multirow{3}{*}{Different}
\\
& E = No Data \tab Freq =No Data   &    &  \\ \cline{2-2}
& 4NFAACj   & \multicolumn{2}{c}{\multirow{3}{*}
 {RMS = 0.0529771}}
\\
& E = No Data \tab Freq =No Data   &    \multicolumn{2}{c}{}  \\ \hline
\end{tabular}
\newpage

\vtab[-2cm]
\tab[-2cm]
\begin{tabular}{c|m{8cm}|c|c}
\# & Moléculas & Restultado esperado & Resultado programa \\\\ \hline\hline
\multirow{4}{*}{\tab[2mm] 288 \tab[2mm]} & 4NFAACg &
\multirow{3}{*}{Different} & \multirow{3}{*}{Different}
\\
& E = No Data \tab Freq =No Data   &    &  \\ \cline{2-2}
& 4NFAACl-3   & \multicolumn{2}{c}{\multirow{3}{*}
 {RMS = 0.00762414}}
\\
& E = No Data \tab Freq =No Data   &    \multicolumn{2}{c}{}  \\ \hline
\multirow{4}{*}{\tab[2mm] 289 \tab[2mm]} & 4NFAACi &
\multirow{3}{*}{Different} & \multirow{3}{*}{Different}
\\
& E = No Data \tab Freq =No Data   &    &  \\ \cline{2-2}
& 4NFAACj   & \multicolumn{2}{c}{\multirow{3}{*}
 {RMS = 0.088298}}
\\
& E = No Data \tab Freq =No Data   &    \multicolumn{2}{c}{}  \\ \hline
\multirow{4}{*}{\tab[2mm] 290 \tab[2mm]} & 4NFAACi &
\multirow{3}{*}{Different} & \multirow{3}{*}{Different}
\\
& E = No Data \tab Freq =No Data   &    &  \\ \cline{2-2}
& 4NFAACl-3   & \multicolumn{2}{c}{\multirow{3}{*}
 {RMS = 0.042945}}
\\
& E = No Data \tab Freq =No Data   &    \multicolumn{2}{c}{}  \\ \hline
\multirow{4}{*}{\tab[2mm] 291 \tab[2mm]} & 4NFAACj &
\multirow{3}{*}{Different} & \multirow{3}{*}{Different}
\\
& E = No Data \tab Freq =No Data   &    &  \\ \cline{2-2}
& 4NFAACl-3   & \multicolumn{2}{c}{\multirow{3}{*}
 {RMS = 0.045353}}
\\
& E = No Data \tab Freq =No Data   &    \multicolumn{2}{c}{}  \\ \hline
\multirow{4}{*}{\tab[2mm] 292 \tab[2mm]} & 2,2-dimethyl-butane\_out\_G09 &
\multirow{3}{*}{\textcolor{Red}{\bf Without comparation}} & \multirow{3}{*}{\textcolor{Red}{\bf Equal}}
\\
& E = No Data \tab Freq =No Data   &    &  \\ \cline{2-2}
& 2,2-dimethyl-butane\_out\_G09\_invertion   & \multicolumn{2}{c}{\multirow{3}{*}
{ RMS = 0}}
\\
& E = No Data \tab Freq =No Data   &    \multicolumn{2}{c}{}  \\ \hline
\multirow{4}{*}{\tab[2mm] 293 \tab[2mm]} & 2,2-dimethyl-butane\_out\_G09 &
\multirow{3}{*}{Different} & \multirow{3}{*}{Different}
\\
& E = No Data \tab Freq =No Data   &    &  \\ \cline{2-2}
& 2,3-dimethyl-butane\_out\_G09   & \multicolumn{2}{c}{\multirow{3}{*}
 {RMS = 0.269312}}
\\
& E = No Data \tab Freq =No Data   &    \multicolumn{2}{c}{}  \\ \hline
\multirow{4}{*}{\tab[2mm] 294 \tab[2mm]} & 2,2-dimethyl-butane\_out\_G09 &
\multirow{3}{*}{Different} & \multirow{3}{*}{Different}
\\
& E = No Data \tab Freq =No Data   &    &  \\ \cline{2-2}
& 2,3-dimethyl-butane\_out\_G09\_invertion   & \multicolumn{2}{c}{\multirow{3}{*}
 {RMS = 0.269312}}
\\
& E = No Data \tab Freq =No Data   &    \multicolumn{2}{c}{}  \\ \hline
\end{tabular}
\newpage

\vtab[-2cm]
\tab[-2cm]
\begin{tabular}{c|m{8cm}|c|c}
\# & Moléculas & Restultado esperado & Resultado programa \\\\ \hline\hline
\multirow{4}{*}{\tab[2mm] 295 \tab[2mm]} & 2,2-dimethyl-butane\_out\_G09 &
\multirow{3}{*}{Different} & \multirow{3}{*}{Different}
\\
& E = No Data \tab Freq =No Data   &    &  \\ \cline{2-2}
& 2-methyl-pentane\_out\_G09   & \multicolumn{2}{c}{\multirow{3}{*}
 {RMS = 0.0426031}}
\\
& E = No Data \tab Freq =No Data   &    \multicolumn{2}{c}{}  \\ \hline
\multirow{4}{*}{\tab[2mm] 296 \tab[2mm]} & 2,2-dimethyl-butane\_out\_G09 &
\multirow{3}{*}{Different} & \multirow{3}{*}{Different}
\\
& E = No Data \tab Freq =No Data   &    &  \\ \cline{2-2}
& 2-methyl-pentane\_out\_G09\_invertion   & \multicolumn{2}{c}{\multirow{3}{*}
 {RMS = 0.0778147}}
\\
& E = No Data \tab Freq =No Data   &    \multicolumn{2}{c}{}  \\ \hline
\multirow{4}{*}{\tab[2mm] 297 \tab[2mm]} & 2,2-dimethyl-butane\_out\_G09 &
\multirow{3}{*}{Different} & \multirow{3}{*}{Different}
\\
& E = No Data \tab Freq =No Data   &    &  \\ \cline{2-2}
& 3-methyl-pentane\_out\_G09   & \multicolumn{2}{c}{\multirow{3}{*}
 {RMS = 0.766027}}
\\
& E = No Data \tab Freq =No Data   &    \multicolumn{2}{c}{}  \\ \hline
\multirow{4}{*}{\tab[2mm] 298 \tab[2mm]} & 2,2-dimethyl-butane\_out\_G09 &
\multirow{3}{*}{Different} & \multirow{3}{*}{Different}
\\
& E = No Data \tab Freq =No Data   &    &  \\ \cline{2-2}
& 3-methyl-pentane\_out\_G09\_invertion   & \multicolumn{2}{c}{\multirow{3}{*}
 {RMS = 0.766028}}
\\
& E = No Data \tab Freq =No Data   &    \multicolumn{2}{c}{}  \\ \hline
\multirow{4}{*}{\tab[2mm] 299 \tab[2mm]} & 2,2-dimethyl-butane\_out\_G09 &
\multirow{3}{*}{Different} & \multirow{3}{*}{Different}
\\
& E = No Data \tab Freq =No Data   &    &  \\ \cline{2-2}
& hexane\_out\_G09   & \multicolumn{2}{c}{\multirow{3}{*}
 {RMS = 0}}
\\
& E = No Data \tab Freq =No Data   &    \multicolumn{2}{c}{}  \\ \hline
\multirow{4}{*}{\tab[2mm] 300 \tab[2mm]} & 2,2-dimethyl-butane\_out\_G09 &
\multirow{3}{*}{Different} & \multirow{3}{*}{Different}
\\
& E = No Data \tab Freq =No Data   &    &  \\ \cline{2-2}
& hexane\_out\_G09\_invertion   & \multicolumn{2}{c}{\multirow{3}{*}
 {RMS = 0}}
\\
& E = No Data \tab Freq =No Data   &    \multicolumn{2}{c}{}  \\ \hline
\multirow{4}{*}{\tab[2mm] 301 \tab[2mm]} & 2,2-dimethyl-butane\_out\_G09\_invertion &
\multirow{3}{*}{Different} & \multirow{3}{*}{Different}
\\
& E = No Data \tab Freq =No Data   &    &  \\ \cline{2-2}
& 2,3-dimethyl-butane\_out\_G09   & \multicolumn{2}{c}{\multirow{3}{*}
 {RMS = 0.269312}}
\\
& E = No Data \tab Freq =No Data   &    \multicolumn{2}{c}{}  \\ \hline
\end{tabular}
\newpage

\vtab[-2cm]
\tab[-2cm]
\begin{tabular}{c|m{8cm}|c|c}
\# & Moléculas & Restultado esperado & Resultado programa \\\\ \hline\hline
\multirow{4}{*}{\tab[2mm] 302 \tab[2mm]} & 2,2-dimethyl-butane\_out\_G09\_invertion &
\multirow{3}{*}{Different} & \multirow{3}{*}{Different}
\\
& E = No Data \tab Freq =No Data   &    &  \\ \cline{2-2}
& 2,3-dimethyl-butane\_out\_G09\_invertion   & \multicolumn{2}{c}{\multirow{3}{*}
 {RMS = 0.269312}}
\\
& E = No Data \tab Freq =No Data   &    \multicolumn{2}{c}{}  \\ \hline
\multirow{4}{*}{\tab[2mm] 303 \tab[2mm]} & 2,2-dimethyl-butane\_out\_G09\_invertion &
\multirow{3}{*}{Different} & \multirow{3}{*}{Different}
\\
& E = No Data \tab Freq =No Data   &    &  \\ \cline{2-2}
& 2-methyl-pentane\_out\_G09   & \multicolumn{2}{c}{\multirow{3}{*}
 {RMS = 0.0426031}}
\\
& E = No Data \tab Freq =No Data   &    \multicolumn{2}{c}{}  \\ \hline
\multirow{4}{*}{\tab[2mm] 304 \tab[2mm]} & 2,2-dimethyl-butane\_out\_G09\_invertion &
\multirow{3}{*}{Different} & \multirow{3}{*}{Different}
\\
& E = No Data \tab Freq =No Data   &    &  \\ \cline{2-2}
& 2-methyl-pentane\_out\_G09\_invertion   & \multicolumn{2}{c}{\multirow{3}{*}
 {RMS = 0.0778147}}
\\
& E = No Data \tab Freq =No Data   &    \multicolumn{2}{c}{}  \\ \hline
\multirow{4}{*}{\tab[2mm] 305 \tab[2mm]} & 2,2-dimethyl-butane\_out\_G09\_invertion &
\multirow{3}{*}{Different} & \multirow{3}{*}{Different}
\\
& E = No Data \tab Freq =No Data   &    &  \\ \cline{2-2}
& 3-methyl-pentane\_out\_G09   & \multicolumn{2}{c}{\multirow{3}{*}
 {RMS = 0.766027}}
\\
& E = No Data \tab Freq =No Data   &    \multicolumn{2}{c}{}  \\ \hline
\multirow{4}{*}{\tab[2mm] 306 \tab[2mm]} & 2,2-dimethyl-butane\_out\_G09\_invertion &
\multirow{3}{*}{Different} & \multirow{3}{*}{Different}
\\
& E = No Data \tab Freq =No Data   &    &  \\ \cline{2-2}
& 3-methyl-pentane\_out\_G09\_invertion   & \multicolumn{2}{c}{\multirow{3}{*}
 {RMS = 0.766028}}
\\
& E = No Data \tab Freq =No Data   &    \multicolumn{2}{c}{}  \\ \hline
\multirow{4}{*}{\tab[2mm] 307 \tab[2mm]} & 2,2-dimethyl-butane\_out\_G09\_invertion &
\multirow{3}{*}{Different} & \multirow{3}{*}{Different}
\\
& E = No Data \tab Freq =No Data   &    &  \\ \cline{2-2}
& hexane\_out\_G09   & \multicolumn{2}{c}{\multirow{3}{*}
 {RMS = 0}}
\\
& E = No Data \tab Freq =No Data   &    \multicolumn{2}{c}{}  \\ \hline
\multirow{4}{*}{\tab[2mm] 308 \tab[2mm]} & 2,2-dimethyl-butane\_out\_G09\_invertion &
\multirow{3}{*}{Different} & \multirow{3}{*}{Different}
\\
& E = No Data \tab Freq =No Data   &    &  \\ \cline{2-2}
& hexane\_out\_G09\_invertion   & \multicolumn{2}{c}{\multirow{3}{*}
 {RMS = 0}}
\\
& E = No Data \tab Freq =No Data   &    \multicolumn{2}{c}{}  \\ \hline
\end{tabular}
\newpage

\vtab[-2cm]
\tab[-2cm]
\begin{tabular}{c|m{8cm}|c|c}
\# & Moléculas & Restultado esperado & Resultado programa \\\\ \hline\hline
\multirow{4}{*}{\tab[2mm] 309 \tab[2mm]} & 2,3-dimethyl-butane\_out\_G09 &
\multirow{3}{*}{\textcolor{Red}{\bf Without comparation}} & \multirow{3}{*}{\textcolor{Red}{\bf Equal}}
\\
& E = No Data \tab Freq =No Data   &    &  \\ \cline{2-2}
& 2,3-dimethyl-butane\_out\_G09\_invertion   & \multicolumn{2}{c}{\multirow{3}{*}
{ RMS = 1.773561E-07}}
\\
& E = No Data \tab Freq =No Data   &    \multicolumn{2}{c}{}  \\ \hline
\multirow{4}{*}{\tab[2mm] 310 \tab[2mm]} & 2,3-dimethyl-butane\_out\_G09 &
\multirow{3}{*}{Different} & \multirow{3}{*}{Different}
\\
& E = No Data \tab Freq =No Data   &    &  \\ \cline{2-2}
& 2-methyl-pentane\_out\_G09   & \multicolumn{2}{c}{\multirow{3}{*}
 {RMS = 0.311916}}
\\
& E = No Data \tab Freq =No Data   &    \multicolumn{2}{c}{}  \\ \hline
\multirow{4}{*}{\tab[2mm] 311 \tab[2mm]} & 2,3-dimethyl-butane\_out\_G09 &
\multirow{3}{*}{Different} & \multirow{3}{*}{Different}
\\
& E = No Data \tab Freq =No Data   &    &  \\ \cline{2-2}
& 2-methyl-pentane\_out\_G09\_invertion   & \multicolumn{2}{c}{\multirow{3}{*}
 {RMS = 0.191498}}
\\
& E = No Data \tab Freq =No Data   &    \multicolumn{2}{c}{}  \\ \hline
\multirow{4}{*}{\tab[2mm] 312 \tab[2mm]} & 2,3-dimethyl-butane\_out\_G09 &
\multirow{3}{*}{Different} & \multirow{3}{*}{Different}
\\
& E = No Data \tab Freq =No Data   &    &  \\ \cline{2-2}
& 3-methyl-pentane\_out\_G09   & \multicolumn{2}{c}{\multirow{3}{*}
 {RMS = 0.496715}}
\\
& E = No Data \tab Freq =No Data   &    \multicolumn{2}{c}{}  \\ \hline
\multirow{4}{*}{\tab[2mm] 313 \tab[2mm]} & 2,3-dimethyl-butane\_out\_G09 &
\multirow{3}{*}{Different} & \multirow{3}{*}{Different}
\\
& E = No Data \tab Freq =No Data   &    &  \\ \cline{2-2}
& 3-methyl-pentane\_out\_G09\_invertion   & \multicolumn{2}{c}{\multirow{3}{*}
 {RMS = 0.496715}}
\\
& E = No Data \tab Freq =No Data   &    \multicolumn{2}{c}{}  \\ \hline
\multirow{4}{*}{\tab[2mm] 314 \tab[2mm]} & 2,3-dimethyl-butane\_out\_G09 &
\multirow{3}{*}{Different} & \multirow{3}{*}{Different}
\\
& E = No Data \tab Freq =No Data   &    &  \\ \cline{2-2}
& hexane\_out\_G09   & \multicolumn{2}{c}{\multirow{3}{*}
 {RMS = 0.269312}}
\\
& E = No Data \tab Freq =No Data   &    \multicolumn{2}{c}{}  \\ \hline
\multirow{4}{*}{\tab[2mm] 315 \tab[2mm]} & 2,3-dimethyl-butane\_out\_G09 &
\multirow{3}{*}{Different} & \multirow{3}{*}{Different}
\\
& E = No Data \tab Freq =No Data   &    &  \\ \cline{2-2}
& hexane\_out\_G09\_invertion   & \multicolumn{2}{c}{\multirow{3}{*}
 {RMS = 0.269312}}
\\
& E = No Data \tab Freq =No Data   &    \multicolumn{2}{c}{}  \\ \hline
\end{tabular}
\newpage

\vtab[-2cm]
\tab[-2cm]
\begin{tabular}{c|m{8cm}|c|c}
\# & Moléculas & Restultado esperado & Resultado programa \\\\ \hline\hline
\multirow{4}{*}{\tab[2mm] 316 \tab[2mm]} & 2,3-dimethyl-butane\_out\_G09\_invertion &
\multirow{3}{*}{Different} & \multirow{3}{*}{Different}
\\
& E = No Data \tab Freq =No Data   &    &  \\ \cline{2-2}
& 2-methyl-pentane\_out\_G09   & \multicolumn{2}{c}{\multirow{3}{*}
 {RMS = 0.311915}}
\\
& E = No Data \tab Freq =No Data   &    \multicolumn{2}{c}{}  \\ \hline
\multirow{4}{*}{\tab[2mm] 317 \tab[2mm]} & 2,3-dimethyl-butane\_out\_G09\_invertion &
\multirow{3}{*}{Different} & \multirow{3}{*}{Different}
\\
& E = No Data \tab Freq =No Data   &    &  \\ \cline{2-2}
& 2-methyl-pentane\_out\_G09\_invertion   & \multicolumn{2}{c}{\multirow{3}{*}
 {RMS = 0.191497}}
\\
& E = No Data \tab Freq =No Data   &    \multicolumn{2}{c}{}  \\ \hline
\multirow{4}{*}{\tab[2mm] 318 \tab[2mm]} & 2,3-dimethyl-butane\_out\_G09\_invertion &
\multirow{3}{*}{Different} & \multirow{3}{*}{Different}
\\
& E = No Data \tab Freq =No Data   &    &  \\ \cline{2-2}
& 3-methyl-pentane\_out\_G09   & \multicolumn{2}{c}{\multirow{3}{*}
 {RMS = 0.496715}}
\\
& E = No Data \tab Freq =No Data   &    \multicolumn{2}{c}{}  \\ \hline
\multirow{4}{*}{\tab[2mm] 319 \tab[2mm]} & 2,3-dimethyl-butane\_out\_G09\_invertion &
\multirow{3}{*}{Different} & \multirow{3}{*}{Different}
\\
& E = No Data \tab Freq =No Data   &    &  \\ \cline{2-2}
& 3-methyl-pentane\_out\_G09\_invertion   & \multicolumn{2}{c}{\multirow{3}{*}
 {RMS = 0.496716}}
\\
& E = No Data \tab Freq =No Data   &    \multicolumn{2}{c}{}  \\ \hline
\multirow{4}{*}{\tab[2mm] 320 \tab[2mm]} & 2,3-dimethyl-butane\_out\_G09\_invertion &
\multirow{3}{*}{Different} & \multirow{3}{*}{Different}
\\
& E = No Data \tab Freq =No Data   &    &  \\ \cline{2-2}
& hexane\_out\_G09   & \multicolumn{2}{c}{\multirow{3}{*}
 {RMS = 0.269312}}
\\
& E = No Data \tab Freq =No Data   &    \multicolumn{2}{c}{}  \\ \hline
\multirow{4}{*}{\tab[2mm] 321 \tab[2mm]} & 2,3-dimethyl-butane\_out\_G09\_invertion &
\multirow{3}{*}{Different} & \multirow{3}{*}{Different}
\\
& E = No Data \tab Freq =No Data   &    &  \\ \cline{2-2}
& hexane\_out\_G09\_invertion   & \multicolumn{2}{c}{\multirow{3}{*}
 {RMS = 0.269312}}
\\
& E = No Data \tab Freq =No Data   &    \multicolumn{2}{c}{}  \\ \hline
\multirow{4}{*}{\tab[2mm] 322 \tab[2mm]} & 2-methyl-pentane\_out\_G09 &
\multirow{3}{*}{\textcolor{Red}{\bf Without comparation}} & \multirow{3}{*}{\textcolor{Red}{\bf Enantiomers}}
\\
& E = No Data \tab Freq =No Data   &    &  \\ \cline{2-2}
& 2-methyl-pentane\_out\_G09\_invertion   & \multicolumn{2}{c}{\multirow{3}{*}
 {RMS = 0.1203334}}
\\
& E = No Data \tab Freq =No Data   &    \multicolumn{2}{c}{}  \\ \hline
\end{tabular}
\newpage

\vtab[-2cm]
\tab[-2cm]
\begin{tabular}{c|m{8cm}|c|c}
\# & Moléculas & Restultado esperado & Resultado programa \\\\ \hline\hline
\multirow{4}{*}{\tab[2mm] 323 \tab[2mm]} & 2-methyl-pentane\_out\_G09 &
\multirow{3}{*}{Different} & \multirow{3}{*}{Different}
\\
& E = No Data \tab Freq =No Data   &    &  \\ \cline{2-2}
& 3-methyl-pentane\_out\_G09   & \multicolumn{2}{c}{\multirow{3}{*}
 {RMS = 0.80863}}
\\
& E = No Data \tab Freq =No Data   &    \multicolumn{2}{c}{}  \\ \hline
\multirow{4}{*}{\tab[2mm] 324 \tab[2mm]} & 2-methyl-pentane\_out\_G09 &
\multirow{3}{*}{Different} & \multirow{3}{*}{Different}
\\
& E = No Data \tab Freq =No Data   &    &  \\ \cline{2-2}
& 3-methyl-pentane\_out\_G09\_invertion   & \multicolumn{2}{c}{\multirow{3}{*}
 {RMS = 0.808631}}
\\
& E = No Data \tab Freq =No Data   &    \multicolumn{2}{c}{}  \\ \hline
\multirow{4}{*}{\tab[2mm] 325 \tab[2mm]} & 2-methyl-pentane\_out\_G09 &
\multirow{3}{*}{Different} & \multirow{3}{*}{Different}
\\
& E = No Data \tab Freq =No Data   &    &  \\ \cline{2-2}
& hexane\_out\_G09   & \multicolumn{2}{c}{\multirow{3}{*}
 {RMS = 0.0426031}}
\\
& E = No Data \tab Freq =No Data   &    \multicolumn{2}{c}{}  \\ \hline
\multirow{4}{*}{\tab[2mm] 326 \tab[2mm]} & 2-methyl-pentane\_out\_G09 &
\multirow{3}{*}{Different} & \multirow{3}{*}{Different}
\\
& E = No Data \tab Freq =No Data   &    &  \\ \cline{2-2}
& hexane\_out\_G09\_invertion   & \multicolumn{2}{c}{\multirow{3}{*}
 {RMS = 0.0426031}}
\\
& E = No Data \tab Freq =No Data   &    \multicolumn{2}{c}{}  \\ \hline
\multirow{4}{*}{\tab[2mm] 327 \tab[2mm]} & 2-methyl-pentane\_out\_G09\_invertion &
\multirow{3}{*}{Different} & \multirow{3}{*}{Different}
\\
& E = No Data \tab Freq =No Data   &    &  \\ \cline{2-2}
& 3-methyl-pentane\_out\_G09   & \multicolumn{2}{c}{\multirow{3}{*}
 {RMS = 0.688213}}
\\
& E = No Data \tab Freq =No Data   &    \multicolumn{2}{c}{}  \\ \hline
\multirow{4}{*}{\tab[2mm] 328 \tab[2mm]} & 2-methyl-pentane\_out\_G09\_invertion &
\multirow{3}{*}{Different} & \multirow{3}{*}{Different}
\\
& E = No Data \tab Freq =No Data   &    &  \\ \cline{2-2}
& 3-methyl-pentane\_out\_G09\_invertion   & \multicolumn{2}{c}{\multirow{3}{*}
 {RMS = 0.688213}}
\\
& E = No Data \tab Freq =No Data   &    \multicolumn{2}{c}{}  \\ \hline
\multirow{4}{*}{\tab[2mm] 329 \tab[2mm]} & 2-methyl-pentane\_out\_G09\_invertion &
\multirow{3}{*}{Different} & \multirow{3}{*}{Different}
\\
& E = No Data \tab Freq =No Data   &    &  \\ \cline{2-2}
& hexane\_out\_G09   & \multicolumn{2}{c}{\multirow{3}{*}
 {RMS = 0.0778147}}
\\
& E = No Data \tab Freq =No Data   &    \multicolumn{2}{c}{}  \\ \hline
\end{tabular}
\newpage

\vtab[-2cm]
\tab[-2cm]
\begin{tabular}{c|m{8cm}|c|c}
\# & Moléculas & Restultado esperado & Resultado programa \\\\ \hline\hline
\multirow{4}{*}{\tab[2mm] 330 \tab[2mm]} & 2-methyl-pentane\_out\_G09\_invertion &
\multirow{3}{*}{Different} & \multirow{3}{*}{Different}
\\
& E = No Data \tab Freq =No Data   &    &  \\ \cline{2-2}
& hexane\_out\_G09\_invertion   & \multicolumn{2}{c}{\multirow{3}{*}
 {RMS = 0.0778147}}
\\
& E = No Data \tab Freq =No Data   &    \multicolumn{2}{c}{}  \\ \hline
\multirow{4}{*}{\tab[2mm] 331 \tab[2mm]} & 3-methyl-pentane\_out\_G09 &
\multirow{3}{*}{\textcolor{Red}{\bf Without comparation}} & \multirow{3}{*}{\textcolor{Red}{\bf Equal}}
\\
& E = No Data \tab Freq =No Data   &    &  \\ \cline{2-2}
& 3-methyl-pentane\_out\_G09\_invertion   & \multicolumn{2}{c}{\multirow{3}{*}
{\textcolor{Red}{ RMS = 0.02098443}}}
\\
& E = No Data \tab Freq =No Data   &    \multicolumn{2}{c}{}  \\ \hline
\multirow{4}{*}{\tab[2mm] 332 \tab[2mm]} & 3-methyl-pentane\_out\_G09 &
\multirow{3}{*}{Different} & \multirow{3}{*}{Different}
\\
& E = No Data \tab Freq =No Data   &    &  \\ \cline{2-2}
& hexane\_out\_G09   & \multicolumn{2}{c}{\multirow{3}{*}
 {RMS = 0.766027}}
\\
& E = No Data \tab Freq =No Data   &    \multicolumn{2}{c}{}  \\ \hline
\multirow{4}{*}{\tab[2mm] 333 \tab[2mm]} & 3-methyl-pentane\_out\_G09 &
\multirow{3}{*}{Different} & \multirow{3}{*}{Different}
\\
& E = No Data \tab Freq =No Data   &    &  \\ \cline{2-2}
& hexane\_out\_G09\_invertion   & \multicolumn{2}{c}{\multirow{3}{*}
 {RMS = 0.766027}}
\\
& E = No Data \tab Freq =No Data   &    \multicolumn{2}{c}{}  \\ \hline
\multirow{4}{*}{\tab[2mm] 334 \tab[2mm]} & 3-methyl-pentane\_out\_G09\_invertion &
\multirow{3}{*}{Different} & \multirow{3}{*}{Different}
\\
& E = No Data \tab Freq =No Data   &    &  \\ \cline{2-2}
& hexane\_out\_G09   & \multicolumn{2}{c}{\multirow{3}{*}
 {RMS = 0.766028}}
\\
& E = No Data \tab Freq =No Data   &    \multicolumn{2}{c}{}  \\ \hline
\multirow{4}{*}{\tab[2mm] 335 \tab[2mm]} & 3-methyl-pentane\_out\_G09\_invertion &
\multirow{3}{*}{Different} & \multirow{3}{*}{Different}
\\
& E = No Data \tab Freq =No Data   &    &  \\ \cline{2-2}
& hexane\_out\_G09\_invertion   & \multicolumn{2}{c}{\multirow{3}{*}
 {RMS = 0.766028}}
\\
& E = No Data \tab Freq =No Data   &    \multicolumn{2}{c}{}  \\ \hline
\multirow{4}{*}{\tab[2mm] 336 \tab[2mm]} & hexane\_out\_G09 &
\multirow{3}{*}{\textcolor{Red}{\bf Without comparation}} & \multirow{3}{*}{\textcolor{Red}{\bf Equal}}
\\
& E = No Data \tab Freq =No Data   &    &  \\ \cline{2-2}
& hexane\_out\_G09\_invertion   & \multicolumn{2}{c}{\multirow{3}{*}
{ RMS = 0}}
\\
& E = No Data \tab Freq =No Data   &    \multicolumn{2}{c}{}  \\ \hline
\end{tabular}
\newpage

\vtab[-2cm]
\tab[-2cm]
\begin{tabular}{c|m{8cm}|c|c}
\# & Moléculas & Restultado esperado & Resultado programa \\\\ \hline\hline
\multirow{4}{*}{\tab[2mm] 337 \tab[2mm]} & neopentane\_Symmetry\_out\_G09 &
\multirow{3}{*}{\textcolor{Red}{\bf Without comparation}} & \multirow{3}{*}{\textcolor{Red}{\bf Enantiomers}}
\\
& E = No Data \tab Freq =No Data   &    &  \\ \cline{2-2}
& neopentane\_Symmetry\_out\_G09\_invertion   & \multicolumn{2}{c}{\multirow{3}{*}
 {RMS = 0.08661368}}
\\
& E = No Data \tab Freq =No Data   &    \multicolumn{2}{c}{}  \\ \hline
\multirow{4}{*}{\tab[2mm] 338 \tab[2mm]} & neopentane\_Symmetry\_out\_G09 &
\multirow{3}{*}{\textcolor{Red}{\bf Without comparation}} & \multirow{3}{*}{\textcolor{Red}{\bf Enantiomers}}
\\
& E = No Data \tab Freq =No Data   &    &  \\ \cline{2-2}
& neopentane\_Symmetry\_out\_G09\_rot\_x45\_y45\_z60   & \multicolumn{2}{c}{\multirow{3}{*}
 {RMS = 0.002619224}}
\\
& E = No Data \tab Freq =No Data   &    \multicolumn{2}{c}{}  \\ \hline
\multirow{4}{*}{\tab[2mm] 339 \tab[2mm]} & neopentane\_Symmetry\_out\_G09 &
\multirow{3}{*}{\textcolor{Red}{\bf Without comparation}} & \multirow{3}{*}{\textcolor{Red}{\bf Enantiomers}}
\\
& E = No Data \tab Freq =No Data   &    &  \\ \cline{2-2}
& neopentane\_out\_G09   & \multicolumn{2}{c}{\multirow{3}{*}
 {RMS = 0.0005811388}}
\\
& E = No Data \tab Freq =No Data   &    \multicolumn{2}{c}{}  \\ \hline
\multirow{4}{*}{\tab[2mm] 340 \tab[2mm]} & neopentane\_Symmetry\_out\_G09 &
\multirow{3}{*}{\textcolor{Red}{\bf Without comparation}} & \multirow{3}{*}{\textcolor{Red}{\bf Enantiomers}}
\\
& E = No Data \tab Freq =No Data   &    &  \\ \cline{2-2}
& neopentane\_out\_G09\_invertion   & \multicolumn{2}{c}{\multirow{3}{*}
 {RMS = 0.08655257}}
\\
& E = No Data \tab Freq =No Data   &    \multicolumn{2}{c}{}  \\ \hline
\multirow{4}{*}{\tab[2mm] 341 \tab[2mm]} & neopentane\_Symmetry\_out\_G09 &
\multirow{3}{*}{\textcolor{Red}{\bf Without comparation}} & \multirow{3}{*}{\textcolor{Red}{\bf Enantiomers}}
\\
& E = No Data \tab Freq =No Data   &    &  \\ \cline{2-2}
& neopentane\_out\_G09\_rot\_x15-y15-z15   & \multicolumn{2}{c}{\multirow{3}{*}
 {RMS = 0.002335402}}
\\
& E = No Data \tab Freq =No Data   &    \multicolumn{2}{c}{}  \\ \hline
\multirow{4}{*}{\tab[2mm] 342 \tab[2mm]} & neopentane\_Symmetry\_out\_G09 &
\multirow{3}{*}{Different} & \multirow{3}{*}{Different}
\\
& E = No Data \tab Freq =No Data   &    &  \\ \cline{2-2}
& tert-butylamine\_out\_G09   & \multicolumn{2}{c}{\multirow{3}{*}
 {RMS = 0.0603092}}
\\
& E = No Data \tab Freq =No Data   &    \multicolumn{2}{c}{}  \\ \hline
\multirow{4}{*}{\tab[2mm] 343 \tab[2mm]} & neopentane\_Symmetry\_out\_G09 &
\multirow{3}{*}{Different} & \multirow{3}{*}{Different}
\\
& E = No Data \tab Freq =No Data   &    &  \\ \cline{2-2}
& tert-butylamine\_out\_G09\_invertion   & \multicolumn{2}{c}{\multirow{3}{*}
 {RMS = 0.0603092}}
\\
& E = No Data \tab Freq =No Data   &    \multicolumn{2}{c}{}  \\ \hline
\end{tabular}
\newpage

\vtab[-2cm]
\tab[-2cm]
\begin{tabular}{c|m{8cm}|c|c}
\# & Moléculas & Restultado esperado & Resultado programa \\\\ \hline\hline
\multirow{4}{*}{\tab[2mm] 344 \tab[2mm]} & neopentane\_Symmetry\_out\_G09 &
\multirow{3}{*}{\textcolor{Red}{\bf Without comparation}} & \multirow{3}{*}{\textcolor{Red}{\bf Different}}
\\
& E = No Data \tab Freq =No Data   &    &  \\ \cline{2-2}
& tert-butylamine\_out\_G09\_rot\_x15\_y15\_z15   & \multicolumn{2}{c}{\multirow{3}{*}
 {RMS = 0.0603092}}
\\
& E = No Data \tab Freq =No Data   &    \multicolumn{2}{c}{}  \\ \hline
\multirow{4}{*}{\tab[2mm] 345 \tab[2mm]} & neopentane\_Symmetry\_out\_G09 &
\multirow{3}{*}{Different} & \multirow{3}{*}{Different}
\\
& E = No Data \tab Freq =No Data   &    &  \\ \cline{2-2}
& tetramethylsilane\_out\_G09   & \multicolumn{2}{c}{\multirow{3}{*}
 {RMS = 0.0587334}}
\\
& E = No Data \tab Freq =No Data   &    \multicolumn{2}{c}{}  \\ \hline
\multirow{4}{*}{\tab[2mm] 346 \tab[2mm]} & neopentane\_Symmetry\_out\_G09 &
\multirow{3}{*}{Different} & \multirow{3}{*}{Different}
\\
& E = No Data \tab Freq =No Data   &    &  \\ \cline{2-2}
& tetramethylsilane\_out\_G09\_invertion   & \multicolumn{2}{c}{\multirow{3}{*}
 {RMS = 0.0587334}}
\\
& E = No Data \tab Freq =No Data   &    \multicolumn{2}{c}{}  \\ \hline
\multirow{4}{*}{\tab[2mm] 347 \tab[2mm]} & neopentane\_Symmetry\_out\_G09\_invertion &
\multirow{3}{*}{\textcolor{Red}{\bf Without comparation}} & \multirow{3}{*}{\textcolor{Red}{\bf Equal}}
\\
& E = No Data \tab Freq =No Data   &    &  \\ \cline{2-2}
& neopentane\_Symmetry\_out\_G09\_rot\_x45\_y45\_z60   & \multicolumn{2}{c}{\multirow{3}{*}
{\textcolor{Red}{ RMS = 0.08923291}}}
\\
& E = No Data \tab Freq =No Data   &    \multicolumn{2}{c}{}  \\ \hline
\multirow{4}{*}{\tab[2mm] 348 \tab[2mm]} & neopentane\_Symmetry\_out\_G09\_invertion &
\multirow{3}{*}{\textcolor{Red}{\bf Without comparation}} & \multirow{3}{*}{\textcolor{Red}{\bf Enantiomers}}
\\
& E = No Data \tab Freq =No Data   &    &  \\ \cline{2-2}
& neopentane\_out\_G09   & \multicolumn{2}{c}{\multirow{3}{*}
 {RMS = 0.08719482}}
\\
& E = No Data \tab Freq =No Data   &    \multicolumn{2}{c}{}  \\ \hline
\multirow{4}{*}{\tab[2mm] 349 \tab[2mm]} & neopentane\_Symmetry\_out\_G09\_invertion &
\multirow{3}{*}{\textcolor{Red}{\bf Without comparation}} & \multirow{3}{*}{\textcolor{Red}{\bf Enantiomers}}
\\
& E = No Data \tab Freq =No Data   &    &  \\ \cline{2-2}
& neopentane\_out\_G09\_invertion   & \multicolumn{2}{c}{\multirow{3}{*}
 {RMS = 6.111066E-05}}
\\
& E = No Data \tab Freq =No Data   &    \multicolumn{2}{c}{}  \\ \hline
\multirow{4}{*}{\tab[2mm] 350 \tab[2mm]} & neopentane\_Symmetry\_out\_G09\_invertion &
\multirow{3}{*}{\textcolor{Red}{\bf Without comparation}} & \multirow{3}{*}{\textcolor{Red}{\bf Enantiomers}}
\\
& E = No Data \tab Freq =No Data   &    &  \\ \cline{2-2}
& neopentane\_out\_G09\_rot\_x15-y15-z15   & \multicolumn{2}{c}{\multirow{3}{*}
 {RMS = 0.08894908}}
\\
& E = No Data \tab Freq =No Data   &    \multicolumn{2}{c}{}  \\ \hline
\end{tabular}
\newpage

\vtab[-2cm]
\tab[-2cm]
\begin{tabular}{c|m{8cm}|c|c}
\# & Moléculas & Restultado esperado & Resultado programa \\\\ \hline\hline
\multirow{4}{*}{\tab[2mm] 351 \tab[2mm]} & neopentane\_Symmetry\_out\_G09\_invertion &
\multirow{3}{*}{Different} & \multirow{3}{*}{Different}
\\
& E = No Data \tab Freq =No Data   &    &  \\ \cline{2-2}
& tert-butylamine\_out\_G09   & \multicolumn{2}{c}{\multirow{3}{*}
 {RMS = 0.0319051}}
\\
& E = No Data \tab Freq =No Data   &    \multicolumn{2}{c}{}  \\ \hline
\multirow{4}{*}{\tab[2mm] 352 \tab[2mm]} & neopentane\_Symmetry\_out\_G09\_invertion &
\multirow{3}{*}{Different} & \multirow{3}{*}{Different}
\\
& E = No Data \tab Freq =No Data   &    &  \\ \cline{2-2}
& tert-butylamine\_out\_G09\_invertion   & \multicolumn{2}{c}{\multirow{3}{*}
 {RMS = 0.0319051}}
\\
& E = No Data \tab Freq =No Data   &    \multicolumn{2}{c}{}  \\ \hline
\multirow{4}{*}{\tab[2mm] 353 \tab[2mm]} & neopentane\_Symmetry\_out\_G09\_invertion &
\multirow{3}{*}{\textcolor{Red}{\bf Without comparation}} & \multirow{3}{*}{\textcolor{Red}{\bf Different}}
\\
& E = No Data \tab Freq =No Data   &    &  \\ \cline{2-2}
& tert-butylamine\_out\_G09\_rot\_x15\_y15\_z15   & \multicolumn{2}{c}{\multirow{3}{*}
 {RMS = 0.0319051}}
\\
& E = No Data \tab Freq =No Data   &    \multicolumn{2}{c}{}  \\ \hline
\multirow{4}{*}{\tab[2mm] 354 \tab[2mm]} & neopentane\_Symmetry\_out\_G09\_invertion &
\multirow{3}{*}{Different} & \multirow{3}{*}{Different}
\\
& E = No Data \tab Freq =No Data   &    &  \\ \cline{2-2}
& tetramethylsilane\_out\_G09   & \multicolumn{2}{c}{\multirow{3}{*}
 {RMS = 0.150948}}
\\
& E = No Data \tab Freq =No Data   &    \multicolumn{2}{c}{}  \\ \hline
\multirow{4}{*}{\tab[2mm] 355 \tab[2mm]} & neopentane\_Symmetry\_out\_G09\_invertion &
\multirow{3}{*}{Different} & \multirow{3}{*}{Different}
\\
& E = No Data \tab Freq =No Data   &    &  \\ \cline{2-2}
& tetramethylsilane\_out\_G09\_invertion   & \multicolumn{2}{c}{\multirow{3}{*}
 {RMS = 0.150948}}
\\
& E = No Data \tab Freq =No Data   &    \multicolumn{2}{c}{}  \\ \hline
\multirow{4}{*}{\tab[2mm] 356 \tab[2mm]} & neopentane\_Symmetry\_out\_G09\_rot\_x45\_y45\_z60 &
\multirow{3}{*}{\textcolor{Red}{\bf Without comparation}} & \multirow{3}{*}{\textcolor{Red}{\bf Enantiomers}}
\\
& E = No Data \tab Freq =No Data   &    &  \\ \cline{2-2}
& neopentane\_out\_G09   & \multicolumn{2}{c}{\multirow{3}{*}
 {RMS = 0.002038085}}
\\
& E = No Data \tab Freq =No Data   &    \multicolumn{2}{c}{}  \\ \hline
\multirow{4}{*}{\tab[2mm] 357 \tab[2mm]} & neopentane\_Symmetry\_out\_G09\_rot\_x45\_y45\_z60 &
\multirow{3}{*}{\textcolor{Red}{\bf Without comparation}} & \multirow{3}{*}{\textcolor{Red}{\bf Enantiomers}}
\\
& E = No Data \tab Freq =No Data   &    &  \\ \cline{2-2}
& neopentane\_out\_G09\_invertion   & \multicolumn{2}{c}{\multirow{3}{*}
 {RMS = 0.0891718}}
\\
& E = No Data \tab Freq =No Data   &    \multicolumn{2}{c}{}  \\ \hline
\end{tabular}
\newpage

\vtab[-2cm]
\tab[-2cm]
\begin{tabular}{c|m{8cm}|c|c}
\# & Moléculas & Restultado esperado & Resultado programa \\\\ \hline\hline
\multirow{4}{*}{\tab[2mm] 358 \tab[2mm]} & neopentane\_Symmetry\_out\_G09\_rot\_x45\_y45\_z60 &
\multirow{3}{*}{\textcolor{Red}{\bf Without comparation}} & \multirow{3}{*}{\textcolor{Red}{\bf Enantiomers}}
\\
& E = No Data \tab Freq =No Data   &    &  \\ \cline{2-2}
& neopentane\_out\_G09\_rot\_x15-y15-z15   & \multicolumn{2}{c}{\multirow{3}{*}
 {RMS = 0.000283822}}
\\
& E = No Data \tab Freq =No Data   &    \multicolumn{2}{c}{}  \\ \hline
\multirow{4}{*}{\tab[2mm] 359 \tab[2mm]} & neopentane\_Symmetry\_out\_G09\_rot\_x45\_y45\_z60 &
\multirow{3}{*}{\textcolor{Red}{\bf Without comparation}} & \multirow{3}{*}{\textcolor{Red}{\bf Different}}
\\
& E = No Data \tab Freq =No Data   &    &  \\ \cline{2-2}
& tert-butylamine\_out\_G09   & \multicolumn{2}{c}{\multirow{3}{*}
 {RMS = 0.0533989}}
\\
& E = No Data \tab Freq =No Data   &    \multicolumn{2}{c}{}  \\ \hline
\multirow{4}{*}{\tab[2mm] 360 \tab[2mm]} & neopentane\_Symmetry\_out\_G09\_rot\_x45\_y45\_z60 &
\multirow{3}{*}{\textcolor{Red}{\bf Without comparation}} & \multirow{3}{*}{\textcolor{Red}{\bf Different}}
\\
& E = No Data \tab Freq =No Data   &    &  \\ \cline{2-2}
& tert-butylamine\_out\_G09\_invertion   & \multicolumn{2}{c}{\multirow{3}{*}
 {RMS = 0.0533989}}
\\
& E = No Data \tab Freq =No Data   &    \multicolumn{2}{c}{}  \\ \hline
\multirow{4}{*}{\tab[2mm] 361 \tab[2mm]} & neopentane\_Symmetry\_out\_G09\_rot\_x45\_y45\_z60 &
\multirow{3}{*}{\textcolor{Red}{\bf Without comparation}} & \multirow{3}{*}{\textcolor{Red}{\bf Different}}
\\
& E = No Data \tab Freq =No Data   &    &  \\ \cline{2-2}
& tert-butylamine\_out\_G09\_rot\_x15\_y15\_z15   & \multicolumn{2}{c}{\multirow{3}{*}
 {RMS = 0.0533989}}
\\
& E = No Data \tab Freq =No Data   &    \multicolumn{2}{c}{}  \\ \hline
\multirow{4}{*}{\tab[2mm] 362 \tab[2mm]} & neopentane\_Symmetry\_out\_G09\_rot\_x45\_y45\_z60 &
\multirow{3}{*}{\textcolor{Red}{\bf Without comparation}} & \multirow{3}{*}{\textcolor{Red}{\bf Different}}
\\
& E = No Data \tab Freq =No Data   &    &  \\ \cline{2-2}
& tetramethylsilane\_out\_G09   & \multicolumn{2}{c}{\multirow{3}{*}
 {RMS = 0.172441}}
\\
& E = No Data \tab Freq =No Data   &    \multicolumn{2}{c}{}  \\ \hline
\multirow{4}{*}{\tab[2mm] 363 \tab[2mm]} & neopentane\_Symmetry\_out\_G09\_rot\_x45\_y45\_z60 &
\multirow{3}{*}{\textcolor{Red}{\bf Without comparation}} & \multirow{3}{*}{\textcolor{Red}{\bf Different}}
\\
& E = No Data \tab Freq =No Data   &    &  \\ \cline{2-2}
& tetramethylsilane\_out\_G09\_invertion   & \multicolumn{2}{c}{\multirow{3}{*}
 {RMS = 0.172441}}
\\
& E = No Data \tab Freq =No Data   &    \multicolumn{2}{c}{}  \\ \hline
\multirow{4}{*}{\tab[2mm] 364 \tab[2mm]} & neopentane\_out\_G09 &
\multirow{3}{*}{\textcolor{Red}{\bf Without comparation}} & \multirow{3}{*}{\textcolor{Red}{\bf Enantiomers}}
\\
& E = No Data \tab Freq =No Data   &    &  \\ \cline{2-2}
& neopentane\_out\_G09\_invertion   & \multicolumn{2}{c}{\multirow{3}{*}
 {RMS = 0.08713371}}
\\
& E = No Data \tab Freq =No Data   &    \multicolumn{2}{c}{}  \\ \hline
\end{tabular}
\newpage

\vtab[-2cm]
\tab[-2cm]
\begin{tabular}{c|m{8cm}|c|c}
\# & Moléculas & Restultado esperado & Resultado programa \\\\ \hline\hline
\multirow{4}{*}{\tab[2mm] 365 \tab[2mm]} & neopentane\_out\_G09 &
\multirow{3}{*}{\textcolor{Red}{\bf Without comparation}} & \multirow{3}{*}{\textcolor{Red}{\bf Enantiomers}}
\\
& E = No Data \tab Freq =No Data   &    &  \\ \cline{2-2}
& neopentane\_out\_G09\_rot\_x15-y15-z15   & \multicolumn{2}{c}{\multirow{3}{*}
 {RMS = 0.001754263}}
\\
& E = No Data \tab Freq =No Data   &    \multicolumn{2}{c}{}  \\ \hline
\multirow{4}{*}{\tab[2mm] 366 \tab[2mm]} & neopentane\_out\_G09 &
\multirow{3}{*}{Different} & \multirow{3}{*}{Different}
\\
& E = No Data \tab Freq =No Data   &    &  \\ \cline{2-2}
& tert-butylamine\_out\_G09   & \multicolumn{2}{c}{\multirow{3}{*}
 {RMS = 0.0603026}}
\\
& E = No Data \tab Freq =No Data   &    \multicolumn{2}{c}{}  \\ \hline
\multirow{4}{*}{\tab[2mm] 367 \tab[2mm]} & neopentane\_out\_G09 &
\multirow{3}{*}{Different} & \multirow{3}{*}{Different}
\\
& E = No Data \tab Freq =No Data   &    &  \\ \cline{2-2}
& tert-butylamine\_out\_G09\_invertion   & \multicolumn{2}{c}{\multirow{3}{*}
 {RMS = 0.0603026}}
\\
& E = No Data \tab Freq =No Data   &    \multicolumn{2}{c}{}  \\ \hline
\multirow{4}{*}{\tab[2mm] 368 \tab[2mm]} & neopentane\_out\_G09 &
\multirow{3}{*}{\textcolor{Red}{\bf Without comparation}} & \multirow{3}{*}{\textcolor{Red}{\bf Different}}
\\
& E = No Data \tab Freq =No Data   &    &  \\ \cline{2-2}
& tert-butylamine\_out\_G09\_rot\_x15\_y15\_z15   & \multicolumn{2}{c}{\multirow{3}{*}
 {RMS = 0.0603026}}
\\
& E = No Data \tab Freq =No Data   &    \multicolumn{2}{c}{}  \\ \hline
\multirow{4}{*}{\tab[2mm] 369 \tab[2mm]} & neopentane\_out\_G09 &
\multirow{3}{*}{Different} & \multirow{3}{*}{Different}
\\
& E = No Data \tab Freq =No Data   &    &  \\ \cline{2-2}
& tetramethylsilane\_out\_G09   & \multicolumn{2}{c}{\multirow{3}{*}
 {RMS = 0.0587399}}
\\
& E = No Data \tab Freq =No Data   &    \multicolumn{2}{c}{}  \\ \hline
\multirow{4}{*}{\tab[2mm] 370 \tab[2mm]} & neopentane\_out\_G09 &
\multirow{3}{*}{Different} & \multirow{3}{*}{Different}
\\
& E = No Data \tab Freq =No Data   &    &  \\ \cline{2-2}
& tetramethylsilane\_out\_G09\_invertion   & \multicolumn{2}{c}{\multirow{3}{*}
 {RMS = 0.0587399}}
\\
& E = No Data \tab Freq =No Data   &    \multicolumn{2}{c}{}  \\ \hline
\multirow{4}{*}{\tab[2mm] 371 \tab[2mm]} & neopentane\_out\_G09\_invertion &
\multirow{3}{*}{\textcolor{Red}{\bf Without comparation}} & \multirow{3}{*}{\textcolor{Red}{\bf Enantiomers}}
\\
& E = No Data \tab Freq =No Data   &    &  \\ \cline{2-2}
& neopentane\_out\_G09\_rot\_x15-y15-z15   & \multicolumn{2}{c}{\multirow{3}{*}
 {RMS = 0.08888797}}
\\
& E = No Data \tab Freq =No Data   &    \multicolumn{2}{c}{}  \\ \hline
\end{tabular}
\newpage

\vtab[-2cm]
\tab[-2cm]
\begin{tabular}{c|m{8cm}|c|c}
\# & Moléculas & Restultado esperado & Resultado programa \\\\ \hline\hline
\multirow{4}{*}{\tab[2mm] 372 \tab[2mm]} & neopentane\_out\_G09\_invertion &
\multirow{3}{*}{Different} & \multirow{3}{*}{Different}
\\
& E = No Data \tab Freq =No Data   &    &  \\ \cline{2-2}
& tert-butylamine\_out\_G09   & \multicolumn{2}{c}{\multirow{3}{*}
 {RMS = 0.0319157}}
\\
& E = No Data \tab Freq =No Data   &    \multicolumn{2}{c}{}  \\ \hline
\multirow{4}{*}{\tab[2mm] 373 \tab[2mm]} & neopentane\_out\_G09\_invertion &
\multirow{3}{*}{Different} & \multirow{3}{*}{Different}
\\
& E = No Data \tab Freq =No Data   &    &  \\ \cline{2-2}
& tert-butylamine\_out\_G09\_invertion   & \multicolumn{2}{c}{\multirow{3}{*}
 {RMS = 0.0319157}}
\\
& E = No Data \tab Freq =No Data   &    \multicolumn{2}{c}{}  \\ \hline
\multirow{4}{*}{\tab[2mm] 374 \tab[2mm]} & neopentane\_out\_G09\_invertion &
\multirow{3}{*}{\textcolor{Red}{\bf Without comparation}} & \multirow{3}{*}{\textcolor{Red}{\bf Different}}
\\
& E = No Data \tab Freq =No Data   &    &  \\ \cline{2-2}
& tert-butylamine\_out\_G09\_rot\_x15\_y15\_z15   & \multicolumn{2}{c}{\multirow{3}{*}
 {RMS = 0.0319157}}
\\
& E = No Data \tab Freq =No Data   &    \multicolumn{2}{c}{}  \\ \hline
\multirow{4}{*}{\tab[2mm] 375 \tab[2mm]} & neopentane\_out\_G09\_invertion &
\multirow{3}{*}{Different} & \multirow{3}{*}{Different}
\\
& E = No Data \tab Freq =No Data   &    &  \\ \cline{2-2}
& tetramethylsilane\_out\_G09   & \multicolumn{2}{c}{\multirow{3}{*}
 {RMS = 0.150958}}
\\
& E = No Data \tab Freq =No Data   &    \multicolumn{2}{c}{}  \\ \hline
\multirow{4}{*}{\tab[2mm] 376 \tab[2mm]} & neopentane\_out\_G09\_invertion &
\multirow{3}{*}{Different} & \multirow{3}{*}{Different}
\\
& E = No Data \tab Freq =No Data   &    &  \\ \cline{2-2}
& tetramethylsilane\_out\_G09\_invertion   & \multicolumn{2}{c}{\multirow{3}{*}
 {RMS = 0.150958}}
\\
& E = No Data \tab Freq =No Data   &    \multicolumn{2}{c}{}  \\ \hline
\multirow{4}{*}{\tab[2mm] 377 \tab[2mm]} & neopentane\_out\_G09\_rot\_x15-y15-z15 &
\multirow{3}{*}{\textcolor{Red}{\bf Without comparation}} & \multirow{3}{*}{\textcolor{Red}{\bf Different}}
\\
& E = No Data \tab Freq =No Data   &    &  \\ \cline{2-2}
& tert-butylamine\_out\_G09   & \multicolumn{2}{c}{\multirow{3}{*}
 {RMS = 0.0670111}}
\\
& E = No Data \tab Freq =No Data   &    \multicolumn{2}{c}{}  \\ \hline
\multirow{4}{*}{\tab[2mm] 378 \tab[2mm]} & neopentane\_out\_G09\_rot\_x15-y15-z15 &
\multirow{3}{*}{\textcolor{Red}{\bf Without comparation}} & \multirow{3}{*}{\textcolor{Red}{\bf Different}}
\\
& E = No Data \tab Freq =No Data   &    &  \\ \cline{2-2}
& tert-butylamine\_out\_G09\_invertion   & \multicolumn{2}{c}{\multirow{3}{*}
 {RMS = 0.0670111}}
\\
& E = No Data \tab Freq =No Data   &    \multicolumn{2}{c}{}  \\ \hline
\end{tabular}
\newpage

\vtab[-2cm]
\tab[-2cm]
\begin{tabular}{c|m{8cm}|c|c}
\# & Moléculas & Restultado esperado & Resultado programa \\\\ \hline\hline
\multirow{4}{*}{\tab[2mm] 379 \tab[2mm]} & neopentane\_out\_G09\_rot\_x15-y15-z15 &
\multirow{3}{*}{\textcolor{Red}{\bf Without comparation}} & \multirow{3}{*}{\textcolor{Red}{\bf Different}}
\\
& E = No Data \tab Freq =No Data   &    &  \\ \cline{2-2}
& tert-butylamine\_out\_G09\_rot\_x15\_y15\_z15   & \multicolumn{2}{c}{\multirow{3}{*}
 {RMS = 0.0670111}}
\\
& E = No Data \tab Freq =No Data   &    \multicolumn{2}{c}{}  \\ \hline
\multirow{4}{*}{\tab[2mm] 380 \tab[2mm]} & neopentane\_out\_G09\_rot\_x15-y15-z15 &
\multirow{3}{*}{\textcolor{Red}{\bf Without comparation}} & \multirow{3}{*}{\textcolor{Red}{\bf Different}}
\\
& E = No Data \tab Freq =No Data   &    &  \\ \cline{2-2}
& tetramethylsilane\_out\_G09   & \multicolumn{2}{c}{\multirow{3}{*}
 {RMS = 0.0520314}}
\\
& E = No Data \tab Freq =No Data   &    \multicolumn{2}{c}{}  \\ \hline
\multirow{4}{*}{\tab[2mm] 381 \tab[2mm]} & neopentane\_out\_G09\_rot\_x15-y15-z15 &
\multirow{3}{*}{\textcolor{Red}{\bf Without comparation}} & \multirow{3}{*}{\textcolor{Red}{\bf Different}}
\\
& E = No Data \tab Freq =No Data   &    &  \\ \cline{2-2}
& tetramethylsilane\_out\_G09\_invertion   & \multicolumn{2}{c}{\multirow{3}{*}
 {RMS = 0.0520314}}
\\
& E = No Data \tab Freq =No Data   &    \multicolumn{2}{c}{}  \\ \hline
\multirow{4}{*}{\tab[2mm] 382 \tab[2mm]} & tert-butylamine\_out\_G09 &
\multirow{3}{*}{\textcolor{Red}{\bf Without comparation}} & \multirow{3}{*}{\textcolor{Red}{\bf Enantiomers}}
\\
& E = No Data \tab Freq =No Data   &    &  \\ \cline{2-2}
& tert-butylamine\_out\_G09\_invertion   & \multicolumn{2}{c}{\multirow{3}{*}
 {RMS = 0.09408316}}
\\
& E = No Data \tab Freq =No Data   &    \multicolumn{2}{c}{}  \\ \hline
\multirow{4}{*}{\tab[2mm] 383 \tab[2mm]} & tert-butylamine\_out\_G09 &
\multirow{3}{*}{\textcolor{Red}{\bf Without comparation}} & \multirow{3}{*}{\textcolor{Red}{\bf Equal}}
\\
& E = No Data \tab Freq =No Data   &    &  \\ \cline{2-2}
& tert-butylamine\_out\_G09\_rot\_x15\_y15\_z15   & \multicolumn{2}{c}{\multirow{3}{*}
{ RMS = 1.188653E-12}}
\\
& E = No Data \tab Freq =No Data   &    \multicolumn{2}{c}{}  \\ \hline
\multirow{4}{*}{\tab[2mm] 384 \tab[2mm]} & tert-butylamine\_out\_G09 &
\multirow{3}{*}{Different} & \multirow{3}{*}{Different}
\\
& E = No Data \tab Freq =No Data   &    &  \\ \cline{2-2}
& tetramethylsilane\_out\_G09   & \multicolumn{2}{c}{\multirow{3}{*}
 {RMS = 0.415062}}
\\
& E = No Data \tab Freq =No Data   &    \multicolumn{2}{c}{}  \\ \hline
\multirow{4}{*}{\tab[2mm] 385 \tab[2mm]} & tert-butylamine\_out\_G09 &
\multirow{3}{*}{Different} & \multirow{3}{*}{Different}
\\
& E = No Data \tab Freq =No Data   &    &  \\ \cline{2-2}
& tetramethylsilane\_out\_G09\_invertion   & \multicolumn{2}{c}{\multirow{3}{*}
 {RMS = 0.415063}}
\\
& E = No Data \tab Freq =No Data   &    \multicolumn{2}{c}{}  \\ \hline
\end{tabular}
\newpage

\vtab[-2cm]
\tab[-2cm]
\begin{tabular}{c|m{8cm}|c|c}
\# & Moléculas & Restultado esperado & Resultado programa \\\\ \hline\hline
\multirow{4}{*}{\tab[2mm] 386 \tab[2mm]} & tert-butylamine\_out\_G09\_invertion &
\multirow{3}{*}{\textcolor{Red}{\bf Without comparation}} & \multirow{3}{*}{\textcolor{Red}{\bf Enantiomers}}
\\
& E = No Data \tab Freq =No Data   &    &  \\ \cline{2-2}
& tert-butylamine\_out\_G09\_rot\_x15\_y15\_z15   & \multicolumn{2}{c}{\multirow{3}{*}
 {RMS = 0.09408316}}
\\
& E = No Data \tab Freq =No Data   &    \multicolumn{2}{c}{}  \\ \hline
\multirow{4}{*}{\tab[2mm] 387 \tab[2mm]} & tert-butylamine\_out\_G09\_invertion &
\multirow{3}{*}{Different} & \multirow{3}{*}{Different}
\\
& E = No Data \tab Freq =No Data   &    &  \\ \cline{2-2}
& tetramethylsilane\_out\_G09   & \multicolumn{2}{c}{\multirow{3}{*}
 {RMS = 0.456096}}
\\
& E = No Data \tab Freq =No Data   &    \multicolumn{2}{c}{}  \\ \hline
\multirow{4}{*}{\tab[2mm] 388 \tab[2mm]} & tert-butylamine\_out\_G09\_invertion &
\multirow{3}{*}{Different} & \multirow{3}{*}{Different}
\\
& E = No Data \tab Freq =No Data   &    &  \\ \cline{2-2}
& tetramethylsilane\_out\_G09\_invertion   & \multicolumn{2}{c}{\multirow{3}{*}
 {RMS = 0.456097}}
\\
& E = No Data \tab Freq =No Data   &    \multicolumn{2}{c}{}  \\ \hline
\multirow{4}{*}{\tab[2mm] 389 \tab[2mm]} & tert-butylamine\_out\_G09\_rot\_x15\_y15\_z15 &
\multirow{3}{*}{\textcolor{Red}{\bf Without comparation}} & \multirow{3}{*}{\textcolor{Red}{\bf Different}}
\\
& E = No Data \tab Freq =No Data   &    &  \\ \cline{2-2}
& tetramethylsilane\_out\_G09   & \multicolumn{2}{c}{\multirow{3}{*}
 {RMS = 0.291987}}
\\
& E = No Data \tab Freq =No Data   &    \multicolumn{2}{c}{}  \\ \hline
\multirow{4}{*}{\tab[2mm] 390 \tab[2mm]} & tert-butylamine\_out\_G09\_rot\_x15\_y15\_z15 &
\multirow{3}{*}{\textcolor{Red}{\bf Without comparation}} & \multirow{3}{*}{\textcolor{Red}{\bf Different}}
\\
& E = No Data \tab Freq =No Data   &    &  \\ \cline{2-2}
& tetramethylsilane\_out\_G09\_invertion   & \multicolumn{2}{c}{\multirow{3}{*}
 {RMS = 0.291987}}
\\
& E = No Data \tab Freq =No Data   &    \multicolumn{2}{c}{}  \\ \hline
\multirow{4}{*}{\tab[2mm] 391 \tab[2mm]} & tetramethylsilane\_out\_G09 &
\multirow{3}{*}{\textcolor{Red}{\bf Without comparation}} & \multirow{3}{*}{\textcolor{Red}{\bf Equal}}
\\
& E = No Data \tab Freq =No Data   &    &  \\ \cline{2-2}
& tetramethylsilane\_out\_G09\_invertion   & \multicolumn{2}{c}{\multirow{3}{*}
{ RMS = 0}}
\\
& E = No Data \tab Freq =No Data   &    \multicolumn{2}{c}{}  \\ \hline
\multirow{4}{*}{\tab[2mm] 392 \tab[2mm]} & Porphin\_out\_G09 &
\multirow{3}{*}{\textcolor{Red}{\bf Without comparation}} & \multirow{3}{*}{\textcolor{Red}{\bf Enantiomers}}
\\
& E = No Data \tab Freq =No Data   &    &  \\ \cline{2-2}
& Porphin\_out\_G09\_invertion   & \multicolumn{2}{c}{\multirow{3}{*}
 {RMS = 0}}
\\
& E = No Data \tab Freq =No Data   &    \multicolumn{2}{c}{}  \\ \hline
\end{tabular}
\newpage

\vtab[-2cm]
\tab[-2cm]
\begin{tabular}{c|m{8cm}|c|c}
\# & Moléculas & Restultado esperado & Resultado programa \\\\ \hline\hline
\multirow{4}{*}{\tab[2mm] 393 \tab[2mm]} & Porphin\_out\_G09 &
\multirow{3}{*}{Different} & \multirow{3}{*}{Different}
\\
& E = No Data \tab Freq =No Data   &    &  \\ \cline{2-2}
& Tetrabenzoporphyrin\_out\_G09   & \multicolumn{2}{c}{\multirow{3}{*}
 {RMS = 0.117363}}
\\
& E = No Data \tab Freq =No Data   &    \multicolumn{2}{c}{}  \\ \hline
\multirow{4}{*}{\tab[2mm] 394 \tab[2mm]} & Porphin\_out\_G09 &
\multirow{3}{*}{Different} & \multirow{3}{*}{Different}
\\
& E = No Data \tab Freq =No Data   &    &  \\ \cline{2-2}
& Tetrabenzoporphyrin\_out\_G09\_invertion   & \multicolumn{2}{c}{\multirow{3}{*}
 {RMS = 0.117363}}
\\
& E = No Data \tab Freq =No Data   &    \multicolumn{2}{c}{}  \\ \hline
\multirow{4}{*}{\tab[2mm] 395 \tab[2mm]} & Porphin\_out\_G09 &
\multirow{3}{*}{Different} & \multirow{3}{*}{Different}
\\
& E = No Data \tab Freq =No Data   &    &  \\ \cline{2-2}
& Tetrabenzoporphyrin\_rotated02\_out\_G09   & \multicolumn{2}{c}{\multirow{3}{*}
 {RMS = 0.167859}}
\\
& E = No Data \tab Freq =No Data   &    \multicolumn{2}{c}{}  \\ \hline
\multirow{4}{*}{\tab[2mm] 396 \tab[2mm]} & Porphin\_out\_G09 &
\multirow{3}{*}{Different} & \multirow{3}{*}{Different}
\\
& E = No Data \tab Freq =No Data   &    &  \\ \cline{2-2}
& Tetrabenzoporphyrin\_rotated02\_out\_G09\_invertion   & \multicolumn{2}{c}{\multirow{3}{*}
 {RMS = 0.167858}}
\\
& E = No Data \tab Freq =No Data   &    \multicolumn{2}{c}{}  \\ \hline
\multirow{4}{*}{\tab[2mm] 397 \tab[2mm]} & Porphin\_out\_G09 &
\multirow{3}{*}{Different} & \multirow{3}{*}{Different}
\\
& E = No Data \tab Freq =No Data   &    &  \\ \cline{2-2}
& Tetrabenzoporphyrin\_rotated\_out\_G09   & \multicolumn{2}{c}{\multirow{3}{*}
 {RMS = 0.117363}}
\\
& E = No Data \tab Freq =No Data   &    \multicolumn{2}{c}{}  \\ \hline
\multirow{4}{*}{\tab[2mm] 398 \tab[2mm]} & Porphin\_out\_G09 &
\multirow{3}{*}{Different} & \multirow{3}{*}{Different}
\\
& E = No Data \tab Freq =No Data   &    &  \\ \cline{2-2}
& Tetrabenzoporphyrin\_rotated\_out\_G09\_invertion   & \multicolumn{2}{c}{\multirow{3}{*}
 {RMS = 0.117363}}
\\
& E = No Data \tab Freq =No Data   &    \multicolumn{2}{c}{}  \\ \hline
\multirow{4}{*}{\tab[2mm] 399 \tab[2mm]} & Porphin\_out\_G09\_invertion &
\multirow{3}{*}{Different} & \multirow{3}{*}{Different}
\\
& E = No Data \tab Freq =No Data   &    &  \\ \cline{2-2}
& Tetrabenzoporphyrin\_out\_G09   & \multicolumn{2}{c}{\multirow{3}{*}
 {RMS = 0.117363}}
\\
& E = No Data \tab Freq =No Data   &    \multicolumn{2}{c}{}  \\ \hline
\end{tabular}
\newpage

\vtab[-2cm]
\tab[-2cm]
\begin{tabular}{c|m{8cm}|c|c}
\# & Moléculas & Restultado esperado & Resultado programa \\\\ \hline\hline
\multirow{4}{*}{\tab[2mm] 400 \tab[2mm]} & Porphin\_out\_G09\_invertion &
\multirow{3}{*}{Different} & \multirow{3}{*}{Different}
\\
& E = No Data \tab Freq =No Data   &    &  \\ \cline{2-2}
& Tetrabenzoporphyrin\_out\_G09\_invertion   & \multicolumn{2}{c}{\multirow{3}{*}
 {RMS = 0.117363}}
\\
& E = No Data \tab Freq =No Data   &    \multicolumn{2}{c}{}  \\ \hline
\multirow{4}{*}{\tab[2mm] 401 \tab[2mm]} & Porphin\_out\_G09\_invertion &
\multirow{3}{*}{Different} & \multirow{3}{*}{Different}
\\
& E = No Data \tab Freq =No Data   &    &  \\ \cline{2-2}
& Tetrabenzoporphyrin\_rotated02\_out\_G09   & \multicolumn{2}{c}{\multirow{3}{*}
 {RMS = 0.167859}}
\\
& E = No Data \tab Freq =No Data   &    \multicolumn{2}{c}{}  \\ \hline
\multirow{4}{*}{\tab[2mm] 402 \tab[2mm]} & Porphin\_out\_G09\_invertion &
\multirow{3}{*}{Different} & \multirow{3}{*}{Different}
\\
& E = No Data \tab Freq =No Data   &    &  \\ \cline{2-2}
& Tetrabenzoporphyrin\_rotated02\_out\_G09\_invertion   & \multicolumn{2}{c}{\multirow{3}{*}
 {RMS = 0.167858}}
\\
& E = No Data \tab Freq =No Data   &    \multicolumn{2}{c}{}  \\ \hline
\multirow{4}{*}{\tab[2mm] 403 \tab[2mm]} & Porphin\_out\_G09\_invertion &
\multirow{3}{*}{Different} & \multirow{3}{*}{Different}
\\
& E = No Data \tab Freq =No Data   &    &  \\ \cline{2-2}
& Tetrabenzoporphyrin\_rotated\_out\_G09   & \multicolumn{2}{c}{\multirow{3}{*}
 {RMS = 0.117363}}
\\
& E = No Data \tab Freq =No Data   &    \multicolumn{2}{c}{}  \\ \hline
\multirow{4}{*}{\tab[2mm] 404 \tab[2mm]} & Porphin\_out\_G09\_invertion &
\multirow{3}{*}{Different} & \multirow{3}{*}{Different}
\\
& E = No Data \tab Freq =No Data   &    &  \\ \cline{2-2}
& Tetrabenzoporphyrin\_rotated\_out\_G09\_invertion   & \multicolumn{2}{c}{\multirow{3}{*}
 {RMS = 0.117363}}
\\
& E = No Data \tab Freq =No Data   &    \multicolumn{2}{c}{}  \\ \hline
\multirow{4}{*}{\tab[2mm] 405 \tab[2mm]} & Tetrabenzoporphyrin\_out\_G09 &
\multirow{3}{*}{\textcolor{Red}{\bf Without comparation}} & \multirow{3}{*}{\textcolor{Red}{\bf Equal}}
\\
& E = No Data \tab Freq =No Data   &    &  \\ \cline{2-2}
& Tetrabenzoporphyrin\_out\_G09\_invertion   & \multicolumn{2}{c}{\multirow{3}{*}
{ RMS = 0}}
\\
& E = No Data \tab Freq =No Data   &    \multicolumn{2}{c}{}  \\ \hline
\multirow{4}{*}{\tab[2mm] 406 \tab[2mm]} & Tetrabenzoporphyrin\_out\_G09 &
\multirow{3}{*}{\textcolor{Red}{\bf Equal}} & \multirow{3}{*}{\textcolor{Red}{\bf Enantiomers}}
\\
& E = No Data \tab Freq =No Data   &    &  \\ \cline{2-2}
& Tetrabenzoporphyrin\_rotated02\_out\_G09   & \multicolumn{2}{c}{\multirow{3}{*}
 {RMS = 0.0001074955}}
\\
& E = No Data \tab Freq =No Data   &    \multicolumn{2}{c}{}  \\ \hline
\end{tabular}
\newpage

\vtab[-2cm]
\tab[-2cm]
\begin{tabular}{c|m{8cm}|c|c}
\# & Moléculas & Restultado esperado & Resultado programa \\\\ \hline\hline
\multirow{4}{*}{\tab[2mm] 407 \tab[2mm]} & Tetrabenzoporphyrin\_out\_G09 &
\multirow{3}{*}{\textcolor{Red}{\bf Equal}} & \multirow{3}{*}{\textcolor{Red}{\bf Enantiomers}}
\\
& E = No Data \tab Freq =No Data   &    &  \\ \cline{2-2}
& Tetrabenzoporphyrin\_rotated02\_out\_G09\_invertion   & \multicolumn{2}{c}{\multirow{3}{*}
 {RMS = 0.0001075617}}
\\
& E = No Data \tab Freq =No Data   &    \multicolumn{2}{c}{}  \\ \hline
\multirow{4}{*}{\tab[2mm] 408 \tab[2mm]} & Tetrabenzoporphyrin\_out\_G09 &
\multirow{3}{*}{Equal} & \multirow{3}{*}{Equal}
\\
& E = No Data \tab Freq =No Data   &    &  \\ \cline{2-2}
& Tetrabenzoporphyrin\_rotated\_out\_G09   & \multicolumn{2}{c}{\multirow{3}{*}
{ RMS = 0}}
\\
& E = No Data \tab Freq =No Data   &    \multicolumn{2}{c}{}  \\ \hline
\multirow{4}{*}{\tab[2mm] 409 \tab[2mm]} & Tetrabenzoporphyrin\_out\_G09 &
\multirow{3}{*}{Equal} & \multirow{3}{*}{Equal}
\\
& E = No Data \tab Freq =No Data   &    &  \\ \cline{2-2}
& Tetrabenzoporphyrin\_rotated\_out\_G09\_invertion   & \multicolumn{2}{c}{\multirow{3}{*}
{ RMS = 0}}
\\
& E = No Data \tab Freq =No Data   &    \multicolumn{2}{c}{}  \\ \hline
\multirow{4}{*}{\tab[2mm] 410 \tab[2mm]} & Tetrabenzoporphyrin\_out\_G09\_invertion &
\multirow{3}{*}{\textcolor{Red}{\bf Equal}} & \multirow{3}{*}{\textcolor{Red}{\bf Enantiomers}}
\\
& E = No Data \tab Freq =No Data   &    &  \\ \cline{2-2}
& Tetrabenzoporphyrin\_rotated02\_out\_G09   & \multicolumn{2}{c}{\multirow{3}{*}
 {RMS = 0.0001074955}}
\\
& E = No Data \tab Freq =No Data   &    \multicolumn{2}{c}{}  \\ \hline
\multirow{4}{*}{\tab[2mm] 411 \tab[2mm]} & Tetrabenzoporphyrin\_out\_G09\_invertion &
\multirow{3}{*}{\textcolor{Red}{\bf Equal}} & \multirow{3}{*}{\textcolor{Red}{\bf Enantiomers}}
\\
& E = No Data \tab Freq =No Data   &    &  \\ \cline{2-2}
& Tetrabenzoporphyrin\_rotated02\_out\_G09\_invertion   & \multicolumn{2}{c}{\multirow{3}{*}
 {RMS = 0.0001075617}}
\\
& E = No Data \tab Freq =No Data   &    \multicolumn{2}{c}{}  \\ \hline
\multirow{4}{*}{\tab[2mm] 412 \tab[2mm]} & Tetrabenzoporphyrin\_out\_G09\_invertion &
\multirow{3}{*}{Equal} & \multirow{3}{*}{Equal}
\\
& E = No Data \tab Freq =No Data   &    &  \\ \cline{2-2}
& Tetrabenzoporphyrin\_rotated\_out\_G09   & \multicolumn{2}{c}{\multirow{3}{*}
{ RMS = 0}}
\\
& E = No Data \tab Freq =No Data   &    \multicolumn{2}{c}{}  \\ \hline
\multirow{4}{*}{\tab[2mm] 413 \tab[2mm]} & Tetrabenzoporphyrin\_out\_G09\_invertion &
\multirow{3}{*}{Equal} & \multirow{3}{*}{Equal}
\\
& E = No Data \tab Freq =No Data   &    &  \\ \cline{2-2}
& Tetrabenzoporphyrin\_rotated\_out\_G09\_invertion   & \multicolumn{2}{c}{\multirow{3}{*}
{ RMS = 0}}
\\
& E = No Data \tab Freq =No Data   &    \multicolumn{2}{c}{}  \\ \hline
\end{tabular}
\newpage

\vtab[-2cm]
\tab[-2cm]
\begin{tabular}{c|m{8cm}|c|c}
\# & Moléculas & Restultado esperado & Resultado programa \\\\ \hline\hline
\multirow{4}{*}{\tab[2mm] 414 \tab[2mm]} & Tetrabenzoporphyrin\_rotated02\_out\_G09 &
\multirow{3}{*}{\textcolor{Red}{\bf Without comparation}} & \multirow{3}{*}{\textcolor{Red}{\bf Equal}}
\\
& E = No Data \tab Freq =No Data   &    &  \\ \cline{2-2}
& Tetrabenzoporphyrin\_rotated02\_out\_G09\_invertion   & \multicolumn{2}{c}{\multirow{3}{*}
{ RMS = 6.620476E-08}}
\\
& E = No Data \tab Freq =No Data   &    \multicolumn{2}{c}{}  \\ \hline
\multirow{4}{*}{\tab[2mm] 415 \tab[2mm]} & Tetrabenzoporphyrin\_rotated02\_out\_G09 &
\multirow{3}{*}{\textcolor{Red}{\bf Equal}} & \multirow{3}{*}{\textcolor{Red}{\bf Enantiomers}}
\\
& E = No Data \tab Freq =No Data   &    &  \\ \cline{2-2}
& Tetrabenzoporphyrin\_rotated\_out\_G09   & \multicolumn{2}{c}{\multirow{3}{*}
 {RMS = 0.0001074955}}
\\
& E = No Data \tab Freq =No Data   &    \multicolumn{2}{c}{}  \\ \hline
\multirow{4}{*}{\tab[2mm] 416 \tab[2mm]} & Tetrabenzoporphyrin\_rotated02\_out\_G09 &
\multirow{3}{*}{\textcolor{Red}{\bf Equal}} & \multirow{3}{*}{\textcolor{Red}{\bf Enantiomers}}
\\
& E = No Data \tab Freq =No Data   &    &  \\ \cline{2-2}
& Tetrabenzoporphyrin\_rotated\_out\_G09\_invertion   & \multicolumn{2}{c}{\multirow{3}{*}
 {RMS = 0.0001074955}}
\\
& E = No Data \tab Freq =No Data   &    \multicolumn{2}{c}{}  \\ \hline
\multirow{4}{*}{\tab[2mm] 417 \tab[2mm]} & Tetrabenzoporphyrin\_rotated02\_out\_G09\_invertion &
\multirow{3}{*}{\textcolor{Red}{\bf Equal}} & \multirow{3}{*}{\textcolor{Red}{\bf Enantiomers}}
\\
& E = No Data \tab Freq =No Data   &    &  \\ \cline{2-2}
& Tetrabenzoporphyrin\_rotated\_out\_G09   & \multicolumn{2}{c}{\multirow{3}{*}
 {RMS = 0.0001075617}}
\\
& E = No Data \tab Freq =No Data   &    \multicolumn{2}{c}{}  \\ \hline
\multirow{4}{*}{\tab[2mm] 418 \tab[2mm]} & Tetrabenzoporphyrin\_rotated02\_out\_G09\_invertion &
\multirow{3}{*}{\textcolor{Red}{\bf Equal}} & \multirow{3}{*}{\textcolor{Red}{\bf Enantiomers}}
\\
& E = No Data \tab Freq =No Data   &    &  \\ \cline{2-2}
& Tetrabenzoporphyrin\_rotated\_out\_G09\_invertion   & \multicolumn{2}{c}{\multirow{3}{*}
 {RMS = 0.0001075617}}
\\
& E = No Data \tab Freq =No Data   &    \multicolumn{2}{c}{}  \\ \hline
\multirow{4}{*}{\tab[2mm] 419 \tab[2mm]} & Tetrabenzoporphyrin\_rotated\_out\_G09 &
\multirow{3}{*}{\textcolor{Red}{\bf Without comparation}} & \multirow{3}{*}{\textcolor{Red}{\bf Equal}}
\\
& E = No Data \tab Freq =No Data   &    &  \\ \cline{2-2}
& Tetrabenzoporphyrin\_rotated\_out\_G09\_invertion   & \multicolumn{2}{c}{\multirow{3}{*}
{ RMS = 0}}
\\
& E = No Data \tab Freq =No Data   &    \multicolumn{2}{c}{}  \\ \hline
\multirow{4}{*}{\tab[2mm] 420 \tab[2mm]} & R-R-R-Fructuose\_out\_G09 &
\multirow{3}{*}{Different} & \multirow{3}{*}{Different}
\\
& E = No Data \tab Freq =No Data   &    &  \\ \cline{2-2}
& R-R-S-Fructuose\_out\_G09   & \multicolumn{2}{c}{\multirow{3}{*}
 {RMS = 0.449321}}
\\
& E = No Data \tab Freq =No Data   &    \multicolumn{2}{c}{}  \\ \hline
\end{tabular}
\newpage

\vtab[-2cm]
\tab[-2cm]
\begin{tabular}{c|m{8cm}|c|c}
\# & Moléculas & Restultado esperado & Resultado programa \\\\ \hline\hline
\multirow{4}{*}{\tab[2mm] 421 \tab[2mm]} & R-R-R-Fructuose\_out\_G09 &
\multirow{3}{*}{Different} & \multirow{3}{*}{Different}
\\
& E = No Data \tab Freq =No Data   &    &  \\ \cline{2-2}
& R-S-R-Fructuose\_out\_G09   & \multicolumn{2}{c}{\multirow{3}{*}
 {RMS = 0.274434}}
\\
& E = No Data \tab Freq =No Data   &    \multicolumn{2}{c}{}  \\ \hline
\multirow{4}{*}{\tab[2mm] 422 \tab[2mm]} & R-R-R-Fructuose\_out\_G09 &
\multirow{3}{*}{Different} & \multirow{3}{*}{Different}
\\
& E = No Data \tab Freq =No Data   &    &  \\ \cline{2-2}
& R-S-S-Fructuose\_out\_G09   & \multicolumn{2}{c}{\multirow{3}{*}
 {RMS = 0.404519}}
\\
& E = No Data \tab Freq =No Data   &    \multicolumn{2}{c}{}  \\ \hline
\multirow{4}{*}{\tab[2mm] 423 \tab[2mm]} & R-R-R-Fructuose\_out\_G09 &
\multirow{3}{*}{Different} & \multirow{3}{*}{Different}
\\
& E = No Data \tab Freq =No Data   &    &  \\ \cline{2-2}
& S-R-R-Fructuose\_out\_G09   & \multicolumn{2}{c}{\multirow{3}{*}
 {RMS = 0.299091}}
\\
& E = No Data \tab Freq =No Data   &    \multicolumn{2}{c}{}  \\ \hline
\multirow{4}{*}{\tab[2mm] 424 \tab[2mm]} & R-R-R-Fructuose\_out\_G09 &
\multirow{3}{*}{Different} & \multirow{3}{*}{Different}
\\
& E = No Data \tab Freq =No Data   &    &  \\ \cline{2-2}
& S-R-S-Fructuose\_out\_G09   & \multicolumn{2}{c}{\multirow{3}{*}
 {RMS = 0.269535}}
\\
& E = No Data \tab Freq =No Data   &    \multicolumn{2}{c}{}  \\ \hline
\multirow{4}{*}{\tab[2mm] 425 \tab[2mm]} & R-R-R-Fructuose\_out\_G09 &
\multirow{3}{*}{Different} & \multirow{3}{*}{Different}
\\
& E = No Data \tab Freq =No Data   &    &  \\ \cline{2-2}
& S-S-R-Fructuose\_out\_G09   & \multicolumn{2}{c}{\multirow{3}{*}
 {RMS = 0.327879}}
\\
& E = No Data \tab Freq =No Data   &    \multicolumn{2}{c}{}  \\ \hline
\multirow{4}{*}{\tab[2mm] 426 \tab[2mm]} & R-R-R-Fructuose\_out\_G09 &
\multirow{3}{*}{\textcolor{Red}{\bf Stereoisomer}} & \multirow{3}{*}{\textcolor{Red}{\bf Different}}
\\
& E = No Data \tab Freq =No Data   &    &  \\ \cline{2-2}
& S-S-S-Fructuose\_out\_G09   & \multicolumn{2}{c}{\multirow{3}{*}
 {RMS = 0.402361}}
\\
& E = No Data \tab Freq =No Data   &    \multicolumn{2}{c}{}  \\ \hline
\multirow{4}{*}{\tab[2mm] 427 \tab[2mm]} & R-R-S-Fructuose\_out\_G09 &
\multirow{3}{*}{Different} & \multirow{3}{*}{Different}
\\
& E = No Data \tab Freq =No Data   &    &  \\ \cline{2-2}
& R-S-R-Fructuose\_out\_G09   & \multicolumn{2}{c}{\multirow{3}{*}
 {RMS = 0.174888}}
\\
& E = No Data \tab Freq =No Data   &    \multicolumn{2}{c}{}  \\ \hline
\end{tabular}
\newpage

\vtab[-2cm]
\tab[-2cm]
\begin{tabular}{c|m{8cm}|c|c}
\# & Moléculas & Restultado esperado & Resultado programa \\\\ \hline\hline
\multirow{4}{*}{\tab[2mm] 428 \tab[2mm]} & R-R-S-Fructuose\_out\_G09 &
\multirow{3}{*}{Different} & \multirow{3}{*}{Different}
\\
& E = No Data \tab Freq =No Data   &    &  \\ \cline{2-2}
& R-S-S-Fructuose\_out\_G09   & \multicolumn{2}{c}{\multirow{3}{*}
 {RMS = 0.0448022}}
\\
& E = No Data \tab Freq =No Data   &    \multicolumn{2}{c}{}  \\ \hline
\multirow{4}{*}{\tab[2mm] 429 \tab[2mm]} & R-R-S-Fructuose\_out\_G09 &
\multirow{3}{*}{Different} & \multirow{3}{*}{Different}
\\
& E = No Data \tab Freq =No Data   &    &  \\ \cline{2-2}
& S-R-R-Fructuose\_out\_G09   & \multicolumn{2}{c}{\multirow{3}{*}
 {RMS = 0.15023}}
\\
& E = No Data \tab Freq =No Data   &    \multicolumn{2}{c}{}  \\ \hline
\multirow{4}{*}{\tab[2mm] 430 \tab[2mm]} & R-R-S-Fructuose\_out\_G09 &
\multirow{3}{*}{Different} & \multirow{3}{*}{Different}
\\
& E = No Data \tab Freq =No Data   &    &  \\ \cline{2-2}
& S-R-S-Fructuose\_out\_G09   & \multicolumn{2}{c}{\multirow{3}{*}
 {RMS = 0.179786}}
\\
& E = No Data \tab Freq =No Data   &    \multicolumn{2}{c}{}  \\ \hline
\multirow{4}{*}{\tab[2mm] 431 \tab[2mm]} & R-R-S-Fructuose\_out\_G09 &
\multirow{3}{*}{\textcolor{Red}{\bf Stereoisomer}} & \multirow{3}{*}{\textcolor{Red}{\bf Enantiomers}}
\\
& E = No Data \tab Freq =No Data   &    &  \\ \cline{2-2}
& S-S-R-Fructuose\_out\_G09   & \multicolumn{2}{c}{\multirow{3}{*}
 {RMS = 0.0005551068}}
\\
& E = No Data \tab Freq =No Data   &    \multicolumn{2}{c}{}  \\ \hline
\multirow{4}{*}{\tab[2mm] 432 \tab[2mm]} & R-R-S-Fructuose\_out\_G09 &
\multirow{3}{*}{Different} & \multirow{3}{*}{Different}
\\
& E = No Data \tab Freq =No Data   &    &  \\ \cline{2-2}
& S-S-S-Fructuose\_out\_G09   & \multicolumn{2}{c}{\multirow{3}{*}
 {RMS = 0.0469601}}
\\
& E = No Data \tab Freq =No Data   &    \multicolumn{2}{c}{}  \\ \hline
\multirow{4}{*}{\tab[2mm] 433 \tab[2mm]} & R-S-R-Fructuose\_out\_G09 &
\multirow{3}{*}{Different} & \multirow{3}{*}{Different}
\\
& E = No Data \tab Freq =No Data   &    &  \\ \cline{2-2}
& R-S-S-Fructuose\_out\_G09   & \multicolumn{2}{c}{\multirow{3}{*}
 {RMS = 0.130085}}
\\
& E = No Data \tab Freq =No Data   &    \multicolumn{2}{c}{}  \\ \hline
\multirow{4}{*}{\tab[2mm] 434 \tab[2mm]} & R-S-R-Fructuose\_out\_G09 &
\multirow{3}{*}{Different} & \multirow{3}{*}{Different}
\\
& E = No Data \tab Freq =No Data   &    &  \\ \cline{2-2}
& S-R-R-Fructuose\_out\_G09   & \multicolumn{2}{c}{\multirow{3}{*}
 {RMS = 0.0246573}}
\\
& E = No Data \tab Freq =No Data   &    \multicolumn{2}{c}{}  \\ \hline
\end{tabular}
\newpage

\vtab[-2cm]
\tab[-2cm]
\begin{tabular}{c|m{8cm}|c|c}
\# & Moléculas & Restultado esperado & Resultado programa \\\\ \hline\hline
\multirow{4}{*}{\tab[2mm] 435 \tab[2mm]} & R-S-R-Fructuose\_out\_G09 &
\multirow{3}{*}{\textcolor{Red}{\bf Stereoisomer}} & \multirow{3}{*}{\textcolor{Red}{\bf Enantiomers}}
\\
& E = No Data \tab Freq =No Data   &    &  \\ \cline{2-2}
& S-R-S-Fructuose\_out\_G09   & \multicolumn{2}{c}{\multirow{3}{*}
 {RMS = 0.005799255}}
\\
& E = No Data \tab Freq =No Data   &    \multicolumn{2}{c}{}  \\ \hline
\multirow{4}{*}{\tab[2mm] 436 \tab[2mm]} & R-S-R-Fructuose\_out\_G09 &
\multirow{3}{*}{Different} & \multirow{3}{*}{Different}
\\
& E = No Data \tab Freq =No Data   &    &  \\ \cline{2-2}
& S-S-R-Fructuose\_out\_G09   & \multicolumn{2}{c}{\multirow{3}{*}
 {RMS = 0.0534457}}
\\
& E = No Data \tab Freq =No Data   &    \multicolumn{2}{c}{}  \\ \hline
\multirow{4}{*}{\tab[2mm] 437 \tab[2mm]} & R-S-R-Fructuose\_out\_G09 &
\multirow{3}{*}{Different} & \multirow{3}{*}{Different}
\\
& E = No Data \tab Freq =No Data   &    &  \\ \cline{2-2}
& S-S-S-Fructuose\_out\_G09   & \multicolumn{2}{c}{\multirow{3}{*}
 {RMS = 0.127928}}
\\
& E = No Data \tab Freq =No Data   &    \multicolumn{2}{c}{}  \\ \hline
\multirow{4}{*}{\tab[2mm] 438 \tab[2mm]} & R-S-S-Fructuose\_out\_G09 &
\multirow{3}{*}{\textcolor{Red}{\bf Stereoisomer}} & \multirow{3}{*}{\textcolor{Red}{\bf Enantiomers}}
\\
& E = No Data \tab Freq =No Data   &    &  \\ \cline{2-2}
& S-R-R-Fructuose\_out\_G09   & \multicolumn{2}{c}{\multirow{3}{*}
 {RMS = 0.0001020726}}
\\
& E = No Data \tab Freq =No Data   &    \multicolumn{2}{c}{}  \\ \hline
\multirow{4}{*}{\tab[2mm] 439 \tab[2mm]} & R-S-S-Fructuose\_out\_G09 &
\multirow{3}{*}{Different} & \multirow{3}{*}{Different}
\\
& E = No Data \tab Freq =No Data   &    &  \\ \cline{2-2}
& S-R-S-Fructuose\_out\_G09   & \multicolumn{2}{c}{\multirow{3}{*}
 {RMS = 0.134984}}
\\
& E = No Data \tab Freq =No Data   &    \multicolumn{2}{c}{}  \\ \hline
\multirow{4}{*}{\tab[2mm] 440 \tab[2mm]} & R-S-S-Fructuose\_out\_G09 &
\multirow{3}{*}{Different} & \multirow{3}{*}{Different}
\\
& E = No Data \tab Freq =No Data   &    &  \\ \cline{2-2}
& S-S-R-Fructuose\_out\_G09   & \multicolumn{2}{c}{\multirow{3}{*}
 {RMS = 0.0766398}}
\\
& E = No Data \tab Freq =No Data   &    \multicolumn{2}{c}{}  \\ \hline
\multirow{4}{*}{\tab[2mm] 441 \tab[2mm]} & R-S-S-Fructuose\_out\_G09 &
\multirow{3}{*}{Different} & \multirow{3}{*}{Different}
\\
& E = No Data \tab Freq =No Data   &    &  \\ \cline{2-2}
& S-S-S-Fructuose\_out\_G09   & \multicolumn{2}{c}{\multirow{3}{*}
 {RMS = 0.00215791}}
\\
& E = No Data \tab Freq =No Data   &    \multicolumn{2}{c}{}  \\ \hline
\end{tabular}
\newpage

\vtab[-2cm]
\tab[-2cm]
\begin{tabular}{c|m{8cm}|c|c}
\# & Moléculas & Restultado esperado & Resultado programa \\\\ \hline\hline
\multirow{4}{*}{\tab[2mm] 442 \tab[2mm]} & S-R-R-Fructuose\_out\_G09 &
\multirow{3}{*}{Different} & \multirow{3}{*}{Different}
\\
& E = No Data \tab Freq =No Data   &    &  \\ \cline{2-2}
& S-R-S-Fructuose\_out\_G09   & \multicolumn{2}{c}{\multirow{3}{*}
 {RMS = 0.029556}}
\\
& E = No Data \tab Freq =No Data   &    \multicolumn{2}{c}{}  \\ \hline
\multirow{4}{*}{\tab[2mm] 443 \tab[2mm]} & S-R-R-Fructuose\_out\_G09 &
\multirow{3}{*}{Different} & \multirow{3}{*}{Different}
\\
& E = No Data \tab Freq =No Data   &    &  \\ \cline{2-2}
& S-S-R-Fructuose\_out\_G09   & \multicolumn{2}{c}{\multirow{3}{*}
 {RMS = 0.0287884}}
\\
& E = No Data \tab Freq =No Data   &    \multicolumn{2}{c}{}  \\ \hline
\multirow{4}{*}{\tab[2mm] 444 \tab[2mm]} & S-R-R-Fructuose\_out\_G09 &
\multirow{3}{*}{Different} & \multirow{3}{*}{Different}
\\
& E = No Data \tab Freq =No Data   &    &  \\ \cline{2-2}
& S-S-S-Fructuose\_out\_G09   & \multicolumn{2}{c}{\multirow{3}{*}
 {RMS = 0.10327}}
\\
& E = No Data \tab Freq =No Data   &    \multicolumn{2}{c}{}  \\ \hline
\multirow{4}{*}{\tab[2mm] 445 \tab[2mm]} & S-R-S-Fructuose\_out\_G09 &
\multirow{3}{*}{Different} & \multirow{3}{*}{Different}
\\
& E = No Data \tab Freq =No Data   &    &  \\ \cline{2-2}
& S-S-R-Fructuose\_out\_G09   & \multicolumn{2}{c}{\multirow{3}{*}
 {RMS = 0.0583444}}
\\
& E = No Data \tab Freq =No Data   &    \multicolumn{2}{c}{}  \\ \hline
\multirow{4}{*}{\tab[2mm] 446 \tab[2mm]} & S-R-S-Fructuose\_out\_G09 &
\multirow{3}{*}{Different} & \multirow{3}{*}{Different}
\\
& E = No Data \tab Freq =No Data   &    &  \\ \cline{2-2}
& S-S-S-Fructuose\_out\_G09   & \multicolumn{2}{c}{\multirow{3}{*}
 {RMS = 0.132826}}
\\
& E = No Data \tab Freq =No Data   &    \multicolumn{2}{c}{}  \\ \hline
\multirow{4}{*}{\tab[2mm] 447 \tab[2mm]} & S-S-R-Fructuose\_out\_G09 &
\multirow{3}{*}{Different} & \multirow{3}{*}{Different}
\\
& E = No Data \tab Freq =No Data   &    &  \\ \cline{2-2}
& S-S-S-Fructuose\_out\_G09   & \multicolumn{2}{c}{\multirow{3}{*}
 {RMS = 0.0744819}}
\\
& E = No Data \tab Freq =No Data   &    \multicolumn{2}{c}{}  \\ \hline
\multirow{4}{*}{\tab[2mm] 448 \tab[2mm]} & r-enan\_out\_G09 &
\multirow{3}{*}{\textcolor{Red}{\bf Without comparation}} & \multirow{3}{*}{\textcolor{Red}{\bf Enantiomers}}
\\
& E = No Data \tab Freq =No Data   &    &  \\ \cline{2-2}
& r-enan\_out\_G09\_invertion\_output   & \multicolumn{2}{c}{\multirow{3}{*}
 {RMS = 0.01735201}}
\\
& E = No Data \tab Freq =No Data   &    \multicolumn{2}{c}{}  \\ \hline
\end{tabular}
\newpage

\vtab[-2cm]
\tab[-2cm]
\begin{tabular}{c|m{8cm}|c|c}
\# & Moléculas & Restultado esperado & Resultado programa \\\\ \hline\hline
\multirow{4}{*}{\tab[2mm] 449 \tab[2mm]} & r-enan\_out\_G09 &
\multirow{3}{*}{Equal} & \multirow{3}{*}{Equal}
\\
& E = No Data \tab Freq =No Data   &    &  \\ \cline{2-2}
& r-enan\_rotated\_out\_G09   & \multicolumn{2}{c}{\multirow{3}{*}
{ RMS = 0}}
\\
& E = No Data \tab Freq =No Data   &    \multicolumn{2}{c}{}  \\ \hline
\multirow{4}{*}{\tab[2mm] 450 \tab[2mm]} & r-enan\_out\_G09 &
\multirow{3}{*}{\textcolor{Red}{\bf Stereoisomer}} & \multirow{3}{*}{\textcolor{Red}{\bf Enantiomers}}
\\
& E = No Data \tab Freq =No Data   &    &  \\ \cline{2-2}
& s-enan\_out\_G09   & \multicolumn{2}{c}{\multirow{3}{*}
 {RMS = 0.00506593}}
\\
& E = No Data \tab Freq =No Data   &    \multicolumn{2}{c}{}  \\ \hline
\multirow{4}{*}{\tab[2mm] 451 \tab[2mm]} & r-enan\_out\_G09 &
\multirow{3}{*}{Equal} & \multirow{3}{*}{Equal}
\\
& E = No Data \tab Freq =No Data   &    &  \\ \cline{2-2}
& s-enan\_out\_G09\_invertion\_output   & \multicolumn{2}{c}{\multirow{3}{*}
{\textcolor{Red}{ RMS = 0.01228343}}}
\\
& E = No Data \tab Freq =No Data   &    \multicolumn{2}{c}{}  \\ \hline
\multirow{4}{*}{\tab[2mm] 452 \tab[2mm]} & r-enan\_out\_G09 &
\multirow{3}{*}{\textcolor{Red}{\bf Stereoisomer}} & \multirow{3}{*}{\textcolor{Red}{\bf Enantiomers}}
\\
& E = No Data \tab Freq =No Data   &    &  \\ \cline{2-2}
& s-enan\_rotated\_out\_G09   & \multicolumn{2}{c}{\multirow{3}{*}
 {RMS = 0.005065924}}
\\
& E = No Data \tab Freq =No Data   &    \multicolumn{2}{c}{}  \\ \hline
\multirow{4}{*}{\tab[2mm] 453 \tab[2mm]} & r-enan\_out\_G09\_invertion\_output &
\multirow{3}{*}{\textcolor{Red}{\bf Stereoisomer}} & \multirow{3}{*}{\textcolor{Red}{\bf Enantiomers}}
\\
& E = No Data \tab Freq =No Data   &    &  \\ \cline{2-2}
& r-enan\_rotated\_out\_G09   & \multicolumn{2}{c}{\multirow{3}{*}
 {RMS = 0.01735201}}
\\
& E = No Data \tab Freq =No Data   &    \multicolumn{2}{c}{}  \\ \hline
\multirow{4}{*}{\tab[2mm] 454 \tab[2mm]} & r-enan\_out\_G09\_invertion\_output &
\multirow{3}{*}{Equal} & \multirow{3}{*}{Equal}
\\
& E = No Data \tab Freq =No Data   &    &  \\ \cline{2-2}
& s-enan\_out\_G09   & \multicolumn{2}{c}{\multirow{3}{*}
{\textcolor{Red}{ RMS = 0.02241794}}}
\\
& E = No Data \tab Freq =No Data   &    \multicolumn{2}{c}{}  \\ \hline
\multirow{4}{*}{\tab[2mm] 455 \tab[2mm]} & r-enan\_out\_G09\_invertion\_output &
\multirow{3}{*}{\textcolor{Red}{\bf Stereoisomer}} & \multirow{3}{*}{\textcolor{Red}{\bf Enantiomers}}
\\
& E = No Data \tab Freq =No Data   &    &  \\ \cline{2-2}
& s-enan\_out\_G09\_invertion\_output   & \multicolumn{2}{c}{\multirow{3}{*}
 {RMS = 0.005068578}}
\\
& E = No Data \tab Freq =No Data   &    \multicolumn{2}{c}{}  \\ \hline
\end{tabular}
\newpage

\vtab[-2cm]
\tab[-2cm]
\begin{tabular}{c|m{8cm}|c|c}
\# & Moléculas & Restultado esperado & Resultado programa \\\\ \hline\hline
\multirow{4}{*}{\tab[2mm] 456 \tab[2mm]} & r-enan\_out\_G09\_invertion\_output &
\multirow{3}{*}{Equal} & \multirow{3}{*}{Equal}
\\
& E = No Data \tab Freq =No Data   &    &  \\ \cline{2-2}
& s-enan\_rotated\_out\_G09   & \multicolumn{2}{c}{\multirow{3}{*}
{\textcolor{Red}{ RMS = 0.02241793}}}
\\
& E = No Data \tab Freq =No Data   &    \multicolumn{2}{c}{}  \\ \hline
\multirow{4}{*}{\tab[2mm] 457 \tab[2mm]} & r-enan\_rotated\_out\_G09 &
\multirow{3}{*}{\textcolor{Red}{\bf Stereoisomer}} & \multirow{3}{*}{\textcolor{Red}{\bf Enantiomers}}
\\
& E = No Data \tab Freq =No Data   &    &  \\ \cline{2-2}
& s-enan\_out\_G09   & \multicolumn{2}{c}{\multirow{3}{*}
 {RMS = 0.00506593}}
\\
& E = No Data \tab Freq =No Data   &    \multicolumn{2}{c}{}  \\ \hline
\multirow{4}{*}{\tab[2mm] 458 \tab[2mm]} & r-enan\_rotated\_out\_G09 &
\multirow{3}{*}{Equal} & \multirow{3}{*}{Equal}
\\
& E = No Data \tab Freq =No Data   &    &  \\ \cline{2-2}
& s-enan\_out\_G09\_invertion\_output   & \multicolumn{2}{c}{\multirow{3}{*}
{\textcolor{Red}{ RMS = 0.01228343}}}
\\
& E = No Data \tab Freq =No Data   &    \multicolumn{2}{c}{}  \\ \hline
\multirow{4}{*}{\tab[2mm] 459 \tab[2mm]} & r-enan\_rotated\_out\_G09 &
\multirow{3}{*}{\textcolor{Red}{\bf Stereoisomer}} & \multirow{3}{*}{\textcolor{Red}{\bf Enantiomers}}
\\
& E = No Data \tab Freq =No Data   &    &  \\ \cline{2-2}
& s-enan\_rotated\_out\_G09   & \multicolumn{2}{c}{\multirow{3}{*}
 {RMS = 0.005065924}}
\\
& E = No Data \tab Freq =No Data   &    \multicolumn{2}{c}{}  \\ \hline
\multirow{4}{*}{\tab[2mm] 460 \tab[2mm]} & s-enan\_out\_G09 &
\multirow{3}{*}{\textcolor{Red}{\bf Without comparation}} & \multirow{3}{*}{\textcolor{Red}{\bf Enantiomers}}
\\
& E = No Data \tab Freq =No Data   &    &  \\ \cline{2-2}
& s-enan\_out\_G09\_invertion\_output   & \multicolumn{2}{c}{\multirow{3}{*}
 {RMS = 0.01734936}}
\\
& E = No Data \tab Freq =No Data   &    \multicolumn{2}{c}{}  \\ \hline
\multirow{4}{*}{\tab[2mm] 461 \tab[2mm]} & s-enan\_out\_G09 &
\multirow{3}{*}{Equal} & \multirow{3}{*}{Equal}
\\
& E = No Data \tab Freq =No Data   &    &  \\ \cline{2-2}
& s-enan\_rotated\_out\_G09   & \multicolumn{2}{c}{\multirow{3}{*}
{ RMS = 6.301958E-09}}
\\
& E = No Data \tab Freq =No Data   &    \multicolumn{2}{c}{}  \\ \hline
\multirow{4}{*}{\tab[2mm] 462 \tab[2mm]} & s-enan\_out\_G09\_invertion\_output &
\multirow{3}{*}{\textcolor{Red}{\bf Stereoisomer}} & \multirow{3}{*}{\textcolor{Red}{\bf Enantiomers}}
\\
& E = No Data \tab Freq =No Data   &    &  \\ \cline{2-2}
& s-enan\_rotated\_out\_G09   & \multicolumn{2}{c}{\multirow{3}{*}
 {RMS = 0.01734936}}
\\
& E = No Data \tab Freq =No Data   &    \multicolumn{2}{c}{}  \\ \hline
\end{tabular}
\newpage

\vtab[-2cm]
\tab[-2cm]
\begin{tabular}{c|m{8cm}|c|c}
\# & Moléculas & Restultado esperado & Resultado programa \\\\ \hline\hline
\multirow{4}{*}{\tab[2mm] 463 \tab[2mm]} & test-R\_out\_G09 &
\multirow{3}{*}{\textcolor{Red}{\bf Without comparation}} & \multirow{3}{*}{\textcolor{Red}{\bf Enantiomers}}
\\
& E = No Data \tab Freq =No Data   &    &  \\ \cline{2-2}
& test-S\_out\_G09   & \multicolumn{2}{c}{\multirow{3}{*}
 {RMS = 1.273513E-17}}
\\
& E = No Data \tab Freq =No Data   &    \multicolumn{2}{c}{}  \\ \hline
\end{tabular}



%-----------------------------------------------------------------------------% 
% Extras
%-----------------------------------------------------------------------------% 
\section{Extras}

Durante la realización de este reporte me surgió la necesidad de crear un par de scripts que puedan ayudar para el trabajo y 
el empleo de esté programa. La tarea del script {\it scriptMake\_All\_Comparations.bsh} es realizar todas las comparaciones 
posibles en una carpeta que contenga varias geometrías en el formato de archivo XYZ, guardando el resultado en otro archivo.\\
El otro script es {\it translate\_comparatation2Latex.bsh} que utiliza la salida del anterior script para crear un resumen de las comparaciones 
y traducirlas a un formato personalizado de latex y, de esta manera, poder visualizar con mayor facilidad las comparaciones.\\
Actualmente el script sigue en desarrollo y se encuentra en el repositiorio del programa.


\end{document}
